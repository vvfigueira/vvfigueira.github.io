\documentclass[preprint,eps,10pt]{revtex4}
%\documentstyle[10pt]{artigo}
%
\usepackage[portuges]{babel}
%\usepackage[latin1]{inputenc}
\usepackage{amsmath,amssymb,lmodern}
\usepackage{indentfirst}
%
\newcommand{\beq}{\begin{equation}}
\newcommand{\eeq}{\end{equation}}
\newcommand{\bea}{\begin{eqnarray}}
\newcommand{\eea}{\end{eqnarray}}
\newcommand{\nn}{\nonumber}
\newcommand{\ex}{{\rm e}}
\newcommand{\Tr}  {\mathop{\rm Tr}}
\newcommand{\bmu} {\bar{\mu}}
\def\lsi{\raise0.3ex\hbox{$<$\kern-0.75em\raise-1.1ex\hbox{$\sim$}}}
\def\gsi{\raise0.3ex\hbox{$>$\kern-0.75em\raise-1.1ex\hbox{$\sim$}}}
\newcommand{\lsim}{\mathop{\lsi}}
\newcommand{\gsim}{\mathop{\gsi}}
%
\oddsidemargin=0cm
\evensidemargin=0cm
\textwidth=16.0cm
\topmargin=-2.5cm
\textheight=26cm
%
\def\baselinestretch{1.0}
\parskip=0.1cm
%
\def\next{\hfill\break} %start next line
\def\ll#1{\leftline{#1}}
\def\rl#1{\rightline{#1}}
\def\page{\vfill\eject}
%
\begin{document}
\thispagestyle{empty}
%
\centerline{\underline{PLANO DE ATIVIDADES}}
%

\vspace{2.0cm}
%
\begin{center}
{\Large{\bf Amplitudes de espalhamento em teorias com derivadas de ordem superior}}
\vskip0.4cm
\end{center}
%
\vspace{1.3cm}
%
%\centerline{\large {\it Gabriel Menezes}}
%
\vskip0.5cm
%
\centerline{Departamento de Física-Matemática}
\vspace{0.5cm}
\centerline{Instituto de Física - Universidade de São Paulo}
\vspace{.06in}
\centerline{São Paulo/SP}
%
\vspace{2.0cm}
%

\centerline{\bf Resumo}

\vskip0.5cm

Este plano de atividades se propõe a realizar um estudo de amplitudes de espalhamento na chamada teoria $(DF)^2$ e, utilizando-se do m\'etodo da c\'opia dupla, estender esses resultados para o caso da supergravidade conforme do tipo Berkovits-Witten. Estes estudos tamb\'em prev\^eem uma maior compreens\~ao do m\'etodo da unitariedade generalizado para o caso de part\'iculas inst\'aveis. Além disso, também nos permitiria uma abordagem sistemático no estudo do comportamento a altas energias das amplitudes de espalhamento em gravidade quadrática.


\newpage


\section{Justificativas para o plano de atividades}


\quad

Um insight importante decorrente do estudo contempor\^aneo das amplitudes de espalhamento \'e que as amplitudes gravitacionais s\~ao mais simples do que se esperaria; com efeito, as amplitudes gravitacionais est\~ao intimamente ligadas \`as amplitudes da teoria de Yang-Mills pela abordagem chamada de c\'opia dupla ({\em double copy}), que determina que as amplitudes gravitacionais sejam obtidas como produto de duas amplitudes de Yang-Mills. A formula\c{c}\~ao original de Kawai, Lewellen e Tye afirma que uma amplitude gravitacional no n\'ivel da \'arvore \'e dada por uma soma sobre os produtos de duas amplitudes Yang-Mills ordenadas por cor no n\'ivel da \'arvore~\cite{KLT}. Mais recentemente, Bern, Carrasco e Johansson demonstraram que os numeradores gravitacionais s\~ao, em certo sentido, o quadrado dos numeradores cinem\'aticos da teoria de Yang-Mills~\cite{BCJ1,BCJ2}. Por sua vez, a abordagem de c\'opia dupla tamb\'em conecta solu\c{c}\~oes cl\'assicas da teoria de Yang-Mills e da gravidade. 

Por outro lado, existem muitas abordagens alternativas para se quantizar o campo gravitacional, mas comparativamente pouco trabalho atual explora a op\c{c}\~ao de se representar a gravidade por uma teoria qu\^antica de campos renormaliz\'avel. H\'a uma cole\c{c}\~ao modesta de trabalhos recentes tentando reviver essa possibilidade~\cite{2,Holdom:16,Mannheim:12,6a,6b,Barra:2019rhz,Tomboulis:15,7a,10a}. 

A gravidade quadr\'atica tem a caracter\'istica positiva de ser renormaliz\'avel~\cite{Stelle:1976gc}, e pode ser assintoticamente livre~\cite{Julve:1978xn, Fradkin:1981hx, Fradkin:1981iu}. Ademais, para se recuperar a relatividade geral no limite de baixas energias, deve-se lan\c{c}ar m\~ao de uma constru\c{c}\~ao delicada de modo a obter os termos usuais a baixas energias. Recentemente apresentamos uma proposta diferente para a gravidade quadr\'atica em que as intera\c{c}\~oes gravitacionais permanecem fracamente acopladas em todas as escalas de energia, mas ocorre uma ``assist\^encia'' por parte de um campo de calibre, cujo papel \'e definir a escala de Planck e induzir a a\c{c}\~ao de Einstein-Hilbert a baixas energias~\cite{DM:18}. Com rela\c{c}\~ao a este \'ultimo efeito, n\'os calculamos a constante de gravita\c{c}\~ao universal de Newton induzida devido \`a cromodin\^amica qu\^antica, encontrando um valor positivo para a mesma~\cite{DM:Newton}. 

Ainda seguindo a mesma linha de investiga\c{c}\~ao, apresentamos uma nova compreens\~ao da resson\^ancia tipo fantasma inst\'avel que aparece em teorias como a gravidade quadr\'atica e as teorias do tipo Lee-Wick. As corre\c{c}\~oes qu\^anticas tornam esta resson\^ancia inst\'avel, de forma que ela n\~ao aparece no espectro assint\'otico~\cite{PRD19a}. Provamos que essas teorias s\~ao unit\'arias para todas as ordens na teoria de perturba\c{c}\~ao~\cite{PRD19}. Por outro lado, essas teorias podem apresentar viola\c{c}\~oes de microcausalidade da ordem da largura do modo fantasma. Para entendermos melhor essa incerteza causal, introduzimos um novo conceito: a seta da causalidade -- a dire\c{c}\~ao do tempo em que ocorrem os processos qu\^anticos causais, que est\'a presente em todas as teorias qu\^anticas~\cite{PRL19}. As leis da f\'isica qu\^antica fornecem uma dire\c{c}\~ao causal. Conforme discutimos, a seta da causalidade determina a seta da termodin\^amica e a seta do tempo~\cite{PPNP20}. Gostaria de enfatizar que o artigo~\cite{PRL19}, publicado pela c\'elebre {\em Physical Review Letters}, foi selecionado para o seleto grupo das {\em Editors' Suggestion} da dita revista.
 
O recente trabalho~\cite{Donoghue:2023yjt} demonstra explicitamente como certas t\'ecnicas tradicionais para se calcular fun\c{c}\~oes beta do grupo de renormaliza\c{c}\~ao n\~ao produzem acoplamentos que correm f\'isicos quando utilizadas em teorias que possuem termos com derivadas de ordem superior; por exemplo, o acoplamento fundamental do modelo sigma n\~ao linear SU(N) com derivada de ordem superior n\~ao corre em nenhuma escala de energia, apesar de resultados relatados anteriormente usando m\'etodos baseados em regulariza\c{c}\~ao com {\em cut-off}. Esses resultados foram levados \`as intera\c{c}\~oes gravitacionais, em particular para gravidade quadr\'atica -- demonstramos que a propriedade de iberdade assintótica acaba por ser compatível com a ausência de táquions~\cite{Buccio:2024hys}. Gostaria de enfatizar que este último artigo, publicado pela c\'elebre {\em Physical Review Letters}, foi selecionado para o seleto grupo das {\em Editors' Suggestion} da dita revista.



\section{Descrição dos objetivos}

De forma abrangente, espera-se uma melhor compreens\~ao da f\'isica envolvida na linha de pesquisa acima delineada. Ademais, espera-se tamb\'em que os resultados obtidos concordem com as conclus\~oes existentes na literatura. Especificamente:

\begin{enumerate}

\item Para iniciarmos nossos estudos, vamos investigar os modelos escalares estudados nos trabalhos~\cite{PRD19,Buccio:2023lzo}. Como são modelos que funcionam como bons representantes escalares de gravidade quadrática, eles devem servir muito bem para preparar o cenário para entendermos com mais detalhe como aplicamos as técnicas descritas na referência~\cite{Herrmann:2018dja} para o caso de teorias com derivadas de ordem superior.

\item Desejamos entender melhor a estrutura analítica das teorias que são obtidas da teoria $(DF)^2$ por meio de cópia dupla. Como é sabido, esta teoria de gauge entra na construção de cópia dupla para amplitudes de várias teorias da gravidade, incluindo algumas teorias de cordas e também a supergravidade conforme não mínima do tipo Berkovits-Witten~\cite{Berkovits:2004jj}. Como a teoria
$(DF)^2$ é uma teoria de gauge com derivadas de ordem superior, ela deve ter um propagador de gauge quártico, cujo comportamento analítico é diferente do usual. Portanto, buscamos entender como isso afeta (ou não) as teorias construídas a partir dela por meio de cópia dupla usando métodos de unitariedade para o cálculo de amplitudes no nível do loop.

\item Utilizando os métodos descritos na referência~\cite{Herrmann:2018dja}, desejamos explorar as propriedades de integrandos da supergravidade conforme do tipo Berkovits-Witten e da teoria $(DF)^2$ para momentos de loop grandes. Esta análise não pode ser feita diretamente para o integrando off-shell completo, mas só se torna bem definida em cortes que nos permitem especificar classificações para as variáveis de loop de forma inequívoca. A região ultravioleta das amplitudes de espalhamento se origina de polos no infinito dos integrandos de loop. A referência acima demonstra que nas amplitudes de gravidade e ${\cal N}=8$ supergravidade esses integrandos escondem uma série de características surpreendentes -- certos polos no infinito estão ausentes, o que requer uma ``conspiração'' entre integrais de Feynman individuais que contribuem para a amplitude. Desejamos ver se essa história se repete no caso da teoria ${\cal N}=4$ supergravidade conforme. Estes estudos tamb\'em prev\^eem uma maior compreens\~ao do m\'etodo da unitariedade generalizado para o caso de part\'iculas inst\'aveis. Além disso, também nos permitiria uma abordagem sistemático no estudo do comportamento a altas energias das amplitudes de espalhamento em gravidade quadrática. 


\end{enumerate}

Acreditamos que este projeto de pesquisa possa contribuir para o avanço do conhecimento sobre o comportamento assintótico das amplitudes de espalhamento em gravidade quadrática. A abordagem inovadora e promissora das amplitudes de espalhamento pode pavimentar novos caminhos para resolver questões da física teórica e estabelecer novas conexões.

\section{Plano de trabalho}


\subsection{Metodologia}


A metodologia geral para todas as linhas de pesquisa discutidas acima seguir\'a os princ\'ipios usuais da teoria qu\^antica de campos em espa\c{c}o-tempo de Minkowski. Em particular, para o estudo de amplitudes de espalhamento, devemos seguir os métodos descritos pelas referências~\cite{74,75,Guevara:19a,Kosower:19,Maybee:19,Elvang:15,Arkani-Hamed:17}. Por outro lado, para o caso específico da teoria $(DF)^2$ e a supergravidade do tipo Berkovits-Witten, iremos nos basear na metodologia apresentada nas referências~\cite{Johansson:2017srf,Johansson:2018ues,Azevedo:2018dgo,Azevedo:2019zbn}. Para analisarmos o comportamento assintótico de integrandos nessas teorias, também iremos nos basear na referência~\cite{Herrmann:2018dja}.  




\subsection{Cronograma de resultados previstos}

À medida que os resultados se
tornarem mais claros, nós ajustaremos o cronograma para produzir
o máximo de produção de pesquisa. Em qualquer caso, como o projeto
será desenvolvido ao longo de 24 meses, propomos os seguintes estágios
principais para cada um dos estudos acima mencionados:
\begin{itemize}
\item \textbf{1º semestre:} Revisão bibliográfica e formulação inicial do
problema. Estudo detalhado das amplitudes da teoria $(DF)^2$ e a supergravidade do tipo Berkovits-Witten.
\item \textbf{2º semestre:} Cálculos preliminares. Desenvolvimento de ferramentas computacionais para simplificação dos cálculos. 
\item \textbf{3º semestre:} Aplicação dos métodos desenvolvidos para se estudar as propriedades de integrandos da supergravidade conforme do tipo Berkovits-Witten e da teoria $(DF)^2$ para momentos de loop grandes.
\item \textbf{4º semestre:} Finalização dos cálculos, análise dos resultados e redação da dissertação. Se for o caso, e se o tempo permitir, preparação e finalização de um artigo científico.
\end{itemize}




\vspace{0.1cm}

\bibliography{refs.bib}


\end{document}

