\documentclass[twoside]{amsart}

\usepackage[brazilian]{babel}
\usepackage{csquotes}
%\usepackage[sorting=none, style=verbose-inote, backend=biber]{biblatex}
\usepackage{amsmath}
\usepackage{amssymb}
\usepackage{bbm}
\usepackage{graphics}
\usepackage{mathtools}
\usepackage[hidelinks]{hyperref}
\usepackage{physics}
\usepackage{enumitem}
\usepackage{slashed}
\usepackage[lmargin=0.5cm,rmargin=0.5cm, tmargin =1cm,bmargin =1cm]{geometry}
\usepackage[explicit]{titlesec}
\usepackage{tensor}
\usepackage{cleveref}
\usepackage{ragged2e}
\usepackage{subcaption}

\usepackage{tcolorbox}
\tcbuselibrary{minted,breakable,xparse,skins}

\definecolor{bg}{gray}{0.95}
\DeclareTCBListing{mintedbox}{O{}m!O{}}{%
  breakable=true,
  listing engine=minted,
  listing only,
  minted language=#2,
  minted style=default,
  minted options={%
    linenos,
    gobble=0,
    breaklines=true,
    breakafter=,,
    fontsize=\small,
    numbersep=8pt,
    #1},
  boxsep=0pt,
  left skip=0pt,
  right skip=0pt,
  left=25pt,
  right=0pt,
  top=3pt,
  bottom=3pt,
  arc=5pt,
  leftrule=0pt,
  rightrule=0pt,
  bottomrule=2pt,
  toprule=2pt,
  colback=bg,
  colframe=orange!70,
  enhanced,
  overlay={%
    \begin{tcbclipinterior}
    \fill[orange!20!white] (frame.south west) rectangle ([xshift=20pt]frame.north west);
    \end{tcbclipinterior}},
  #3}

\AtBeginDocument{\renewcommand*{\hbar}{{\mkern-1mu\mathchar'26\mkern-8mu\textnormal{h}}}}
\AtBeginDocument{\newcommand{\e}{\textnormal{e}}}
\AtBeginDocument{\newcommand{\im}{\textnormal{i}}}
\AtBeginDocument{\newcommand{\luz}{\textnormal{c}}}
\AtBeginDocument{\newcommand{\grav}{\textnormal{G}}}
\AtBeginDocument{\newcommand{\kb}{{\textnormal{k}_{\textnormal{B}}}}}
\newcommand{\Dd}[1]{\mathcal D #1}
\newcommand{\trp}[1]{{#1}^{\textnormal{T}}}
\newcommand{\Det}[1]{\textup{Det} #1}

\numberwithin{equation}{section}

\newtheorem{teo}{Teorema}[section]
\newtheorem{defi}{Definição}[section]
\newtheorem{lem}{Lema}[section]
\newtheorem{hip}{Hipótese}[subsection]

\pagestyle{plain}

\titleformat\section[block]
    {\bfseries\centering}
    {\Large Programa \thesection.}
    {\baselineskip}
    {\underbar{#1}}

\titleformat\subsection[block]
    {\bfseries\centering}
    {\Large \thesection.(\Alph{subsection})}
    {\baselineskip}
    {}

%\AddToHook{cmd/section/before}{\clearpage}

%\addbibresource{ref.bib}

\title
{
    \Huge EP2
}
\author
{
    \large Vicente V. Figueira, NUSP: 11809301
}
%\date{\today}

\begin{document}

\maketitle

%\tableofcontents

%%%%%%%%%%%%%%%%%%%%%%%%%%%%%%%%%%%%%%%%%%%%%%%%%%%%%%%%%%%%%

%\begin{refsection}

%\setcounter{section}{1}

\section{\Large Solução de Sistemas de Equações Lineares}

\setcounter{subsection}{1}

\subsection{}

Conforme mostrado no Item \textbf{a}, nosso problema consiste de resolver o seguinte sistema linear de equações,

\begin{align}
    \mqty[0.0 & 5.3 & -1.8\\ 11.9 & 0.0 & 1.8 \\ 1 & -1 & -1]\mqty[I_1\\ I_2\\ I_3]&=\mqty[3.1\\15.0\\0]
\end{align}

Faremos isso neste item primeiro pelo método de Eliminação de Gauss utilizando Pivoteamento Parcial. Optamos por 
representar a matrix em nosso programa como uma matrix aumentada,

\begin{align}
    \mqty[\mqty{0.0 & 5.3 & -1.8\\ 11.9 & 0.0 & 1.8 \\ 1 & -1 & -1}\mqty|\\ \\ \\|\mqty{3.1\\15.0\\0}]
\end{align}

O programa que realiza os passos do Pivoteamento pode ser visto abaixo,

\begin{mintedbox}{python}
def fTriangulacao(tMatrix):

    vDimensao = len(tMatrix) # Tamanho da matrix dada como input

    #########################################

    # Procura pela posição do Pivô

    for vColuna in range(vDimensao-1):

        vPivot = 0 
        vLinhaPivot = 0 

        for vLinha in range(vColuna, vDimensao): 

            if abs( tMatrix[vLinha][vColuna] ) > abs( vPivot ):

                vPivot = tMatrix[vLinha][vColuna]
                vLinhaPivot = vLinha

    #########################################

    # Troca linhas de lugar para realocar o Pivô
    # e printa ao final de cada permutação de linhas

        vTemp1 = tMatrix[vColuna][:]
        vTemp2 = tMatrix[vLinhaPivot][:]

        tMatrix[vColuna] = vTemp2
        tMatrix[vLinhaPivot] = vTemp1

        print("Pivoteamento:")
        print(tMatrix)

    #########################################

    # Zera os espaços abaixos da posição do Pivô
    # e printa ao final de cada passo

        for vLinha in range(vColuna + 1, vDimensao):

            vMlc = -tMatrix[vLinha][vColuna] / tMatrix[vColuna][vColuna]

            for vTempCol in range(vColuna, vDimensao + 1):

                tMatrix[vLinha][vTempCol] = tMatrix[vColuna][vTempCol] * vMlc + tMatrix[vLinha][vTempCol]

            print("Introdução de zeros abaixo do pivô:")
            print(tMatrix)

    return tMatrix

def fSolucionar(tMatrix):

    vDimensao = len(tMatrix)
    tSolucao = []

    #########################################

    # Inicia a solução como um array nulo

    for vLinhaTemp in range(vDimensao):

        tSolucao.append(0)

    #########################################

    # Realiza o método de substituição para trás
    # para obter a solução final

    for vLinha in range(vDimensao-1, -1, -1):

        vFator = 1 / tMatrix[vLinha][vLinha]
        vSoma = 0

        for vColuna in range(vLinha+1, vDimensao):

            vSoma = vSoma + tSolucao[vColuna] * tMatrix[vLinha][vColuna]

        tSolucao[vLinha] = vFator * (tMatrix[vLinha][vDimensao] - vSoma)

    return tSolucao

    #########################################

    # Inicia os valores corretos para a matrix e o vetor b

tMatrix = [[0.0,5.3,-1.8],[11.9,0.0,1.8],[1,-1,-1]]
tB = [3.1,15.0,0]
tX = []

    #########################################

    # Transforma a Matrix em aumentada

vDimensao = len(tMatrix)

for vLinha in range(vDimensao):

    tMatrix[vLinha].append(tB[vLinha])

    #########################################

    # Triangula a matrix

tMatrix = fTriangulacao(tMatrix)

print("Matrix Triangulada:")
print(tMatrix)

    #########################################

    # Resolve o sistema

tX = fSolucionar(tMatrix)

print("Solução Final:")
print(tX)
\end{mintedbox}

Este programa também printa os resulados passo a passo, que são --- Mostrando até a 4ª casa decimal --- :

\begin{align}
    &\textnormal{Pivoteamento:}\nonumber\\
    &[[11.9, 0.0, 1.8, 15.0],\nonumber\\ &[0.0, 5.3, -1.8, 3.1],\nonumber\\ &[1, -1, -1, 0]]\nonumber\\
    &\textnormal{Introdução de zeros abaixo do pivô}:\nonumber\\
    &[[11.9, 0.0, 1.8, 15.0],\nonumber\\ & [0.0, 5.3, -1.8, 3.1],\nonumber\\ & [1, -1, -1, 0]]\nonumber\\
    &\textnormal{Introdução de zeros abaixo do pivô}:\nonumber\\
    &[[11.9, 0.0, 1.8, 15.0],\nonumber\\ & [0.0, 5.3, -1.8, 3.1],\nonumber\\ & [0.0, -1.0, -1.1513, -1.2605]]\nonumber\\
    &\textnormal{Pivoteamento}:\nonumber\\
    &[[11.9, 0.0, 1.8, 15.0],\nonumber\\ & [0.0, 5.3, -1.8, 3.1],\nonumber\\ & [0.0, -1.0, -1.1513, -1.2605]]\nonumber\\
    &\textnormal{Introdução de zeros abaixo do pivô:}\nonumber\\
    &[[11.9, 0.0, 1.8, 15.0],\nonumber\\ & [0.0, 5.3, -1.8, 3.1],\nonumber\\ & [0.0, 0.0, -1.4909, -0.6756]]\nonumber\\
    &\textnormal{Matrix Triangulada:}\nonumber\\
    &[[11.9, 0.0, 1.8, 15.0],\nonumber\\ & [0.0, 5.3, -1.8, 3.1],\nonumber\\ & [0.0, 0.0, -1.4909, -0.6756]]\nonumber\\
    &\textnormal{Solução Final:}\nonumber\\
    &[1.1920, 0.7388, 0.4532]\nonumber
\end{align}

Isso nos dá um resultado da matrix triangular final como:

\begin{align}
    \mqty[ 11.9 & 0.0 & 1.8 \\ 0.0 & 5.3 & -1.8\\0 & 0 & -1.4909]
\end{align}

E com a solução do sistema sendo:

\begin{align}
    \mqty[I_1\\I_2\\I_3]=\mqty[1.1920\\ 0.7388\\ 0.4532]
\end{align}

\subsection{}

Agora iremos resolver o mesmo sistema de equações utilizando do método de Jacobi. Para isso vamos utilizar da matrix 
já com suas duas primeiras linhas permutadas. 

\begin{align}
    \mqty[ 11.9 & 0.0 & 1.8 \\ 0.0 & 5.3 & -1.8\\1 & -1 & -1]\mqty[ I_2\\I_1\\ I_3]&=\mqty[15.0\\3.1\\0]
\end{align}

O critério de parada utilizado foi a precisão de `$\epsilon=10^{-3}$' entre passos da solução com 
o ``chute'' para a solução inicial sendo $\qty[0,0,0]$. O programa pode ser visto 
abaixo,

\begin{mintedbox}{python}
# Precisão para a parada do programa
vPrecisao = 1.E-3

def fJacobi(tMatrix, tX0, tB):
    
    #########################################

    # Definição de Varráveis inicias. Erro inicial
    # é escolhido arbitrariamente

    vPasso = 0
    vErro = 10
    tXPasso = tX0[:]
    vDimensao = len(tX0)

    print("Passo: ", vPasso)
    print("Erro: --")
    print("Solução neste passo: ", tXPasso)

    #########################################

    # Loop principal

    while vErro > vPrecisao:

    # Salva o passo atual para o calculo de erros

        tXProxPasso = tXPasso[:]

    #########################################

    # Atualiza o Passo atual

        for vLinha in range(vDimensao):

            vSoma = 0

            for vColuna in range(vDimensao):

    # Ignora os elementos da diagonal

                if vColuna == vLinha:

                    continue

                vSoma = vSoma + tXProxPasso[vColuna] * tMatrix[vLinha][vColuna]

            vFator = 1 / tMatrix[vLinha][vLinha]
            tXPasso[vLinha] = vFator * (tB[vLinha] - vSoma)

    #########################################

    # Calcula o erro 

        vErro = 0

        for vLinha in range(vDimensao):

            vErroTemp = abs(tXPasso[vLinha] - tXProxPasso[vLinha])

            if vErroTemp > vErro:

                vErro = vErroTemp

    # Atualiza os contadores e printa o resultado 
    # de cada passo

        vPasso = vPasso + 1
    
        print("Passo: ", vPasso)
        print("Erro: ", vErro)
        print("Solução neste passo: ", tXPasso)

    return(tXPasso)

    #########################################

    # Define os parametros de nosso problema

tMatrix= [[11.9,0.0,1.8],[0.0,5.3,-1.8],[1,-1,-1]]
tB = [15.0,3.1,0]
tX0 = [0,0,0]

    #########################################

    # Resolve o sistema

tSolucao = fJacobi(tMatrix, tX0, tB)

print("Solução Final: ", tSolucao)
\end{mintedbox}

O retorno do programa é,
\centering
{
    \begin{tabular}{rrrrrrr}
        \hline
           Passo  &  Erro &   $\qty[I_1,I_2,I_3]$ \\
        \hline
        1
 &  1.261
 &  [1.261, 0.585, -0.0] \\
   2
 &  0.671
 &  [1.261, 0.585, 0.676] \\
   3
 &  0.229
 &  [1.158, 0.814, 0.676] \\
   4
 &  0.332
 &  [1.158, 0.814, 0.344]\\
   5
 &  0.113
 &  [1.208, 0.702, 0.344]\\
   6
 &  0.163
 &  [1.208, 0.702, 0.507]\\
   7
 &  0.055
 &  [1.184, 0.757, 0.507]\\
   8
 &  0.080
 &  [1.184, 0.757, 0.427]\\
   9
 &  0.027
 &  [1.196, 0.730, 0.427]\\
   10
 &  0.039
 &  [1.196, 0.730, 0.466]\\
   11
 &  0.013
 &  [1.190, 0.743, 0.466]\\
   12
 &  0.019
 &  [1.190, 0.743, 0.447]\\
   13
 &  0.007
 &  [1.193, 0.737, 0.447]\\
   14
 &  0.009
 &  [1.193, 0.737, 0.456]\\
   15
 &  0.003
 &  [1.191, 0.740, 0.456]\\
   16
 &  0.005
 &  [1.191, 0.740, 0.452]\\
   17
 &  0.001
 &  [1.192, 0.738, 0.452]\\
   18
 &  0.002
 &  [1.192, 0.738, 0.454]\\
   19
 &  0.0007
 &  [1.192, 0.739, 0.4543]\\
        \hline
        \end{tabular}
}

\justifying

Na tabela mostramos apenas até o 3º algarismo significativo, a solução final é,
   
$$\textnormal{Solução Final:   }  [1.1918465707994006, 0.7390614731107796, 0.45390322638173963]$$
$$\textnormal{Erro Final:   }  0.0007735890376808774$$

Que é compatível com o resultado obtido pelo método anterior.

\subsection{}

O nosso sistema é, após permutar a primeira com a segunda linha,

\begin{align}
    \mqty[ 11.9 & 0.0 & 1.8 \\ 0.0 & 5.3 & -1.8\\1 & -1 & -1]\mqty[ I_2\\I_1\\ I_3]&=\mqty[15.0\\3.1\\0]
\end{align}

Que resulta em uma matrix

\begin{align}
    \mqty[11.9 & 0.0 & 1.8 \\ 0.0 & 5.3 & -1.8\\1 & -1 & -1]
\end{align}

Assim podemos obter cada `$a_{ij}$', onde `$i$' refere a linha e $j$ a coluna. A construção da matrix `$\mathbb J$' é então,

\begin{align}
    \mqty[0 & 0 & \frac{1.8}{11.9} \\ 0 & 0 & -\frac{1.8}{5.3}\\-1 & 1 & 0 ]
\end{align}

Cujos auto-valores calculados são,

\begin{align}
    \lambda_1 &= 0\\
    \lambda_2 &= \frac{6\sqrt{542402}}{6307}\im\\
    \lambda_3 &=- \frac{6\sqrt{542402}}{6307}\im
\end{align}

E portanto o raio espectral é,

\begin{align}
    \rho_s &=\norm{\lambda_2}=\frac{6\sqrt{542402}}{6307}
\end{align}

We have then to solve for,

\begin{align}
    \rho_s^k&=10^{-3}\\
    k\ln\qty(\rho_s^k)&=-3\ln\qty(10)\\
    k&=-3\frac{\ln\qty(10)}{\ln\qty(\rho_s^k)}\\
    k&=-3\frac{\ln\qty(10)}{\ln\qty(\frac{6\sqrt{542402}}{6307})}\\
    k&=19.4
\end{align}

O que é extremamente compatível com o resultado obtido no item anterior, que convergiu em exatos 19 passos!

\subsection{}

Agora vamos utilizar do método de Gauss-Seidel, a principal diferença deste método com o método 
de Jacobi está na linha 50. Foi utilizado também uma precisão de `$10^{-3}$' e valores inicial da 
solução como sendo `$\qty[0,0,0]$'. O código pode ser visto abaixo,

\begin{mintedbox}{python}
# Precisão para a parada do Programa
vPrecisao = 1.E-3

def fGaussSeidel(tMatrix, tX0, tB):

    #########################################

    # Definição dos parâmetros do problema,
    # a escolha do erro é arbitrária

    vPasso = 0
    vErro = 10
    tXPasso = tX0[:]
    vDimensao = len(tX0)

    print("Passo: ", vPasso)
    print("Erro: --")
    print("Solução neste passo: ", tXPasso)

    #########################################

    # Loop principal do método

    while vErro > vPrecisao:

    # Salva o Valor do passo atual para 
    # calcular o erro posteriormente

        tXProxPasso = tXPasso[:]

    #########################################

    # Calculo do próximo passo

        for vLinha in range(vDimensao):

            vSoma = 0

    # Ignora os elementos da diagonal

            for vColuna in range(vDimensao):

                if vColuna == vLinha:

                    continue

    # Diferença do método é utilizar nesta soma tXPasso
    # ao invés de utilizar tXProxPasso

                vSoma = vSoma + tXPasso[vColuna] * tMatrix[vLinha][vColuna]

            vFator = 1 / tMatrix[vLinha][vLinha]
            tXPasso[vLinha] = vFator * (tB[vLinha] - vSoma)

    #########################################

    # Cálculo do erro

        vErro = 0

        for i in range(vDimensao):

            vErroTemp = abs(tXPasso[i] - tXProxPasso[i])

            if vErroTemp > vErro:

                vErro = vErroTemp

    #########################################

    # Aumenta o contador de passos e printa
    
        vPasso = vPasso + 1

        print("Passo: ", vPasso)
        print("Erro: ", vErro)
        print("Solução neste passo: ", tXPasso)

    return(tXPasso)

    #########################################

    # Define os valores do problema

tMatrix= [[11.9,0.0,1.8],[0.0,5.3,-1.8],[1,-1,-1]]
tB = [15.0,3.1,0]
tX0 = [0,0,0]

    #########################################

    # Resolve o sistema

tSolucao = fGaussSeidel(tMatrix, tX0, tB)

print("Solução Final: ", tSolucao)
\end{mintedbox}

O retorno do programa é,

\centering
{
    \begin{tabular}{rrrrrrr}
        \hline
           Passo  &  Erro &   $\qty[I_1,I_2,I_3]$ \\
        \hline
        0
        & --
        &  [0, 0, 0]\\
          1
        &  1.261
        &  [1.261, 0.585, 0.671]\\
          2
        &  0.332
        &  [1.158, 0.814, 0.344]\\
          3
        &  0.163
        &  [1.208, 0.702, 0.507]\\
          4
        &  0.080
        &  [1.184, 0.757, 0.427]\\
          5
        &  0.039
        &  [1.196, 0.730, 0.466]\\
          6
        &  0.0193
        &  [1.190, 0.743, 0.447]\\
          7
        &  0.009
        &  [1.193, 0.737, 0.456]\\
          8
        &  0.005
        &  [1.191, 0.740, 0.452]\\
          9
        &  0.002
        &  [1.192, 0.738, 0.454]\\
          10
        &  0.001
        &  [1.192, 0.739, 0.453]\\
          11
        &  0.0005
        &  [1.192, 0.739, 0.453]\\
        \hline
        \end{tabular}
}

\justifying

Novamente na tabela mostramos apenas os 3 primeiros algarismos significativos para facilitar a 
leitura. O resultado final é,

$$\textnormal{Solução Final:   }  [1.192015699509284, 0.7386817312904751, 0.453333968218809]$$
$$\textnormal{Erro Final:   } 0.0005488705301880392$$

Que é também compatível com os outros métodos, porém converge mais rapidamente.

\end{document}