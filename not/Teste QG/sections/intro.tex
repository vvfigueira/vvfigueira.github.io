\section{Introdução}

O Espaço-Tempo (A)dS é definido como, \[\mp\qty(x^{-1})^2-\qty(x^0)^2+\qty(x^1)^2+\qty(x^2)^2+\qty(x^3)^2=\mp L^2\] 
Onde o sinal de cima é para AdS, e o sinal de baixo para dS. A álgebra de isometria no embedding 5 dimensional é: 
\begin{align*}
    \comm{J^{IK}}{J^{LM}}&=-4\im\eta^{[I|[L}J^{M]|K]}\\
    \comm{J^{IK}}{J^{LM}}&=-\im\eta^{IL}J^{MK}+\im\eta^{IM}J^{LK}+\im\eta^{KL}J^{MI}-\im\eta^{KM}J^{LI}
\end{align*}
Da qual podemos interpretar como geradores de translação $J^{-1\alpha}$,
\begin{align*}
    \comm{J^{-1\alpha}}{J^{-1\beta}}&=-\im\eta^{-1-1}J^{\beta\alpha}+\im\eta^{-1\beta}J^{-1\alpha}+\im\eta^{\alpha-1}J^{\beta-1}-\im\eta^{\alpha\beta}J^{-1-1}\\
    \comm{J^{-1\alpha}}{J^{-1\beta}}&=\mp\im J^{\alpha\beta}
\end{align*}
As outras relações de comutação são,
\begin{align*}
    \comm{J^{\alpha\beta}}{J^{-1\mu}}&=-\im\eta^{\alpha-1}J^{\mu\beta}+\im\eta^{\alpha\mu}J^{-1\beta}+\im\eta^{\beta-1}J^{\mu\alpha}-\im\eta^{\beta\mu}J^{-1\alpha}\\
    \comm{J^{\alpha\beta}}{J^{-1\mu}}&=\im\eta^{\alpha\mu}J^{-1\beta}-\im\eta^{\beta\mu}J^{-1\alpha}\\
    \comm{J^{\alpha\beta}}{J^{-1\mu}}&=-2\im J^{-1[\alpha}\eta^{\beta]\mu}
\end{align*}
E,
\begin{align*}
    \comm{J^{\alpha\beta}}{J^{\mu\nu}}&=-4\im\eta^{[\alpha|[\mu}J^{\nu]|\beta]}
\end{align*}
Como os geradores de translação necessitam ter dimensão, $P^\alpha=\frac1LJ^{-1\alpha}$. A álgebra completa é,
\[\comm{J^{\alpha\beta}}{J^{\mu\nu}}=-4\im\eta^{[\alpha|[\mu}J^{\nu]|\beta]},\ \ \ \comm{J^{\alpha\beta}}{P^\mu}=-2\im P^{[\alpha}\eta^{\beta]\mu},\ \ \ \comm{P^{\alpha}}{P^{\beta}}=\mp\frac{\im}{L^2} J^{\alpha\beta}\]
O bilinear mais geral para essa álgebra é,
\[\expval{J_{\alpha\beta},J_{\mu\nu}}=\pm2\lambda\eta_{\alpha[\mu}\eta_{\nu]\beta},\ \ \ \expval{J_{\alpha\beta},P_\mu}=0,\ \ \ \expval{P_\alpha,P_\mu}=\frac{\lambda}{L^2}\eta_{\alpha\mu}\]
A ação de Einstein-Hilbert com termo de constante cosmológica é,
\begin{align*}
    S_{\textnormal{EH}}&=\frac{1}{2\kappa}\int\limits_M\star\vb{R}_{\alpha\beta}\wedge\vb e^\alpha\wedge\vb e^\beta-\frac{\Lambda}{\kappa4!}\int\limits_M\epsilon_{\alpha\beta\mu\nu}\vb e^\alpha\wedge\vb e^\beta\wedge\vb e^\mu\wedge\vb e^\nu\\
    S_{\textnormal{EH}}&=\frac{1}{2\kappa}\eta_{\mu\alpha}\eta_{\beta\nu}\int\limits_M\star\vb{R}^{\mu\nu}\wedge\vb e^\alpha\wedge\vb e^\beta\pm\frac{3\cdot 2}{\kappa L^24!}\int\limits_M\vb e^\alpha\wedge\vb e^\beta\wedge\star\qty(\vb e_\alpha\wedge\vb e_\beta)\\
    S_{\textnormal{EH}}&=\frac{\pm}{4\lambda\kappa}\pm2\lambda\eta_{\mu[\alpha}\eta_{\beta]\nu}\int\limits_M\star\vb{R}^{\mu\nu}\wedge\vb e^\alpha\wedge\vb e^\beta\pm\frac{3\cdot 2}{\lambda\kappa L^24!}\lambda\eta_{\mu[\alpha}\eta_{\beta]\nu}\int\limits_M\vb e^\mu\wedge\vb e^\nu\wedge\star\qty(\vb e^\alpha\wedge\vb e^\beta)\\
    S_{\textnormal{EH}}&=\frac{\pm}{4\lambda\kappa}\expval{J_{\mu\nu},J_{\alpha\beta}}\int\limits_M\star\vb{R}^{\mu\nu}\wedge\vb e^\alpha\wedge\vb e^\beta+\frac{3}{\lambda\kappa L^24!}\expval{J_{\mu\nu},J_{\alpha\beta}}\int\limits_M\vb e^\mu\wedge\vb e^\nu\wedge\star\qty(\vb e^\alpha\wedge\vb e^\beta)\\
    S_{\textnormal{EH}}&=\frac{\pm}{4\lambda\kappa}\pm\im L ^2\expval{J_{\mu\nu},\comm{P_\alpha}{P_\beta}}\int\limits_M\star\vb{R}^{\mu\nu}\wedge\vb e^\alpha\wedge\vb e^\beta+\frac{3\qty(\pm)^2\im^2L^4}{\lambda\kappa L^24!}\expval{\comm{P_\mu}{P_\nu},\comm{P_\alpha}{P_\beta}}\int\limits_M\vb e^\mu\wedge\vb e^\nu\wedge\star\qty(\vb e^\alpha\wedge\vb e^\beta)\\
    S_{\textnormal{EH}}&=-\frac{L^2}{4\lambda\kappa}\expval{\im J_{\mu\nu},\comm{\im P_\alpha}{\im P_\beta}}\int\limits_M\star\vb{R}^{\mu\nu}\wedge\vb e^\alpha\wedge\vb e^\beta-\frac{L^2}{8\lambda\kappa}\expval{\comm{\im P_\mu}{\im P_\nu},\comm{\im P_\alpha}{\im P_\beta}}\int\limits_M\vb e^\mu\wedge\vb e^\nu\wedge\star\qty(\vb e^\alpha\wedge\vb e^\beta)\\
    S_{\textnormal{EH}}&=-\frac{L^2}{2\lambda\kappa}\int\limits_M\expval{\star\vb{R}\ \wedgecomma\ \wedgecomm{\vb e}{\vb e}}-\frac{L^2}{8\lambda\kappa}\int\limits_M\expval{\wedgecomm{\vb e}{\vb e}\ \wedgecomma \star\wedgecomm{\vb e}{\vb e}}
\end{align*}
Com $\vb R=\frac12\im J_{\mu\nu}\vb R^{\mu\nu}$ e $\vb e=\im P_\mu\vb e^\mu$. Note,
\begin{align*}
    \vb R&=\frac12\im\vb R_{\alpha\beta}J^{\alpha\beta}=\frac12\vb d\boldsymbol\omega_{\alpha\beta}\im J^{\alpha\beta}+\frac12\tensor{{\boldsymbol\omega}}{_\alpha^\rho}\wedge\tensor{{\boldsymbol\omega}}{_\rho_\beta}\im J^{\alpha\beta}=\vb d\boldsymbol\omega+\frac12\tensor{{\boldsymbol\omega}}{_\mu_\nu}\wedge\tensor{{\boldsymbol\omega}}{_\rho_\sigma}\eta^{\rho\nu}\im J^{\mu\sigma}=\vb d\boldsymbol\omega+\frac18\tensor{{\boldsymbol\omega}}{_\mu_\nu}\wedge\tensor{{\boldsymbol\omega}}{_\rho_\sigma}4\im \eta^{[\rho|[\nu}J^{\mu]|\sigma]}\\
    \vb R&=\vb d\boldsymbol\omega-\frac18\tensor{{\boldsymbol\omega}}{_\mu_\nu}\wedge\tensor{{\boldsymbol\omega}}{_\rho_\sigma}\comm{J^{\rho\sigma}}{J^{\nu\mu}}=\vb d\boldsymbol\omega+\frac12\tensor{{\boldsymbol\omega}}{_\mu_\nu}\wedge\tensor{{\boldsymbol\omega}}{_\rho_\sigma}\comm{\frac\im 2J^{\mu\nu}}{\frac\im2J^{\rho\sigma}}=\vb d\boldsymbol\omega+\frac12\wedgecomm{\boldsymbol\omega}{\boldsymbol\omega}
\end{align*}
Seja então,
\begin{align*}
    \vb F&=\vb d\qty(\boldsymbol \omega+\vb e)+\frac12\wedgecomm{\boldsymbol \omega+\vb e}{\boldsymbol \omega+\vb e}\\
    \vb F&=\vb d\boldsymbol \omega+\frac12\wedgecomm{\boldsymbol \omega}{\boldsymbol \omega}+\frac12\wedgecomm{\vb e}{\vb e}+\vb d\vb e+\wedgecomm{\boldsymbol \omega}{\vb e}\\
    \vb F&=\vb R+\frac12\wedgecomm{\vb e}{\vb e}+\vb T
\end{align*}
Logo, 
\begin{align*}
    \tilde S&=\int\limits_{M}\expval{\vb F\ \wedgecomma \star \vb F}\\
    \tilde S&=\int\limits_{M}\expval{\vb R\ \wedgecomma \star \vb R}+\frac12\int\limits_{M}\expval{\vb R\ \wedgecomma \star \wedgecomm{\vb e}{\vb e}}+\frac12\int\limits_{M}\expval{ \wedgecomm{\vb e}{\vb e}\ \wedgecomma \star\vb R}+\int\limits_{M}\expval{\vb T\ \wedgecomma \star \vb T}+\frac14\int\limits_{M}\expval{\wedgecomm{\vb e}{\vb e}\ \wedgecomma \star \wedgecomm{\vb e}{\vb e}}\\
    \tilde S&=\int\limits_{M}\expval{\vb R\ \wedgecomma \star \vb R}+\int\limits_{M}\expval{\vb R\ \wedgecomma \star \wedgecomm{\vb e}{\vb e}}+\frac14\int\limits_{M}\expval{\wedgecomm{\vb e}{\vb e}\ \wedgecomma \star \wedgecomm{\vb e}{\vb e}}+\int\limits_{M}\expval{\vb T\ \wedgecomma \star \vb T}
\end{align*}
Isto é, a ação de Yang-Mills para o grupo de isometria de (A)dS é a ação de Einstein-Hilbert com os termos adicionais do tensor de Riemann quadrado 
e o tensor de torção quadrado. A equação de movimento para $\boldsymbol \omega$ é,
\begin{align*}
    0=\delta_{\boldsymbol\omega}\tilde S&=2\int\limits_{M}\expval{\delta_{\boldsymbol \omega}\vb R\ \wedgecomma \star \vb R}+\int\limits_{M}\expval{\delta_{\boldsymbol \omega}\vb R\ \wedgecomma \star \wedgecomm{\vb e}{\vb e}}+2\int\limits_{M}\expval{\delta_{\boldsymbol \omega}\vb T\ \wedgecomma \star \vb T}\\
    0&=2\int\limits_{M}\expval{\vb d\delta\boldsymbol \omega+\wedgecomm{\delta\boldsymbol\omega}{\boldsymbol\omega}\ \wedgecomma \star\qty( \vb R+\frac12\wedgecomm{\vb e}{\vb e})}+2\int\limits_{M}\expval{\wedgecomm{\delta\boldsymbol\omega}{\vb e}\ \wedgecomma \star \vb T}\\
    0&=\int\limits_{M}\expval{\delta\boldsymbol \omega\ \wedgecomma\ \vb d\star\qty( \vb R+\frac12\wedgecomm{\vb e}{\vb e})+\wedgecomm{\boldsymbol\omega}{\star\qty( \vb R+\frac12\wedgecomm{\vb e}{\vb e})}+\wedgecomm{\vb e}{\star\vb T}}\\
    0&=\vb d_\nabla\star \vb R+\frac12\vb d_\nabla\star\wedgecomm{\vb e}{\vb e}+\wedgecomm{\vb e}{\star\vb d_\nabla \vb e}
\end{align*}