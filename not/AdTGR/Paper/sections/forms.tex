\section{The Einstein-Hilbert Action in the Form Language}

We'll begin with a quick recap of the vielbein/spin connection formalism \cite{wald,carlip,gravitation,donnay}. For now we'll keep the discussion general in $D$ dimensions, and, 
only later on, we'll go to the special case of $D=2+1$. The vielbein\footnote{We'll denote the \textit{vector} --- (1,0) tensor --- vielbein with a tilde, only to be distinguishable from the 
associated \textit{covetor} --- (0,1) tensor --- vielbein, which we'll denote without the tilde due to being way more important to us.}, $\tilde{\vb e}_\mu$, are a basis of the vector field space $\mathfrak X\qty(M)$. Notice, 
the index $\mu$ is only indexing which vector from the ones present in the basis we are talking about. It isn't a coordinate index --- one that is 
related to a specific component decomposition in a specific chart ---, as this is not a coordinate basis --- i.e. $\boldsymbol\partial_a$ ---. We can 
in fact do miracles with it, as diagonalize the metric,
\begin{align*}
    \textnormal{diag}\mqty(-1& 1&\cdots&1)=\eta_{\mu\nu}&=\vb g\qty(\tilde{\vb e}_\mu,\tilde{\vb e}_\nu)=g_{ab}\vb d{x^a}\qty(\tilde{\vb e}_\mu)\otimes\vb d{x^b}\qty(\tilde{\vb e}_\nu)\\
    \eta_{\mu\nu}&=g_{ab}\tensor{e}{_\mu^c}\tensor{e}{_\nu^d}\vb d{x^a}\qty(\boldsymbol\partial_c)\otimes\vb d{x^b}\qty(\boldsymbol\partial_d)\\
    \eta_{\mu\nu}&=g_{ab}\tensor{e}{_\mu^a}\tensor{e}{_\nu^b}
\end{align*}

The whole point of introducing the vielbein is switch the degrees of freedom from the 
metric to the inertial frame basis, which can be seen as a downgrade, due to this process enlarging the number of degrees of freedom from $\frac12D\qty(D+1)$ 
to $D^2$. This is only the naive counting, without considering the redundancies, as the number of physical degrees of freedom must be the same. 
For this to be true, is only possible if we enlarge also the redundancies, to kill the extra degrees of freedom we introduced. As mentioned, this could be seen as 
not desirable, but, for us this turn out to be essencial. What are these new redundancies? They're the choice of labeling $\mu$ in $\tilde{\vb e}_\mu$. As long as the 
new relabel also respect the defining property of the vielbein, it's a redundant transformation. Notice that these transformations are exactly local Lorentz ones, that is, 
given a set of functions $\tensor{\Lambda}{^\mu_\nu}:M\rightarrow\mathbb R$, a new set of vector fields $\tensor{\Lambda}{^\mu_\nu}\tilde{\vb e}_\mu$ is a vielbein iff, 
\begin{align*}
    \vb g\qty(\tensor{\Lambda}{^\alpha_\mu}\tilde{\vb e}_\alpha,\tensor{\Lambda}{^\beta_\nu}\tilde{\vb e}_\beta)&=\tensor{\Lambda}{^\alpha_\mu}\tensor{\Lambda}{^\beta_\nu}\vb g\qty(\tilde{\vb e}_\alpha,\tilde{\vb e}_\beta)\\
    \eta_{\mu\nu}&=\tensor{\Lambda}{^\alpha_\mu}\tensor{\Lambda}{^\beta_\nu}\eta_{\alpha\beta}
\end{align*}
In other words, $\tensor{\Lambda}{^\mu_\nu}$ must be a local $SO\qty(D-1,1)$ element\footnote{Actually, the group is $O\qty(D-1,1)$, but, we'll only be interested in the orientation preserving transformations.}. With this new redundancy taken into account, the degrees of freedom match, $D^2$ 
from the vielbein, minus $D$ diffeomorphisms $\phi^\ast\tilde{\vb e}_\mu\sim\tilde{\vb e}_\mu$, $D$ Bianchi identities and $\frac12D\qty(D-1)$ local $SO\qty(D-1,1)$ transformations, giving $D^2-2D-\frac12D\qty(D-1)=\frac12D\qty(D-3)$, exactly 
the same counting using only the metric! Notice that now the redundancies are diffeomorphisms and local $SO\qty(D-1,1)$ transformations. This already seems like we're gauging the whole Poincaré group, which will come up later.

From the condition that $\vb g$ must be non-degenerate, we get that the matrix of components $\tensor{e}{_\mu^a}$ must be invertible. This ensures 
the existence of $\tensor{e}{^\mu_a}$ ,from which we can construct the dual vector field $\vb e^\mu=\tensor{e}{^\mu_a}\vb d{x^a}$. This certify that 
we have a basis of the whole tensor space, so it's possible to decompose any tensor in it,
\begin{align*}
    \eta_{\mu\nu}\vb e^\mu\otimes\vb e^\nu&=\vb g\qty(\tilde{\vb e}_\mu,\tilde{\vb e}_\nu)\vb e^\mu\otimes\vb e^\nu=\vb g
\end{align*}

Up to now we have been considering a metric compatible torsionless affine connection, but, this turns out to not be the optimal choice, as for this kind of connection 
there is an differential/algebraical restraint between the connection and the metric. At least that's what we should expect in a coordinate basis. What we would like is 
to have a connection linearly independent of the metric/vielbein, but without sacrificing the metricity condition, this is completely hopeless in a 
coordinate basis. In a non-coordinate basis this is achievable! We just have to remind that, an affine connection can be defined in any basis of $\mathfrak X\qty(M)$. What 
is usually done is $\nabla_{\vb X}\boldsymbol\partial_b=X^a\tensor{\Gamma}{_a^c_b}\boldsymbol\partial_c$, but it's much more interesting to define it with respect to the vielbein 
basis,
\begin{align*}
    \nabla_{\vb X}\qty(\tilde{\vb e}_\nu)&=\tensor{{\boldsymbol\omega(\vb X)}}{^\mu_\nu}\tilde{\vb e}_\mu=X^a\tensor{\omega}{_a^\mu_\nu}\tilde{\vb e}_\mu
\end{align*}
Here $\tensor{{\boldsymbol\omega}}{^\mu_\nu}=\tensor{\omega}{_a^\mu_\nu}\vb dx^a$ is named the spin connection. It can be seen as a $\mathfrak{gl}\qty(D-1,1)$--valued $(0,1)$ tensor, 
or, as we'll adopt here, a $\mathfrak{gl}\qty(D-1,1)$--valued $1$--form.
The advantage of working with the vielbein is that the Lorentz index $\mu$ 
does not change upon coordinate/chart/diffeomorphism transformations, it acts as if was an internal symmetry. Thus, 
$\tensor{{\boldsymbol\omega}}{^\mu_\nu}$ do transform exactly as a tensor should. This is already a enormous dichotomy 
with the standard formulation, where the Christoffel connection doesn't transform in a good manner. 
Now we'll impose the metric compatibility of the connection. 
This is another scenario where the vielbein formalism come to hand, as this condition imposes no additional differential/algebraical constrains among the vielbein and spin connection.
\begin{align*}
    \vb g\qty(\tilde{\vb e}_\alpha,\nabla_{\vb X}\qty(\tilde{\vb e}_\nu))&=X^a\tensor{\omega}{_a^\mu_\nu}\vb g\qty(\tilde{\vb e}_\alpha,\tilde{\vb e}_\mu),\ \ \ \textnormal{symmetrize }\alpha\leftrightarrow\nu\\
    \vb g\qty(\tilde{\vb e}_\alpha,\nabla_{\vb X}\qty(\tilde{\vb e}_\nu))+\vb g\qty(\tilde{\vb e}_\nu,\nabla_{\vb X}\qty(\tilde{\vb e}_\alpha))&=X^a\tensor{\omega}{_a^\mu_\nu}\vb g\qty(\tilde{\vb e}_\alpha,\tilde{\vb e}_\mu)+X^a\tensor{\omega}{_a^\mu_\alpha}\vb g\qty(\tilde{\vb e}_\nu,\tilde{\vb e}_\mu),\ \ \ \textnormal{metric symmetry}\\
    \vb g\qty(\tilde{\vb e}_\alpha,\nabla_{\vb X}\qty(\tilde{\vb e}_\nu))+\vb g\qty(\nabla_{\vb X}\qty(\tilde{\vb e}_\alpha),\tilde{\vb e}_\nu)&=X^a\tensor{\omega}{_a_\alpha_\nu}+X^a\tensor{\omega}{_a_\nu_\alpha},\ \ \ \textnormal{Leibnitz rule}\\
    \nabla_{\vb X}\qty(\vb g\qty(\tilde{\vb e}_\alpha,\tilde{\vb e}_\nu))-\nabla_{\vb X}\qty(\vb g)\qty(\tilde{\vb e}_\alpha,\tilde{\vb e}_\nu)&=X^a\tensor{\omega}{_a_\alpha_\nu}+X^a\tensor{\omega}{_a_\nu_\alpha},\ \ \ \nabla_{\vb X}\qty(\eta_{\alpha\nu})=0\\
    -\nabla_{\vb X}\qty(\vb g)\qty(\tilde{\vb e}_\alpha,\tilde{\vb e}_\nu)&=X^a\tensor{\omega}{_a_\alpha_\nu}+X^a\tensor{\omega}{_a_\nu_\alpha},\ \ \ \textnormal{metricity}\\
    -\tensor{\omega}{_a_\nu_\alpha}&=\tensor{\omega}{_a_\alpha_\nu}
\end{align*}

That is, the metric compatible spin connection is anti-symmetric in the non-coordinate indices, exactly the property 
satisfied by the generators of the $SO\qty(D-1,1)$ group. Thus, the spin connection can be seen as a $\mathfrak{so}\qty(D-1,1)$-valued 1-form. It has 
everything in its favor to be interpreted as a gauge field of the $SO\qty(D-1,1)$ group. To confirm this, notice how it changes under a gauge transformation 
of the vielbein basis $\tilde{\vb e}_\nu\rightarrow\tensor{\Lambda}{^\mu_\nu}\tilde{\vb e}_\mu$:
\begin{align*}
    \nabla_{\vb X}\qty(\tensor{\Lambda}{^\mu_\nu}\tilde{\vb e}_\mu)&=\nabla_{\vb X }\qty(\tensor{\Lambda}{^\mu_\nu})\tilde{\vb e}_\mu+\tensor{\Lambda}{^\mu_\nu}\nabla_{\vb X}\qty(\tilde{\vb e}_\mu),\ \ \ \textnormal{connection definition}\\
    \nabla_{\vb X}\qty(\tensor{\Lambda}{^\mu_\nu}\tilde{\vb e}_\mu)&=\vb X\qty(\tensor{\Lambda}{^\mu_\nu})\tilde{\vb e}_\mu+\tensor{\Lambda}{^\mu_\nu}\tensor{{\boldsymbol\omega\qty(\vb X)}}{^\alpha_\mu}\tilde{\vb e}_\alpha,\ \ \ \tensor{{\Lambda^{-1}}}{^\beta_\alpha}\tensor{\Lambda}{^\sigma_\beta}=\tensor{\delta}{^\sigma_\alpha}\\
    \nabla_{\vb X}\qty(\tensor{\Lambda}{^\mu_\nu}\tilde{\vb e}_\mu)&=\vb d\tensor{\Lambda}{^\mu_\nu}\qty(\vb X)\tensor{{\Lambda^{-1}}}{^\beta_\mu}\tensor{\Lambda}{^\sigma_\beta}\tilde{\vb e}_\sigma+\tensor{\Lambda}{^\mu_\nu}\tensor{{\boldsymbol\omega\qty(\vb X)}}{^\alpha_\mu}\tensor{{\Lambda^{-1}}}{^\beta_\alpha}\tensor{\Lambda}{^\sigma_\beta}\tilde{\vb e}_\sigma\\
    \nabla_{\vb X}\qty(\tensor{\Lambda}{^\mu_\nu}\tilde{\vb e}_\mu)&=\qty(\tensor{{\Lambda^{-1}}}{^\beta_\mu}\vb d{\tensor{\Lambda}{^\mu_\nu}}+\tensor{{\Lambda^{-1}}}{^\beta_\alpha}\tensor{{\boldsymbol\omega}}{^\alpha_\mu}\tensor{\Lambda}{^\mu_\nu})\qty(\vb X)\tensor{\Lambda}{^\sigma_\beta}\tilde{\vb e}_\sigma
\end{align*}

Hence, under a gauge transformation of $\tilde{\vb e}_\nu\rightarrow\tensor{\Lambda}{^\mu_\nu}\tilde{\vb e}_\mu$, the spin connection transforms exactly as a connection of the gauge group $SO\qty(D-1,1)$, 
$\boldsymbol \omega\rightarrow \Lambda^{-1}\boldsymbol\omega\Lambda+\Lambda^{-1}\vb d{\Lambda}$, which is what we're looking for. Before pursuing further the gauged translations, we're going to obtain a new interpretation 
for the Riemann tensor, using what we just learned from the spin connection. Notice, the usual interpretation of the Riemann tensor, is of it being a $(1,3)$ tensor, but, as naturally --- without the need for a metric compatible 
torsionless connection --- it's anti-symmetric in the first two entries, we can switch the point of view from a $(1,3)$ tensor to a $(1,1)$ tensor valued 2--form, or, in an even better way, an End$\qty(\mathfrak X\qty(M))$--valued 2--form, 
which, when decomposed in the non-coordinate basis, will turn out to be a $\mathfrak{so}\qty(D-1,1)$--valued 2--form, as we'll shown now. Starting from the definition\footnote{Here, care must be taken. In our conventions, the covariant derivative acts non-trivially only 
in vectors, and acts as a normal derivative in functions. That is, for $X^a$ being \textit{components} of a vector, $\nabla_{\vb Y}X^a=\vb Y\qty(X^a)=Y^b\partial_b X^a$, in contrast to $\nabla_{\vb Y}\boldsymbol\partial_a=Y^b\tensor{\Gamma}{_b^c_a}\boldsymbol\partial_c$.},
\begin{align*}
    \textbf{Riem}\qty(\vb X,\vb Y)\tilde{\vb e}_\mu&=\qty(\nabla_{\vb X}\nabla_{\vb Y}-\nabla_{\vb Y}\nabla_{\vb X}-\nabla_{\comm{\vb X}{\vb Y}})\tilde{\vb e}_\mu\\
    \textbf{Riem}\qty(\vb X,\vb Y)\tilde{\vb e}_\mu&=\nabla_{\vb X}\qty(Y^b\tensor{\omega}{_b^\nu_\mu}\tilde{\vb e}_\nu)-\nabla_{\vb Y}\qty(X^a\tensor{\omega}{_a^\nu_\mu}\tilde{\vb e}_\nu)-\comm{\vb X}{\vb Y}^b\tensor{\omega}{_b^\nu_\mu}\tilde{\vb e}_\nu\\
    \textbf{Riem}\qty(\vb X,\vb Y)\tilde{\vb e}_\mu&=Y^b\nabla_{\vb X}\qty(\tensor{\omega}{_b^\nu_\mu}\tilde{\vb e}_\nu)-X^a\nabla_{\vb Y}\qty(\tensor{\omega}{_a^\nu_\mu}\tilde{\vb e}_\nu)+\nabla_{\vb X}\qty(Y^b)\tensor{\omega}{_b^\nu_\mu}\tilde{\vb e}_\nu-\nabla_{\vb Y}\qty(X^b)\tensor{\omega}{_b^\nu_\mu}\tilde{\vb e}_\nu-\comm{\vb X}{\vb Y}^b\tensor{\omega}{_b^\nu_\mu}\tilde{\vb e}_\nu\\
    \textbf{Riem}\qty(\vb X,\vb Y)\tilde{\vb e}_\mu&=X^aY^b\nabla_{a}\qty(\tensor{\omega}{_b^\nu_\mu}\tilde{\vb e}_\nu)-X^aY^b\nabla_{b}\qty(\tensor{\omega}{_a^\nu_\mu}\tilde{\vb e}_\nu)+\qty(\nabla_{\vb X}\qty(Y^b)-\nabla_{\vb Y}\qty(X^b))\tensor{\omega}{_b^\nu_\mu}\tilde{\vb e}_\nu-\comm{\vb X}{\vb Y}^b\tensor{\omega}{_b^\nu_\mu}\tilde{\vb e}_\nu\\
    \textbf{Riem}\qty(\vb X,\vb Y)\tilde{\vb e}_\mu&=X^aY^b\nabla_{a}\qty(\tensor{\omega}{_b^\nu_\mu})\tilde{\vb e}_\nu+X^aY^b\tensor{\omega}{_b^\nu_\mu}\tensor{\omega}{_a^\alpha_\nu}\tilde{\vb e}_\alpha-X^aY^b\nabla_{b}\qty(\tensor{\omega}{_a^\nu_\mu})\tilde{\vb e}_\nu-X^aY^b\tensor{\omega}{_a^\nu_\mu}\tensor{\omega}{_b^\alpha_\nu}\tilde{\vb e}_\alpha\\
    &\quad\quad\quad+\qty(X^a\partial_a Y^b-Y^a\partial_aX^b)\tensor{\omega}{_b^\nu_\mu}\tilde{\vb e}_\nu-\qty(X^a\partial_aY^b-Y^a\partial_a X^b)\tensor{\omega}{_b^\nu_\mu}\tilde{\vb e}_\nu\\
    \textbf{Riem}\qty(\vb X,\vb Y)\tilde{\vb e}_\mu&=X^aY^b\left(\partial_{a}\qty(\tensor{\omega}{_b^\nu_\mu})+\tensor{\omega}{_b^\alpha_\mu}\tensor{\omega}{_a^\nu_\alpha}-\partial_{b}\qty(\tensor{\omega}{_a^\nu_\mu})-\tensor{\omega}{_a^\alpha_\mu}\tensor{\omega}{_b^\nu_\alpha}\right)\tilde{\vb e}_\nu\\
    \textbf{Riem}\qty(\vb X,\vb Y)\tilde{\vb e}_\mu&=X^aY^b\left(\partial_{a}\tensor{\omega}{_b^\nu_\mu}-\partial_{b}\tensor{\omega}{_a^\nu_\mu}+\tensor{\omega}{_a^\nu_\alpha}\tensor{\omega}{_b^\alpha_\mu}-\tensor{\omega}{_b^\nu_\alpha}\tensor{\omega}{_a^\alpha_\mu}\right)\tilde{\vb e}_\nu\\
    X^aY^b\tensor{R}{_a_b^\nu_\mu}\tilde{\vb e}_\nu&=X^aY^b\tensor{\left(\vb d{\boldsymbol\omega}+\boldsymbol\omega\wedge\boldsymbol\omega\right)}{_a_b^\nu_\mu}\tilde{\vb e}_\nu
\end{align*}

This settles down the interpretation of the Riemann tensor being a $\mathfrak{so}\qty(D-1,1)$--valued 2--form, and also provides a striking resemblance to the usual gauge force field in non-abelian 
theories, $\vb F=\vb d\vb A+\vb A\wedge \vb A$. It's also easily related to the usual coordinate Riemann tensor,
\begin{align*}
    \textbf{Riem}\qty(\vb X,\vb Y)\tilde{\vb e}_\mu&=X^aY^b\tensor{R}{_a_b^\nu_\mu}\tilde{\vb e}_\nu\\
    \textbf{Riem}\qty(\vb X,\vb Y)\qty(\tensor{e}{_\mu^e}\boldsymbol\partial_e)&=X^aY^b\tensor{R}{_a_b^\nu_\mu}\tensor{e}{_\nu^c}\boldsymbol\partial_c\\
    \tensor{e}{_\mu^e}\textbf{Riem}\qty(\vb X,\vb Y)\boldsymbol\partial_e&=X^aY^b\tensor{R}{_a_b^\nu_\mu}\tensor{e}{_\nu^c}\boldsymbol\partial_c\\
    \tensor{e}{_\mu^e}X^aY^b\tensor{R}{_a_b^c_e}\boldsymbol\partial_c&=X^aY^b\tensor{R}{_a_b^\nu_\mu}\tensor{e}{_\nu^c}\boldsymbol\partial_c\\
    \tensor{e}{^\mu_d}\tensor{e}{_\mu^e}\tensor{R}{_a_b^c_e}&=\tensor{e}{^\mu_d}\tensor{R}{_a_b^\nu_\mu}\tensor{e}{_\nu^c}\\
    \tensor{R}{_a_b^c_d}&=\tensor{e}{^\mu_d}\tensor{R}{_a_b^\nu_\mu}\tensor{e}{_\nu^c}
\end{align*}
which we'll use when rewriting the Einstein-Hilbert Action. It doesn't hurt to stress that we are only assuming metricity, and not torsionless --- which will come about naturally later ---. 
Lastly, we define our Riemann Curvature 2-form,
\begin{align*}
    \tensor{{\vb R}}{^\nu_\mu}&=\frac12\tensor{R}{_a_b^\nu_\mu}\vb d{x^a}\wedge\vb d{x^b}=\tensor{\vb d{\boldsymbol\omega}}{^\nu_\mu}+\tensor{{\boldsymbol\omega}}{^\nu_\alpha}\wedge\tensor{{\boldsymbol\omega}}{^\alpha_\mu}%\tensor{{\vb R}}{^\nu_\mu}&=\frac12\tensor{R}{_a_b^\nu_\mu}\tensor{e}{_\alpha^a}\tensor{e}{^\alpha_c}\tensor{e}{_\beta^b}\tensor{e}{^\beta_d}\dd{x^c}\wedge\dd{x^d}\\%\tensor{{\vb R}}{^\nu_\mu}&=\frac12\tensor{R}{_a_b^\nu_\mu}\tensor{e}{_\alpha^a}\tensor{e}{_\beta^b}{\tilde{\vb e}^\alpha}\wedge{\tilde{\vb e}^\beta}
\end{align*}
We'll talk more about the absence of $\vb e^\mu$ later --- which signifies this is the gauge force field of only the inhomogeneous part of the Poincaré group ---.

Now we start the real deal of rewriting the 
Einstein-Hilbert Action in terms of the curvature 2--form, and the vielbein 1--form. Starting with the volume form\footnote{While it's widely known, the Levi-Civita symbol, $\epsilon_{a_1\cdots a_D}$, in a coordinate basis isn't a tensor. 
For a non-coordinate basis, $\epsilon_{\mu_1\cdots\mu_D}$, it does not need additional factors of the metric determinant, one more usefulness of the vielbein basis.},
\begin{align*}
    \dd[D]{x}\sqrt{\abs{g}}&=\sqrt{\abs{\Det\qty[g_{ab}]}}\dd{x^0}\wedge\cdots\wedge\dd{x^{D-1}}\\
    \dd[D]{x}\sqrt{\abs{g}}&=\sqrt{\abs{\Det\qty[\tensor{e}{^\mu_a}\eta_{\mu\nu}\tensor{e}{^\nu_b}]}}\dd{x^0}\wedge\cdots\wedge\dd{x^{D-1}}\\
    \dd[D]{x}\sqrt{\abs{g}}&=\sqrt{\abs{\Det\qty[\tensor{e}{^\mu_a}]\Det\qty[\eta_{\mu\nu}]\Det\qty[\tensor{e}{^\nu_b}]}}\dd{x^0}\wedge\cdots\wedge\dd{x^{D-1}}\\
    \dd[D]{x}\sqrt{\abs{g}}&=\sqrt{\qty(\Det\qty[\tensor{e}{^\mu_a}])^2}\dd{x^0}\wedge\cdots\wedge\dd{x^{D-1}}\\
    \dd[D]{x}\sqrt{\abs{g}}&=\Det\qty[\tensor{e}{^\mu_a}]\dd{x^0}\wedge\cdots\wedge\dd{x^{D-1}}\\
    \dd[D]{x}\sqrt{\abs{g}}&=\epsilon_{\mu_1\cdots\mu_{D}}\tensor{e}{^{\mu_0}_0}\cdots\tensor{e}{^{\mu_{D}}_{D-1}}\dd{x^0}\wedge\cdots\wedge\dd{x^{D-1}}\\
    \dd[D]{x}\sqrt{\abs{g}}&=\epsilon_{\mu_1\cdots\mu_{D}}\tensor{e}{^{\mu_0}_0}\dd{x^0}\wedge\cdots\wedge\tensor{e}{^{\mu_{D}}_{D-1}}\dd{x^{D-1}}\\
    \dd[D]{x}\sqrt{\abs{g}}&=\frac{1}{D!}\epsilon_{\mu_1\cdots\mu_{D}}\tensor{e}{^{\mu_1}_{a_1}}\dd{x^{a_1}}\wedge\cdots\wedge\tensor{e}{^{\mu_{D}}_{a_{D}}}\dd{x^{a_{D}}}\\
    \dd[D]{x}\sqrt{\abs{g}}&=\frac{1}{D!}\epsilon_{\mu_1\cdots\mu_{D}}\vb e^{\mu_1}\wedge\cdots\wedge\vb e^{\mu_{D}}\numberthis\label{VolumeForm}
\end{align*}

The other ingredient we need, is the Ricci scalar,
\begin{align*}
    R&=g^{ab}\tensor{R}{_c_b^c_a}\\
    R&=\tensor{e}{_\rho^a}\tensor{e}{^\rho^b}\tensor{R}{_c_b_d_a}\tensor{e}{_\alpha^c}\tensor{e}{^\alpha^d},\ \ \ \textnormal{vielbein definition}\\
    R&=\eta^{\rho\sigma}\eta^{\alpha\beta}\tensor{e}{_\rho^a}\tensor{e}{_\sigma^b}\tensor{R}{_c_b_d_a}\tensor{e}{_\alpha^c}\tensor{e}{_\beta^d}\\
    R&=\eta^{\rho\sigma}\eta^{\alpha\beta}\tensor{R}{_c_b_\beta_\rho}\tensor{e}{_\alpha^c}\tensor{e}{_\sigma^b},\ \ \ \textnormal{anti-symmetry in the first two index}\\
    R&=\frac12\qty(\eta^{\rho\sigma}\eta^{\alpha\beta}-\eta^{\rho\alpha}\eta^{\sigma\beta})\tensor{R}{_c_b_\beta_\rho}\tensor{e}{_\alpha^c}\tensor{e}{_\sigma^b}\\
    R&=\frac{-1}{2\qty(D-2)!}\tensor{\epsilon}{^{\nu_1}^\cdots^{\nu_{D-2}}^\beta^\rho}\tensor{\epsilon}{_{\nu_1}_\cdots_{\nu_{D-2}}^\alpha^\sigma}\tensor{R}{_c_b_\beta_\rho}\tensor{e}{_\alpha^c}\tensor{e}{_\sigma^b}\numberthis\label{RicciScalar}
\end{align*}

Combining these two results, \eqref{VolumeForm}, \eqref{RicciScalar}, we can rewrite the EH Action in the forms language. That is, the expression we're expecting to obtain is an integral over a $D$--form. In the middle of the computation, 
we'll need to introduce also the Hodge star operator,
\begin{align*}
    S_{\textnormal{EH}}&=\frac{1}{2\kappa}\int\limits_M\dd[D]{x}\sqrt{\abs{g}}R\\
    S_{\textnormal{EH}}&=\frac{-1}{4D!\qty(D-2)!\kappa}\int\limits_M\vb e^{\mu_1}\wedge\cdots\wedge\vb e^{\mu_{D}}\epsilon_{\mu_1\cdots\mu_{D}}\tensor{\epsilon}{^{\nu_1}^\cdots^{\nu_{D-2}}^\beta^\rho}\tensor{\epsilon}{_{\nu_1}_\cdots_{\nu_{D-2}}^\alpha^\sigma}\tensor{R}{_c_b_\beta_\rho}\tensor{e}{_\alpha^c}\tensor{e}{_\sigma^b}\\
    S_{\textnormal{EH}}&=\frac{1}{4\qty(D-2)!\kappa}\int\limits_M\vb e^{\mu_1}\wedge\cdots\wedge\vb e^{\mu_{D}}\tensor{\eta}{_{\mu_1}^{[\nu_1}}\cdots\tensor{\eta}{_{\mu_{D-2}}^{\nu_{D-2}}}\tensor{\eta}{_{\mu_{D-1}}^{\beta}}\tensor{\eta}{_{\mu_{D}}^{\rho]}}\tensor{\epsilon}{_{\nu_1}_\cdots_{\nu_{D-2}}^\alpha^\sigma}\tensor{R}{_c_b_\beta_\rho}\tensor{e}{_\alpha^c}\tensor{e}{_\sigma^b}\\
    S_{\textnormal{EH}}&=\frac{1}{4\qty(D-2)!\kappa}\int\limits_M\vb e^{\nu_1}\wedge\cdots\wedge\vb e^{\nu_{D-2}}\wedge\vb e^{\beta}\wedge\vb e^{\rho}\tensor{\epsilon}{_{\nu_1}_\cdots_{\nu_{D-2}}^\alpha^\sigma}\tensor{R}{_c_b_\beta_\rho}\tensor{e}{_\alpha^c}\tensor{e}{_\sigma^b}\\
    S_{\textnormal{EH}}&=\frac{1}{2\kappa}\int\limits_M\frac{1}{2\qty(D-2)!}\tensor{R}{_c_b_\beta_\rho}\tensor{e}{_\alpha^c}\tensor{e}{_\sigma^b}\tensor{\epsilon}{^\alpha^\sigma_{\nu_1}_\cdots_{\nu_{D-2}}}\vb e^{\nu_1}\wedge\cdots\wedge\vb e^{\nu_{D-2}}\wedge\vb e^{\beta}\wedge\vb e^{\rho}\\
    S_{\textnormal{EH}}&=\frac{1}{2\kappa}\int\limits_M\frac{1}{2\qty(D-2)!}\tensor{R}{_c_b_\beta_\rho}\tensor{\epsilon}{^\alpha^\sigma_{\nu_1}_\cdots_{\nu_{D-2}}}\tensor{e}{_\alpha^c}\tensor{e}{_\sigma^b}\tensor{e}{^{\nu_1}_{a_1}}\cdots\tensor{e}{^{\nu_{D-2}}_{a_{D-2}}}\vb dx^{a_1}\wedge\cdots\wedge\vb dx^{a_{D-2}}\wedge\vb e^{\beta}\wedge\vb e^{\rho}\\
    S_{\textnormal{EH}}&=\frac{1}{2\kappa}\int\limits_M\frac{1}{2\qty(D-2)!}\tensor{R}{_c_b_\beta_\rho}\Det\qty[\tensor{e}{^\nu_a}]\tensor{\epsilon}{^c^b_{a_1}_\cdots_{a_{D-2}}}\vb dx^{a_1}\wedge\cdots\wedge\vb dx^{a_{D-2}}\wedge\vb e^{\beta}\wedge\vb e^{\rho}\\
    S_{\textnormal{EH}}&=\frac{1}{2\kappa}\int\limits_M\frac{1}{2}\tensor{R}{_c_b_\beta_\rho}\star\qty(\vb dx^{c}\wedge\vb dx^{b})\wedge\vb e^{\beta}\wedge\vb e^{\rho}\\
    S_{\textnormal{EH}}&=\frac{1}{2\kappa}\int\limits_M\star\vb R_{\beta\rho}\wedge\vb e^{\beta}\wedge\vb e^{\rho}\numberthis\label{EHAction2}
\end{align*}

This is our final result for the EH Action. To see that it's a consistent expression, we can do the counting of the form degree. $\vb R$ is a 2--form, which makes $\star\vb R$ a $\qty(D-2)$--form. As $\vb e$ is a 
1--form, the final expression is a $D-2+2=D$--form. The next sections are dedicated to check the redundancies and equations of motion of the Action.