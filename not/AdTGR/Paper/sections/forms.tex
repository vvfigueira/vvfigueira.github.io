\section{The Einstein-Hilbert Action in the Form Language}

We'll begin with a quick recap of the vielbein/spin connection formalism, for now we'll keep the discussion general in $D$ dimensions, and 
only later on we'll go to the special case of $D=2+1$. The vielbein\footnote{We'll denote the \textit{vector} --- (1,0) tensor --- vielbein with a tilde, only to be distinguishable from the 
associated \textit{covetor} --- (0,1) tensor --- vielbein, which we'll denote without the tilde due to being way more important to us.} $\tilde{\vb e}_\mu$ are a basis of the vector field space $\mathfrak X\qty(M)$, notice, 
the index $\mu$ is only indexing which vector from the $D$ present in the basis are we talking about, it isn't a coordinate index --- one that is 
related to a specific component decomposition in a specific chart ---, as this is not a coordinate basis --- i.e. $\boldsymbol\partial_a$ ---, we can 
in fact do miracles with it, as diagonalize the metric,
\begin{align*}
    \textnormal{diag}\mqty(-1& 1&\cdots&1)=\eta_{\mu\nu}&=\vb g\qty(\tilde{\vb e}_\mu,\tilde{\vb e}_\nu)=g_{ab}\vb d{x^a}\qty(\tilde{\vb e}_\mu)\otimes\vb d{x^b}\qty(\tilde{\vb e}_\nu)\\
    \eta_{\mu\nu}&=g_{ab}\tensor{e}{_\mu^c}\tensor{e}{_\nu^d}\vb d{x^a}\qty(\boldsymbol\partial_c)\otimes\vb d{x^b}\qty(\boldsymbol\partial_d)\\
    \eta_{\mu\nu}&=g_{ab}\tensor{e}{_\mu^a}\tensor{e}{_\nu^b}
\end{align*}

The whole point of introducing the vielbein is switch the degrees of freedom from the 
metric to the inertial frame basis, which can be seen as a downgrade, due to this process enlarging the number of degrees of freedom from $\frac12D\qty(D+1)$ 
to $D^2$, but, of course, this is only the naive counting, without considering the redundancies, as the number of physical degrees of freedom must be the same. 
For this to be true, is only possible if we enlarge also the redundancies, to kill the extra degrees of freedom we introduced, as mentioned, this could be seen as 
not desirable, but, for us this turn out to be essencial. What are these new redundancies? They're the choice of labeling $\mu$ in $\tilde{\vb e}_\mu$, as long as the 
new relabel also respect the defining property of the vielbein it's a redundant transformation, notice that these transformations are exactly local Lorentz ones, that is, 
given a set of functions $\tensor{\Lambda}{^\mu_\nu}:M\rightarrow\mathbb R$, a new set of vector fields $\tensor{\Lambda}{^\mu_\nu}\tilde{\vb e}_\mu$ is a vielbein iff, 
\begin{align*}
    \vb g\qty(\tensor{\Lambda}{^\alpha_\mu}\tilde{\vb e}_\alpha,\tensor{\Lambda}{^\beta_\nu}\tilde{\vb e}_\beta)&=\tensor{\Lambda}{^\alpha_\mu}\tensor{\Lambda}{^\beta_\nu}\vb g\qty(\tilde{\vb e}_\alpha,\tilde{\vb e}_\beta)\\
    \eta_{\mu\nu}&=\tensor{\Lambda}{^\alpha_\mu}\tensor{\Lambda}{^\beta_\nu}\eta_{\alpha\beta}
\end{align*}
in other words, $\tensor{\Lambda}{^\mu_\nu}$ must be a local $SO\qty(1,D-1)$ element\footnote{Actually the group is $O\qty(1,D-1)$, but, we'll only be interested in the orientation preserving transformations.}, with this new redundancy taken in account the degrees of freedom match, $D^2$ 
from the vielbein, minus $D$ diffeomorphisms $\phi^\ast\tilde{\vb e}_\mu\sim\tilde{\vb e}_\mu$, $D$ Bianchi identities and $\frac12D\qty(D-1)$ local $SO\qty(1,D-1)$ transformations, giving $D^2-2D-\frac12D\qty(D-1)=\frac12D\qty(D-3)$ exactly 
the same counting using only the metric! Notice that now the redundancies are diffeomorphisms and local $SO\qty(1,D-1)$ transformations, this already seems like we're gauging the whole Poincaré group, which will come up later.

From the condition that $\vb g$ must be non-degenerate we get that the matrix of components $\tensor{e}{_\mu^a}$ must be invertible, this ensures 
the existence of $\tensor{e}{^\mu_a}$ ,from which we can construct the dual vector field $\vb e^\mu=\tensor{e}{^\mu_a}\vb d{x^a}$ this ensures 
we have a basis of the whole tensor space, so it's possible to decompose any tensor in it,
\begin{align*}
    \eta_{\mu\nu}\vb e^\mu\otimes\vb e^\nu&=\vb g\qty(\tilde{\vb e}_\mu,\tilde{\vb e}_\nu)\vb e^\mu\otimes\vb e^\nu=\vb g
\end{align*}

Up to now we have been considering a metric compatible torsionless affine connection, but, this turn out to not be the optimal choice, as for this kind of connection 
there is an differential/algebraical restraint between the connection and the metric, at least that what we should expect in a coordinate basis. What we would like is 
to have a connection linearly independent of the metric/vielbein, but without sacrificing the metricity condition, this is completely hopeless in a 
coordinate basis, but, in a non-coordinate basis this is achievable! We just have to remind that an affine connection can be defined in any basis of $\mathfrak X\qty(M)$, what 
is usually done is, $\nabla_{\vb X}\boldsymbol\partial_b=X^a\tensor{\Gamma}{_a^c_b}\boldsymbol\partial_c$, but it's much more interesting to define it with respect to the vielbein 
basis,
% Only having a metric, or in this case, only having a vielbein, isn't sufficient to construct a theory, we also need a affine connection/covariant 
% derivative. A useful way to define it is to define in a vector field basis, as we already have the vielbein as a vector field basis it's more 
% convenient to define in respect to it,
\begin{align*}
    \nabla_{\vb X}\qty(\tilde{\vb e}_\nu)&=\tensor{{\boldsymbol\omega(\vb X)}}{^\mu_\nu}\tilde{\vb e}_\mu=X^a\tensor{\omega}{_a^\mu_\nu}\tilde{\vb e}_\mu
\end{align*}

Where $\tensor{{\boldsymbol\omega}}{^\mu_\nu}=\tensor{\omega}{_a^\mu_\nu}\vb dx^a$ is named the spin connection, it can be seen as a $\mathfrak{gl}\qty(1,D-1)$--valued $(0,1)$ tensor, 
or, as we'll adopt here, a $\mathfrak{gl}\qty(1,D-1)$--valued $1$--form.
The ease of working with the vielbein is that the Lorentz index $\mu$ 
does not change upon coordinate/chart/diffeomorphism transformations, it acts as if was an internal symmetry, thus, 
$\tensor{{\boldsymbol\omega}}{^\mu_\nu}$ do transform exactly as a tensor should, this is already a enormous dichotomy 
with the standard formulation where the Christoffel connection doesn't transform in a good manner.
% notice that we have a 
% redundancy of how to choose the vielbein basis, due to $\tensor{\Lambda}{^\mu_\nu}\vb e_\mu$, $\tensor{\Lambda}{^\mu_\nu}:M\rightarrow\mathbb R$, being an equally good basis,
% as long as $\tensor{\Lambda}{^\alpha_\mu}\tensor{\Lambda}{^\beta_\nu}\eta_{\alpha\beta}=\eta_{\mu\nu}$, that is, as we enlarge the number of components going from the metric to 
% the vielbein, we also enlarge the redundancies, now, additionally to Diff$\qty(M)$ we have local $SO\qty(1,D-1)$ transformations as redundancies, this is as 
% we was gauging the whole Poincaré group, with Diff$\qty(M)$ being the gauging of the translations.
Now we'll impose the metric compatibility of the connection, 
this is another scenario where the vielbein formalism como to hand, as this condition imposes no additional differential/algebraical constrains among the vielbein and spin connection.
\begin{align*}
    \vb g\qty(\tilde{\vb e}_\alpha,\nabla_{\vb X}\qty(\tilde{\vb e}_\nu))&=X^a\tensor{\omega}{_a^\mu_\nu}\vb g\qty(\tilde{\vb e}_\alpha,\tilde{\vb e}_\mu),\ \ \ \textnormal{symmetrize }\alpha\leftrightarrow\nu\\
    \vb g\qty(\tilde{\vb e}_\alpha,\nabla_{\vb X}\qty(\tilde{\vb e}_\nu))+\vb g\qty(\tilde{\vb e}_\nu,\nabla_{\vb X}\qty(\tilde{\vb e}_\alpha))&=X^a\tensor{\omega}{_a^\mu_\nu}\vb g\qty(\tilde{\vb e}_\alpha,\tilde{\vb e}_\mu)+X^a\tensor{\omega}{_a^\mu_\alpha}\vb g\qty(\tilde{\vb e}_\nu,\tilde{\vb e}_\mu),\ \ \ \textnormal{metric symmetry}\\
    \vb g\qty(\tilde{\vb e}_\alpha,\nabla_{\vb X}\qty(\tilde{\vb e}_\nu))+\vb g\qty(\nabla_{\vb X}\qty(\tilde{\vb e}_\alpha),\tilde{\vb e}_\nu)&=X^a\tensor{\omega}{_a_\alpha_\nu}+X^a\tensor{\omega}{_a_\nu_\alpha},\ \ \ \textnormal{Leibnitz rule}\\
    \nabla_{\vb X}\qty(\vb g\qty(\tilde{\vb e}_\alpha,\tilde{\vb e}_\nu))-\nabla_{\vb X}\qty(\vb g)\qty(\tilde{\vb e}_\alpha,\tilde{\vb e}_\nu)&=X^a\tensor{\omega}{_a_\alpha_\nu}+X^a\tensor{\omega}{_a_\nu_\alpha},\ \ \ \nabla_{\vb X}\qty(\eta_{\alpha\nu})=0\\
    -\nabla_{\vb X}\qty(\vb g)\qty(\tilde{\vb e}_\alpha,\tilde{\vb e}_\nu)&=X^a\tensor{\omega}{_a_\alpha_\nu}+X^a\tensor{\omega}{_a_\nu_\alpha},\ \ \ \textnormal{metricity}\\
    -\tensor{\omega}{_a_\nu_\alpha}&=\tensor{\omega}{_a_\alpha_\nu}
\end{align*}

That is, the metric compatible spin connection is anti-symmetric in the non-coordinate indices, exactly the property 
satisfied by the generators of the $SO\qty(1,D-1)$ group, thus, the spin connection can be seen as a $\mathfrak{so}\qty(1,D-1)$-valued 1-form, it has 
everything in it's favor to be interpreted as a gauge field of the $SO\qty(1,D-1)$ group, to confirm this notice how it changes under a gauge transformation 
of the vielbein basis $\tilde{\vb e}_\nu\rightarrow\tensor{\Lambda}{^\mu_\nu}\tilde{\vb e}_\mu$,
\begin{align*}
    \nabla_{\vb X}\qty(\tensor{\Lambda}{^\mu_\nu}\tilde{\vb e}_\mu)&=\nabla_{\vb X }\qty(\tensor{\Lambda}{^\mu_\nu})\tilde{\vb e}_\mu+\tensor{\Lambda}{^\mu_\nu}\nabla_{\vb X}\qty(\tilde{\vb e}_\mu),\ \ \ \textnormal{connection definition}\\
    \nabla_{\vb X}\qty(\tensor{\Lambda}{^\mu_\nu}\tilde{\vb e}_\mu)&=\vb X\qty(\tensor{\Lambda}{^\mu_\nu})\tilde{\vb e}_\mu+\tensor{\Lambda}{^\mu_\nu}\tensor{{\boldsymbol\omega\qty(\vb X)}}{^\alpha_\mu}\tilde{\vb e}_\alpha,\ \ \ \tensor{{\Lambda^{-1}}}{^\beta_\alpha}\tensor{\Lambda}{^\sigma_\beta}=\tensor{\delta}{^\sigma_\alpha}\\
    \nabla_{\vb X}\qty(\tensor{\Lambda}{^\mu_\nu}\tilde{\vb e}_\mu)&=\vb d\tensor{\Lambda}{^\mu_\nu}\qty(\vb X)\tensor{{\Lambda^{-1}}}{^\beta_\mu}\tensor{\Lambda}{^\sigma_\beta}\tilde{\vb e}_\sigma+\tensor{\Lambda}{^\mu_\nu}\tensor{{\boldsymbol\omega\qty(\vb X)}}{^\alpha_\mu}\tensor{{\Lambda^{-1}}}{^\beta_\alpha}\tensor{\Lambda}{^\sigma_\beta}\tilde{\vb e}_\sigma\\
    \nabla_{\vb X}\qty(\tensor{\Lambda}{^\mu_\nu}\tilde{\vb e}_\mu)&=\qty(\tensor{{\Lambda^{-1}}}{^\beta_\mu}\vb d{\tensor{\Lambda}{^\mu_\nu}}+\tensor{{\Lambda^{-1}}}{^\beta_\alpha}\tensor{{\boldsymbol\omega}}{^\alpha_\mu}\tensor{\Lambda}{^\mu_\nu})\qty(\vb X)\tensor{\Lambda}{^\sigma_\beta}\tilde{\vb e}_\sigma
\end{align*}

That is, under a gauge transformation of $\tilde{\vb e}_\nu\rightarrow\tensor{\Lambda}{^\mu_\nu}\tilde{\vb e}_\mu$, the spin connection transforms exactly as a connection of the gague group $SO\qty(1,D-1)$, that is, 
$\boldsymbol \omega\rightarrow \Lambda^{-1}\boldsymbol\omega\Lambda+\Lambda^{-1}\vb d{\Lambda}$, which is what we're searching. Before pursuing further the gauged translations, we're going to obtain a new interpretation 
for the Riemann tensor, using what we just learned from the spin connection, notice, the usual interpretation of the Riemann tensor is of it being a $(1,3)$ tensor, but, as naturally --- without the need for a metric compatible 
torsionless connection --- it's anti-symmetric in the first two entries, we can switch the point of view from a $(1,3)$ tensor to a $(1,1)$ tensor valued 2--form, or, in an even better way, an End$\qty(\mathfrak X\qty(M))$--valued 2--form, 
which, when decomposed in the non-coordinate basis, will turn out to be a $\mathfrak{so}\qty(1,D-1)$--valued 2--form, as we'll shown now. Starting from the definition\footnote{Here, care must be taken, in our conventions, the covariant derivative acts non-trivially only 
in vectors, and acts as a normal derivative in functions, that is, for $X^a$ being \textit{components} of a vector, $\nabla_{\vb Y}X^a=\vb Y\qty(X^a)=Y^b\partial_b X^a$, in contrast to $\nabla_{\vb Y}\boldsymbol\partial_a=Y^b\tensor{\Gamma}{_b^c_a}\boldsymbol\partial_c$.},
\begin{align*}
    \textbf{Riem}\qty(\vb X,\vb Y)\tilde{\vb e}_\mu&=\qty(\nabla_{\vb X}\nabla_{\vb Y}-\nabla_{\vb Y}\nabla_{\vb X}-\nabla_{\comm{\vb X}{\vb Y}})\tilde{\vb e}_\mu\\
    \textbf{Riem}\qty(\vb X,\vb Y)\tilde{\vb e}_\mu&=\nabla_{\vb X}\qty(Y^b\tensor{\omega}{_b^\nu_\mu}\tilde{\vb e}_\nu)-\nabla_{\vb Y}\qty(X^a\tensor{\omega}{_a^\nu_\mu}\tilde{\vb e}_\nu)-\comm{\vb X}{\vb Y}^b\tensor{\omega}{_b^\nu_\mu}\tilde{\vb e}_\nu\\
    \textbf{Riem}\qty(\vb X,\vb Y)\tilde{\vb e}_\mu&=Y^b\nabla_{\vb X}\qty(\tensor{\omega}{_b^\nu_\mu}\tilde{\vb e}_\nu)-X^a\nabla_{\vb Y}\qty(\tensor{\omega}{_a^\nu_\mu}\tilde{\vb e}_\nu)+\nabla_{\vb X}\qty(Y^b)\tensor{\omega}{_b^\nu_\mu}\tilde{\vb e}_\nu-\nabla_{\vb Y}\qty(X^b)\tensor{\omega}{_b^\nu_\mu}\tilde{\vb e}_\nu-\comm{\vb X}{\vb Y}^b\tensor{\omega}{_b^\nu_\mu}\tilde{\vb e}_\nu\\
    \textbf{Riem}\qty(\vb X,\vb Y)\tilde{\vb e}_\mu&=X^aY^b\nabla_{a}\qty(\tensor{\omega}{_b^\nu_\mu}\tilde{\vb e}_\nu)-X^aY^b\nabla_{b}\qty(\tensor{\omega}{_a^\nu_\mu}\tilde{\vb e}_\nu)+\qty(\nabla_{\vb X}\qty(Y^b)-\nabla_{\vb Y}\qty(X^b))\tensor{\omega}{_b^\nu_\mu}\tilde{\vb e}_\nu-\comm{\vb X}{\vb Y}^b\tensor{\omega}{_b^\nu_\mu}\tilde{\vb e}_\nu\\
    \textbf{Riem}\qty(\vb X,\vb Y)\tilde{\vb e}_\mu&=X^aY^b\nabla_{a}\qty(\tensor{\omega}{_b^\nu_\mu})\tilde{\vb e}_\nu+X^aY^b\tensor{\omega}{_b^\nu_\mu}\tensor{\omega}{_a^\alpha_\nu}\tilde{\vb e}_\alpha-X^aY^b\nabla_{b}\qty(\tensor{\omega}{_a^\nu_\mu})\tilde{\vb e}_\nu-X^aY^b\tensor{\omega}{_a^\nu_\mu}\tensor{\omega}{_b^\alpha_\nu}\tilde{\vb e}_\alpha\\
    &\quad\quad\quad+\qty(X^a\partial_a Y^b-Y^a\partial_aX^b)\tensor{\omega}{_b^\nu_\mu}\tilde{\vb e}_\nu-\qty(X^a\partial_aY^b-Y^a\partial_a X^b)\tensor{\omega}{_b^\nu_\mu}\tilde{\vb e}_\nu\\
    \textbf{Riem}\qty(\vb X,\vb Y)\tilde{\vb e}_\mu&=X^aY^b\left(\partial_{a}\qty(\tensor{\omega}{_b^\nu_\mu})+\tensor{\omega}{_b^\alpha_\mu}\tensor{\omega}{_a^\nu_\alpha}-\partial_{b}\qty(\tensor{\omega}{_a^\nu_\mu})-\tensor{\omega}{_a^\alpha_\mu}\tensor{\omega}{_b^\nu_\alpha}\right)\tilde{\vb e}_\nu\\
    \textbf{Riem}\qty(\vb X,\vb Y)\tilde{\vb e}_\mu&=X^aY^b\left(\partial_{a}\tensor{\omega}{_b^\nu_\mu}-\partial_{b}\tensor{\omega}{_a^\nu_\mu}+\tensor{\omega}{_a^\nu_\alpha}\tensor{\omega}{_b^\alpha_\mu}-\tensor{\omega}{_b^\nu_\alpha}\tensor{\omega}{_a^\alpha_\mu}\right)\tilde{\vb e}_\nu\\
    X^aY^b\tensor{R}{_a_b^\nu_\mu}\tilde{\vb e}_\nu&=X^aY^b\tensor{\left(\vb d{\boldsymbol\omega}+\boldsymbol\omega\wedge\boldsymbol\omega\right)}{_a_b^\nu_\mu}\tilde{\vb e}_\nu
\end{align*}

What settle down the interpretation of the Riemann tensor being a $\mathfrak{so}\qty(1,D-1)$--valued 2--form, and also provides a striking resemblance to the usual gauge force field in non-abelian 
theories, $\vb F=\vb d\vb A+\vb A\wedge \vb A$, it's also is easily related to the usual coordinate Riemann tensor,
\begin{align*}
    \textbf{Riem}\qty(\vb X,\vb Y)\tilde{\vb e}_\mu&=X^aY^b\tensor{R}{_a_b^\nu_\mu}\tilde{\vb e}_\nu\\
    \textbf{Riem}\qty(\vb X,\vb Y)\qty(\tensor{e}{_\mu^e}\boldsymbol\partial_e)&=X^aY^b\tensor{R}{_a_b^\nu_\mu}\tensor{e}{_\nu^c}\boldsymbol\partial_c\\
    \tensor{e}{_\mu^e}\textbf{Riem}\qty(\vb X,\vb Y)\boldsymbol\partial_e&=X^aY^b\tensor{R}{_a_b^\nu_\mu}\tensor{e}{_\nu^c}\boldsymbol\partial_c\\
    \tensor{e}{_\mu^e}X^aY^b\tensor{R}{_a_b^c_e}\boldsymbol\partial_c&=X^aY^b\tensor{R}{_a_b^\nu_\mu}\tensor{e}{_\nu^c}\boldsymbol\partial_c\\
    \tensor{e}{^\mu_d}\tensor{e}{_\mu^e}\tensor{R}{_a_b^c_e}&=\tensor{e}{^\mu_d}\tensor{R}{_a_b^\nu_\mu}\tensor{e}{_\nu^c}\\
    \tensor{R}{_a_b^c_d}&=\tensor{e}{^\mu_d}\tensor{R}{_a_b^\nu_\mu}\tensor{e}{_\nu^c}
\end{align*}
notice that the anti-symmetry of the first two indices of the Riemann tensor is defining property of it, doesn't depending on the type of connection 
chosen, but, the anti-symmetry in the last two indices is up to the metricity condition and the symmetry between pairs is up to torsionless condition, 
up to now we'll assume only the metricity one. With all of this being said, we can construct our Riemann/Curvature 2-form,
\begin{align*}
    \tensor{{\vb R}}{^\nu_\mu}&=\frac12\tensor{R}{_a_b^\nu_\mu}\dd{x^a}\wedge\dd{x^b}=\tensor{\dd{\boldsymbol\omega}}{^\nu_\mu}+\tensor{{\boldsymbol\omega}}{^\nu_\alpha}\wedge\tensor{{\boldsymbol\omega}}{^\alpha_\mu}%\tensor{{\vb R}}{^\nu_\mu}&=\frac12\tensor{R}{_a_b^\nu_\mu}\tensor{e}{_\alpha^a}\tensor{e}{^\alpha_c}\tensor{e}{_\beta^b}\tensor{e}{^\beta_d}\dd{x^c}\wedge\dd{x^d}\\%\tensor{{\vb R}}{^\nu_\mu}&=\frac12\tensor{R}{_a_b^\nu_\mu}\tensor{e}{_\alpha^a}\tensor{e}{_\beta^b}{\tilde{\vb e}^\alpha}\wedge{\tilde{\vb e}^\beta}
\end{align*}
this is of the general form of $\vb F=\dd{\vb A}+\vb A\wedge\vb A$ as for a gauge theory, we'll talk more about the absence of $\tilde{\vb e} ^\mu$ later, from now we'll start writing the 
Einstein-Hilbert Action in terms of these variables introduced, starting with the volume form,
\begin{align*}
    \dd[D]{x}\sqrt{\abs{g}}&=\sqrt{\abs{\Det\qty[g_{ab}]}}\dd{x^0}\wedge\cdots\wedge\dd{x^{D-1}}\\
    \dd[D]{x}\sqrt{\abs{g}}&=\sqrt{\abs{\Det\qty[\tensor{e}{^\mu_a}\eta_{\mu\nu}\tensor{e}{^\nu_b}]}}\dd{x^0}\wedge\cdots\wedge\dd{x^{D-1}}\\
    \dd[D]{x}\sqrt{\abs{g}}&=\sqrt{\abs{\Det\qty[\tensor{e}{^\mu_a}]\Det\qty[\eta_{\mu\nu}]\Det\qty[\tensor{e}{^\nu_b}]}}\dd{x^0}\wedge\cdots\wedge\dd{x^{D-1}}\\
    \dd[D]{x}\sqrt{\abs{g}}&=\sqrt{\qty(\Det\qty[\tensor{e}{^\mu_a}])^2}\dd{x^0}\wedge\cdots\wedge\dd{x^{D-1}}\\
    \dd[D]{x}\sqrt{\abs{g}}&=\Det\qty[\tensor{e}{^\mu_a}]\dd{x^0}\wedge\cdots\wedge\dd{x^{D-1}}\\
    \dd[D]{x}\sqrt{\abs{g}}&=\epsilon_{\mu_1\cdots\mu_{D}}\tensor{e}{^{\mu_0}_0}\cdots\tensor{e}{^{\mu_{D}}_{D-1}}\dd{x^0}\wedge\cdots\wedge\dd{x^{D-1}}\\
    \dd[D]{x}\sqrt{\abs{g}}&=\epsilon_{\mu_1\cdots\mu_{D}}\tensor{e}{^{\mu_0}_0}\dd{x^0}\wedge\cdots\wedge\tensor{e}{^{\mu_{D}}_{D-1}}\dd{x^{D-1}}\\
    \dd[D]{x}\sqrt{\abs{g}}&=\frac{1}{D!}\epsilon_{\mu_1\cdots\mu_{D}}\tensor{e}{^{\mu_1}_{a_1}}\dd{x^{a_1}}\wedge\cdots\wedge\tensor{e}{^{\mu_{D}}_{a_{D}}}\dd{x^{a_{D}}}\\
    \dd[D]{x}\sqrt{\abs{g}}&=\frac{1}{D!}\epsilon_{\mu_1\cdots\mu_{D}}\tilde{\vb e}^{\mu_1}\wedge\cdots\wedge\tilde{\vb e}^{\mu_{D}}
\end{align*}

And now we express the Ricci scalar,
\begin{align*}
    R&=g^{ab}\tensor{R}{_c_b^c_a}\\
    R&=\tensor{e}{_\rho^a}\tensor{e}{^\rho^b}\tensor{R}{_c_b_d_a}\tensor{e}{_\alpha^c}\tensor{e}{^\alpha^d}\\
    R&=\eta^{\rho\sigma}\eta^{\alpha\beta}\tensor{e}{_\rho^a}\tensor{e}{_\sigma^b}\tensor{R}{_c_b_d_a}\tensor{e}{_\alpha^c}\tensor{e}{_\beta^d}\\
    R&=\eta^{\rho\sigma}\eta^{\alpha\beta}\tensor{R}{_\alpha_\sigma_\beta_\rho}\\
    R&=\frac12\qty(\eta^{\rho\sigma}\eta^{\alpha\beta}-\eta^{\rho\alpha}\eta^{\sigma\beta})\tensor{R}{_\alpha_\sigma_\beta_\rho}\\
    R&=\frac{-1}{2\qty(D-2)!}\tensor{\epsilon}{^{\nu_1}^\cdots^{\nu_{D-2}}^\beta^\rho}\tensor{\epsilon}{_{\nu_1}_\cdots_{\nu_{D-2}}^\alpha^\sigma}\tensor{R}{_\alpha_\sigma_\beta_\rho}
\end{align*}

Putting everything together,
\begin{align*}
    S_{\textnormal{EH}}&=\frac{1}{2\kappa}\int\limits_M\dd[D]{x}\sqrt{\abs{g}}R\\
    S_{\textnormal{EH}}&=\frac{-1}{4D!\qty(D-2)!\kappa}\int\limits_M\vb e^{\mu_1}\wedge\cdots\wedge\vb e^{\mu_{D}}\epsilon_{\mu_1\cdots\mu_{D}}\tensor{\epsilon}{^{\nu_1}^\cdots^{\nu_{D-2}}^\beta^\rho}\tensor{\epsilon}{_{\nu_1}_\cdots_{\nu_{D-2}}^\alpha^\sigma}\tensor{R}{_\alpha_\sigma_\beta_\rho}\\
    S_{\textnormal{EH}}&=\frac{1}{4\qty(D-2)!\kappa}\int\limits_M\vb e^{\mu_1}\wedge\cdots\wedge\vb e^{\mu_{D}}\tensor{\eta}{_{\mu_1}^{[\nu_1}}\cdots\tensor{\eta}{_{\mu_{D-2}}^{\nu_{D-2}}}\tensor{\eta}{_{\mu_{D-1}}^{\beta}}\tensor{\eta}{_{\mu_{D}}^{\rho]}}\tensor{\epsilon}{_{\nu_1}_\cdots_{\nu_{D-2}}^\alpha^\sigma}\tensor{R}{_\alpha_\sigma_\beta_\rho}\\
    S_{\textnormal{EH}}&=\frac{1}{4\qty(D-2)!\kappa}\int\limits_M\vb e^{\nu_1}\wedge\cdots\wedge\vb e^{\nu_{D-2}}\wedge\vb e^{\beta}\wedge\vb e^{\rho}\tensor{\epsilon}{_{\nu_1}_\cdots_{\nu_{D-2}}^\alpha^\sigma}\tensor{R}{_\alpha_\sigma_\beta_\rho}\\
    S_{\textnormal{EH}}&=\frac{1}{2\kappa}\int\limits_M\frac{1}{2\qty(D-2)!}R_{\alpha\sigma\beta\rho}\tensor{\epsilon}{^\alpha^\sigma_{\nu_1}_\cdots_{\nu_{D-2}}}\vb e^{\nu_1}\wedge\cdots\wedge\vb e^{\nu_{D-2}}\wedge\vb e^{\beta}\wedge\vb e^{\rho}\\
    S_{\textnormal{EH}}&=\frac{1}{2\kappa}\int\limits_M\star\vb R_{\beta\rho}\wedge\vb e^{\beta}\wedge\vb e^{\rho}\\
    S_{\textnormal{EH}}&=\frac{1}{2\kappa}\int\limits_M\vb R_{\beta\rho}\wedge\star\qty(\vb e^{\beta}\wedge\vb e^{\rho})\\
    S_{\textnormal{EH}}&=\frac{1}{2\kappa}\int\limits_M\frac{1}{\qty(D-2)!}\tensor{\epsilon}{^\beta^\rho_{\alpha_1}_\cdots_{\alpha_{D-2}}}\vb R_{\beta\rho}\wedge\vb e^{\alpha_1}\wedge\cdots\wedge\vb e^{\alpha_{D-2}}\\
    S_{\textnormal{EH}}&=\frac{1}{2\qty(D-2)!\kappa}\int\limits_M\tensor{\epsilon}{_{\alpha_1}_\cdots_{\alpha_{D}}}\vb e^{\alpha_1}\wedge\cdots\wedge\vb e^{\alpha_{D-2}}\wedge\vb R^{\alpha_{D-1}\alpha_{D}}
\end{align*}

For $D=4$ this gives,
\begin{align*}
    S_{\textnormal{EH}}\qty[\vb e,\boldsymbol\omega]&=\frac{1}{4\kappa}\epsilon_{\mu\nu\alpha\beta}\int\limits_M\vb e^{\mu}\wedge\vb e^\nu\wedge\vb R^{\alpha\beta}
\end{align*}

It's not clear how this should be interpreted as a gauge theory, due to we wanting to interpret $\boldsymbol\omega$ as the gauge 1-form of Lorentz transformations and $\tilde{\vb e}$ as the gauge 1-form of 
translations, the Einstein Hilbert Action should then be something of the form,
\begin{align*}
    \int\limits_M\Tr\qty[\vb A\wedge\vb A\wedge\vb F]
\end{align*}
for a combined gauge field $\vb A$, where both $\boldsymbol\omega$ and $\vb e$ contribute in some way, sadly, there is no such gauge theory in this form, 
first, due to the trace being of three lie algebra valued forms, this vanishes, secondly, this is not invariant under a gauge transformation $\vb A\rightarrow g^{-1}\vb Ag+g^{-1}\dd{g}$. So, 
if it's to $3+1$ gravity to be a gauge theory in this sense, it certainly is realized in a different way, but, an interesting case happens to $D=2+1$,
\begin{align*}
    S_{\textnormal{EH}}\qty[\vb e,\boldsymbol\omega]&=\frac{1}{2\kappa}\epsilon_{\mu\alpha\beta}\int\limits_M\vb e^{\mu}\wedge\vb R^{\alpha\beta}
\end{align*}

This already seems a lot like a proper gauge theory of the Chern-Simons type,
Equations of motion are,
\begin{align*}
    S_{\textnormal{EH}}\qty[\vb e+\delta\vb e,\boldsymbol\omega]-S_{\textnormal{EH}}\qty[\vb e,\boldsymbol\omega]&=\frac{1}{2\kappa}\epsilon_{\mu\alpha\beta}\int\limits_M\delta\vb e^{\mu}\wedge\vb R^{\alpha\beta}=0\\
    0&=-\frac12\epsilon_{\mu\alpha\beta}\vb R^{\alpha\beta}\\
    0&=-\frac12\epsilon_{\mu\alpha\beta}\frac12\tensor{R}{_\rho_\sigma^\alpha^\beta}\vb e^\rho\wedge\vb e^\sigma\\
    0&=-\frac14\epsilon_{\mu\alpha\beta}\tensor{R}{_\rho_\sigma^\alpha^\beta}\star\qty(\vb e^\rho\wedge\vb e^\sigma)\\
    0&=-\frac14\epsilon_{\mu\alpha\beta}\tensor{R}{_\rho_\sigma^\alpha^\beta}\tensor{\epsilon}{^\rho^\sigma_\kappa}\vb e^\kappa\\
    0&=-\frac14\epsilon_{\mu\alpha\beta}\tensor{\epsilon}{^\rho^\sigma^\kappa}\tensor{R}{_\rho_\sigma^\alpha^\beta}\vb e_\kappa\\
    0&=-\frac{3!}{4}\tensor{\eta}{_\mu^{[\rho}}\tensor{\eta}{_\alpha^{\sigma}}\tensor{\eta}{_\beta^{\kappa]}}\tensor{R}{_\rho_\sigma^\alpha^\beta}\vb e_\kappa\\
    0&=-\frac{3!}{4}\tensor{R}{_\rho_\sigma^{[\sigma}^\kappa}\tensor{\eta}{_\mu^{\rho]}}\vb e_\kappa\\
    0&=-\frac14\qty(\tensor{R}{_\rho_\sigma^{\sigma}^\kappa}\tensor{\eta}{_\mu^{\rho}}+\tensor{R}{_\rho_\sigma^\kappa^\rho}\tensor{\eta}{_\mu^{\sigma}}+\tensor{R}{_\rho_\sigma^\rho^\sigma}\tensor{\eta}{_\mu^\kappa}-\tensor{R}{_\rho_\sigma^{\rho}^\kappa}\tensor{\eta}{_\mu^{\sigma}}-\tensor{R}{_\rho_\sigma^\kappa^\sigma}\tensor{\eta}{_\mu^{\rho}}-\tensor{R}{_\rho_\sigma^\sigma^\rho}\tensor{\eta}{_\mu^\kappa})\vb e_\kappa\\
    0&=-\frac14\qty(-\tensor{R}{_\mu^\kappa}-\tensor{R}{_\mu^\kappa}+R\tensor{\eta}{_\mu^\kappa}-\tensor{R}{_\mu^\kappa}-\tensor{R}{_\mu^\kappa}+R\tensor{\eta}{_\mu^\kappa})\vb e_\kappa\\
    0&=-\frac14(-4\tensor{R}{_\mu^\kappa}+2R\tensor{\eta}{_\mu^\kappa})\vb e_\kappa\\
    0&=\qty(\tensor{R}{_\mu_\kappa}-\frac12R\tensor{\eta}{_\mu_\kappa})\vb e^\kappa
\end{align*}

And,
\begin{align*}
    S_{\textnormal{EH}}\qty[\vb e,\boldsymbol\omega+\delta\boldsymbol\omega]-S_{\textnormal{EH}}\qty[\vb e,\boldsymbol\omega]&=\frac{1}{2\kappa}\epsilon_{\mu\alpha\beta}\int\limits_M\vb e^{\mu}\wedge\qty(\dd{\delta\boldsymbol\omega}+\delta\boldsymbol\omega\wedge\boldsymbol\omega+\boldsymbol\omega\wedge\delta\boldsymbol\omega)^{\alpha\beta}=0\\
    0&=\frac{1}{2\kappa}\epsilon_{\mu\alpha\beta}\int\limits_M\qty(-\dd{\qty(\vb e^\mu\wedge\delta\boldsymbol\omega^{\alpha\beta})}+\dd{\vb e^\mu}\wedge\delta\boldsymbol\omega^{\alpha\beta}+\vb e^\mu\wedge\qty(\delta\boldsymbol\omega\wedge\boldsymbol\omega)^{\alpha\beta}+\vb e^\mu\wedge\qty(\boldsymbol\omega\wedge\delta\boldsymbol\omega)^{\alpha\beta})\\
    0&=\frac{1}{2}\epsilon_{\mu\alpha\beta}\int\limits_M\qty(\dd{\vb e^\mu}\wedge\delta\boldsymbol\omega^{\alpha\beta}+\vb e^\mu\wedge\tensor{\delta\boldsymbol\omega}{^\alpha^\gamma}\wedge\tensor{\boldsymbol\omega}{_\gamma^\beta}+\vb e^\mu\wedge\tensor{\boldsymbol\omega}{^\alpha^\gamma}\wedge\tensor{\delta\boldsymbol\omega}{_\gamma^\beta})\\
    0&=\frac{1}{2}\epsilon_{\mu\alpha\beta}\int\limits_M\qty(\dd{\vb e^\mu}\wedge\delta\boldsymbol\omega^{\alpha\beta}-\vb e^\mu\wedge\tensor{\boldsymbol\omega}{_\gamma^\beta}\wedge\tensor{\delta\boldsymbol\omega}{^\alpha^\gamma}+\vb e^\mu\wedge\tensor{\boldsymbol\omega}{^\alpha_\gamma}\wedge\tensor{\delta\boldsymbol\omega}{^\gamma^\beta})\\
    0&=\frac{1}{2}\epsilon_{\mu\alpha\beta}\int\limits_M\dd{\vb e^\mu}\wedge\delta\boldsymbol\omega^{\alpha\beta}-\frac{1}{2}\epsilon_{\mu\alpha\beta}\vb e^\mu\wedge\tensor{\boldsymbol\omega}{_\gamma^\beta}\wedge\tensor{\delta\boldsymbol\omega}{^\alpha^\gamma}+\frac{1}{2}\epsilon_{\mu\alpha\beta}\vb e^\mu\wedge\tensor{\boldsymbol\omega}{^\alpha_\gamma}\wedge\tensor{\delta\boldsymbol\omega}{^\gamma^\beta}\\
    0&=\frac{1}{2}\epsilon_{\mu\alpha\beta}\int\limits_M\dd{\vb e^\mu}\wedge\delta\boldsymbol\omega^{\alpha\beta}-\frac{1}{2}\epsilon_{\mu\alpha\gamma}\vb e^\mu\wedge\tensor{\boldsymbol\omega}{_\beta^\gamma}\wedge\tensor{\delta\boldsymbol\omega}{^\alpha^\beta}+\frac{1}{2}\epsilon_{\mu\gamma\beta}\vb e^\mu\wedge\tensor{\boldsymbol\omega}{^\gamma_\alpha}\wedge\tensor{\delta\boldsymbol\omega}{^\alpha^\beta}\\
    0&=\frac{1}{2}\int\limits_M\qty(\epsilon_{\mu\alpha\beta}\dd{\vb e^\mu}-\epsilon_{\mu\alpha\gamma}\vb e^\mu\wedge\tensor{\boldsymbol\omega}{_\beta^\gamma}+\epsilon_{\mu\gamma\beta}\vb e^\mu\wedge\tensor{\boldsymbol\omega}{^\gamma_\alpha})\wedge\tensor{\delta\boldsymbol\omega}{^\alpha^\beta}\\
    0&=\frac{1}{2}\epsilon_{\mu\alpha\beta}\dd{\vb e^\mu}-\frac{1}{2}\epsilon_{\mu\alpha\gamma}\vb e^\mu\wedge\tensor{\boldsymbol\omega}{_\beta^\gamma}+\frac{1}{2}\epsilon_{\mu\gamma\beta}\vb e^\mu\wedge\tensor{\boldsymbol\omega}{^\gamma_\alpha}\\
    0&=\frac{1}{2}\epsilon^{\alpha\beta\nu}\epsilon_{\mu\alpha\beta}\dd{\vb e^\mu}-\frac{1}{2}\epsilon^{\alpha\beta\nu}\epsilon_{\mu\alpha\gamma}\vb e^\mu\wedge\tensor{\boldsymbol\omega}{_\beta^\gamma}+\frac{1}{2}\epsilon^{\alpha\beta\nu}\epsilon_{\mu\gamma\beta}\vb e^\mu\wedge\tensor{\boldsymbol\omega}{^\gamma_\alpha}\\
    0&=\dd{\vb e^\nu}-\tensor{\eta}{_\gamma^{[\beta}}\tensor{\eta}{_\mu^{\nu]}}\vb e^\mu\wedge\tensor{\boldsymbol\omega}{_\beta^\gamma}+\tensor{\eta}{_\mu^{[\nu}}\tensor{\eta}{_\gamma^{\alpha]}}\vb e^\mu\wedge\tensor{\boldsymbol\omega}{^\gamma_\alpha}\\
    0&=\dd{\vb e^\nu}-\frac{1}{2}\qty(\tensor{\eta}{_\gamma^{\beta}}\tensor{\eta}{_\mu^{\nu}}-\tensor{\eta}{_\gamma^{\nu}}\tensor{\eta}{_\mu^{\beta}})\vb e^\mu\wedge\tensor{\boldsymbol\omega}{_\beta^\gamma}+\frac{1}{2}\qty(\tensor{\eta}{_\mu^{\nu}}\tensor{\eta}{_\gamma^{\alpha}}-\tensor{\eta}{_\mu^{\alpha}}\tensor{\eta}{_\gamma^{\nu}})\vb e^\mu\wedge\tensor{\boldsymbol\omega}{^\gamma_\alpha}\\
    0&=\dd{\vb e^\nu}+\frac{1}{2}\tensor{\eta}{_\gamma^{\nu}}\tensor{\eta}{_\mu^{\beta}}\vb e^\mu\wedge\tensor{\boldsymbol\omega}{_\beta^\gamma}-\frac{1}{2}\tensor{\eta}{_\mu^{\alpha}}\tensor{\eta}{_\gamma^{\nu}}\vb e^\mu\wedge\tensor{\boldsymbol\omega}{^\gamma_\alpha}\\
    0&=\dd{\vb e^\nu}+\frac{1}{2}\vb e^\beta\wedge\tensor{\boldsymbol\omega}{_\beta^\nu}-\frac{1}{2}\vb e^\alpha\wedge\tensor{\boldsymbol\omega}{^\nu_\alpha}\\
    0&=\dd{\vb e^\nu}+\tensor{\boldsymbol\omega}{^\nu_\alpha}\wedge\vb e^\alpha
\end{align*}

It's not really feasible to give this a gauge theory approach, only if we're in 2+1, in this case there is an isomorphism, $\vb e^\mu\rightarrow \vb e_{\alpha\beta}=\epsilon_{\mu\alpha\beta}\vb e^\mu$, so that,
\begin{align*}
    S_{\textnormal{EH}}&=\frac{1}{2\kappa}\epsilon_{\mu\alpha\beta}\int\limits_M\vb e^\mu\wedge \vb R^{\alpha\beta}\\
    S_{\textnormal{EH}}&=-\frac{1}{2\kappa}\int\limits_M\vb e_{\beta\alpha}\wedge \vb R^{\alpha\beta}\\
    S_{\textnormal{EH}}&=-\frac{1}{2\kappa}\int\limits_M\Tr\qty[\vb e\wedge \vb R]\\
    S_{\textnormal{EH}}&=-\frac{1}{2\kappa}\int\limits_M\Tr\qty[\vb e\wedge\qty(\dd{\boldsymbol\omega}+\frac12\wedgecomm{\boldsymbol\omega}{\boldsymbol\omega})]
\end{align*}

This is a lot similar to Chern-Simons theory,

\begin{align*}
    S_{\textnormal{CS}}\qty[\vb A]&=k\int\limits_M\Tr\qty[\vb A\wedge\dd{\vb A}+\frac13\vb A\wedge\wedgecomm{\vb A}{\vb A}]
\end{align*}

Let's try, $\vb A^x=\boldsymbol\omega+x\vb e$,

\begin{align*}
    S_{\textnormal{CS}}\qty[\vb A^x]&=k\int\limits_M\Tr\qty[\qty(\boldsymbol\omega+x\vb e)\wedge\dd{\qty(\boldsymbol\omega+x\vb e)}+\frac13\qty(\boldsymbol\omega+x\vb e)\wedge\wedgecomm{\boldsymbol\omega+x\vb e}{\boldsymbol\omega+x\vb e}]\\
    S_{\textnormal{CS}}\qty[\vb A^x]&=k\int\limits_M\Tr\biggl[\boldsymbol\omega\wedge\dd{\boldsymbol\omega}+x\boldsymbol\omega\wedge\dd{\vb e}+x\vb e\wedge\dd{\boldsymbol\omega}+x^2\vb e\wedge\dd{\vb e}\\
    &\quad\quad\quad\left.+\frac23\qty(\boldsymbol\omega+x\vb e)\wedge\qty(\boldsymbol\omega+x\vb e)\wedge\qty(\boldsymbol\omega+x\vb e)\right]\\
    S_{\textnormal{CS}}\qty[\vb A^x]&=k\int\limits_M\Tr\biggl[\boldsymbol\omega\wedge\dd{\boldsymbol\omega}+x\dd{\vb e}\wedge\boldsymbol\omega+x\vb e\wedge\dd{\boldsymbol\omega}-\frac12x^2\dd{\qty(\vb e\wedge\vb e)}\\
    &\quad\quad\quad\left.+\frac23\boldsymbol\omega\wedge\boldsymbol\omega\wedge\boldsymbol\omega+2x\boldsymbol\omega\wedge\boldsymbol\omega\wedge\vb e+2x^2\vb e\wedge\vb e\wedge \boldsymbol\omega+\frac23x^3\vb e\wedge\vb e\wedge\vb e\right]\\
    S_{\textnormal{CS}}\qty[\vb A^x]&=k\int\limits_M\Tr\left[\boldsymbol\omega\wedge\qty(\dd{\boldsymbol\omega}+\frac23\boldsymbol\omega\wedge\boldsymbol\omega)+x\dd{\qty(\vb e\wedge\boldsymbol\omega)}+x\vb e\wedge\dd{\boldsymbol\omega}+x\vb e\wedge\qty(\dd{\boldsymbol\omega}+2\boldsymbol\omega\wedge\boldsymbol\omega)+2x^2\vb e\wedge\boldsymbol\omega\wedge\vb e\right.\\
    &\quad\quad\quad\left.+\frac23x^3\vb e\wedge\vb e\wedge\vb e\right]\\
    S_{\textnormal{CS}}\qty[\vb A^x]&=S_{\textnormal{CS}}\qty[\boldsymbol\omega]+k\int\limits_M\Tr\qty[x\dd{\qty(\vb e\wedge\boldsymbol\omega)}+2x\vb e\wedge\qty(\dd{\boldsymbol\omega}+\boldsymbol\omega\wedge\boldsymbol\omega)+2x^2\vb e\wedge\boldsymbol\omega\wedge\vb e+\frac23x^3\vb e\wedge\vb e\wedge\vb e]\\
    S_{\textnormal{CS}}\qty[\vb A^x]&=S_{\textnormal{CS}}\qty[\boldsymbol\omega]+k\int\limits_M\Tr\qty[x\dd{\qty(\vb e\wedge\boldsymbol\omega)}+2x\vb e\wedge\vb R+2x^2\vb e\wedge\boldsymbol\omega\wedge\vb e+\frac23x^3\vb e\wedge\vb e\wedge\vb e]\\
    S_{\textnormal{CS}}\qty[\vb A^x]&=S_{\textnormal{CS}}\qty[\boldsymbol\omega]-4xk\kappa S_{\textnormal{EH}}+k\int\limits_M\Tr\qty[x\dd{\qty(\vb e\wedge\boldsymbol\omega)}+2x^2\vb e\wedge\boldsymbol\omega\wedge\vb e+\frac23x^3\vb e\wedge\vb e\wedge\vb e]
\end{align*}

We can simplify if we sum two contributions,

\begin{align*}
    S_{\textnormal{CS}}\qty[\vb A^x]-S_{\textnormal{CS}}\qty[\vb A^{-x}]&=-8xk\kappa S_{\textnormal{EH}}+2kx\int\limits_{\partial M}\Tr\qty[\vb e\wedge\boldsymbol\omega]+\frac43x^3k\int\limits_M\Tr\qty[\vb e\wedge\vb e\wedge\vb e]\\
    \frac{1}{8xk\kappa}\qty(S_{\textnormal{CS}}\qty[\vb A^{-x}]-S_{\textnormal{CS}}\qty[\vb A^x])&=S_{\textnormal{EH}}-\frac{1}{4\kappa}\int\limits_{\partial M}\Tr\qty[\vb e\wedge\boldsymbol\omega]-\frac{x^2}{3!\kappa}\int\limits_M\Tr\qty[\vb e\wedge\vb e\wedge\vb e]
\end{align*}

And also the not so usual action,

\begin{align*}
    S_{\textnormal{CS}}\qty[\vb A^x]+S_{\textnormal{CS}}\qty[\vb A^{-x}]&=2S_{\textnormal{CS}}\qty[\boldsymbol\omega]+4x^2k\int\limits_M\Tr\qty[\vb e\wedge\boldsymbol\omega\wedge\vb e]
\end{align*}

Chern-Simons Equation of Motion,

\begin{align*}
    S_{\textnormal{CS}}\qty[\vb A+\delta\vb A]-S_{\textnormal{CS}}\qty[\vb A]&=k\int\limits_M\Tr\qty[\delta\vb A\wedge\dd{\vb A}+\vb A\wedge\dd{\delta\vb A}+\frac23\delta\vb A\wedge\vb A\wedge\vb A+\frac23\vb A\wedge\delta\vb A\wedge\vb A+\frac23\vb A\wedge\vb A\wedge\delta\vb A]\\
    S_{\textnormal{CS}}\qty[\vb A+\delta\vb A]-S_{\textnormal{CS}}\qty[\vb A]&=k\int\limits_M\Tr\qty[\dd{\vb A}\wedge\delta\vb A-\dd{\qty(\vb A\wedge\delta\vb A)}+\dd{\vb A}\wedge\delta \vb A+2\vb A\wedge\vb A\wedge\delta\vb A]\\
    S_{\textnormal{CS}}\qty[\vb A+\delta\vb A]-S_{\textnormal{CS}}\qty[\vb A]&=2k\int\limits_M\Tr\qty[\qty(\dd{\vb A}+\vb A\wedge\vb A)\wedge\delta\vb A]-k\int\limits_M\Tr\qty[\dd{\qty(\vb A\wedge\delta\vb A)}]\\
    S_{\textnormal{CS}}\qty[\vb A+\delta\vb A]-S_{\textnormal{CS}}\qty[\vb A]&=2k\int\limits_M\Tr\qty[\vb F\wedge\delta\vb A]-k\int\limits_{\partial M}\Tr\qty[\vb A\wedge\delta\vb A]
\end{align*}

Chern-Simons Gauge Invariance,

\begin{align*}
    S_{\textnormal{CS}}\qty[g\vb Ag^{-1}+g\dd{g^{-1}}]&=k\int\limits_M\Tr\qty[\qty(g\vb Ag^{-1}+g\dd{g^{-1}})\wedge g \vb Fg^{-1}-\frac13\qty(g\vb Ag^{-1}+g\dd{g^{-1}})\wedge\qty(g\vb Ag^{-1}+g\dd{g^{-1}})\wedge\qty(g\vb Ag^{-1}+g\dd{g^{-1}})]\\
    S_{\textnormal{CS}}\qty[g\vb Ag^{-1}+g\dd{g^{-1}}]&=k\int\limits_M\Tr\left[\vb A\wedge\vb F+\dd{g^{-1}}g\wedge \vb F-\frac13\vb A\wedge\vb A\wedge\vb A-\frac13g\dd{g^{-1}}\wedge g\dd{g^{-1}}\wedge g\dd{g^{-1}}\right.\\
    &\quad\quad\quad\left.-g\vb Ag^{-1}\wedge g\vb Ag^{-1}\wedge g\dd{g^{-1}}-g\vb A g^{-1}\wedge g\dd{g^{-1}}\wedge g\dd{g^{-1}}\right]\\
    S_{\textnormal{CS}}\qty[g\vb Ag^{-1}+g\dd{g^{-1}}]&=S_{\textnormal{CS}}\qty[\vb A]+k\int\limits_M\Tr\left[\dd{g^{-1}}g\wedge \vb F-\frac13\dd{g^{-1}}g\wedge \dd{g^{-1}}g\wedge \dd{g^{-1}}g\right.\\
    &\quad\quad\quad\left.-\dd{g^{-1}}g\wedge\vb A\wedge \vb A-\vb A \wedge \dd{g^{-1}}g\wedge \dd{g^{-1}}g\right]\\
    S_{\textnormal{CS}}\qty[g\vb Ag^{-1}+g\dd{g^{-1}}]&=S_{\textnormal{CS}}\qty[\vb A]+k\int\limits_M\Tr\left[\dd{g^{-1}}g\wedge \dd{\vb A}-\frac13\dd{g^{-1}}g\wedge \dd{g^{-1}}g\wedge \dd{g^{-1}}g+\dd{g^{-1}}g\wedge g^{-1}\dd{g}\wedge\vb A \right]\\
    S_{\textnormal{CS}}\qty[g\vb Ag^{-1}+g\dd{g^{-1}}]&=S_{\textnormal{CS}}\qty[\vb A]+k\int\limits_M\Tr\left[\dd{g^{-1}}g\wedge \dd{\vb A}-\frac13\dd{g^{-1}}g\wedge \dd{g^{-1}}g\wedge \dd{g^{-1}}g-\dd{\qty(\dd{g^{-1}}g\wedge\vb A )}-\dd{g^{-1}}g\wedge\dd{\vb A}\right]\\
    S_{\textnormal{CS}}\qty[g\vb Ag^{-1}+g\dd{g^{-1}}]&=S_{\textnormal{CS}}\qty[\vb A]-k\int\limits_M\Tr\left[\frac13\dd{g^{-1}}g\wedge \dd{g^{-1}}g\wedge \dd{g^{-1}}g+\dd{\qty(\dd{g^{-1}}g\wedge\vb A )}\right]
\end{align*}