\section{Concluding Remarks}

The two main results we have achieved here are summarized in \ref{CSEHEquivalence1}, \ref{CSEHEquivalence2}, that is, 
the equivalence of, classical, $2+1$--dimensional gravity as a gauge theory of the groups $ISO(2,1)$, $SO(3,1)$ and $SO(2,2)$, 
depending on the sign of the cosmological constant. What does this provides to us? Besides the realization 
of common sense knowledge of gravity being a gauge theory of the Poincaré Group, which we argued is not trivial to $D>3$, it also 
provides an well defined quantization procedure to this lower dimensional toy model, the one inherited of the Chern-Simons, more than 
that, the resulting theory is even renormalizable and possesses a zero beta function, this is due to the cosmological constant being a 
structure constant of the Lie Algebra and the $\kappa$ being related to the Chern-Simons level $k$, which is allowed only a 
discrete number of values. Nevertheless, quantization of $2+1$ gravity has also been argued only 
to be equivalent to Chern-Simons perturbatively, that is, expanded around a classical solutions, which in gravity 
have the restraint of the vielbein to be invertible, so it's not clear what sense should made of a non-classical solution as $\vb A=0\Rightarrow\vb e=\boldsymbol\omega=0$, 
and wether or not it should be included in the path integral, according to our cousin String Theory, we should allow only for invertible metrics, 
this might seem to be the case here, but the matter is not settled yet.