\section{Introduction}

General Relativity has been known for being a highly complex theory, to which, apart from highly symmetric ones, few solutions are known. One of 
the main reasons of this is the non-linear character of the Action and the Equations of Motion,
\begin{align*}
    S_{\textnormal{EH}}&=\frac{1}{2\kappa}\int\limits_M\dd[D]{x}\sqrt{\abs{g}}g^{ab}\tensor{R}{_c_b^c_a},\ \ \ R_{ab}-\frac12Rg_{ab}=T_{ab}\numberthis\label{EHAction}
\end{align*}
from the non-polinomial term $\sqrt{\abs{g}}$, the inverse field $g^{ab}$, and also $\tensor{R}{_c_b^c_a}$ where lies more 
contributions of inverse fields and derivatives of them. Thus, in the study of this theory, and couplings of it to matter, prospective of obtaining 
analytical solutions in non-highly symmetric scenarios is faded to doom, also, these peculiarities prevents the theory from being interpreted as a 
gauge theory in the usual sense\footnote{With \textit{gauge theory in the usual sense} we mean a theory in which the fundamental degree of freedom is a 
field $\vb A$ --- usually an adjoint $\mathfrak g$--valued 1--form --- which has as redundancy a local realization of a Lie Group $G$ which acts on the field as $\vb A\rightarrow g\vb Ag^{-1}+g\vb dg^{-1}$.}, 
these combined motifs have proven gravity in $D=3+1$ dimensions to be stubborn to the usual quantization methods, and even to classical non-vacuum solutions. 
Nevertheless, there is hope of grasping a better understanding of it --- either qualitative or quantitative --- by looking to simpler toy models, which, 
have already proven it's usefulness as in String Theory, with $D=1+1$ gravity, so, we would like to pursue a similar line of thought and 
ask ourselves, what can $D=2+1$ gravity teach us? We'll try to focus more on wether or not it can be stated as a usual gauge theory.

But first, why should $D=2+1$ be any easier of dealing with than $D=3+1$? The answer lies in the number of dynamical degrees of freedom of the theory, which in gravity, 
are deeply tied to the space-time dimensions, in the most common realization of it, the dynamical fundamental field is considered to be the metric, $\vb g$, 
which is a symmetric, non-degenerate $(0,2)$--tensor, to which is associated a metric-compatible, torsionless connection $\nabla_{\qty(\,\cdot\,)}\qty(\,\cdot\,):\mathfrak X\qty(M)\times \mathfrak X\qty(M)\rightarrow\mathfrak X(M)$, 
which under these conditions is totally determined by the metric, hence, all of the degrees of freedom of the theory are metric ones, which are $\frac12D\qty(D+1)$, but, as we have redundancies/gauge 
transformations of $D$ diffeomorphisms, $\phi_\ast\vb g\sim\vb g$, we have to discount them, also, there are $D$ additional constrains coming from the Bianchi identity, $\nabla_a\qty(R^{ab}-\frac12 Rg^{ab})=0$, 
so the total number of degree of freedom in this theory is, $\frac12D\qty(D+1)-2D=\frac12D\qty(D-3)$,
% a common choice is lowering the dimension of the problem, in this case, from $3+1$ to $2+1$ dimensions, why should this be a interesting theory to study? First, 
% the dynamical degrees of freedom of GR are tied to the dimension of the space-time, for a metric compatible, torsionless connection --- which is 
% naturally totally determined by the metric --- there are $\frac12D(D+1)$ free components of the metric, which are subject to $D$ redundancies from diffeomorphisms, 
% and also $D$ Bianchi identities,this gives as a total of $\frac12 D(D-3)$ free components
in $D=3+1$ this is our well known two polarizations of the 
metric, but, for $D=2+1$ this is zero, which can be interpreted as the metric having no dynamical degrees of freedom, that is, the equation of motion 
is merely an algebraic condition, this is compatible with our knowledge of independent components of the Riemann tensor, $\frac{1}{12}D^2(D^2-1)$, exactly $20$ 
in $D=3+1$, but $6$ in $D=2+1$, notice that $6=\frac123(3+1)$, the same number of degrees of freedom of a symmetric $(0,2)$ tensor, in other words, 
in $D=2+1$ the Riemann tensor is totally determined by the knowledge of the Ricci tensor, which is totally determined algebraically by the equations of motion. 
This is consistent with the metric doesn't having degrees 
of freedom, due to being known that dynamical propagation of gravity is linked to the Weyl tensor, and, if the Riemann tensor is totally determined by the 
Ricci tensor, there is no degree of freedom in the Weyl tensor, thus, no dynamics.
This is our hope to ``solve'' this theory, as it's non-dynamical, we expect it to be ``trivial'', or at least ``exact'' --- we have to define what we mean by this ---, 
similarly to what was done to lower dimensional electrodynamics by Schwinger.

% Metric:

% \begin{align*}
%     \dd{s}^2&=-A\qty(r)\dd{t}^2+\frac{1}{A\qty(r)}\dd{r}^2+r^2\dd{\Omega}^2\\
%     A\qty(r)&=-B\qty(2-3b_1B)\frac1r+\qty(1-3b_1B)+b_1r+b_2r^2
% \end{align*}

% Ricci:

% \begin{align*}
%     R&=6b_1\qty(B-r)\frac{1}{r^2}-12b_2\\
%     R'&=-12\frac{b_1}{r^3}\qty(B-\frac12r)
% \end{align*}

% Equation:

% \begin{align*}
%     6\beta\square R&=\gamma R+4\gamma\Lambda\\
%     \dv{}{r}\qty(A\qty(r)r^2R')&=\frac{\gamma}{6\beta}\qty( R+4\Lambda)r^2\\
%     -12b_1\dv{}{r}\qty(A\qty(r)\frac1r\qty(B-\frac12 r))&=\frac{\gamma}{6\beta}\qty( 6b_1\qty(B-r)\frac{1}{r^2}-12b_2+4\Lambda)r^2\\
%     -12b_1\dv{}{r}\qty(\qty(-B\qty(2-3b_1B)\frac1r+\qty(1-3b_1B)+b_1r+b_2r^2)\qty(\frac Br-\frac12))&=\frac{\gamma}{6\beta}\qty( 6b_1\qty(B-r)\frac{1}{r^2}-12b_2+4\Lambda)r^2\\
%     -12b_1\qty(B\qty(2-3b_1B)\frac{1}{r^2}+b_1+2b_2r)\qty(\frac Br-\frac12)+12b_1\qty(-B\qty(2-3b_1B)\frac{1}{r}+\qty(1-3b_1B)+b_1r+b_2r^2)\frac{B}{r^2}&=\frac{\gamma}{6\beta}\qty( 6b_1\qty(B-r)\frac{1}{r^2}-12b_2+4\Lambda)r^2\\
%     -r^{-3}24b_1B\qty(2-3b_1B)+r^{-2}12b_1B\qty(3-6b_1B)+6b_1^2-12b_1b_2B+r12b_1b_2&=\frac{\gamma}{6\beta}\qty( 6b_1B-6b_1r-4r^2\qty(3b_2-\Lambda))
% \end{align*}

% Thus:
% \begin{align*}
%     B\qty(2-3b_1B)&=0\\
%     6b_1-12b_2B&=\frac\gamma\beta B\\
%     12b_2&=-\frac\gamma\beta \\
%     3b_2&=\Lambda
% \end{align*}