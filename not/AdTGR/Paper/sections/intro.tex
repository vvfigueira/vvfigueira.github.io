\section{Introduction}

General Relativity has been known for being a highly complex theory, to which few solutions are known, apart from highly symmetric ones, one of 
the main reasons of this is the non-linear character of the Action and the Equations of Motion,
\begin{align*}
    S_{\textnormal{EH}}&=\frac{1}{2\kappa}\int\limits_M\dd[D]{x}\sqrt{\abs{g}}g^{ab}\tensor{R}{_c_b^c_a}
\end{align*}

Apart from the non-polinomial term $\sqrt{\abs{g}}$, and the inverse field $g^{ab}$ showing up in the Action, inside $\tensor{R}{_c_b^c_a}$ lies more 
contributions of inverse fields and derivatives of them. Thus, in the study of this theory, and couplings of it to matter, prospective of obtaining 
analytical solutions in non-highly symmetric scenarios is faded to doom, but, we can try to understand more of this if we look at simpler toy models, 
a common choice is lowering the dimension of the problem, in this case, from $3+1$ to $2+1$ dimensions, why should this be a interesting theory to study? First, 
the dynamical degrees of freedom of GR are tied to the dimension of the space-time, for a metric compatible, torsionless connection --- which is 
naturally totally determined by the metric --- there are $\frac12D(D+1)$ free components of the metric, which are subject to $D$ redundancies from diffeomorphisms, 
and also $D$ Bianchi identities,this gives as a total of $\frac12 D(D-3)$ free components, in $D=3+1$ this is our well known two polarizations of the 
metric, but for $D=2+1$ this is zero, which can be interpreted as the metric having no dynamical degrees of freedom, that is, the equation of motion 
is merely an algebraic condition, this is compatible with our knowledge of independent components of the Riemann tensor, $\frac{1}{12}D^2(D^2-1)$, exactly $20$ 
in $D=3+1$, but $6$ in $D=2+1$, notice that $6=\frac123(3+1)$, the same number of degrees of freedom of a symmetric $(0,2)$ tensor, in other words, 
in $D=2+1$ the Riemann tensor is totally determined by the knowledge of the Ricci tensor, which is totally determined algebraically by the equations of motion. 
This is compatible with the metric doesn't having degrees 
of freedom, due to being known that dynamical propagation of gravity is linked to the Weyl tensor, and, if the Riemann tensor is totally determined by the 
Ricci tensor, there is no degree of freedom in the Weyl tensor, thus, no dynamics.
This is our hope to solve this theory, as it's non-dynamical, we expect it to be ``trivial'', or at least ``exact'' --- we have to define what we mean by this ---, 
similarly to what was done to lower dimensional electrodynamics by Schwinger.

% Metric:

% \begin{align*}
%     \dd{s}^2&=-A\qty(r)\dd{t}^2+\frac{1}{A\qty(r)}\dd{r}^2+r^2\dd{\Omega}^2\\
%     A\qty(r)&=-B\qty(2-3b_1B)\frac1r+\qty(1-3b_1B)+b_1r+b_2r^2
% \end{align*}

% Ricci:

% \begin{align*}
%     R&=6b_1\qty(B-r)\frac{1}{r^2}-12b_2\\
%     R'&=-12\frac{b_1}{r^3}\qty(B-\frac12r)
% \end{align*}

% Equation:

% \begin{align*}
%     6\beta\square R&=\gamma R+4\gamma\Lambda\\
%     \dv{}{r}\qty(A\qty(r)r^2R')&=\frac{\gamma}{6\beta}\qty( R+4\Lambda)r^2\\
%     -12b_1\dv{}{r}\qty(A\qty(r)\frac1r\qty(B-\frac12 r))&=\frac{\gamma}{6\beta}\qty( 6b_1\qty(B-r)\frac{1}{r^2}-12b_2+4\Lambda)r^2\\
%     -12b_1\dv{}{r}\qty(\qty(-B\qty(2-3b_1B)\frac1r+\qty(1-3b_1B)+b_1r+b_2r^2)\qty(\frac Br-\frac12))&=\frac{\gamma}{6\beta}\qty( 6b_1\qty(B-r)\frac{1}{r^2}-12b_2+4\Lambda)r^2\\
%     -12b_1\qty(B\qty(2-3b_1B)\frac{1}{r^2}+b_1+2b_2r)\qty(\frac Br-\frac12)+12b_1\qty(-B\qty(2-3b_1B)\frac{1}{r}+\qty(1-3b_1B)+b_1r+b_2r^2)\frac{B}{r^2}&=\frac{\gamma}{6\beta}\qty( 6b_1\qty(B-r)\frac{1}{r^2}-12b_2+4\Lambda)r^2\\
%     -r^{-3}24b_1B\qty(2-3b_1B)+r^{-2}12b_1B\qty(3-6b_1B)+6b_1^2-12b_1b_2B+r12b_1b_2&=\frac{\gamma}{6\beta}\qty( 6b_1B-6b_1r-4r^2\qty(3b_2-\Lambda))
% \end{align*}

% Thus:
% \begin{align*}
%     B\qty(2-3b_1B)&=0\\
%     6b_1-12b_2B&=\frac\gamma\beta B\\
%     12b_2&=-\frac\gamma\beta \\
%     3b_2&=\Lambda
% \end{align*}