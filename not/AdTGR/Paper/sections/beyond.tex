\section{About D=3+1}

With our astonishing success achieved in $D=2+1$, a natural follow up is to ask what can be said about $D=3+1$. In the 
derivation of \ref{BilinearAnsatz}, we discarded $D\neq3$ reasoning that no non-zero action could be build, from $\vb e,\boldsymbol\omega$ with a 
multilinear trace in the algebra $\mathfrak{iso}\qty(D-1,1)$, due to the explicit appearance of $\comm{P_\mu}{P_\nu}$ inside the trace, which is naturally 
zero for $\mathfrak{iso}\qty(D-1,1)$, nevertheless, in our pursue of the inclusion of the cosmological constant, we concluded that 
for non-zero cosmological constant space-times, we should not be looking towards $\mathfrak{iso}\qty(D-1,1)$, but actually to $\mathfrak{so}\qty(D,1)$ and 
$\mathfrak{so}\qty(D-1,2)$, which have a non-zero translation commutation relation, \ref{SOAlgebra}. This means we can indeed build non-zero actions using a 
bilinear form in $D\neq 3$, as long as we stick to the non-zero cosmological constant case, to see how can this be done, we pick up from \ref{EHAction3}, with the cosmological constant term,
\begin{align*}
    S_{\textnormal{EH}}&=\frac{1}{4\kappa}\int\limits_M\epsilon_{\alpha\beta\mu\nu}\vb e^\alpha\wedge\vb e^\beta\wedge\vb R^{\mu\nu}-\frac{\Lambda}{\kappa4!}\int\limits_M\epsilon_{\alpha\beta\mu\nu}\vb e^\alpha\wedge\vb e^\beta\wedge\vb e^\mu\wedge\vb e^\nu
\end{align*}
which as always, we'll try to write as a bilinear, notice that each two vielbein terms will have to couple to a commutator of the translations, due to the wedge anti-symmetry, so our guess has to be,
\begin{align*}
    S_{\textnormal{EH}}&\stackrel{?}{=}-\frac{1}{4\Lambda\kappa}\expval{\comm{P_\alpha}{P_\beta},J_{\mu\nu}}\int\limits_M\vb e^\alpha\wedge\vb e^\beta\wedge\vb R^{\mu\nu}-\frac{1}{\Lambda\kappa4!}\expval{\comm{P_\alpha}{P_\beta},\comm{P_\mu}{P_\nu}}\int\limits_M\vb e^\alpha\wedge\vb e^\beta\wedge\vb e^\mu\wedge\vb e^\nu\\
    S_{\textnormal{EH}}&\stackrel{?}{=}-\frac{1}{2\Lambda\kappa}\int\limits_M\expval{\wedgecomm{\vb e}{\vb e}\ \wedgecomma\ \vb R}-\frac{1}{\Lambda\kappa4!}\int\limits_M\expval{\wedgecomm{\vb e}{\vb e}\ \wedgecomma\ \wedgecomm{\vb e}{\vb e}}\numberthis\label{EHActionTry1}
\end{align*}
this is indeed consistent, as in there exists a invariant symmetric non-degenerate bilinear form such,
\begin{align*}
    \expval{J_{\alpha\beta},J_{\mu\nu}}=\frac12\tensor{\epsilon}{_\rho_\sigma_\alpha_\beta}\tensor{\epsilon}{^\rho^\sigma_\mu_\nu},\ \ \ \expval{J_{\alpha\beta},P_\mu}=0,\ \ \ \expval{P_\mu,P_\nu}=\pm\frac{1}{L^2}\eta_{\mu\nu}
\end{align*}
the only problem with \ref{EHActionTry1} is the linearity in derivatives, as there is no gauge theory in $D=3+1$ with only linear derivatives terms, so it's not clear of what can be made of it. Of course, if we're interested in non-minimal 
Actions, it's possible to add further terms as $\expval{\vb R\ \wedgecomma\star\vb R}$ or $\expval{\vb R\ \wedgecomma\ \vb R}$, which are much more similar to Yang-Mills theories, but this would go more in the interpretation of 
decoupling the translations from the Poincaré group and interpreting gravity as a gauge theory of $SO\qty(D-1,1)$ coupled to a inertial frame field, which does not seem to be self consistent as for a kinetic term $\expval{\vb R\ \wedgecomma\star\vb R}$ 
is already coupled to the vielbein through the Hodge star, and a term like $\expval{\vb R\ \wedgecomma\ \vb R}$ does not provide dynamics as is topological.
% \begin{align*}
%     S_{\textnormal{CS}}\qty[\vb A]&=\frac{k}{4\pi}\int\limits_M\Tr\biggl[\boldsymbol\omega\ \wedgecomma\ \vb d{\vb e}+\vb e\ \wedgecomma\ \vb d{\boldsymbol\omega}+\vb e\ \wedgecomma\ \vb d{\vb e}\left.+\frac13\qty(\boldsymbol\omega+\vb e)\ \wedgecomma\ \qty(\wedgecomm{\boldsymbol\omega}{\vb e}+\wedgecomm{\vb e}{\boldsymbol\omega})\right]\\
%     S_{\textnormal{CS}}\qty[\vb A]&=\frac{k}{4\pi}\int\limits_M\Tr\biggl[\boldsymbol\omega\wedge\dd{\boldsymbol\omega}+x\dd{\vb e}\wedge\boldsymbol\omega+x\vb e\wedge\dd{\boldsymbol\omega}-\frac12x^2\dd{\qty(\vb e\wedge\vb e)}\\
%     &\quad\quad\quad\left.+\frac23\boldsymbol\omega\wedge\boldsymbol\omega\wedge\boldsymbol\omega+2x\boldsymbol\omega\wedge\boldsymbol\omega\wedge\vb e+2x^2\vb e\wedge\vb e\wedge \boldsymbol\omega+\frac23x^3\vb e\wedge\vb e\wedge\vb e\right]\\
%     S_{\textnormal{CS}}\qty[\vb A]&=\frac{k}{4\pi}\int\limits_M\Tr\left[\boldsymbol\omega\wedge\qty(\dd{\boldsymbol\omega}+\frac23\boldsymbol\omega\wedge\boldsymbol\omega)+x\dd{\qty(\vb e\wedge\boldsymbol\omega)}+x\vb e\wedge\dd{\boldsymbol\omega}+x\vb e\wedge\qty(\dd{\boldsymbol\omega}+2\boldsymbol\omega\wedge\boldsymbol\omega)+2x^2\vb e\wedge\boldsymbol\omega\wedge\vb e\right.\\
%     &\quad\quad\quad\left.+\frac23x^3\vb e\wedge\vb e\wedge\vb e\right]\\
%     S_{\textnormal{CS}}\qty[\vb A]&=S_{\textnormal{CS}}\qty[\boldsymbol\omega]+\frac{k}{4\pi}\int\limits_M\Tr\qty[x\dd{\qty(\vb e\wedge\boldsymbol\omega)}+2x\vb e\wedge\qty(\dd{\boldsymbol\omega}+\boldsymbol\omega\wedge\boldsymbol\omega)+2x^2\vb e\wedge\boldsymbol\omega\wedge\vb e+\frac23x^3\vb e\wedge\vb e\wedge\vb e]\\
%     S_{\textnormal{CS}}\qty[\vb A]&=S_{\textnormal{CS}}\qty[\boldsymbol\omega]+\frac{k}{4\pi}\int\limits_M\Tr\qty[x\dd{\qty(\vb e\wedge\boldsymbol\omega)}+2x\vb e\wedge\vb R+2x^2\vb e\wedge\boldsymbol\omega\wedge\vb e+\frac23x^3\vb e\wedge\vb e\wedge\vb e]\\
%     S_{\textnormal{CS}}\qty[\vb A]&=S_{\textnormal{CS}}\qty[\boldsymbol\omega]-\frac{xk}{\pi}\kappa S_{\textnormal{EH}}+\frac{k}{4\pi}\int\limits_M\Tr\qty[x\dd{\qty(\vb e\wedge\boldsymbol\omega)}+2x^2\vb e\wedge\boldsymbol\omega\wedge\vb e+\frac23x^3\vb e\wedge\vb e\wedge\vb e]
% \end{align*}

% \begin{align*}
%     S_{\textnormal{CS}}\qty[\vb A^x]&=\frac{k}{4\pi}\int\limits_M\Tr\qty[\qty(\boldsymbol\omega+x\vb e)\ \wedgecomma\ \qty(\vb d{\qty(\boldsymbol\omega+x\vb e)}+\frac13\wedgecomm{\boldsymbol\omega+x\vb e}{\boldsymbol\omega+x\vb e})]\\
%     S_{\textnormal{CS}}\qty[\vb A^x]&=\frac{k}{4\pi}\int\limits_M\Tr\qty[\boldsymbol\omega\ \wedgecomma\ \qty(\vb d\boldsymbol\omega+\frac13\wedgecomm{\boldsymbol\omega}{\boldsymbol\omega})]+\frac{kx}{4\pi}\int\limits_M\Tr\biggl[x\boldsymbol\omega\wedge\dd{\vb e}+x\vb e\wedge\dd{\boldsymbol\omega}+x^2\vb e\wedge\dd{\vb e}\\
%     &\quad\quad\quad\left.+\frac23\qty(\boldsymbol\omega+x\vb e)\wedge\qty(\boldsymbol\omega+x\vb e)\wedge\qty(\boldsymbol\omega+x\vb e)\right]\\
%     S_{\textnormal{CS}}\qty[\vb A^x]&=\frac{k}{4\pi}\int\limits_M\Tr\biggl[\boldsymbol\omega\wedge\dd{\boldsymbol\omega}+x\dd{\vb e}\wedge\boldsymbol\omega+x\vb e\wedge\dd{\boldsymbol\omega}-\frac12x^2\dd{\qty(\vb e\wedge\vb e)}\\
%     &\quad\quad\quad\left.+\frac23\boldsymbol\omega\wedge\boldsymbol\omega\wedge\boldsymbol\omega+2x\boldsymbol\omega\wedge\boldsymbol\omega\wedge\vb e+2x^2\vb e\wedge\vb e\wedge \boldsymbol\omega+\frac23x^3\vb e\wedge\vb e\wedge\vb e\right]\\
%     S_{\textnormal{CS}}\qty[\vb A^x]&=\frac{k}{4\pi}\int\limits_M\Tr\left[\boldsymbol\omega\wedge\qty(\dd{\boldsymbol\omega}+\frac23\boldsymbol\omega\wedge\boldsymbol\omega)+x\dd{\qty(\vb e\wedge\boldsymbol\omega)}+x\vb e\wedge\dd{\boldsymbol\omega}+x\vb e\wedge\qty(\dd{\boldsymbol\omega}+2\boldsymbol\omega\wedge\boldsymbol\omega)+2x^2\vb e\wedge\boldsymbol\omega\wedge\vb e\right.\\
%     &\quad\quad\quad\left.+\frac23x^3\vb e\wedge\vb e\wedge\vb e\right]\\
%     S_{\textnormal{CS}}\qty[\vb A^x]&=S_{\textnormal{CS}}\qty[\boldsymbol\omega]+\frac{k}{4\pi}\int\limits_M\Tr\qty[x\dd{\qty(\vb e\wedge\boldsymbol\omega)}+2x\vb e\wedge\qty(\dd{\boldsymbol\omega}+\boldsymbol\omega\wedge\boldsymbol\omega)+2x^2\vb e\wedge\boldsymbol\omega\wedge\vb e+\frac23x^3\vb e\wedge\vb e\wedge\vb e]\\
%     S_{\textnormal{CS}}\qty[\vb A^x]&=S_{\textnormal{CS}}\qty[\boldsymbol\omega]+\frac{k}{4\pi}\int\limits_M\Tr\qty[x\dd{\qty(\vb e\wedge\boldsymbol\omega)}+2x\vb e\wedge\vb R+2x^2\vb e\wedge\boldsymbol\omega\wedge\vb e+\frac23x^3\vb e\wedge\vb e\wedge\vb e]\\
%     S_{\textnormal{CS}}\qty[\vb A^x]&=S_{\textnormal{CS}}\qty[\boldsymbol\omega]-\frac{xk}{\pi}\kappa S_{\textnormal{EH}}+\frac{k}{4\pi}\int\limits_M\Tr\qty[x\dd{\qty(\vb e\wedge\boldsymbol\omega)}+2x^2\vb e\wedge\boldsymbol\omega\wedge\vb e+\frac23x^3\vb e\wedge\vb e\wedge\vb e]
% \end{align*}

% We can simplify if we sum two contributions,

% \begin{align*}
%     S_{\textnormal{CS}}\qty[\vb A^x]-S_{\textnormal{CS}}\qty[\vb A^{-x}]&=-8xk\kappa S_{\textnormal{EH}}+2kx\int\limits_{\partial M}\Tr\qty[\vb e\wedge\boldsymbol\omega]+\frac43x^3k\int\limits_M\Tr\qty[\vb e\wedge\vb e\wedge\vb e]\\
%     \frac{1}{8xk\kappa}\qty(S_{\textnormal{CS}}\qty[\vb A^{-x}]-S_{\textnormal{CS}}\qty[\vb A^x])&=S_{\textnormal{EH}}-\frac{1}{4\kappa}\int\limits_{\partial M}\Tr\qty[\vb e\wedge\boldsymbol\omega]-\frac{x^2}{3!\kappa}\int\limits_M\Tr\qty[\vb e\wedge\vb e\wedge\vb e]
% \end{align*}

% And also the not so usual action,

% \begin{align*}
%     S_{\textnormal{CS}}\qty[\vb A^x]+S_{\textnormal{CS}}\qty[\vb A^{-x}]&=2S_{\textnormal{CS}}\qty[\boldsymbol\omega]+4x^2k\int\limits_M\Tr\qty[\vb e\wedge\boldsymbol\omega\wedge\vb e]
% \end{align*}

% Chern-Simons Equation of Motion,

% \begin{align*}
%     S_{\textnormal{CS}}\qty[\vb A+\delta\vb A]-S_{\textnormal{CS}}\qty[\vb A]&=k\int\limits_M\Tr\qty[\delta\vb A\wedge\dd{\vb A}+\vb A\wedge\dd{\delta\vb A}+\frac23\delta\vb A\wedge\vb A\wedge\vb A+\frac23\vb A\wedge\delta\vb A\wedge\vb A+\frac23\vb A\wedge\vb A\wedge\delta\vb A]\\
%     S_{\textnormal{CS}}\qty[\vb A+\delta\vb A]-S_{\textnormal{CS}}\qty[\vb A]&=k\int\limits_M\Tr\qty[\dd{\vb A}\wedge\delta\vb A-\dd{\qty(\vb A\wedge\delta\vb A)}+\dd{\vb A}\wedge\delta \vb A+2\vb A\wedge\vb A\wedge\delta\vb A]\\
%     S_{\textnormal{CS}}\qty[\vb A+\delta\vb A]-S_{\textnormal{CS}}\qty[\vb A]&=2k\int\limits_M\Tr\qty[\qty(\dd{\vb A}+\vb A\wedge\vb A)\wedge\delta\vb A]-k\int\limits_M\Tr\qty[\dd{\qty(\vb A\wedge\delta\vb A)}]\\
%     S_{\textnormal{CS}}\qty[\vb A+\delta\vb A]-S_{\textnormal{CS}}\qty[\vb A]&=2k\int\limits_M\Tr\qty[\vb F\wedge\delta\vb A]-k\int\limits_{\partial M}\Tr\qty[\vb A\wedge\delta\vb A]
% \end{align*}

% Chern-Simons Gauge Invariance,

% \begin{align*}
%     S_{\textnormal{CS}}\qty[g\vb Ag^{-1}+g\dd{g^{-1}}]&=k\int\limits_M\Tr\qty[\qty(g\vb Ag^{-1}+g\dd{g^{-1}})\wedge g \vb Fg^{-1}-\frac13\qty(g\vb Ag^{-1}+g\dd{g^{-1}})\wedge\qty(g\vb Ag^{-1}+g\dd{g^{-1}})\wedge\qty(g\vb Ag^{-1}+g\dd{g^{-1}})]\\
%     S_{\textnormal{CS}}\qty[g\vb Ag^{-1}+g\dd{g^{-1}}]&=k\int\limits_M\Tr\left[\vb A\wedge\vb F+\dd{g^{-1}}g\wedge \vb F-\frac13\vb A\wedge\vb A\wedge\vb A-\frac13g\dd{g^{-1}}\wedge g\dd{g^{-1}}\wedge g\dd{g^{-1}}\right.\\
%     &\quad\quad\quad\left.-g\vb Ag^{-1}\wedge g\vb Ag^{-1}\wedge g\dd{g^{-1}}-g\vb A g^{-1}\wedge g\dd{g^{-1}}\wedge g\dd{g^{-1}}\right]\\
%     S_{\textnormal{CS}}\qty[g\vb Ag^{-1}+g\dd{g^{-1}}]&=S_{\textnormal{CS}}\qty[\vb A]+k\int\limits_M\Tr\left[\dd{g^{-1}}g\wedge \vb F-\frac13\dd{g^{-1}}g\wedge \dd{g^{-1}}g\wedge \dd{g^{-1}}g\right.\\
%     &\quad\quad\quad\left.-\dd{g^{-1}}g\wedge\vb A\wedge \vb A-\vb A \wedge \dd{g^{-1}}g\wedge \dd{g^{-1}}g\right]\\
%     S_{\textnormal{CS}}\qty[g\vb Ag^{-1}+g\dd{g^{-1}}]&=S_{\textnormal{CS}}\qty[\vb A]+k\int\limits_M\Tr\left[\dd{g^{-1}}g\wedge \dd{\vb A}-\frac13\dd{g^{-1}}g\wedge \dd{g^{-1}}g\wedge \dd{g^{-1}}g+\dd{g^{-1}}g\wedge g^{-1}\dd{g}\wedge\vb A \right]\\
%     S_{\textnormal{CS}}\qty[g\vb Ag^{-1}+g\dd{g^{-1}}]&=S_{\textnormal{CS}}\qty[\vb A]+k\int\limits_M\Tr\left[\dd{g^{-1}}g\wedge \dd{\vb A}-\frac13\dd{g^{-1}}g\wedge \dd{g^{-1}}g\wedge \dd{g^{-1}}g-\dd{\qty(\dd{g^{-1}}g\wedge\vb A )}-\dd{g^{-1}}g\wedge\dd{\vb A}\right]\\
%     S_{\textnormal{CS}}\qty[g\vb Ag^{-1}+g\dd{g^{-1}}]&=S_{\textnormal{CS}}\qty[\vb A]-k\int\limits_M\Tr\left[\frac13\dd{g^{-1}}g\wedge \dd{g^{-1}}g\wedge \dd{g^{-1}}g+\dd{\qty(\dd{g^{-1}}g\wedge\vb A )}\right]
% \end{align*}














% This already seems a lot like a proper gauge theory of the Chern-Simons type,
% Equations of motion are,
% \begin{align*}
%     S_{\textnormal{EH}}\qty[\vb e+\delta\vb e,\boldsymbol\omega]-S_{\textnormal{EH}}\qty[\vb e,\boldsymbol\omega]&=\frac{1}{2\kappa}\epsilon_{\mu\alpha\beta}\int\limits_M\delta\vb e^{\mu}\wedge\vb R^{\alpha\beta}=0\\
%     0&=-\frac12\epsilon_{\mu\alpha\beta}\vb R^{\alpha\beta}\\
%     0&=-\frac12\epsilon_{\mu\alpha\beta}\frac12\tensor{R}{_\rho_\sigma^\alpha^\beta}\vb e^\rho\wedge\vb e^\sigma\\
%     0&=-\frac14\epsilon_{\mu\alpha\beta}\tensor{R}{_\rho_\sigma^\alpha^\beta}\star\qty(\vb e^\rho\wedge\vb e^\sigma)\\
%     0&=-\frac14\epsilon_{\mu\alpha\beta}\tensor{R}{_\rho_\sigma^\alpha^\beta}\tensor{\epsilon}{^\rho^\sigma_\kappa}\vb e^\kappa\\
%     0&=-\frac14\epsilon_{\mu\alpha\beta}\tensor{\epsilon}{^\rho^\sigma^\kappa}\tensor{R}{_\rho_\sigma^\alpha^\beta}\vb e_\kappa\\
%     0&=-\frac{3!}{4}\tensor{\eta}{_\mu^{[\rho}}\tensor{\eta}{_\alpha^{\sigma}}\tensor{\eta}{_\beta^{\kappa]}}\tensor{R}{_\rho_\sigma^\alpha^\beta}\vb e_\kappa\\
%     0&=-\frac{3!}{4}\tensor{R}{_\rho_\sigma^{[\sigma}^\kappa}\tensor{\eta}{_\mu^{\rho]}}\vb e_\kappa\\
%     0&=-\frac14\qty(\tensor{R}{_\rho_\sigma^{\sigma}^\kappa}\tensor{\eta}{_\mu^{\rho}}+\tensor{R}{_\rho_\sigma^\kappa^\rho}\tensor{\eta}{_\mu^{\sigma}}+\tensor{R}{_\rho_\sigma^\rho^\sigma}\tensor{\eta}{_\mu^\kappa}-\tensor{R}{_\rho_\sigma^{\rho}^\kappa}\tensor{\eta}{_\mu^{\sigma}}-\tensor{R}{_\rho_\sigma^\kappa^\sigma}\tensor{\eta}{_\mu^{\rho}}-\tensor{R}{_\rho_\sigma^\sigma^\rho}\tensor{\eta}{_\mu^\kappa})\vb e_\kappa\\
%     0&=-\frac14\qty(-\tensor{R}{_\mu^\kappa}-\tensor{R}{_\mu^\kappa}+R\tensor{\eta}{_\mu^\kappa}-\tensor{R}{_\mu^\kappa}-\tensor{R}{_\mu^\kappa}+R\tensor{\eta}{_\mu^\kappa})\vb e_\kappa\\
%     0&=-\frac14(-4\tensor{R}{_\mu^\kappa}+2R\tensor{\eta}{_\mu^\kappa})\vb e_\kappa\\
%     0&=\qty(\tensor{R}{_\mu_\kappa}-\frac12R\tensor{\eta}{_\mu_\kappa})\vb e^\kappa
% \end{align*}