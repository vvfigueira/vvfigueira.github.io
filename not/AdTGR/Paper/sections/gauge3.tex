\section{Inclusion of a Cosmological Constant}

Having we successfully rewritten the EH Action in terms of a gauge theory, the next natural desire is to incorporate a cosmological 
constant, as it's of relevance for our universe. First, we must work out what is the appearance of this term in the form language, let's 
recall the action with the cosmological constant,
\begin{align*}
    S_{\textnormal{EH}}\rightarrow\frac{1}{2\kappa}\int\limits_M\dd[D]{x}\sqrt{\abs{g}}\qty(R-2\Lambda)\numberthis\label{EHActionWithCosmologicalConstant}
\end{align*}
so all we need is to rewrite the volume form, which we have already done in \ref{VolumeForm}, just substituting it back \ref{VolumeForm} and \ref{BilinearAnsatz},
\begin{align*}
    S_{\textnormal{EH}}&=\frac{1}{2\kappa}\int\limits_{M}\epsilon_{\mu\alpha\beta}\vb e^\mu\wedge\vb{R}^{\alpha\beta}-\frac{\Lambda}{\kappa3!}\int\limits_M\epsilon_{\mu\alpha\beta}\vb e^\mu\wedge\vb e^\alpha\wedge\vb e^\beta
\end{align*}
in the same line as we have been doing, we want the action to be writable in terms of the variables $\vb e=\vb e^\mu P_\mu$, the first term we already showed how can this be done, 
for the second one is where some subtleties start to appear. First, the striking difference between these two terms is that the second one has three instances of a gauge field, to be able to insert inside our 
bilinear three instances of the gauge field is only possible through using the anti-symmetry of the wedge product to produce a commutator, that is, we expect,
\begin{align*}
    -\frac{\Lambda}{\kappa 3!}\int\limits_M\epsilon_{\mu\alpha\beta}\vb e^\mu\wedge\vb e^\alpha\wedge\vb e^\beta\stackrel{?}{=}-\frac{\Lambda}{\kappa 3!}\int\limits_M\expval{P_\mu,\comm{P_\alpha}{P_\beta}}\vb e^\mu\wedge\vb e^\alpha\wedge\vb e^\beta=-\frac{\Lambda}{\kappa 3!}\int\limits_M\expval{\vb e\ \wedgecomma\ \wedgecomm{\vb e}{\vb e}}\stackrel{!}{=}0
\end{align*}

This is the only possible way of inserting three terms inside a bilinear, yet, it's flawed, the reason is obvious, $\comm{P_\alpha}{P_\beta}=0$. This means that it's inconsistent to add a cosmological constant 
and still requires the gauge group to be $ISO\qty(2,1)$. Is this correct? Yes! When introduced the cosmological constant, the solutions will of course be related to the (A)dS space-times, and in particular to the 
vacuum (A)dS space-times, the isometry group isn't $ISO(D-1,1)$ but actually is $SO(D-1,2)$ for AdS --- $\Lambda<0$ --- and $SO(D,1)$ for dS --- $\Lambda>0$ ---. Thus, alike to the $\Lambda=0$ case we worked out, 
we concluded that we should gauge $ISO(D-1,1)$, the isometry group of Minkowski space-time, analogously, with $\Lambda\neq 0$ we can only suppose then that we must gauge the respective 
isometry groups of the (A)dS space-times. Does this allows the inclusion of a cosmological constant? First, we have to look what is to be taken as translations from the groups $SO(3,1)$ and $SO(2,2)$, 
we'll not dwell on details here, but, all of the $6$ generators from these two groups can be rearranged in a similar form of the algebra of $ISO\qty(2,1)$ --- which can be seen as 
a non-central extension of the Poincaré group/algebra, this is consistent with the interpretation of Poincaré group being a Inönü-Wigner contraction of (A)dS ---,
\begin{align*}
    \comm{J^{\alpha\beta}}{J^{\mu\nu}}=4\eta^{[\alpha|[\mu}J^{\nu]|\beta]},\ \ \ \comm{J^{\alpha\beta}}{P^\mu}&=2P^{[\alpha}\eta^{\beta]\mu}, \ \ \ \comm{P^\mu}{P^\nu}=\pm\frac{1}{L^2}J^{\mu\nu}\ \ \begin{cases}+\rightarrow AdS\\-\rightarrow dS\end{cases}\numberthis\label{SOAlgebra}
\end{align*}
in which $L$ is the radius of the (A)dS space-time, it's related to the cosmological constant by, $\abs{\Lambda}=L^{-2}$, this perfectly matches with our guess to the form of this term,
\begin{align*}
    -\frac{\Lambda}{\kappa 3!}\int\limits_M\epsilon_{\mu\alpha\beta}\vb e^\mu\wedge\vb e^\alpha\wedge\vb e^\beta=-\frac{\Lambda}{\kappa 3!}\int\limits_M\expval{P_\mu,J_{\alpha\beta}}\vb e^\mu\wedge\vb e^\alpha\wedge\vb e^\beta\stackrel{!}{=}\frac{1}{\kappa 3!}\int\limits_M\expval{P_\mu,\comm{P_\alpha}{P_\beta}}\vb e^\mu\wedge\vb e^\alpha\wedge\vb e^\beta=\frac{1}{\kappa 3!}\int\limits_M\expval{\vb e\ \wedgecomma\ \wedgecomm{\vb e}{\vb e}}
\end{align*}
well, as long as the invariant symmetric non-degenerate bilinear still exists for these groups, a fact that would have to be checked, 
in fact there do exists such a bilinear, which is sadly not as pretty as our former,
\begin{align*}
    \expval{J_{\alpha\beta},P_\mu}=\epsilon_{\alpha\beta\mu},\ \ \ \expval{J_{\alpha\beta},J_{\mu\nu}}=\tensor{\epsilon}{^\rho_\alpha_\beta}\tensor{\epsilon}{_\rho_\mu_\nu},\ \ \ \expval{P_\mu,P_\nu}=\pm\frac{1}{L^2}\eta_{\mu\nu}\numberthis\label{BilinearSO}
\end{align*}
the existence of this bilinear is almost a miracle happening only in $D=2+1$, for $D\neq 3$ the algebras of the isometry groups of (A)dS 
allows only one bilinear, the Killing form, which if used here in $D=2+1$ would not give the desired result, as we'll comment later on. 
Now we can rework the form of the Chern-Simons Action, but, as the bilinear isn't that simple it's not really possible to get rid of the majority of the unwanted terms, so, we'll have to play a little with definitions and set the connection $\vb A^\pm=\boldsymbol\omega\pm\vb e$,
\begin{align*}
    S_{\textnormal{CS}}\qty[\vb A^\pm]&=\frac{k}{4\pi}\int\limits_M\expval{\qty(\boldsymbol\omega\pm\vb e)\ \wedgecomma\ \qty(\vb d{\qty(\boldsymbol\omega\pm\vb e)}+\frac13\wedgecomm{\boldsymbol\omega\pm\vb e}{\boldsymbol\omega\pm\vb e})}\\
    S_{\textnormal{CS}}\qty[\vb A^\pm]&=\frac{k}{4\pi}\int\limits_M\expval{\boldsymbol\omega\ \wedgecomma\ \qty(\vb d\boldsymbol\omega+\frac13\wedgecomm{\boldsymbol\omega}{\boldsymbol\omega})}\pm\frac{k}{4\pi}\int\limits_M\biggl\{\expval{\boldsymbol\omega\ \wedgecomma\ \vb d{\vb e}}+\expval{\vb e\ \wedgecomma\ \vb d{\boldsymbol\omega}}\mp\expval{\vb e\ \wedgecomma\ \vb d{\vb e}}\\
    &\quad\quad\quad\left.+\frac13\expval{\boldsymbol\omega\ \wedgecomma\ \qty(\wedgecomm{\boldsymbol\omega}{\vb e}+\wedgecomm{\vb e}{\boldsymbol\omega}\mp\wedgecomm{\vb e}{\vb e})}+\frac13\expval{\vb e\ \wedgecomma\ \qty(\wedgecomm{\boldsymbol\omega}{\boldsymbol\omega}\mp\wedgecomm{\boldsymbol\omega}{\vb e}\mp\wedgecomm{\vb e}{\boldsymbol\omega}+\wedgecomm{\vb e}{\vb e})}\right\}
\end{align*}
here lies the reason for us to considering the connection with a generic sign $\pm$, if we sum $S_{\textnormal{CS}}\qty[\vb A^+]-S_{\textnormal{CS}}\qty[\vb A^-]$ all the unwanted terms will vanish due them being even powers of $\vb e$,
\begin{align*}
    S_{\textnormal{CS}}\qty[\vb A^+]-S_{\textnormal{CS}}\qty[\vb A^-]&=2\frac{k}{4\pi}\int\limits_M\qty{\expval{\boldsymbol\omega\ \wedgecomma\ \vb d{\vb e}}+\expval{\vb e\ \wedgecomma\ \vb d{\boldsymbol\omega}}+\frac13\expval{\boldsymbol\omega\ \wedgecomma\ \qty(\wedgecomm{\boldsymbol\omega}{\vb e}+\wedgecomm{\vb e}{\boldsymbol\omega})}+\frac13\expval{\vb e\ \wedgecomma\ \qty(\wedgecomm{\boldsymbol\omega}{\boldsymbol\omega}+\wedgecomm{\vb e}{\vb e})}}
\end{align*}
now we follow be the usual means, grouping the remaining terms, 
integrating by parts, and using the invariance of the bilinear form,
\begin{align*}
    S_{\textnormal{CS}}\qty[\vb A^+]-S_{\textnormal{CS}}\qty[\vb A^-]&=\frac{k}{2\pi}\int\limits_M\left\{-\vb d\expval{\boldsymbol\omega\ \wedgecomma\ {\vb e}}+\expval{\vb d\boldsymbol\omega\ \wedgecomma\ {\vb e}}+\expval{\vb e\ \wedgecomma\ \vb d{\boldsymbol\omega}}+\frac23\expval{\wedgecomm{\boldsymbol\omega}{\boldsymbol\omega}\ \wedgecomma\ \vb e}+\frac13\expval{\vb e\ \wedgecomma\ \qty(\wedgecomm{\boldsymbol\omega}{\boldsymbol\omega}+\wedgecomm{\vb e}{\vb e})}\right\}\\
    S_{\textnormal{CS}}\qty[\vb A^+]-S_{\textnormal{CS}}\qty[\vb A^-]&=\frac{k}{2\pi}\int\limits_M\left\{-\vb d\expval{\boldsymbol\omega\ \wedgecomma\ {\vb e}}+2\expval{\vb e\ \wedgecomma\ \qty(\vb d{\boldsymbol\omega}+\frac12\wedgecomm{\boldsymbol\omega}{\boldsymbol\omega})}+\frac13\expval{\vb e\ \wedgecomma\ \wedgecomm{\vb e}{\vb e}}\right\}\\
    \frac{\pi}{k\kappa}\qty(S_{\textnormal{CS}}\qty[\vb A^+]-S_{\textnormal{CS}}\qty[\vb A^-])&=\frac{1}{2\kappa}\int\limits_{\partial M}\expval{{\vb e}\ \wedgecomma\ \boldsymbol\omega}+\frac1\kappa\int\limits_{M}\expval{\vb e\ \wedgecomma\ \vb R}+\frac{1}{\kappa3!}\int\limits_M\expval{\vb e\ \wedgecomma\ \wedgecomm{\vb e}{\vb e}}\numberthis\label{CSEHEquivalence2}
\end{align*}
exactly the result we're expecting to obtain, a relation between the EH Action with a cosmological constant term, and a --- actually two --- Chern-Simons theory! As previously discussed this process additionally gives half of the GHY boundary term.