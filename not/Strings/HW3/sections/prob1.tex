\problem{}
\probitem{}

For an operator $\mathcal O$ to be BRST closed, it means $\comm{Q_{\textnormal{BRST}}}{\mathcal O}=0$, and by 
now we're very familiar with a commutator being written as a contour integral, that is,
\begin{align*}
    \comm{Q_{\textnormal{BRST}}}{\mathcal O\qty(w,\bar w)}&=\frac{1}{2\pi\im}\oint\limits_{C_w}\qty(\dd{z}j_{\textnormal{BRST}}\qty(z)-\dd{\bar z}\tilde  j_{\textnormal{BRST}}\qty(\bar z))\mathcal O\qty(w,\bar w)\numberthis\label{commutator:closed}
\end{align*}
Where the BRST current is given by, \[j_{\textnormal{BRST}}\qty(z)=-\frac{1}{\alpha'}\cnord{c\partial X^\mu\partial X_\mu}\qty(z)+\cnord{bc\partial c}\qty(z)+\frac32\cnord{\partial^2 c}\qty(z)\numberthis\label{brstcurrent}\] 
And also $C_w$ is any closed contour encircling counterclockwise the point $w$. Actually, this definition is only good for 
closed strings --- in which operators can be inserted at any point ---, but, for the open string --- our case here ---, 
operators have to be inserted at the boundary Im$\qty(w)=0$. Hence, as we have to our disposal only half of the complex plane, 
it's impossible to have a closed curve $C_w$ that encloses a point at Im$\qty(w)=0$ --- unless the point itself 
belongs to the curve, under which the Cauchy residue theorem stops holding ---. Thus, the definition of the commutator 
for the open string is different,
\begin{align*}
    \comm{Q_{\textnormal{BRST}}}{\mathcal O\qty(w,\bar w)}&=\frac{1}{2\pi\im}\int\limits_{C_w'}\qty(\dd{z}j_{\textnormal{BRST}}\qty(z)-\dd{\bar z}\tilde  j_{\textnormal{BRST}}\qty(\bar z))\mathcal O\qty(w,\bar w)\numberthis\label{commutator:open}
\end{align*}
Where $C'_w$ is any counterclockwise oriented open curve starting and ending at Im$(z)=0$, such that the point $w$, Im$(w)=0$, lies in the interior of $C'_w$. 
To get a simpler expression, we can do the so called \textit{doubling trick}. The open string boundary conditions forces us to 
have at Im$(z)=0$ the following, \[\bar\partial X^\mu = \partial X^\mu,\ \ \ \tilde b=b,\ \ \ \tilde c=c,\ \ \ \textnormal{Im}\qty(z)=0\] 
Among other things, this also imply that at Im$(z)=0$ we have $\tilde j_{\textnormal{BRST}}=j_{\textnormal{BRST}}$. The doubling trick 
consists of we assigning $j_{\textnormal{BRST}}$ as being a operator over the whole $\mathbb C$. In the upper half it's already 
well defined, but in the lower half we define it as,
\[j_{\textnormal{BRST}}\qty(z)=\tilde j_{\textnormal{BRST}}\qty(z^\ast),\ \ \ \textnormal{Im}\qty(z)<0\]
Due to the open string boundary conditions this definition is continuous at Im$(z)=0$. We can use this to simplify \cref{commutator:open},
\begin{align*}
    \comm{Q_{\textnormal{BRST}}}{\mathcal O\qty(w,\bar w)}&=\frac{1}{2\pi\im}\int\limits_{C_w'}\dd{z}j_{\textnormal{BRST}}\qty(z)\mathcal O\qty(w,\bar w)-\frac{1}{2\pi\im}\int\limits_{C_w'}\dd{\bar z}\tilde  j_{\textnormal{BRST}}\qty(\bar z)O\qty(w,\bar w)\\
    \comm{Q_{\textnormal{BRST}}}{\mathcal O\qty(w,\bar w)}&=\frac{1}{2\pi\im}\int\limits_{C_w'}\dd{z}j_{\textnormal{BRST}}\qty(z)\mathcal O\qty(w,\bar w)-\frac{1}{2\pi\im}\int\limits_{{\bar C}_w'}\dd{ z}j_{\textnormal{BRST}}\qty( z)O\qty(w,\bar w)\\
    \comm{Q_{\textnormal{BRST}}}{\mathcal O\qty(w,\bar w)}&=\frac{1}{2\pi\im}\int\limits_{C_w'}\dd{z}j_{\textnormal{BRST}}\qty(z)\mathcal O\qty(w,\bar w)+\frac{1}{2\pi\im}\int\limits_{{{\bar C}_w}^{'-1}}\dd{ z}j_{\textnormal{BRST}}\qty( z)O\qty(w,\bar w)\\
    \comm{Q_{\textnormal{BRST}}}{\mathcal O\qty(w,\bar w)}&=\frac{1}{2\pi\im}\int\limits_{C_w'\cup {{\bar C}_w}^{'-1}}\dd{z}j_{\textnormal{BRST}}\qty(z)\mathcal O\qty(w,\bar w)\\
    \comm{Q_{\textnormal{BRST}}}{\mathcal O\qty(w,\bar w)}&=\frac{1}{2\pi\im}\oint\limits_{C_w}\dd{z}j_{\textnormal{BRST}}\qty(z)\mathcal O\qty(w,\bar w)\numberthis\label{commutator:open2}
\end{align*}
Here $\bar C_w'$ is the complex conjugated curve to $C_w'$, and ${{\bar C}_w}^{'-1}$ is the reverse orientation version of the curve $\bar C_w'$. 
Lastly, $C_w'\cup {{\bar C}_w}^{'-1}$ is the closed curve got by gluing the end of $C_w'$ with the start of $ {{\bar C}_w}^{'-1}$, which is 
not hard to see that is a counterclockwise oriented closed curve around $w$. The last expression, \cref{commutator:open2}, ease the 
calculation of the commutator in comparison to \cref{commutator:open}, because now we can use the Cauchy residue theorem.

Our main interest here is when $\mathcal O$ is a open string vertex operator, specially, \[V^a\qty(w,\bar w;\epsilon, k)=\lambda^a\epsilon_\mu\cnord{c\partial X^\mu \exp\qty(\im k\cdot X)}\qty(w,\bar w)\numberthis\label{vertexoperator1a}\]
It's clear that $V^a$ is not just a holomorphic, or just an anti-holomorphic, operator. What spoils it is the $\exp\qty(\im k\cdot X)$ part. Hence, 
the dependence on $\bar w$ is mandatory, also, as this is a open string vertex operator, it must be inserted at Im$(w)=0\Rightarrow w=\bar w$. 
Thus, $V^a$ actually is not dependent on $\bar w$, and we will omit such dependence here\footnote{If we where to 
be completely rigorous, we would need to change our normal ordering prescription.}. 

By \cref{commutator:open2}, we only need the knowledge of the first order pole of the OPE between the BRST current and the vertex operator in 
order to know the commutator. Hence, we'll start computing the OPE between those, it is in the same manner we did in the second 
homework set. First compute the normal ordering of the expression you want to know the OPE, this will involve among 
the OPE, other normal ordered terms, after gathering the divergent contributions we can set up the 
normal ordered terms as non-singular and obtain the OPE. As the full expression of the BRST 
current is quite big, and has a lot of terms, we'll compute each three contributions separately,
\begin{align*}
    -\cnord{\frac{1}{\alpha'}\cnord{c\partial X^\mu\partial X_\mu}\qty(z)V^a\qty(w;\epsilon,k)}&=-\frac{1}{\alpha'}\cnord{c\partial X^\mu\partial X_\mu}\qty(z)V^a\qty(w;\epsilon,k)\\
    &\quad\quad\quad-2\frac{\lambda^a\epsilon_\mu}{\alpha'}\cnord{\wick{(c\partial \c1X^\alpha\partial X_\alpha)\qty(z)(c\partial\c1 X^\mu\exp\qty(\im k\cdot X))\qty(w)}}\\
    &\quad\quad\quad-2\frac{\lambda^a\epsilon_\mu}{\alpha'}\cnord{\wick{(c\partial \c1X^\alpha\partial \c2X_\alpha)\qty(z)(c\partial\c1 X^\mu\exp(\im k\cdot\c2 X))\qty(w)}}\\
    &\quad\quad\quad-2\frac{\lambda^a\epsilon_\mu}{\alpha'}\cnord{\wick{(c\partial X^\alpha\partial \c2X_\alpha)\qty(z)(c\partial X^\mu\exp(\im k\cdot\c2 X))\qty(w)}}\\
    &\quad\quad\quad-\frac{\lambda^a\epsilon_\mu}{\alpha'}\cnord{\wick{(c\partial \c1X^\alpha\partial \c2X_\alpha)\qty(z)(c\partial X^\mu\exp(\im k\cdot\c1 X\mkern-4mu \c2 {\mathclap{\phantom{X}}}))\qty(w)}}\\
    -\cnord{\frac{1}{\alpha'}\cnord{c\partial X^\mu\partial X_\mu}\qty(z)V^a\qty(w;\epsilon,k)}&=-\frac{1}{\alpha'}\cnord{c\partial X^\mu\partial X_\mu}\qty(z)V^a\qty(w;\epsilon,k)\\
    &\quad\quad\quad-2\frac{\lambda^a\epsilon_\mu}{2}\eta^{\alpha\mu}\partial_z\partial_w\ln\abs{z-w}^2\cnord{(c\partial X_\alpha)\qty(z)(c\exp\qty(\im k\cdot X))\qty(w)}\\
    &+2\frac{\alpha'\lambda^a\epsilon_\mu}{4}\eta^{\alpha\mu}\im k_\alpha\partial_z\partial_w\ln\abs{z-w}^2\partial_z\ln\abs{z-w}^2\cnord{c\qty(z)(c\exp(\im k\cdot X))\qty(w)}\\
    &\quad\quad\quad-2\frac{\lambda^a\epsilon_\mu}{2}\im k_\alpha\partial_z\ln\abs{z-w}^2\cnord{(c\partial X^\alpha)\qty(z)(c\partial X^\mu\exp(\im k\cdot X))\qty(w)}\\
    &\quad\quad\quad+\frac{\alpha'\lambda^a\epsilon_\mu}{4}\im k_\alpha \im k^\alpha\qty(\partial_z\ln\abs{z-w}^2)^2\cnord{c\qty(z)(c\partial X^\mu\exp(\im k\cdot X))\qty(w)}\\
    -\cnord{\frac{1}{\alpha'}\cnord{c\partial X^\mu\partial X_\mu}\qty(z)V^a\qty(w;\epsilon,k)}&=-\frac{1}{\alpha'}\cnord{c\partial X^\mu\partial X_\mu}\qty(z)V^a\qty(w;\epsilon,k)\\
    &\quad\quad\quad-\frac{\lambda^a\epsilon_\mu}{\qty(z-w)^2}\cnord{(c\partial X^\mu)\qty(z)(c\exp\qty(\im k\cdot X))\qty(w)}\\
    &\quad\quad\quad+\frac{\im\alpha'\lambda^ak\cdot\epsilon}{2\qty(z-w)^3}\cnord{c\qty(z)(c\exp(\im k\cdot X))\qty(w)}\\
    &\quad\quad\quad-\frac{\im k_\alpha\lambda^a\epsilon_\mu}{z-w}\cnord{(c\partial X^\alpha)\qty(z)(c\partial X^\mu\exp(\im k\cdot X))\qty(w)}\\
    &\quad\quad\quad-\frac{k^2\alpha'\lambda^a\epsilon_\mu}{4\qty(z-w)^2}\cnord{c\qty(z)(c\partial X^\mu\exp(\im k\cdot X))\qty(w)}
\end{align*}
now we expand each function of $z$ in Taylor series around $z=w$, keeping only the terms which have a single simple pole,
\begin{align*}
    -\frac{1}{\alpha'}\cnord{c\partial X^\mu\partial X_\mu}\qty(z)V^a\qty(w;\epsilon,k)&=\frac{\lambda^a\epsilon_\mu}{z-w}\cnord{\partial cc\partial X^\mu\exp\qty(\im k\cdot X)}\qty(w)\\
    &\quad\quad\quad-\frac{\im\alpha'\lambda^ak\cdot\epsilon}{4\qty(z-w)}\cnord{\partial^2 cc\exp(\im k\cdot X)}\qty(w)\\
    &\quad\quad\quad+\frac{k^2\alpha'\lambda^a\epsilon_\mu}{4\qty(z-w)}\cnord{\partial cc\partial X^\mu\exp(\im k\cdot X)}\qty(w)\\
    &\quad\quad\quad+ \textnormal{ no single simple pole}\\
    -\frac{1}{\alpha'}\cnord{c\partial X^\mu\partial X_\mu}\qty(z)V^a\qty(w;\epsilon,k)&=\frac{\lambda^a\epsilon_\mu}{z-w}\qty(1+\frac{k^2\alpha'}{4})\cnord{\partial cc\partial X^\mu\exp\qty(\im k\cdot X)}\qty(w)\\
    &\quad\quad\quad-\frac{\im\alpha'\lambda^ak\cdot\epsilon}{4\qty(z-w)}\cnord{\partial^2 cc\exp(\im k\cdot X)}\qty(w)\\
    &\quad\quad\quad + \textnormal{ no single simple pole}\numberthis\label{first1a}
\end{align*}
The second term of the BRST current OPE gives,
\begin{align*}
    \cnord{\cnord{bc\partial c}\qty(z) V^a\qty(w;\epsilon,k)}&=\cnord{bc\partial c}\qty(z) V^a\qty(w;\epsilon,k)\\
    &\quad\quad\quad+\lambda^a\epsilon_\mu\cnord{\wick{(\c1 bc\partial c)\qty(z) (\c1c\partial X^\mu\exp\qty(\im k\cdot X))\qty(w)}}\\
    \cnord{\cnord{bc\partial c}\qty(z) V^a\qty(w;\epsilon,k)}&=\cnord{bc\partial c}\qty(z) V^a\qty(w;\epsilon,k)\\
    &\quad\quad\quad-\frac{\lambda^a\epsilon_\mu}{z-w}\cnord{(c\partial c)\qty(z) (\partial X^\mu\exp\qty(\im k\cdot X))\qty(w)}\\
    \cnord{bc\partial c}\qty(z) V^a\qty(w;\epsilon,k)&=\frac{\lambda^a\epsilon_\mu}{z-w}\cnord{(c\partial c)\qty(z) (\partial X^\mu\exp\qty(\im k\cdot X))\qty(w)}+\textnormal{ regular}\\
    \cnord{bc\partial c}\qty(z) V^a\qty(w;\epsilon,k)&=-\frac{\lambda^a\epsilon_\mu}{z-w}\cnord{\partial cc \partial X^\mu\exp\qty(\im k\cdot X)}\qty(w)\\
    &\quad\quad\quad+ \textnormal{ no single simple pole}\numberthis\label{second1a}
\end{align*}
The third term of the BRST current OPE is already non-singular,
\begin{align*}
    \cnord{\frac32\cnord{\partial ^2c}\qty(z) V^a\qty(w;\epsilon,k)}&=\frac32\partial ^2c\qty(z) V^a\qty(w;\epsilon,k)\\
    \frac32\partial ^2c\qty(z) V^a\qty(w;\epsilon,k)&= \textnormal{ no single simple pole}\numberthis\label{third1a}
\end{align*}
Summing the three contributions, \cref{first1a,second1a,third1a},
\begin{align*}
    j_{\textnormal{BRST}}\qty(z)V^a\qty(w;\epsilon, k)&=\frac{\lambda^a\epsilon_\mu}{z-w}\qty(1+\frac{k^2\alpha'}{4})\cnord{\partial cc\partial X^\mu\exp\qty(\im k\cdot X)}\qty(w)\\
    &\quad\quad\quad-\frac{\im\alpha'\lambda^ak\cdot\epsilon}{4\qty(z-w)}\cnord{\partial^2 cc\exp(\im k\cdot X)}\qty(w)\\
    &\quad\quad\quad-\frac{\lambda^a\epsilon_\mu}{z-w}\cnord{\partial cc \partial X^\mu\exp\qty(\im k\cdot X)}\qty(w)\\
    &\quad\quad\quad + \textnormal{ no single simple pole}\\
    j_{\textnormal{BRST}}\qty(z)V^a\qty(w;\epsilon, k)&=\frac{\lambda^a\epsilon_\mu}{z-w}\frac{k^2\alpha'}{4}\cnord{\partial cc\partial X^\mu\exp\qty(\im k\cdot X)}\qty(w)\\
    &\quad\quad\quad-\frac{\im\alpha'\lambda^ak\cdot\epsilon}{4\qty(z-w)}\cnord{\partial^2 cc\exp(\im k\cdot X)}\qty(w)\\
    &\quad\quad\quad + \textnormal{ no single simple pole}\numberthis\label{brstope1a}
\end{align*}
Using \cref{commutator:open2},
\begin{align*}
    \comm{Q_{\textnormal{BRST}}}{V^a\qty(w,\epsilon,k)}&=\frac{1}{2\pi\im}\oint\limits_{C_w}\dd{z}j_{\textnormal{BRST}}\qty(z)V^a\qty(w,\bar w)\\
    \comm{Q_{\textnormal{BRST}}}{V^a\qty(w,\epsilon,k)}&=\frac{1}{2\pi\im}\oint\limits_{C_w}\dd{z}\frac{\lambda^a\epsilon_\mu}{z-w}\frac{k^2\alpha'}{4}\cnord{\partial cc\partial X^\mu\exp\qty(\im k\cdot X)}\qty(w)\\
    &\quad\quad\quad-\frac{1}{2\pi\im}\oint\limits_{C_w}\dd{z}\frac{\im\alpha'\lambda^ak\cdot\epsilon}{4\qty(z-w)}\cnord{\partial^2 cc\exp(\im k\cdot X)}\qty(w)\\
    &\quad\quad\quad+\frac{1}{2\pi\im}\oint\limits_{C_w}\dd{z} \textnormal{ no single simple pole}\\
    \comm{Q_{\textnormal{BRST}}}{V^a\qty(w,\epsilon,k)}&=\lambda^a\epsilon_\mu\frac{k^2\alpha'}{4}\cnord{\partial cc\partial X^\mu\exp\qty(\im k\cdot X)}\qty(w)\\
    &\quad\quad\quad-\frac{\im\alpha'\lambda^ak\cdot\epsilon}{4}\cnord{\partial^2 cc\exp(\im k\cdot X)}\qty(w)\\
\end{align*}
The two operators showing up here, $\cnord{\partial cc\partial X^\mu\exp\qty(\im k\cdot X)}\qty(w)$ and $\cnord{\partial^2 cc\exp(\im k\cdot X)}\qty(w)$, 
are linear independent and also non identically zero. Hence, if we want to guarantee that $\comm{Q_{\textnormal{BRST}}}{V^a\qty(w,\epsilon,k)}=0$ 
we must set the coefficients in front of the two operator to zero. As $\lambda^a,\alpha'$ are both non zero, and we cannot guarantee that 
$\epsilon_\mu \cnord{\partial cc\partial X^\mu\exp\qty(\im k\cdot X)}\qty(w)$ is zero, the only option is to set to zero,
\[k^2=k\cdot \epsilon=0\]
Which is what we wanted to show.



% We also have to obtain the same OPE but for the anti-holomorphic BRST current,
% \[\tilde j_{\textnormal{BRST}}\qty(\bar z)=-\frac{1}{\alpha'}\cnord{\tilde c\bar\partial X^\mu\bar\partial X_\mu}\qty(\bar z)+\cnord{\tilde b\tilde c\bar\partial\tilde c}\qty(\bar z)+\frac32\cnord{{\bar\partial}^2\tilde c}\qty(\bar z)\numberthis\label{antibrstcurrent}\]
% The last two terms of this current, when contracted with the vertex operator, doesn't give any new poles, as all OPE's are finite. 
% The only possible poles are associated with the first term of the BRST current,
% \begin{align*}
%     \cnord{\tilde j_{\textnormal{BRST}}\qty(\bar z)V^a\qty(w;\epsilon, k)}&=j_{\textnormal{BRST}}\qty(z)V^a\qty(w;\epsilon, k)\\
%     &\quad\quad\quad-2\frac{\lambda^a\epsilon_\mu}{\alpha'}\cnord{\wick{(\tilde c\bar\partial \c1X^\alpha\bar\partial X_\alpha)(\bar z)(c\partial \c1X^\mu\exp\qty(\im k\cdot X))(w)}}\\
%     &\quad\quad\quad-2\frac{\lambda^a\epsilon_\mu}{\alpha'}\cnord{\wick{(\tilde c\bar\partial \c1X^\alpha\bar\partial X_\alpha)(\bar z)(c\partial X^\mu\exp(\im k\cdot \c1X))(w)}}\\
%     &\quad\quad\quad-2\frac{\lambda^a\epsilon_\mu}{\alpha'}\cnord{\wick{(\tilde c\bar\partial \c1X^\alpha\bar\partial \c2X_\alpha)(\bar z)(c\partial \c1X^\mu\exp(\im k\cdot\c2 X))(w)}}\\
%     &\quad\quad\quad-\frac{\lambda^a\epsilon_\mu}{\alpha'}\cnord{\wick{(\tilde c\bar\partial \c1X^\alpha\bar\partial\c2 X_\alpha)(\bar z)(c\partial X^\mu\exp(\im k\cdot\c1 X\mkern-4mu \c2 {\mathclap{\phantom{X}}}))(w)}}\\
%     \cnord{\tilde j_{\textnormal{BRST}}\qty(\bar z)V^a\qty(w;\epsilon, k)}&=j_{\textnormal{BRST}}\qty(z)V^a\qty(w;\epsilon, k)\\
%     &\quad\quad\quad-2\frac{\lambda^a\epsilon_\mu}{2}\eta^{\alpha\mu}\partial_{\bar z}\partial_w\ln\abs{z-w}^2\cnord{{(\tilde c\bar\partial X_\alpha)(\bar z)(c\exp\qty(\im k\cdot X))(w)}}\\
%     &\quad\quad\quad-2\frac{\lambda^a\epsilon_\mu}{2}\im k^\alpha\partial_{\bar z}\ln\abs{z-w}^2\cnord{{(\tilde c\bar\partial X_\alpha)(\bar z)(c\partial X^\mu\exp(\im k\cdot X))(w)}}\\
%     &\quad\quad\quad+2\frac{\alpha'\lambda^a\epsilon_\mu}{4}\tensor{\eta}{_\alpha^\mu}\partial_{\bar z}\partial_w\ln\abs{z-w}^2\im k^\alpha\partial_{\bar z}\ln\abs{z-w}^2\cnord{{\tilde c(\bar z)(c\exp(\im k\cdot X))(w)}}\\
%     &\quad\quad\quad+\frac{\alpha'\lambda^a\epsilon_\mu}{4}\im k_\alpha \im k^\alpha\qty(\partial_{\bar z}\ln\abs{z-w}^2)^2\cnord{{\tilde c(\bar z)(c\partial X^\mu\exp(\im k\cdot X))(w)}}\\
%     j_{\textnormal{BRST}}\qty(z)V^a\qty(w;\epsilon, k)&=\lambda^a\epsilon_\mu2\pi\delta^{\qty(2)}\qty(z-w)\cnord{{(\tilde c\bar\partial X^\mu)(\bar z)(c\exp\qty(\im k\cdot X))(w)}}\\
%     &\quad\quad\quad+\frac{\lambda^a\epsilon_\mu}{\bar z-\bar w}\im k^\alpha\cnord{{(\tilde c\bar\partial X_\alpha)(\bar z)(c\partial X^\mu\exp(\im k\cdot X))(w)}}\\
%     &\quad\quad\quad+\frac{\im\alpha'\lambda^ak\cdot\epsilon}{2\qty(\bar z-\bar w)}2\pi\delta^{\qty(2)}\qty(z-w)\cnord{{\tilde c(\bar z)(c\exp(\im k\cdot X))(w)}}\\
%     &\quad\quad\quad+\frac{k^2\alpha'\lambda^a\epsilon_\mu}{4\qty(\bar z-\bar w)^2}\cnord{{\tilde c(\bar z)(c\partial X^\mu\exp(\im k\cdot X))(w)}}\\
%     &\quad\quad\quad+\textnormal{ no single simple pole}
% \end{align*}
% Now what remains is to Taylor expand everything. The terms with a Dirac delta 
% are zero, due to the integration contour in \cref{commutator} not passing through the point $z=w$. It's 
% \begin{align*}
%     j_{\textnormal{BRST}}\qty(z)V^a\qty(w;\epsilon, k)&=\frac{\lambda^a\epsilon_\mu}{\bar z-\bar w}\im k^\alpha\cnord{{(\tilde c\bar\partial X_\alpha)(\bar z)(c\partial X^\mu\exp(\im k\cdot X))(w)}}\\
%     &\quad\quad\quad+\frac{k^2\alpha'\lambda^a\epsilon_\mu}{4\qty(\bar z-\bar w)^2}\cnord{{\tilde c(\bar z)(c\partial X^\mu\exp(\im k\cdot X))(w)}}\\
%     &\quad\quad\quad+\textnormal{ no single simple pole}
% \end{align*}

% \begin{align*}
%     \comm{Q_{\textnormal{BRST}}}{X^\mu}&=\qty(c\partial+\tilde c\bar\partial)X^\mu\\
%     \comm{Q_{\textnormal{BRST}}}{b}&=T^X+T^g\\
%     \comm{Q_{\textnormal{BRST}}}{c}&=c\partial c
% \end{align*}
% with,
% \begin{align*}
%     T^X&=\frac{1}{\alpha'}\qnord{\partial X^\mu\partial X_\mu},\ \ \ T^g=\cnord{c\partial b}-2\cnord{b\partial c}
% \end{align*}
% so,
% \begin{align*}
%     \comm{Q_{\textnormal{BRST}}}{V^a}&=\lambda^a\epsilon_\mu\comm{Q_{\textnormal{BRST}}}{c\partial X^\mu\exp\qty(\im k\cdot X)}\\
%     \comm{Q_{\textnormal{BRST}}}{V^a}&=\lambda^a\epsilon_\mu\comm{Q_{\textnormal{BRST}}}{c}\partial X^\mu\exp\qty(\im k\cdot X)-\lambda^a\epsilon_\mu c\comm{Q_{\textnormal{BRST}}}{\partial X^\mu\exp\qty(\im k\cdot X)}\\
%     \comm{Q_{\textnormal{BRST}}}{V^a}&=\lambda^a\epsilon_\mu c\partial c\partial X^\mu\exp\qty(\im k\cdot X)\\
%     &\quad\quad\quad-\lambda^a\epsilon_\mu c\partial X^\mu\comm{Q_{\textnormal{BRST}}}{\exp\qty(\im k\cdot X)}-\lambda^a\epsilon_\mu c\comm{Q_{\textnormal{BRST}}}{\partial X^\mu}\exp\qty(\im k\cdot X)\\
%     \comm{Q_{\textnormal{BRST}}}{V^a}&=\lambda^a\epsilon_\mu c\partial c\partial X^\mu\exp\qty(\im k\cdot X)\\
%     &\quad\quad\quad-\lambda^a\epsilon_\mu c\partial X^\mu\comm{Q_{\textnormal{BRST}}}{\exp\qty(\im k\cdot X)}-\lambda^a\epsilon_\mu c\partial\comm{Q_{\textnormal{BRST}}}{ X^\mu}\exp\qty(\im k\cdot X)\\
%     \comm{Q_{\textnormal{BRST}}}{V^a}&=\lambda^a\epsilon_\mu c\partial c\partial X^\mu\exp\qty(\im k\cdot X)\\
%     &\quad\quad\quad-\lambda^a\epsilon_\mu c\partial X^\mu\comm{Q_{\textnormal{BRST}}}{\exp\qty(\im k\cdot X)}-\lambda^a\epsilon_\mu c\partial\qty(c\partial+\tilde c\bar\partial)X^\mu\exp\qty(\im k\cdot X)\\
%     \comm{Q_{\textnormal{BRST}}}{V^a}&=\lambda^a\epsilon_\mu c\partial c\partial X^\mu\exp\qty(\im k\cdot X)\\
%     &\quad\quad\quad-\lambda^a\epsilon_\mu c\partial X^\mu\comm{Q_{\textnormal{BRST}}}{\exp\qty(\im k\cdot X)}-\lambda^a\epsilon_\mu c\partial\qty(c\partial+\tilde c\bar\partial)X^\mu\exp\qty(\im k\cdot X)\\
%     &\quad\quad\quad-\lambda^a\epsilon_\mu cc\partial-\lambda^a\epsilon_\mu c\partial\qty(\tilde c\bar\partial)X^\mu\exp\qty(\im k\cdot X)
% \end{align*}
\probitem{}

We need to show that $V^a\qty(w;\epsilon+ak,k)$, $a\in\mathbb R$, is just $V^a\qty(w;\epsilon,k)$ plus an BRST exact operator. What is a 
BRST exact operator? By the same means of the definition of a BRST closed operator, a BRST exact operator is one such $\mathcal O\qty(w,\bar w )
=\comm{Q_{\textnormal{BRST}}}{\mathcal O'\qty(w,\bar w)}$. Let's show this is the case,
\begin{align*}
    V^a\qty(w;\epsilon+ak,k)&=\lambda^a\epsilon_\mu\cnord{c\partial X^\mu\exp\qty(\im k\cdot X)}\qty(w)+a\lambda^ak_\mu\cnord{c\partial X^\mu\exp\qty(\im k\cdot X)}\qty(w)\\
    V^a\qty(w;\epsilon+ak,k)&=V^a\qty(w;\epsilon,k)+a\lambda^ak_\mu\cnord{c\partial X^\mu\exp\qty(\im k\cdot X)}\qty(w)\numberthis\label{tempeq2}
\end{align*}
The last term in the second line is our interest, notice,
\begin{align*}
    \lambda^ak_\mu\cnord{c\partial X^\mu\exp\qty(\im k\cdot X)}\qty(w)&=\lambda^a\oint\frac{\dd{z}}{2\pi\im}\qty(\frac{k_\mu\cnord{c\partial X^\mu\exp\qty(\im k\cdot X)}\qty(w)}{z-w}+\textnormal{ no single simple pole})\numberthis\label{tempeq1}
\end{align*}
As this term has to be BRST exact, the integrand has to be an OPE of some operator with the BRST current, 
the choice here of operator is obvious, as we have already an $c\partial X^\mu$, what seems a remainder of the 
first term of the BRST current, we'll try,
\begin{align*}
    \cnord{j_{\textnormal{BRST}}\qty(z)\cnord{\exp\qty(\im k\cdot X)}\qty(w)}&=j_{\textnormal{BRST}}\qty(z)\cnord{\exp\qty(\im k\cdot X)}\qty(w)\\
    &\quad\quad\quad-\frac{2}{\alpha'}\cnord{\wick{(c\partial X^\mu\partial \c1X_\mu)(z)\exp(\im k\cdot\c1 X)(w)}}\\
    &\quad\quad\quad-\frac{1}{\alpha'}\cnord{\wick{(c\partial \c2X^\mu\partial \c1X_\mu)(z)\exp(\im k\cdot\c1 X\mkern-4mu \c2 {\mathclap{\phantom{X}}})(w)}}\\
    &\quad\quad\quad+\textnormal{ no single simple pole}
\end{align*}
Both second and third terms of the BRST current doesn't have any wick contractions with this operator, hence, do not contribute with meaningful 
poles,
\begin{align*}
    \cnord{j_{\textnormal{BRST}}\qty(z)\cnord{\exp\qty(\im k\cdot X)}\qty(w)}&=j_{\textnormal{BRST}}\qty(z)\cnord{\exp\qty(\im k\cdot X)}\qty(w)\\
    &\quad\quad\quad-\frac{2}{\alpha'}\im k^\mu\frac{\alpha'}{2}\partial_z\ln\abs{z-w}^2\cnord{{(c\partial X_\mu)(z)\exp(\im k\cdot X)(w)}}\\
    &\quad\quad\quad+\frac{1}{\alpha'}\im^2 k^2\frac{{\alpha'}^2}{4}\qty(\partial_z\ln\abs{z-w}^2)^2\cnord{{c(z)\exp(\im k\cdot X)(w)}}\\
    &\quad\quad\quad+\textnormal{ no single simple pole}
\end{align*}
Expanding on Taylor, and also using $k^2=0$,
\begin{align*}
    j_{\textnormal{BRST}}\qty(z)\cnord{\exp\qty(\im k\cdot X)}\qty(w)&=\frac{\im k^\mu}{z-w}\cnord{{c\partial X_\mu\exp(\im k\cdot X)}}(w)\\
    &\quad\quad\quad+\textnormal{ no single simple pole}
\end{align*}
Using this result back in \cref{tempeq1},
\begin{align*}
    \lambda^ak_\mu\cnord{c\partial X^\mu\exp\qty(\im k\cdot X)}\qty(w)&=\lambda^a\oint\frac{\dd{z}}{2\pi\im}\qty(\frac{k_\mu\cnord{c\partial X^\mu\exp\qty(\im k\cdot X)}\qty(w)}{z-w}+\textnormal{ no single simple pole})\\
    \lambda^ak_\mu\cnord{c\partial X^\mu\exp\qty(\im k\cdot X)}\qty(w)&=-\im\lambda^a\oint\frac{\dd{z}}{2\pi\im}j_{\textnormal{BRST}}\qty(z)\cnord{\exp\qty(\im k\cdot X)}\qty(w)\\
    \lambda^ak_\mu\cnord{c\partial X^\mu\exp\qty(\im k\cdot X)}\qty(w)&=-\im\lambda^a\comm{Q_{\textnormal{BRST}}}{\cnord{\exp\qty(\im k\cdot X)}\qty(w)}
\end{align*}
At last, substituting in \cref{tempeq2},
\begin{align*}
    V^a\qty(w;\epsilon+ak,k)&=V^a\qty(w;\epsilon,k)-a\im\lambda^a\comm{Q_{\textnormal{BRST}}}{\cnord{\exp\qty(\im k\cdot X)}\qty(w)}
\end{align*}
Exactly what we wanted to show.

\probitem{}

Let's first state what is our approach here, it's well known that a three point correlation function in a CFT of 
fields with definite conformal weights has a closed form,
\[\expval{\phi_i\qty(z_i)\phi_j\qty(z_j)\phi_k\qty(z_k)}=C_{ijk}\abs{z_{ij}}^{h_k-h_i-h_j}\abs{z_{jk}}^{h_i-h_k-h_j}\abs{z_{ki}}^{h_j-h_i-h_k}\numberthis\label{threepointcft}\]
We can use this to our advantage as $V^a$, for the open string, which are inserted at $\bar w=w$, behaves as they were holomorphic fields, 
we just need to know the conformal weights of them, to get this we compute the OPE with $T=-\frac{1}{\alpha'}\cnord{\partial X^\alpha\partial X_\alpha}-\cnord{\partial b c}-2\cnord{b\partial c}$, 
\begin{align*}
    \cnord{T\qty(z) V^a\qty(w)}&=T\qty(z) V^a\qty(w)-\lambda^a\epsilon_\mu\frac{2}{\alpha'}\cnord{\wick{(\partial\c1 X^\alpha\partial X_\alpha)(z)(c\partial\c1 X^\mu\exp(\im k\cdot X))(w)}}\\
    &\quad\quad\quad-\lambda^a\epsilon_\mu\frac{2}{\alpha'}\cnord{\wick{(\partial\c1 X^\alpha\partial X_\alpha)(z)(c\partial X^\mu\exp(\im k\cdot \c1X))(w)}}\\
    &\quad\quad\quad-\lambda^a\epsilon_\mu\frac{2}{\alpha'}\cnord{\wick{(\partial\c1 X^\alpha\partial\c2 X_\alpha)(z)(c\partial\c2 X^\mu\exp(\im k\cdot \c1X))(w)}}\\
    &\quad\quad\quad-\lambda^a\epsilon_\mu\frac{1}{\alpha'}\cnord{\wick{(\partial\c1 X^\alpha\partial\c2 X_\alpha)(z)(c\partial X^\mu\exp(\im k\cdot \c1X\mkern-4mu \c2 {\mathclap{\phantom{X}}}))(w)}}\\
    &\quad\quad\quad-\lambda^a\epsilon_\mu\cnord{\wick{(\partial \c1 bc)(z)(\c1c\partial X^\mu\exp(\im k\cdot X))(w)}}\\
    &\quad\quad\quad-2\lambda^a\epsilon_\mu\cnord{\wick{(  \c1b\partial c)(z)(\c1c\partial X^\mu\exp(\im k\cdot X))(w)}}\\
    \cnord{T\qty(z) V^a\qty(w)}&=T\qty(z) V^a\qty(w)-\lambda^a\epsilon_\mu\eta^{\mu\alpha}\partial_z\partial_w\ln\abs{z-w}^2\cnord{{\partial X_\alpha(z)(c\exp(\im k\cdot X))(w)}}\\
    &\quad\quad\quad-\lambda^a\epsilon_\mu\im k^\alpha\partial_z\ln\abs{z-w}^2\cnord{{\partial X_\alpha(z)(c\partial X^\mu\exp(\im k\cdot X))(w)}}\\
    &\quad\quad\quad+\lambda^a\epsilon_\mu\frac{\alpha'}{2}\im k_\alpha\eta^{\alpha\mu}\partial_z\ln\abs{z-w}^2\partial_z\partial_w\ln\abs{z-w}^2\cnord{c\exp(\im k\cdot X)}(w)\\
    &\quad\quad\quad+\lambda^a\epsilon_\mu\frac{\alpha'}{4}\im^2 k^2\qty(\partial_z\ln\abs{z-w}^2)^2\cnord{c\partial X^\mu\exp(\im k\cdot X)}(w)\\
    &\quad\quad\quad-\lambda^a\epsilon_\mu\partial_z\qty(\frac{1}{z-w})\cnord{c(z)(\partial X^\mu\exp(\im k\cdot X))(w)}\\
    &\quad\quad\quad-2\lambda^a\epsilon_\mu\frac{1}{z-w}\cnord{\partial c(z)(\partial X^\mu\exp(\im k\cdot X))(w)}
\end{align*}
Setting now $k^2=k\cdot \epsilon=0$ and expanding in Taylor,
\begin{align*}
    T\qty(z) V^a\qty(w)&=\lambda^a\epsilon_\mu\frac{1}{\qty(z-w)^2}\cnord{{\partial X^\mu c\exp(\im k\cdot X)}}(w)\\
    &\quad\quad\quad+\lambda^a\epsilon_\mu\frac{1}{z-w}\cnord{{\partial^2 X^\mu c\exp(\im k\cdot X)}}(w)\\
    &\quad\quad\quad+\frac{\lambda^a\epsilon_\mu\im k^\alpha}{z-w}\cnord{{\partial X_\alpha c\partial X^\mu\exp(\im k\cdot X)}}(w)\\
    &\quad\quad\quad-\lambda^a\epsilon_\mu\frac{1}{\qty(z-w)^2}\cnord{c\partial X^\mu\exp(\im k\cdot X)}(w)\\
    &\quad\quad\quad-\lambda^a\epsilon_\mu\frac{1}{z-w}\cnord{\partial c\partial X^\mu\exp(\im k\cdot X)}(w)\\
    &\quad\quad\quad+2\lambda^a\epsilon_\mu\frac{1}{z-w}\cnord{\partial c\partial X^\mu\exp(\im k\cdot X)}(w)\\
    &\quad\quad\quad+\textnormal{ regular}\\
    T\qty(z) V^a\qty(w)&=+\lambda^a\epsilon_\mu\frac{1}{z-w}\cnord{{c\partial^2 X^\mu \exp(\im k\cdot X)}}(w)\\
    &\quad\quad\quad+\frac{\lambda^a\epsilon_\mu\im k^\alpha}{z-w}\cnord{{c\partial X_\alpha \partial X^\mu\exp(\im k\cdot X)}}(w)\\
    &\quad\quad\quad+\lambda^a\epsilon_\mu\frac{1}{z-w}\cnord{\partial c\partial X^\mu\exp(\im k\cdot X)}(w)\\
    &\quad\quad\quad+\textnormal{ regular}\\
    T\qty(z) V^a\qty(w)&=\frac{1}{z-w}\partial V^a\qty(w)+\textnormal{ regular}
\end{align*}
An astonishing result of $h=0$ for $V^a$. Using this result in \cref{threepointcft},
\begin{align*}
    \expval{V^{a_1}\qty(w_1;\epsilon_1,k_1)V^{a_2}\qty(w_2;\epsilon_2,k_2)V^{a_3}\qty(w_3;\epsilon_3,k_3)}&=\lambda^{a_1}\lambda^{a_2}\lambda^{a_3}C_{123}\qty(\epsilon_1,k_1;\epsilon_2,k_2;\epsilon_3,k_3)
\end{align*}

\probitem{}