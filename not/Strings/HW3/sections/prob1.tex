\problem{}
\probitem{}

For an operator $\mathcal O$ to be BRST closed, it means $\comm{Q_{\textnormal{BRST}}}{\mathcal O}=0$, and by 
now we're very familiar with a commutator being written as a contour integral, that is,
\begin{align*}
    \comm{Q_{\textnormal{BRST}}}{\mathcal O\qty(w,\bar w)}&=\frac{1}{2\pi\im}\oint\limits_{C_w}\qty(\dd{z}j_{\textnormal{BRST}}\qty(z)-\dd{\bar z}\tilde  j_{\textnormal{BRST}}\qty(\bar z))\mathcal O\qty(w,\bar w)\numberthis\label{commutator:closed}
\end{align*}
Where the BRST current is given by, \[j_{\textnormal{BRST}}\qty(z)=-\frac{1}{\alpha'}\cnord{c\partial X^\mu\partial X_\mu}\qty(z)+\cnord{bc\partial c}\qty(z)+\frac32\cnord{\partial^2 c}\qty(z)\numberthis\label{brstcurrent}\] 
And also $C_w$ is any closed contour encircling counterclockwise the point $w$. Actually, this definition is only good for 
closed strings --- in which operators can be inserted at any point ---, but, for the open string --- our case here ---, 
operators have to be inserted at the boundary Im$\qty(w)=0$. Hence, as we have to our disposal only half of the complex plane, 
it's impossible to have a closed curve $C_w$ that encloses a point at Im$\qty(w)=0$ --- unless the point itself 
belongs to the curve, under which the Cauchy residue theorem stops holding ---. Thus, the definition of the commutator 
for the open string is different,
\begin{align*}
    \comm{Q_{\textnormal{BRST}}}{\mathcal O\qty(w,\bar w)}&=\frac{1}{2\pi\im}\int\limits_{C_w'}\qty(\dd{z}j_{\textnormal{BRST}}\qty(z)-\dd{\bar z}\tilde  j_{\textnormal{BRST}}\qty(\bar z))\mathcal O\qty(w,\bar w)\numberthis\label{commutator:open}
\end{align*}
Where $C'_w$ is any counterclockwise oriented open curve starting and ending at Im$(z)=0$, such that the point $w$, Im$(w)=0$, lies in the interior of $C'_w$. 
To get a simpler expression, we can do the so called \textit{doubling trick}. The open string boundary conditions forces us to 
have at Im$(z)=0$ the following, \[\bar\partial X^\mu = \partial X^\mu,\ \ \ \tilde b=b,\ \ \ \tilde c=c,\ \ \ \textnormal{Im}\qty(z)=0\] 
Among other things, this also imply that at Im$(z)=0$ we have $\tilde j_{\textnormal{BRST}}=j_{\textnormal{BRST}}$. The doubling trick 
consists of we assigning $j_{\textnormal{BRST}}$ as being a operator over the whole $\mathbb C$. In the upper half it's already 
well defined, but in the lower half we define it as,
\[j_{\textnormal{BRST}}\qty(z)=\tilde j_{\textnormal{BRST}}\qty(z^\ast),\ \ \ \textnormal{Im}\qty(z)<0\]
Due to the open string boundary conditions this definition is continuous at Im$(z)=0$. We can use this to simplify \cref{commutator:open},
\begin{align*}
    \comm{Q_{\textnormal{BRST}}}{\mathcal O\qty(w,\bar w)}&=\frac{1}{2\pi\im}\int\limits_{C_w'}\dd{z}j_{\textnormal{BRST}}\qty(z)\mathcal O\qty(w,\bar w)-\frac{1}{2\pi\im}\int\limits_{C_w'}\dd{\bar z}\tilde  j_{\textnormal{BRST}}\qty(\bar z)O\qty(w,\bar w)\\
    \comm{Q_{\textnormal{BRST}}}{\mathcal O\qty(w,\bar w)}&=\frac{1}{2\pi\im}\int\limits_{C_w'}\dd{z}j_{\textnormal{BRST}}\qty(z)\mathcal O\qty(w,\bar w)-\frac{1}{2\pi\im}\int\limits_{{\bar C}_w'}\dd{ z}j_{\textnormal{BRST}}\qty( z)O\qty(w,\bar w)\\
    \comm{Q_{\textnormal{BRST}}}{\mathcal O\qty(w,\bar w)}&=\frac{1}{2\pi\im}\int\limits_{C_w'}\dd{z}j_{\textnormal{BRST}}\qty(z)\mathcal O\qty(w,\bar w)+\frac{1}{2\pi\im}\int\limits_{{{\bar C}_w}^{'-1}}\dd{ z}j_{\textnormal{BRST}}\qty( z)O\qty(w,\bar w)\\
    \comm{Q_{\textnormal{BRST}}}{\mathcal O\qty(w,\bar w)}&=\frac{1}{2\pi\im}\int\limits_{C_w'\cup {{\bar C}_w}^{'-1}}\dd{z}j_{\textnormal{BRST}}\qty(z)\mathcal O\qty(w,\bar w)\\
    \comm{Q_{\textnormal{BRST}}}{\mathcal O\qty(w,\bar w)}&=\frac{1}{2\pi\im}\oint\limits_{C_w}\dd{z}j_{\textnormal{BRST}}\qty(z)\mathcal O\qty(w,\bar w)\numberthis\label{commutator:open2}
\end{align*}
Here $\bar C_w'$ is the complex conjugated curve to $C_w'$, and ${{\bar C}_w}^{'-1}$ is the reverse orientation version of the curve $\bar C_w'$. 
Lastly, $C_w'\cup {{\bar C}_w}^{'-1}$ is the closed curve got by gluing the end of $C_w'$ with the start of $ {{\bar C}_w}^{'-1}$, which is 
not hard to see that is a counterclockwise oriented closed curve around $w$. The last expression, \cref{commutator:open2}, ease the 
calculation of the commutator in comparison to \cref{commutator:open}, because now we can use the Cauchy residue theorem.

Our main interest here is when $\mathcal O$ is a open string vertex operator, specially, \[V^a\qty(w,\bar w;\epsilon, k)=\lambda^a\epsilon_\mu\cnord{c\partial X^\mu \exp\qty(\im k\cdot X)}\qty(w,\bar w)\numberthis\label{vertexoperator1a}\]
It's clear that $V^a$ is not just a holomorphic, or just an anti-holomorphic, operator. What spoils it is the $\exp\qty(\im k\cdot X)$ part. Hence, 
the dependence on $\bar w$ is mandatory, also, as this is a open string vertex operator, it must be inserted at Im$(w)=0\Rightarrow w=\bar w$. 
Thus, $V^a$ actually is not dependent on $\bar w$, and we will omit such dependence here\footnote{If we where to 
be completely rigorous, we would need to change our normal ordering prescription.}. 

By \cref{commutator:open2}, we only need the knowledge of the first order pole of the OPE between the BRST current and the vertex operator in 
order to know the commutator. Hence, we'll start computing the OPE between those, it is in the same manner we did in the second 
homework set. First compute the normal ordering of the expression you want to know the OPE, this will involve among 
the OPE, other normal ordered terms, after gathering the divergent contributions we can set up the 
normal ordered terms as non-singular and obtain the OPE. As the full expression of the BRST 
current is quite big, and has a lot of terms, we'll compute each three contributions separately,
\begin{align*}
    -\cnord{\frac{1}{\alpha'}\cnord{c\partial X^\mu\partial X_\mu}\qty(z)V^a\qty(w;\epsilon,k)}&=-\frac{1}{\alpha'}\cnord{c\partial X^\mu\partial X_\mu}\qty(z)V^a\qty(w;\epsilon,k)\\
    &\quad\quad\quad-2\frac{\lambda^a\epsilon_\mu}{\alpha'}\cnord{\wick{(c\partial \c1X^\alpha\partial X_\alpha)\qty(z)(c\partial\c1 X^\mu\exp\qty(\im k\cdot X))\qty(w)}}\\
    &\quad\quad\quad-2\frac{\lambda^a\epsilon_\mu}{\alpha'}\cnord{\wick{(c\partial \c1X^\alpha\partial \c2X_\alpha)\qty(z)(c\partial\c1 X^\mu\exp(\im k\cdot\c2 X))\qty(w)}}\\
    &\quad\quad\quad-2\frac{\lambda^a\epsilon_\mu}{\alpha'}\cnord{\wick{(c\partial X^\alpha\partial \c2X_\alpha)\qty(z)(c\partial X^\mu\exp(\im k\cdot\c2 X))\qty(w)}}\\
    &\quad\quad\quad-\frac{\lambda^a\epsilon_\mu}{\alpha'}\cnord{\wick{(c\partial \c1X^\alpha\partial \c2X_\alpha)\qty(z)(c\partial X^\mu\exp(\im k\cdot\c1 X\mkern-4mu \c2 {\mathclap{\phantom{X}}}))\qty(w)}}\\
    -\cnord{\frac{1}{\alpha'}\cnord{c\partial X^\mu\partial X_\mu}\qty(z)V^a\qty(w;\epsilon,k)}&=-\frac{1}{\alpha'}\cnord{c\partial X^\mu\partial X_\mu}\qty(z)V^a\qty(w;\epsilon,k)\\
    &\quad\quad\quad-2\frac{\lambda^a\epsilon_\mu}{2}\eta^{\alpha\mu}\partial_z\partial_w\ln\abs{z-w}^2\cnord{(c\partial X_\alpha)\qty(z)(c\exp\qty(\im k\cdot X))\qty(w)}\\
    &+2\frac{\alpha'\lambda^a\epsilon_\mu}{4}\eta^{\alpha\mu}\im k_\alpha\partial_z\partial_w\ln\abs{z-w}^2\partial_z\ln\abs{z-w}^2\cnord{c\qty(z)(c\exp(\im k\cdot X))\qty(w)}\\
    &\quad\quad\quad-2\frac{\lambda^a\epsilon_\mu}{2}\im k_\alpha\partial_z\ln\abs{z-w}^2\cnord{(c\partial X^\alpha)\qty(z)(c\partial X^\mu\exp(\im k\cdot X))\qty(w)}\\
    &\quad\quad\quad+\frac{\alpha'\lambda^a\epsilon_\mu}{4}\im k_\alpha \im k^\alpha\qty(\partial_z\ln\abs{z-w}^2)^2\cnord{c\qty(z)(c\partial X^\mu\exp(\im k\cdot X))\qty(w)}\\
    -\cnord{\frac{1}{\alpha'}\cnord{c\partial X^\mu\partial X_\mu}\qty(z)V^a\qty(w;\epsilon,k)}&=-\frac{1}{\alpha'}\cnord{c\partial X^\mu\partial X_\mu}\qty(z)V^a\qty(w;\epsilon,k)\\
    &\quad\quad\quad-\frac{\lambda^a\epsilon_\mu}{\qty(z-w)^2}\cnord{(c\partial X^\mu)\qty(z)(c\exp\qty(\im k\cdot X))\qty(w)}\\
    &\quad\quad\quad+\frac{\im\alpha'\lambda^ak\cdot\epsilon}{2\qty(z-w)^3}\cnord{c\qty(z)(c\exp(\im k\cdot X))\qty(w)}\\
    &\quad\quad\quad-\frac{\im k_\alpha\lambda^a\epsilon_\mu}{z-w}\cnord{(c\partial X^\alpha)\qty(z)(c\partial X^\mu\exp(\im k\cdot X))\qty(w)}\\
    &\quad\quad\quad-\frac{k^2\alpha'\lambda^a\epsilon_\mu}{4\qty(z-w)^2}\cnord{c\qty(z)(c\partial X^\mu\exp(\im k\cdot X))\qty(w)}
\end{align*}
now we expand each function of $z$ in Taylor series around $z=w$, keeping only the terms which have a single simple pole,
\begin{align*}
    -\frac{1}{\alpha'}\cnord{c\partial X^\mu\partial X_\mu}\qty(z)V^a\qty(w;\epsilon,k)&=\frac{\lambda^a\epsilon_\mu}{z-w}\cnord{\partial cc\partial X^\mu\exp\qty(\im k\cdot X)}\qty(w)\\
    &\quad\quad\quad-\frac{\im\alpha'\lambda^ak\cdot\epsilon}{4\qty(z-w)}\cnord{\partial^2 cc\exp(\im k\cdot X)}\qty(w)\\
    &\quad\quad\quad+\frac{k^2\alpha'\lambda^a\epsilon_\mu}{4\qty(z-w)}\cnord{\partial cc\partial X^\mu\exp(\im k\cdot X)}\qty(w)\\
    &\quad\quad\quad+ \textnormal{ no single simple pole}\\
    -\frac{1}{\alpha'}\cnord{c\partial X^\mu\partial X_\mu}\qty(z)V^a\qty(w;\epsilon,k)&=\frac{\lambda^a\epsilon_\mu}{z-w}\qty(1+\frac{k^2\alpha'}{4})\cnord{\partial cc\partial X^\mu\exp\qty(\im k\cdot X)}\qty(w)\\
    &\quad\quad\quad-\frac{\im\alpha'\lambda^ak\cdot\epsilon}{4\qty(z-w)}\cnord{\partial^2 cc\exp(\im k\cdot X)}\qty(w)\\
    &\quad\quad\quad + \textnormal{ no single simple pole}\numberthis\label{first1a}
\end{align*}
The second term of the BRST current OPE gives,
\begin{align*}
    \cnord{\cnord{bc\partial c}\qty(z) V^a\qty(w;\epsilon,k)}&=\cnord{bc\partial c}\qty(z) V^a\qty(w;\epsilon,k)\\
    &\quad\quad\quad+\lambda^a\epsilon_\mu\cnord{\wick{(\c1 bc\partial c)\qty(z) (\c1c\partial X^\mu\exp\qty(\im k\cdot X))\qty(w)}}\\
    \cnord{\cnord{bc\partial c}\qty(z) V^a\qty(w;\epsilon,k)}&=\cnord{bc\partial c}\qty(z) V^a\qty(w;\epsilon,k)\\
    &\quad\quad\quad-\frac{\lambda^a\epsilon_\mu}{z-w}\cnord{(c\partial c)\qty(z) (\partial X^\mu\exp\qty(\im k\cdot X))\qty(w)}\\
    \cnord{bc\partial c}\qty(z) V^a\qty(w;\epsilon,k)&=\frac{\lambda^a\epsilon_\mu}{z-w}\cnord{(c\partial c)\qty(z) (\partial X^\mu\exp\qty(\im k\cdot X))\qty(w)}+\textnormal{ regular}\\
    \cnord{bc\partial c}\qty(z) V^a\qty(w;\epsilon,k)&=-\frac{\lambda^a\epsilon_\mu}{z-w}\cnord{\partial cc \partial X^\mu\exp\qty(\im k\cdot X)}\qty(w)\\
    &\quad\quad\quad+ \textnormal{ no single simple pole}\numberthis\label{second1a}
\end{align*}
The third term of the BRST current OPE is already non-singular,
\begin{align*}
    \cnord{\frac32\cnord{\partial ^2c}\qty(z) V^a\qty(w;\epsilon,k)}&=\frac32\partial ^2c\qty(z) V^a\qty(w;\epsilon,k)\\
    \frac32\partial ^2c\qty(z) V^a\qty(w;\epsilon,k)&= \textnormal{ no single simple pole}\numberthis\label{third1a}
\end{align*}
Summing the three contributions, \cref{first1a,second1a,third1a},
\begin{align*}
    j_{\textnormal{BRST}}\qty(z)V^a\qty(w;\epsilon, k)&=\frac{\lambda^a\epsilon_\mu}{z-w}\qty(1+\frac{k^2\alpha'}{4})\cnord{\partial cc\partial X^\mu\exp\qty(\im k\cdot X)}\qty(w)\\
    &\quad\quad\quad-\frac{\im\alpha'\lambda^ak\cdot\epsilon}{4\qty(z-w)}\cnord{\partial^2 cc\exp(\im k\cdot X)}\qty(w)\\
    &\quad\quad\quad-\frac{\lambda^a\epsilon_\mu}{z-w}\cnord{\partial cc \partial X^\mu\exp\qty(\im k\cdot X)}\qty(w)\\
    &\quad\quad\quad + \textnormal{ no single simple pole}\\
    j_{\textnormal{BRST}}\qty(z)V^a\qty(w;\epsilon, k)&=\frac{\lambda^a\epsilon_\mu}{z-w}\frac{k^2\alpha'}{4}\cnord{\partial cc\partial X^\mu\exp\qty(\im k\cdot X)}\qty(w)\\
    &\quad\quad\quad-\frac{\im\alpha'\lambda^ak\cdot\epsilon}{4\qty(z-w)}\cnord{\partial^2 cc\exp(\im k\cdot X)}\qty(w)\\
    &\quad\quad\quad + \textnormal{ no single simple pole}\numberthis\label{brstope1a}
\end{align*}
Using \cref{commutator:open2},
\begin{align*}
    \comm{Q_{\textnormal{BRST}}}{V^a\qty(w,\epsilon,k)}&=\frac{1}{2\pi\im}\oint\limits_{C_w}\dd{z}j_{\textnormal{BRST}}\qty(z)V^a\qty(w,\bar w)\\
    \comm{Q_{\textnormal{BRST}}}{V^a\qty(w,\epsilon,k)}&=\frac{1}{2\pi\im}\oint\limits_{C_w}\dd{z}\frac{\lambda^a\epsilon_\mu}{z-w}\frac{k^2\alpha'}{4}\cnord{\partial cc\partial X^\mu\exp\qty(\im k\cdot X)}\qty(w)\\
    &\quad\quad\quad-\frac{1}{2\pi\im}\oint\limits_{C_w}\dd{z}\frac{\im\alpha'\lambda^ak\cdot\epsilon}{4\qty(z-w)}\cnord{\partial^2 cc\exp(\im k\cdot X)}\qty(w)\\
    &\quad\quad\quad+\frac{1}{2\pi\im}\oint\limits_{C_w}\dd{z} \textnormal{ no single simple pole}\\
    \comm{Q_{\textnormal{BRST}}}{V^a\qty(w,\epsilon,k)}&=\lambda^a\epsilon_\mu\frac{k^2\alpha'}{4}\cnord{\partial cc\partial X^\mu\exp\qty(\im k\cdot X)}\qty(w)\\
    &\quad\quad\quad-\frac{\im\alpha'\lambda^ak\cdot\epsilon}{4}\cnord{\partial^2 cc\exp(\im k\cdot X)}\qty(w)\\
\end{align*}
The two operators showing up here, $\cnord{\partial cc\partial X^\mu\exp\qty(\im k\cdot X)}\qty(w)$ and $\cnord{\partial^2 cc\exp(\im k\cdot X)}\qty(w)$, 
are linear independent and also non identically zero. Hence, if we want to guarantee that $\comm{Q_{\textnormal{BRST}}}{V^a\qty(w,\epsilon,k)}=0$ 
we must set the coefficients in front of the two operator to zero. As $\lambda^a,\alpha'$ are both non zero, and we cannot guarantee that 
$\epsilon_\mu \cnord{\partial cc\partial X^\mu\exp\qty(\im k\cdot X)}\qty(w)$ is zero, the only option is to set to zero,
\[k^2=k\cdot \epsilon=0\]
Which is what we wanted to show.



% We also have to obtain the same OPE but for the anti-holomorphic BRST current,
% \[\tilde j_{\textnormal{BRST}}\qty(\bar z)=-\frac{1}{\alpha'}\cnord{\tilde c\bar\partial X^\mu\bar\partial X_\mu}\qty(\bar z)+\cnord{\tilde b\tilde c\bar\partial\tilde c}\qty(\bar z)+\frac32\cnord{{\bar\partial}^2\tilde c}\qty(\bar z)\numberthis\label{antibrstcurrent}\]
% The last two terms of this current, when contracted with the vertex operator, doesn't give any new poles, as all OPE's are finite. 
% The only possible poles are associated with the first term of the BRST current,
% \begin{align*}
%     \cnord{\tilde j_{\textnormal{BRST}}\qty(\bar z)V^a\qty(w;\epsilon, k)}&=j_{\textnormal{BRST}}\qty(z)V^a\qty(w;\epsilon, k)\\
%     &\quad\quad\quad-2\frac{\lambda^a\epsilon_\mu}{\alpha'}\cnord{\wick{(\tilde c\bar\partial \c1X^\alpha\bar\partial X_\alpha)(\bar z)(c\partial \c1X^\mu\exp\qty(\im k\cdot X))(w)}}\\
%     &\quad\quad\quad-2\frac{\lambda^a\epsilon_\mu}{\alpha'}\cnord{\wick{(\tilde c\bar\partial \c1X^\alpha\bar\partial X_\alpha)(\bar z)(c\partial X^\mu\exp(\im k\cdot \c1X))(w)}}\\
%     &\quad\quad\quad-2\frac{\lambda^a\epsilon_\mu}{\alpha'}\cnord{\wick{(\tilde c\bar\partial \c1X^\alpha\bar\partial \c2X_\alpha)(\bar z)(c\partial \c1X^\mu\exp(\im k\cdot\c2 X))(w)}}\\
%     &\quad\quad\quad-\frac{\lambda^a\epsilon_\mu}{\alpha'}\cnord{\wick{(\tilde c\bar\partial \c1X^\alpha\bar\partial\c2 X_\alpha)(\bar z)(c\partial X^\mu\exp(\im k\cdot\c1 X\mkern-4mu \c2 {\mathclap{\phantom{X}}}))(w)}}\\
%     \cnord{\tilde j_{\textnormal{BRST}}\qty(\bar z)V^a\qty(w;\epsilon, k)}&=j_{\textnormal{BRST}}\qty(z)V^a\qty(w;\epsilon, k)\\
%     &\quad\quad\quad-2\frac{\lambda^a\epsilon_\mu}{2}\eta^{\alpha\mu}\partial_{\bar z}\partial_w\ln\abs{z-w}^2\cnord{{(\tilde c\bar\partial X_\alpha)(\bar z)(c\exp\qty(\im k\cdot X))(w)}}\\
%     &\quad\quad\quad-2\frac{\lambda^a\epsilon_\mu}{2}\im k^\alpha\partial_{\bar z}\ln\abs{z-w}^2\cnord{{(\tilde c\bar\partial X_\alpha)(\bar z)(c\partial X^\mu\exp(\im k\cdot X))(w)}}\\
%     &\quad\quad\quad+2\frac{\alpha'\lambda^a\epsilon_\mu}{4}\tensor{\eta}{_\alpha^\mu}\partial_{\bar z}\partial_w\ln\abs{z-w}^2\im k^\alpha\partial_{\bar z}\ln\abs{z-w}^2\cnord{{\tilde c(\bar z)(c\exp(\im k\cdot X))(w)}}\\
%     &\quad\quad\quad+\frac{\alpha'\lambda^a\epsilon_\mu}{4}\im k_\alpha \im k^\alpha\qty(\partial_{\bar z}\ln\abs{z-w}^2)^2\cnord{{\tilde c(\bar z)(c\partial X^\mu\exp(\im k\cdot X))(w)}}\\
%     j_{\textnormal{BRST}}\qty(z)V^a\qty(w;\epsilon, k)&=\lambda^a\epsilon_\mu2\pi\delta^{\qty(2)}\qty(z-w)\cnord{{(\tilde c\bar\partial X^\mu)(\bar z)(c\exp\qty(\im k\cdot X))(w)}}\\
%     &\quad\quad\quad+\frac{\lambda^a\epsilon_\mu}{\bar z-\bar w}\im k^\alpha\cnord{{(\tilde c\bar\partial X_\alpha)(\bar z)(c\partial X^\mu\exp(\im k\cdot X))(w)}}\\
%     &\quad\quad\quad+\frac{\im\alpha'\lambda^ak\cdot\epsilon}{2\qty(\bar z-\bar w)}2\pi\delta^{\qty(2)}\qty(z-w)\cnord{{\tilde c(\bar z)(c\exp(\im k\cdot X))(w)}}\\
%     &\quad\quad\quad+\frac{k^2\alpha'\lambda^a\epsilon_\mu}{4\qty(\bar z-\bar w)^2}\cnord{{\tilde c(\bar z)(c\partial X^\mu\exp(\im k\cdot X))(w)}}\\
%     &\quad\quad\quad+\textnormal{ no single simple pole}
% \end{align*}
% Now what remains is to Taylor expand everything. The terms with a Dirac delta 
% are zero, due to the integration contour in \cref{commutator} not passing through the point $z=w$. It's 
% \begin{align*}
%     j_{\textnormal{BRST}}\qty(z)V^a\qty(w;\epsilon, k)&=\frac{\lambda^a\epsilon_\mu}{\bar z-\bar w}\im k^\alpha\cnord{{(\tilde c\bar\partial X_\alpha)(\bar z)(c\partial X^\mu\exp(\im k\cdot X))(w)}}\\
%     &\quad\quad\quad+\frac{k^2\alpha'\lambda^a\epsilon_\mu}{4\qty(\bar z-\bar w)^2}\cnord{{\tilde c(\bar z)(c\partial X^\mu\exp(\im k\cdot X))(w)}}\\
%     &\quad\quad\quad+\textnormal{ no single simple pole}
% \end{align*}

% \begin{align*}
%     \comm{Q_{\textnormal{BRST}}}{X^\mu}&=\qty(c\partial+\tilde c\bar\partial)X^\mu\\
%     \comm{Q_{\textnormal{BRST}}}{b}&=T^X+T^g\\
%     \comm{Q_{\textnormal{BRST}}}{c}&=c\partial c
% \end{align*}
% with,
% \begin{align*}
%     T^X&=\frac{1}{\alpha'}\qnord{\partial X^\mu\partial X_\mu},\ \ \ T^g=\cnord{c\partial b}-2\cnord{b\partial c}
% \end{align*}
% so,
% \begin{align*}
%     \comm{Q_{\textnormal{BRST}}}{V^a}&=\lambda^a\epsilon_\mu\comm{Q_{\textnormal{BRST}}}{c\partial X^\mu\exp\qty(\im k\cdot X)}\\
%     \comm{Q_{\textnormal{BRST}}}{V^a}&=\lambda^a\epsilon_\mu\comm{Q_{\textnormal{BRST}}}{c}\partial X^\mu\exp\qty(\im k\cdot X)-\lambda^a\epsilon_\mu c\comm{Q_{\textnormal{BRST}}}{\partial X^\mu\exp\qty(\im k\cdot X)}\\
%     \comm{Q_{\textnormal{BRST}}}{V^a}&=\lambda^a\epsilon_\mu c\partial c\partial X^\mu\exp\qty(\im k\cdot X)\\
%     &\quad\quad\quad-\lambda^a\epsilon_\mu c\partial X^\mu\comm{Q_{\textnormal{BRST}}}{\exp\qty(\im k\cdot X)}-\lambda^a\epsilon_\mu c\comm{Q_{\textnormal{BRST}}}{\partial X^\mu}\exp\qty(\im k\cdot X)\\
%     \comm{Q_{\textnormal{BRST}}}{V^a}&=\lambda^a\epsilon_\mu c\partial c\partial X^\mu\exp\qty(\im k\cdot X)\\
%     &\quad\quad\quad-\lambda^a\epsilon_\mu c\partial X^\mu\comm{Q_{\textnormal{BRST}}}{\exp\qty(\im k\cdot X)}-\lambda^a\epsilon_\mu c\partial\comm{Q_{\textnormal{BRST}}}{ X^\mu}\exp\qty(\im k\cdot X)\\
%     \comm{Q_{\textnormal{BRST}}}{V^a}&=\lambda^a\epsilon_\mu c\partial c\partial X^\mu\exp\qty(\im k\cdot X)\\
%     &\quad\quad\quad-\lambda^a\epsilon_\mu c\partial X^\mu\comm{Q_{\textnormal{BRST}}}{\exp\qty(\im k\cdot X)}-\lambda^a\epsilon_\mu c\partial\qty(c\partial+\tilde c\bar\partial)X^\mu\exp\qty(\im k\cdot X)\\
%     \comm{Q_{\textnormal{BRST}}}{V^a}&=\lambda^a\epsilon_\mu c\partial c\partial X^\mu\exp\qty(\im k\cdot X)\\
%     &\quad\quad\quad-\lambda^a\epsilon_\mu c\partial X^\mu\comm{Q_{\textnormal{BRST}}}{\exp\qty(\im k\cdot X)}-\lambda^a\epsilon_\mu c\partial\qty(c\partial+\tilde c\bar\partial)X^\mu\exp\qty(\im k\cdot X)\\
%     &\quad\quad\quad-\lambda^a\epsilon_\mu cc\partial-\lambda^a\epsilon_\mu c\partial\qty(\tilde c\bar\partial)X^\mu\exp\qty(\im k\cdot X)
% \end{align*}
\probitem{}

We need to show that $V^a\qty(w;\epsilon+ak,k)$, $a\in\mathbb R$, is just $V^a\qty(w;\epsilon,k)$ plus an BRST exact operator. What is a 
BRST exact operator? By the same means of the definition of a BRST closed operator, a BRST exact operator is one such $\mathcal O\qty(w,\bar w )
=\comm{Q_{\textnormal{BRST}}}{\mathcal O'\qty(w,\bar w)}$. Let's show this is the case,
\begin{align*}
    V^a\qty(w;\epsilon+ak,k)&=\lambda^a\epsilon_\mu\cnord{c\partial X^\mu\exp\qty(\im k\cdot X)}\qty(w)+a\lambda^ak_\mu\cnord{c\partial X^\mu\exp\qty(\im k\cdot X)}\qty(w)\\
    V^a\qty(w;\epsilon+ak,k)&=V^a\qty(w;\epsilon,k)+a\lambda^ak_\mu\cnord{c\partial X^\mu\exp\qty(\im k\cdot X)}\qty(w)\numberthis\label{tempeq2}
\end{align*}
The last term in the second line is our interest, notice,
\begin{align*}
    \lambda^ak_\mu\cnord{c\partial X^\mu\exp\qty(\im k\cdot X)}\qty(w)&=\lambda^a\oint\frac{\dd{z}}{2\pi\im}\qty(\frac{k_\mu\cnord{c\partial X^\mu\exp\qty(\im k\cdot X)}\qty(w)}{z-w}+\textnormal{ no single simple pole})\numberthis\label{tempeq1}
\end{align*}
As this term has to be BRST exact, the integrand has to be an OPE of some operator with the BRST current, 
the choice here of operator is obvious, as we have already an $c\partial X^\mu$, what seems a remainder of the 
first term of the BRST current, we'll try,
\begin{align*}
    \cnord{j_{\textnormal{BRST}}\qty(z)\cnord{\exp\qty(\im k\cdot X)}\qty(w)}&=j_{\textnormal{BRST}}\qty(z)\cnord{\exp\qty(\im k\cdot X)}\qty(w)\\
    &\quad\quad\quad-\frac{2}{\alpha'}\cnord{\wick{(c\partial X^\mu\partial \c1X_\mu)(z)\exp(\im k\cdot\c1 X)(w)}}\\
    &\quad\quad\quad-\frac{1}{\alpha'}\cnord{\wick{(c\partial \c2X^\mu\partial \c1X_\mu)(z)\exp(\im k\cdot\c1 X\mkern-4mu \c2 {\mathclap{\phantom{X}}})(w)}}\\
    &\quad\quad\quad+\textnormal{ no single simple pole}
\end{align*}
Both second and third terms of the BRST current doesn't have any wick contractions with this operator, hence, do not contribute with meaningful 
poles,
\begin{align*}
    \cnord{j_{\textnormal{BRST}}\qty(z)\cnord{\exp\qty(\im k\cdot X)}\qty(w)}&=j_{\textnormal{BRST}}\qty(z)\cnord{\exp\qty(\im k\cdot X)}\qty(w)\\
    &\quad\quad\quad-\frac{2}{\alpha'}\im k^\mu\frac{\alpha'}{2}\partial_z\ln\abs{z-w}^2\cnord{{(c\partial X_\mu)(z)\exp(\im k\cdot X)(w)}}\\
    &\quad\quad\quad+\frac{1}{\alpha'}\im^2 k^2\frac{{\alpha'}^2}{4}\qty(\partial_z\ln\abs{z-w}^2)^2\cnord{{c(z)\exp(\im k\cdot X)(w)}}\\
    &\quad\quad\quad+\textnormal{ no single simple pole}
\end{align*}
Expanding on Taylor, and also using $k^2=0$,
\begin{align*}
    j_{\textnormal{BRST}}\qty(z)\cnord{\exp\qty(\im k\cdot X)}\qty(w)&=\frac{\im k^\mu}{z-w}\cnord{{c\partial X_\mu\exp(\im k\cdot X)}}(w)\\
    &\quad\quad\quad+\textnormal{ no single simple pole}
\end{align*}
Using this result back in \cref{tempeq1},
\begin{align*}
    \lambda^ak_\mu\cnord{c\partial X^\mu\exp\qty(\im k\cdot X)}\qty(w)&=\lambda^a\oint\frac{\dd{z}}{2\pi\im}\qty(\frac{k_\mu\cnord{c\partial X^\mu\exp\qty(\im k\cdot X)}\qty(w)}{z-w}+\textnormal{ no single simple pole})\\
    \lambda^ak_\mu\cnord{c\partial X^\mu\exp\qty(\im k\cdot X)}\qty(w)&=-\im\lambda^a\oint\frac{\dd{z}}{2\pi\im}j_{\textnormal{BRST}}\qty(z)\cnord{\exp\qty(\im k\cdot X)}\qty(w)\\
    \lambda^ak_\mu\cnord{c\partial X^\mu\exp\qty(\im k\cdot X)}\qty(w)&=-\im\lambda^a\comm{Q_{\textnormal{BRST}}}{\cnord{\exp\qty(\im k\cdot X)}\qty(w)}
\end{align*}
At last, substituting in \cref{tempeq2},
\begin{align*}
    V^a\qty(w;\epsilon+ak,k)&=V^a\qty(w;\epsilon,k)-a\im\lambda^a\comm{Q_{\textnormal{BRST}}}{\cnord{\exp\qty(\im k\cdot X)}\qty(w)}
\end{align*}
Exactly what we wanted to show.

\probitem{}

This problem will be very heavy on CFT, let's state a few know results of correlation functions in CFTs,
\begin{align*}
    \expval{\phi_i\qty(z_i)}&=\delta_{h_i,0}C_i\numberthis\label{onepointcft}\\
    \expval{\phi_i\qty(z_i)\phi_j\qty(z_j)}&=\delta_{h_i,h_j}C_{ij}\abs{z_{ij}}^{h_i+h_j}\numberthis\label{twopointcft}\\
    \expval{\phi_i\qty(z_i)\phi_j\qty(z_j)\phi_k\qty(z_k)}&=C_{ijk}\abs{z_{ij}}^{h_k-h_i-h_j}\abs{z_{jk}}^{h_i-h_k-h_j}\abs{z_{ki}}^{h_j-h_i-h_k}\numberthis\label{threepointcft}
\end{align*}
We'll abuse these equations. The reason they're so useful is the two point one, which is zero unless the two fields have the 
same conformal weight. For this we'll need to know the conformal weight of $V^a$ --- for the open string, which is inserted at $\bar w=w$, behaves as they were holomorphic fields ---. 
To know the conformal weights of it is to compute the OPE with $T=-\frac{1}{\alpha'}\cnord{\partial X^\alpha\partial X_\alpha}-\cnord{\partial b c}-2\cnord{b\partial c}$, 
\begin{align*}
    \cnord{T\qty(z) V^a\qty(w)}&=T\qty(z) V^a\qty(w)-\lambda^a\epsilon_\mu\frac{2}{\alpha'}\cnord{\wick{(\partial\c1 X^\alpha\partial X_\alpha)(z)(c\partial\c1 X^\mu\exp(\im k\cdot X))(w)}}\\
    &\quad\quad\quad-\lambda^a\epsilon_\mu\frac{2}{\alpha'}\cnord{\wick{(\partial\c1 X^\alpha\partial X_\alpha)(z)(c\partial X^\mu\exp(\im k\cdot \c1X))(w)}}\\
    &\quad\quad\quad-\lambda^a\epsilon_\mu\frac{2}{\alpha'}\cnord{\wick{(\partial\c1 X^\alpha\partial\c2 X_\alpha)(z)(c\partial\c2 X^\mu\exp(\im k\cdot \c1X))(w)}}\\
    &\quad\quad\quad-\lambda^a\epsilon_\mu\frac{1}{\alpha'}\cnord{\wick{(\partial\c1 X^\alpha\partial\c2 X_\alpha)(z)(c\partial X^\mu\exp(\im k\cdot \c1X\mkern-4mu \c2 {\mathclap{\phantom{X}}}))(w)}}\\
    &\quad\quad\quad-\lambda^a\epsilon_\mu\cnord{\wick{(\partial \c1 bc)(z)(\c1c\partial X^\mu\exp(\im k\cdot X))(w)}}\\
    &\quad\quad\quad-2\lambda^a\epsilon_\mu\cnord{\wick{(  \c1b\partial c)(z)(\c1c\partial X^\mu\exp(\im k\cdot X))(w)}}\\
    \cnord{T\qty(z) V^a\qty(w)}&=T\qty(z) V^a\qty(w)-\lambda^a\epsilon_\mu\eta^{\mu\alpha}\partial_z\partial_w\ln\abs{z-w}^2\cnord{{\partial X_\alpha(z)(c\exp(\im k\cdot X))(w)}}\\
    &\quad\quad\quad-\lambda^a\epsilon_\mu\im k^\alpha\partial_z\ln\abs{z-w}^2\cnord{{\partial X_\alpha(z)(c\partial X^\mu\exp(\im k\cdot X))(w)}}\\
    &\quad\quad\quad+\lambda^a\epsilon_\mu\frac{\alpha'}{2}\im k_\alpha\eta^{\alpha\mu}\partial_z\ln\abs{z-w}^2\partial_z\partial_w\ln\abs{z-w}^2\cnord{c\exp(\im k\cdot X)}(w)\\
    &\quad\quad\quad+\lambda^a\epsilon_\mu\frac{\alpha'}{4}\im^2 k^2\qty(\partial_z\ln\abs{z-w}^2)^2\cnord{c\partial X^\mu\exp(\im k\cdot X)}(w)\\
    &\quad\quad\quad-\lambda^a\epsilon_\mu\partial_z\qty(\frac{1}{z-w})\cnord{c(z)(\partial X^\mu\exp(\im k\cdot X))(w)}\\
    &\quad\quad\quad-2\lambda^a\epsilon_\mu\frac{1}{z-w}\cnord{\partial c(z)(\partial X^\mu\exp(\im k\cdot X))(w)}
\end{align*}
Setting now $k^2=k\cdot \epsilon=0$ and expanding in Taylor,
\begin{align*}
    T\qty(z) V^a\qty(w)&=\lambda^a\epsilon_\mu\frac{1}{\qty(z-w)^2}\cnord{{\partial X^\mu c\exp(\im k\cdot X)}}(w)\\
    &\quad\quad\quad+\lambda^a\epsilon_\mu\frac{1}{z-w}\cnord{{\partial^2 X^\mu c\exp(\im k\cdot X)}}(w)\\
    &\quad\quad\quad+\frac{\lambda^a\epsilon_\mu\im k^\alpha}{z-w}\cnord{{\partial X_\alpha c\partial X^\mu\exp(\im k\cdot X)}}(w)\\
    &\quad\quad\quad-\lambda^a\epsilon_\mu\frac{1}{\qty(z-w)^2}\cnord{c\partial X^\mu\exp(\im k\cdot X)}(w)\\
    &\quad\quad\quad-\lambda^a\epsilon_\mu\frac{1}{z-w}\cnord{\partial c\partial X^\mu\exp(\im k\cdot X)}(w)\\
    &\quad\quad\quad+2\lambda^a\epsilon_\mu\frac{1}{z-w}\cnord{\partial c\partial X^\mu\exp(\im k\cdot X)}(w)\\
    &\quad\quad\quad+\textnormal{ regular}\\
    T\qty(z) V^a\qty(w)&=+\lambda^a\epsilon_\mu\frac{1}{z-w}\cnord{{c\partial^2 X^\mu \exp(\im k\cdot X)}}(w)\\
    &\quad\quad\quad+\frac{\lambda^a\epsilon_\mu\im k^\alpha}{z-w}\cnord{{c\partial X_\alpha \partial X^\mu\exp(\im k\cdot X)}}(w)\\
    &\quad\quad\quad+\lambda^a\epsilon_\mu\frac{1}{z-w}\cnord{\partial c\partial X^\mu\exp(\im k\cdot X)}(w)\\
    &\quad\quad\quad+\textnormal{ regular}\\
    T\qty(z) V^a\qty(w)&=\frac{1}{z-w}\partial V^a\qty(w)+\textnormal{ regular}
\end{align*}
An astonishing result of $h=0$ for $V^a$. We'll proceed as follow, first, we'll compute the OPE $VV$, 
which is given in terms of normal ordered terms, hence, our three point correlator will be transformed into a 
two point correlator. By \cref{twopointcft} and by the zero conformal weight of $V^a$, the only non zero contribution 
between the OPE $VV$ and the remaining $V$ in the correlator will be of the zero conformal weight part of the $VV$ 
OPE with $V$. At last, by \cref{onepointcft}, the only non zero contribution to the three point correlator will 
be the zero conformal weight part of the OPE between the zero conformal weight part of the $VV$ OPE 
with $V$. Before proceeding with this OPE computation, we'll need a lot's of other results which now we 
develop.

\subsubsection{Momentum conservation delta}

In further results we'll need to use the conservation of momentum $k_1+k_2+k_3=0$, 
which now we prove. Remember, 
we have as a conserved current the target-space momentum, $j^\mu\qty(z)=\frac{\im}{\alpha'}\partial X^\mu\qty(z)$, let's compute the OPE of this current with $V^a$,
\begin{align*}
    \cnord{j^\mu\qty(z)V^a\qty(w)}&=j^\mu\qty(z) V^a\qty(w)+\frac{\im\lambda^a\epsilon_\nu}{\alpha'}\cnord{\wick{\partial \c1X^\mu(z)(c\partial\c1 X^\nu\exp(\im k\cdot X))(w)}}\\
    &\quad\quad\quad+\frac{\im\lambda^a\epsilon_\nu}{\alpha'}\cnord{\wick{\partial\c1 X^\mu(z)(c\partial X^\nu\exp(\im k\cdot \c1X))(w)}}\\
    \cnord{j^\mu\qty(z)V^a\qty(w)}&=j^\mu\qty(z) V^a\qty(w)+\frac{\im\lambda^a\epsilon_\nu}{2}\eta^{\mu\nu}\partial_z\partial_w\ln\abs{z-w}^2\cnord{{c\exp(\im k\cdot X)}}(w)\\
    &\quad\quad\quad+\frac{\im^2\lambda^a\epsilon_\nu}{2}k^\mu\partial_z\ln\abs{z-w}^2\cnord{{c\partial X^\nu\exp(\im k\cdot X)}}(w)\\
    j^\mu\qty(z) V^a\qty(w)&=-\frac{\im\lambda^a\epsilon^\mu}{2\qty(z-w)^2}\cnord{{c\exp(\im k\cdot X)}}(w)+\frac{k^\mu V^a\qty(w)}{2(z-w)}+\textnormal{ regular}
\end{align*}
The conserved charge associated with the target-space conserved momentum --- for the open string --- is, $P^\mu=\oint\frac{\dd{z}}{2\pi\im}j^\mu\qty(z)$, 
hence,
\begin{align*}
    \comm{P^\mu}{V^a\qty(w)}&=\oint\limits_{C_w}\frac{\dd{z}}{2\pi\im}j^\mu\qty(z) V^a\qty(w)\\
    \comm{P^\mu}{V^a\qty(w)}&=\oint\limits_{C_w}\frac{\dd{z}}{2\pi\im}\qty(-\frac{\im\lambda^a\epsilon^\mu}{2\qty(z-w)^2}\cnord{{c\exp(\im k\cdot X)}}(w)+\frac{k^\mu V^a\qty(w)}{2(z-w)}+\textnormal{ regular})\\
    \comm{P^\mu}{V^a\qty(w)}&=\frac{k^\mu}{2}V^a(w)\numberthis\label{pvcomm}
\end{align*}
One may wonder about the usefulness of such expression. When computing $\expval{\mathcal O\qty(w,\bar w)}$, we're doing an 
expectation value of the operator $\mathcal O\qty(w,\bar w)$ over a state, in this case the vacuum. But, the vacuum 
itself is an eigenstate of the operator $P^\mu$, and as $P^{\dagger\mu}=P ^\mu$, the following is valid for 
any operator, \[\qty(P^{\dagger\mu}\Psi,\mathcal O\qty(w,\bar w)\Psi)-\qty(\Psi,\mathcal O\qty(w,\bar w)P^{\mu}\Psi)=\expval{P ^\mu\mathcal O\qty(w,\bar w)}-\expval{\mathcal O\qty(w,\bar w)P ^\mu}= \expval{\comm{P ^\mu}{\mathcal O\qty(w,\bar w)}}=0\]
In particular, this is valid to,
\[\expval{\comm{P ^\mu}{V^{a_1}\qty(w_1;\epsilon_1,k_1)V^{a_2}\qty(w_2;\epsilon_2,k_2)V^{a_3}\qty(w_3;\epsilon_3,k_3)}}=0\numberthis\label{tempeq3}\]
But,
\begin{align*}
    \comm{P ^\mu}{V^{a_1}V^{a_2}V^{a_3}}&=\comm{P ^\mu}{V^{a_1}}V^{a_2}V^{a_3}+V^{a_1}\comm{P ^\mu}{V^{a_2}}V^{a_3}+V^{a_1}V^{a_2}\comm{P ^\mu}{V^{a_3}}\\
    \comm{P ^\mu}{V^{a_1}V^{a_2}V^{a_3}}&=\frac{k_1^\mu}{2}V^{a_1}V^{a_2}V^{a_3}+\frac{k_2^\mu}{2}V^{a_1}V^{a_2}V^{a_3}+\frac{k_3^\mu}{2}V^{a_1}V^{a_2}V^{a_3}\\
    \comm{P ^\mu}{V^{a_1}V^{a_2}V^{a_3}}&=\frac{k_1^\mu+k_2^\mu+k_3^\mu}{2}V^{a_1}V^{a_2}V^{a_3}
\end{align*}
Combining this with \cref{tempeq3},
\begin{align*}
    0&=\expval{\comm{P ^\mu}{V^{a_1}V^{a_2}V^{a_3}}}=\frac{k_1^\mu+k_2^\mu+k_3^\mu}{2}\expval{V^{a_1}V^{a_2}V^{a_3}}
\end{align*}
This result is the same as saying,\[\expval{V^{a_1}V^{a_2}V^{a_3}}\propto\delta^{\qty(26)}\qty(k_1+k_2+k_3)\]

\subsubsection{Trace color factors}

Each vertex operator is \textit{dressed} with respect to it's color group by $\lambda^a$. 
So that the final correlator also should be dressed with respect to some number $\lambda^{a_1a_2a_3}$. 
We can fully determine such. $V^a$ is fermionic, hence, from the path integral point of view,
\begin{align*}
    \expval{V^{a_1}V^{a_3}V^{a_2}}&=\lambda^{a_1a_3a_2}\expval{\cnord{\epsilon_1\cdot \partial X\exp(\im k_1\cdot X)}\cnord{\epsilon_3\cdot \partial X\exp(\im k_3\cdot X)}\cnord{\epsilon_2\cdot \partial X\exp(\im k_2\cdot X)}}\\
    \expval{V^{a_1}V^{a_3}V^{a_2}}&=-\lambda^{a_1a_3a_2}\expval{\cnord{\epsilon_1\cdot \partial X\exp(\im k_1\cdot X)}\cnord{\epsilon_2\cdot \partial X\exp(\im k_2\cdot X)}\cnord{\epsilon_3\cdot \partial X\exp(\im k_3\cdot X)}}\\
    \expval{V^{a_1}V^{a_3}V^{a_2}}&=-\frac{\lambda^{a_1a_3a_2}}{\lambda^{a_1a_2a_3}}\expval{V^{a_1}V^{a_2}V^{a_3}}
\end{align*}
But, from the path integral point of view, $\expval{V^{a_1}V^{a_3}V^{a_2}}$ and $\expval{V^{a_1}V^{a_2}V^{a_3}}$ are equal, as long as 
the ordering of the points stay the same, hence, the above argument proves that $-\lambda^{a_1a_3a_2}=-\lambda^{a_1a_2a_3}$. 
And, similar arguments can be made to prove that $\lambda^{a_1a_2a_3}$ is in fact totally antisymmetric. There is a unique, 
up to multiplicative factors, three index totally antisymmetric object that can be constructed out of generators of the group $\lambda^a$, 
this is the structure constant $f^{a_1a_2a_3}$, which has a nice representation as $\Tr\qty[\lambda^{a_1}\comm{\lambda^{a_2}}{\lambda^{a_3}}]$. 
This completely determines the dependence on $\lambda$s,
\[\expval{V^{a_1}V^{a_2}V^{a_3}}\propto\Tr\qty[\lambda^{a_1}\comm{\lambda^{a_2}}{\lambda^{a_3}}]\]

\subsubsection{VV OPE}

Now we compute the $VV$ OPE, for now we forget about the color factors,
\begin{align*}
    \cnord{V_1V_2}&=V_1V_2+\epsilon_{1\mu}\epsilon_{2\nu}\cnord{\wick{(c\partial\c1 X^\mu\exp(\im k_1\cdot X))(w_1)(c\partial\c1 X^\nu\exp(\im k_2\cdot X))(w_2)}}\\
    &\quad\quad\quad+\epsilon_{1\mu}\epsilon_{2\nu}\cnord{\wick{(c\partial\c1 X^\mu\exp(\im k_1\cdot X))(w_1)(c\partial X^\nu\exp(\im k_2\cdot\c1 X))(w_2)}}\\
    &\quad\quad\quad-\epsilon_{1\mu}\epsilon_{2\nu}\cnord{\wick[offset=1em]{(c\partial\c1 X^\nu\exp(\im  k_2\cdot{X}))(w_2)(c\partial {X^\mu}\exp(\im k_1\cdot\c1 X))(w_1)}}\\
    &\quad\quad\quad-\epsilon_{1\mu}\epsilon_{2\nu}\cnord{\wick[offset=1em]{(c\partial\c1 X^\nu\c2\exp(\im  k_2\cdot{X}))(w_2)(c\partial \c2{X^\mu}\exp(\im k_1\cdot\c1 X))(w_1)}}\\
    \cnord{V_1V_2}&=V_1V_2+\frac{\alpha'}{2(w_1-w_2)^2}\epsilon_{1}\cdot\epsilon_{2}\cnord{{(c\exp(\im k_1\cdot X))(w_1)(c\exp(\im k_2\cdot X))(w_2)}}\\
    &\quad\quad\quad+\frac{\alpha'\im}{2(w_1-w_2)}\epsilon_{1}\cdot k_2\epsilon_{2\nu}\cnord{{(c\exp(\im k_1\cdot X))(w_1)(c\partial X^\nu\exp(\im k_2\cdot X))(w_2)}}\\
    &\quad\quad\quad+\frac{\alpha'\im}{2(w_2-w_1)}\epsilon_{1\mu}\epsilon_{2}\cdot k_1\cnord{{(c\partial X^\mu\exp(\im k_1\cdot X))(w_1)(c\exp(\im k_2\cdot X))(w_2)}}\\
    &\quad\quad\quad-\frac{{\alpha'}^2\im^2}{4(w_2-w_1)(w_1-w_2)}\epsilon_{1}\cdot k_2\epsilon_{2}\cdot k_1\cnord{{(c\exp(\im k_1\cdot X))(w_1)(c\exp(\im k_2\cdot X))(w_2)}}
\end{align*}
There are no contraction between two exponential because they are regular, as we know, 
$\cnord{\exp(\im k_1\cdot X)}\qty(z)\cnord{\exp(\im k_2\cdot X)}(w)=\abs{z-w}^{\alpha'k_1\cdot k_2}\cnord{\exp(\im k_1\cdot X)(z)\exp(\im k_2\cdot X)(w)}$, 
but, using the on-shell conditions and the conservation of momentum, $k_1\cdot k_2=\frac12\qty(k_1+k_2)^2=\frac12k_3^2=0$, hence, the product is regular. As we mentioned before, we only need the zero conformal weight part of this OPE, to obtain this is easy, first, 
expand the OPE in Taylor around $w_1=w_2$, then do 
the counting by using that $c$ has $-1$, $X$ has $0$ and $\exp(\im k\cdot X)$ has $\frac{\alpha'}{4}k^2=0$. Other way is 
to look for the $w_1-w_2$ prefactor after doing the Taylor expansion, the zero conformal weight contributions have no 
such prefactor. Doing this,
\begin{align*}
    V_1V_2&=-\frac{\alpha'}{4}\epsilon_{1}\cdot\epsilon_{2}\cnord{{\partial^2 c\exp(\im k_1\cdot X)c\exp(\im k_2\cdot X)}}(w_2)\\
    &\quad\quad\quad-\frac{\alpha'\im}{2}k_{1\mu}\epsilon_{1}\cdot\epsilon_{2}\cnord{{\partial c\partial X^\mu\exp(\im k_1\cdot X)c\exp(\im k_2\cdot X)}}(w_2)\\
    &\quad\quad\quad-\frac{\alpha'\im}{2}\epsilon_{1}\cdot k_2\epsilon_{2\nu}\cnord{{\partial c\exp(\im k_1\cdot X)c\partial X^\nu\exp(\im k_2\cdot X)}}(w_2)\\
    &\quad\quad\quad+\frac{\alpha'\im}{2}\epsilon_{1\mu}\epsilon_{2}\cdot k_1\cnord{{\partial c\partial X^\mu\exp(\im k_1\cdot X)c\exp(\im k_2\cdot X)}}(w_2)\\
    &\quad\quad\quad+\frac{{\alpha'}^2}{8}\epsilon_{1}\cdot k_2\epsilon_{2}\cdot k_1\cnord{{\partial^2(c\exp(\im k_1\cdot X))c\exp(\im k_2\cdot X)}}(w_2)\\
    &\quad\quad\quad+\textnormal{ $h=0$ terms}\\
    V_1V_2&=-\frac{\alpha'}{4}\qty(\epsilon_{1}\cdot\epsilon_{2}-\frac{\alpha'}{2}\epsilon_{1}\cdot k_2\epsilon_{2}\cdot k_1)\cnord{{\partial^2 cc\exp(\im (k_1+k_2)\cdot X)}}(w_2)\\
    &\quad-\frac{\alpha'\im}{2}\qty(k_{1\mu}\epsilon_{1}\cdot\epsilon_{2}+\epsilon_{1}\cdot k_2\epsilon_{2\mu}-\epsilon_{1\mu}\epsilon_{2}\cdot k_1-\frac{\alpha'}{2}k_{1\mu}\epsilon_1\cdot k_2\epsilon_2\cdot k_1)\cnord{{\partial cc\partial X^\mu\exp(\im(k_1+ k_2)\cdot X)}}(w_2)\\
    &\quad\quad\quad+\textnormal{ $h\neq0$ terms}\numberthis\label{VVope}
\end{align*}
The $h\neq0$ part of the OPE do not contributes to the three point function, so we can forget about it. Now, we compute the OPE of this OPE 
with the last $V$. We do it step by step, first with the first term of \cref{VVope}. Again, the exponential does not need to be contracted 
with another exponential, also we only keep terms with zero conformal weight,
\begin{align*}
    \cnord{{\partial^2 cc\exp(\im (k_1+k_2)\cdot X)}}(w_2)V_3&=\cnord{\cnord{{\partial^2 cc\exp(\im (k_1+k_2)\cdot X)}}(w_2)V_3}\\
    &\quad\quad+\epsilon_{3\alpha}\cnord{\wick{(c\partial \c1X^\alpha\exp(\im k_3\cdot X))(w_3)(\partial^2 cc\exp(\im (k_1+k_2)\cdot\c1 X))(w_2)}}\\
    \cnord{{\partial^2 cc\exp(\im (k_1+k_2)\cdot X)}}(w_2)V_3&=\cnord{\cnord{{\partial^2 cc\exp(\im (k_1+k_2)\cdot X)}}(w_2)V_3}\\
    &-\frac{\alpha'\im\epsilon_{3}\cdot\qty(k_1+k_2)}{2(w_3-w_2)}\cnord{{(\partial^2 cc\exp(\im (k_1+k_2)\cdot X))(w_2)(c\exp(\im k_3\cdot X))(w_3)}}\\
    \cnord{{\partial^2 cc\exp(\im (k_1+k_2)\cdot X)}}(w_2)V_3&=\frac{\alpha'\im\epsilon_{3}\cdot\qty(k_1+k_2)}{2}\cnord{{\partial^2 c\partial cc\exp(\im (k_1+k_2+k_3)\cdot X)}}(w_3)\\
    &\quad\quad\quad+\textnormal{ $h\neq0$ terms}\numberthis\label{VVVope1}\\
    \cnord{{\partial^2 cc\exp(\im (k_1+k_2)\cdot X)}}(w_2)V_3&=\textnormal{ $h\neq0$ terms}\numberthis\label{VVVope1}
\end{align*}
Where we used $k_1+k_2=-k_3$. Now the second term of \cref{VVope},
\begin{align*}
    \cnord{{\partial cc\partial X^\mu\exp(\im(k_1+ k_2)\cdot X)}}&(w_2)V_3=-\epsilon_{3\alpha}\cnord{\wick{(\partial cc\partial\c1 X^\mu\exp(\im(k_1+ k_2)\cdot X))(w_2)(c\partial\c1 X^\alpha\exp(\im k_3\cdot X))(w_3)}}\\
    &-\epsilon_{3\alpha}\cnord{\wick{(\partial cc\partial\c1 X^\mu\exp(\im(k_1+ k_2)\cdot X))(w_2)(c\partial X^\alpha\exp(\im k_3\cdot\c1 X))(w_3)}}\\
    &+\epsilon_{3\alpha}\cnord{\wick{(c\partial\c1 X^\alpha\exp(\im k_3\cdot X))(w_3)(\partial cc\partial X^\mu\exp(\im (k_1+k_2)\cdot\c1 X))(w_2)}}\\
    &+\epsilon_{3\alpha}\cnord{\wick{(c\partial\c1 X^\alpha\c2\exp(\im k_3\cdot X))(w_3)(\partial cc\partial\c2 X^\mu\exp(\im (k_1+k_2)\cdot\c1 X))(w_2)}}\\
    &\quad\quad\quad+\textnormal{ $h\neq0$ terms}\\
    \cnord{{\partial cc\partial X^\mu\exp(\im(k_1+ k_2)\cdot X)}}&(w_2)V_3=-\frac{\alpha'\epsilon_{3}^\mu}{2(w_2-w_3)^2}\cnord{{(\partial cc\exp(\im(k_1+ k_2)\cdot X))(w_2)(c\exp(\im k_3\cdot X))(w_3)}}\\
    &-\frac{\alpha'\im k_3^\mu\epsilon_{3\alpha}}{2(w_2-w_3)}\cnord{{(\partial cc\exp(\im(k_1+ k_2)\cdot X))(w_2)(c\partial X^\alpha\exp(\im k_3\cdot X))(w_3)}}\\
    &-\frac{\alpha'\im (k_1+k_2)\cdot\epsilon_{3}}{2(w_3-w_2)}\cnord{{(\partial cc\partial X^\mu\exp(\im (k_1+k_2)\cdot X))(w_2)(c\exp(\im k_3\cdot X))(w_3)}}\\
    &+\frac{{\alpha'}^2\im^2 k_3^\mu(k_1+k_2)\cdot\epsilon_{3}}{4(w_3-w_2)(w_2-w_3)}\cnord{{(\partial cc\exp(\im (k_1+k_2)\cdot X))(w_2)(c\exp(\im k_3\cdot X))(w_3)}}\\
    &\quad\quad\quad+\textnormal{ $h\neq0$ terms}\\
    \cnord{{\partial cc\partial X^\mu\exp(\im(k_1+ k_2)\cdot X)}}&(w_2)V_3=-\frac{\alpha'\epsilon_{3}^\mu}{2}\cnord{{\partial^2 c\partial c c\exp(\im(k_1+k_2+ k_3)\cdot X)}}(w_3)\\
    &\quad\quad\quad+\textnormal{ $h\neq0$ terms}\numberthis\label{VVVope2}
\end{align*}
Where we used $k_1+k_2=-k_3$. Putting together \cref{VVope,VVVope1,VVVope2},
\begin{align*}
    V_1V_2V_3&=\frac{{\alpha'}^2\im}{4}\cnord{{\partial^2 c\partial cc\exp(\im(k_1+k_2+ k_3)\cdot X)}}(w_3)\times\\
    &\quad\quad\quad\times\left[\epsilon_3^\mu\qty(k_{1\mu}\epsilon_{1}\cdot\epsilon_{2}+\epsilon_{1}\cdot k_2\epsilon_{2\mu}-\epsilon_{1\mu}\epsilon_{2}\cdot k_1-\frac{\alpha'}{2}k_{1\mu}\epsilon_1\cdot k_2\epsilon_2\cdot k_1)\right]\\
\end{align*}
Using that $\epsilon_3\cdot k_1=-\epsilon_3 \cdot k_2$ and similar,
\begin{align*}
    V_1V_2V_3&=\frac{{\alpha'}^2\im}{8}\cnord{{\partial^2 c\partial cc\exp(\im(k_1+k_2+ k_3)\cdot X)}}(w_3)\times\\
    &\quad\quad\quad\times\left[\epsilon_3\cdot k_{12}\epsilon_{1}\cdot\epsilon_{2}+\epsilon_{1}\cdot k_{23}\epsilon_{2}\cdot \epsilon_3+\epsilon_{2}\cdot k_{31}\epsilon_3\cdot\epsilon_{1}+\frac{\alpha'}{8}\epsilon_3\cdot k_{12}\epsilon_1\cdot k_{23}\epsilon_2\cdot k_{31}\right]\\
\end{align*}
In which $k_{ij}=k_i-k_j$. So the expectation value is, now dressing with color factors,
\begin{align*}
    \expval{V^{a_1}V^{a_2}V^{a_3}}&=\Tr\qty[\lambda^{a_1}\comm{\lambda^{a_2}}{\lambda^{a_3}}]\frac{{\alpha'}^2\im}{8}\expval{\cnord{{\partial^2 c\partial cc\exp(\im(k_1+k_2+ k_3)\cdot X)}}(w_3)}\times\\
    &\quad\quad\quad\times\left[\epsilon_3\cdot k_{12}\epsilon_{1}\cdot\epsilon_{2}+\epsilon_{1}\cdot k_{23}\epsilon_{2}\cdot \epsilon_3+\epsilon_{2}\cdot k_{31}\epsilon_3\cdot\epsilon_{1}+\frac{\alpha'}{8}\epsilon_3\cdot k_{12}\epsilon_1\cdot k_{23}\epsilon_2\cdot k_{31}\right]\\
\end{align*}
The expectation value on the right hand side cannot depend on the point $w_3$, as there is just a single operator inside it, hence, 
it is just a scalar factor. In fact it's from this term that we get the conservation of momentum delta, 
this is easier to see from the path integral formalism, where we have $\int\Dd{X}\exp(\im \sum k\cdot X)\propto \delta^{\qty(26)}\qty(k_1+k_2+k_3)$. 
We won't derive this is detail as we already have derived that there must be a conservation of momentum delta inside it, and as we already 
argued that this expectation value is a scalar, and by the on-shell conditions $k_1^2=k_2^2=k_3^2=0$ and the conservation of momentum 
no scalar can be build out of the momentums $k_i\cdot k_j=0$, hence, this expectation value is just the conservation of momentum delta 
times a number,
\begin{align*}
    \expval{V^{a_1}V^{a_2}V^{a_3}}&=C\Tr\qty[\lambda^{a_1}\comm{\lambda^{a_2}}{\lambda^{a_3}}]\frac{{\alpha'}^2\im}{8}\delta^{\qty(26)}\qty(k_1+k_2+k_3)\times\\
    &\quad\quad\quad\times\left[\epsilon_3\cdot k_{12}\epsilon_{1}\cdot\epsilon_{2}+\epsilon_{1}\cdot k_{23}\epsilon_{2}\cdot \epsilon_3+\epsilon_{2}\cdot k_{31}\epsilon_3\cdot\epsilon_{1}+\frac{\alpha'}{8}\epsilon_3\cdot k_{12}\epsilon_1\cdot k_{23}\epsilon_2\cdot k_{31}\right]\\
\end{align*}
The three first kinematic factors are the usual Yang-Mills three point interaction, the fourth one is the first stringy contribution, 
as can be seen from the power counting of $\alpha'$.


% Using this result in \cref{threepointcft},
% \begin{align*}
%     \expval{V^{a_1}\qty(w_1;\epsilon_1,k_1)V^{a_2}\qty(w_2;\epsilon_2,k_2)V^{a_3}\qty(w_3;\epsilon_3,k_3)}&=\lambda^{a_1a_2a_3}C\qty(\epsilon_1,k_1;\epsilon_2,k_2;\epsilon_3,k_3)\numberthis\label{threepointfirst}
% \end{align*}
% Where we included a \textit{dressing} factor to the correlator, $\lambda^{a_1a_2a_3}\in\mathbb C$. The only motif why is acceptable to write this expression for an open string is due to all operators being inserted 
% in the boundary, otherwise there could be contributions like $\abs{w_1-\bar w_2}^n$. This might not seem much progress, but, notice, the left hand side is only linear on each of the polarizations, and as 
% these are just usual vector that can go out of the expectation value, the right hand side also is only linear in the 
% polarization vector,
% \begin{align*}
%     \expval{V^{a_1}\qty(w_1;\epsilon_1,k_1)V^{a_2}\qty(w_2;\epsilon_2,k_2)V^{a_3}\qty(w_3;\epsilon_3,k_3)}&=\lambda^{a_1a_2a_3}\epsilon_{1\mu_1}\epsilon_{2\mu_2}\epsilon_{3\mu_3}C^{\mu_1\mu_2\mu_3}\qty(k_1,k_2,k_3)\numberthis\label{threepointsecond}
% \end{align*}
% To get more information about what kind of function $C^{\mu_1\mu_2\mu_3}\qty(k_1,k_2,k_3)$ is, 


% %RETIRADO


% It's clear that this result is independent of both $\lambda,\epsilon$, hence, it imply that there should be 
% a delta function inside $C$ to ensure the \textit{conservation of momentum}, that is, we should replace our 
% definition of $C$ by,
% \[C^{\mu_1\mu_2\mu_3}\qty(k_1,k_2,k_3)\rightarrow \delta^{\qty(26)}\qty(k_1+k_2+k_3)C^{\mu_1\mu_2\mu_3}\qty(k_1,k_2,k_3)\]
% Up to now our expression is,
% \[\expval{V^{a_1}V^{a_2}V^{a_3}}=\lambda^{a_1a_2a_3}\delta^{(26)}\qty(k_1+k_2+k_3)\epsilon_{1\mu_1}\epsilon_{2\mu_2}\epsilon_{3\mu_3}C^{\mu_1\mu_2\mu_3}\qty(k_1,k_2,k_3)\numberthis\label{threepointthird}\]

% %RETIRADO

% To obtain any information further is necessary to use the gauge invariance, in particular, as shown in the last 
% item, $V^a\qty(w;\epsilon+k,k)=V^a\qty(w;\epsilon,k)+\comm{Q_{\textnormal{BRST}}}{\mathcal O}$. As 
% any physical state must be BRST closed, the vacuum itself also needs, also, as $V^a$ commutes with 
% the BRST operator,
% \[\expval{\comm{Q_{\textnormal{BRST}}}{\mathcal O}V^{a_2}V^{a_3}}=0\]
% Thus, the right hand side of \cref{threepointfourth} is also invariant under the gauge transformation, what 
% is equivalent to say,
% \[k_{1\mu_1}\epsilon_{2\mu_2}\epsilon_{3\mu_3}C^{\mu_1\mu_2\mu_3}\qty(k_1,k_2,k_3)=0\numberthis\label{orthogonalitycondition}\]
% With the conservation of momentum imposed, of course. This last expression is our holy graal. It is saying that $\epsilon_{2\mu_2}\epsilon_{3\mu_3}C^{\mu_1\mu_2\mu_3}\qty(k_1,k_2,k_3)$ 
% is orthogonal to $k_{1\mu_1}$, luckily, there are few combinations of vectors build out of $k_1,k_2,k_3,\epsilon_2,\epsilon_3$ that are linear in $\epsilon_2,\epsilon_3$ and are orthogonal, 
% \begin{align*}
%     k_1\cdot k_1&=0\\
%     k_1\cdot k_2&=\frac12\qty(k_1+k_2)^2=\frac12 k_3^2=0\\
%     k_1\cdot k_3&=\frac12\qty(k_1+k_3)^2=\frac12 k_2^2=0\\
%     k_1\cdot\qty(\epsilon_2\qty(k_1\cdot \epsilon_3)-\epsilon_3\qty(k_1\cdot \epsilon_2))&=0
% \end{align*}
% The last one is the most non trivial one. The first one, $k_1$, can be neglected, as it would have to be contracted with $\epsilon_1$, 
% which gives zero. We can determine $\epsilon_{2\mu_2}\epsilon_{3\mu_3}C^{\mu_1\mu_2\mu_3}\qty(k_1,k_2,k_3)$ up to proportionality 
% functions of Lorentz scalars,
% \begin{align*}
%     \epsilon_{2\mu_2}\epsilon_{3\mu_3}C^{\mu_1\mu_2\mu_3}\qty(k_1,k_2,k_3)&=a_1\qty(k_1,\epsilon_2,\epsilon_3)k_2^{\mu_1}+b_1\qty(\epsilon_2,\epsilon_3)k_3^{\mu_1}+c_1\qty(\epsilon_2^{\mu_1}\qty(k_1\cdot \epsilon_3)-\epsilon_3^{\mu_1}\qty(k_1\cdot \epsilon_2))
% \end{align*}

\probitem{}