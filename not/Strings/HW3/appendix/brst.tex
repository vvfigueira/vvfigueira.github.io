\section{BRST}
\subsection{Faddeev-Popov Gauge Fixing}

Our Action functional is,
\begin{align*}
    S_{X}+\lambda\chi&=\frac{1}{4\pi\alpha'}\int\limits_M\dd[2]{\sigma}\sqrt{h}h^{ab}\partial_a X^\mu\partial_b X_\mu+\frac\lambda{4\pi}\int\limits_M\dd[2]{\sigma}\sqrt{h}R+\frac\lambda{2\pi}\int\limits_{\partial M}\dd{s}K\numberthis\label{bosonicaction}
\end{align*}
we would like to define the quantum theory by means of the path integral, that is, we expect that,
\begin{align*}
    Z&\stackrel{?}{=}\int\Dd{X}\Dd{h}\exp\qty(-S_X\qty[X,h]-\lambda\chi)\numberthis\label{try1}
\end{align*}
should give a well defined theory, but, already from \ref{try1} there're several problems that arise, one of them is: \textit{What should be interpreted from the path integral itself? We haven't 
defined any manifold to our metric $h$ and scalar fields $X$ to live in, also, even if we had defined such, the path integral relies on explicit coordinate points, $\Dd{h}=\prod\limits_{\sigma}\dd{h_{ab}\qty(\sigma)}$, 
which are highly dependent on charts.}

This is a valid claim, our way to avoid it is to \textit{define} $\Dd{h}$ to mean: \textit{Sum over all \textbf{allowed} two dimensional Riemannian manifolds, and all 
possible metric structures in these.} Here, \textit{\textbf{allowed}} requires a prescription, which manifolds are or aren't allowed impacts the obtained string theory. Happily, 
every two dimensional manifold has a definite value for the Euler Characteristic $\chi$, hence, we can sort them out by it,
\begin{align*}
    Z&\stackrel{?}{=}\sum\limits_{\qty{M}}\int\limits_{\textnormal{Met}\qty(M)}\Dd{h}\int\Dd{X}\exp\qty(-S_X\qty[X,h]-\lambda\chi)\\
    Z&\stackrel{?}{=}\sum\limits_{\qty{\chi}}\exp\qty(-\lambda \chi)\sum\limits_{\qty{M_\chi}}\int\limits_{\textnormal{Met}\qty(M_\chi)}\Dd{h}\int\Dd{X}\exp\qty(-S_X\qty[X,h])\numberthis\label{try2}
\end{align*}
Where $M$ is to be understood as a two dimensional Riemannian manifold and $M_\chi$ is one with Euler Characteristic $\chi$, 
$\textnormal{Met}\qty(M_\chi)$ is the space of all metrics which can be assigned to $M_\chi$, we have written $\sum\limits_{\qty{M_\chi}}$ in the 
special case of there being more than one manifold with same Euler Characteristic\footnote{As we're interested only in Differentiable Manifolds, more than manifold should read: More than one equivalence class of Differentiable Manifolds.}, also, the functional integral over $X$ should be read as integrating over all 
maps from $M_\chi$ to $\mathbb R^{1,D-1}$. While this is better defined than before, i.e. not coordinate dependent, we still have a few problems, first, 
it's know that \ref{bosonicaction} has a Gauge Group of Diff$\qty(M)\times$Weyl$\qty(M)$, but, in our second try of a definition of the path integral, we're integrating the metrics over Met$\qty(M_\chi)$, 
it's clear that may happen of two elements of Met$\qty(M_\chi)$ be equivalent under a Diff$\qty(M_\chi)\times$Weyl$\qty(M_\chi)$ transformation, to put in more 
clear terms, we're worried if exists $h',h\in\textnormal{Met}\qty(M_\chi)$ such,
\begin{align*}
    h'_{ab}\qty(\sigma'\qty(\sigma))=\exp\qty(2\omega\qty(\sigma))\pdv{ \sigma^c}{\sigma'^a}\pdv{ \sigma^d}{\sigma'^b} h_{cd}\qty( \sigma)
\end{align*}
the existence of those kinds of elements is troublesome, as Diff$\qty(M_\chi)\times$Weyl$\qty(M_\chi)$ is a infinite dimensional group of redundancies, 
this means we're over-counting physical configurations by a infinite amount. The solution is to look for an equivalence class of metrics under this Gauge Group 
action,
\begin{align*}
    \mathscr{M}_\chi=\textnormal{Met}\qty(M_\chi)/\textnormal{Diff}\qty(M_\chi)\times\textnormal{Weyl}\qty(M_\chi)
\end{align*}
the equivalence class is to be understood as\footnote{In all charts.},
\begin{align*}
    h'\sim h\Leftrightarrow h'_{ab}\qty(\sigma'\qty(\sigma))=\exp\qty(2\omega\qty(\sigma))\pdv{ \sigma^c}{\sigma'^a}\pdv{ \sigma^d}{\sigma'^b} h_{cd}\qty( \sigma)
\end{align*}
that is, two metrics are the same representative of an element of $\mathscr M_\chi$ iff they differ by a composition of a Diffeomorphism and Weyl transformation. We'll denote a given composition of a Diffeomorphism followed by a Weyl transformation by $\zeta$,
\begin{align*}
    h'=\zeta\circ h
\end{align*}

Notice that the set of equivalence class of metrics, or, the set of inequivalent Diff$\qty(M_\chi)\times$Weyl$\qty(M_\chi)$ metrics $\mathscr M_\chi$ is highly 
dependent on the topology of $M_\chi$, for example, for $M_\chi\cong\mathbb R^2\cong\mathbb C$, it's trivial, there is just one point in the set $\mathscr M_\chi$, in other words, 
every metric is equivalent, which isn't true for more complex topologies.

Thus, it's possible for us to set up a well defined version of the path integral, just replace Met$\qty(M_\chi)$ by $\mathscr M_\chi$\footnote{By making this procedure, we 
eliminate the redundancies of Diff$\qty(M_\chi)\times$Weyl$\qty(M_\chi)$ with except of a measure zero subset of transformations, known as \textit{conformal killing group} --- CKG ---, so we're still over-counting 
the physical contributions, but, this time by a finite number, this doesn't spoil the well-definiteness of the path integral, but do spoil the normalization. There is ways of correcting this, but we'll no dwell upon.},
\begin{align*}
    Z&=\sum\limits_{\qty{\chi}}\exp\qty(-\lambda g)\sum\limits_{\qty{M_\chi}}\int\limits_{\mathscr M_\chi}\Dd{h}\int\Dd{X}\exp\qty(-S_X\qty[X,h])\numberthis\label{try3}
\end{align*}
where the integration is to be understood as by choosing for each equivalence class in $\mathscr M_\chi$ a representative element in Met$\qty(M_g)$. 
While this is a satisfactory definition for the path integral, it feels a little clunky, we rather have an path integral over all the possible metrics --- in the sense defined before ---, 
well, this is achievable. First, for each equivalence class of $\mathscr M_\chi$ elect one representative element of Met$\qty(M_\chi)$, we'll denote these elements 
as $\hat h\qty(\vb t)$ --- here $\vb t$ is a parametrization of the correspondent equivalence class in $\mathscr M_\chi$, we haven't proved here, and won't, but $\mathscr M_\chi$ is a finite $N$ dimensional manifold, hence, $\vb t$ is a $N$-tuple of real numbers ---, by construction, these representatives are 
inequivalent under Diff$\qty(M_\chi)\times$Weyl$\qty(M_\chi)$, hence,
\begin{align*}
    \zeta_1\circ \hat h\qty(\vb t_1)=\zeta_2\circ\hat h\qty(\vb t_2)\Leftrightarrow \vb t_1=\vb t_2\textnormal{ and }\zeta_1=\zeta_2
\end{align*}
so that every element in Met$\qty(M_g)$ can be written as a unique\footnote{Apart from the measure zero CKG.\label{CKG}} composition of a given $\zeta$ into a given $\hat h\qty(\vb t)$. Now, we rewrite 
the pictorial integral over $\mathscr M_\chi$ is a more formal way, using the parametrization we just described,
\begin{align*}
    Z&=\sum\limits_{\qty{\chi}}\exp\qty(-\lambda g)\sum\limits_{\qty{M_\chi}}\int\limits_{\mathscr M_\chi}\dd[N]{\vb t}\int\Dd{X}\exp\qty(-S_X\qty[X,\hat h\qty(\vb t)])\\
    Z&=\sum\limits_{\qty{\chi}}\exp\qty(-\lambda g)\sum\limits_{\qty{M_\chi}}\int\limits_{\mathscr M_\chi}\dd[N]{\vb t}\int\limits_{\textnormal{Diff}\qty(M_\chi)\times\textnormal{Weyl}\qty(M_\chi)}\Dd{\zeta}\delta\qty(\zeta)\int\Dd{X}\exp\qty(-S_X\qty[X,\hat h\qty(\vb t)])
\end{align*}
in the last line we introduced a one by integrating over the delta functional, as this integral picks only $\zeta=0$, what should be understood as $\zeta=\textnormal{id}$ in the group, 
we can deform a little the integration to,
\begin{align*}
    Z&=\sum\limits_{\qty{\chi}}\exp\qty(-\lambda g)\sum\limits_{\qty{M_\chi}}\int\limits_{\mathscr M_\chi}\dd[N]{\vb t}\int\limits_{\textnormal{Diff}\qty(M_\chi)\times\textnormal{Weyl}\qty(M_\chi)}\Dd{\zeta}\delta\qty(\zeta)\int\Dd{X}\exp\qty(-S_X\qty[X, \zeta\circ\hat h\qty(\vb t)])\numberthis\label{try4}
\end{align*}

This is almost in the form that we would like, notice that we're integrating over the set of representative of the inequivalent metrics, $\dd[N]{\vb t}$, and also over the whole group Diff$\qty(M_\chi)\times$Weyl$\qty(M_\chi)$, $\Dd{\zeta}$, 
by construction, \textbf{every} metric in Met$\qty(M_\chi)$ can be written uniquely\footnote{\ref{CKG}.} as,
\begin{align*}
    h=\zeta_{\vb t}\circ \hat h\qty(\vb t)
\end{align*}
in other words, to integrate over $\dd[N]{\vb t}\Dd{\zeta}$ is to integrate over all metrics of the form $\zeta\circ \hat h\qty(\vb t)$, which is to integrate over 
all metrics $h=\zeta\circ \hat h\qty(\vb t)$ in Met$\qty(M_\chi)$! We cannot yet make this change, due to the presence of an explicit dependence in $\zeta$ at the functional delta. 
We'll eliminate it by means of a change of variable of the functional delta, notice that,
\begin{align*}
    \delta\qty(\hat h\qty(\vb t)-\zeta\circ \hat h\qty(\vb t))
\end{align*}
picks up just the contribution of $\zeta=0$, so it's a good candidate for a change of variables,
\begin{align*}
    \delta\qty(\hat h\qty(\vb t)-\zeta\circ \hat h\qty(\vb t))=\delta\qty(\zeta)\abs{\Det\qty[\fdv{}{\zeta}\qty(\hat h\qty(\vb t)-\zeta\circ\hat h\qty(\vb t))\eval_{\zeta=0}]}^{-1}
\end{align*}

in this way is possible to separate the integral over all metrics $\int\Dd{h}$ into an integration over all inequivalent metrics $\int\Dd{\hat h}$ 
and an integration over all possible Diff$\times$Weyl transformations $\int\Dd{\zeta}$, so that the partition function can be rewrote as,
\begin{align*}
    Z&\stackrel{?}{=}\int\Dd{X}\Dd{\hat h}\Dd{\zeta}\exp\qty(-S_X\qty[X,\zeta\circ h])
\end{align*}
this still has the same problem of before, we're over-integrating the physical configurations, that is, $\hat h$ are the physical configurations, 
but we're integrating also over the whole Diff$\times$Weyl group in $\Dd{\zeta}$. One way of circumventing this problem is introducing by 
hand a Dirac delta to force $\zeta=0$, what also forces we to integrate only over one copy of the physical configurations,
\begin{align*}
    Z&=\int\Dd{X}\Dd{\hat h}\Dd{\zeta}\delta\qty(\zeta)\exp\qty(-S_X\qty[X,\zeta\circ h])
\end{align*}
but this is not the most general way, we could set $\zeta=f\qty(\sigma)$, for a arbitrary function, and this would still give the same theory,
\begin{align*}
    Z&=\int\Dd{X}\Dd{\hat h}\Dd{\zeta}\delta\qty(\zeta-f)\exp\qty(-S_X\qty[X,\zeta\circ h])
\end{align*}
we can go even further and give a function $G\qty(\zeta)$ such that the solution to $G\qty(\zeta)=0$ is only $\zeta=f$, so that we can use the relations between 
Dirac deltas,
\begin{align*}
    \delta\qty(G\qty(\zeta))=\abs{\Det\qty[\fdv{G}{\zeta}]\eval_{\zeta=f}}^{-1}\delta\qty(\zeta-f)
\end{align*}
to obtain,
\begin{align*}
    Z&=\int\Dd{X}\Dd{\hat h}\Dd{\zeta}\delta\qty(\zeta-f)\exp\qty(-S_X\qty[X,\zeta\circ h])\\
    Z&=\int\Dd{X}\Dd{\hat h}\Dd{\zeta}\abs{\Det\qty[\fdv{G}{\zeta}]\eval_{\zeta=f}}\delta\qty(G\qty(\zeta))\exp\qty(-S_X\qty[X,\zeta\circ \hat h])\\
    Z&=\int\Dd{X}\Dd{\hat h}\Dd{\zeta}\abs{\Det\qty[\fdv{G}{\zeta}]\eval_{\zeta=f}}\delta\qty(G\qty(\zeta))\exp\qty(-S_X\qty[X,\zeta\circ \hat h])
\end{align*}

There are some details here, as $\zeta$ is to represent both a Weyl and a Diff, it has to represent both a function $\omega$ and a vector field $\xi$ such 
that,
\begin{align*}
    \zeta \circ \hat h&=\hat h+2\omega \hat h+\pounds_\xi \hat h+\mathcal O\qty(\omega^2,\xi^2,\omega\xi)\\
    \qty[\zeta \circ\hat  h]_{ab}&=\hat h_{ab}+2\omega \hat h_{ab}+2\nabla_{(a}\xi_{b)}+\mathcal O\qty(\omega^2,\xi^2,\omega\xi)
\end{align*}
this means both $\zeta=f$ and $G\qty(\zeta)=0$ are in fact a collection of various equations. In particular, we'll choose
\begin{align*}
    G_{ab}\qty(\zeta)&=\qty[\tilde h]_{ab}-\qty[\zeta\circ \hat h]_{ab}
\end{align*}
for a particular metric $\tilde h$. As $G_{ab}\qty(\zeta)$ is in fact a function of $h=\zeta\circ \hat h$ alone,
\begin{align*}
    G_{ab}\qty(\zeta)&=\qty[\tilde h]_{ab}-\qty[\zeta\circ \hat h]_{ab}=\qty[\tilde h]_{ab}-\qty[h]_{ab}=G_{ab}\qty(h)
\end{align*}
we can rewrite as,
\begin{align*}
    Z&=\int\Dd{X}\Dd{\hat h}\Dd{\zeta}\abs{\Det\qty[\fdv{G_{ab}}{\zeta}]\eval_{\zeta=f}}\delta\qty(G_{ab}\qty(\zeta))\exp\qty(-S_X\qty[X,\zeta\circ \hat h])\\
    Z&=\int\Dd{X}\Dd{\hat h}\Dd{\zeta}\abs{\Det\qty[\fdv{G_{ab}}{\zeta}]\eval_{G_{ab}\qty(h)=0}}\delta\qty(G_{ab}\qty(h))\exp\qty(-S_X\qty[X,h])
\end{align*}

Notice that every term in the integrand depends on $\zeta$ only through $h=\zeta\circ \hat h$, this is what we do want, so that we can 
recombine the integration measure $\int\Dd{\hat h}\Dd{\zeta}=\int\Dd{h}$, the only problem in this procedure is the term,
\begin{align*}
    \Det\qty[\fdv{G_{ab}}{\zeta}]\eval_{G_{ab}\qty(h)=0}
\end{align*}
which is manifestly dependent on $\zeta$, or at least looks like it is. We'll prove it only depends on $\zeta$ through 
$h = \zeta\circ \hat h$. The point is, if
\begin{align*}
    G_{ab}\qty(h)={\tilde h}_{ab}-h_{ab}={\tilde h}_{ab}-\qty[\zeta\circ \hat h]_{ab}=0
\end{align*}
is needed to have a solution, then exists the transformation $\tilde\zeta$, such that,
\begin{align*}
    \tilde\zeta\circ \hat h=\tilde h
\end{align*}
as this transformation is also an element of the gauge \textbf{group}, it certainly has an inverse, $\tilde\zeta^{-1}\circ \tilde \zeta \circ \hat h=\hat h$, thus, 
\begin{align*}
    G_{ab}\qty(h)&=\tilde h-\zeta\circ \hat h\\
    G_{ab}\qty(h)&=\tilde h-\zeta\circ \tilde\zeta^{-1}\circ \tilde \zeta\circ\hat h\\
    G_{ab}\qty(h)&=\tilde h-\zeta\circ \tilde\zeta^{-1}\circ \tilde h\\
    G_{ab}\qty(h)&=\tilde h-\zeta'\circ \tilde h
\end{align*}
where we defined the new gauge transformation $\zeta'=\zeta\circ \tilde \zeta^{-1}$, notice that,
\begin{align*}
    h&=\zeta\circ \hat h\\
    h&=\zeta\circ \tilde\zeta^{-1}\circ \tilde \zeta \circ\hat h\\
    h&=\zeta'\circ \tilde h
\end{align*}