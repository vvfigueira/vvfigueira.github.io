\section{Super Riemann Surfaces and Consequences}
\label{sec:form}

% \subsection{Intuitive Description of SRS}

% As mentioned in the last section, for the action\footnote{We're going to forget about the 
% world-sheet gravitino.} \eqref{action:sst1} to be on-shell invariant under \eqref{susy:pol1}, 
% it's needed to add an auxiliary field in it. We'll try to understand why, and at the same time introduce the 
% superspace formalism. We know that the supersymmetry algebra contains a commutation relation such,
% $$\comm{Q}{Q}\propto P$$
% so that they have to be interpreted as a space-time --- in this case world-sheet --- symmetry (redundancy), 
% instead of a internal symmetry. This is interesting, because we can understand the world-sheet as being 
% the quotient group, $$\mathbb R^2\cong\mathbb C\cong ISO\qty(1,1)/SO(1,1)$$ what implies that we 
% can also understand our space-time (world-sheet) with SUSY --- Super-Space ---, as being the quotient group of the 
% Super Poincaré group with respect to the Lorentz group, 
% $$\textnormal{Super-Space}\cong\mathbb C^{1|1}\cong ISO\qty(1,1|1)/SO(1,1)$$

% Why does this is of relevance? Due to we being able to write a generic element of the group $ISO(1,1|1)$ as,
% $$ISO(1,1|1)\ni g(\sigma,\theta,\omega)=\exp\qty(-\im \sigma_aP^a-\im \theta_A Q^A+\frac\im2\omega_{ab}J^{ab})$$
% if we factor out the Lorentz group, we obtain an expression for the elements of the Super-Space, they're 
% parametrized by two bosonic coordinates $\sigma^a$, and two fermionic coordinates $\theta^A$ --- 
% the fermionic nature is guarantee by the fermionic nature of the SUSY generators $Q^A$ ---. As 
% we're ultimately interested in the complex structure, we switch to $z,\bar z,\theta,\bar\theta$ by 
% the usual substitutions, $z=\sigma^1-\sigma^0$ and similarly for $\theta$. These coordinates we introduced 
% are useful because they allow for a differential representation of the SUSY algebra, which is analogous 
% to the differential representation of the translations which we're accustomed $P\sim L_{-1}\sim \partial_z$. 
% Inspection shows that the right choice is $Q_\theta=\partial_\theta-\theta\partial_z$, with the analogous 
% anti-holomorphic one, $$\comm{Q_\theta}{Q_\theta}=2\partial_\theta\partial_\theta-2\theta\partial_z\partial_\theta-2\partial_\theta\qty(\theta\partial_z)+2\theta\partial_z\qty(\theta\partial_z)=-2\partial_z$$
% what neatly satisfy the correct algebra. The good thing about this kind of differential representation is that 
% possesses a natural action on functions of the Super-Space --- which we'll shown in a bit ---, instead of the mysterious action in \eqref{susy:pol1}. 
% Together with this differential representation of the SUSY generator, it's useful to introduce a 
% \textit{covariant derivative} $D_\theta$, in the sense that it preserves the supersymmetry transformation 
% of the object it's acting on, $$D_\theta=\partial_\theta+\theta\partial_z,\ \ \ \comm{D_\theta}{Q_\theta}=0,\ \ \ \comm{D_\theta}{D_\theta}=2\partial_z$$
% while we'll only be able to show that this is the right choice in the next subsection, there are a few 
% motifs behind this definition. Remember, our main goal here is to obtain a geometric visualization of this 
% supersymmetry, in other words, a geometric visualization of the Super Conformal group. As we know, a conformal 
% transformation can be defined as being a coordinate change such that $\partial_z$ is changed to a 
% multiple of itself, as we already argued here, in the Super Conformal group, we have not only $z$, 
% but also $\theta$, so we need a differential operator such: (i) It commutes with the SUSY generator. 
% (ii) A Super Conformal transformation can be defined as a coordinate change $z,\theta\rightarrow z'(z,\bar z,\theta,\bar\theta),\theta'(z,\bar z,\theta,\bar\theta)$ 
% that preserves this differential operator. Notice that $\partial_z$ is consistent with condition (i), but, 
% if we try to impose condition (ii) we gain only the usual bosonic conformal transformations. Our claim is, 
% the most general differential operator that satisfy both conditions is a multiple of $D_\theta$. We'll 
% not prove here, but under such a super conformal transformation this differential operator transforms as 
% $D_\theta=\qty(D_\theta\theta')D_{\theta'}$. 

% The analogy proposes us to define Super Fields. 
% A Super Field is a function of the Super-Space $\mathbb A\qty(z,\bar z,\theta,\bar\theta)$, it's said to 
% have weights $(h,\tilde h)$ if it changes as, $$\qty(D_\theta\theta')^{2h}\qty(D_{\bar\theta}{\bar\theta}')^{2\tilde h}\mathbb A'\qty(z',\bar z',\theta',\bar\theta')=\mathbb A\qty(z,\bar z,\theta,\bar\theta)$$
% a super field turns out to be a useful construction due to the natural action of a SUSY transformation,
% $$\comm{\mathbb A\qty(z,\theta)}{Q}=-\im Q_\theta\mathbb A\qty(z,\theta)$$
% and the natural transformation of the measure,
% $$\dd[2]{z'}\dd[2]{\theta'}=\dd[2]{z}\dd[2]{\theta}D_\theta\theta'D_{\bar\theta}{\bar\theta}'$$
% which allows for a easy construction of an action, as the measure transforms as a weight $(-\frac12,-\frac12)$ 
% super field, we just need to integrate a $(\frac12,\frac12)$ super field, and as each covariant 
% derivative transforms as $(\frac12,0)$, a natural candidate is the derivative of a $(0,0)$ super field $\mathbb X^\mu$. 
% Due to the fermionic nature of the $\theta$, this can be expanded as, using a little of foresight,
% $$\mathbb X^\mu\qty(z,\bar z,\theta,\bar\theta)=X^\mu\qty(z,\bar z)+\im\theta\psi^\mu\qty(z)+\im\bar\theta{\tilde\psi}^\mu\qty(\bar z)+\bar\theta\theta F^\mu\qty(z,\bar z)$$
% here we see already our familiar fields $X^\mu,\psi^\mu$, and the presence of an additional field $F^\mu$, 
% which transforms as $(\frac12,\frac12)$, and also non trivially under the SUSY, as hinted by $Q_\theta\mathbb X^\mu$, 
% this field is necessary to ensure the off-shell SUSY invariance, and it modifies the \eqref{susy:pol1}. 
% As we mentioned before, a super conformal invariant action can be build as,
% \begin{align}
%     S&=\frac{1}{4\pi}\int\limits_\Sigma\dd[2]{z}\dd[2]{\theta}D_{\bar\theta}\mathbb X^\mu D_\theta\mathbb X_\mu\label{action:sst2}
% \end{align}
% where the fermionic integration will of course only extract the term proportional to $\bar\theta\theta$, this 
% computation is straightforward, giving,
% \begin{align*}
%     S&=\frac{1}{4\pi}\int\limits_\Sigma\dd[2]{z}\qty(\partial_zX^\mu\partial_{\bar z}X^\mu+\psi^\mu\partial_{\bar z}\psi_\mu+{\tilde\psi}^\mu\partial_z{\tilde\psi}_\mu+F^\mu F_\mu)
% \end{align*}
% exactly our starting action! With of course the auxiliary field $F^\mu$, which has the trivial equation of motion $F^\mu=0$. 
% This implies that the action \eqref{action:sst1} is just \eqref{action:sst2} with the auxiliary field integrated out.

\subsection{Definition of SRS}

As we foretold, to obtain a geometrical description of the superconformal structure we need a geometrical 
description of the generators $T(z),G(z)$, in the bosonic case we have such a description as differential 
operators such $T(z)\rightarrow L_n\sim z^{n+1}\partial_z$. But, to do so for $G(z)$ --- which is 
grassmann odd --- would require to have a grassmann odd coordinate. This is the start of the history of 
SRS. They're a special type of \textbf{Complex supermanifolds}, so, in order to understand them, 
we need first to understand what is a complex supermanifold.
\begin{definition}
    A \textbf{complex supermanifold} $\Sigma$ of dimension $1|1$ is a space locally isomorphic 
    to $\mathbb C^{1|1}$, that is, it's locally covered by coordinate charts $z|\theta:U\subset\Sigma\rightarrow\mathbb C^{1|1}$ 
    such that $z$ is a complex grassmann even coordinate, and $\theta$ is a complex grassmann odd coordinate.
\end{definition}
Of course we have to say something about the transition functions between overlapping charts. Due to the 
complex nature, we really just have two kinds of choice, either we impose that all transition 
functions are just continuous, or that they are holomorphic. It's clear that for our context 
the latter is way more useful, so we'll choose it. Maybe it's needed clarification on what we mean by 
a transition function $(z_i|\theta_i)^{-1}\circ z_j|\theta_j:z_j|\theta_j\qty(U_j)\subset\mathbb C^{1|1}\rightarrow \mathbb C^{1|1}$ 
to be holomorphic.
\begin{definition}
    A function $f(z,\theta):U\subset\mathbb C^{1|1}\rightarrow\mathbb C^{1|1}$ is said to be holomorphic iff 
    the expansion in powers of $\theta$ --- which is finite due to the oddness ---, $$f\qty(z,\theta)=f_0\qty(z)+\theta f_1\qty(z)$$ 
    has both $f_0,f_1: U\cap\mathbb C\subset\mathbb C\rightarrow\mathbb C^{1|1}$ holomorphic.
\end{definition}
Another definition which will be of use for us is the notion of non-zero objects in a supermanifold. 
\begin{definition}
    Any \textbf{non-zero} object --- function, vector, ... --- is one such that is non-zero up to grassmann odd 
    variables.
\end{definition}
With this we have all the ingredients to state the definition of a SRS, the point is: we want for the SRS to have a notion 
of superconformal transformations --- which is compatible with the super Virasoro algebra from SST --- as the RS had. The good thing about RS is, they don't need actually more structure than 
being a complex manifold, as any transformation $z\rightarrow f(z)$ is indeed a conformal transformation. Two 
distinct characteristic of conformal transformations in SR are: they preserve the tangent space, that is 
$\partial_z\rightarrow\partial_z z'\partial_{z'}$, and they preserve angles. These properties cannot be 
used immediately to give a prescription of super conformal transformation. If we try to set them as being 
all the set of transformations which preserve the tangent space of a complex supermanifold we would 
get actually an algebra that is way bigger than the super Virasoro algebra, and to define them as being 
\textit{angle preserving} would require to introduce a metric, which is too much of additional structure.

The idea is to look at what is really essential for us to have, if we're hoping to get any non-trivial theory 
with super conformal transformations, we need some kind of derivative operator which transforms covariantly under 
such. This is not the end, as we need to guarantee --- in a coordinate invariant way --- that covariant transformations 
of this derivative operator --- which we'll call super conformal transformations --- do in fact \textit{mix} the 
grassmann odd and even coordinates. While seem kind of arbitrary, this requirement already excludes lot's of 
possible choices as $\partial_z$ and $\partial_\theta$. Let's sum up what we concluded, to set up a SRS we 
need a notion of super conformal transformation, which we concluded is equivalent of choosing a 
covariant derivative --- what can be seen as choosing a subspace of the tangent space $T\Sigma$ ---, but no just 
any covariant derivative, one such that guarantees that we're in fact \textit{coupling} both 
coordinates. There are just two possible choices for the dimension of the subspace we're choosing, $0|1$ or $1|0$. 
We argue that it's impossible to choose a $1|0$ dimensional subspace of $T\Sigma$ and guarantee that in some sense 
the covariance under it \textit{couples} both coordinates, this is so because no operation done with only grassmann 
even variables can return a grassmann odd variable. But, the opposite is true! It's possible to compose two 
grassmann odd variables to give a grassmann even one. Hence, we conclude that to define a notion 
of super conformal transformation is needed to choose a $0|1$ dimensional subspace of $T\Sigma$, what about the 
condition to ensure \textit{mixing} of coordinates? Let's save for the definition\cite{witten:revisited,witten:moduli,witten:integration},
\begin{definition}
    A \textbf{Super Riemann Surface} $\Sigma$ is a complex supermanifold of dimension $1|1$ that 
    possesses a distinct subbundle/subspace $\mathcal D\subset T\Sigma$ of dimension $0|1$ that satisfy the 
    completely non-integrability condition, $$\forall D\in\mathcal D: D\textnormal{ is non-zero}\Rightarrow \comm{D}{D}\notin \mathcal D$$ 
    $\mathcal D$ is said to be a complex structure of this SRS.\label{definition:srs}
\end{definition}
Here we see the beauty of this definition, as we discussed, we have the distinguished structure as a $0|1$ dimensional 
subspace of $T\Sigma$, and the condition that it must satisfy is the so called \textit{completely non-integrability}, 
which is dependent on the graded Lie Bracket operation defined for vector fields $\comm{\cdot}{\cdot}$, what is 
important here is that $\comm{D}{D}$ is only non-zero if $D$ is odd, and in this case, by statistics, it returns 
a even vector field. Here the hypothesis of $D$ being non-zero is crucial, otherwise, $\comm{D}{D}$ could be 
zero and so, despite being even, could still belong to $\mathcal D$. Hence, the completely non-integrability 
condition provides a notion of \textit{coupling} both coordinates without having to evoke new structure in 
our manifold. It's instructive to see what kind of elements belongs to this subbundle,
\begin{lemma}
    Given a SRS with distinct subbundle $\mathcal D$, it's always possible to construct a 
    coordinate system $z|\theta$, such that locally, any element $D\in\mathcal D$ is of the 
    form, \[D_\theta\coloneq D_{U_{z|\theta}}=\partial_\theta+\theta\partial_z,\ \ \ D_\theta^2\coloneq\frac12\comm{D_\theta}{D_\theta}=\partial_z\numberthis\label{covariantderivative}\] we call 
    this coordinate system a superconformal one.
\end{lemma}
\begin{proof}
    In a given coordinate system/chart $z|\theta$ we always can decompose any element of the tangent space as,
    $$D_{U_{z|\theta}}=a\qty(z,\theta)\partial_\theta+b\qty(z,\theta)\partial_z$$ the condition of $D_{U_{z|\theta}}$ being 
    non-zero, as stated in \cref{definition:srs}, is equivalent to $a\qty(z,\theta)\neq 0$, hence, it's possible to scale 
    $\theta\rightarrow a\theta$, doing this and also expanding, due to the grassmann odd character, 
    $b\qty(z,\theta)=b_0\qty(z)+\theta b_1\qty(z)$, 
    $$D_{U_{z|\theta}}=\partial_\theta+(b_0+\theta b_1)\partial_z$$ which now we compute the graded Lie Bracket,
    \begin{align*}
        \comm{D_{U_{z|\theta}}}{D_{U_{z|\theta}}}&=\comm{\partial_\theta+(b_0+\theta b_1)\partial_z}{\partial_\theta+(b_0+\theta b_1)\partial_z}\\
        \comm{D_{U_{z|\theta}}}{D_{U_{z|\theta}}}&=2\comm{\partial_\theta}{(b_0+\theta b_1)\partial_z}+\comm{(b_0+\theta b_1)\partial_z}{(b_0+\theta b_1)\partial_z}\\
        \comm{D_{U_{z|\theta}}}{D_{U_{z|\theta}}}&=2\partial_\theta(b_0+\theta b_1)\partial_z+2\qty(b_0+\theta b_1)\comm{\partial_\theta}{\partial_z}+2(b_0+\theta b_1)\qty(\partial_z(b_0+\theta b_1)\partial_z+(b_0+\theta b_1)\partial^2_z)\\
        \comm{D_{U_{z|\theta}}}{D_{U_{z|\theta}}}&=2b_1\partial_z+2\theta b_1\partial_z b_0\partial_z-2\theta b_0\partial_z b_1\partial_z\numberthis\label{temp:eq1}
    \end{align*}
    so, the only way $\comm{D_{U_{z|\theta}}}{D_{U_{z|\theta}}}\in\mathcal D$ could possibly be true is if the above expression is identically zero, which is 
    only possible if $b_1=0$. Thus, the completely non-integrability condition gives the requirement $b_1\neq 0$, hence, we can perform a 
    further change of coordinates $\theta\rightarrow -b_1^{-1}b_0+\theta$, $z\rightarrow b_1 z$,
    $$D_{\theta}\coloneq D_{U_{z|\theta}}=\partial_\theta+\theta \partial_z$$ And lastly, 
    setting $b_1=1, b_0=0$ in \cref{temp:eq1} we get, $D^2_\theta = \partial_z$.
\end{proof}
The existence of such structure is what settles apart a SRS from a generic complex supermanifold. 
To grasp a better understanding of this non-integrability condition we have to take a look at the 
second equation in \eqref{covariantderivative}, for a usual coordinate basis of the tangent space 
we always have $\comm{\partial_I}{\partial_J}=0$, which kind of induces a splitting of the 
tangent space as $\mathbb C^{1|0}\times\cdots\times\mathbb C^{0|1}\times\cdots$, but of course, 
as this is a coordinate basis, this splitting is not some kind of inner property of the manifold, 
it's a coordinate dependent gimmick. But, with our definition being coordinate independent, 
we're saying that existis a global splitting of the tangent space $T\Sigma\cong \mathbb C^{1|0}\times \mathbb C^{0|1}\cong T\Sigma/\mathcal D\times \mathcal D\cong \mathcal D ^2\times\mathcal D$.

We will now pursue 
why does this structure is able to reconstruct the off-shell super Virasoro algebra in SST.

\subsection{Superconformality in SRS}

As we foretold, having a distinct subbundle $\mathcal D$, is possible for us to define super conformal 
coordinate changes. The idea is, vector fields naturally introduce coordinate changes by integral curves, 
hence,
\begin{definition}
    A vector field $W\in T\Sigma$ is said to generate a super conformal coordinate transformation on a 
    SRS $\Sigma$, if it preserves the subbundle $\mathcal D$, that is, 
    $$W\in T\Sigma \textnormal{ generates superconformal transformation}\Leftrightarrow\forall D\in\mathcal D:\, \comm{W}{D}\in\mathcal D.$$
\end{definition}
As good as having a formal definition may be, our true interest here is to obtain these transformations in a given basis, 
in particular, a superconformal basis,
\begin{lemma}
    The set of all vector fields that generate superconformal transformation can be decomposed in a basis, 
    with a superconformal coordinate system, of one even and one odd vector fields such, 
    \[G_f=f\qty(z)\qty(\partial_\theta-\theta\partial_z),\ \ \ L_g=g(z)\partial_z+\frac12 g'\qty(z)\theta\partial_\theta\numberthis\label{nsgenerators}\]
\end{lemma}
\begin{proof}
We will compute it using $D_\theta$ and a decomposition of $W=a\partial_\theta+b\partial_z$
\begin{align*}
    \comm{W}{D_\theta}&=\comm{a\partial_\theta+b\partial_z}{D_\theta}=a\comm{\partial_\theta}{D_\theta}\mp D_\theta a\partial_\theta+b\comm{\partial_z}{D_\theta}\mp D_\theta b\partial_z\\
    \comm{W}{D_\theta}&=a\partial_z\mp D_\theta a\partial_\theta\mp D_\theta b\partial_z=\mp D_\theta a\partial_\theta\mp \qty(D_\theta b\mp a)\partial_z
\end{align*}
Where the signs reefer to $W$ being even, top sign, or odd, bottom sign. To say that $\comm{W}{D_\theta}\in\mathcal D$ is the same 
to say that $\comm{W}{D_\theta}\propto D_\theta$, hence, the condition is,
\begin{align*}
    \mp \qty(D_\theta b\mp a)&=\mp D_\theta a\theta\\
    D_\theta b&= D_\theta a\theta\pm a
\end{align*}
In this form it may seem difficult to solve it, but, we will use our virtue of foresight to propose an ansatz, $b=-a\theta$,
\begin{align*}
    D_\theta (-a\theta)&= D_\theta a\theta\pm a\Rightarrow -D_\theta a\theta\pm a= D_\theta a\theta\pm a\Rightarrow D_\theta a\theta=0\Rightarrow \partial_\theta a\theta=0\Rightarrow \begin{cases}a&=a\qty(z)\\b&=-a\qty(z)\theta\end{cases}
\end{align*}
With this ansatz we got exactly an solution, that is, one family of odd vector fields that generate superconformal transformations are $G_f=f\qty(z)\qty(\partial_\theta-\theta\partial_z)$. 
But, this is not the end, as we have two dimensions, one odd and one even, we know that there is one more solution to this equation,
\begin{align*}
    D_\theta b&=D_\theta a\theta\pm a\\
    D^2_\theta b&=D^2_\theta a\theta\pm D_\theta a\pm D_\theta a\\
    \partial_z b&=\partial _z a\theta \pm2\partial_\theta a\pm2\theta\partial_z a=-\partial_za\theta\pm2\partial_\theta a
\end{align*}
it's clear that the condition $\partial_\theta a=0$ will return our already found solution, hence, we try to 
solve this equation for $\partial_za\theta=0\Rightarrow \partial_z b=\pm 2\partial_\theta a$, substituting 
back in the original equation,
\begin{align*}
    \partial_\theta b+\theta\partial_z b&=\partial_\theta a\theta\pm a\\
    \partial_\theta b\pm2\theta\partial_\theta a&=\partial_\theta a\theta\pm a\partial_\theta \theta\\
    \partial_\theta b+2\partial_\theta a\theta&=\partial_\theta a\theta\pm a\partial_\theta \theta\\
    \partial_\theta b&=-\partial_\theta a\theta\pm a\partial_\theta \theta=-\partial_\theta\qty(a\theta)
\end{align*}
As we already solve for $b=-a\theta$, the only other possible solution that can be obtained from here is 
$\partial_\theta b=\partial_\theta\qty(a\theta)=0\Rightarrow a=c\qty(z)\theta,\ b=b\qty(z)$, and the last consistency 
condition gives,
\begin{align*}
    \partial_z b&=\pm 2\partial_\theta a\Rightarrow\partial_z b= 2c\qty(z)\Rightarrow \begin{cases}a &= \frac12 \partial_z b\qty(z)\theta\\b &= b\qty(z)\end{cases} 
\end{align*}
which give to us the second and last linear independent solution, a family of even vector fields that generate 
superconformal transformations $L_g=g\qty(z)\partial_z+\frac12g'\qty(z)\theta\partial_\theta$.
\end{proof}
It's no coincidence the naming we utilized in \cref{nsgenerators}, as those two are the generators of the 
superconformal transformations, they should also be related to the SST super Virasoro generators $L_n,G_r$, 
and actually this is true, they indeed furnish a differential representation of the super Virasoro,
\begin{lemma}
    The vector field basis, \cref{nsgenerators}, of superconformal transformations for $f\qty(z)=z^{r+\frac12},g\qty(z)=-z^{n+1}$, 
    \[L_n=-z^{n+1}\partial_z-\frac{n+1}{2}z^n\theta\partial_\theta,\ \ \ G_r=z^{r+\frac12}\qty(\partial_\theta-\theta\partial_z),\ \ \ n\in\mathbb Z,\ r\in\mathbb Z+\frac12\]
    furnishes a representation of the super Virasoro algebra for the NS sector,
    \begin{align*}
        \comm{L_m}{L_n}&=\qty(m-n)L_{m+n}\\
        \comm{G_r}{G_s}&=2L_{r+s}\\
        \comm{L_m}{G_r}&=\qty(\frac m2-r)G_{m+r}
    \end{align*}
\end{lemma}
\begin{proof}
    The computation is very straightforward, we start with $L-L$,
    \begin{align*}
        \comm{L_m}{L_n}&=\comm{z^{m+1}\partial_z+\frac{m+1}{2}z^m\theta\partial_\theta}{z^{n+1}\partial_z+\frac{n+1}{2}z^n\theta\partial_\theta}\\
        \comm{L_m}{L_n}&=\comm{z^{m+1}\partial_z}{z^{n+1}\partial_z}+\comm{z^{m+1}\partial_z}{\frac{n+1}{2}z^n\theta\partial_\theta}+\comm{\frac{m+1}{2}z^m\theta\partial_\theta}{z^{n+1}\partial_z}\\
        \comm{L_m}{L_n}&=z^{m+1}\partial_zz^{n+1}\partial_z-z^{n+1}\partial_zz^{m+1}\partial_z+\frac{n+1}{2}z^{m+1}\theta\partial_zz^n\partial_\theta-\frac{m+1}{2}z^{n+1}\theta\partial_zz^m\partial_\theta\\
        \comm{L_m}{L_n}&=\qty(n+1)z^{m+n+1}\partial_z-\qty(m+1)z^{n+m+1}\partial_z+\frac{\qty(n+1)n}{2}z^{m+n}\theta\partial_\theta-\frac{\qty(m+1)m}{2}z^{n+m}\theta\partial_\theta\\
        \comm{L_m}{L_n}&=-\qty(m-n)z^{m+n+1}\partial_z+\qty(n^2+n-m^2-m)\frac{z^{n+m}}{2}\theta\partial_\theta\\
        \comm{L_m}{L_n}&=-\qty(m-n)z^{m+n+1}\partial_z-\qty(m-n)\qty(n+m+1)\frac{z^{n+m}}{2}\theta\partial_\theta=\qty(m-n)L_{m+n}
    \end{align*}
    now $G-G$,
    \begin{align*}
        \comm{G_r}{G_s}&=\comm{z^{r+\frac12}\qty(\partial_\theta-\theta\partial_z)}{z^{s+\frac12}\qty(\partial_\theta-\theta\partial_z)}\\
        \comm{G_r}{G_s}&=-\qty(z^{r+\frac12}\theta\partial_z z^{s+\frac12}+z^{s+\frac12}\theta\partial_z z^{r+\frac12})\qty(\partial_\theta-\theta\partial_z)+z^{r+s+1}\comm{\qty(\partial_\theta-\theta\partial_z)}{\qty(\partial_\theta-\theta\partial_z)}\\
        \comm{G_r}{G_s}&=-\qty(s+\frac12+r+\frac12)z^{r+s}\theta\partial_\theta-2z^{r+s+1}\partial_z=2\qty(-z^{r+s+1}\partial_z-\frac{r+s+1}{2}z^{r+s}\theta\partial_\theta)\\
        \comm{G_r}{G_s}&=2L_{r+s}
    \end{align*}
    and lastly $L-G$,
    \begin{align*}
        \comm{L_m}{G_r}&=
    \end{align*}
\end{proof}