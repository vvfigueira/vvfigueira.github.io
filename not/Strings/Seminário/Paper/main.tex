\documentclass[a4paper,11pt]{article}

\pdfoutput=1

\usepackage{jheppub}
\usepackage[T1]{fontenc}

\usepackage[english]{babel}
\usepackage{csquotes}
\usepackage{amsmath}
\usepackage{amsthm}
\usepackage{amssymb}
\usepackage{bbm}
\usepackage{graphics}
\usepackage{mathtools}
\usepackage{physics}
\usepackage{enumitem}
\usepackage{slashed}
\usepackage{tensor}
\usepackage{tensor}
\usepackage[english]{cleveref}
\usepackage{stackengine}

\newtheorem{definition}{Definition}[section]
\newtheorem{lemma}{Lemma}[section]

\renewcommand\qedsymbol{$\blacksquare$}

\AtBeginDocument{\renewcommand*{\hbar}{{\mkern-1mu\mathchar'26\mkern-8mu\textnormal{h}}}}
\AtBeginDocument{\newcommand{\e}{\textnormal{e}}}
\AtBeginDocument{\newcommand{\im}{\textnormal{i}}}
\AtBeginDocument{\newcommand{\luz}{\textnormal{c}}}
\AtBeginDocument{\newcommand{\grav}{\textnormal{G}}}
\AtBeginDocument{\newcommand{\kb}{{\textnormal{k}_{\textnormal{B}}}}}
\newcommand{\Dd}[1]{\mathcal D #1}
\newcommand{\Det}[1]{\textup{Det} #1}
\newcommand{\sgn}[1]{\mbox{sgn}\qty(#1)}
\newcommand{\cqd}{\hfill$\blacksquare$}
\newcommand{\dbar}{\mbox{\dj}}

\newcommand\numberthis{\addtocounter{equation}{1}\tag{\theequation}}

\title{Super Riemann Surfaces}

\author{Vicente Viater Figueira}
\affiliation{IFUSP}
\emailAdd{vfigueira@usp.br}

\abstract{This work provides a brief description of Super Riemann Surfaces (SRS). 
Starting with the basic definitions of what is a complex supermanifold, we try to motivate SRS as 
being a complex supermanifold with additional structure to allow for a notion of superconformal transformations, 
those are shown to match the same notion of superconformal transformation --- super Virasoro algebra --- in 
superstring theory (SST). We try to work out the peculiarities with the R sector, and with the introduction 
of both NS and R punctures, it's concluded that the R sector is associated to degeneracies with 
the complex structure of the SRS.}

\begin{document}

\maketitle
\flushbottom

\section{Introdução}

O Espaço-Tempo (A)dS é definido como, \[\mp\qty(x^{-1})^2-\qty(x^0)^2+\qty(x^1)^2+\qty(x^2)^2+\qty(x^3)^2=\mp L^2\] 
Onde o sinal de cima é para AdS, e o sinal de baixo para dS. A álgebra de isometria no embedding 5 dimensional é: 
\begin{align*}
    \comm{J^{IK}}{J^{LM}}&=-4\im\eta^{[I|[L}J^{M]|K]}\\
    \comm{J^{IK}}{J^{LM}}&=-\im\eta^{IL}J^{MK}+\im\eta^{IM}J^{LK}+\im\eta^{KL}J^{MI}-\im\eta^{KM}J^{LI}
\end{align*}
Da qual podemos interpretar como geradores de translação $J^{-1\alpha}$,
\begin{align*}
    \comm{J^{-1\alpha}}{J^{-1\beta}}&=-\im\eta^{-1-1}J^{\beta\alpha}+\im\eta^{-1\beta}J^{-1\alpha}+\im\eta^{\alpha-1}J^{\beta-1}-\im\eta^{\alpha\beta}J^{-1-1}\\
    \comm{J^{-1\alpha}}{J^{-1\beta}}&=\mp\im J^{\alpha\beta}
\end{align*}
As outras relações de comutação são,
\begin{align*}
    \comm{J^{\alpha\beta}}{J^{-1\mu}}&=-\im\eta^{\alpha-1}J^{\mu\beta}+\im\eta^{\alpha\mu}J^{-1\beta}+\im\eta^{\beta-1}J^{\mu\alpha}-\im\eta^{\beta\mu}J^{-1\alpha}\\
    \comm{J^{\alpha\beta}}{J^{-1\mu}}&=\im\eta^{\alpha\mu}J^{-1\beta}-\im\eta^{\beta\mu}J^{-1\alpha}\\
    \comm{J^{\alpha\beta}}{J^{-1\mu}}&=-2\im J^{-1[\alpha}\eta^{\beta]\mu}
\end{align*}
E,
\begin{align*}
    \comm{J^{\alpha\beta}}{J^{\mu\nu}}&=-4\im\eta^{[\alpha|[\mu}J^{\nu]|\beta]}
\end{align*}
Como os geradores de translação necessitam ter dimensão, $P^\alpha=\frac1LJ^{-1\alpha}$. A álgebra completa é,
\[\comm{J^{\alpha\beta}}{J^{\mu\nu}}=-4\im\eta^{[\alpha|[\mu}J^{\nu]|\beta]},\ \ \ \comm{J^{\alpha\beta}}{P^\mu}=-2\im P^{[\alpha}\eta^{\beta]\mu},\ \ \ \comm{P^{\alpha}}{P^{\beta}}=\mp\frac{\im}{L^2} J^{\alpha\beta}\]
O bilinear mais geral para essa álgebra é,
\[\expval{J_{\alpha\beta},J_{\mu\nu}}=\pm2\lambda\eta_{\alpha[\mu}\eta_{\nu]\beta},\ \ \ \expval{J_{\alpha\beta},P_\mu}=0,\ \ \ \expval{P_\alpha,P_\mu}=\frac{\lambda}{L^2}\eta_{\alpha\mu}\]
A ação de Einstein-Hilbert com termo de constante cosmológica é,
\begin{align*}
    S_{\textnormal{EH}}&=\frac{1}{2\kappa}\int\limits_M\star\vb{R}_{\alpha\beta}\wedge\vb e^\alpha\wedge\vb e^\beta-\frac{\Lambda}{\kappa4!}\int\limits_M\epsilon_{\alpha\beta\mu\nu}\vb e^\alpha\wedge\vb e^\beta\wedge\vb e^\mu\wedge\vb e^\nu\\
    S_{\textnormal{EH}}&=\frac{1}{2\kappa}\eta_{\mu\alpha}\eta_{\beta\nu}\int\limits_M\star\vb{R}^{\mu\nu}\wedge\vb e^\alpha\wedge\vb e^\beta\pm\frac{3\cdot 2}{\kappa L^24!}\int\limits_M\vb e^\alpha\wedge\vb e^\beta\wedge\star\qty(\vb e_\alpha\wedge\vb e_\beta)\\
    S_{\textnormal{EH}}&=\frac{\pm}{4\lambda\kappa}\pm2\lambda\eta_{\mu[\alpha}\eta_{\beta]\nu}\int\limits_M\star\vb{R}^{\mu\nu}\wedge\vb e^\alpha\wedge\vb e^\beta\pm\frac{3\cdot 2}{\lambda\kappa L^24!}\lambda\eta_{\mu[\alpha}\eta_{\beta]\nu}\int\limits_M\vb e^\mu\wedge\vb e^\nu\wedge\star\qty(\vb e^\alpha\wedge\vb e^\beta)\\
    S_{\textnormal{EH}}&=\frac{\pm}{4\lambda\kappa}\expval{J_{\mu\nu},J_{\alpha\beta}}\int\limits_M\star\vb{R}^{\mu\nu}\wedge\vb e^\alpha\wedge\vb e^\beta+\frac{3}{\lambda\kappa L^24!}\expval{J_{\mu\nu},J_{\alpha\beta}}\int\limits_M\vb e^\mu\wedge\vb e^\nu\wedge\star\qty(\vb e^\alpha\wedge\vb e^\beta)\\
    S_{\textnormal{EH}}&=\frac{\pm}{4\lambda\kappa}\pm\im L ^2\expval{J_{\mu\nu},\comm{P_\alpha}{P_\beta}}\int\limits_M\star\vb{R}^{\mu\nu}\wedge\vb e^\alpha\wedge\vb e^\beta+\frac{3\qty(\pm)^2\im^2L^4}{\lambda\kappa L^24!}\expval{\comm{P_\mu}{P_\nu},\comm{P_\alpha}{P_\beta}}\int\limits_M\vb e^\mu\wedge\vb e^\nu\wedge\star\qty(\vb e^\alpha\wedge\vb e^\beta)\\
    S_{\textnormal{EH}}&=-\frac{L^2}{4\lambda\kappa}\expval{\im J_{\mu\nu},\comm{\im P_\alpha}{\im P_\beta}}\int\limits_M\star\vb{R}^{\mu\nu}\wedge\vb e^\alpha\wedge\vb e^\beta-\frac{L^2}{8\lambda\kappa}\expval{\comm{\im P_\mu}{\im P_\nu},\comm{\im P_\alpha}{\im P_\beta}}\int\limits_M\vb e^\mu\wedge\vb e^\nu\wedge\star\qty(\vb e^\alpha\wedge\vb e^\beta)\\
    S_{\textnormal{EH}}&=-\frac{L^2}{2\lambda\kappa}\int\limits_M\expval{\star\vb{R}\ \wedgecomma\ \wedgecomm{\vb e}{\vb e}}-\frac{L^2}{8\lambda\kappa}\int\limits_M\expval{\wedgecomm{\vb e}{\vb e}\ \wedgecomma \star\wedgecomm{\vb e}{\vb e}}
\end{align*}
Com $\vb R=\frac12\im J_{\mu\nu}\vb R^{\mu\nu}$ e $\vb e=\im P_\mu\vb e^\mu$. Note,
\begin{align*}
    \vb R&=\frac12\im\vb R_{\alpha\beta}J^{\alpha\beta}=\frac12\vb d\boldsymbol\omega_{\alpha\beta}\im J^{\alpha\beta}+\frac12\tensor{{\boldsymbol\omega}}{_\alpha^\rho}\wedge\tensor{{\boldsymbol\omega}}{_\rho_\beta}\im J^{\alpha\beta}=\vb d\boldsymbol\omega+\frac12\tensor{{\boldsymbol\omega}}{_\mu_\nu}\wedge\tensor{{\boldsymbol\omega}}{_\rho_\sigma}\eta^{\rho\nu}\im J^{\mu\sigma}=\vb d\boldsymbol\omega+\frac18\tensor{{\boldsymbol\omega}}{_\mu_\nu}\wedge\tensor{{\boldsymbol\omega}}{_\rho_\sigma}4\im \eta^{[\rho|[\nu}J^{\mu]|\sigma]}\\
    \vb R&=\vb d\boldsymbol\omega-\frac18\tensor{{\boldsymbol\omega}}{_\mu_\nu}\wedge\tensor{{\boldsymbol\omega}}{_\rho_\sigma}\comm{J^{\rho\sigma}}{J^{\nu\mu}}=\vb d\boldsymbol\omega+\frac12\tensor{{\boldsymbol\omega}}{_\mu_\nu}\wedge\tensor{{\boldsymbol\omega}}{_\rho_\sigma}\comm{\frac\im 2J^{\mu\nu}}{\frac\im2J^{\rho\sigma}}=\vb d\boldsymbol\omega+\frac12\wedgecomm{\boldsymbol\omega}{\boldsymbol\omega}
\end{align*}
Seja então,
\begin{align*}
    \vb F&=\vb d\qty(\boldsymbol \omega+\vb e)+\frac12\wedgecomm{\boldsymbol \omega+\vb e}{\boldsymbol \omega+\vb e}\\
    \vb F&=\vb d\boldsymbol \omega+\frac12\wedgecomm{\boldsymbol \omega}{\boldsymbol \omega}+\frac12\wedgecomm{\vb e}{\vb e}+\vb d\vb e+\wedgecomm{\boldsymbol \omega}{\vb e}\\
    \vb F&=\vb R+\frac12\wedgecomm{\vb e}{\vb e}+\vb T
\end{align*}
Logo, 
\begin{align*}
    \tilde S&=\int\limits_{M}\expval{\vb F\ \wedgecomma \star \vb F}\\
    \tilde S&=\int\limits_{M}\expval{\vb R\ \wedgecomma \star \vb R}+\frac12\int\limits_{M}\expval{\vb R\ \wedgecomma \star \wedgecomm{\vb e}{\vb e}}+\frac12\int\limits_{M}\expval{ \wedgecomm{\vb e}{\vb e}\ \wedgecomma \star\vb R}+\int\limits_{M}\expval{\vb T\ \wedgecomma \star \vb T}+\frac14\int\limits_{M}\expval{\wedgecomm{\vb e}{\vb e}\ \wedgecomma \star \wedgecomm{\vb e}{\vb e}}\\
    \tilde S&=\int\limits_{M}\expval{\vb R\ \wedgecomma \star \vb R}+\int\limits_{M}\expval{\vb R\ \wedgecomma \star \wedgecomm{\vb e}{\vb e}}+\frac14\int\limits_{M}\expval{\wedgecomm{\vb e}{\vb e}\ \wedgecomma \star \wedgecomm{\vb e}{\vb e}}+\int\limits_{M}\expval{\vb T\ \wedgecomma \star \vb T}
\end{align*}
Isto é, a ação de Yang-Mills para o grupo de isometria de (A)dS é a ação de Einstein-Hilbert com os termos adicionais do tensor de Riemann quadrado 
e o tensor de torção quadrado. A equação de movimento para $\boldsymbol \omega$ é,
\begin{align*}
    0=\delta_{\boldsymbol\omega}\tilde S&=2\int\limits_{M}\expval{\delta_{\boldsymbol \omega}\vb R\ \wedgecomma \star \vb R}+\int\limits_{M}\expval{\delta_{\boldsymbol \omega}\vb R\ \wedgecomma \star \wedgecomm{\vb e}{\vb e}}+2\int\limits_{M}\expval{\delta_{\boldsymbol \omega}\vb T\ \wedgecomma \star \vb T}\\
    0&=2\int\limits_{M}\expval{\vb d\delta\boldsymbol \omega+\wedgecomm{\delta\boldsymbol\omega}{\boldsymbol\omega}\ \wedgecomma \star\qty( \vb R+\frac12\wedgecomm{\vb e}{\vb e})}+2\int\limits_{M}\expval{\wedgecomm{\delta\boldsymbol\omega}{\vb e}\ \wedgecomma \star \vb T}\\
    0&=\int\limits_{M}\expval{\delta\boldsymbol \omega\ \wedgecomma\ \vb d\star\qty( \vb R+\frac12\wedgecomm{\vb e}{\vb e})+\wedgecomm{\boldsymbol\omega}{\star\qty( \vb R+\frac12\wedgecomm{\vb e}{\vb e})}+\wedgecomm{\vb e}{\star\vb T}}\\
    0&=\vb d_\nabla\star \vb R+\frac12\vb d_\nabla\star\wedgecomm{\vb e}{\vb e}+\wedgecomm{\vb e}{\star\vb d_\nabla \vb e}
\end{align*}

\section{Super Riemann Surfaces and Consequences}
\label{sec:form}

\subsection{Definition of SRS}

As we foretold, to obtain a geometrical description of the superconformal structure we need a geometrical 
description of the generators $T(z),G(z)$, in the bosonic case we have such a description as differential 
operators such $T(z)\rightarrow L_n\sim z^{n+1}\partial_z$. But, to do so for $G(z)$ --- which is 
grassmann odd --- would require to have a grassmann odd coordinate. This is the start of the history of 
SRS. They're a special type of \textbf{Complex supermanifolds}, so, in order to understand them, 
we need first to understand what is a complex supermanifold.
\begin{definition}
    A \textbf{complex supermanifold} $\Sigma$ of dimension $1|1$ is a space locally isomorphic 
    to $\mathbb C^{1|1}$, that is, it's locally covered by coordinate charts $z|\theta:U\subset\Sigma\rightarrow\mathbb C^{1|1}$ 
    such that $z$ is a complex grassmann even coordinate, and $\theta$ is a complex grassmann odd coordinate.
\end{definition}
Of course we have to say something about the transition functions between overlapping charts. Due to the 
complex nature, we really just have two kinds of choice, either we impose that all transition 
functions are just continuous, or that they are holomorphic. It's clear that for our context 
the latter is way more useful, so we'll choose it. Maybe it's needed clarification on what we mean by 
a transition function $(z_i|\theta_i)^{-1}\circ z_j|\theta_j:z_j|\theta_j\qty(U_j)\subset\mathbb C^{1|1}\rightarrow \mathbb C^{1|1}$ 
to be holomorphic.
\begin{definition}
    A function $f(z,\theta):U\subset\mathbb C^{1|1}\rightarrow\mathbb C^{1|1}$ is said to be holomorphic iff 
    the expansion in powers of $\theta$ --- which is finite due to the oddness ---, $$f\qty(z,\theta)=f_0\qty(z)+\theta f_1\qty(z)$$ 
    has both $f_0,f_1: U\cap\mathbb C\subset\mathbb C\rightarrow\mathbb C^{1|1}$ holomorphic.
\end{definition}
Another definition which will be of use for us is the notion of non-zero objects in a supermanifold. 
\begin{definition}
    Any \textbf{non-zero} object --- function, vector, ... --- is one such that is non-zero up to grassmann odd 
    variables.
\end{definition}
With this we have all the ingredients to state the definition of a SRS, the point is: we want for the SRS to have a notion 
of superconformal transformations --- which is compatible with the super Virasoro algebra from SST --- as the RS had. The good thing about RS is, they don't need actually more structure than 
being a complex manifold, as any transformation $z\rightarrow f(z)$ is indeed a conformal transformation. Two 
distinct characteristic of conformal transformations in SR are: they preserve the tangent space, that is 
$\partial_z\rightarrow\partial_z z'\partial_{z'}$, and they preserve angles. These properties cannot be 
used immediately to give a prescription of super conformal transformation. If we try to set them as being 
all the set of transformations which preserve the tangent space of a complex supermanifold we would 
get actually an algebra that is way bigger than the super Virasoro algebra, and to define them as being 
\textit{angle preserving} would require to introduce a metric, which is too much of additional structure.

The idea is to look at what is really essential for us to have, if we're hoping to get any non-trivial theory 
with super conformal transformations, we need some kind of derivative operator which transforms covariantly under 
such. This is not the end, as we need to guarantee --- in a coordinate invariant way --- that covariant transformations 
of this derivative operator --- which we'll call super conformal transformations --- do in fact \textit{mix} the 
grassmann odd and even coordinates. While seem kind of arbitrary, this requirement already excludes lot's of 
possible choices as $\partial_z$ and $\partial_\theta$. Let's sum up what we concluded, to set up a SRS we 
need a notion of super conformal transformation, which we concluded is equivalent of choosing a 
covariant derivative --- what can be seen as choosing a subspace of the tangent space $T\Sigma$ ---, but no just 
any covariant derivative, one such that guarantees that we're in fact \textit{coupling} both 
coordinates. There are just two possible choices for the dimension of the subspace we're choosing, $0|1$ or $1|0$. 
We argue that it's impossible to choose a $1|0$ dimensional subspace of $T\Sigma$ and guarantee that in some sense 
the covariance under it \textit{couples} both coordinates, this is so because no operation done with only grassmann 
even variables can return a grassmann odd variable. But, the opposite is true! It's possible to compose two 
grassmann odd variables to give a grassmann even one. Hence, we conclude that to define a notion 
of super conformal transformation is needed to choose a $0|1$ dimensional subspace of $T\Sigma$, what about the 
condition to ensure \textit{mixing} of coordinates? Let's save for the definition\cite{witten:revisited,witten:moduli,witten:integration},
\begin{definition}
    A \textbf{Super Riemann Surface} $\Sigma$ is a complex supermanifold of dimension $1|1$ that 
    possesses a distinct subbundle/subspace $\mathcal D\subset T\Sigma$ of dimension $0|1$ that satisfy the 
    completely non-integrability condition, $$\forall D\in\mathcal D: D\textnormal{ is non-zero}\Rightarrow \comm{D}{D}\notin \mathcal D$$ 
    $\mathcal D$ is said to be a complex structure of this SRS.\label{definition:srs}
\end{definition}
Here we see the beauty of this definition, as we discussed, we have the distinguished structure as a $0|1$ dimensional 
subspace of $T\Sigma$, and the condition that it must satisfy is the so called \textit{completely non-integrability}, 
which is dependent on the graded Lie Bracket operation defined for vector fields $\comm{\cdot}{\cdot}$, what is 
important here is that $\comm{D}{D}$ is only non-zero if $D$ is odd, and in this case, by statistics, it returns 
a even vector field. Here the hypothesis of $D$ being non-zero is crucial, otherwise, $\comm{D}{D}$ could be 
zero and so, despite being even, could still belong to $\mathcal D$. Hence, the completely non-integrability 
condition provides a notion of \textit{coupling} both coordinates without having to evoke new structure in 
our manifold. It's instructive to see what kind of elements belongs to this subbundle,
\begin{lemma}\label{lemma:formd}
    Given a SRS with distinct subbundle $\mathcal D$, it's always possible to construct a 
    coordinate system $z|\theta$, such that locally, any element $D\in\mathcal D$ is of the 
    form, \[D_\theta\coloneq D_{U_{z|\theta}}=\partial_\theta+\theta\partial_z,\ \ \ D_\theta^2\coloneq\frac12\comm{D_\theta}{D_\theta}=\partial_z\numberthis\label{covariantderivative}\] we call 
    this coordinate system a superconformal one.
\end{lemma}
\begin{proof}
    In a given coordinate system/chart $z|\theta$ we always can decompose any element of the tangent space as,
    $$D_{U_{z|\theta}}=a\qty(z,\theta)\partial_\theta+b\qty(z,\theta)\partial_z$$ the condition of $D_{U_{z|\theta}}$ being 
    non-zero, as stated in \cref{definition:srs}, is equivalent to $a\qty(z,\theta)\neq 0$, hence, it's possible to scale 
    $\theta\rightarrow a\theta$, doing this and also expanding, due to the grassmann odd character, 
    $b\qty(z,\theta)=b_0\qty(z)+\theta b_1\qty(z)$, 
    $$D_{U_{z|\theta}}=\partial_\theta+(b_0+\theta b_1)\partial_z$$ which now we compute the graded Lie Bracket,
    \begin{align*}
        \comm{D_{U_{z|\theta}}}{D_{U_{z|\theta}}}&=\comm{\partial_\theta+(b_0+\theta b_1)\partial_z}{\partial_\theta+(b_0+\theta b_1)\partial_z}\\
        \comm{D_{U_{z|\theta}}}{D_{U_{z|\theta}}}&=2\comm{\partial_\theta}{(b_0+\theta b_1)\partial_z}+\comm{(b_0+\theta b_1)\partial_z}{(b_0+\theta b_1)\partial_z}\\
        \comm{D_{U_{z|\theta}}}{D_{U_{z|\theta}}}&=2\partial_\theta(b_0+\theta b_1)\partial_z+2\qty(b_0+\theta b_1)\comm{\partial_\theta}{\partial_z}+2(b_0+\theta b_1)\qty(\partial_z(b_0+\theta b_1)\partial_z+(b_0+\theta b_1)\partial^2_z)\\
        \comm{D_{U_{z|\theta}}}{D_{U_{z|\theta}}}&=2b_1\partial_z+2\theta b_1\partial_z b_0\partial_z-2\theta b_0\partial_z b_1\partial_z\numberthis\label{temp:eq1}
    \end{align*}
    so, the only way $\comm{D_{U_{z|\theta}}}{D_{U_{z|\theta}}}\in\mathcal D$ could possibly be true is if the above expression is identically zero, which is 
    only possible if $b_1=0$. Thus, the completely non-integrability condition gives the requirement $b_1\neq 0$, hence, we can perform a 
    further change of coordinates $\theta\rightarrow -b_1^{-1}b_0+\theta$, $z\rightarrow b_1 z$,
    $$D_{\theta}\coloneq D_{U_{z|\theta}}=\partial_\theta+\theta \partial_z$$ And lastly, 
    setting $b_1=1, b_0=0$ in \cref{temp:eq1} we get, $D^2_\theta = \partial_z$.
\end{proof}
The existence of such structure is what settles apart a SRS from a generic complex supermanifold. 
To grasp a better understanding of this non-integrability condition we have to take a look at the 
second equation in \eqref{covariantderivative}, for a usual coordinate basis of the tangent space 
we always have $\comm{\partial_I}{\partial_J}=0$, which kind of induces a splitting of the 
tangent space as $\mathbb C^{1|0}\times\cdots\times\mathbb C^{0|1}\times\cdots$, but of course, 
as this is a coordinate basis, this splitting is not some kind of inner property of the manifold, 
it's a coordinate dependent gimmick. But, with our definition being coordinate independent, 
we're saying that existis a global splitting of the tangent space $T\Sigma\cong \mathbb C^{1|0}\times \mathbb C^{0|1}\cong T\Sigma/\mathcal D\times \mathcal D\cong \mathcal D ^2\times\mathcal D$.

We will now pursue 
why does this structure is able to reconstruct the off-shell super Virasoro algebra in SST.

\subsection{Superconformality in SRS}

As we foretold, having a distinct subbundle $\mathcal D$, is possible for us to define super conformal 
coordinate changes. The idea is, vector fields naturally introduce coordinate changes by integral curves, 
hence,
\begin{definition}
    A vector field $W\in T\Sigma$ is said to generate a super conformal coordinate transformation on a 
    SRS $\Sigma$, if it preserves the subbundle $\mathcal D$, that is, 
    $$W\in T\Sigma \textnormal{ generates superconformal transformation}\Leftrightarrow\forall D\in\mathcal D:\, \comm{W}{D}\in\mathcal D.$$
\end{definition}
As good as having a formal definition may be, our true interest here is to obtain these transformations in a given basis, 
in particular, a superconformal basis,
\begin{lemma}\label{lemma:basissuperconformal}
    The set of all vector fields that generate superconformal transformation can be decomposed in a basis, 
    with a superconformal coordinate system, of one even and one odd vector fields such, 
    \[G_f=f\qty(z)\qty(\partial_\theta-\theta\partial_z),\ \ \ L_g=g(z)\partial_z+\frac12 g'\qty(z)\theta\partial_\theta\numberthis\label{nsgenerators}\]
\end{lemma}
\begin{proof}
We will compute it using $D_\theta$ and a decomposition of $W=a\partial_\theta+b\partial_z$
\begin{align*}
    \comm{W}{D_\theta}&=\comm{a\partial_\theta+b\partial_z}{D_\theta}=a\comm{\partial_\theta}{D_\theta}\mp D_\theta a\partial_\theta+b\comm{\partial_z}{D_\theta}\mp D_\theta b\partial_z\\
    \comm{W}{D_\theta}&=a\partial_z\mp D_\theta a\partial_\theta\mp D_\theta b\partial_z=\mp D_\theta a\partial_\theta\mp \qty(D_\theta b\mp a)\partial_z
\end{align*}
Where the signs reefer to $W$ being even, top sign, or odd, bottom sign. To say that $\comm{W}{D_\theta}\in\mathcal D$ is the same 
to say that $\comm{W}{D_\theta}\propto D_\theta$, hence, the condition is,
\begin{align*}
    \mp \qty(D_\theta b\mp a)&=\mp D_\theta a\theta\\
    D_\theta b&= D_\theta a\theta\pm a
\end{align*}
In this form it may seem difficult to solve it, but, we will use our virtue of foresight to propose an ansatz, $b=-a\theta$,
\begin{align*}
    D_\theta (-a\theta)&= D_\theta a\theta\pm a\Rightarrow -D_\theta a\theta\pm a= D_\theta a\theta\pm a\Rightarrow D_\theta a\theta=0\Rightarrow \partial_\theta a\theta=0\Rightarrow \begin{cases}a&=a\qty(z)\\b&=-a\qty(z)\theta\end{cases}
\end{align*}
With this ansatz we got exactly an solution, that is, one family of odd vector fields that generate superconformal transformations are $G_f=f\qty(z)\qty(\partial_\theta-\theta\partial_z)$. 
But, this is not the end, as we have two dimensions, one odd and one even, we know that there is one more solution to this equation,
\begin{align*}
    D_\theta b&=D_\theta a\theta\pm a\\
    D^2_\theta b&=D^2_\theta a\theta\pm D_\theta a\pm D_\theta a\\
    \partial_z b&=\partial _z a\theta \pm2\partial_\theta a\pm2\theta\partial_z a=-\partial_za\theta\pm2\partial_\theta a
\end{align*}
it's clear that the condition $\partial_\theta a=0$ will return our already found solution, hence, we try to 
solve this equation for $\partial_za\theta=0\Rightarrow \partial_z b=\pm 2\partial_\theta a$, substituting 
back in the original equation,
\begin{align*}
    \partial_\theta b+\theta\partial_z b&=\partial_\theta a\theta\pm a\\
    \partial_\theta b\pm2\theta\partial_\theta a&=\partial_\theta a\theta\pm a\partial_\theta \theta\\
    \partial_\theta b+2\partial_\theta a\theta&=\partial_\theta a\theta\pm a\partial_\theta \theta\\
    \partial_\theta b&=-\partial_\theta a\theta\pm a\partial_\theta \theta=-\partial_\theta\qty(a\theta)
\end{align*}
As we already solve for $b=-a\theta$, the only other possible solution that can be obtained from here is 
$\partial_\theta b=\partial_\theta\qty(a\theta)=0\Rightarrow a=c\qty(z)\theta,\ b=b\qty(z)$, and the last consistency 
condition gives,
\begin{align*}
    \partial_z b&=\pm 2\partial_\theta a\Rightarrow\partial_z b= 2c\qty(z)\Rightarrow \begin{cases}a &= \frac12 \partial_z b\qty(z)\theta\\b &= b\qty(z)\end{cases} 
\end{align*}
which give to us the second and last linear independent solution, a family of even vector fields that generate 
superconformal transformations $L_g=g\qty(z)\partial_z+\frac12g'\qty(z)\theta\partial_\theta$.
\end{proof}
It's no coincidence the naming we utilized in \cref{nsgenerators}, as those two are the generators of the 
superconformal transformations, they should also be related to the SST super Virasoro generators $L_n,G_r$, 
and actually this is true, they indeed furnish a differential representation of the super Virasoro,
\begin{lemma}\label{lemma:nsgenerators}
    The vector field basis, \cref{nsgenerators}, of superconformal transformations for $f\qty(z)=z^{r+\frac12},g\qty(z)=-z^{n+1}$, 
    \[L_n=-z^{n+1}\partial_z-\frac{n+1}{2}z^n\theta\partial_\theta,\ \ \ G_r=z^{r+\frac12}\qty(\partial_\theta-\theta\partial_z),\ \ \ n\in\mathbb Z,\ r\in\mathbb Z+\frac12\numberthis\label{generators:ns2}\]
    furnishes a representation of the super Virasoro algebra for the NS sector,
    \begin{subequations}\label{supervirasoro:ns}\begin{align}
        \comm{L_m}{L_n}&=\qty(m-n)L_{m+n}\\
        \comm{G_r}{G_s}&=2L_{r+s}\\
        \comm{L_m}{G_r}&=\qty(\frac m2-r)G_{m+r}
    \end{align}\end{subequations}
\end{lemma}
\begin{proof}
    The computation is very straightforward, we start with $L-L$,
    \begin{align*}
        \comm{L_m}{L_n}&=\comm{z^{m+1}\partial_z+\frac{m+1}{2}z^m\theta\partial_\theta}{z^{n+1}\partial_z+\frac{n+1}{2}z^n\theta\partial_\theta}\\
        \comm{L_m}{L_n}&=\comm{z^{m+1}\partial_z}{z^{n+1}\partial_z}+\comm{z^{m+1}\partial_z}{\frac{n+1}{2}z^n\theta\partial_\theta}+\comm{\frac{m+1}{2}z^m\theta\partial_\theta}{z^{n+1}\partial_z}\\
        \comm{L_m}{L_n}&=z^{m+1}\partial_zz^{n+1}\partial_z-z^{n+1}\partial_zz^{m+1}\partial_z+\frac{n+1}{2}z^{m+1}\theta\partial_zz^n\partial_\theta-\frac{m+1}{2}z^{n+1}\theta\partial_zz^m\partial_\theta\\
        \comm{L_m}{L_n}&=\qty(n+1)z^{m+n+1}\partial_z-\qty(m+1)z^{n+m+1}\partial_z+\frac{\qty(n+1)n}{2}z^{m+n}\theta\partial_\theta-\frac{\qty(m+1)m}{2}z^{n+m}\theta\partial_\theta\\
        \comm{L_m}{L_n}&=-\qty(m-n)z^{m+n+1}\partial_z+\qty(n^2+n-m^2-m)\frac{z^{n+m}}{2}\theta\partial_\theta\\
        \comm{L_m}{L_n}&=-\qty(m-n)z^{m+n+1}\partial_z-\qty(m-n)\qty(n+m+1)\frac{z^{n+m}}{2}\theta\partial_\theta=\qty(m-n)L_{m+n}
    \end{align*}
    now $G-G$,
    \begin{align*}
        \comm{G_r}{G_s}&=\comm{z^{r+\frac12}\qty(\partial_\theta-\theta\partial_z)}{z^{s+\frac12}\qty(\partial_\theta-\theta\partial_z)}\\
        \comm{G_r}{G_s}&=-\qty(z^{r+\frac12}\theta\partial_z z^{s+\frac12}+z^{s+\frac12}\theta\partial_z z^{r+\frac12})\qty(\partial_\theta-\theta\partial_z)+z^{r+s+1}\comm{\qty(\partial_\theta-\theta\partial_z)}{\qty(\partial_\theta-\theta\partial_z)}\\
        \comm{G_r}{G_s}&=-\qty(s+\frac12+r+\frac12)z^{r+s}\theta\partial_\theta-2z^{r+s+1}\partial_z=2\qty(-z^{r+s+1}\partial_z-\frac{r+s+1}{2}z^{r+s}\theta\partial_\theta)\\
        \comm{G_r}{G_s}&=2L_{r+s}
    \end{align*}
    and lastly $L-G$,
    \begin{align*}
        \comm{L_m}{G_r}&=\comm{-z^{m+1}\partial_z-\frac{m+1}{2}z^m\theta\partial_\theta}{z^{r+\frac12}\qty(\partial_\theta-\theta\partial_z)}\\
        \comm{L_m}{G_r}&=-z^{m+1}\partial_z z^{r+\frac12}\partial_\theta+\frac{m+1}{2}z^{m+r+\frac12}\partial_\theta+z^{m+1}\partial_z z^{r+\frac12}\theta\partial_z-z^{r+\frac12}\theta\partial_zz^{m+1}\partial_z\\
        &\quad\quad\quad+\frac{m+1}{2}z^{m+r+\frac12}\theta\partial_z\\
        \comm{L_m}{G_r}&=\qty(\frac m2+\frac12 -r-\frac12)z^{m+r+\frac12}\partial_\theta+\qty(r+\frac12-\frac m2 -\frac12)z^{m+r+\frac12}\theta\partial_z\\
        \comm{L_m}{G_r}&=\qty(\frac m2-r)z^{m+r+\frac12}\qty(\partial_\theta-\theta\partial_z)=\qty(\frac m2-r)G_{m+r}
    \end{align*}
\end{proof}
As we have mentioned before, this last result is one of the main reasons why SRS are useful, they make the superconformal algebra 
on-shell, it's instructive to see explicitly from \cref{supervirasoro:ns} the realization of SUSY, $\comm{G_{-\frac12}}{G_{-\frac12}}=2L_{-1}=-2\partial_z$. Nevertheless, there are still many points left to explanation, as stated in \cref{lemma:nsgenerators}, the 
resulting super Virasoro algebra from SRS are in the NS sector, that is, they only work for states with NS boundary conditions. 
As a reminder, NS boundary conditions are imposed in the world-sheet fermions \eqref{action:sst1} as $\psi^\mu\qty(w+2\pi\im)=-\psi^\mu\qty(w)$, 
with $w=\ln z$ the cylinder coordinate. We will not prove here, but, it's impossible to reconstruct the super Virasoro 
algebra in the R sector --- $\psi^\mu\qty(w+2\pi\im)=\psi^\mu\qty(w)\Rightarrow G_r,\ r\in\mathbb Z$ --- with \eqref{nsgenerators}. 
While this may seem a big problem, there is an interesting interpretation of why this is the case, and in fact this is no obstruction 
to the introduction of R sector fermions, for now we just cite what is the form of the super 
Virasoro generators for the R sector, and later on there will be a explanation on why this should be true,
\begin{lemma}
    The following vector fields, \[L_n=-z^{n+1}\partial_z-\frac n2z^n\theta\partial_\theta,\ \ \ G_r=z^r\qty(\partial_\theta-\theta z\partial_z),\ \ \ n,r\in\mathbb Z\numberthis\label{rgenerators}\] 
    furnishes a differential representation of the super Virasoro algebra for the R sector,
    \begin{subequations}\begin{align}
        \comm{L_m}{L_n}&=\qty(m-n)L_{m+n}\\
        \comm{G_r}{G_s}&=2L_{r+s}\\
        \comm{L_m}{G_r}&=\qty(\frac m2-r)G_{m+r}
    \end{align}\label{supervirasoro:r}\end{subequations}
\end{lemma}
\begin{proof}
    Again, the calculation is very straightforward, we start with $L-L$,
    \begin{align*}
        \comm{L_m}{L_n}&=\comm{z^{m+1}\partial_z+\frac{m}{2}z^m\theta\partial_\theta}{z^{n+1}\partial_z+\frac{n}{2}z^n\theta\partial_\theta}\\
        \comm{L_m}{L_n}&=\comm{z^{m+1}\partial_z}{z^{n+1}\partial_z}+\comm{z^{m+1}\partial_z}{\frac{n}{2}z^n\theta\partial_\theta}+\comm{\frac{m}{2}z^m\theta\partial_\theta}{z^{n+1}\partial_z}\\
        \comm{L_m}{L_n}&=z^{m+1}\partial_zz^{n+1}\partial_z-z^{n+1}\partial_zz^{m+1}\partial_z+\frac{n}{2}z^{m+1}\theta\partial_zz^n\partial_\theta-\frac{m}{2}z^{n+1}\theta\partial_zz^m\partial_\theta\\
        \comm{L_m}{L_n}&=\qty(n+1)z^{m+n+1}\partial_z-\qty(m+1)z^{n+m+1}\partial_z+\frac{n^2}{2}z^{m+n}\theta\partial_\theta-\frac{m^2}{2}z^{n+m}\theta\partial_\theta\\
        \comm{L_m}{L_n}&=-\qty(m-n)z^{m+n+1}\partial_z+\qty(n^2-m^2)\frac{z^{n+m}}{2}\theta\partial_\theta\\
        \comm{L_m}{L_n}&=-\qty(m-n)z^{m+n+1}\partial_z-\qty(m-n)\qty(n+m)\frac{z^{n+m}}{2}\theta\partial_\theta=\qty(m-n)L_{m+n}
    \end{align*}
    now $G-G$,
    \begin{align*}
        \comm{G_r}{G_s}&=\comm{z^{r}\qty(\partial_\theta-\theta z\partial_z)}{z^{s}\qty(\partial_\theta-\theta z\partial_z)}\\
        \comm{G_r}{G_s}&=-\qty(z^{r+1}\theta\partial_z z^{s}+z^{s+1}\theta\partial_z z^{r})\qty(\partial_\theta-\theta z\partial_z)+z^{r+s}\comm{\qty(\partial_\theta-\theta z\partial_z)}{\qty(\partial_\theta-\theta z\partial_z)}\\
        \comm{G_r}{G_s}&=-\qty(s+r)z^{r+s}\theta\partial_\theta-2z^{r+s}z\partial_z=2\qty(-z^{r+s+1}\partial_z-\frac{r+s}{2}z^{r+s}\theta\partial_\theta)\\
        \comm{G_r}{G_s}&=2L_{r+s}
    \end{align*}
    and lastly $L-G$,
    \begin{align*}
        \comm{L_m}{G_r}&=\comm{-z^{m+1}\partial_z-\frac{m}{2}z^m\theta\partial_\theta}{z^{r}\qty(\partial_\theta-\theta z\partial_z)}\\
        \comm{L_m}{G_r}&=-z^{m+1}\partial_z z^{r}\partial_\theta+\frac{m}{2}z^{m+r}\partial_\theta+z^{m+1}\partial_z z^{r+1}\theta\partial_z-z^{r+1}\theta\partial_zz^{m+1}\partial_z\\
        &\quad\quad\quad+\frac{m}{2}z^{m+r+1}\theta\partial_z\\
        \comm{L_m}{G_r}&=\qty(\frac m2-r)z^{m+r}\partial_\theta+\qty(r+1-\frac m2 -1)z^{m+r+1}\theta\partial_z\\
        \comm{L_m}{G_r}&=\qty(\frac m2-r)z^{m+r}\qty(\partial_\theta-\theta z\partial_z)=\qty(\frac m2-r)G_{m+r}
    \end{align*}
\end{proof}
While might seem taken out of a hat, this shows two main points: (i) that is possible to somewhat \textit{embed} the R sector in SRS, but, (ii) the fact that \eqref{rgenerators} cannot be written 
as \eqref{nsgenerators}, together with \cref{lemma:basissuperconformal}, shows that the super Virasoro R sector cannot possibly 
preserve our distinct subspace $\mathcal D$. There is a small flaw in this argument that may shed some hope in the interpretation of 
R fermions, which we are going to postpone. For now we follow working with just the NS sector.

\subsection{Super fields and weights}

Having now constructed SRS, and also shown that they do satisfy an on-shell super Virasoro algebra, the next step is to 
do a matching of how a theory formulated in SRS can give the action in \cref{action:sst1}. 
The main ideia here is to follow analogous to BST, first to describe objects that transform covariantly 
under superconformal transformations, and later use them to construct a superconformal invariant action. It's instructive 
to first understand how $\theta$ behaves under dilations, which for now is the closest we have to a conformal weight. 
As we know, dilations are the integral curves of, \cref{generators:ns2}, $L_0=-z\partial_z-\frac12\theta\partial_\theta$, 
which are trivially $z|\theta\rightarrow \lambda z|\lambda^{\frac12}\theta$. This kind of is a reminiscent of $z$ having 
conformal weight $-1$ and $\theta$ having conformal weight $-\frac12$, this is not really true, as we used global transformations, 
to get a real definition of what may be superconformal weights we have to look at general transformations. By now we are fully 
aware that superconformal transformations preserve $\mathcal D$, other way to pose it is, superconformal transformations of 
a superconformal coordinate system $z|\theta$ to another superconformal coordinate system $\hat z|\hat\theta$ changes $D_\theta$ to 
$F\qty(z,\theta)D_{\hat\theta}$, hence, it should be true that,
\begin{align*}
    D_\theta=F\qty(z,\theta)D_{\hat\theta}\Rightarrow D_\theta\hat\theta&=F\qty(z,\theta)D_{\hat\theta}\hat\theta\\
    D_\theta\hat\theta&=F\qty(z,\theta)\Rightarrow D_\theta=D_\theta\hat\theta D_{\hat\theta}\numberthis\label{superconformal:definition}
\end{align*}
which is the definition we will be taking of a superconformal transformation. Plenty other consistency conditions can be 
obtained from \cref{superconformal:definition} by applying it to $z,\hat z,\theta$. Those are specially useful if one want to 
show what is the most general form of a superconformal transformation explicitly, but, there isn't so much of insight coming 
from this, and thus we won't attempt it here. What indeed is of interest to us is how \cref{superconformal:definition} is 
similar to the conformal transformation of the BST counterpart, $\partial_z=\partial_z z'\partial_{z'}$, which is used to define the conformal weights scaling,
\[\qty(\partial_z z')^{h}\qty(\partial_{\bar z}{\bar z}')^{\tilde h}{\phi}'\qty(z',{\bar z}')=\phi\qty(z,\bar z)\]
but, as $D^2_\theta=\partial_z$, this suggest that we should consider the following as a definition of the superconformal scaling 
weights, \[\qty(D_\theta \theta')^{2h}\phi'\qty(z',\theta')=\phi\qty(z,\theta)\] notice the $2h$ instead of $h$. But, still, there 
is something off with this definition, the lack of conjugated variables $\bar z,\bar\theta$, there is a reason for this. 
In the presence of both complex even and odd variables, $z,\theta$, it's in fact impossible to define a notion of 
complex conjugation $z|\theta\rightarrow\bar z|\bar \theta$, hence, we cannot talk about $z,\bar z|\theta,\bar\theta$ 
in a single SRS. This is no problem, as we can always set the full world-sheet $\Sigma$ as being a product of two SRS, 
$\Sigma_L,\Sigma_R$, one with local holomorphic coordinates $z|\theta$, and the other with local anti-holomorphic 
coordinates $\bar z|\bar \theta$. Notice, here not necessarily $\bar z=z^\ast$. Now, in the full world-sheet $\Sigma\cong\Sigma_L\times \Sigma_R$, 
it's possible to define superconformal weights as, \[\qty(D_\theta \theta')^{2h}\qty(D_{\bar\theta} {\bar\theta}')^{2h}\phi'\qty(z',{\bar z}',\theta',{\bar\theta}')=\phi\qty(z,{\bar z},\theta,{\bar\theta})\] 
One last matter that we haven't touched yet is the integration measure. In possess of superconformal fields, we have to 
decide over what we should integrate them to obtain a superconformally invariant action. The answer is kind of 
trivial, we should integrate over top forms! Sadly, there are lots of subtleties involving forms and integration in 
supermanifolds which we won't deal with here. We're just to cite a result that we will need.
\begin{lemma}\label{berezinian}
    In a SRS $\Sigma_L$ there is a natural way to define a 2--form with dimensions $1|1$ called the Berezinian, Ber$\qty(\Sigma_L)$, which in 
    a coordinate system $z|\theta$ is denoted as $\qty[\dd{z}|\dd{\theta}]$. It provides a natural way of defining line 
    integrals in $\Sigma_L$,
    \[\int\limits_{\gamma\subset\Sigma_L}\qty[\dd{z}|\dd{\theta}]\qty(f_0\qty(z)+\theta f_1\qty(z))=\int\limits_{\gamma\cap\mathbb C^{1|0}\subset\Sigma_L\cap\mathbb C^{1|0}}\dd{z}f_1\qty(z)\]
    It also transform covariantly under superconformal transformations, \[\qty[z'|\theta']=\qty[z|\theta]D_\theta\theta'\numberthis\label{measuretransformation}\]
\end{lemma}
\begin{proof} See \cite{witten:integration}.\end{proof}
Might seem that that the Berezinian has a downside, it only sets up a procedure for doing line integrals, not volume ones --- 
the ones we're interested in ---. This is not really of a problem, as we mentioned before, it's impossible to set up $\bar z,\bar\theta$ 
with just a single holomorphic SRS, this is also a reason that we cannot construct a volume form in a holomorphic SRS, hence, 
the resolution of this problem is what we already done, it is to work with $\Sigma=\Sigma_L\times\Sigma_R$, and thus set up 
the Berezinian as Ber$\qty(\Sigma)=\textnormal{Ber}\qty(\Sigma_L)\otimes\textnormal{Ber}\qty(\Sigma_R)$. Now, Ber$\qty(\Sigma)$ 
in local coordinates $z,\bar z|\theta,\bar\theta$ is $\qty[\dd{z},\dd{\bar z}|\dd{\theta},\dd{\bar\theta}]=\qty[\dd{z}|\dd{\theta}]\otimes\qty[\dd{\bar z}|\dd{\bar\theta}]$ So 
that now we can pose our action as being an integral such, \[S=\int\limits_\Sigma\qty[\dd{z},\dd{\bar z}|\dd{\theta},\dd{\bar\theta}]\phi\qty(z,\bar z,\theta,\bar\theta)\] 
By the transformation properties, \cref{measuretransformation}, we need $\phi$ to be a $\qty(\frac12,\frac12)$ weight superfield in order to 
the action to be invariant. Notice, if we want a non-trivial theory, we better have derivatives, and the only covariant ones at our disposal are 
$D_\theta,D_{\bar\theta}$, which by \cref{superconformal:definition} rises the weights by $\qty(\frac12,0)$ and $\qty(0,\frac12)$. 
Hence, the easiest way to set up an action is from a complex $\qty(0,0)$ superfield $\mathbb X^\mu\qty(z,\bar z,\theta,\bar\theta)$ 
using the derivatives $D_\theta,D_{\bar\theta}$, \[S=-\int\limits_\Sigma\qty[\dd{z},\dd{\bar z}|\dd{\theta},\dd{\bar\theta}]D_\theta \mathbb X^\mu D_{\bar\theta}\mathbb X_\mu\numberthis\label{action:srssst1}\] 
This is manifestly superconformally invariant, and, by now doesn't seem a lot similar to \cref{action:sst1}, but, we have a trick up in our sleave, 
due to the oddness of $\theta,\bar\theta$ we can expand $\mathbb X^\mu$ in powers of them,
\[\mathbb X^\mu\qty(z,\bar z,\theta,\bar\theta)=X^\mu\qty(z,\bar z)+\im\theta\psi^\mu\qty(z,\bar z)+\im\bar\theta{\tilde\psi}^\mu\qty(z,\bar z)+\bar\theta\theta F^\mu\qty(z,\bar z)\numberthis\label{xdecomposition}\]
Here we used our virtue of foresight to name each component of the superfield as the ones in \cref{action:sst1}, this is possible because as 
we saw, $\theta,\bar\theta$ kind of transforms --- under global transformations --- as would a field of conformal weights $\qty(-\frac12,0),\qty(0,-\frac12)$. 
This forces $\psi^\mu,{\tilde\psi}^\mu$ to transforms as $\qty(\frac12,0),\qty(0,\frac12)$ --- as they do naturally in SST --- to keep every term in the expansion a scalar. 
The $F^\mu$ field doesn't have a counterpart in SST because it is a auxiliary field. We shown that SRS makes the superconformal/super Virasoro 
algebra/redundancy off-shell, this is partially the role of the $F^\mu$, despite having a trivial equation of motion --- which we'll show --- 
it has a non-trivial transformation under the superconformal transformations. Hence, when integrated out, we get exactly the SST action, but at 
the price of losing one of the fields, turning the superconformal redundancy valid only on-shell. Let us obtain the SST action from our 
try \eqref{action:srssst1},
\begin{lemma}
    With the identification \eqref{xdecomposition},
    \[S=-\int\limits_\Sigma\qty[\dd{z},\dd{\bar z}|\dd{\theta},\dd{\bar\theta}]D_\theta \mathbb X^\mu D_{\bar\theta}\mathbb X_\mu=\int\limits_{\Sigma_{\textnormal{red}}}\dd{z}\dd{\bar z}\qty(\partial X^\mu\bar\partial X_\mu+\psi^\mu\bar\partial\psi_\mu+{\tilde\psi}^\mu\partial{\tilde\psi}_\mu+F^\mu F_\mu)\numberthis\label{action:sst3}\]
\end{lemma}
\begin{proof}
    The idea here is simple, as stated in \cref{berezinian}, the integration measure will pick up just terms proportional to $\bar\theta\theta$, 
    hence, we just need to worry about them, we start computing what is $D_\theta\mathbb X^\mu$,
    \begin{align*}
        D_\theta \mathbb X^\mu&=\qty(\partial_\theta+\theta\partial_z)\qty(X^\mu+\im\theta\psi^\mu+\im\bar\theta{\tilde\psi}^\mu+\bar\theta\theta F^\mu)\\
        D_\theta \mathbb X^\mu&=\im\psi^\mu-\bar\theta F^\mu+\theta\partial\qty(X^\mu+\im\bar\theta{\tilde\psi}^\mu)
    \end{align*}
    analogously,
    \begin{align*}
        D_{\bar\theta} \mathbb X^\mu&=\qty(\partial_{\bar\theta}+\bar\theta\partial_{\bar z})\qty(X^\mu+\im\theta\psi^\mu+\im\bar\theta{\tilde\psi}^\mu+\bar\theta\theta F^\mu)\\
        D_{\bar\theta} \mathbb X^\mu&=\im{\tilde\psi}^\mu+\theta F^\mu+\bar\theta\bar\partial\qty(X^\mu+\im\theta{\psi}^\mu)
    \end{align*}
    so that,
    \begin{align*}
        D_\theta \mathbb X^\mu D_{\bar\theta} \mathbb X_\mu&=\qty(\im\psi_\mu-\bar\theta F_\mu+\theta\partial X_\mu+\im\theta\bar\theta\partial{\tilde\psi}_\mu)\qty(\im{\tilde\psi}_\mu+\theta F_\mu+\bar\theta\bar\partial X_\mu+\im\bar\theta\theta\bar\partial{\psi}_\mu)\\
        D_\theta \mathbb X^\mu D_{\bar\theta} \mathbb X_\mu&=-\bar\theta\theta\psi^\mu\bar\partial\psi_\mu-\bar\theta\theta F^\mu F_\mu-\bar\theta\theta\partial X^\mu\bar\partial X_\mu-\bar\theta\theta{\tilde\psi}^\mu\partial{\tilde\psi}_\mu+\textnormal{non }\bar\theta\theta\textnormal{ proportional terms}
    \end{align*}
    from which we can readily integrate keeping only the $\bar\theta\theta$ terms,
    \begin{align*}
        S=-\int\limits_\Sigma\qty[\dd{z},\dd{\bar z}|\dd{\theta},\dd{\bar\theta}]D_\theta \mathbb X^\mu D_{\bar\theta}\mathbb X_\mu=\int\limits_{\Sigma_{\textnormal{red}}}\dd{z}\dd{\bar z}\qty(\partial X^\mu\bar\partial X_\mu+\psi^\mu\bar\partial\psi_\mu+{\tilde\psi}^\mu\partial{\tilde\psi}_\mu+F^\mu F_\mu)
    \end{align*}
\end{proof}

This last result makes the connection between theories formulated in SRS, and SST natural. It's clear from \cref{action:sst3} that the 
equation of motion for $F^\mu$ is just $F^\mu=0$, hence it can be trivially integrated out, giving exactly \cref{action:sst1}. One might 
worry about the ghosts, but, these can be also naturally incorporated in the superfield formalism, we won't do this here. What is interesting to 
see is what are the SUSY transformations, \cref{susy:pol1}, in this superfield formalism. For this is necessary to remember that 
the SUSY generator is $G_{-\frac12}$ --- at least for the NS sector ---, which has an action as,
\begin{align*}
    \delta_\epsilon\mathbb X^\mu&=\qty(\epsilon G_{-\frac12}+\epsilon^\ast \bar G_{-\frac12})\mathbb X^\mu=\qty(\epsilon\qty(\partial_\theta-\theta\partial_z)+\epsilon^\ast\qty(\partial_{\bar\theta}-\bar\theta\partial_{\bar z}))\qty(X^\mu+\im\theta\psi^\mu+\im\bar\theta{\tilde\psi}^\mu+\bar\theta\theta F^\mu)\\
    \delta_\epsilon\mathbb X^\mu&=\epsilon\qty(\im\psi^\mu-\bar\theta F^\mu)-\epsilon\theta\partial\qty(X^\mu+\im\bar\theta{\tilde\psi}^\mu)+\epsilon^\ast\qty(\im{\tilde\psi}^\mu+\theta F^\mu)-\epsilon^\ast\bar\theta\bar\partial\qty(X^\mu+\im\theta{\psi}^\mu)
\end{align*}
If we set up $\delta_\epsilon\mathbb X^\mu=\delta_\epsilon X^\mu+\im\theta\delta_\epsilon\psi^\mu+\im\bar\theta\delta_\epsilon{\tilde\psi}^\mu+\bar\theta\theta\delta_\epsilon F^\mu$, then we can do the matching of the last expression to 
give the following transformation laws,
\begin{align*}\begin{cases}
    \delta_\epsilon X^\mu&=\im\epsilon\psi^\mu+\im\epsilon^\ast{\tilde\psi}^\mu\\
    \delta_\epsilon \psi^\mu&=-\im\epsilon\partial X^\mu+\im\epsilon^\ast F^\mu\\
    \delta_\epsilon {\tilde\psi}^\mu&=-\im\epsilon^\ast\bar\partial X^\mu-\im\epsilon F^\mu\\
    \delta_\epsilon F^\mu&=-\im\epsilon^\ast\bar\partial\psi^\mu+\im\epsilon\partial{\tilde\psi}^\mu\end{cases}\Rightarrow \textnormal{on-shell }F^\mu=0\Rightarrow\begin{cases}
    \delta_\epsilon X^\mu&=\im\epsilon\psi^\mu+\im\epsilon^\ast{\tilde\psi}^\mu\\
    \delta_\epsilon \psi^\mu&=-\im\epsilon\partial X^\mu\\
    \delta_\epsilon {\tilde\psi}^\mu&=-\im\epsilon^\ast\bar\partial X^\mu\end{cases}
\end{align*}
Which is exactly \cref{susy:pol1}, with just extra factors of $-\im$ due to conventions. 

We tried 
in this section to show that SRS can provide a natural environment for superconformal theories, there are a 
lot of ways we could continue our discussion, we choose here to close with a few remarks about 
vertex operators, insertions and punctures.

\section{Punctures in SRS}
\label{sec:punc}


\section{Concluding Remarks}

In this brief work we have shown the basics of Super Riemann Surfaces, in particular, 
beyond the definitions, we proved how the SRS can achieve the on-shell super Virasoro 
algebra, and the relations of the R sector with degeneracies of the complex 
structure of the SRS. There are more issues with SRS that are of interest to SST, 
in particular one that we haven't touched here, the moduli space. For the 
superstring, the moduli space itself is a SRS, that is, have both odd and 
even variables, and punctures --- both NS and R ones --- alter the 
dimensionality of it. This is a essential matter for the SST perturbation 
theory, as the partition function needs the integration over the moduli space, 
but sadly, to cover it also would require a much bigger work.

% \appendix

% \section{Mathematical Toolkit}
\label{app:tool}


\bibliographystyle{JHEP}
\bibliography{main}

\end{document}
