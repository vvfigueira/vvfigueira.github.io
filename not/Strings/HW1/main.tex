\documentclass[a4paper, 12pt]{article}
\usepackage[a4paper,
    left=2cm,
    right=2cm,
    top=2cm,
    bottom=2cm]{geometry}
\usepackage[font=small,labelfont=bf,
   justification=justified,
   format=plain]{caption}
\makeindex
\usepackage[english]{babel}
\usepackage{amsthm}
\usepackage{graphicx}
\usepackage{setspace}
\usepackage{amsmath}
\usepackage{physics}
\usepackage{amssymb}
\usepackage{hyperref}
\usepackage{cleveref}
\usepackage{float}
\usepackage{mathtools}
\usepackage{enumitem}
\usepackage{slashed}
\usepackage{mathrsfs}
\usepackage[compat=1.1.0]{tikz-feynman}
\usepackage{tensor}
\usepackage{bbm}
\usepackage{simpler-wick}

\newtheoremstyle{dotless}% name
    {}% space above
    {}% space below
    {\itshape}% body font
    {}% indent amount
    {\bfseries}% theorem head font
    {}% punctuation after theorem head
    { }% space after theorem head
    {}% theorem head spec

\theoremstyle{dotless}

\newtheorem{p}{Exercício}%[section]

\title{\textbf{Homework I}}

\author{\textbf{Vicente V. Figueira --- NUSP 11809301}}

\date{\today}

\newcommand{\chap}[1]{
    \let\secstore\thesection
    \let\thesection\mythesection
    \section{#1}
    \let\thesection\secstore
}
\newcommand{\insrt}[4]{
    \begin{figure}[H]
        \centering
        \includegraphics[width=#2\linewidth]{#1}
        \caption{#3}
        \label{#4}
    \end{figure}
}
\newcommand{\letra}[1]{\begin{itemize}
    \item{\textbf{Item (#1):}}
    \end{itemize}
}
\newcommand{\prob}[1]{\begin{p}\label{#1}\end{p}}
\newcommand{\cqd}{$\hfill\blacksquare$}

\AtBeginDocument{\renewcommand*{\hbar}{{\mkern-1mu\mathchar'26\mkern-8mu\textnormal{h}}}}
\AtBeginDocument{\newcommand{\e}{\textnormal{e}}}
\AtBeginDocument{\newcommand{\im}{\textnormal{i}}}
\AtBeginDocument{\newcommand{\luz}{\textnormal{c}}}
\AtBeginDocument{\newcommand{\grav}{\textnormal{G}}}
\AtBeginDocument{\newcommand{\kb}{{\textnormal{k}_{\textnormal{B}}}}}
\newcommand{\Dd}[1]{\mathcal D #1}
\newcommand{\trp}[1]{{#1}^{\textnormal{T}}}
\newcommand{\Det}[1]{\textup{Det} #1}
\newcommand{\sign}[1]{\textnormal{sign} #1}

\numberwithin{equation}{section}

\newtheoremstyle{dotless}% name
    {}% space above
    {}% space below
    {}% body font
    {}% indent amount
    {\bfseries}% theorem head font
    {}% punctuation after theorem head
    { }% space after theorem head
    {}% theorem head spec

\theoremstyle{dotless}

\newtheorem{teo}{Teorema}[section]
\newtheorem{defin}{Definição}[section]
\newtheorem{corl}{Corolário}[teo]
\newtheorem{lemm}{Lema}[defin]
\newtheorem{exem}{Exemplo}[section]
\newtheorem{prov}{Prova:}[section]
\newenvironment{prova}{\paragraph{\nbold{Prova:}}}{\hfill$\blacksquare$\par}
\newcommand{\nbold}[1]{\normalfont{\textbf{#1}}}
\newcommand{\unit}{\mbox{\normalfont{id$_{\vb H}$}}}

%% \doublespacing

\makeatletter
\def\@seccntformat#1{%
    \ifcsname the#1\endcsname
        \ifcsname #1notheorem\endcsname
            \hspace{-1em} % Oculta visualmente no texto
        \else
            \csname the#1\endcsname\quad % Mantém para teoremas
        \fi
    \fi}
\let\sectionnotheorem\relax
\let\subsectionnotheorem\relax

% Modificação para evitar que seções/subseções tenham número no sumário
\let\latex@numberline\numberline
\def\numberline#1{%
    \ifcsname thechapter\endcsname
        \ifnum\value{secnumdepth}>0 % Verifica se numeração está ativada
            \if\relax#1\relax
                % Se for uma seção/subseção, oculta o número no sumário
            \else
                \latex@numberline{#1} % Mantém para teoremas e outros
            \fi
        \fi
    \fi}
\makeatother

\begin{document}

\maketitle

%% \tableofcontents

\problem{}
\probitem{}

For an operator $\mathcal O$ to be BRST closed, it means $\comm{Q_{\textnormal{BRST}}}{\mathcal O}=0$, just remembering the usual BRST transformations\footnote{We're using a graded commutator notation, that is, it's to be interpreted as 
either a commutator or an anti-commutator depending on the statistic of what is inside.},
\begin{align*}
    \comm{Q_{\textnormal{BRST}}}{X^\mu}&=\qty(c\partial+\tilde c\bar\partial)X^\mu\\
    \comm{Q_{\textnormal{BRST}}}{b}&=T^X+T^g\\
    \comm{Q_{\textnormal{BRST}}}{c}&=c\partial c
\end{align*}
with,
\begin{align*}
    T^X&=\frac{1}{\alpha'}\cnord{\partial X^\mu\partial X_\mu},\ \ \ T^g=\cnord{c\partial b}-2\cnord{b\partial c}
\end{align*}
so,
\begin{align*}
    \comm{Q_{\textnormal{BRST}}}{V^a}&=\lambda^a\epsilon_\mu\comm{Q_{\textnormal{BRST}}}{c\partial X^\mu\exp\qty(\im k\cdot X)}\\
    \comm{Q_{\textnormal{BRST}}}{V^a}&=\lambda^a\epsilon_\mu\comm{Q_{\textnormal{BRST}}}{c}\partial X^\mu\exp\qty(\im k\cdot X)-\lambda^a\epsilon_\mu c\comm{Q_{\textnormal{BRST}}}{\partial X^\mu\exp\qty(\im k\cdot X)}\\
    \comm{Q_{\textnormal{BRST}}}{V^a}&=\lambda^a\epsilon_\mu c\partial c\partial X^\mu\exp\qty(\im k\cdot X)\\
    &\quad\quad\quad-\lambda^a\epsilon_\mu c\partial X^\mu\comm{Q_{\textnormal{BRST}}}{\exp\qty(\im k\cdot X)}-\lambda^a\epsilon_\mu c\comm{Q_{\textnormal{BRST}}}{\partial X^\mu}\exp\qty(\im k\cdot X)\\
    \comm{Q_{\textnormal{BRST}}}{V^a}&=\lambda^a\epsilon_\mu c\partial c\partial X^\mu\exp\qty(\im k\cdot X)\\
    &\quad\quad\quad-\lambda^a\epsilon_\mu c\partial X^\mu\comm{Q_{\textnormal{BRST}}}{\exp\qty(\im k\cdot X)}-\lambda^a\epsilon_\mu c\partial\comm{Q_{\textnormal{BRST}}}{ X^\mu}\exp\qty(\im k\cdot X)\\
    \comm{Q_{\textnormal{BRST}}}{V^a}&=\lambda^a\epsilon_\mu c\partial c\partial X^\mu\exp\qty(\im k\cdot X)\\
    &\quad\quad\quad-\lambda^a\epsilon_\mu c\partial X^\mu\comm{Q_{\textnormal{BRST}}}{\exp\qty(\im k\cdot X)}-\lambda^a\epsilon_\mu c\partial\qty(c\partial+\tilde c\bar\partial)X^\mu\exp\qty(\im k\cdot X)\\
    \comm{Q_{\textnormal{BRST}}}{V^a}&=\lambda^a\epsilon_\mu c\partial c\partial X^\mu\exp\qty(\im k\cdot X)\\
    &\quad\quad\quad-\lambda^a\epsilon_\mu c\partial X^\mu\comm{Q_{\textnormal{BRST}}}{\exp\qty(\im k\cdot X)}-\lambda^a\epsilon_\mu c\partial\qty(c\partial+\tilde c\bar\partial)X^\mu\exp\qty(\im k\cdot X)\\
    &\quad\quad\quad-\lambda^a\epsilon_\mu cc\partial-\lambda^a\epsilon_\mu c\partial\qty(\tilde c\bar\partial)X^\mu\exp\qty(\im k\cdot X)
\end{align*}
\probitem{}
\probitem{}
\probitem{}

\end{document}