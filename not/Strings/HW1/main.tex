\documentclass[a4paper, 12pt]{article}
\usepackage[a4paper,
    left=2cm,
    right=2cm,
    top=2cm,
    bottom=2cm]{geometry}
\usepackage[font=small,labelfont=bf,
   justification=justified,
   format=plain]{caption}
\makeindex
\usepackage[english]{babel}
\usepackage{amsthm}
\usepackage{graphicx}
\usepackage{setspace}
\usepackage{amsmath}
\usepackage{physics}
\usepackage{amssymb}
\usepackage{hyperref}
\usepackage{cleveref}
\usepackage{float}
\usepackage{mathtools}
\usepackage{enumitem}
\usepackage{slashed}
\usepackage{mathrsfs}
\usepackage[compat=1.1.0]{tikz-feynman}
\usepackage{tensor}
\usepackage{bbm}
\usepackage{simpler-wick}
\usepackage[compact,explicit]{titlesec}


\newtheoremstyle{dotless}% name
    {}% space above
    {}% space below
    {\itshape}% body font
    {}% indent amount
    {\bfseries}% theorem head font
    {}% punctuation after theorem head
    { }% space after theorem head
    {}% theorem head spec

\theoremstyle{dotless}

\newtheorem{p}{Exercício}%[section]

\title{\textbf{Homework I}}

\author{\textbf{Vicente V. Figueira --- NUSP 11809301}}

\date{\today}

\newcommand{\chap}[1]{
    \let\secstore\thesection
    \let\thesection\mythesection
    \section{#1}
    \let\thesection\secstore
}
\newcommand{\insrt}[4]{
    \begin{figure}[H]
        \centering
        \includegraphics[width=#2\linewidth]{#1}
        \caption{#3}
        \label{#4}
    \end{figure}
}
\newcommand{\letra}[1]{\begin{itemize}
    \item{\textbf{Item (#1):}}
    \end{itemize}
}
\newcommand{\prob}[1]{\begin{p}\label{#1}\end{p}}
\newcommand{\cqd}{$\hfill\blacksquare$}

\AtBeginDocument{\renewcommand*{\hbar}{{\mkern-1mu\mathchar'26\mkern-8mu\textnormal{h}}}}
\AtBeginDocument{\newcommand{\e}{\textnormal{e}}}
\AtBeginDocument{\newcommand{\im}{\textnormal{i}}}
\AtBeginDocument{\newcommand{\luz}{\textnormal{c}}}
\AtBeginDocument{\newcommand{\grav}{\textnormal{G}}}
\AtBeginDocument{\newcommand{\kb}{{\textnormal{k}_{\textnormal{B}}}}}
\newcommand{\Dd}[1]{\mathcal D #1}
\newcommand{\trp}[1]{{#1}^{\textnormal{T}}}
\newcommand{\Det}[1]{\textup{Det} #1}
\newcommand{\sign}[1]{\textnormal{sign} #1}
\newcommand{\sidev}{{\raisebox{-0.5ex}{$'$}}}

\newcommand\numberthis{\addtocounter{equation}{1}\tag{\theequation}}

\numberwithin{equation}{section}

\newtheoremstyle{dotless}% name
    {}% space above
    {}% space below
    {}% body font
    {}% indent amount
    {\bfseries}% theorem head font
    {}% punctuation after theorem head
    { }% space after theorem head
    {}% theorem head spec

\theoremstyle{dotless}

\newtheorem{teo}{Teorema}[section]
\newtheorem{defin}{Definição}[section]
\newtheorem{corl}{Corolário}[teo]
\newtheorem{lemm}{Lema}[defin]
\newtheorem{exem}{Exemplo}[section]
\newtheorem{prov}{Prova:}[section]
\newenvironment{prova}{\paragraph{\nbold{Prova:}}}{\hfill$\blacksquare$\par}
\newcommand{\nbold}[1]{\normalfont{\textbf{#1}}}
\newcommand{\unit}{\mbox{\normalfont{id$_{\vb H}$}}}

%% \doublespacing

%\titleformat{\section}[runin]{\large\bfseries}{}{0pt}{Problem \thesection}

\newcommand{\problem}{%
  \refstepcounter{section}%
  \addcontentsline{toc}{section}{Problem \thesection}%
  \vspace{1.5em}
  \noindent
  {\large\bfseries Problem \thesection}%
  \par\nopagebreak\vspace{0.5em}
}


\renewcommand{\thesubsection}{\thesection.\Alph{subsection})}

\allowdisplaybreaks

\begin{document}

\maketitle

\tableofcontents

\problem{}
\probitem{}

For an operator $\mathcal O$ to be BRST closed, it means $\comm{Q_{\textnormal{BRST}}}{\mathcal O}=0$, just remembering the usual BRST transformations\footnote{We're using a graded commutator notation, that is, it's to be interpreted as 
either a commutator or an anti-commutator depending on the statistic of what is inside.},
\begin{align*}
    \comm{Q_{\textnormal{BRST}}}{X^\mu}&=\qty(c\partial+\tilde c\bar\partial)X^\mu\\
    \comm{Q_{\textnormal{BRST}}}{b}&=T^X+T^g\\
    \comm{Q_{\textnormal{BRST}}}{c}&=c\partial c
\end{align*}
with,
\begin{align*}
    T^X&=\frac{1}{\alpha'}\cnord{\partial X^\mu\partial X_\mu},\ \ \ T^g=\cnord{c\partial b}-2\cnord{b\partial c}
\end{align*}
so,
\begin{align*}
    \comm{Q_{\textnormal{BRST}}}{V^a}&=\lambda^a\epsilon_\mu\comm{Q_{\textnormal{BRST}}}{c\partial X^\mu\exp\qty(\im k\cdot X)}\\
    \comm{Q_{\textnormal{BRST}}}{V^a}&=\lambda^a\epsilon_\mu\comm{Q_{\textnormal{BRST}}}{c}\partial X^\mu\exp\qty(\im k\cdot X)-\lambda^a\epsilon_\mu c\comm{Q_{\textnormal{BRST}}}{\partial X^\mu\exp\qty(\im k\cdot X)}\\
    \comm{Q_{\textnormal{BRST}}}{V^a}&=\lambda^a\epsilon_\mu c\partial c\partial X^\mu\exp\qty(\im k\cdot X)\\
    &\quad\quad\quad-\lambda^a\epsilon_\mu c\partial X^\mu\comm{Q_{\textnormal{BRST}}}{\exp\qty(\im k\cdot X)}-\lambda^a\epsilon_\mu c\comm{Q_{\textnormal{BRST}}}{\partial X^\mu}\exp\qty(\im k\cdot X)\\
    \comm{Q_{\textnormal{BRST}}}{V^a}&=\lambda^a\epsilon_\mu c\partial c\partial X^\mu\exp\qty(\im k\cdot X)\\
    &\quad\quad\quad-\lambda^a\epsilon_\mu c\partial X^\mu\comm{Q_{\textnormal{BRST}}}{\exp\qty(\im k\cdot X)}-\lambda^a\epsilon_\mu c\partial\comm{Q_{\textnormal{BRST}}}{ X^\mu}\exp\qty(\im k\cdot X)\\
    \comm{Q_{\textnormal{BRST}}}{V^a}&=\lambda^a\epsilon_\mu c\partial c\partial X^\mu\exp\qty(\im k\cdot X)\\
    &\quad\quad\quad-\lambda^a\epsilon_\mu c\partial X^\mu\comm{Q_{\textnormal{BRST}}}{\exp\qty(\im k\cdot X)}-\lambda^a\epsilon_\mu c\partial\qty(c\partial+\tilde c\bar\partial)X^\mu\exp\qty(\im k\cdot X)\\
    \comm{Q_{\textnormal{BRST}}}{V^a}&=\lambda^a\epsilon_\mu c\partial c\partial X^\mu\exp\qty(\im k\cdot X)\\
    &\quad\quad\quad-\lambda^a\epsilon_\mu c\partial X^\mu\comm{Q_{\textnormal{BRST}}}{\exp\qty(\im k\cdot X)}-\lambda^a\epsilon_\mu c\partial\qty(c\partial+\tilde c\bar\partial)X^\mu\exp\qty(\im k\cdot X)\\
    &\quad\quad\quad-\lambda^a\epsilon_\mu cc\partial-\lambda^a\epsilon_\mu c\partial\qty(\tilde c\bar\partial)X^\mu\exp\qty(\im k\cdot X)
\end{align*}
\probitem{}
\probitem{}
\probitem{}

\newpage

\section{Problem 2}
\subsection{2.A)}

The Polyakov Action is given by,

\begin{align*}
    S_{\textnormal{P}}&=-\frac{1}{4\pi\alpha'}\int\dd[2]{\sigma}\sqrt{h}h^{ab}g_{\mu\nu}\partial_a X^\mu \partial_b X^\nu
\end{align*}

With $h^{ab}$ being the world-sheet metric, $h=\norm{\Det\qty[h_{ab}]}$, and $g_{\mu\nu}=\textnormal{diag}\qty(-1,1,\cdots,1)$ the 
target space metric. A Poincare transformation of the fields $X$ is,

\begin{align*}
    X^\mu\qty(\tau,\sigma)&\rightarrow\tilde X^\mu\qty(\tau,\sigma)=\tensor{\Lambda}{^\mu_\nu}X^\nu\qty(\tau,\sigma)+a^\mu\\
    \partial_a X^\mu&\rightarrow\partial_a\tilde X^\mu=\tensor{\Lambda}{^\mu_\nu}\partial_a X^\nu
\end{align*}

With of course $\Lambda$ satisfying the defining property of a Lorentz transformation,

\begin{align*}
    g_{\mu\nu}\tensor{\Lambda}{^\mu_\alpha}\tensor{\Lambda}{^\nu_\beta}=g_{\alpha\beta}
\end{align*}

The transformed Action is,

\begin{align*}
    \tilde S_{\textnormal{P}}&=-\frac{1}{4\pi\alpha'}\int\dd[2]{\sigma}\sqrt{h}h^{ab}g_{\mu\nu}\partial_a \tilde X^\mu \partial_b \tilde X^\nu\\
    \tilde S_{\textnormal{P}}&=-\frac{1}{4\pi\alpha'}\int\dd[2]{\sigma}\sqrt{h}h^{ab}g_{\mu\nu}\tensor{\Lambda}{^\mu_\alpha}\tensor{\Lambda}{^\nu_\beta}\partial_a  X^\alpha \partial_b X^\beta\\
    \tilde S_{\textnormal{P}}&=-\frac{1}{4\pi\alpha'}\int\dd[2]{\sigma}\sqrt{h}h^{ab}g_{\alpha\beta}\partial_a  X^\alpha \partial_b X^\beta=S_{\textnormal{P}}
\end{align*}

Hence the Poincare group is indeed a global symmetry of the Action. To obtain the conserved currents we have to first know what are the 
equations of motion,

\begin{align*}
    \delta S_{\textnormal{P}}&=-\frac{1}{4\pi\alpha'}\int\dd[2]{\sigma}\qty{\sqrt{h}h^{ab}g_{\mu\nu}2\partial_aX^\mu\delta\partial_b X^\nu+\sqrt{h}\delta h^{ab}g_{\mu\nu}\partial_aX^\mu\partial_b X^\nu-\frac12\sqrt{h}h_{ab}\delta h^{ab}h^{cd}g_{\mu\nu}\partial_cX^\mu\partial_d X^\nu}\\
    \delta S_{\textnormal{P}}&=-\frac{1}{4\pi\alpha'}\int\dd[2]{\sigma}\qty{\sqrt{h}h^{ab}2\partial_aX^\mu\delta\partial_b X_\mu+\sqrt{h}\delta h^{ab}\partial_aX^\mu\partial_b X_\mu-\frac12\sqrt{h}h_{ab}\delta h^{ab}\partial_cX^\mu\partial^c X_\mu}\\
    \delta S_{\textnormal{P}}&=-\frac{1}{2\pi\alpha'}\int\dd[2]{\sigma}\partial_b\qty[\sqrt{h}\partial^bX^\mu\delta X_\mu]+\frac{1}{2\pi\alpha'}\int\dd[2]{\sigma}\partial_b\qty[\sqrt{h}\partial^bX^\mu]\delta X_\mu\\
    &\quad\quad\quad-\frac{1}{4\pi\alpha'}\int\dd[2]{\sigma}\delta h^{ab}\qty[\sqrt{h}\partial_aX^\mu\partial_b X_\mu-\frac12\sqrt{h}h_{ab}\partial_cX^\mu\partial^c X_\mu]
\end{align*}

Each of the three terms has to vanish independently, the first of them is just a Boundary Condition,

\begin{align*}
    \int\dd[2]{\sigma}\partial_b\qty[\sqrt{h}\partial^bX^\mu\delta X_\mu]=0\numberthis\label{boundaryc}
\end{align*}

The second gives the equations for $X$,

\begin{align*}
    \partial_a\qty[\sqrt{h}\partial^aX^\mu]=0\numberthis\label{eomx}
\end{align*}

And the last one give the equations for $h$,

\begin{align*}
    \sqrt{h}\partial_aX^\mu\partial_bX_\mu-\frac12\sqrt{h}h_{ab}\partial_cX^\mu\partial^cX_\mu=0\numberthis\label{eomh}
\end{align*}

Armed with these, we can consider now just a variation on $X$, which is a symmetry of the Action, in our case this will be a Poincare transformation,

\begin{align*}
    \delta S_{\textnormal{P}}&=-\frac{1}{2\pi\alpha'}\int\dd[2]{\sigma}\sqrt{h}\partial^bX^\mu\partial_b \delta X_\mu\\
    \delta S_{\textnormal{P}}&=-\frac{1}{2\pi\alpha'}\int\dd[2]{\sigma}\qty{\partial_b\qty[\sqrt{h}\partial^bX^\mu\delta X_\mu]-\partial_b\qty[\sqrt{h}\partial^bX^\mu]\delta X_\mu}
\end{align*}

Imposing the fields to obey the equations of motion, \ref{eomx}, the second term vanishes identically. And also using our 
already derived result that $\tilde S_{\textnormal{P}}=S_{\textnormal{P}}\Rightarrow\delta S_{\textnormal{P}}=0$, we get the simple 
expression, for $\delta X$ being the variation under a Poincare transformation,

\begin{align*}
    \delta S_{\textnormal{P}}&=-\frac{1}{2\pi\alpha'}\int\dd[2]{\sigma}\partial_b\qty[\sqrt{h}\partial^bX^\mu\delta X_\mu]=0
\end{align*}

In the case of a pure translation, $\delta X=\tilde X-X=a$,

\begin{align*}
    -\frac{1}{2\pi\alpha'}a_\mu\int\dd[2]{\sigma}\partial_b\qty[\sqrt{h}\partial^bX^\mu]=0
\end{align*}

From where we can read the conserved current associated with translations,

\begin{align*}
    \mathcal P^{b\mu}&=-\frac{\sqrt{h}}{2\pi\alpha'}\partial^bX^\mu,\ \ \ \partial_b\mathcal P^{b\mu}=0\numberthis\label{defcalp}
\end{align*}

We can do the same for a Lorentz transformation, $\delta X^\mu=\tilde X^\mu-X^\mu=\tensor{\omega}{^\mu_\nu}X_\nu$, with of course $\omega_{\mu\nu}=-\omega_{\nu\mu}$, 
being the infinitesimal part of the Lorentz transformation, $\Lambda=\mathbbm 1+\omega$,

\begin{align*}
    -\frac{1}{2\pi\alpha'}\int\dd[2]{\sigma}\partial_b\qty[\sqrt{h}\partial^bX^\mu\omega_{\mu\nu}X^\nu]&=0\\
    -\frac{1}{2\pi\alpha'}\omega_{\mu\nu}\int\dd[2]{\sigma}\partial_b\qty[\sqrt{h}\partial^bX^\mu X^\nu-\sqrt{h}\partial^bX^\nu X^\mu]&=0
\end{align*}

So that the conserved current associated with Lorentz transformations is,

\begin{align*}
    \mathcal M^{b\mu\nu}=-\frac{\sqrt{h}}{2\pi\alpha'}\qty[X^\mu\partial^b X^\nu - X^\nu\partial^b X^\mu]=X^\mu\mathcal P^{b\nu}-X^\nu\mathcal P^{b\mu},\ \ \ \partial_b\mathcal M^{b\mu\nu}=0
\end{align*}

Lastly, the conserved charges that follow from the conserved currents are,

\begin{align*}
    P^\mu&=\int\dd{\sigma}\mathcal P^{\tau\mu}=-\frac{1}{2\pi\alpha'}\int\dd{\sigma}\sqrt{h}\partial^\tau X^\mu\numberthis\label{p}\\
    M^{\mu\nu}&=\int\dd{\sigma}\mathcal M^{\tau\mu\nu}=-\frac{1}{2\pi\alpha'}\int\dd{\sigma}\sqrt{h}\qty[X^\mu\partial^\tau X^\nu -X^\nu\partial^\tau X^\mu ]\numberthis\label{m}
\end{align*}

\subsection{2.B)}

We now turn to the matter of verifying that the conserved charges derived here do obey the Poincare algebra, for 
this we'll need the Poisson Brackets, which are defined with respect to $X^\mu\qty(t,\sigma)$, and it's conjugate 
momentum $\pdv{\mathcal L}{\partial_\tau X^\mu}=\tensor{{\mathcal P}}{^\tau_\mu}\equiv\Pi_\mu\qty(\tau,\sigma)$, which 
fortunately we already computed. The metric $h$ does not enter in the Poisson Brackets because it's not dynamical, 
it has three degrees of freedom, but we also have three gauge redundancies, just enough to make it non-dynamical. 
The fundamental Poisson Bracket relations are,

\begin{align*}
    \acomm{X^\mu\qty(\tau,\sigma)}{X^\nu\qty(\tau,\sigma')}=0,\ \ \ \acomm{\Pi^\mu\qty(\tau,\sigma)}{\Pi^\nu\qty(\tau,\sigma')}=0,\ \ \ \acomm{X^\mu\qty(\tau,\sigma)}{\Pi^\nu\qty(\tau,\sigma')}=\delta\qty(\sigma-\sigma')g^{\mu\nu}
\end{align*}

Just for completeness, we'll rewrite \ref{p},\ref{m} in function of the canonical variables,

\begin{align*}
    P^\mu&=\int\dd{\sigma}\Pi^\mu\\
    M^{\mu\nu}&=\int\dd{\sigma}\qty[X^\mu\Pi^\nu -X^\nu\Pi^\mu ]
\end{align*}

We'll start by the $P-P$ --- we'll not keep track of the $\tau$ dependence in the conserved charges, because, they are conserved. But 
nevertheless, everything is assumed to be evaluated at equal $\tau$ ---,

\begin{align*}
    \acomm{P^\mu}{P^\nu}=\int\dd{\sigma}\dd{\sigma'}\acomm{\Pi^\mu\qty(\tau,\sigma)}{\Pi^\nu\qty(\tau,\sigma')}=0
\end{align*}

Next the $P-M$,

\begin{align*}
    \acomm{P^\mu}{M^{\alpha\beta}}&=\int\dd{\sigma}\dd{\sigma'}\acomm{\Pi^\mu\qty(\tau,\sigma)}{X^\alpha\qty(\tau,\sigma')\Pi^\beta\qty(\tau,\sigma')-X^\beta\qty(\tau,\sigma')\Pi^\alpha\qty(\tau,\sigma')}\\
    &=\int\dd{\sigma}\dd{\sigma'}\qty[X^\alpha\qty(\tau,\sigma')\acomm{\Pi^\mu\qty(\tau,\sigma)}{\Pi^\beta\qty(\tau,\sigma')}+\acomm{\Pi^\mu\qty(\tau,\sigma)}{X^\alpha\qty(\tau,\sigma')}\Pi^\beta\qty(\tau,\sigma')-\qty(\alpha\leftrightarrow\beta)]\\
    &=\int\dd{\sigma}\dd{\sigma'}\qty[-\Pi^\beta\qty(\tau,\sigma')g^{\mu\alpha}\delta\qty(\sigma-\sigma')-\qty(\alpha\leftrightarrow\beta)]\\
    &=\int\dd{\sigma}\dd{\sigma'}\qty[-\Pi^\beta\qty(\tau,\sigma')g^{\mu\alpha}\delta\qty(\sigma-\sigma')+\Pi^\alpha\qty(\tau,\sigma')g^{\mu\beta}\delta\qty(\sigma-\sigma')]\\
    &=\int\dd{\sigma}\qty[\Pi^\alpha\qty(\tau,\sigma)g^{\mu\beta}-\Pi^\beta\qty(\tau,\sigma)g^{\mu\alpha}]\\
    \acomm{P^\mu}{M^{\alpha\beta}}&=g^{\mu\beta}P^\alpha -g^{\mu\alpha}P^\beta 
\end{align*}

And lastly, the $M-M$,

\begin{align*}
    \acomm{M^{\mu\nu}}{M^{\alpha\beta}}&=\int\dd{\sigma}\dd{\sigma'}\acomm{X^\mu\qty(\tau,\sigma)\Pi^\nu\qty(\tau,\sigma)-X^\nu\qty(\tau,\sigma)\Pi^\mu\qty(\tau,\sigma)}{X^\alpha\qty(\tau,\sigma')\Pi^\beta\qty(\tau,\sigma')-X^\beta\qty(\tau,\sigma')\Pi^\alpha\qty(\tau,\sigma')}\\
    &=\int\dd{\sigma}\dd{\sigma'}\acomm{X^\mu\qty(\tau,\sigma)\Pi^\nu\qty(\tau,\sigma)-X^\nu\qty(\tau,\sigma)\Pi^\mu\qty(\tau,\sigma)}{X^\alpha\qty(\tau,\sigma')\Pi^\beta\qty(\tau,\sigma')}-\qty(\alpha\leftrightarrow\beta)\\
    &=\qty[\int\dd{\sigma}\dd{\sigma'}\acomm{X^\mu\qty(\tau,\sigma)\Pi^\nu\qty(\tau,\sigma)}{X^\alpha\qty(\tau,\sigma')\Pi^\beta\qty(\tau,\sigma')}-\qty(\alpha\leftrightarrow\beta)]-\qty(\mu\leftrightarrow\nu)
\end{align*}

Notice that,

\begin{align*}
    &\acomm{X^\mu\qty(\tau,\sigma)\Pi^\nu\qty(\tau,\sigma)}{X^\alpha\qty(\tau,\sigma')\Pi^\beta\qty(\tau,\sigma')}\\
    &=\acomm{X^\mu\qty(\tau,\sigma)}{X^\alpha\qty(\tau,\sigma')\Pi^\beta\qty(\tau,\sigma')}\Pi^\nu\qty(\tau,\sigma)+X^\mu\qty(\tau,\sigma)\acomm{\Pi^\nu\qty(\tau,\sigma)}{X^\alpha\qty(\tau,\sigma')\Pi^\beta\qty(\tau,\sigma')}\\
    &=X^\alpha\qty(\tau,\sigma')\acomm{X^\mu\qty(\tau,\sigma)}{\Pi^\beta\qty(\tau,\sigma')}\Pi^\nu\qty(\tau,\sigma)+X^\mu\qty(\tau,\sigma)\acomm{\Pi^\nu\qty(\tau,\sigma)}{X^\alpha\qty(\tau,\sigma')}\Pi^\beta\qty(\tau,\sigma')\\
    &=X^\alpha\qty(\tau,\sigma')g^{\mu\beta}\delta\qty(\sigma-\sigma')\Pi^\nu\qty(\tau,\sigma)-X^\mu\qty(\tau,\sigma)g^{\nu\alpha}\delta\qty(\sigma-\sigma')\Pi^\beta\qty(\tau,\sigma')
\end{align*}

Using this back in our expression,

\begin{align*}
    \acomm{M^{\mu\nu}}{M^{\alpha\beta}}&=\qty[\int\dd{\sigma}X^\alpha\qty(\tau,\sigma)g^{\mu\beta}\Pi^\nu\qty(\tau,\sigma)-\Pi^\mu\qty(\tau,\sigma)g^{\nu\alpha}\Pi^\beta\qty(\tau,\sigma)-\qty(\alpha\leftrightarrow\beta)]-\qty(\mu\leftrightarrow\nu)\\
    &=\int\dd{\sigma}\qty[X^\alpha\qty(\tau,\sigma)g^{\mu\beta}\Pi^\nu\qty(\tau,\sigma)-X^\mu\qty(\tau,\sigma)g^{\nu\alpha}\Pi^\beta\qty(\tau,\sigma)]\\
    &\quad\quad\quad-\int\dd{\sigma}\qty[X^\beta\qty(\tau,\sigma)g^{\mu\alpha}\Pi^\nu\qty(\tau,\sigma)-X^\mu\qty(\tau,\sigma)g^{\nu\beta}\Pi^\alpha\qty(\tau,\sigma)]-\qty(\mu\leftrightarrow\nu)\\
    &=\int\dd{\sigma}\qty[X^\alpha\qty(\tau,\sigma)g^{\mu\beta}\Pi^\nu\qty(\tau,\sigma)-X^\mu\qty(\tau,\sigma)g^{\nu\alpha}\Pi^\beta\qty(\tau,\sigma)]\\
    &\quad\quad\quad-\int\dd{\sigma}\qty[X^\beta\qty(\tau,\sigma)g^{\mu\alpha}\Pi^\nu\qty(\tau,\sigma)-X^\mu\qty(\tau,\sigma)g^{\nu\beta}\Pi^\alpha\qty(\tau,\sigma)]\\
    &\quad\quad\quad-\int\dd{\sigma}\qty[X^\alpha\qty(\tau,\sigma)g^{\nu\beta}\Pi^\mu\qty(\tau,\sigma)-X^\nu\qty(\tau,\sigma)g^{\mu\alpha}\Pi^\beta\qty(\tau,\sigma)]\\
    &\quad\quad\quad+\int\dd{\sigma}\qty[X^\beta\qty(\tau,\sigma)g^{\nu\alpha}\Pi^\mu\qty(\tau,\sigma)-X^\nu\qty(\tau,\sigma)g^{\mu\beta}\Pi^\alpha\qty(\tau,\sigma)]
\end{align*}

Collecting the terms with same metric index,

\begin{align*}
    \acomm{M^{\mu\nu}}{M^{\alpha\beta}}&=g^{\mu\beta}\int\dd{\sigma}\qty[X^\alpha\qty(\tau,\sigma)\Pi^\nu\qty(\tau,\sigma)-X^\nu\qty(\tau,\sigma)\Pi^\alpha\qty(\tau,\sigma)]\\
    &\quad\quad\quad+g^{\nu\beta}\int\dd{\sigma}\qty[X^\mu\qty(\tau,\sigma)\Pi^\alpha\qty(\tau,\sigma)-X^\alpha\qty(\tau,\sigma)\Pi^\mu\qty(\tau,\sigma)]\\
    &\quad\quad\quad+g^{\mu\alpha}\int\dd{\sigma}\qty[X^\nu\qty(\tau,\sigma)\Pi^\beta\qty(\tau,\sigma)-X^\beta\qty(\tau,\sigma)\Pi^\nu\qty(\tau,\sigma)]\\
    &\quad\quad\quad+g^{\nu\alpha}\int\dd{\sigma}\qty[X^\beta\qty(\tau,\sigma)\Pi^\mu\qty(\tau,\sigma)-X^\mu\qty(\tau,\sigma)\Pi^\beta\qty(\tau,\sigma)]\\
    \acomm{M^{\mu\nu}}{M^{\alpha\beta}}&=g^{\mu\beta}M^{\alpha\nu}+g^{\nu\beta}M^{\mu\alpha}+g^{\mu\alpha}M^{\nu\beta}+g^{\nu\alpha}M^{\beta\mu}\\
    \acomm{M^{\mu\nu}}{M^{\alpha\beta}}&=g^{\mu\alpha}M^{\nu\beta}-g^{\mu\beta}M^{\nu\alpha}+g^{\nu\beta}M^{\mu\alpha}-g^{\nu\alpha}M^{\mu\beta}
\end{align*}

Summarizing,

\begin{align*}
    \acomm{P^\mu}{P^\nu}&=0\\
    \acomm{P^\mu}{M^{\alpha\beta}}&=g^{\mu\beta}P^\alpha -g^{\mu\alpha}P^\beta\\
    \acomm{M^{\mu\nu}}{M^{\alpha\beta}}&=g^{\mu\alpha}M^{\nu\beta}-g^{\mu\beta}M^{\nu\alpha}+g^{\nu\beta}M^{\mu\alpha}-g^{\nu\alpha}M^{\mu\beta}\numberthis\label{lorentzalgebra}
\end{align*}

Which is exactly the algebra of the Poincare Group!

\newpage

\section{Problem 3}
\subsection{3.A)}

The Gamma Function can be represented in the complex plane domain, $\textnormal{Re}\qty(s)>1$, as the following integral,

\begin{align*}
    \Gamma(s)&=\int\limits_0^\infty\dd{t}\exp\qty(-t)t^{s-1},\ \ \ \textnormal{Re}\qty(s)>1
\end{align*}

Which is also the subset of the complex plane in which this integral converges, of course this representation of the Gamma Function 
in a open set is sufficient for obtain an analytical continuation to the whole complex plane. Obviously, the integral is invariant under 
relabeling the dummy variable $t$, we make the following choice $t\rightarrow nt$ --- Assuming $n>0$ ---,

\begin{align*}
    \Gamma(s)&=\int\limits_0^\infty\dd{\qty(nt)}\exp\qty(-nt)\qty(nt)^{s-1},\ \ \ \textnormal{Re}\qty(s)>1\\
    \Gamma(s)&=n^s\int\limits_0^\infty\dd{t}\exp\qty(-nt)t^{s-1},\ \ \ \textnormal{Re}\qty(s)>1\\
    n^{-s}\Gamma\qty(s)&=\int\limits_0^\infty\dd{t}\exp\qty(-nt)t^{s-1},\ \ \ \textnormal{Re}\qty(s)>1\\
    \sum\limits_{n=1}^\infty n^{-s}\Gamma\qty(s)&=\sum\limits_{n=1}^\infty\int\limits_0^\infty\dd{t}\exp\qty(-nt)t^{s-1},\ \ \ \textnormal{Re}\qty(s)>1
\end{align*}    

The sum in the left-hand side is recognized as the representation for the Zeta Function in the domain $\textnormal{Re}\qty(s)>1$, which is also 
the domain of convergence of the sum,

\begin{align*}
    \zeta(s)\Gamma\qty(s)&=\sum\limits_{n=1}^\infty\int\limits_0^\infty\dd{t}\exp\qty(-nt)t^{s-1},\ \ \ \textnormal{Re}\qty(s)>1\\
\end{align*}

About the right-hand side, to be able to exchange the integral and the sum is sufficient that,

\begin{align*}
    &\int\limits_0^\infty\dd{t}\sum\limits_{n=1}^\infty\norm{\exp\qty(-nt)t^{s-1}}<\infty,\ \ \ \textnormal{Re}\qty(s)>1\\
    &\int\limits_0^\infty\dd{t}\sum\limits_{n=1}^\infty\exp\qty(-nt)\norm{t^{s-1}}<\infty,\ \ \ \textnormal{Re}\qty(s)>1\\
    &\int\limits_0^\infty\dd{t}\sum\limits_{n=1}^\infty\exp\qty(-nt)t^{\textnormal{Re}\qty(s)-1}<\infty,\ \ \ \textnormal{Re}\qty(s)>1
\end{align*}

The sum now is a simple geometric series, giving,

\begin{align*}
    &\int\limits_0^\infty\dd{t}\frac{t^{\textnormal{Re}\qty(s)-1}}{\exp\qty(t)-1}<\infty,\ \ \ \textnormal{Re}\qty(s)>1
\end{align*}

The dangerous behavior that could make the integral diverges is the one at $t\rightarrow0$, thankfully, as $\textnormal{Re}\qty(s)>1$, the 
possible zero in the denominator is cancelled by the stronger zero in the numerator, hence, the integral converges. That is, 
is permissible to do the exchange of the sum and integral, and we get,

\begin{align*}
    \zeta(s)\Gamma\qty(s)&=\int\limits_0^\infty\dd{t}\sum\limits_{n=1}^\infty\exp\qty(-nt)t^{s-1},\ \ \ \textnormal{Re}\qty(s)>1\\
    \zeta(s)\Gamma\qty(s)&=\int\limits_0^\infty\dd{t}\frac{t^{s-1}}{\exp\qty(t)-1},\ \ \ \textnormal{Re}\qty(s)>1
\end{align*}

\subsection{3.B)}

The objective here is to make an analytical continuation to $\textnormal{Re}\qty(s)>-2$ of the expression found in the later item. 
First of all, the reason the later expression is only well defined in $\textnormal{Re}\qty(s)>1$, is due to the divergence of the 
integrand at $t\rightarrow 0$ for $\textnormal{Re}\qty(s)\leq1$

\begin{align*}
    \zeta(s)\Gamma\qty(s)&=\int\limits_0^1\dd{t}\frac{t^{s-1}}{\exp\qty(t)-1}+\int\limits_1^\infty\dd{t}\frac{t^{s-1}}{\exp\qty(t)-1},\ \ \ \textnormal{Re}\qty(s)>1\\
    \zeta(s)\Gamma\qty(s)&=\int\limits_0^1\dd{t}t^{s-1}\qty[\frac{1}{\exp\qty(t)-1}-\frac1t]+\int\limits_1^\infty\dd{t}\frac{t^{s-1}}{\exp\qty(t)-1},\ \ \ \textnormal{Re}\qty(s)>1\\
\end{align*}

\subsection{3.C)}

\newpage

\problem{}
\probitem{}

Just restating the hypothesis,

\begin{align*}
    X\qty(z)X\qty(w)&=-\ln\qty(z-w)+\textnormal{regular}\\
    T\qty(z)&=-\frac12\cnord{\partial X\partial X}\qty(z)\\
    V_n\qty(z)&=\cnord{\exp\qty(\im n X)}\qty(z,\bar z)
\end{align*}

To compute the $TV_n$ OPE, we start by computing the normal ordered product,

\begin{align*}
    \cnord{T\qty(z)V_n\qty(w)}&=T\qty(z)V_n\qty(w)-\frac12\cnord{\wick{(\partial \c1X\partial X)\qty(z)\exp(in \c1X)\qty(w)}}\\
    &\quad\quad\quad-\frac12\cnord{\wick{(\partial X\partial \c1X)\qty(z)\exp(in \c1X)\qty(w)}}-\frac12\cnord{\wick{(\partial \c1X\partial\c2 X)\qty(z)\exp(in \c1 X\mkern-4mu \c2 {\mathclap{\phantom{X}}})\qty(w)}}\\
    \cnord{T\qty(z)V_n\qty(w)}&=T\qty(z)V_n\qty(w)-\frac{\im n}{2}\partial_z\qty(\ln\qty(z-w))\cnord{\partial X\qty(z)\exp(in X)\qty(w)}\\
    &\quad\quad\quad-\frac{\im n}{2}\partial_z\qty(\ln\qty(z-w))\cnord{\partial X\exp(in X)\qty(w)}\\
    &\quad\quad\quad+\frac{\im n \im n}{2}\partial_z\qty(\ln\qty(z-w))\partial_z\qty(\ln\qty(z-w))\cnord{\exp(in X)\qty(w)}\\
    \cnord{T\qty(z)V_n\qty(w)}&=T\qty(z)V_n\qty(w)-\frac{\im n}{2\qty(z-w)}\cnord{\partial X\qty(z)\exp(in X)\qty(w)}\\
    &\quad\quad\quad-\frac{\im n}{2\qty(z-w)}\cnord{\partial X\qty(z)\exp(in X)\qty(w)}\\
    &\quad\quad\quad-\frac{n^2}{2\qty(z-w)^2}\cnord{\exp(in X)\qty(w)}\\
    \cnord{T\qty(z)V_n\qty(w)}&=T\qty(z)V_n\qty(w)-\frac{\im n}{\qty(z-w)}\cnord{\partial X\qty(z)\exp(in X)\qty(w)}-\frac{n^2}{2\qty(z-w)^2}\cnord{\exp(in X)}\qty(w)\\
    \cnord{T\qty(z)V_n\qty(w)}&=T\qty(z)V_n\qty(w)-\frac{\im n}{\qty(z-w)}\cnord{\partial X\exp(in X)}\qty(w)-\frac{n^2V_n\qty(w)}{2\qty(z-w)^2}+\textnormal{regular}\\
    \cnord{T\qty(z)V_n\qty(w)}&=T\qty(z)V_n\qty(w)-\frac{1}{\qty(z-w)}\partial\cnord{\exp(in X)}\qty(w)-\frac{n^2V_n\qty(w)}{2\qty(z-w)^2}+\textnormal{regular}\\
    \cnord{T\qty(z)V_n\qty(w)}&=T\qty(z)V_n\qty(w)-\frac{\partial V_n\qty(w)}{\qty(z-w)}-\frac{n^2V_n\qty(w)}{2\qty(z-w)^2}+\textnormal{regular}
\end{align*}

That is, the OPE is,

\begin{align*}
    T\qty(z)V_n\qty(w)&=\frac{\partial V_n\qty(w)}{\qty(z-w)}+\frac{n^2V_n\qty(w)}{2\qty(z-w)^2}+\textnormal{regular}\numberthis\label{TVope}
\end{align*}

\probitem{}

First, the OPE of $j$ with itself,

\begin{align*}
    \cnord{j\qty(z)j\qty(w)}&=j\qty(z)j\qty(w)-\cnord{\wick{\partial \c1X\qty(z)\partial \c1X\qty(w)}}\\
    \cnord{j\qty(z)j\qty(w)}&=j\qty(z)j\qty(w)-\partial_z\partial_w\qty(\ln\qty(z-w))\\
    \cnord{j\qty(z)j\qty(w)}&=j\qty(z)j\qty(w)+\partial_z\frac{1}{z-w}\\
    \cnord{j\qty(z)j\qty(w)}&=j\qty(z)j\qty(w)-\frac{1}{\qty(z-w)^2}
\end{align*}

So that,

\begin{align*}
    j\qty(z)j\qty(w)&=\frac{1}{\qty(z-w)^2}+\textnormal{regular}
\end{align*}

Now the $jV_n$ OPE,

\begin{align*}
    \cnord{j\qty(z)V_n\qty(w)}&=j\qty(z)V_n\qty(w)+\im \cnord{\wick{\c1\partial X\qty(z)\exp(\im n\c1X)\qty(w)}}\\
    \cnord{j\qty(z)V_n\qty(w)}&=j\qty(z)V_n\qty(w)+\im\im n\partial_z\qty(\ln\qty(z-w)) \cnord{\exp\qty(\im nX)}\qty(w)\\
    \cnord{j\qty(z)V_n\qty(w)}&=j\qty(z)V_n\qty(w)-\frac{n}{z-w}\cnord{\exp\qty(\im nX)}\qty(w)\\
    \cnord{j\qty(z)V_n\qty(w)}&=j\qty(z)V_n\qty(w)-\frac{n}{z-w}V_n\qty(w)
\end{align*}

So that the OPE is,

\begin{align*}
    j\qty(z)V_n\qty(w)&=\frac{n}{z-w}V_n\qty(w)+\textnormal{regular}
\end{align*}

In the last problem we already derived how to obtain the algebra of modes from the OPE, we simply integrate over a contour 
around $w$ with weight $z^p$, we need to remember that $j$ has conformal weight $\qty(1,0)$, so that the integration with respect to $z^p$ 
around the origin gives $j_p$,

\begin{align*}
    \comm{j_p}{j\qty(w)}=\oint\limits_C\frac{\dd{z}}{2\pi\im}z^p j\qty(z)j\qty(w)&=\oint\limits_C\frac{\dd{z}}{2\pi\im}\frac{z^p}{\qty(z-w)^2}+\oint\limits_C\frac{\dd{z}}{2\pi\im}z^p\textnormal{regular}\\
    \comm{j_p}{V_n\qty(w)}=\oint\limits_C\frac{\dd{z}}{2\pi\im}z^p j\qty(z)V_n\qty(w)&=\oint\limits_C\frac{\dd{z}}{2\pi\im}\frac{nz^p}{z-w}V_n\qty(w)+\oint\limits_C\frac{\dd{z}}{2\pi\im}z^p \textnormal{regular}
\end{align*}

The regular part doesn't have any pole at $z=w$, hence,

\begin{align*}
    \comm{j_p}{j\qty(w)}&=\oint\limits_C\frac{\dd{z}}{2\pi\im}\frac{z^p}{\qty(z-w)^2}=\partial_z\qty(z^p)\eval_{z=w}=pw^{p-1}\\
    \comm{j_p}{V_n\qty(w)}&=\oint\limits_C\frac{\dd{z}}{2\pi\im}\frac{nz^p}{z-w}V_n\qty(w)=nw^pV_n\qty(w)\numberthis\label{jVcom}
\end{align*}

We integrate over again to obtain the commutator of the modes only, this time with a contour around the origin, but, notice from \ref{TVope}, 
$V_n$ has conformal weight $\qty(\frac{n^2}{2},0)$, thus, when integrating around the origin with $w^q$, this will give the mode $V_{n\qty(q+1-\frac{n^2}{2})}$,

\begin{align*}
    \comm{j_p}{j_q}&=\oint\limits_C\frac{\dd{w}}{2\pi\im}w^q\comm{j_n}{j\qty(w)}=\oint\limits_C\frac{\dd{w}}{2\pi\im}w^qpw^{p-1}=p\delta_{p+q,0}\\
    \comm{j_p}{V_{n\qty(q+1-\frac{n^2}{2})}}&=\oint\limits_C\frac{\dd{w}}{2\pi\im}w^q\comm{j_n}{V_n\qty(w)}=\oint\limits_C\frac{\dd{w}}{2\pi\im}w^qnw^pV_n\qty(w)=nV_{n\qty(p+q+1-\frac{n^2}{2})}
\end{align*}

From the first algebra, the $jj$ one, we get that $j$ is the current associated with some $SO\qty(2)$ or $U\qty(1)$ algebra, as there is no 
structure constants and just one current, and the first algebra is just the Kac-Moody central extension of this $SO\qty(2)$ or $U\qty(1)$ algebra. 
The usual conserved charge obtained from a conserved current is just $j_0$, from where we can see that,

\begin{align*}
    \comm{j_0}{V_{n\qty(q+1-\frac{n^2}{2})}}&=nV_{n\qty(q+1-\frac{n^2}{2})}
\end{align*}

Or even from \ref{jVcom},

\begin{align*}
    \comm{j_0}{V_n\qty(w)}&=nV_n\qty(w)
\end{align*}

This is saying to us that, under transformations generated by $j_0$ --- Which are the symmetries that originate the $j$ current ---, the modes of $V_n$, 
and of course also $V_n$ itself, change just by a scaling of $n$, the correct relation between commutator with current and change of operators is,

\begin{align*}
    \comm{j_0}{A}=\frac{1}{\im }\dv{}{t} A\eval_{t=0}
\end{align*}

Where $t$ is a parametrization of the transformation, in our case,

\begin{align*}
    \dv{}{t}V_n\qty(w)\eval_{t=0}&=\im nV_n\qty(w)\\
    \dv{}{t}V_n\qty(w;t)&=\im nV_n\qty(w;t)\Rightarrow V_n\qty(w;t)=V_n\qty(w)\exp\qty(\im n t)
\end{align*}

This makes clear that the transformation generated by $j_0$ is a $U\qty(1)$ transformation, or in particular, 
$V_n\qty(w)$ transforms under a $U\qty(1)$ representation of the symmetry generated by $j_0$, with `\textit{charge}' $n$.

\probitem{}

The energy momentum tensor has to have conformal weight $\qty(2,0)$ and be symmetric under the $U\qty(1)$, 

\begin{align*}
    \psi\rightarrow \exp\qty(\im t)\psi,\ \ \ \tilde\psi\rightarrow\exp\qty(-\im t)\tilde\psi
\end{align*}

There are just two terms which are compatible with this,

\begin{align*}
    T_\psi\qty(z)&=\alpha\cnord{\tilde\psi\partial\psi}\qty(z)+\beta\cnord{\partial\tilde\psi\psi}\qty(z)
\end{align*}

To fix $\alpha,\beta$ we have to compute the OPE's $T\psi,T\tilde\psi$,

\begin{align*}
    \cnord{T_\psi\qty(z)\psi\qty(w)}&=T_\psi\qty(z)\psi\qty(w)+\alpha\cnord{\wick[offset=1.2em]{(\c1{\tilde\psi}\partial\psi)\qty(z)\c1\psi\qty(w)}}+\beta\cnord{\wick[offset=1.2em]{(\partial\c1{\tilde\psi}\psi)\qty(z)\c1\psi\qty(w)}}\\
    \cnord{T_\psi\qty(z)\psi\qty(w)}&=T_\psi\qty(z)\psi\qty(w)+\alpha\frac{1}{z-w}\partial\psi\qty(z)+\beta\partial_z\qty(\frac{1}{z-w})\psi\qty(z)\\
    \cnord{T_\psi\qty(z)\psi\qty(w)}&=T_\psi\qty(z)\psi\qty(w)+\alpha\frac{1}{z-w}\partial\psi\qty(w)-\frac{\beta}{\qty(z-w)^2}\psi\qty(z)+\textnormal{regular}\\
    T_\psi\qty(z)\psi\qty(w)&=-\alpha\frac{1}{z-w}\partial\psi\qty(w)+\frac{\beta}{\qty(z-w)^2}\psi\qty(z)+\textnormal{regular}
\end{align*}

This fixes $\alpha=-1,\beta=\frac12$, otherwise we have the wrong conformal transformations, so,

\begin{align*}
    T_\psi\qty(z)&=-\cnord{\tilde\psi\partial\psi}\qty(z)+\frac12\cnord{\partial\tilde\psi\psi}\qty(z)
\end{align*}

For the current, we need it to have conformal weight $\qty(1,0)$ and to be hermitian, the only possible term is,

\begin{align*}
    j_\psi\qty(z)&=\alpha\cnord{\tilde\psi\psi}\qty(z)
\end{align*}
 
With $\alpha$ real. To fix it we compute the $j\psi$ OPE,

\begin{align*}
    \cnord{j_\psi\qty(z)\psi\qty(w)}&=j_\psi\qty(z)\psi\qty(w)+\alpha\cnord{\wick[offset=1.2em]{(\c1{\tilde\psi}\psi)\qty(z)\c1\psi\qty(w)}}\\
    \cnord{j_\psi\qty(z)\psi\qty(w)}&=j_\psi\qty(z)\psi\qty(w)+\frac{\alpha}{z-w}\psi\qty(z)\\
    \cnord{j_\psi\qty(z)\psi\qty(w)}&=j_\psi\qty(z)\psi\qty(w)+\frac{\alpha}{z-w}\psi\qty(w)+\textnormal{regular}\\
    j_\psi\qty(z)\psi\qty(w)&=-\frac{\alpha}{z-w}\psi\qty(w)+\textnormal{regular}\\
    \oint\limits_C\frac{\dd{z}}{2\pi\im}j_\psi\qty(z)\psi\qty(w)&=-\oint\limits_C\frac{\dd{z}}{2\pi\im}\frac{\alpha}{z-w}\psi\qty(w)+\oint\limits_C\frac{\dd{z}}{2\pi\im}\textnormal{regular}
\end{align*}

Where $C$ is any contour that encloses $w$, again, the regular terms have no poles at $z=w$,

\begin{align*}
    \comm{j_0}{\psi\qty(w)}&=\oint\limits_C\frac{\dd{z}}{2\pi\im}j_\psi\qty(z)\psi\qty(w)=-\oint\limits_C\frac{\dd{z}}{2\pi\im}\frac{\alpha}{z-w}\psi\qty(w)\\
    \comm{j_0}{\psi\qty(w)}&=-\oint\limits_C\frac{\dd{z}}{2\pi\im}\frac{\alpha}{z-w}\psi\qty(w)=-\alpha\psi\qty(w)\\
    \frac{1}{\im}\dv{}{t}\qty[\exp\qty(\im t)\psi\qty(w)]\eval_{t=0}&=-\alpha\psi\qty(w)\Rightarrow \alpha=-1
\end{align*}

In other words, $\alpha$ is specified once we set the transformation of $\psi$, in this case setting the charge of it to be $1$ is enough. So,

\begin{align*}
    j_\psi\qty(z)&=-\cnord{\tilde\psi\psi}\qty(z)
\end{align*}

And finally, we compute the $Tj_\psi$ OPE,

\begin{align*}
    \cnord{T\qty(z)j_\psi\qty(w)}&=T\qty(z)j_\psi\qty(w)+\cnord{\wick[offset=1.2em]{(\tilde\psi\partial\psi)\qty(z)(\tilde\psi\psi)\qty(w)}}-\frac12\cnord{\wick[offset=1.2em]{(\partial\tilde\psi\psi)\qty(z)(\tilde\psi\psi)\qty(w)}}
\end{align*}

\probitem{}
\probitem{}

\newpage

\problem{}
\probitem{}
\probitem{}

\newpage

\problem{}
\probitem{}

We have already done this for general $N$ real fermions in problem \ref{3d}. Let us just cite here the 
results,

\begin{align*}
    j^{\qty[ab]}\qty(z)&=\frac1 2\tensor{T}{^{\qty[ab]}_k_l}\cnord{\psi^k\psi^l}\qty(z)\\
    j^A\qty(z)&=\frac1 2\tensor{T}{^A_k_l}\cnord{\psi^k\psi^l}\qty(z)\\
    j^A\qty(z_1)j ^B\qty(z_2)&=\frac{\im\tensor{f}{^A^B_C}}{\qty(z_1-z_2)}j^C\qty(z_2)+\frac{\Tr\qty[T^A T^B]}{2\qty(z_1-z_2)^2}+\textnormal{regular}
\end{align*}

Where in this case $\tensor{T}{^{\qty[ab]}_k_l}=\tensor{T}{^A_k_l}$ are the generators of $SO\qty(4)$, and $\tensor{f}{^A^B_C}$ are the structure constants such 
that $a,b,k,l=1,2,3,4$ and $A,B,C=1,2,3,4,5,6$. That $\psi^k$ transforms in the fundamental representation is a 
trivial fact, due to they consisting of a real --- the Grassmannian nature of them do not interfere --- vector of four components, exactly the 
representation on which the $4x4$ orthogonal, determinant one matrices act. If it form a $SO\qty(4)$ Kac-Moody algebra can 
be already seem from our derivation of a generic $SO\qty(N)$ Kac-Moody algebra, which we cited above.

\probitem{}

To keep the conventions already established in problems \ref{3c},\ref{4c}, we're going to choose the following 
for grouping the four fermions into a pair of complex ones,

\begin{align*}
    {\Psi}^{\dot 1}=\frac{1}{\sqrt{2}}\psi^1-\frac{\im}{\sqrt2}\psi^2\\
    {\tilde\Psi}^{\dot 1}=\frac{1}{\sqrt{2}}\psi^1+\frac{\im}{\sqrt2}\psi^2\\
    {\Psi}^{\dot 2}=\frac{1}{\sqrt{2}}\psi^3-\frac{\im}{\sqrt2}\psi^4\\
    {\tilde\Psi}^{\dot 2}=\frac{1}{\sqrt{2}}\psi^3+\frac{\im}{\sqrt2}\psi^4
\end{align*}

This choice makes the following,


\begin{align*}
    -\cnord{{\tilde\Psi}^{\dot 1}\Psi^{\dot 1}}&=-\frac12\cnord{\psi^1\psi^1+\im\psi^2\psi^1-\psi^1\im\psi^2+\psi^2\psi^2}\\
    -\cnord{{\tilde\Psi}^{\dot 1}\Psi^{\dot 1}}&=\im\cnord{\psi^1\psi^2}
\end{align*}

As outcome of problems \ref{3c},\ref{4c} this should be seen as a good relation to hold. To bosonize each pair ${\tilde\Psi}^{\dot a}\Psi^{\dot a}$ --- We'll use $\dot a,\dot b=1,2$ for the complex fermions we 
leave $j,k,l=1,2,3,4$ for the real ones ---, 
we follow the ideas of problem \ref{4}, to each pair of complex fermions we attribute a chiral boson $X^{\dot a}\qty(z)$, then 
we should have, as was already argued in problem \ref{4d}, the following correspondence,

\begin{subequations}\label{teste1}
\begin{align}
    \cnord{\exp\qty(\im X^{\dot a})}&= \Psi^{\dot a}\\
    \cnord{\exp\qty(-\im X^{\dot a})}&= {\tilde\Psi}^{\dot a}
\end{align}
\end{subequations}

We choose a particular way of bosonizing, a more generic choice could be $\cnord{\exp\qty(\im X^{\dot a})}=\alpha \Psi^{\dot a},\cnord{\exp\qty(-\im X^{\dot a})}=\alpha^{-1} {\tilde\Psi}^{\dot a}$. 
But what matters is that now we have a description of the fermionic theory by a bosonic one, consisting of two compactified chiral bosons $X^{\dot a}\qty(z)=X^{\dot a}\qty(z)+2\pi$. 

\probitem{}

The symmetry of the bosonic theory is,

\begin{align*}
    X^{\dot a}\qty(z)\rightarrow X^{\dot a}\qty(z)+t^{\dot a}\ \qty(\textnormal{mod }2\pi)
\end{align*}

What from \ref{teste1} can be seen as a realization of a $U\qty(1)$ symmetry for each index $\dot a$, as there is 
two of them, and the symmetries are disconnected, this means a $U\qty(1)\times U\qty(1)$ symmetry,

\begin{align*}
    \Psi^{\dot a}=\cnord{\exp\qty(\im X^{\dot a})}&\rightarrow\cnord{\exp\qty(\im X^{\dot a}+\im t^{\dot a})},\ \ \ t^{\dot a}\in[0,2\pi)\\
    \Psi^{\dot a}&\rightarrow\exp\qty(\im t^{\dot a})\cnord{\exp\qty(\im X^{\dot a})},\ \ \ t^{\dot a}\in[0,2\pi)\\
    \Psi^{\dot a}&\rightarrow\exp\qty(\im t^{\dot a})\Psi^{\dot a},\ \ \ t^{\dot a}\in[0,2\pi)
\end{align*}

From the last line is clear that we have a $U\qty(1)$ symmetry for each index, specially due to the chiral boson being 
compactified. Of course, this transformation above implies the transformation of the complex one also,

\begin{align*}
    {\tilde\Psi}^{\dot a}=\cnord{\exp\qty(-\im X^{\dot a})}&\rightarrow\cnord{\exp\qty(-\im X^{\dot a}-\im t^{\dot a})},\ \ \ t^{\dot a}\in[0,2\pi)\\
    {\tilde\Psi}^{\dot a}&\rightarrow\exp\qty(-\im t^{\dot a})\cnord{\exp\qty(-\im X^{\dot a})},\ \ \ t^{\dot a}\in[0,2\pi)\\
    {\tilde\Psi}^{\dot a}&\rightarrow\exp\qty(-\im t^{\dot a}){\tilde\Psi}^{\dot a},\ \ \ t^{\dot a}\in[0,2\pi)
\end{align*}

So now it's clear that ${\tilde\Psi}^{\dot a}$ have charge $-1$ with the respective $U\qty(1)$'s, and $\Psi^{\dot a}$ have charge $+1$ with the 
respective $U\qty(1)$'s. What we have shown is that $U\qty(1)\times U\qty(1)\subset SO\qty(4)$, so it's possible for us to work out the action of each of these $U\qty(1)$'s over the vector representation, 
let's name $U\qty(1)\times U\qty(1)$ by $U\qty(1)_{\dot 1}\times U\qty(1)_{\dot 2}$, the nomenclature should be self evident. With a little 
help from group theory, it's known that \textbf{locally} $SO\qty(4)$ is isomorphic to $SU\qty(2)\times SU\qty(2)$, that is not true globally, as the 
two groups have different topologies. But nevertheless, under a quantization, any global classical symmetry has to under go a double cover, here is no 
different, our symmetry group $SO\qty(4)$ has to under go to a double cover, which is by definition the group $Spin\qty(4)$, which luckily is 
globally isomorphic to $SU\qty(2)\times SU\qty(2)$. Just recalling everything, our classical theory of four fermions 
enjoyed a $SO\qty(4)$ symmetry, that is, generically if we stayed at classical analysis, we would be interested in 
operators which transform under representations of $SO\qty(4)$, but, as ultimately we're interested in the quantum theory, 
this enlargers the symmetry group, and we actually have to search for operators which transform in representations of the double cover of $SO\qty(4)$, 
that is $Spin\qty(2)\simeq SU\qty(2)\times SU\qty(2)$, while $SO\qty(4)$ just allows for integer charges/spins, it's double cover 
allows for half-integers charges/spins, this is a well known fact, as $SU\qty(2)$ possesses representations which have a half integer 
quadratic Casimir operator, the fundamental representation as example. But, we just found an explicit representation of a subgroup of $SO\qty(4)$ by the 
bosonization of the fermionic theory, $U\qty(1)_{\dot 1}\times U\qty(1)_{\dot 2}\subset SO\qty(4)\Rightarrow U\qty(1)_{\dot 1}\times U\qty(1)_{\dot 2}\subset SU\qty(2)\times SU\qty(2)$, 
let us just stress some facts here, we have found \textbf{two abelian }$\vb{U\qty(1)}$ \textbf{subgroups of }$\vb{SU\qty(2)\times SU\qty(2)}$, of course $U\qty(1)$ 
by itself is an abelian group, but, they're abelian \textbf{among themselves}, this is what the $\times$ is saying, and this can be seen both from the bosonization 
as well from the fermionic representation, the transformations for $X^{\dot a}\rightarrow X^{\dot a}+t^{\dot a}\ \qty(\textnormal{mod }2\pi)$ are totally independent 
for each index, as well as,
\begin{align*}
    \begin{cases}
        \Psi^{\dot a}&\rightarrow\exp\qty(\im t^{\dot a})\Psi^{\dot a},\ \ \ t^{\dot a}\in[0,2\pi)\\
        {\tilde\Psi}^{\dot a}&\rightarrow\exp\qty(-\im t^{\dot a}){\tilde\Psi}^{\dot a},\ \ \ t^{\dot a}\in[0,2\pi)
    \end{cases}
\end{align*}
are independent for each index. This raises an eyebrow, as certainly $SO\qty(4)$ is not abelian, an neither is $SU\qty(2)\times SU\qty(2)$, but, 
the double cover has naturally an $\times$ in the definition, this means it's composed of two copies of $SU\qty(2)$ which are abelian among themselves! 
It's clear that $SU\qty(2)$ itself is not abelian, and also is clear that $U\qty(1)\subset SU\qty(2)$, thus, the only possible way we could extract two abelian 
copies of $U\qty(1)$ from $SU\qty(2)\times SU\qty(2)$ is if we extracted one copy of each, this means we have a bigger identification of the symmetry $U\qty(1)_{\dot1}\times U\qty(1)_{\dot 2}$ as being,

\begin{align*}
    U\qty(1)_{\dot 1}\times U\qty(2)_{\dot 2}\subset SU\qty(2)_{\dot 1}\times SU\qty(2)_{\dot 2}
\end{align*}

We just identified which $U\qty(1)$ came from which $SU\qty(1)$, this particular $U\qty(1)_{\dot a}\subset SU\qty(2)_{\dot a}$ we picked up 
can be seen to be represented in a diagonal form in $\Psi^{\dot a},{\tilde\Psi}^{\dot a}$, as at in each $\mathfrak{su}\qty(2)_{\dot a}$ we 
can diagonalize at maximum one generator and the single Casimir --- which will have positive half-integer eigenvalues for the Casimir $j$, 
and $m=-j,-j+1,\cdots,j$ for the chosen generator ---, it's clear that 
\begin{align*}
    \begin{cases}
        \Psi^{\dot a}&\rightarrow\exp\qty(\im j t^{\dot a})\Psi^{\dot a},\ \ \ t^{\dot a}\in[0,2\pi),\ \ \ j=1\\
        {\tilde\Psi}^{\dot a}&\rightarrow\exp\qty(-\im j t^{\dot a}){\tilde\Psi}^{\dot a},\ \ \ t^{\dot a}\in[0,2\pi),\ \ \ j=1
    \end{cases}
\end{align*}
implies they transform in the spin $1$ representation of each particular $SU\qty(2)$. After all these remarks, 
we finally know how to ask what we want, we want operators that transform under the spin $\frac12$ representation 
of the double cover of $SO\qty(4)$, but now we know what this means, this means we want operators that transform as,
\begin{align*}
    \begin{cases}
        \Phi^{\dot a}&\rightarrow\exp\qty(\im j t^{\dot a})\Phi^{\dot a},\ \ \ t^{\dot a}\in[0,2\pi)\\
        {\tilde\Phi}^{\dot a}&\rightarrow\exp\qty(-\im j t^{\dot a}){\tilde\Phi}^{\dot a},\ \ \ t^{\dot a}\in[0,2\pi)
    \end{cases}
\end{align*}
for $j=\frac12$. As for each index this is a spin $\frac12$ representation of $SU\qty(2)_{\dot a}$. But we have a natural way of 
analyzing the transformations being done by the diagonal generator of $SU\qty(2)_{\dot a}$, the symmetry in the bosonic theory! 
$X^{\dot a}\rightarrow X^{\dot a}+t^{\dot a}\ \qty(\textnormal{mod }2\pi)$ correspond to a transformation generated by the diagonal 
generator of each $SU\qty(2)_{\dot a}$, thus, is easy to get the desired result, consider,
\begin{align*}
    \Phi^{\dot a}=\cnord{\exp\qty(\frac\im2 X^{\dot a})}&\rightarrow\cnord{\exp\qty(\frac\im2 X^{\dot a}+\frac\im2 t^{\dot a})},\ \ \ t^{\dot a}\in[0,2\pi)\\
    \Phi^{\dot a}&\rightarrow\exp\qty(\frac\im2 t^{\dot a})\cnord{\exp\qty(\frac12\im X^{\dot a})},\ \ \ t^{\dot a}\in[0,2\pi)\\
    \Phi^{\dot a}&\rightarrow\exp\qty(\frac\im2 t^{\dot a})\Phi^{\dot a},\ \ \ t^{\dot a}\in[0,2\pi)
\end{align*}
This is exactly the desired transformation with $j=\frac12$, what means $\Phi^{\dot a}$ transform as the spin $\frac12$ representation 
of the double cover of $SO\qty(4)$, the same can be done for,
\begin{align*}
    {\tilde\Phi}^{\dot a}=\cnord{\exp\qty(-\frac\im2 X^{\dot a})}&\rightarrow\cnord{\exp\qty(-\frac\im2 X^{\dot a}-\frac\im2 t^{\dot a})},\ \ \ t^{\dot a}\in[0,2\pi)\\
    {\tilde\Phi}^{\dot a}&\rightarrow\exp\qty(-\frac\im2 t^{\dot a})\cnord{\exp\qty(-\frac\im2 X^{\dot a})},\ \ \ t^{\dot a}\in[0,2\pi)\\
    {\tilde\Phi}^{\dot a}&\rightarrow\exp\qty(-\frac\im2 t^{\dot a}){\tilde\Phi}^{\dot a},\ \ \ t^{\dot a}\in[0,2\pi)
\end{align*}

These have not a polinomial definition on the fermionic theory, but, the bosonic theory provides a good representation of these spin $\frac12$ operators,

\begin{align*}
    \cnord{\exp\qty(\frac\im2 X^{\dot a})},\ \ \ \cnord{\exp\qty(-\frac\im2 X^{\dot a})}
\end{align*}

%\newpage

%\appendix

%\section{Light-Cone Gauge for the Open String}
\label{app-openclass}

Here, we're going to fix the gauge and solve the classical equations of motion for the open string with 
Neumann Boundary Conditions at both ends. To get the classical solution we just have to solve both
\ref{eomx} and \ref{eomh}. Of course, we do know the Polyakov Action has  Diff$\times$Weyl gauge freedom, hence, 
none of the components of the metric $h$ are in fact dynamical, because by fixing this gauge freedom we actually 
fix all the components of $h$, what makes of \ref{eomh} not an equation of motion for $h$, but merely a set of 
constrains regarding the choice of gauge for $h$. But, fixing all three components of the metric isn't the 
only possible choice for picking a gauge, another option is to fix $\tau$, and two of the components 
of $h$, as we have in total 3 gauge degrees of freedom. We could in principle assign any function to $\tau$, but 
what seems to be the optimal choice is to work in the Light-cone gauge. First, for any target space vector we define,

\begin{align*}
    A^\pm=\frac{1}{\sqrt2}\qty(A^0\pm A^1)
\end{align*}

And now we state the Light-cone gauge condition, which fixes the $\tau$ reparametrization, given by\footnote{The factor of $2$ here is just convention, it get along pretty well with the 
domain we'll choose of $\sigma\in\qty[0,\pi]$ as a lot of simplifications happen, $\alpha'$ is here just for making $\tau$ dimensionless, and $p^+$ is 
the only conserved charge with no $\tau,\sigma$ dependence and with the correct target space index.},

\begin{align*}
    X^+\qty(\tau,\sigma)=2\alpha'p^+\tau
\end{align*}

Where $p^+$ is the light-cone component of the conserved charge associated with target space translations --- the momentum, \ref{p} ---. This 
condition do not allow for any more reparametrizations of $\tau$, as,

\begin{align*}
    {X'}^+\qty(\tau',\sigma')&=X^+\qty(\tau,\sigma)\\
    2\alpha'p^+\tau'&=2\alpha'p^+\tau
\end{align*}

Which states the only reparametrization compatible with the light-cone gauge condition is $\tau'\qty(\tau,\sigma)=\tau$. Hence, from our 
initial freedom we can only make $\sigma$ reparametrizations now. One way of fixing this remaining reparametrization, is to choose $\sigma$ proportional 
to the energy carried by the string, actually, we're going to choose a reparametrization such that the energy density in the string, at fixed $\tau$, is independent of 
$\sigma$. One might suspect that the energy density is given by $\mathcal P^{\tau0}$, as in \ref{defcalp}, but, in the light-cone coordinates, the one that has 
the role of energy density is $\mathcal P^{\tau+}$, we'll choose a sigma parametrization such that $\partial_\sigma\mathcal P^{\tau+}=0$, to see how can this 
be done we take the transformation of this object under a reparametrization, $\tau'=\tau,\sigma'=\sigma'\qty(\sigma,\tau)$, as,

\begin{align*}
    \mathcal P^{\tau+}&=-\frac{\sqrt h\partial^\tau X^+}{2\pi\alpha'}=-\frac{\sqrt h h^{\tau\tau}}{2\pi\alpha'}2\alpha'p^+=-\sqrt h h^{\tau\tau}\frac{p^+}{\pi}\\
    {\mathcal P'}^{\tau+}\qty(\tau,\sigma')&=-\pdv{\sigma}{\sigma'}\sqrt h h^{\tau\tau}\frac{p^+}{\pi}=\pdv{\sigma}{\sigma'}\mathcal P^{\tau+}
\end{align*}

Hence, given ${\mathcal P'}^{\tau+}\qty(\tau,\sigma')$, we can choose $\sigma=\sigma\qty(\tau,\sigma')$, so that $\partial_\sigma\mathcal P^{\tau+}=0$, as long as,

\begin{align*}
    \pdv{}{\sigma}\qty[{\pdv{\sigma'}{\sigma}}{\mathcal P'}^{\tau+}]&=0\Rightarrow{\pdv[2]{\sigma'}{\sigma}}{\mathcal P'}^{\tau+}+\qty(\pdv{\sigma'}{\sigma})^2\pdv{{\mathcal P'}^{\tau+}}{\sigma'}=0
\end{align*}

This is simply a differential equation which can be solved. But this really fixes all the $\sigma$ reparametrization? Let's see if it's possible to 
make further changes of $\sigma\rightarrow \sigma''$ preserving this condition,

\begin{align*}
    {\pdv{}{\sigma}}\qty[{\pdv{\sigma''}{\sigma}}{\pdv{\sigma'}{\sigma''}}]{\mathcal P'}^{\tau+}+\qty(\pdv{\sigma'}{\sigma''}\pdv{\sigma''}{\sigma})^2\pdv{{\mathcal P'}^{\tau+}}{\sigma'}&=0\\
    {\pdv[2]{\sigma''}{\sigma}}{\pdv{\sigma'}{\sigma''}}{\mathcal P'}^{\tau+}+\qty(\pdv{\sigma''}{\sigma})^2\qty[{\pdv[2]{{\sigma'}}{{\sigma''}}}{\mathcal P'}^{\tau+}
    +\qty(\pdv{\sigma'}{\sigma''})^2\pdv{{\mathcal P'}^{\tau+}}{\sigma'}
    ]&=0
\end{align*}

Imposing that $\sigma''$ also satisfy the parametrization condition,

\begin{align*}
    {\pdv[2]{\sigma''}{\sigma}}{\pdv{\sigma'}{\sigma''}}{\mathcal P'}^{\tau+}&=0\Rightarrow {\pdv[2]{\sigma''}{\sigma}}=0\Rightarrow \sigma''=a\sigma+b\numberthis\label{sigmaresidual}
\end{align*}

Our condition of constancy of $\mathcal P^{\tau+}$ in sigma, has a residual affine reparametrization, which we can use to set $\sigma=0$ in one of the ends of the 
string, and $\sigma=\pi$ in the other one. This completely fixes the $\sigma$ parametrization. What remains now is to fix the Weyl redundancy, as necessarily 
$h_{ab}$ has a inverse, $\Det\qty[h_{ab}]\neq 0$, thus, by a Weyl transformation is always possible to make $\Det\qty[h_{ab}]=-1$ --- We're using Lorentzian signature ---, which 
will be our choice of fixing the Weyl redundancy. There is 
no more room for any transformation now, hence, the gauge is fully fixed. Notice, as $\mathcal P^{\tau+}$ is constant in $\sigma$, we have the following,

\begin{align*}
    p^+&=\int\limits_0^\pi\dd{\sigma}\mathcal P^{\tau+}=\mathcal P^{\tau+}\pi=-p^+\sqrt h h^{\tau\tau}\\
    -1&=h^{\tau\tau}\numberthis\label{httcond}
\end{align*}

Furthermore, using the equation of motion \ref{eomx} at $\mu=+$, we get,

\begin{align*}
    0&=\partial_\tau\qty[h^{\tau\sigma}\partial_\sigma X^+h^{\tau\tau}\partial_\tau X^+]+\partial_\sigma\qty[h^{\sigma\sigma}\partial_\sigma X^++h^{\sigma\tau}\partial_\tau X^+]\\
    0&=\partial_\tau h^{\tau\tau}+\partial_\sigma h^{\sigma\tau}\\
    0&=\partial_\sigma h^{\sigma\tau}
\end{align*}

But, using the Neumann Boundary conditions \ref{boundaryc} at $\mu=+$,

\begin{align*}
    0&=\partial^\sigma X^+\eval_{\sigma=0}^{\sigma=\pi}\\
    0&=h^{\sigma\tau}\partial_\tau X^++h^{\sigma\sigma}\partial_\sigma X^+\eval_{\sigma=0}^{\sigma=\pi}\\
    0&=h^{\tau\sigma}\eval_{\sigma=0}^{\sigma=\pi}\Rightarrow h^{\tau\sigma}\eval_{\sigma=0}=h^{\tau\sigma}\eval_{\sigma=\pi}=0
\end{align*}

Together with $\partial_\sigma h^{\tau\sigma}=0$, this simply states that $h^{\tau\sigma}=h^{\sigma\tau}\equiv0$. This, with $h^{\tau\tau}=-1$ 
and the determinant condition, says that $h=\textnormal{Diag}\mqty(-1&1)$. That simplifies the equations of motion \ref{eomx} to,

\begin{align*}
    \partial_\tau\partial_\tau X^\mu-\partial_\sigma\partial_\sigma X^\mu={\ddot X}^\mu-{X''}^\mu=0
\end{align*}

And also the consistency conditions \ref{eomh} as,

\begin{align*}
    0=&\partial_a X^\mu \partial_b X_\mu-\frac12 h_{ab}\qty(-{\dot X}^2+{X'}^2)\\
    0=&\begin{cases}
        \partial_\tau X^\mu \partial_\tau X_\mu+\frac12\qty(-{\dot X}^2+{X'}^2)&=\frac12\qty[{\dot X}^2+{X'}^2]\\
        \partial_\tau X^\mu \partial_\sigma X_\mu&=\dot X\cdot X'\\
        \partial_\sigma X^\mu \partial_\sigma X_\mu-\frac12\qty(-{\dot X}^2+{X'}^2)&=\frac12\qty[{\dot X}^2+{X'}^2]
    \end{cases}
\end{align*}

These constrains can be recast as,

\begin{align*}
    {\dot X}^2+{X'}^2\pm&=0\\
    \qty(\dot X\pm X')^2&=0\numberthis\label{constrains}
\end{align*}

Lastly, but not less important, we have the Boundary Conditions, \ref{boundaryc}, which take the form,

\begin{align*}
    \partial_\sigma X^\mu\eval_{\sigma=0}=\partial_\sigma X^\mu\eval_{\sigma=\pi}=0
\end{align*}

We have now to solve the equation of motion with the right Boundary conditions, for this, is easier to change 
coordinates to $\sigma^\pm=\tau\pm\sigma$, in which the new equations of motion read,

\begin{align*}
    0&=\pdv[2]{}{\tau} X^\mu-\pdv[2]{}{\sigma}X^\mu\\
    0&=\pdv{}{\tau}\qty[\pdv{\sigma^+}{\tau}\pdv{}{\sigma^+}X^\mu+\pdv{\sigma^-}{\tau}\pdv{}{\sigma^-}X^\mu]-\pdv{}{\sigma}\qty[\pdv{\sigma^+}{\sigma}\pdv{}{\sigma^+}X^\mu+\pdv{\sigma^-}{\sigma}\pdv{}{\sigma^-}X^\mu]\\
    0&=\pdv{}{\tau}\qty[\pdv{}{\sigma^+}X^\mu+\pdv{}{\sigma^-}X^\mu]-\pdv{}{\sigma}\qty[\pdv{}{\sigma^+}X^\mu-\pdv{}{\sigma^-}X^\mu]\\
    0&=\qty(\pdv{}{\sigma^+}+\pdv{}{\sigma^-})\qty[\pdv{}{\sigma^+}X^\mu+\pdv{}{\sigma^-}X^\mu]-\qty(\pdv{}{\sigma^+}-\pdv{}{\sigma^-})\qty[\pdv{}{\sigma^+}X^\mu-\pdv{}{\sigma^-}X^\mu]\\
    0&=\partial_+\partial_-X^\mu\numberthis\label{fulleqom}
\end{align*}

This is easily solved for,

\begin{align*}
    X^\mu=\frac12\qty(f^\mu\qty(\sigma^+)+g^\mu\qty(\sigma^-))
\end{align*}

Imposing the boundary condition at $\sigma=0$,

\begin{align*}
    {X'}^\mu\qty(\tau,\sigma=0)&=\frac12\qty({f'}^\mu\qty(\tau)-{g'}^\mu\qty(\tau))=0
\end{align*}

This states that $g^\mu$ is equal to $f^\mu$ apart from a constant, which we'll absorb in the definition of $f^\mu$. Now, the boundary condition 
on $\sigma=\pi$,

\begin{align*}
    X^\mu&=\frac12\qty(f^\mu\qty(\sigma^+)+f^\mu\qty(\sigma^-))\\
    {X'}^\mu\qty(\tau,\pi)&=\frac12\qty({f'}^\mu\qty(\tau+\pi)-{f'}^\mu\qty(\tau-\pi))=0
\end{align*}

That is, ${f'}^\mu$ is periodic with period $2\pi$. The most general real function with period $2\pi$ is\footnote{The $\alpha'$ factors here are just to make the $\alpha_n^\mu$ dimensionless.},

\begin{align*}
    {f'}^\mu\qty(u)&=f_1^\mu+\sqrt{2\alpha'}\sum\limits_{n=1}^\infty\qty({\alpha^\ast_n}^\mu\exp\qty(\im n u)+\alpha_n^\mu\exp\qty(-\im n u))\\
    f^\mu\qty(u)&=f_0^\mu+f_1^\mu u-\im\sqrt{2\alpha'}\sum\limits_{n=1}^\infty\frac1n\qty({\alpha^\ast_n}^\mu\exp\qty(\im n u)-\alpha_n^\mu\exp\qty(-\im n u))
\end{align*}

Defining $\alpha_{-n}^\mu={\alpha^\ast_n}^\mu$,

\begin{align*}
    f^\mu\qty(u)&=f_0^\mu+f_1^\mu u+\im\sqrt{2\alpha'}\sum\limits_{n\in\mathbb Z^\ast}\frac{\alpha_n^\mu}{n}\exp\qty(-\im n u)
\end{align*}

Now, back in $X$,

\begin{align*}
    X^\mu&=f_0^\mu+\frac12f_1^\mu\qty(\sigma^++\sigma^-)+\im\sqrt{2\alpha'}\sum\limits_{n\in\mathbb Z^\ast}\frac{\alpha_n^\mu}{n}\frac12\qty[\exp\qty(-\im n \sigma^+)+\exp\qty(-\im n\sigma^-)]\\
    X^\mu&=f_0^\mu+f_1^\mu\tau+\im\sqrt{2\alpha'}\sum\limits_{n\in\mathbb Z^\ast}\frac{\alpha_n^\mu}{n}\exp\qty(-\im n\tau)\frac12\qty[\exp\qty(-\im n \sigma)+\exp\qty(\im n\sigma)]\\
    X^\mu&=f_0^\mu+f_1^\mu\tau+\im\sqrt{2\alpha'}\sum\limits_{n\in\mathbb Z^\ast}\frac{\alpha_n^\mu}{n}\exp\qty(-\im n\tau)\cos\qty(n\sigma)   
\end{align*}

Notice that,

\begin{align*}
    p^\mu&=\int\limits_0^\pi\dd{\sigma}\mathcal P^{\tau\mu}=\int\limits_0^\pi\dd{\sigma}\frac{\partial_\tau X^\mu}{2\pi\alpha'},\ \ \ \textnormal{Because }h=\textnormal{Diag}\mqty(-1&1)\\
    p^\mu&=\frac{1}{2\pi\alpha'}\int\limits_0^\pi\dd{\sigma}\qty[f_1^\mu+\sqrt{2\alpha'}\sum\limits_{n\in\mathbb Z^\ast}\alpha_n^\mu\exp\qty(-\im n\tau)\cos\qty(n\sigma)]=\frac{f_1^\mu}{2\alpha'}   
\end{align*}

So that,

\begin{align*}
    X^\mu&=f_0^\mu+2\alpha' p^\mu\tau+\im\sqrt{2\alpha'}\sum\limits_{n\in\mathbb Z^\ast}\frac{\alpha_n^\mu}{n}\exp\qty(-\im n\tau)\cos\qty(n\sigma)   
\end{align*}

And it's clear that,

\begin{align*}
    \frac1\pi\int\limits_0^{\pi}\dd{\sigma}X^\mu\qty(0,\sigma)&=f_0^\mu
\end{align*}

That is, $f_0^\mu$ is the mean position at $\tau=0$, so we relabel it to be $f_0^\mu=x_0^\mu$. It's also useful to define a additional mode, $\alpha_0^\mu=\sqrt{2\alpha'}p^\mu$,

\begin{align*}
    X^\mu&=x_0^\mu+2\alpha' p^\mu\tau+\im\sqrt{2\alpha'}\sum\limits_{n\in\mathbb Z^\ast}\frac{\alpha_n^\mu}{n}\exp\qty(-\im n\tau)\cos\qty(n\sigma)\\
    X^\mu&=x_0^\mu+\sqrt{2\alpha'} \alpha_0^\mu\tau+\im\sqrt{2\alpha'}\sum\limits_{n\in\mathbb Z^\ast}\frac{\alpha_n^\mu}{n}\exp\qty(-\im n\tau)\cos\qty(n\sigma)   
\end{align*}

This is not the end of it, because we need to make sure the two constrains are satisfied, namely, \ref{constrains}. With the convention of 
uppercase latin index referring to non-light-cone components, $I=2,\cdots,D-1$, the constrains are,

\begin{align*}
    0&=-2\qty({\dot X}^-\pm{X'}^-)\qty({\dot X}^+\pm{X'}^+)+\qty({\dot X}^I\pm{X'}^I)^2\\
    4\alpha'p^+\qty({\dot X}^-\pm{X'}^-)&=\qty({\dot X}^I\pm{X'}^I)^2\\
    {\dot X}^-\pm{X'}^-&=\frac{1}{4\alpha'p^+}\qty({\dot X}^I\pm{X'}^I)^2
\end{align*}

That is, the two constrains implies that $X^-$ is not dynamical, and, the collection of $X^I$ fully determine $X^-$, 
apart from a single integration constant, $x_0^-$. To see this is helpful to note,

\begin{align*}
    {\dot X}^\mu&=\sqrt{2\alpha'} \alpha_0^\mu+\sqrt{2\alpha'}\sum\limits_{n\in\mathbb Z^\ast}\alpha_n^\mu\exp\qty(-\im n\tau)\cos\qty(n\sigma)\\
    {\dot X}^\mu&=\sqrt{2\alpha'} \alpha_0^\mu\exp(-\im 0 \tau)\cos\qty(0\sigma)+\sqrt{2\alpha'}\sum\limits_{n\in\mathbb Z^\ast}\alpha_n^\mu\exp\qty(-\im n\tau)\cos\qty(n\sigma)\\
    {\dot X}^\mu&=\sqrt{2\alpha'}\sum\limits_{n\in\mathbb Z}\alpha_n^\mu\exp\qty(-\im n\tau)\cos\qty(n\sigma)
\end{align*}

And,

\begin{align*}
    {X'}^\mu&=-\im\sqrt{2\alpha'}\sum\limits_{n\in\mathbb Z^\ast}\alpha_n^\mu\exp\qty(-\im n\tau)\sin\qty(n\sigma)\\
    {X'}^\mu&=-\im\sqrt{2\alpha'}\alpha_0^\mu\exp\qty(-\im 0\tau)\sin(0\sigma)-\im\sqrt{2\alpha'}\sum\limits_{n\in\mathbb Z^\ast}\alpha_n^\mu\exp\qty(-\im n\tau)\sin\qty(n\sigma)\\
    {X'}^\mu&=-\im\sqrt{2\alpha'}\sum\limits_{n\in\mathbb Z}\alpha_n^\mu\exp\qty(-\im n\tau)\sin\qty(n\sigma)
\end{align*}

Which implies,

\begin{align*}
    {\dot X}^\mu\pm{X'}^\mu&=\sqrt{2\alpha'}\sum\limits_{n\in\mathbb Z}\alpha_n^\mu\exp\qty(-\im n\tau)\qty(\cos\qty(n\sigma)\mp\im\sin\qty(n\sigma))\\
    {\dot X}^\mu\pm{X'}^\mu&=\sqrt{2\alpha'}\sum\limits_{n\in\mathbb Z}\alpha_n^\mu\exp\qty(-\im n\qty(\tau\pm \sigma))\numberthis\label{pmdot}
\end{align*}

So that,

\begin{align*}
    {\dot X}^-\pm{X'}^-&=\frac{1}{4\alpha'{p^+}}\qty({\dot X}^I\pm{X'}^I)^2\\
    \sqrt{2\alpha'}\sum\limits_{n\in\mathbb Z}\alpha^-_n\exp\qty(-\im n\qty(\tau\pm\sigma))&=\frac{2\alpha'}{4\alpha' p^+}\sum\limits_{p,q\in\mathbb Z}\alpha^I_p\alpha^I_q\exp\qty(-\im\qty(p+q)\qty(\tau\pm\sigma))\\
    \sqrt{2\alpha'}\sum\limits_{n\in\mathbb Z}\alpha^-_n\exp\qty(-\im n\qty(\tau\pm\sigma))&=\frac{1}{2p^+}\sum\limits_{n,p\in\mathbb Z}\alpha^I_p\alpha^I_{n-p}\exp\qty(-\im n\qty(\tau\pm\sigma))\\
    \sqrt{2\alpha'}\sum\limits_{n\in\mathbb Z}\alpha^-_n\exp\qty(-\im n\qty(\tau\pm\sigma))&=\frac{1}{p^+}\sum\limits_{n\in\mathbb Z}\qty(\frac12\sum\limits_{p\in\mathbb Z}\alpha^I_p\alpha^I_{n-p})\exp\qty(-\im n\qty(\tau\pm\sigma))\\
    \sqrt{2\alpha'}\alpha_n^-&=\frac{1}{2p^+}\sum\limits_{p\in\mathbb Z}\alpha_p^I\alpha_{n-p}^I=\frac{1}{p^+}L_n^\perp\numberthis\label{virasorodef}
\end{align*}

Hence, all the fourier modes of $X^-$ are completely determined by the transverse modes ones, apart from of course the integration constant $x^-_0$. In the 
last passage we also defined the Virasoro modes $L_n^\perp$. This is it, we fully solved the equation of motion with all the constrains and boundary 
conditions, let's rewrite them here just for completeness,

\begin{align*}
    X^I&=x_0^I+\sqrt{2\alpha'} \alpha_0^I\tau+\im\sqrt{2\alpha'}\sum\limits_{n\in\mathbb Z^\ast}\frac{\alpha_n^I}{n}\exp\qty(-\im n\tau)\cos\qty(n\sigma)\numberthis\label{solopen1}\\
    X^-&=x_0^-+\sqrt{2\alpha'} \alpha_0^-\tau+\im\sqrt{2\alpha'}\sum\limits_{n\in\mathbb Z^\ast}\frac{\alpha_n^-}{n}\exp\qty(-\im n\tau)\cos\qty(n\sigma),\ \ \ \alpha^-_n=\frac{1}{\sqrt{2\alpha'}p^+}L^\perp_n\numberthis\label{solopen2}\\
    X^+&=2\alpha'p^+\tau\numberthis\label{solopen3}
\end{align*}

%\newpage

%\section{Light-Cone Gauge for the Closed String}
\label{app-closedclass}

Here, we're going to fix the gauge and solve the classical equations of motion for the closed string. The procedure here 
is pretty much the same as we already done in the Appendix \ref{app-openclass}, the gauge fixing conditions are very similar 
to the ones used there, that is,

\begin{align*}
    X^+\qty(\tau,\sigma)&=\alpha'p^+\tau\\
    \partial_\sigma\mathcal P^{\tau +}&=0\\
    \Det\qty[h_{ab}]&=-1
\end{align*}

This three conditions almost fully fix the gauge, but, as discussed already in Appendix \ref{app-openclass}, the second condition in this 
list allows for a residual gauge \ref{sigmaresidual},

\begin{align*}
    \sigma'=a\sigma+b\numberthis\label{sresclosed}
\end{align*}

But now we have to impose we're dealing with a closed string, that is, the $\sigma$ parameter lives in a topology of a 
circle, in other words, exists $l\in\mathbb R$, such that,

\begin{align*}
    X\qty(\tau,l+\sigma)=X\qty(\tau,\sigma)
\end{align*}

Or in an even better description, $\sigma\sim \sigma+l\textnormal{ mod }l$. Notice that we can use \ref{sresclosed} partially to 
set $l$ to whatever value we like, we'll choose $l=2\pi$, which facilitates some computations, but, the translational 
part of \ref{sresclosed}, $\sigma'=\sigma+b\qty(\tau)$ is impossible to get fully rid off, we have no privileged point to call $\sigma=0$, but, we can remove the 
$\tau$ dependence in it by choosing a specific family of $\sigma=\textnormal{cte}$ curves, this choice will be of the 
$\sigma=\textnormal{cte}$ lines to be orthogonal to the $\tau=\textnormal{cte}$ lines, this is of course another condition 
on the metric,

\begin{align*}
    h^{\sigma\tau}\qty(\tau,\sigma_0)=0
\end{align*}

Where $\sigma_0$ is just any fixed value. This leaves again a residual reparametrization, a solid translation of the $\sigma$ 
coordinate, $\sigma'=\sigma+a$, this cannot be resolved, and we'll have to live with.
With also the same reasoning of \ref{httcond} we can obtain,

\begin{align*}
    p^+&=\int\limits_{\sigma_0}^{\sigma_0+2\pi}\dd{\sigma}\mathcal P^{\tau+}=\mathcal P^{\tau+}2\pi=-\frac{2\pi\alpha' p^+\sqrt h h^{\tau\tau}}{2\pi\alpha'}\\
    -1&=h^{\tau\tau}\numberthis\label{httcondclosed}
\end{align*}

Furthermore, using the equation of motion \ref{eomx} at $\mu=+$, we get,

\begin{align*}
    0&=\partial_\tau\qty[h^{\tau\sigma}\partial_\sigma X^+h^{\tau\tau}\partial_\tau X^+]+\partial_\sigma\qty[h^{\sigma\sigma}\partial_\sigma X^++h^{\sigma\tau}\partial_\tau X^+]\\
    0&=\partial_\tau h^{\tau\tau}+\partial_\sigma h^{\sigma\tau}\\
    0&=\partial_\sigma h^{\sigma\tau}
\end{align*}

Together with $h^{\tau\sigma}\qty(\tau,\sigma_0)=0$, this simply states that $h^{\tau\sigma}=h^{\sigma\tau}\equiv0$. This, with $h^{\tau\tau}=-1$ 
and the determinant condition, says that $h=\textnormal{Diag}\mqty(-1&1)$. That simplifies the equations of motion \ref{eomx} to,

\begin{align*}
    \partial_\tau\partial_\tau X^\mu-\partial_\sigma\partial_\sigma X^\mu={\ddot X}^\mu-{X''}^\mu=0
\end{align*}

And also the consistency conditions \ref{eomh} as,

\begin{align*}
    0=&\partial_a X^\mu \partial_b X_\mu-\frac12 h_{ab}\qty(-{\dot X}^2+{X'}^2)\\
    0=&\begin{cases}
        \partial_\tau X^\mu \partial_\tau X_\mu+\frac12\qty(-{\dot X}^2+{X'}^2)&=\frac12\qty[{\dot X}^2+{X'}^2]\\
        \partial_\tau X^\mu \partial_\sigma X_\mu&=\dot X\cdot X'\\
        \partial_\sigma X^\mu \partial_\sigma X_\mu-\frac12\qty(-{\dot X}^2+{X'}^2)&=\frac12\qty[{\dot X}^2+{X'}^2]
    \end{cases}
\end{align*}

These constrains can be recast as,

\begin{align*}
    {\dot X}^2+{X'}^2\pm&=0\\
    \qty(\dot X\pm X')^2&=0\numberthis\label{constrains2closed}
\end{align*}

And luckily, we don't need to worry about the boundary condition \ref{boundaryc}, as the $\sigma$ lives in a circle, there is no 
boundary on which we could impose this boundary equations, instead we have another consistency equation, which can be seen as 
an additional boundary condition, as we're trying to solve with $\sigma$ in $\mathbb R$ instead of in a circle,

\begin{align*}
    X^\mu\qty(\tau,\sigma+2\pi)&=X^\mu\qty(\tau,\sigma)
\end{align*}

We have now to solve the equation of motion with the consistency conditions, we already computed the equations of motion 
in the variables $\sigma^\pm=\tau\pm\sigma$, \ref{fulleqom},

\begin{align*}
    \partial_+\partial_-X^\mu&=0
\end{align*}

This is easily solved for,

\begin{align*}
    X^\mu=X_L^\mu\qty(\sigma^+)+X_R^\mu\qty(\sigma^-)
\end{align*}

Imposing the consistency condition,

\begin{align*}
    X^\mu\qty(\tau,\sigma+2\pi)&=X^\mu\qty(\tau,\sigma)\\
    X_L^\mu\qty(\sigma^++2\pi)+X_R^\mu\qty(\sigma^--2\pi)&=X_L^\mu\qty(\sigma^+)+X_R^\mu\qty(\sigma^-)\\
    X_L^\mu\qty(\sigma^++2\pi)-X_L^\mu\qty(\sigma^+)&=X_R^\mu\qty(\sigma^-)-X_R^\mu\qty(\sigma^--2\pi)
\end{align*}

As $\sigma^\pm$ are independent variables,

\begin{align*}
    {X_L'}^\mu\qty(\sigma^++2\pi)-{X_L'}^\mu\qty(\sigma^+)&=0={X_R'}^\mu\qty(\sigma^-)-{X_R'}^\mu\qty(\sigma^--2\pi)
\end{align*}

This states that the two functions $X'_{L,R}$ are both periodic in $2\pi$, hence, we can expand then in Fourier modes, 
we write it as,

\begin{align*}
    {X'_L}^\mu\qty(\sigma^+)&=\sqrt{\frac{\alpha'}{2}}\sum\limits_{n\in\mathbb Z}{\bar\alpha}^\mu_n\exp\qty(-\im n \sigma^+)\\
    {X'_R}^\mu\qty(\sigma^-)&=\sqrt{\frac{\alpha'}{2}}\sum\limits_{n\in\mathbb Z}{\alpha}^\mu_n\exp\qty(-\im n \sigma^-)
\end{align*}

Integrating this equation we get,

\begin{align*}
    X_L^\mu\qty(\sigma^+)&=\frac12 x_{0,L}^\mu+\sqrt{\frac{\alpha'}{2}}{\bar\alpha}_0^\mu\sigma^++\im\sqrt{\frac{\alpha'}{2}}\sum\limits_{n\in\mathbb Z^\ast}\frac{{\bar\alpha}^\mu_n}{n}\exp\qty(-\im n \sigma^+)\\
    X_R^\mu\qty(\sigma^-)&=\frac12 x_{0,R}^\mu+\sqrt{\frac{\alpha'}{2}}\alpha_0^\mu\sigma^-+\im\sqrt{\frac{\alpha'}{2}}\sum\limits_{n\in\mathbb Z^\ast}\frac{{\alpha}^\mu_n}{n}\exp\qty(-\im n \sigma^-)
\end{align*}

Let's see what does the consistency condition of the closed string imply also,

\begin{align*}
    X_L^\mu\qty(\sigma^++2\pi)-X_L^\mu\qty(\sigma^+)&=X_R^\mu\qty(\sigma^-)-X_R^\mu\qty(\sigma^--2\pi)\\
    \sqrt{\frac{\alpha'}{2}}{\bar\alpha}_0^\mu2\pi&=\sqrt{\frac{\alpha'}{2}}{\bar\alpha}_0^\mu2\pi\\
    {\bar\alpha}_0^\mu&=\alpha_0^\mu\numberthis\label{lvmat1}
\end{align*}

This allows us to sum both the parts to obtain the full solution,

\begin{align*}
    X^\mu\qty(\tau,\sigma)&=\frac12\qty(x_{0,L}^\mu+x_{0,R}^\mu)+\sqrt{\frac{\alpha'}{2}}{\bar\alpha}_0^\mu\qty(\sigma^++\sigma^-)+\im\sqrt{\frac{\alpha'}{2}}\sum\limits_{n\in\mathbb Z^\ast}\frac{1}{n}\qty({\bar\alpha}^\mu_n\exp\qty(-\im n \sigma^+)+{\alpha}^\mu_n\exp\qty(-\im n \sigma^-))\\
    X^\mu\qty(\tau,\sigma)&=x_{0}^\mu+\sqrt{2\alpha'}{\bar\alpha}_0^\mu\tau+\im\sqrt{\frac{\alpha'}{2}}\sum\limits_{n\in\mathbb Z^\ast}\frac{\e^{-\im n\tau}}{n}\qty({\bar\alpha}^\mu_n\e^{-\im n\sigma}+{\alpha}^\mu_n\e^{\im n\sigma})
\end{align*}

Notice that,

\begin{align*}
    p^\mu&=\int\limits_{\sigma_0}^{\sigma_0+2\pi}\dd{\sigma}\mathcal P^{\tau\mu}=\int\limits_{\sigma_0}^{\sigma_0+2\pi}\dd{\sigma}\frac{\partial_\tau X^\mu}{2\pi\alpha'},\ \ \ \textnormal{Because }h=\textnormal{Diag}\mqty(-1&1)\\
    p^\mu&=\frac{1}{2\pi\alpha'}\int\limits_{\sigma_0}^{2\pi}\dd{\sigma}\qty[\sqrt{2\alpha'}{\bar\alpha_0}^\mu+\sqrt{\frac{\alpha'}{2}}\sum\limits_{n\in\mathbb Z^\ast}\e^{-\im n\tau}\qty({\bar\alpha}_n^\mu\e^{-\im n\sigma}+{\alpha}_n^\mu\e^{\im n\sigma})]=\sqrt{\frac{2}{\alpha'}}{\bar\alpha}_0^\mu\\
    \alpha_0^\mu&={\bar\alpha}_0^\mu=\sqrt{\frac{\alpha'}{2}}p^\mu
\end{align*}

Rewriting again,

\begin{align*}
    X^\mu\qty(\tau,\sigma)&=x_{0}^\mu+\sqrt{2\alpha'}{\bar\alpha}_0^\mu\tau+\im\sqrt{\frac{\alpha'}{2}}\sum\limits_{n\in\mathbb Z^\ast}\frac{\e^{-\im n\tau}}{n}\qty({\bar\alpha}^\mu_n\e^{-\im n\sigma}+{\alpha}^\mu_n\e^{\im n\sigma})
\end{align*}

This is not the end of it, because we need to make sure the two constrains are satisfied, namely, \ref{constrains2closed}. With the convention of 
uppercase Latin index referring to non-light-cone components, $I=2,\cdots,D-1$, the constrains are,

\begin{align*}
    0&=-2\qty({\dot X}^-\pm{X'}^-)\qty({\dot X}^+\pm{X'}^+)+\qty({\dot X}^I\pm{X'}^I)^2\\
    2\alpha'p^+\qty({\dot X}^-\pm{X'}^-)&=\qty({\dot X}^I\pm{X'}^I)^2\\
    {\dot X}^-\pm{X'}^-&=\frac{1}{2\alpha'p^+}\qty({\dot X}^I\pm{X'}^I)^2
\end{align*}

That is, the two constrains implies that $X^-$ is not dynamical, and, the collection of $X^I$ fully determine $X^-$, 
apart from a single integration constant, $x_0^-$. To see this is helpful to note,

\begin{align*}
    {\dot X}^\mu&=\sqrt{2\alpha'}{\bar\alpha}_0^\mu+\sqrt{\frac{\alpha'}{2}}\sum\limits_{n\in\mathbb Z^\ast}\e^{-\im n\tau}\qty({\bar\alpha}^\mu_n\e^{-\im n\sigma}+{\alpha}^\mu_n\e^{\im n\sigma})\\
    {\dot X}^\mu&=\sqrt{\frac{\alpha'}{2}}\qty(\alpha_0^\mu+{\bar\alpha}_0^\mu)+\sqrt{\frac{\alpha'}{2}}\sum\limits_{n\in\mathbb Z^\ast}\e^{-\im n\tau}\qty({\bar\alpha}^\mu_n\e^{-\im n\sigma}+{\alpha}^\mu_n\e^{\im n\sigma})\\
    {\dot X}^\mu&=\sqrt{\frac{\alpha'}{2}}\sum\limits_{n\in\mathbb Z}\e^{-\im n\tau}\qty({\bar\alpha}^\mu_n\e^{-\im n\sigma}+{\alpha}^\mu_n\e^{\im n\sigma})
\end{align*}

And,

\begin{align*}
    {X'}^\mu&=\sqrt{\frac{\alpha'}{2}}\sum\limits_{n\in\mathbb Z^\ast}\e^{-\im n\tau}\qty({\bar\alpha}^\mu_n\e^{-\im n\sigma}-{\alpha}^\mu_n\e^{\im n\sigma})\\
    {X'}^\mu&=\sqrt{\frac{\alpha'}{2}}\sum\limits_{n\in\mathbb Z}\e^{-\im n\tau}\qty({\bar\alpha}^\mu_n\e^{-\im n\sigma}-{\alpha}^\mu_n\e^{\im n\sigma})
\end{align*}

Which implies,

\begin{align*}
    {\dot X}^\mu+{X'}^\mu&=\sqrt{2\alpha'}\sum\limits_{n\in\mathbb Z}{\bar\alpha}_n^\mu\exp\qty(-\im n\sigma^+)\numberthis\label{+dot}\\
    {\dot X}^\mu-{X'}^\mu&=\sqrt{2\alpha'}\sum\limits_{n\in\mathbb Z}{\alpha}_n^\mu\exp\qty(-\im n\sigma^-)\numberthis\label{-dot}
\end{align*}

So that,

\begin{align*}
    {\dot X}^-+{X'}^-&=\frac{1}{2\alpha'{p^+}}\qty({\dot X}^I+{X'}^I)^2\\
    \sqrt{2\alpha'}\sum\limits_{n\in\mathbb Z}{\bar\alpha}^-_n\exp\qty(-\im n\sigma^+)&=\frac{2\alpha'}{2\alpha' p^+}\sum\limits_{p,q\in\mathbb Z}{\bar\alpha}^I_p{\bar\alpha}^I_q\exp\qty(-\im\qty(p+q)\sigma^+)\\
    \sqrt{2\alpha'}\sum\limits_{n\in\mathbb Z}{\bar\alpha}^-_n\exp\qty(-\im n\sigma^+)&=\frac{1}{p^+}\sum\limits_{n,p\in\mathbb Z}{\bar\alpha}^I_p{\bar\alpha}^I_{n-p}\exp\qty(-\im n\qty(\tau\pm\sigma))\\
    \sqrt{2\alpha'}\sum\limits_{n\in\mathbb Z}{\bar\alpha}^-_n\exp\qty(-\im n\sigma^+)&=\frac{2}{p^+}\sum\limits_{n\in\mathbb Z}\qty(\frac12\sum\limits_{p\in\mathbb Z}{\bar\alpha}^I_p{\bar\alpha}^I_{n-p})\exp\qty(-\im n\qty(\tau\pm\sigma))\\
    \sqrt{2\alpha'}{\bar\alpha}_n^-&=\frac{1}{p^+}\sum\limits_{p\in\mathbb Z}\alpha_p^I\alpha_{n-p}^I=\frac{2}{p^+}{\bar L}_n^\perp\numberthis\label{virasorodefbarclosed}
\end{align*}

And,

\begin{align*}
    {\dot X}^-\pm{X'}^-&=\frac{1}{2\alpha'{p^+}}\qty({\dot X}^I\pm{X'}^I)^2\\
    \sqrt{2\alpha'}\sum\limits_{n\in\mathbb Z}\alpha^-_n\exp\qty(-\im n\qty(\tau\pm\sigma))&=\frac{2\alpha'}{2\alpha' p^+}\sum\limits_{p,q\in\mathbb Z}\alpha^I_p\alpha^I_q\exp\qty(-\im\qty(p+q)\qty(\tau\pm\sigma))\\
    \sqrt{2\alpha'}\sum\limits_{n\in\mathbb Z}\alpha^-_n\exp\qty(-\im n\qty(\tau\pm\sigma))&=\frac{1}{p^+}\sum\limits_{n,p\in\mathbb Z}\alpha^I_p\alpha^I_{n-p}\exp\qty(-\im n\qty(\tau\pm\sigma))\\
    \sqrt{2\alpha'}\sum\limits_{n\in\mathbb Z}\alpha^-_n\exp\qty(-\im n\qty(\tau\pm\sigma))&=\frac{2}{p^+}\sum\limits_{n\in\mathbb Z}\qty(\frac12\sum\limits_{p\in\mathbb Z}\alpha^I_p\alpha^I_{n-p})\exp\qty(-\im n\qty(\tau\pm\sigma))\\
    \sqrt{2\alpha'}\alpha_n^-&=\frac{1}{p^+}\sum\limits_{p\in\mathbb Z}\alpha_p^I\alpha_{n-p}^I=\frac{2}{p^+}L_n^\perp\numberthis\label{virasorodefclosed}
\end{align*}

Hence, all the Fourier modes of $X^-$ are completely determined by the transverse modes ones, apart from of course the integration constant $x^-_0$. In the 
last passage we also defined the Virasoro modes $L_n^\perp$. This is it, we fully solved the equation of motion with all the constrains and boundary 
conditions, but there's a catch, imposing $\sigma$ lives in a circle gives \ref{lvmat1}, which for $\mu=-$ implies a non-trivial constrain,

\begin{align*}
    {\bar L}^\perp_0=L^\perp_0\numberthis\label{lvmat2}
\end{align*}

As a consequence we get the full set $\alpha_n^I,{\bar\alpha}_n^I$ cannot be chosen freely. Let's rewrite here the full solution for completeness,

\begin{align*}
    X^I&=x_{0}^I+\sqrt{2\alpha'}{\bar\alpha}_0^I\tau+\im\sqrt{\frac{\alpha'}{2}}\sum\limits_{n\in\mathbb Z^\ast}\frac{\e^{-\im n\tau}}{n}\qty({\bar\alpha}^I_n\e^{-\im n\sigma}+{\alpha}^I_n\e^{\im n\sigma})\\
    X^-&=x_{0}^-+\frac{2}{p^+}L^\perp_0\tau+\frac{\im}{p^+}\sum\limits_{n\in\mathbb Z^\ast}\frac{\e^{-\im n\tau}}{n}\qty({\bar L}^\perp_n\e^{-\im n\sigma}+{L}^\perp_n\e^{\im n\sigma})\\
    X^+&=\alpha'p^+\tau\\
    L^\perp_0&={\bar L}^\perp_0
\end{align*}


\end{document}