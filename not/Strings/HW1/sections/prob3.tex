\section{Problem 3}
\subsection{3.A)}

The Gamma Function can be represented in the complex plane domain, $\textnormal{Re}\qty(s)>1$, as the following integral,

\begin{align*}
    \Gamma(s)&=\int\limits_0^\infty\dd{t}\exp\qty(-t)t^{s-1},\ \ \ \textnormal{Re}\qty(s)>1
\end{align*}

Which is also the subset of the complex plane in which this integral converges, of course this representation of the Gamma Function 
in a open set is sufficient for obtain an analytical continuation to the whole complex plane. Obviously, the integral is invariant under 
relabeling the dummy variable $t$, we make the following choice $t\rightarrow nt$ --- Assuming $n>0$ ---,

\begin{align*}
    \Gamma(s)&=\int\limits_0^\infty\dd{\qty(nt)}\exp\qty(-nt)\qty(nt)^{s-1},\ \ \ \textnormal{Re}\qty(s)>1\\
    \Gamma(s)&=n^s\int\limits_0^\infty\dd{t}\exp\qty(-nt)t^{s-1},\ \ \ \textnormal{Re}\qty(s)>1\\
    n^{-s}\Gamma\qty(s)&=\int\limits_0^\infty\dd{t}\exp\qty(-nt)t^{s-1},\ \ \ \textnormal{Re}\qty(s)>1\\
    \sum\limits_{n=1}^\infty n^{-s}\Gamma\qty(s)&=\sum\limits_{n=1}^\infty\int\limits_0^\infty\dd{t}\exp\qty(-nt)t^{s-1},\ \ \ \textnormal{Re}\qty(s)>1
\end{align*}    

The sum in the left-hand side is recognized as the representation for the Zeta Function in the domain $\textnormal{Re}\qty(s)>1$, which is also 
the domain of convergence of the sum,

\begin{align*}
    \zeta(s)\Gamma\qty(s)&=\sum\limits_{n=1}^\infty\int\limits_0^\infty\dd{t}\exp\qty(-nt)t^{s-1},\ \ \ \textnormal{Re}\qty(s)>1\\
\end{align*}

About the right-hand side, to be able to exchange the integral and the sum is sufficient that,

\begin{align*}
    &\int\limits_0^\infty\dd{t}\sum\limits_{n=1}^\infty\norm{\exp\qty(-nt)t^{s-1}}<\infty,\ \ \ \textnormal{Re}\qty(s)>1\\
    &\int\limits_0^\infty\dd{t}\sum\limits_{n=1}^\infty\exp\qty(-nt)\norm{t^{s-1}}<\infty,\ \ \ \textnormal{Re}\qty(s)>1\\
    &\int\limits_0^\infty\dd{t}\sum\limits_{n=1}^\infty\exp\qty(-nt)t^{\textnormal{Re}\qty(s)-1}<\infty,\ \ \ \textnormal{Re}\qty(s)>1
\end{align*}

The sum now is a simple geometric series, giving,

\begin{align*}
    &\int\limits_0^\infty\dd{t}\frac{t^{\textnormal{Re}\qty(s)-1}}{\exp\qty(t)-1}<\infty,\ \ \ \textnormal{Re}\qty(s)>1
\end{align*}

The dangerous behavior that could make the integral diverges is the one at $t\rightarrow0$, thankfully, as $\textnormal{Re}\qty(s)>1$, the 
possible zero in the denominator is cancelled by the stronger zero in the numerator, hence, the integral converges. That is, 
is permissible to do the exchange of the sum and integral, and we get,

\begin{align*}
    \zeta(s)\Gamma\qty(s)&=\int\limits_0^\infty\dd{t}\sum\limits_{n=1}^\infty\exp\qty(-nt)t^{s-1},\ \ \ \textnormal{Re}\qty(s)>1\\
    \zeta(s)\Gamma\qty(s)&=\int\limits_0^\infty\dd{t}\frac{t^{s-1}}{\exp\qty(t)-1},\ \ \ \textnormal{Re}\qty(s)>1
\end{align*}

\subsection{3.B)}

The objective here is to make an analytical continuation to $\textnormal{Re}\qty(s)>-2$ of the expression found in the later item. 
First of all, the reason the later expression is only well defined in $\textnormal{Re}\qty(s)>1$, is due to the divergence of the 
integrand at $t\rightarrow 0$ for $\textnormal{Re}\qty(s)\leq1$

\begin{align*}
    \zeta(s)\Gamma\qty(s)&=\int\limits_0^1\dd{t}\frac{t^{s-1}}{\exp\qty(t)-1}+\int\limits_1^\infty\dd{t}\frac{t^{s-1}}{\exp\qty(t)-1},\ \ \ \textnormal{Re}\qty(s)>1\\
    \zeta(s)\Gamma\qty(s)&=\int\limits_0^1\dd{t}t^{s-1}\qty[\frac{1}{\exp\qty(t)-1}-\frac1t]+\int\limits_1^\infty\dd{t}\frac{t^{s-1}}{\exp\qty(t)-1},\ \ \ \textnormal{Re}\qty(s)>1\\
\end{align*}

\subsection{3.C)}