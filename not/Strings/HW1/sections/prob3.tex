\problem{}
\subsection{}

The Gamma Function can be represented in the complex plane domain, $\textnormal{Re}\qty(s)>1$, as the following integral,

\begin{align*}
    \Gamma(s)&=\int\limits_0^\infty\dd{t}\exp\qty(-t)t^{s-1},\ \ \ \textnormal{Re}\qty(s)>1\numberthis\label{gamma}
\end{align*}

Which is also the subset of the complex plane in which this integral converges, of course this representation of the Gamma Function 
in a open set is sufficient for obtain an analytical continuation to the whole complex plane. Obviously, the integral is invariant under 
relabeling the dummy variable $t$, we make the following choice $t\rightarrow nt$ --- Assuming $n>0$ ---,

\begin{align*}
    \Gamma(s)&=\int\limits_0^\infty\dd{\qty(nt)}\exp\qty(-nt)\qty(nt)^{s-1},\ \ \ \textnormal{Re}\qty(s)>1\\
    \Gamma(s)&=n^s\int\limits_0^\infty\dd{t}\exp\qty(-nt)t^{s-1},\ \ \ \textnormal{Re}\qty(s)>1\\
    n^{-s}\Gamma\qty(s)&=\int\limits_0^\infty\dd{t}\exp\qty(-nt)t^{s-1},\ \ \ \textnormal{Re}\qty(s)>1\\
    \sum\limits_{n=1}^\infty n^{-s}\Gamma\qty(s)&=\sum\limits_{n=1}^\infty\int\limits_0^\infty\dd{t}\exp\qty(-nt)t^{s-1},\ \ \ \textnormal{Re}\qty(s)>1
\end{align*}    

The sum in the left-hand side is recognized as the representation for the Zeta Function in the domain $\textnormal{Re}\qty(s)>1$, which is also 
the domain of convergence of the sum,

\begin{align*}
    \zeta\qty(s)=\sum\limits_{n=1}^\infty n^{-s},\ \ \ \textnormal{Re}\qty(s)>1
\end{align*}

So that,

\begin{align*}
    \zeta(s)\Gamma\qty(s)&=\sum\limits_{n=1}^\infty\int\limits_0^\infty\dd{t}\exp\qty(-nt)t^{s-1},\ \ \ \textnormal{Re}\qty(s)>1
\end{align*}

About the right-hand side, to be able to exchange the integral and the sum is sufficient that,

\begin{align*}
    &\int\limits_0^\infty\dd{t}\sum\limits_{n=1}^\infty\norm{\exp\qty(-nt)t^{s-1}}<\infty,\ \ \ \textnormal{Re}\qty(s)>1\\
    &\int\limits_0^\infty\dd{t}\sum\limits_{n=1}^\infty\exp\qty(-nt)\norm{t^{s-1}}<\infty,\ \ \ \textnormal{Re}\qty(s)>1\\
    &\int\limits_0^\infty\dd{t}\sum\limits_{n=1}^\infty\exp\qty(-nt)t^{\textnormal{Re}\qty(s)-1}<\infty,\ \ \ \textnormal{Re}\qty(s)>1
\end{align*}

The sum now is a simple geometric series, giving,

\begin{align*}
    &\int\limits_0^\infty\dd{t}\frac{t^{\textnormal{Re}\qty(s)-1}}{\exp\qty(t)-1}<\infty,\ \ \ \textnormal{Re}\qty(s)>1
\end{align*}

The dangerous behavior that could make the integral diverges is the one at $t\rightarrow0$, an indeed, $\textnormal{Re}\qty(s)>1$, is 
sufficient for the convergence of this integral, which can be seen at,

\begin{align*}
    \int\limits_0^\epsilon\dd{t}\frac{t^{\textnormal{Re}\qty(s)-1}}{\exp\qty(t)-1}\approx\int\limits_0^\epsilon\dd{t}\frac{t^{\textnormal{Re}\qty(s)-1}}{t+\mathcal O\qty(t^2)}\approx\int\limits_0^\epsilon t^{\textnormal{Re}\qty(s)-2}=\frac{t^{\textnormal{Re}\qty(s)-1}}{\textnormal{Re}\qty(s)-1}\eval_0^\epsilon
\end{align*}

Which shows the integral is really finite at $t\rightarrow 0$ with $\textnormal{Re}\qty(s)>1$, hence, switching the integral and the 
sum is justified, so,

\begin{align*}
    \zeta(s)\Gamma\qty(s)&=\sum\limits_{n=1}^\infty\int\limits_0^\infty\dd{t}\exp\qty(-nt)t^{s-1},\ \ \ \textnormal{Re}\qty(s)>1\\
    \zeta(s)\Gamma\qty(s)&=\dd{t}\int\limits_0^\infty\sum\limits_{n=1}^\infty\exp\qty(-nt)t^{s-1},\ \ \ \textnormal{Re}\qty(s)>1
\end{align*}

Where again we have the sum of a geometric series, giving,

\begin{align*}
    \zeta(s)\Gamma\qty(s)&=\int\limits_0^\infty\dd{t}\frac{t^{s-1}}{\exp\qty(t)-1},\ \ \ \textnormal{Re}\qty(s)>1
\end{align*}

\subsection{}

The objective here is to make an analytical continuation to $\textnormal{Re}\qty(s)>-2$ of the expression found in the later item. 
First of all, the reason the later expression is only well defined in $\textnormal{Re}\qty(s)>1$, is due to the divergence of the 
integrand at $t\rightarrow 0$ for $\textnormal{Re}\qty(s)\leq1$, this is only because $\qty(\exp\qty(t)-1)^{-1}$ has a simple pole at $t=0$, which 
is also the only pole of this function, so to get the Laurent series we first find the residue of it,

\begin{align*}
    \textnormal{Res}_{t=0}\qty(\frac{1}{\exp\qty(t)-1})&=\frac{t}{\exp\qty(t)-1}\eval_{t=0}\\
    \textnormal{Res}_{t=0}\qty(\frac{1}{\exp\qty(t)-1})&=\frac{t}{t+\mathcal O\qty(t^2)}\eval_{t=0}\\
    \textnormal{Res}_{t=0}\qty(\frac{1}{\exp\qty(t)-1})&=\frac{1}{1+\mathcal O\qty(t)}\eval_{t=0}\\
    \textnormal{Res}_{t=0}\qty(\frac{1}{\exp\qty(t)-1})&=1
\end{align*}

As this is the only pole, we get a Laurent series starting as,

\begin{align*}
    \frac{1}{\exp\qty(t)-1}&=\frac1t+\mathcal O\qty(t^0)
\end{align*}

To get the following terms we just make a trivial Taylor series of the function $\qty(\exp\qty(t)-1)^{-1}-t^{-1}$

\begin{align*}
    \qty[\frac{1}{\exp\qty(t)-1}-\frac1t]\eval_0&=\frac{1+t-\exp\qty(t)}{t\qty[\exp\qty(t)-1]}\eval_0\\
    \qty[\frac{1}{\exp\qty(t)-1}-\frac1t]\eval_0&=\frac{-\frac{t^2}{2}+\mathcal O\qty(t^3)}{t\qty[t+\mathcal O\qty(t^2)]}\eval_0\\
    \qty[\frac{1}{\exp\qty(t)-1}-\frac1t]\eval_0&=\frac{-\frac{t^2}{2}+\mathcal O\qty(t^3)}{t^2\qty[1+\mathcal O\qty(t)]}\eval_0\\
    \qty[\frac{1}{\exp\qty(t)-1}-\frac1t]\eval_0&=-\frac12
\end{align*}

In other words,

\begin{align*}
    \frac{1}{\exp\qty(t)-1}&=\frac1t-\frac12+\mathcal O\qty(t)
\end{align*}

The next term of the series will be,

\begin{align*}
    \dv{}{t}\qty[\frac{1}{\exp\qty(t)-1}-\frac1t]\eval_0&=\frac{1}{t^2}-\frac{\exp\qty(t)}{\qty[\exp\qty(t)-1]^2}\\
    &=\frac{\exp\qty(t)+\exp\qty(-t)-2-t^2}{t^2\qty[\exp\qty(t)+\exp\qty(-t)-2]}\eval_0\\
    &=\frac{2\frac{t^4}{4!}+\mathcal O\qty(t^6)}{t^2\qty[t^2+\mathcal O\qty(t^4)]}\eval_0\\
    &=\frac{1}{12}\frac{t^4+\mathcal O\qty(t^6)}{t^4\qty[1+\mathcal O\qty(t^2)]}\eval_0\\
    &=\frac{1}{12}
\end{align*}

So up to first order we have,

\begin{align*}
    \frac{1}{\exp\qty(t)-1}&=\frac1t-\frac12+\frac{t}{12}+\mathcal O\qty(t^2)\numberthis\label{laurent}
\end{align*}

Why have we done this? Because we do can soften the behavior of the integrand near $t\rightarrow0$ if we 
subtract leading terms of the expansion of $\qty(\exp\qty(t)-1)^{-1}$, each leading term that we subtract, is 
equivalent to gaining a power of $t$ in the numerator, which does soften the behavior near $t\rightarrow 0$, but also 
makes it worse in the region $t\rightarrow\infty$, and as our only problem is related with the small $t$ region, 
we can divide the integral in two parts,

\begin{align*}
    \zeta(s)\Gamma\qty(s)&=\int\limits_0^1\dd{t}\frac{t^{s-1}}{\exp\qty(t)-1}+\int\limits_1^\infty\dd{t}\frac{t^{s-1}}{\exp\qty(t)-1},\ \ \ \textnormal{Re}\qty(s)>1\\
    \zeta(s)\Gamma\qty(s)&=\int\limits_0^1\dd{t}t^{s-1}\qty[\frac{1}{\exp\qty(t)-1}-\frac1t+\frac12-\frac{t}{12}+\frac1t-\frac12+\frac{t}{12}]+\int\limits_1^\infty\dd{t}\frac{t^{s-1}}{\exp\qty(t)-1},\ \ \ \textnormal{Re}\qty(s)>1
\end{align*}

Where we simply added and subtracted the leading terms of the expansion, the integral of the last three of them is 
trivial and can be done to give,

\begin{align*}
    \zeta(s)\Gamma\qty(s)&=\int\limits_0^1\dd{t}t^{s-1}\qty[\frac{1}{\exp\qty(t)-1}-\frac1t+\frac12-\frac{t}{12}]+\int\limits_0^1\dd{t}\qty[t^{s-2}-\frac{t^{s-1}}{2}+\frac{t^s}{12}]+\int\limits_1^\infty\dd{t}\frac{t^{s-1}}{\exp\qty(t)-1}\\
    \zeta(s)\Gamma\qty(s)&=\int\limits_0^1\dd{t}t^{s-1}\qty[\frac{1}{\exp\qty(t)-1}-\frac1t+\frac12-\frac{t}{12}]+\frac{1}{s-1}-\frac{1}{2s}+\frac{1}{12\qty(s+1)}+\int\limits_1^\infty\dd{t}\frac{t^{s-1}}{\exp\qty(t)-1}   
\end{align*}

Just what we wanted.

\subsection{}

Naively, this last expression should be well defined only for $\textnormal{Re}\qty(s)>1$, let's see this term by term, starting by the last one,

\begin{align*}
    \int\limits_1^\infty\dd{t}\frac{t^{s-1}}{\exp\qty(t)-1}
\end{align*}

This is finite for all $s$, as it is exponentially decaying and is bounded in the integration interval, this term is well defined for all $s$. 
The next three ones are,

\begin{align*}
    \frac{1}{s-1}-\frac{1}{2s}+\frac{1}{12\qty(s+1)}
\end{align*}

Also these are well defined in the whole complex plane, with three poles at $s=-1,0,1$. Finally we have,

\begin{align*}
    \int\limits_0^1\dd{t}t^{s-1}\qty[\frac{1}{\exp\qty(t)-1}-\frac1t+\frac12-\frac{t}{12}]
\end{align*}

The only potential not well defined behavior that can occur is near $t=0$, but we have already developed a 
series expansion for the expression in brackets, \ref{laurent}, that means, near the critical value of $t=0$, 
the integrand goes like,

\begin{align*}
    \int\limits_0^1\dd{t}t^{s-1}\qty[\frac{1}{\exp\qty(t)-1}-\frac1t+\frac12-\frac{t}{12}]\approx\int\limits_0^1\dd{t}t^{s-1}\mathcal O\qty(t^2)\approx \int\limits_0^1\dd{t}t^{s+1}=\frac{t^{s+2}}{s+2}\eval_0^1
\end{align*}

This is well defined as long as $\textnormal{Re}\qty(s)>-2$. Hence, the expression,

\begin{align*}
    \zeta(s)\Gamma\qty(s)&=\int\limits_0^1\dd{t}t^{s-1}\qty[\frac{1}{\exp\qty(t)-1}-\frac1t+\frac12-\frac{t}{12}]+\frac{1}{s-1}-\frac{1}{2s}+\frac{1}{12\qty(s+1)}+\int\limits_1^\infty\dd{t}\frac{t^{s-1}}{\exp\qty(t)-1}\numberthis\label{zetag}   
\end{align*}

Is well defined as long as $\textnormal{Re}\qty(s)>-2$. One might worry about the poles, but, these are the natural structure of $\zeta\qty(s)\Gamma\qty(s)$, 
to be well defined does not mean to don't have poles, but means the representation can be assigned a number in an unique manner. 
What is left to us now is to find the values $\zeta\qty(0),\zeta(-1)$, notice that our representation has a poles in both these 
values of the argument, in fact these poles are structures of $\Gamma\qty(s)$, and not $\zeta(s)$. That the Gamma Function indeed has poles in those values 
can be seen from,

\begin{align*}
    \Gamma\qty(s+1)&=s\Gamma\qty(s)\Rightarrow\begin{cases}\Gamma\qty(0)&=\frac{\Gamma\qty(1)}{0}\\\Gamma\qty(-1)&=\frac{\Gamma\qty(0)}{-1}\end{cases}
\end{align*}

And because the poles in our representation are just simple poles, they could not have been poles also in $\zeta$, as the two functions are multiplying 
if there were a pole in $\zeta(0),\zeta(-1)$ they would have been apparent in our representation as double poles. Due to the absence of those, the poles at $s=0,-1$ are indeed only due to the Gamma Function. 
This guarantees us that $\zeta(0),\zeta(-1)$ are both finite, and to determine those we just need to evaluate the residue of the expression. First, the residue of the Gamma Function,

\begin{align*}
    \textnormal{Res}_{s=0}\qty(\Gamma\qty(s))&=s\Gamma\qty(s)\eval_{s=0}=\Gamma\qty(s+1)\eval_{s=0}=\Gamma\qty(1)=1\\
    \textnormal{Res}_{s=-1}\qty(\Gamma\qty(s))&=\qty(s+1)\Gamma\qty(s)\eval_{s=-1}=\frac{\qty(s+1)s\Gamma\qty(s)}{s}\eval_{s=-1}=\frac{\Gamma\qty(s+2)}{s}\eval_{s=-1}=-1
\end{align*}

As we argued that $\zeta(0),\zeta(-1)$ should be finite, what will happen is that when we multiply $\zeta$ by $\Gamma$, the residues of the poles of the Gamma Function will be 
multiplied by the value of the Zeta Function at that point, that is,

\begin{align*}
    \textnormal{Res}_{s=0}\qty(\zeta\qty(s)\Gamma\qty(s))&=\zeta\qty(0)\textnormal{Res}_{s=0}\qty(\Gamma\qty(s))
\end{align*}

But, as can be seen directly from \ref{zetag}, the only contribution for the residue at $s=0$ will be by $-\frac{1}{2s}$, as all the other terms are finite at $s=0$, 
thus,

\begin{align*}
    \textnormal{Res}_{s=0}\qty(\zeta\qty(s)\Gamma\qty(s))&=-\frac12=\zeta\qty(0)\textnormal{Res}_{s=0}\qty(\Gamma\qty(s))=\zeta(0)\\
    \zeta\qty(0)&=-\frac12
\end{align*}

Analogously we have,

\begin{align*}
    \textnormal{Res}_{s=-1}\qty(\zeta\qty(s)\Gamma\qty(s))&=\zeta\qty(-1)\textnormal{Res}_{s=-1}\qty(\Gamma\qty(s))
\end{align*}

Again, as we discussed previously, all the terms are finite at $s=-1$, except for $\frac{1}{12\qty(s+1)}$, hence, the residue will be,

\begin{align*}
    \textnormal{Res}_{s=-1}\qty(\zeta\qty(s)\Gamma\qty(s))&=\frac{1}{12}=\zeta\qty(-1)\textnormal{Res}_{s=-1}\qty(\Gamma\qty(s))=-\zeta\qty(-1)\\
    \zeta(-1)&=-\frac{1}{12}
\end{align*}

As desired.