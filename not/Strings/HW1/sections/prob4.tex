\section{Problem 4}
\subsection{4.A)}

First, one remark, we're going to derive everything in this problem for the \textbf{open} string with 
Neumann Boundary Conditions at both ends. To get the classical solution we just have to solve both
\ref{eomx} and \ref{eomh}. Of course, we do know the Polyakov Action has  Diff$\times$Weyl gauge freedom, hence, 
none of the components of the metric $h$ are in fact dynamical, because by fixing this gauge freedom we actually 
fix all the components of $h$, what makes of \ref{eomh} not an equation of motion for $h$, but merely a set of 
constrains regarding the choice of gauge for $h$. But, fixing all three components of the metric isn't the 
only possible choice for picking a gauge, another option is to fix $\tau$, and two of the components 
of $h$, as we have in total 3 gauge degrees of freedom. We could in principle assign any function to $\tau$, but 
what seems to be the optimal choice is to work in the Light-cone gauge. First, for any target space vector we define,

\begin{align*}
    A^\pm=\frac{1}{\sqrt2}\qty(A^0+A^1)
\end{align*}

And now we state the Light-cone gauge condition, which is the pining $\tau$ condition, is then given by,

\begin{align*}
    X^+\qty(\tau,\sigma)=2\alpha'p^+\tau
\end{align*}

Where $p^+$ is the light-cone component of the conserved charge associated with target space translations --- the momentum, \ref{p} ---. This 
condition do not allow for any more reparametrizations of $\tau$, as,

\begin{align*}
    {X'}^+\qty(\tau',\sigma')&=X^+\qty(\tau,\sigma)\\
    2\alpha'p^+\tau'&=2\alpha'p^+\tau
\end{align*}

Which states the only reparametrization compatible with the light-cone gauge condition is $\tau'\qty(\tau,\sigma)=\tau$. Hence, from our 
initial freedom we can only make $\sigma$ reparametrizations now. One way of fixing this remaining reparametrization, is to choose $\sigma$ proportional 
to the energy carried by the string, actually, we're going to choose a reparametrization such that the energy density in the string, at fixed $\tau$, is independent of 
$\sigma$. One might suspect that the energy density is given by $\mathcal P^{\tau0}$, as in \ref{defcalp}, but, in the light-cone coordinates, the one that has 
the role of energy density is $\mathcal P^{\tau+}$, we'll choose a sigma parametrization such that $\partial_\sigma\mathcal P^{\tau+}=0$, to see how can this 
be done we take the transformation of this object under a reparametrization, $\tau'=\tau,\sigma'=\sigma'\qty(\sigma,\tau)$, as,

\begin{align*}
    \mathcal P^{\tau+}&=-\frac{\sqrt h\partial^\tau X^+}{2\pi\alpha'}=-\frac{\sqrt h h^{\tau\tau}}{2\pi\alpha'}2\alpha'p^+=-\sqrt h h^{\tau\tau}\frac{p^+}{\pi}\\
    {\mathcal P'}^{\tau+}\qty(\tau,\sigma')&=-\pdv{\sigma}{\sigma'}\sqrt h h^{\tau\tau}\frac{p^+}{\pi}=\pdv{\sigma}{\sigma'}\mathcal P^{\tau+}
\end{align*}

Hence, given ${\mathcal P'}^{\tau+}\qty(\tau,\sigma')$, we can choose $\sigma=\sigma\qty(\tau,\sigma')$, so that $\partial_\sigma\mathcal P^{\tau+}=0$, as long as,

\begin{align*}
    \pdv{}{\sigma}\qty[{\pdv{\sigma'}{\sigma}}{\mathcal P'}^{\tau+}]&=0\Rightarrow{\pdv[2]{\sigma'}{\sigma}}{\mathcal P'}^{\tau+}+\qty(\pdv{\sigma'}{\sigma})^2\pdv{{\mathcal P'}^{\tau+}}{\sigma'}=0
\end{align*}

This is simply a differential equation which can be solved. But this really fixes all the $\sigma$ reparametrization? Let's see if it's possible to 
make further changes of $\sigma\rightarrow \sigma''$ preserving this condition,

\begin{align*}
    {\pdv{}{\sigma}}\qty[{\pdv{\sigma''}{\sigma}}{\pdv{\sigma'}{\sigma''}}]{\mathcal P'}^{\tau+}+\qty(\pdv{\sigma'}{\sigma''}\pdv{\sigma''}{\sigma})^2\pdv{{\mathcal P'}^{\tau+}}{\sigma'}&=0\\
    {\pdv[2]{\sigma''}{\sigma}}{\pdv{\sigma'}{\sigma''}}{\mathcal P'}^{\tau+}+\qty(\pdv{\sigma''}{\sigma})^2\qty[{\pdv[2]{{\sigma'}}{{\sigma''}}}{\mathcal P'}^{\tau+}
    +\qty(\pdv{\sigma'}{\sigma''})^2\pdv{{\mathcal P'}^{\tau+}}{\sigma'}
    ]&=0
\end{align*}

Imposing that $\sigma''$ also satisfy the parametrization condition,

\begin{align*}
    {\pdv[2]{\sigma''}{\sigma}}{\pdv{\sigma'}{\sigma''}}{\mathcal P'}^{\tau+}&=0\Rightarrow {\pdv[2]{\sigma''}{\sigma}}=0\Rightarrow \sigma''=a\sigma+b
\end{align*}

Our condition of constancy of $\mathcal P^{\tau+}$ in sigma, has a residual affine reparametrization, which we can use to set $\sigma=0$ in one of the ends of the 
string, and $\sigma=\pi$ in the other one. This completely fixes the $\sigma$ parametrization. What remains now is to fix the Weyl redundancy, as necessarily 
$h_{ab}$ has a inverse, $\Det\qty[h_{ab}]\neq 0$, thus, by a Weyl transformation is always possible to make $\Det\qty[h_{ab}]=-1$ --- We're using Lorentzian signature ---, which 
will be our choice of fixing the Weyl redundancy. There is 
no more room for any transformation now, hence, the gauge is fully fixed. Notice, as $\mathcal P^{\tau+}$ is constant in $\sigma$, we have the following,

\begin{align*}
    p^+&=\int\limits_0^\pi\dd{\sigma}\mathcal P^{\tau+}=\mathcal P^{\tau+}\pi=-p^+\sqrt h h^{\tau\tau}\\
    -1&=h^{\tau\tau}
\end{align*}

Furthermore, using the equation of motion \ref{eomx} at $\mu=+$, we get,

\begin{align*}
    0&=\partial_\tau\qty[h^{\tau\sigma}\partial_\sigma X^+h^{\tau\tau}\partial_\tau X^+]+\partial_\sigma\qty[h^{\sigma\sigma}\partial_\sigma X^++h^{\sigma\tau}\partial_\tau X^+]\\
    0&=\partial_\tau h^{\tau\tau}+\partial_\sigma h^{\sigma\tau}\\
    0&=\partial_\sigma h^{\sigma\tau}
\end{align*}

But, using the Neumann Boundary conditions \ref{boundaryc} at $\mu=+$,

\begin{align*}
    0&=\partial^\sigma X^+\eval_{\sigma=0}^{\sigma=\pi}\\
    0&=h^{\sigma\tau}\partial_\tau X^++h^{\sigma\sigma}\partial_\sigma X^+\eval_{\sigma=0}^{\sigma=\pi}\\
    0&=h^{\tau\sigma}\eval_{\sigma=0}^{\sigma=\pi}\Rightarrow h^{\tau\sigma}\eval_{\sigma=0}=h^{\tau\sigma}\eval_{\sigma=\pi}=0
\end{align*}

Together with $\partial_\sigma h^{\tau\sigma}=0$, this simply states that $h^{\tau\sigma}=h^{\sigma\tau}\equiv0$. This, with $h^{\tau\tau}=-1$, 
and with the determinant condition, says that $h=\textnormal{Diag}\mqty(-1&1)$. That simplifies the equations of motion \ref{eomx} to,

\begin{align*}
    \partial_\tau\partial_\tau X^\mu-\partial_\sigma\partial_\sigma X^\mu={\ddot X}^\mu-{X''}^\mu=0
\end{align*}

And also the consistency conditions \ref{eomh} as,

\begin{align*}
    0=&\partial_a X^\mu \partial_b X_\mu-\frac12 h_{ab}\qty(-{\dot X}^2+{X'}^2)\\
    0=&\begin{cases}
        \partial_\tau X^\mu \partial_\tau X_\mu+\frac12\qty(-{\dot X}^2+{X'}^2)&=\frac12\qty[{\dot X}^2+{X'}^2]\\
        \partial_\tau X^\mu \partial_\sigma X_\mu&=\dot X\cdot X'\\
        \partial_\sigma X^\mu \partial_\sigma X_\mu-\frac12\qty(-{\dot X}^2+{X'}^2)&=\frac12\qty[{\dot X}^2+{X'}^2]
    \end{cases}
\end{align*}

These constrains can be recast as,

\begin{align*}
    \qty(\dot X\pm X')^2=0\numberthis\label{constrains}
\end{align*}

Lastly, but not less important, we have the Boundary Conditions, \ref{boundaryc}, which take the form,

\begin{align*}
    \partial_\sigma X^\mu\eval_{\sigma=0}=\partial_\sigma X^\mu\eval_{\sigma=\pi}=0
\end{align*}

We have now to solve the equation of motion with the right Boundary conditions, for this, is easier to change 
coordinates to $\sigma^\pm=\tau\pm\sigma$,

\begin{align*}
    0&=\pdv[2]{}{\tau} X^\mu-\pdv[2]{}{\sigma}X^\mu\\
    0&=\pdv{}{\tau}\qty[\pdv{\sigma^+}{\tau}\pdv{}{\sigma^+}X^\mu+\pdv{\sigma^-}{\tau}\pdv{}{\sigma^-}X^\mu]-\pdv{}{\sigma}\qty[\pdv{\sigma^+}{\sigma}\pdv{}{\sigma^+}X^\mu+\pdv{\sigma^-}{\sigma}\pdv{}{\sigma^-}X^\mu]\\
    0&=\pdv{}{\tau}\qty[\pdv{}{\sigma^+}X^\mu+\pdv{}{\sigma^-}X^\mu]-\pdv{}{\sigma}\qty[\pdv{}{\sigma^+}X^\mu-\pdv{}{\sigma^-}X^\mu]\\
    0&=\qty(\pdv{}{\sigma^+}+\pdv{}{\sigma^-})\qty[\pdv{}{\sigma^+}X^\mu+\pdv{}{\sigma^-}X^\mu]-\qty(\pdv{}{\sigma^+}-\pdv{}{\sigma^-})\qty[\pdv{}{\sigma^+}X^\mu-\pdv{}{\sigma^-}X^\mu]\\
    0&=\partial_+\partial_-X^\mu
\end{align*}

This is easily solved for,

\begin{align*}
    X^\mu=\frac12\qty(f^\mu\qty(\sigma^+)+g^\mu\qty(\sigma^-))
\end{align*}

Imposing the boundary condition at $\sigma=0$,

\begin{align*}
    {X'}^\mu\qty(\tau,\sigma=0)&=\frac12\qty({f'}^\mu\qty(\tau)-{g'}^\mu\qty(\tau))=0
\end{align*}

This states that $g^\mu$ is equal to $f^\mu$ apart from a constant, which we'll absorb in the definition of $f^\mu$. Now, the boundary condition 
on $\sigma=\pi$,

\begin{align*}
    X^\mu&=\frac12\qty(f^\mu\qty(\sigma^+)+f^\mu\qty(\sigma^-))\\
    {X'}^\mu\qty(\tau,\pi)&=\frac12\qty({f'}^\mu\qty(\tau+\pi)-{f'}^\mu\qty(\tau-\pi))=0
\end{align*}

That is, ${f'}^\mu$ is periodic with period $2\pi$. The most general real function with period $2\pi$ is,

\begin{align*}
    {f'}^\mu\qty(u)&=f_1^\mu+\sqrt{2\alpha'}\sum\limits_{n=1}^\infty\qty({\alpha^\ast_n}^\mu\exp\qty(\im n u)+\alpha_n^\mu\exp\qty(-\im n u))\\
    f^\mu\qty(u)&=f_0^\mu+f_1^\mu u-\im\sqrt{2\alpha'}\sum\limits_{n=1}^\infty\frac1n\qty({\alpha^\ast_n}^\mu\exp\qty(\im n u)-\alpha_n^\mu\exp\qty(-\im n u))
\end{align*}

Defining $\alpha_{-n}^\mu={\alpha^\ast_n}^\mu$,

\begin{align*}
    f^\mu\qty(u)&=f_0^\mu+f_1^\mu u+\im\sqrt{2\alpha'}\sum\limits_{n\in\mathbb Z^\ast}\frac{\alpha_n^\mu}{n}\exp\qty(-\im n u)
\end{align*}

Now, back in $X$,

\begin{align*}
    X^\mu&=f_0^\mu+\frac12f_1^\mu\qty(\sigma^++\sigma^-)+\im\sqrt{2\alpha'}\sum\limits_{n\in\mathbb Z^\ast}\frac{\alpha_n^\mu}{n}\frac12\qty[\exp\qty(-\im n \sigma^+)+\exp\qty(-\im n\sigma^-)]\\
    X^\mu&=f_0^\mu+f_1^\mu\tau+\im\sqrt{2\alpha'}\sum\limits_{n\in\mathbb Z^\ast}\frac{\alpha_n^\mu}{n}\exp\qty(-\im n\tau)\frac12\qty[\exp\qty(-\im n \sigma)+\exp\qty(\im n\sigma)]\\
    X^\mu&=f_0^\mu+f_1^\mu\tau+\im\sqrt{2\alpha'}\sum\limits_{n\in\mathbb Z^\ast}\frac{\alpha_n^\mu}{n}\exp\qty(-\im n\tau)\cos\qty(n\sigma)   
\end{align*}

Setting $f_0^\mu=x_0^\mu$, and also noting that,

\begin{align*}
    p^\mu&=\int\limits_0^\pi\dd{\sigma}\mathcal P^{\tau\mu}=\int\limits_0^\pi\dd{\sigma}\frac{\partial_\tau X^\mu}{2\pi\alpha'}\\
    p^\mu&=\frac{1}{2\pi\alpha'}\int\limits_0^\pi\dd{\sigma}\qty[f_1^\mu+\sqrt{2\alpha'}\sum\limits_{n\in\mathbb Z^\ast}\alpha_n^\mu\exp\qty(-\im n\tau)\cos\qty(n\sigma)]=\frac{f_1^\mu}{2\alpha'}   
\end{align*}

So that,

\begin{align*}
    X^\mu&=x_0^\mu+2\alpha' p^\mu\tau+\im\sqrt{2\alpha'}\sum\limits_{n\in\mathbb Z^\ast}\frac{\alpha_n^\mu}{n}\exp\qty(-\im n\tau)\cos\qty(n\sigma)   
\end{align*}

Also is useful to define a additional mode, $\alpha_0^\mu=\sqrt{2\alpha'}p^\mu$. 

This is not the end of it, because we need to make sure the two constrains are satisfied, namely, \ref{constrains}. With the convention of 
uppercase latin index referring to non-light-cone components, $I=2,\cdots,D-1$, the constrains are,

\begin{align*}
    0&=-2\qty({\dot X}^-\pm{X'}^-)\qty({\dot X}^+\pm{X'}^+)+\qty({\dot X}^I\pm{X'}^I)^2\\
    4\alpha'p^+\qty({\dot X}^-\pm{X'}^-)&=\qty({\dot X}^I\pm{X'}^I)^2\\
    {\dot X}^-\pm{X'}^-&=\frac{1}{4\alpha'p^+}\qty({\dot X}^I\pm{X'}^I)^2
\end{align*}

That is, the two constrains implies that $X^-$ is not dynamical, and, the collection of $X^I$ fully determine $X^-$, 
apart from a single integration constant, $x_0^-$. To see this is helpful to note,

\begin{align*}
    {\dot X}^\mu&=\sqrt{2\alpha'}\sum\limits_{n\in\mathbb Z}\alpha_n^\mu\exp\qty(-\im n\tau)\cos\qty(n\sigma)\\
    {X'}&=-\im\sqrt{2\alpha'}\sum\limits_{n\in\mathbb Z}\alpha_n^\mu\exp\qty(-\im n\tau)\sin\qty(n\sigma)
\end{align*}

Which implies,

\begin{align*}
    {\dot X}^\mu\pm{X'}^\mu&=\sqrt{2\alpha'}\sum\limits_{n\in\mathbb Z}\alpha_n^\mu\exp\qty(-\im n\qty(\tau\pm \sigma))\numberthis\label{pmdot}
\end{align*}

So that,

\begin{align*}
    {\dot X}^-\pm{X'}^-&=\frac{1}{4\alpha'\pi}\qty({\dot X}^I\pm{X'}^I)^2\\
    \sqrt{2\alpha'}\sum\limits_{n\in\mathbb Z}\alpha^-_n\exp\qty(-\im n\qty(\tau\pm\sigma))&=\frac{2\alpha'}{4\alpha' p^+}\sum\limits_{p,q\in\mathbb Z}\alpha^I_p\alpha^I_q\exp\qty(-\im\qty(p+q)\qty(\tau\pm\sigma))\\
    \sqrt{2\alpha'}\sum\limits_{n\in\mathbb Z}\alpha^-_n\exp\qty(-\im n\qty(\tau\pm\sigma))&=\frac{1}{2p^+}\sum\limits_{n,p\in\mathbb Z}\alpha^I_p\alpha^I_{n-p}\exp\qty(-\im n\qty(\tau\pm\sigma))\\
    \sqrt{2\alpha'}\sum\limits_{n\in\mathbb Z}\alpha^-_n\exp\qty(-\im n\qty(\tau\pm\sigma))&=\frac{1}{p^+}\sum\limits_{n\in\mathbb Z}\qty(\frac12\sum\limits_{p\in\mathbb Z}\alpha^I_p\alpha^I_{n-p})\exp\qty(-\im n\qty(\tau\pm\sigma))\\
    \sqrt{2\alpha'}\alpha_n^-&=\frac{1}{2p^+}\sum\limits_{p\in\mathbb Z}\alpha_p^I\alpha_{n-p}^I=\frac{1}{p^+}L_n^\perp\numberthis\label{virasorodef}
\end{align*}

Hence, all the fourier modes of $X^-$ are completely determined by the transverse modes ones, apart from of course the integration constant $x^-_0$. In the 
last passage we also defined the Virasoro modes $L_n^\perp$. This is it, we fully solved the equation of motion with all the constrains and boundary 
conditions, the degrees of freedom we found are:

\begin{align*}
    X^I\qty(\tau,\sigma),p^+,x_0^-
\end{align*}

From this list of classical degrees of freedom we can start the quantization of 
our theory! This is done by imposing the canonical commutation relations dictated by the Poisson bracket between the degrees of freedom and their correspondent 
conjugated momentum, those are given by,

\begin{align*}
    \comm{X^I\qty(\tau,\sigma)}{\mathcal P^{\tau J}\qty(\tau,\sigma')}&=\im g^{IJ}\delta\qty(\sigma-\sigma'),\ \ \ \comm{x_0^-}{p^+}=-\im
\end{align*}

Any other commutator between these $4$ operators is zero. Of course now, $X^-$ has to be considered as a function 
of the $X^I$ and $\mathcal P^{\tau I}$, so, it has non trivial commutators. The first thing we need to do here is go from these 
canonical commutation relations, to the commutation relations of the modes $\alpha_n^I$. The best way of doing this is working with the 
expression \ref{pmdot}, and computing the following commutator,

\begin{align*}
    \comm{{\dot X}^I\pm{X'}^I}{{\dot X}^I\pm{X'}^I}
\end{align*}

For this is useful to note,

\begin{align*}
    \comm{X^I\qty(\tau,\sigma)}{X^J\qty(\tau,\sigma')}=0&\Rightarrow\comm{{X'}^I\qty(\tau,\sigma)}{{X'}^J\qty(\tau,\sigma')}=0\\
    \comm{X^I\qty(\tau,\sigma)}{\mathcal P^{\tau J}\qty(\tau,\sigma')}=\im g^{IJ}\delta\qty(\sigma-\sigma')&\Rightarrow\comm{{X'}^I\qty(\tau,\sigma)}{{\dot X}^J\qty(\tau,\sigma')}=2\pi\alpha'\im g^{IJ}\dv{}{\sigma}\delta\qty(\sigma-\sigma')\\
    \comm{\mathcal P^{\tau I}\qty(\tau,\sigma)}{\mathcal P^{\tau J}\qty(\tau,\sigma')}=0&\Rightarrow\comm{{\dot X}^I\qty(\tau,\sigma)}{{\dot X}^J\qty(\tau,\sigma')}=0
\end{align*}

Hence, the non-vanishing contributions are,

\begin{align*}
    \comm{\qty({\dot X}^I\pm{X'}^I)\qty(\tau,\sigma)}{\qty({\dot X}^I\pm{X'}^I)\qty(\tau,\sigma')}&=\pm\comm{{\dot X}^I\qty(\tau,\sigma)}{{X'}^I\qty(\tau,\sigma')}\pm\comm{{X'}^I\qty(\tau,\sigma)}{{\dot X}^I\qty(\tau,\sigma')}\\
    \comm{\qty({\dot X}^I\pm{X'}^I)\qty(\tau,\sigma)}{\qty({\dot X}^I\pm{X'}^I)\qty(\tau,\sigma')}&=\mp2\pi\alpha'\im g^{IJ}\dv{}{\sigma'}\delta\qty(\sigma'-\sigma)\pm2\pi\alpha'\im g^{IJ}\dv{}{\sigma}\delta\qty(\sigma-\sigma')\\
    \comm{\qty({\dot X}^I\pm{X'}^I)\qty(\tau,\sigma)}{\qty({\dot X}^I\pm{X'}^I)\qty(\tau,\sigma')}&=\pm4\pi\alpha'\im g^{IJ}\dv{}{\sigma}\delta\qty(\sigma-\sigma')\numberthis\label{commpm}
\end{align*}

And with opposite signs,

\begin{align*}
    \comm{\qty({\dot X}^I\pm{X'}^I)\qty(\tau,\sigma)}{\qty({\dot X}^I\mp{X'}^I)\qty(\tau,\sigma')}&=\mp\comm{{\dot X}^I\qty(\tau,\sigma)}{{X'}^I\qty(\tau,\sigma')}\pm\comm{{X'}^I\qty(\tau,\sigma)}{{\dot X}^I\qty(\tau,\sigma')}\\
    \comm{\qty({\dot X}^I\pm{X'}^I)\qty(\tau,\sigma)}{\qty({\dot X}^I\mp{X'}^I)\qty(\tau,\sigma')}&=\pm2\pi\alpha'\im g^{IJ}\dv{}{\sigma'}\delta\qty(\sigma'-\sigma)\pm2\pi\alpha'\im g^{IJ}\dv{}{\sigma}\delta\qty(\sigma-\sigma')\\
    \comm{\qty({\dot X}^I\pm{X'}^I)\qty(\tau,\sigma)}{\qty({\dot X}^I\mp{X'}^I)\qty(\tau,\sigma')}&=0\numberthis\label{commmp}
\end{align*}

But these expressions, and of course $X$ also, are only defined for $\sigma\in\qty[0,\pi]$, as we want to use \ref{pmdot} to 
isolate the fourier modes, we'll need to integrate over $2\pi$, the only option here is to find an extension of this expression to the whole interval 
$\qty[0,2\pi]$. As everything is periodic in $\sigma$ with $2\pi$ period, it's sufficient to find an extension to $\sigma\in\qty[-\pi,\pi]$. 
See that,

\begin{align*}
    \qty({\dot X}^I+{X'}^I)\qty(\tau,\sigma)&=\sqrt{2\alpha'}\sum\limits_{n\in\mathbb Z}\alpha_n^I\exp\qty(-\im n\qty(\tau+ \sigma)),\ \ \ \sigma\in\qty[0,\pi]\\
    \qty({\dot X}^I-{X'}^I)\qty(\tau,\sigma)&=\sqrt{2\alpha'}\sum\limits_{n\in\mathbb Z}\alpha_n^I\exp\qty(-\im n\qty(\tau- \sigma)),\ \ \ \sigma\in\qty[0,\pi]
\end{align*}

Performing a change of variables $\sigma\rightarrow-\sigma$ in the second expression,

\begin{align*}
    \qty({\dot X}^I+{X'}^I)\qty(\tau,\sigma)&=\sqrt{2\alpha'}\sum\limits_{n\in\mathbb Z}\alpha_n^I\exp\qty(-\im n\qty(\tau+ \sigma)),\ \ \ \sigma\in\qty[0,\pi]\\
    \qty({\dot X}^I-{X'}^I)\qty(\tau,-\sigma)&=\sqrt{2\alpha'}\sum\limits_{n\in\mathbb Z}\alpha_n^I\exp\qty(-\im n\qty(\tau+\sigma)),\ \ \ \sigma\in\qty[-\pi,0]
\end{align*}

That's interesting, because we found a representation of the modes in the whole domain $\qty[-\pi,\pi]$,

\begin{align*}
    A^I\qty(\tau,\sigma)=\sqrt{2\alpha'}\sum\limits_{n\in\mathbb Z}\alpha_n^I\exp\qty(-\im n\qty(\tau+ \sigma))=\begin{cases}
        \qty({\dot X}^I+{X'}^I)\qty(\tau,\sigma),& \sigma\in\qty[0,\pi]\\
        \qty({\dot X}^I-{X'}^I)\qty(\tau,-\sigma),& \sigma\in\qty[-\pi,0]
    \end{cases}
\end{align*}

And of course, as everything where, we still have $A\qty(\tau,\sigma+2\pi)=A\qty(\tau,\sigma)$. Also, by \ref{commpm} we do have, using the $+$ sign,

\begin{align*}
    \comm{A^I\qty(\tau,\sigma)}{A^J\qty(\tau,\sigma')}=4\pi\alpha'\im g^{IJ}\dv{}{\sigma}\delta\qty(\sigma-\sigma'),\ \ \ \sigma,\sigma'\in\qty[0,\pi]
\end{align*}

By \ref{commmp}, we get that when $\sigma\in\qty[0,\pi]$ and $\sigma'\in\qty[-\pi,0]$ this commutator is zero, which is consistent as the Dirac delta is 
zero in this domain. At last we use \ref{commpm} with both $-$ sign, and $-\sigma,-\sigma'\in\qty[-\pi,0]$, which get us,

\begin{align*}
    \comm{A^I\qty(\tau,\sigma)}{A^J\qty(\tau,\sigma')}=4\pi\alpha'\im g^{IJ}\dv{}{\sigma}\delta\qty(\sigma-\sigma'),\ \ \ \sigma,\sigma'\in\qty[-\pi,0]
\end{align*}

Putting all together, what we have is,

\begin{align*}
    \comm{A^I\qty(\tau,\sigma)}{A^J\qty(\tau,\sigma')}=4\pi\alpha'\im g^{IJ}\dv{}{\sigma}\delta\qty(\sigma-\sigma'),\ \ \ \sigma,\sigma'\in\qty[-\pi,\pi]
\end{align*}

And, also as everything is periodic in $2\pi$, this fully determines the commutation relation over $\sigma,\sigma'\in\qty[0,2\pi]$. Inserting now the definition of 
$A\qty(\tau,\sigma)$,

\begin{align*}
    4\pi\alpha'\im g^{IJ}\dv{}{\sigma}\delta\qty(\sigma-\sigma')&=2\alpha'\sum\limits_{n',m'\in\mathbb Z}\e^{-\im n'\qty(\tau+\sigma')}\e^{-\im m'\qty(\tau+\sigma)}\comm{\alpha_{m'}^I}{\alpha_{n'}^J}\\
    2\pi\im g^{IJ}\int_0^{2\pi}\frac{\dd{\sigma}}{2\pi}\dv{}{\sigma}\delta\qty(\sigma-\sigma')\e^{\im m\sigma}&=\int_0^{2\pi}\frac{\dd{\sigma}}{2\pi}\e^{\im m\sigma}\sum\limits_{m',n'\in\mathbb Z}\e^{-\im \tau\qty(n'+m')}\e^{-\im \sigma' n'-\im\sigma m'}\comm{\alpha_{m'}^I}{\alpha_{n'}^J}\\
    mg^{IJ}\e^{\im m\sigma'}&=\sum\limits_{m',n'\in\mathbb Z}\e^{-\im \tau\qty(n'+m')}\e^{-\im \sigma' n'}\delta_{m,m'}\comm{\alpha_{m'}^I}{\alpha_{n'}^J}\\
    \int_0^{2\pi}\frac{\dd{\sigma}}{2\pi}\e^{\im n\sigma'}mg^{IJ}\e^{\im m\sigma'}&=\int_0^{2\pi}\frac{\dd{\sigma}}{2\pi}\e^{\im n\sigma'}\sum\limits_{n'\in\mathbb Z}\e^{-\im \tau\qty(n'+m)}\e^{-\im \sigma' n'}\comm{\alpha_{m}^I}{\alpha_{n'}^J}\\
    mg^{IJ}\delta_{m+n,0}&=\sum\limits_{n'\in\mathbb Z}\delta_{n,n'}\e^{-\im \tau\qty(n'+m)}\comm{\alpha_{m}^I}{\alpha_{n'}^J}\\
    mg^{IJ}\delta_{m+n,0}\e^{\im \tau\qty(n+m)}&=\comm{\alpha_{m}^I}{\alpha_{n}^J}\\
    \comm{\alpha_{m}^I}{\alpha_{n}^J}&=mg^{IJ}\delta_{m+n,0}\numberthis\label{alphacomm}
\end{align*}

This is not the end, we still have one more commutation relation to get,

\begin{align*}
    2\pi\alpha'\im g^{IJ}\delta\qty(\sigma-\sigma')&=\comm{X^I\qty(\tau,\sigma)}{{\dot X}^J\qty(\tau,\sigma')}\\
    \int\limits_0^\pi\dd{\sigma}2\pi\alpha'\im g^{IJ}\delta\qty(\sigma-\sigma')&=\int\limits_0^\pi\dd{\sigma}\comm{X^I\qty(\tau,\sigma)}{{\dot X}^J\qty(\tau,\sigma')},\ \ \ \int\limits_0^\pi\dd{\sigma}\cos\qty(n\sigma)=0,\ n\in\mathbb Z\\
    2\pi\alpha'\im g^{IJ}&=\pi\comm{x_0^I+\sqrt{2\alpha'}\alpha_0^I\tau}{{\dot X}^J\qty(\tau,\sigma')}\\
    2\alpha'\im g^{IJ}&=\sqrt{2\alpha'}\sum\limits_{n'\in\mathbb Z}\exp\qty(-\im n'\tau)\cos\qty(n'\sigma')\qty{\comm{x_0^I}{\alpha_{n'}^J}+\sqrt{2\alpha'}\tau\comm{\alpha_0^I}{\alpha_{n'}^J}}\\
    \sqrt{2\alpha'}\im g^{IJ}&=\sum\limits_{n'\in\mathbb Z}\exp\qty(-\im n'\tau)\cos\qty(n'\sigma')\comm{x_0^I}{\alpha_{n'}^J}\\
    \int\limits_0^\pi\frac{\dd{\sigma'}}{\pi}\cos\qty(n\sigma')\sqrt{2\alpha'}\im g^{IJ}&=\int\limits_0^\pi\frac{\dd{\sigma'}}{\pi}\cos\qty(n\sigma')\sum\limits_{n'\in\mathbb Z}\exp\qty(-\im n'\tau)\cos\qty(n'\sigma')\comm{x_0^I}{\alpha_{n'}^J}\\
    \delta_{n,0}\sqrt{2\alpha'}\im g^{IJ}&=\sum\limits_{n'\in\mathbb Z}\exp\qty(-\im n'\tau)\delta_{n',n}\comm{x_0^I}{\alpha_{n'}^J}\\
    \comm{x_0^I}{\alpha_{n}^J}&=\delta_{n,0}\sqrt{2\alpha'}\im g^{IJ}\exp\qty(-\im n\tau)=\delta_{n,0}\sqrt{2\alpha'}\im g^{IJ}
\end{align*}

The last thing we need to discuss before going to the Lorentz generators is about the Virasoro operators, these were defined as,

\begin{align*}
    L^\perp_n&=\frac12\sum\limits_{p\in\mathbb Z}\alpha_{n-p}^I\alpha_{p}
\end{align*}

We should be aware of possible ambiguities in the ordering of these, as we now the commutation relations \ref{alphacomm}, two alphas fail 
to commute only if their mode number sum up to $0$, but, notice that $n-p+p=0\rightarrow n=0$, hence, the only not well defined Virasoro mode is 
$L^\perp_0$. As the difference between any two ordering prescriptions is always proportional to the identity operator, we'll \textbf{define} $L^\perp_0$ 
to be on the \textbf{normal ordered} prescription --- All $\alpha_n^I,\ n\geq 0$ need to be to the right of all $\alpha_n^I,\ n<0$ ---, and wherever there is 
mention to this Virasoro mode we should use the normal ordered one plus an addition undetermined normal ordering constant,

\begin{align*}
    L^\perp_0\rightarrow L_0^\perp+a
\end{align*}

As an example, in \ref{virasorodef}, with $n=0$, the quantum version should read,

\begin{align*}
    2\alpha'p^-=\sqrt{2\alpha'}\alpha^-_0&=\frac{1}{p^+}\qty(L_0^\perp+a)\\
    L_0^\perp&=\frac12\alpha_0^I\alpha_0^I+\sum\limits_{p\in\mathbb N^\ast}\alpha_{-p}^I\alpha_p^I
\end{align*}

We can write a manifestly normal ordered form for all $n$,

\begin{align*}
    L_n^\perp=\frac12\sum\limits_{p\geq 0}\alpha_{n-p}^I\alpha_p^I+\frac12\sum\limits_{p<0}\alpha_p^I\alpha_{m-p}^I
\end{align*}

Using these, every calculation we do is manifestly normal ordered, which will prevent us from making mistakes. As classically we had, $\qty(\alpha_n^I)^\ast=\alpha_{-n}^I$, in the quantization we have, $\qty(\alpha_n^I)^\dagger=\alpha_{-n}^I$. This 
allows us to conclude that, $\qty(L^\perp_n)^\dagger=L_{-n}^\perp$. A few more properties we'll need are the commutation 
relations of the Virasoro modes with all the other objects, we start with,

\begin{align*}
    \comm{L_m^\perp}{\alpha^J_n}&=\frac12\sum\limits_{p\geq 0}\comm{\alpha_{m-p}^I\alpha_p^I}{\alpha_n^J}+\frac12\sum\limits_{p< 0}\comm{\alpha_{p}^I\alpha_{m-p}^I}{\alpha_n^J}\\
    &=\frac12\sum\limits_{p\geq0}\qty{\alpha_{m-p}^I\comm{\alpha_p^I}{\alpha_n^J}+\comm{\alpha_{m-p}^I}{\alpha_n^J}\alpha_p^I}+\frac12\sum\limits_{p<0}\qty{\alpha_{p}^I\comm{\alpha_{m-p}^I}{\alpha_n^J}+\comm{\alpha_{p}^I}{\alpha_n^J}\alpha_{m-p}^I}\\
    &=\frac12\sum\limits_{p\in\mathbb Z}\qty{\alpha_{m-p}^I g^{IJ}p\delta_{p+n,0}+\qty(m-p)\delta_{m-p+n,0}g^{IJ}\alpha_p^I}\\
    &=\frac12\qty{-\alpha_{m+n}^Jn+\qty(m-m-n)\alpha_{m+n}^J}\\
    \comm{L_m^\perp}{\alpha^J_n}&=-n\alpha_{n+m}^J\numberthis\label{virasoroalpha}
\end{align*}

Now,

\begin{align*}
    \comm{L_m^\perp}{x_0^J}&=\frac12\sum\limits_{p\geq0}\comm{\alpha_{m-p}^I\alpha_p^I}{x_0^J}+\frac12\sum\limits_{p<0}\comm{\alpha_{p}^I\alpha_{m-p}^I}{x_0^J}\\
    &=\frac12\sum\limits_{p\geq0}\qty{\alpha_{m-p}^I\comm{\alpha_p^I}{x_0^J}+\comm{\alpha_{m-p}^I}{x_0^J}\alpha_p^I}+\frac12\sum\limits_{p<0}\qty{\alpha_{p}^I\comm{\alpha_{m-p}^I}{x_0^J}+\comm{\alpha_{p}^I}{x_0^J}\alpha_{m-p}^I}\\
    &=\frac12\sum\limits_{p\in\mathbb Z}\qty{-\im\sqrt{2\alpha'}g^{IJ}\delta_{p,0}\alpha_{m-p}^I-\im\sqrt{2\alpha'}g^{IJ}\delta_{m-p,0}\alpha_p^I}\\
    &=-\im\sqrt{2\alpha'}\frac12\qty{\alpha_{m}^J+\alpha_m^J}\\
    \comm{L_m^\perp}{x_0^J}&=-\im\sqrt{2\alpha'}\alpha_m^J
\end{align*}

And lastly, but not less important, we have to know the commutation relation between the Virasoro modes themselves, this is more subtle, 
because we defined them being normal ordered, thus, every step of the calculation we have to make sure all terms are normal ordered,

\begin{align*}
    \comm{L_m^\perp}{L_n^\perp}&=\frac12\sum\limits_{p\geq 0}\comm{\alpha_{m-p}^I\alpha_{p}^I}{L^\perp_n}+\frac12\sum\limits_{p< 0}\comm{\alpha_{p}^I\alpha_{m-p}^I}{L^\perp_n}\\
    &=\frac12\sum\limits_{p\geq 0}\qty{\alpha_{m-p}^I\comm{\alpha_{p}^I}{L^\perp_n}+\comm{\alpha_{m-p}^I}{L^\perp_n}\alpha_{p}^I}\\
    &\quad\quad\quad+\frac12\sum\limits_{p< 0}\qty{\alpha_{p}^I\comm{\alpha_{m-p}^I}{L^\perp_n}+\comm{\alpha_{p}^I}{L^\perp_n}\alpha_{m-p}^I}\\
    &=\frac12\sum\limits_{p\geq 0}\qty{p\alpha_{m-p}^I\alpha_{p+n}^I+\qty(m-p)\alpha_{n+m-p}^I\alpha_{p}^I}\\
    &\quad\quad\quad+\frac12\sum\limits_{p< 0}\qty{\qty(m-p)\alpha_{p}^I\alpha_{m+n-p}^I+p\alpha_{p+n}^I\alpha_{m-p}^I}\\
    &=\frac12\sum\limits_{p\geq 0}\qty(m-p)\alpha_{n+m-p}^I\alpha_{p}^I+\frac12\sum\limits_{p< 0}\qty(m-p)\alpha_{p}^I\alpha_{m+n-p}^I\\
    &\quad\quad\quad+\frac12\sum\limits_{p\geq 0}p\alpha_{m-p}^I\alpha_{p+n}^I+\frac12\sum\limits_{p< 0}p\alpha_{p+n}^I\alpha_{m-p}^I\\
    &=\frac12\sum\limits_{p\geq 0}\qty(m-p)\alpha_{n+m-p}^I\alpha_{p}^I+\frac12\sum\limits_{p< 0}\qty(m-p)\alpha_{p}^I\alpha_{m+n-p}^I\\
    &\quad\quad\quad+\frac12\sum\limits_{p\geq n}\qty(p-n)\alpha_{m+n-p}^I\alpha_{p}^I+\frac12\sum\limits_{p< n}\qty(p-n)\alpha_{p}^I\alpha_{m+n-p}^I\\
    &=\frac12\sum\limits_{p\geq 0}\qty(m-p)\alpha_{n+m-p}^I\alpha_{p}^I+\frac12\sum\limits_{p< 0}\qty(m-p)\alpha_{p}^I\alpha_{m+n-p}^I\\
    &\quad\quad\quad+\frac12\sum\limits_{p\geq 0}\qty(p-n)\alpha_{m+n-p}^I\alpha_{p}^I+\frac12\sum\limits_{p< 0}\qty(p-n)\alpha_{p}^I\alpha_{m+n-p}^I\\
    &\quad\quad\quad+\frac12\qty(\frac12-\frac{n}{2\abs{n}})\qty[\sum\limits_{p= n}^{-1}\qty(p-n)\alpha_{m+n-p}^I\alpha_{p}^I-\sum\limits_{p= n}^{-1}\qty(p-n)\alpha_{p}^I\alpha_{m+n-p}^I]\\
    &\quad\quad\quad+\frac12\qty(\frac12+\frac{n}{2\abs{n}})\qty[-\sum\limits_{p= 0}^{n-1}\qty(p-n)\alpha_{m+n-p}^I\alpha_{p}^I+\sum\limits_{p= 0}^{n-1}\qty(p-n)\alpha_{p}^I\alpha_{m+n-p}^I]\\
    &=\frac12\qty(m-n)\sum\limits_{p\geq 0}\alpha_{n+m-p}^I\alpha_{p}^I+\frac12\qty(m-n)\sum\limits_{p< 0}\alpha_{p}^I\alpha_{m+n-p}^I\\
    &\quad\quad\quad+\frac12\qty(\frac12-\frac{n}{2\abs{n}})\qty[\sum\limits_{p= n}^{-1}\qty(p-n)\qty{\comm{\alpha_{m+n-p}^I}{\alpha_{p}^I}+\alpha_{p}^I\alpha_{m+n-p}^I}-\sum\limits_{p= n}^{-1}\qty(p-n)\alpha_{p}^I\alpha_{m+n-p}^I]\\
    &\quad\quad\quad+\frac12\qty(\frac12+\frac{n}{2\abs{n}})\qty[-\sum\limits_{p= 0}^{n-1}\qty(p-n)\alpha_{m+n-p}^I\alpha_{p}^I+\sum\limits_{p= 0}^{n-1}\qty(p-n)\qty{\comm{\alpha_{p}^I}{\alpha_{m+n-p}^I}+\alpha_{m+n-p}^I\alpha_{p}^I}]\\
    &=\qty(m-n)L_{m+n}^\perp\\
    &\quad\quad\quad-\frac{g^{II}}{2}\qty(\frac12-\frac{n}{2\abs{n}})\sum\limits_{p= n}^{-1}\qty(p-n)p\delta_{m+n,0}+\frac{g^{II}}{2}\qty(\frac12+\frac{n}{2\abs{n}})\sum\limits_{p= 0}^{n-1}\qty(p-n)p\delta_{m+n,0}\\
    &=\qty(m-n)L_{m+n}^\perp\\
    &\quad\quad\quad+\frac{D-2}{2}\qty(\frac12-\frac{n}{2\abs{n}})\sum\limits_{p= 0}^{\abs{n}-1}\qty(-p-n)p\delta_{m+n,0}+\frac{D-2}{2}\qty(\frac12+\frac{n}{2\abs{n}})\sum\limits_{p= 0}^{\abs{n}-1}\qty(p-n)p\delta_{m+n,0}\\
    &=\qty(m-n)L_{m+n}^\perp\\
    &\quad\quad\quad+\frac{D-2}{2}\delta_{m+n,0}\sum\limits_{p= 0}^{\abs{n}-1}p\qty[\qty(\frac12-\frac{n}{2\abs{n}})\qty(-p-n)+\qty(\frac12+\frac{n}{2\abs{n}})\qty(p-n)]\\
    &=\qty(m-n)L_{m+n}^\perp\\
    &\quad\quad\quad+\frac{D-2}{2}\delta_{m+n,0}\sum\limits_{p= 0}^{\abs{n}-1}p\qty[-n+\frac{n}{2\abs{n}}\qty(p+n)+\frac{n}{2\abs{n}}\qty(p-n)]\\
    &=\qty(m-n)L_{m+n}^\perp+\frac{D-2}{2}\delta_{m+n,0}\frac{n}{\abs{n}}\sum\limits_{p= 0}^{\abs{n}-1}p\qty[p-\abs{n}]
\end{align*}

In the middle of the calculus we introduced factors of $\frac12\qty(1\pm \frac{n}{\abs{n}})$ just to account for the two possible cases, $n>0$ and $n<0$ --- Of course, the case $n=0$ is trivial 
due to not being necessary to introduce any other factors to ensure the normal ordering of the expression---. 
Now, we're going to prove by induction that the value of the sum is,

\begin{align*}
    \sum\limits_{p=0}^{\abs{n}-1}p\qty(p-\abs{n})=\frac16\qty(\abs{n}-\abs{n}^3),\ \ \ \abs{n}\geq 1
\end{align*}

It's trivial to check it's validity from $\abs{n}=1$, now, suppose it's valid for $\abs{n}=k$,

\begin{align*}
    \sum\limits_{p=0}^{k}p\qty(p-k-1)&=\sum\limits_{p=0}^{k}p\qty(p-k)-\sum\limits_{p=0}^kp\\
    \sum\limits_{p=0}^{k}p\qty(p-k-1)&=\sum\limits_{p=0}^{k-1}p\qty(p-k)-\sum\limits_{p=0}^kp\\
    \sum\limits_{p=0}^{k}p\qty(p-k-1)&=\frac16\qty(k-k^3)-\frac12k\qty(k+1)\\
    \sum\limits_{p=0}^{k}p\qty(p-k-1)&=\frac16\qty(-3k^2-2k-k^3)\\
    \sum\limits_{p=0}^{k}p\qty(p-k-1)&=\frac16\qty(k+1-1-3k-3k^2-k^3)=\frac16\qty(k+1-\qty(k+1)^3)
\end{align*}

Which finishes our proof. Hence,

\begin{align*}
    \comm{L_m^\perp}{L_n^\perp}&=\qty(m-n)L_{m+n}^\perp+\frac{D-2}{2}\delta_{m+n,0}\frac{n}{\abs{n}}\sum\limits_{p= 0}^{\abs{n}-1}p\qty[p-\abs{n}]\\
    \comm{L_m^\perp}{L_n^\perp}&=\qty(m-n)L_{m+n}^\perp+\frac{D-2}{12}\delta_{m+n,0}\frac{n}{\abs{n}}\qty(\abs{n}-\abs{n}^3)\\
    \comm{L_m^\perp}{L_n^\perp}&=\qty(m-n)L_{m+n}^\perp+\frac{D-2}{12}\delta_{m+n,0}\qty(n-n^3)\\
    \comm{L_m^\perp}{L_n^\perp}&=\qty(m-n)L_{m+n}^\perp+\frac{D-2}{12}\qty(m^3-m)\delta_{m+n,0}
\end{align*}

This completes the set of all needed commutation relations. We can now discuss the Lorentz generators in the Light-cone gauge. 
We have already computed them in \ref{m}, but of course we have two remarks, neither they are quantum, nor are in the light-cone gauge, 
about the later, the actual quantum light-cone gauge Lorentz generators \textbf{should} satisfy the same algebra as \ref{lorentzalgebra} in the 
light-cone coordinates. The failure to met this requirement is related to an anomaly in this global symmetry of the Quantum Poincare Action. 
And about the former, we'll change the definition accordingly to ensure the quantum generators do satisfy being Hermitian. With that being said, 
let's evaluate the classical version of \ref{m} in the light-cone gauge,

\begin{align*}
    M^{\mu\nu}&\stackrel{?}{=}\frac{1}{2\pi\alpha'}\int\limits_0^\pi\dd{\sigma}\qty(X^\mu{\dot X}^\nu-X^\nu{\dot X}^\mu)\\
    &\stackrel{?}{=}\frac{1}{2\pi\alpha'}\int\limits_0^\pi\dd{\sigma}\qty(x_0^\mu+2\alpha'p^\mu\tau+\im\sqrt{2\alpha'}\sum\limits_{n\in\mathbb Z^\ast}\frac{\alpha^\mu_n}{n}\exp\qty(-\im n\tau)\cos\qty(n\sigma))\times\\
    &\quad\quad\quad\times\sqrt{2\alpha'}\sum\limits_{m\in\mathbb Z}\alpha^\nu_m\exp\qty(-\im m\tau)\cos\qty(m\sigma)-\qty(\mu\leftrightarrow\nu)\\
    &\stackrel{?}{=}\frac{1}{2\pi\alpha'}\left[\sqrt{2\alpha'}\sum\limits_{m\in\mathbb Z}x_0^\mu\alpha^\nu_m\exp\qty(-\im m\tau)\pi\delta_{m,0}\right.\\
    &\quad\quad\quad+2\alpha'\tau\sqrt{2\alpha'}\sum\limits_{m\in\mathbb Z}p^\mu\alpha^\nu_m\exp\qty(-\im m\tau)\pi\delta_{m,0}\\
    &\quad\quad\quad\left.+\im 2\alpha'\sum\limits_{n\in\mathbb Z^\ast}\sum\limits_{m\in\mathbb Z}\frac{\alpha^\mu_n}{n}\exp\qty(-\im n\tau)\alpha^\nu_m\exp\qty(-\im m\tau)\frac\pi2\qty(\delta_{m,n}+\delta_{m,-n})\right]-\qty(\mu\leftrightarrow\nu)\\
    M^{\mu\nu}&\stackrel{?}{=}\left[\frac{1}{\sqrt{2\alpha'}}x_0^\mu\alpha^\nu_0\right.+\tau\sqrt{2\alpha'}p^\mu\alpha^\nu_0+\frac\im2 \sum\limits_{n\in\mathbb Z^\ast}\frac{1}{n}\alpha^\mu_n\alpha^\nu_n\exp\qty(-\im \qty(n+m)\tau)\left.+\frac\im2\sum\limits_{n\in\mathbb Z^\ast}\frac{1}{n}\alpha^\mu_n\alpha^\nu_{-n}\right]-\qty(\mu\leftrightarrow\nu)
\end{align*}

Employing the before mentioned equality $\alpha^\mu_0=\sqrt{2\alpha'}p^\mu$,

\begin{align*}
    M^{\mu\nu}&\stackrel{?}{=}\left[x_0^\mu p^\nu-x_0^\nu p^\mu+\tau\alpha_0^\mu \alpha_0^\nu-\tau\alpha_0^\nu \alpha_0^\mu+\frac\im2\sum\limits_{n\in\mathbb Z^\ast}\frac{1}{n}\qty(\alpha^\mu_n\alpha^\nu_{-n}-\alpha^\nu_n\alpha^\mu_{-n})\right.\\
    &\quad\quad\quad\left.+\frac\im2 \sum\limits_{n\in\mathbb Z^\ast}\frac{1}{n}\qty(\alpha^\mu_n\alpha^\nu_n-\alpha^\nu_n\alpha^\mu_n)\exp\qty(-\im 2n\tau)\right]\\
    M^{\mu\nu}&\stackrel{?}{=}\left[x_0^\mu p^\nu-x_0^\nu p^\mu+\frac\im2\sum\limits_{n\in\mathbb N^\ast}\frac{1}{n}\qty(\alpha^\mu_n\alpha^\nu_{-n}-\alpha^\nu_n\alpha^\mu_{-n}-\alpha^\mu_{-n}\alpha^\nu_{n}+\alpha^\nu_{-n}\alpha^\mu_{n})\right]\\
    M^{\mu\nu}&\stackrel{?}{=}\left[x_0^\mu p^\nu-x_0^\nu p^\mu+\im\sum\limits_{n\in\mathbb N^\ast}\frac{1}{n}\qty(\alpha^\mu_n\alpha^\nu_{-n}-\alpha^\nu_n\alpha^\mu_{-n})\right]
\end{align*}

This is the classical version of the generators in the light-cone gauge. Does it is a good Quantum Lorentz Generator Operator? 
For a positive answer, it has to be both Hermitian and normal-ordered, let's consider one by one,

\begin{align*}
    M^{IJ}&\stackrel{?}{=}\left[x_0^I p^J-x_0^J p^I+\im\sum\limits_{n\in\mathbb N^\ast}\frac{1}{n}\qty(\alpha^I_n\alpha^J_{-n}-\alpha^J_n\alpha^I_{-n})\right]\\
    \qty(M^{IJ})^\dagger-M^{IJ}&\stackrel{?}{=}\left[p^Jx_0^I -p^Ix_0^J -x_0^I p^J+x_0^J p^I-\im\sum\limits_{n\in\mathbb N^\ast}\frac{1}{n}\qty(\alpha^J_{n}\alpha^I_{-n}-\alpha^I_{n}\alpha^J_{-n}+\alpha^I_n\alpha^J_{-n}-\alpha^J_n\alpha^I_{-n})\right]\\
    \qty(M^{IJ})^\dagger-M^{IJ}&\stackrel{?}{=}\left[\comm{p^J}{x_0^I} -\comm{p^I}{x_0^J}\right]\\
    \qty(M^{IJ})^\dagger-M^{IJ}&\stackrel{?}{=}\left[-\im g^{IJ} +\im  g^{IJ}\right]=0
\end{align*}

Yes! And about normal-ordered? No. Both terms inside the sum are reversed normal-ordered, just reversing those, 
our quantum operator related to the Lorentz Generators is,

\begin{align*}
    M^{IJ}&=\left[x_0^I p^J-x_0^J p^I-\im\sum\limits_{n\in\mathbb N^\ast}\frac{1}{n}\qty(\alpha^I_{-n}\alpha^J_n-\alpha^J_{-n}\alpha^I_n)\right]
\end{align*}

Now, already changing to the normal-ordered one, and looking at Hermiticity,

\begin{align*}
    M^{I+}&\stackrel{?}{=}\left[x_0^I p^+-x_0^+ p^I-\im\sum\limits_{n\in\mathbb N^\ast}\frac{1}{n}\qty(\alpha^I_{-n}\alpha^+_{n}-\alpha^+_{-n}\alpha^I_{n})\right]\\
    M^{I+}&\stackrel{?}{=}x_0^I p^+
\end{align*}

Which is of course Hermitian, hence,

\begin{align*}
    M^{I+}&=x_0^I p^+
\end{align*}

By the same reasoning, that is, $x_0^+=0=\alpha_n^+,\ n\neq 0$,

\begin{align*}
    M^{-+}&\stackrel{?}{=}x_0^- p^+
\end{align*}

Which fails to be Hermitian, due the canonical commutation relations. One way to avoid this is to symmetrize it,

\begin{align*}
    M^{-+}&=\frac{1}{2}\qty(x_0^-p^++p^+x_0^-)
\end{align*}

Which now is both normal ordered and hermitian. Now,

\begin{align*}
    M^{++}&\stackrel{?}{=}x_0^+p^+=0
\end{align*}

Which is expected by the anti-symmetry of the generators, we have another which is zero by the anti-symmetry,

\begin{align*}
    M^{--}&\stackrel{?}{=}\qty[x_0^-p^--x_0^-p^--\im\sum\limits_{n\in\mathbb N^\ast}\frac1n\qty(\alpha_{-n}^-\alpha^-_{n}-\alpha^-_{-n}\alpha^-_n)]=0
\end{align*}

At last, we have the most important generator,

\begin{align*}
    M^{-I}&\stackrel{?}{=}\left[x_0^- p^I-x_0^I p^--\im\sum\limits_{n\in\mathbb N^\ast}\frac{1}{n}\qty(\alpha^-_{-n}\alpha^I_{n}-\alpha^I_{-n}\alpha^-_{n})\right]
\end{align*}

Is it Hermitian? No. 

\begin{align*}
    \qty(M^{-I})^\dagger-M^{-I}&\stackrel{?}{=}\left[p^Ix_0^- -p^-x_0^I -x_0^- p^I+x_0^I p^-+\im\sum\limits_{n\in\mathbb N^\ast}\frac{1}{n}\qty(\alpha^I_{-n}\alpha^-_{n}-\alpha^-_{-n}\alpha^I_{n}+\alpha^-_{-n}\alpha^I_{n}-\alpha^I_{-n}\alpha^-_{n})\right]\\
    \qty(M^{-I})^\dagger-M^{-I}&\stackrel{?}{=}\left[\comm{p^I}{x_0^-} -\comm{p^-}{x_0^I}\right]\\
    \qty(M^{-I})^\dagger-M^{-I}&\stackrel{?}{=}-\comm{p^-}{x_0^I}\\
    \qty(M^{-I})^\dagger-M^{-I}&\stackrel{?}{=}-\frac{1}{\sqrt{2\alpha'}}\comm{\alpha_0^-}{x_0^I}\\
    \qty(M^{-I})^\dagger-M^{-I}&\stackrel{?}{=}-\frac{1}{2\alpha'p^+}\comm{L^\perp_0+a}{x_0^I}=\frac{\im}{\sqrt{2\alpha'}p^+}\alpha^I_0
\end{align*}

Again, if we symmetrize the $x_0^Ip^-$ term this issue is resolved,

\begin{align*}
    M^{-I}&\stackrel{?}{=}x_0^- p^I-\frac12\qty(x_0^I p^-+p^-x_0^I)-\im\sum\limits_{n\in\mathbb N^\ast}\frac{1}{n}\qty(\alpha^-_{-n}\alpha^I_{n}-\alpha^I_{-n}\alpha^-_{n})
\end{align*}

Is this expression normal-ordered? Yes! Due to the $\alpha_n^-$ being proportional to the Virasoro modes, which are already normal-ordered. Hence, the expression for this 
Lorentz Generator is, using Virasoro modes,

\begin{align*}
    M^{-I}&=x_0^- p^I-\frac{1}{4\alpha'p^+}\qty(x_0^I\qty(L_0^\perp+a)+\qty(L_0^\perp+a)x_0^I)-\frac{\im}{\sqrt{2\alpha'p^+}}\sum\limits_{n\in\mathbb N^\ast}\frac{1}{n}\qty(L^\perp_{-n}\alpha^I_{n}-\alpha^I_{-n}L^\perp_n)
\end{align*}

\subsection{4.B)}