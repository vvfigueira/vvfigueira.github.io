\problem{}
\subsection{}

The Polyakov Action is given by,

\begin{align*}
    S_{\textnormal{P}}&=-\frac{1}{4\pi\alpha'}\int\dd[2]{\sigma}\sqrt{h}h^{ab}g_{\mu\nu}\partial_a X^\mu \partial_b X^\nu
\end{align*}

With $h^{ab}$ being the world-sheet metric, $h=\norm{\Det\qty[h_{ab}]}$, and $g_{\mu\nu}=\textnormal{diag}\qty(-1,1,\cdots,1)$ the 
target space metric. A Poincare transformation of the fields $X$ is,

\begin{align*}
    X^\mu\qty(\tau,\sigma)&\rightarrow\tilde X^\mu\qty(\tau,\sigma)=\tensor{\Lambda}{^\mu_\nu}X^\nu\qty(\tau,\sigma)+a^\mu\\
    \partial_a X^\mu&\rightarrow\partial_a\tilde X^\mu=\tensor{\Lambda}{^\mu_\nu}\partial_a X^\nu
\end{align*}

With of course $\Lambda$ satisfying the defining property of a Lorentz transformation,

\begin{align*}
    g_{\mu\nu}\tensor{\Lambda}{^\mu_\alpha}\tensor{\Lambda}{^\nu_\beta}=g_{\alpha\beta}
\end{align*}

The transformed Action is,

\begin{align*}
    \tilde S_{\textnormal{P}}&=-\frac{1}{4\pi\alpha'}\int\dd[2]{\sigma}\sqrt{h}h^{ab}g_{\mu\nu}\partial_a \tilde X^\mu \partial_b \tilde X^\nu\\
    \tilde S_{\textnormal{P}}&=-\frac{1}{4\pi\alpha'}\int\dd[2]{\sigma}\sqrt{h}h^{ab}g_{\mu\nu}\tensor{\Lambda}{^\mu_\alpha}\tensor{\Lambda}{^\nu_\beta}\partial_a  X^\alpha \partial_b X^\beta\\
    \tilde S_{\textnormal{P}}&=-\frac{1}{4\pi\alpha'}\int\dd[2]{\sigma}\sqrt{h}h^{ab}g_{\alpha\beta}\partial_a  X^\alpha \partial_b X^\beta=S_{\textnormal{P}}
\end{align*}

Hence the Poincare group is indeed a global symmetry of the Action. To obtain the conserved currents we have to first know what are the 
equations of motion,

\begin{align*}
    \delta S_{\textnormal{P}}&=-\frac{1}{4\pi\alpha'}\int\dd[2]{\sigma}\qty{\sqrt{h}h^{ab}g_{\mu\nu}2\partial_aX^\mu\delta\partial_b X^\nu+\sqrt{h}\delta h^{ab}g_{\mu\nu}\partial_aX^\mu\partial_b X^\nu-\frac12\sqrt{h}h_{ab}\delta h^{ab}h^{cd}g_{\mu\nu}\partial_cX^\mu\partial_d X^\nu}\\
    \delta S_{\textnormal{P}}&=-\frac{1}{4\pi\alpha'}\int\dd[2]{\sigma}\qty{\sqrt{h}h^{ab}2\partial_aX^\mu\delta\partial_b X_\mu+\sqrt{h}\delta h^{ab}\partial_aX^\mu\partial_b X_\mu-\frac12\sqrt{h}h_{ab}\delta h^{ab}\partial_cX^\mu\partial^c X_\mu}\\
    \delta S_{\textnormal{P}}&=-\frac{1}{2\pi\alpha'}\int\dd[2]{\sigma}\partial_b\qty[\sqrt{h}\partial^bX^\mu\delta X_\mu]+\frac{1}{2\pi\alpha'}\int\dd[2]{\sigma}\partial_b\qty[\sqrt{h}\partial^bX^\mu]\delta X_\mu\\
    &\quad\quad\quad-\frac{1}{4\pi\alpha'}\int\dd[2]{\sigma}\delta h^{ab}\qty[\sqrt{h}\partial_aX^\mu\partial_b X_\mu-\frac12\sqrt{h}h_{ab}\partial_cX^\mu\partial^c X_\mu]
\end{align*}

Each of the three terms has to vanish independently, the first of them is just a Boundary Condition,

\begin{align*}
    \int\dd[2]{\sigma}\partial_b\qty[\sqrt{h}\partial^bX^\mu\delta X_\mu]=0\numberthis\label{boundaryc}
\end{align*}

The second gives the equations for $X$,

\begin{align*}
    \partial_a\qty[\sqrt{h}\partial^aX^\mu]=0\numberthis\label{eomx}
\end{align*}

And the last one give the equations for $h$,

\begin{align*}
    \sqrt{h}\partial_aX^\mu\partial_bX_\mu-\frac12\sqrt{h}h_{ab}\partial_cX^\mu\partial^cX_\mu=0\numberthis\label{eomh}
\end{align*}

Armed with these, we can consider now just a variation on $X$, which is a symmetry of the Action, in our case this will be a Poincare transformation,

\begin{align*}
    \delta S_{\textnormal{P}}&=-\frac{1}{2\pi\alpha'}\int\dd[2]{\sigma}\sqrt{h}\partial^bX^\mu\partial_b \delta X_\mu\\
    \delta S_{\textnormal{P}}&=-\frac{1}{2\pi\alpha'}\int\dd[2]{\sigma}\qty{\partial_b\qty[\sqrt{h}\partial^bX^\mu\delta X_\mu]-\partial_b\qty[\sqrt{h}\partial^bX^\mu]\delta X_\mu}
\end{align*}

Imposing the fields to obey the equations of motion, \ref{eomx}, the second term vanishes identically. And also using our 
already derived result that $\tilde S_{\textnormal{P}}=S_{\textnormal{P}}\Rightarrow\delta S_{\textnormal{P}}=0$, we get the simple 
expression, for $\delta X$ being the variation under a Poincare transformation,

\begin{align*}
    \delta S_{\textnormal{P}}&=-\frac{1}{2\pi\alpha'}\int\dd[2]{\sigma}\partial_b\qty[\sqrt{h}\partial^bX^\mu\delta X_\mu]=0
\end{align*}

In the case of a pure translation, $\delta X=\tilde X-X=a$,

\begin{align*}
    -\frac{1}{2\pi\alpha'}a_\mu\int\dd[2]{\sigma}\partial_b\qty[\sqrt{h}\partial^bX^\mu]=0
\end{align*}

From where we can read the conserved current associated with translations,

\begin{align*}
    \mathcal P^{b\mu}&=-\frac{\sqrt{h}}{2\pi\alpha'}\partial^bX^\mu,\ \ \ \partial_b\mathcal P^{b\mu}=0\numberthis\label{defcalp}
\end{align*}

We can do the same for a Lorentz transformation, $\delta X^\mu=\tilde X^\mu-X^\mu=\tensor{\omega}{^\mu_\nu}X_\nu$, with of course $\omega_{\mu\nu}=-\omega_{\nu\mu}$, 
being the infinitesimal part of the Lorentz transformation, $\Lambda=\mathbbm 1+\omega$,

\begin{align*}
    -\frac{1}{2\pi\alpha'}\int\dd[2]{\sigma}\partial_b\qty[\sqrt{h}\partial^bX^\mu\omega_{\mu\nu}X^\nu]&=0\\
    -\frac{1}{2\pi\alpha'}\omega_{\mu\nu}\int\dd[2]{\sigma}\partial_b\qty[\sqrt{h}\partial^bX^\mu X^\nu-\sqrt{h}\partial^bX^\nu X^\mu]&=0
\end{align*}

So that the conserved current associated with Lorentz transformations is,

\begin{align*}
    \mathcal M^{b\mu\nu}=-\frac{\sqrt{h}}{2\pi\alpha'}\qty[X^\mu\partial^b X^\nu - X^\nu\partial^b X^\mu]=X^\mu\mathcal P^{b\nu}-X^\nu\mathcal P^{b\mu},\ \ \ \partial_b\mathcal M^{b\mu\nu}=0
\end{align*}

Lastly, the conserved charges that follow from the conserved currents are,

\begin{align*}
    P^\mu&=\int\dd{\sigma}\mathcal P^{\tau\mu}=-\frac{1}{2\pi\alpha'}\int\dd{\sigma}\sqrt{h}\partial^\tau X^\mu\numberthis\label{p}\\
    M^{\mu\nu}&=\int\dd{\sigma}\mathcal M^{\tau\mu\nu}=-\frac{1}{2\pi\alpha'}\int\dd{\sigma}\sqrt{h}\qty[X^\mu\partial^\tau X^\nu -X^\nu\partial^\tau X^\mu ]\numberthis\label{m}
\end{align*}

\subsection{}

We now turn to the matter of verifying that the conserved charges derived here do obey the Poincare algebra, for 
this we'll need the Poisson Brackets, which are defined with respect to $X^\mu\qty(t,\sigma)$, and it's conjugate 
momentum $\pdv{\mathcal L}{\partial_\tau X^\mu}=\tensor{{\mathcal P}}{^\tau_\mu}\equiv\Pi_\mu\qty(\tau,\sigma)$, which 
fortunately we already computed. The metric $h$ does not enter in the Poisson Brackets because it's not dynamical, 
it has three degrees of freedom, but we also have three gauge redundancies, just enough to make it non-dynamical. 
The fundamental Poisson Bracket relations are,

\begin{align*}
    \acomm{X^\mu\qty(\tau,\sigma)}{X^\nu\qty(\tau,\sigma')}=0,\ \ \ \acomm{\Pi^\mu\qty(\tau,\sigma)}{\Pi^\nu\qty(\tau,\sigma')}=0,\ \ \ \acomm{X^\mu\qty(\tau,\sigma)}{\Pi^\nu\qty(\tau,\sigma')}=\delta\qty(\sigma-\sigma')g^{\mu\nu}
\end{align*}

Just for completeness, we'll rewrite \ref{p},\ref{m} in function of the canonical variables,

\begin{align*}
    P^\mu&=\int\dd{\sigma}\Pi^\mu\\
    M^{\mu\nu}&=\int\dd{\sigma}\qty[X^\mu\Pi^\nu -X^\nu\Pi^\mu ]
\end{align*}

We'll start by the $P-P$ --- we'll not keep track of the $\tau$ dependence in the conserved charges, because, they are conserved. But 
nevertheless, everything is assumed to be evaluated at equal $\tau$ ---,

\begin{align*}
    \acomm{P^\mu}{P^\nu}=\int\dd{\sigma}\dd{\sigma'}\acomm{\Pi^\mu\qty(\tau,\sigma)}{\Pi^\nu\qty(\tau,\sigma')}=0
\end{align*}

Next the $P-M$,

\begin{align*}
    \acomm{P^\mu}{M^{\alpha\beta}}&=\int\dd{\sigma}\dd{\sigma'}\acomm{\Pi^\mu\qty(\tau,\sigma)}{X^\alpha\qty(\tau,\sigma')\Pi^\beta\qty(\tau,\sigma')-X^\beta\qty(\tau,\sigma')\Pi^\alpha\qty(\tau,\sigma')}\\
    &=\int\dd{\sigma}\dd{\sigma'}\qty[X^\alpha\qty(\tau,\sigma')\acomm{\Pi^\mu\qty(\tau,\sigma)}{\Pi^\beta\qty(\tau,\sigma')}+\acomm{\Pi^\mu\qty(\tau,\sigma)}{X^\alpha\qty(\tau,\sigma')}\Pi^\beta\qty(\tau,\sigma')-\qty(\alpha\leftrightarrow\beta)]\\
    &=\int\dd{\sigma}\dd{\sigma'}\qty[-\Pi^\beta\qty(\tau,\sigma')g^{\mu\alpha}\delta\qty(\sigma-\sigma')-\qty(\alpha\leftrightarrow\beta)]\\
    &=\int\dd{\sigma}\dd{\sigma'}\qty[-\Pi^\beta\qty(\tau,\sigma')g^{\mu\alpha}\delta\qty(\sigma-\sigma')+\Pi^\alpha\qty(\tau,\sigma')g^{\mu\beta}\delta\qty(\sigma-\sigma')]\\
    &=\int\dd{\sigma}\qty[\Pi^\alpha\qty(\tau,\sigma)g^{\mu\beta}-\Pi^\beta\qty(\tau,\sigma)g^{\mu\alpha}]\\
    \acomm{P^\mu}{M^{\alpha\beta}}&=g^{\mu\beta}P^\alpha -g^{\mu\alpha}P^\beta 
\end{align*}

And lastly, the $M-M$,

\begin{align*}
    \acomm{M^{\mu\nu}}{M^{\alpha\beta}}&=\int\dd{\sigma}\dd{\sigma'}\acomm{X^\mu\qty(\tau,\sigma)\Pi^\nu\qty(\tau,\sigma)-X^\nu\qty(\tau,\sigma)\Pi^\mu\qty(\tau,\sigma)}{X^\alpha\qty(\tau,\sigma')\Pi^\beta\qty(\tau,\sigma')-X^\beta\qty(\tau,\sigma')\Pi^\alpha\qty(\tau,\sigma')}\\
    &=\int\dd{\sigma}\dd{\sigma'}\acomm{X^\mu\qty(\tau,\sigma)\Pi^\nu\qty(\tau,\sigma)-X^\nu\qty(\tau,\sigma)\Pi^\mu\qty(\tau,\sigma)}{X^\alpha\qty(\tau,\sigma')\Pi^\beta\qty(\tau,\sigma')}-\qty(\alpha\leftrightarrow\beta)\\
    &=\qty[\int\dd{\sigma}\dd{\sigma'}\acomm{X^\mu\qty(\tau,\sigma)\Pi^\nu\qty(\tau,\sigma)}{X^\alpha\qty(\tau,\sigma')\Pi^\beta\qty(\tau,\sigma')}-\qty(\alpha\leftrightarrow\beta)]-\qty(\mu\leftrightarrow\nu)
\end{align*}

Notice that,

\begin{align*}
    &\acomm{X^\mu\qty(\tau,\sigma)\Pi^\nu\qty(\tau,\sigma)}{X^\alpha\qty(\tau,\sigma')\Pi^\beta\qty(\tau,\sigma')}\\
    &=\acomm{X^\mu\qty(\tau,\sigma)}{X^\alpha\qty(\tau,\sigma')\Pi^\beta\qty(\tau,\sigma')}\Pi^\nu\qty(\tau,\sigma)+X^\mu\qty(\tau,\sigma)\acomm{\Pi^\nu\qty(\tau,\sigma)}{X^\alpha\qty(\tau,\sigma')\Pi^\beta\qty(\tau,\sigma')}\\
    &=X^\alpha\qty(\tau,\sigma')\acomm{X^\mu\qty(\tau,\sigma)}{\Pi^\beta\qty(\tau,\sigma')}\Pi^\nu\qty(\tau,\sigma)+X^\mu\qty(\tau,\sigma)\acomm{\Pi^\nu\qty(\tau,\sigma)}{X^\alpha\qty(\tau,\sigma')}\Pi^\beta\qty(\tau,\sigma')\\
    &=X^\alpha\qty(\tau,\sigma')g^{\mu\beta}\delta\qty(\sigma-\sigma')\Pi^\nu\qty(\tau,\sigma)-X^\mu\qty(\tau,\sigma)g^{\nu\alpha}\delta\qty(\sigma-\sigma')\Pi^\beta\qty(\tau,\sigma')
\end{align*}

Using this back in our expression,

\begin{align*}
    \acomm{M^{\mu\nu}}{M^{\alpha\beta}}&=\qty[\int\dd{\sigma}X^\alpha\qty(\tau,\sigma)g^{\mu\beta}\Pi^\nu\qty(\tau,\sigma)-\Pi^\mu\qty(\tau,\sigma)g^{\nu\alpha}\Pi^\beta\qty(\tau,\sigma)-\qty(\alpha\leftrightarrow\beta)]-\qty(\mu\leftrightarrow\nu)\\
    &=\int\dd{\sigma}\qty[X^\alpha\qty(\tau,\sigma)g^{\mu\beta}\Pi^\nu\qty(\tau,\sigma)-X^\mu\qty(\tau,\sigma)g^{\nu\alpha}\Pi^\beta\qty(\tau,\sigma)]\\
    &\quad\quad\quad-\int\dd{\sigma}\qty[X^\beta\qty(\tau,\sigma)g^{\mu\alpha}\Pi^\nu\qty(\tau,\sigma)-X^\mu\qty(\tau,\sigma)g^{\nu\beta}\Pi^\alpha\qty(\tau,\sigma)]-\qty(\mu\leftrightarrow\nu)\\
    &=\int\dd{\sigma}\qty[X^\alpha\qty(\tau,\sigma)g^{\mu\beta}\Pi^\nu\qty(\tau,\sigma)-X^\mu\qty(\tau,\sigma)g^{\nu\alpha}\Pi^\beta\qty(\tau,\sigma)]\\
    &\quad\quad\quad-\int\dd{\sigma}\qty[X^\beta\qty(\tau,\sigma)g^{\mu\alpha}\Pi^\nu\qty(\tau,\sigma)-X^\mu\qty(\tau,\sigma)g^{\nu\beta}\Pi^\alpha\qty(\tau,\sigma)]\\
    &\quad\quad\quad-\int\dd{\sigma}\qty[X^\alpha\qty(\tau,\sigma)g^{\nu\beta}\Pi^\mu\qty(\tau,\sigma)-X^\nu\qty(\tau,\sigma)g^{\mu\alpha}\Pi^\beta\qty(\tau,\sigma)]\\
    &\quad\quad\quad+\int\dd{\sigma}\qty[X^\beta\qty(\tau,\sigma)g^{\nu\alpha}\Pi^\mu\qty(\tau,\sigma)-X^\nu\qty(\tau,\sigma)g^{\mu\beta}\Pi^\alpha\qty(\tau,\sigma)]
\end{align*}

Collecting the terms with same metric index,

\begin{align*}
    \acomm{M^{\mu\nu}}{M^{\alpha\beta}}&=g^{\mu\beta}\int\dd{\sigma}\qty[X^\alpha\qty(\tau,\sigma)\Pi^\nu\qty(\tau,\sigma)-X^\nu\qty(\tau,\sigma)\Pi^\alpha\qty(\tau,\sigma)]\\
    &\quad\quad\quad+g^{\nu\beta}\int\dd{\sigma}\qty[X^\mu\qty(\tau,\sigma)\Pi^\alpha\qty(\tau,\sigma)-X^\alpha\qty(\tau,\sigma)\Pi^\mu\qty(\tau,\sigma)]\\
    &\quad\quad\quad+g^{\mu\alpha}\int\dd{\sigma}\qty[X^\nu\qty(\tau,\sigma)\Pi^\beta\qty(\tau,\sigma)-X^\beta\qty(\tau,\sigma)\Pi^\nu\qty(\tau,\sigma)]\\
    &\quad\quad\quad+g^{\nu\alpha}\int\dd{\sigma}\qty[X^\beta\qty(\tau,\sigma)\Pi^\mu\qty(\tau,\sigma)-X^\mu\qty(\tau,\sigma)\Pi^\beta\qty(\tau,\sigma)]\\
    \acomm{M^{\mu\nu}}{M^{\alpha\beta}}&=g^{\mu\beta}M^{\alpha\nu}+g^{\nu\beta}M^{\mu\alpha}+g^{\mu\alpha}M^{\nu\beta}+g^{\nu\alpha}M^{\beta\mu}\\
    \acomm{M^{\mu\nu}}{M^{\alpha\beta}}&=g^{\mu\alpha}M^{\nu\beta}-g^{\mu\beta}M^{\nu\alpha}+g^{\nu\beta}M^{\mu\alpha}-g^{\nu\alpha}M^{\mu\beta}
\end{align*}

Summarizing,

\begin{align*}
    \acomm{P^\mu}{P^\nu}&=0\\
    \acomm{P^\mu}{M^{\alpha\beta}}&=g^{\mu\beta}P^\alpha -g^{\mu\alpha}P^\beta\\
    \acomm{M^{\mu\nu}}{M^{\alpha\beta}}&=g^{\mu\alpha}M^{\nu\beta}-g^{\mu\beta}M^{\nu\alpha}+g^{\nu\beta}M^{\mu\alpha}-g^{\nu\alpha}M^{\mu\beta}\numberthis\label{lorentzalgebra}
\end{align*}

Which is exactly the algebra of the Poincare Group!