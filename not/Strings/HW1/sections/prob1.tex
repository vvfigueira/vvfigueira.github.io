\section{Problem 1}
\subsection{1.A)}

The Nambu-Goto Action is given by:

\begin{align*}
    S_{\textnormal{NG}}=-\frac{T_0}{\luz}\int\limits_{-\infty}^{+\infty}\dd{\tau}\int\limits_0^\pi\dd{\sigma}\sqrt{\qty(\dot X\cdot X^\sidev)^2- \qty(\dot X\cdot \dot X)\qty(X^\sidev\cdot X^\sidev) }\numberthis\label{ngaction}
\end{align*}

Where we made the abbreviations,

\begin{align*}
    \dot X^\mu=\pdv{{X^\mu}}{\tau},\ \ \ {X'}^\mu=\pdv{{X^\mu}}{\sigma}
\end{align*}

The choice of the static gauge, together with considering the string being stretched only along the $X^1$ 
direction can be written as,

\begin{align*}
    X^\mu\qty(\tau,\sigma)=\qty(\luz\tau,f\qty(\sigma),0,\cdots,0),\ \ \ f\qty(0)=0\ \&\ f\qty(\pi)=a
\end{align*}

Let's first compute what are the equations of motion,

\begin{align*}
    \delta S_{\textnormal{NG}}&=-\frac{T_0}{\luz}\int\dd[2]{\sigma}\frac{2\qty(\dot X\cdot X^\sidev)\qty({\dot X}^\alpha\delta{X^\sidev}_\alpha+{X^\sidev}^\alpha\delta{\dot X}_\alpha)-2{\dot X}^2{X^\sidev}^\alpha\delta{X^\sidev}_\alpha-2{X^\sidev}^2{\dot X}^\alpha\delta{\dot X}_\alpha}{2\sqrt{\qty(\dot X\cdot X^\sidev)^2- \qty(\dot X\cdot \dot X)\qty(X^\sidev\cdot X^\sidev) }}\\
    &=-\frac{T_0}{\luz}\int\dd[2]{\sigma}\frac{\delta{\dot X}_\alpha\qty[\qty(\dot X\cdot X^\sidev){X^\sidev}^\alpha-{X^\sidev}^2{\dot X}^\alpha]+\delta{X^\sidev}_\alpha\qty[\qty(\dot X\cdot X^\sidev){\dot X}^\alpha-{\dot X}^2{X^\sidev}^\alpha]}{\sqrt{\qty(\dot X\cdot X^\sidev)^2- \qty(\dot X\cdot \dot X)\qty(X^\sidev\cdot X^\sidev) }}
\end{align*}

We define the conjugate momenta as to simplify our expression,

\begin{align*}
    \mathcal P^{\tau\alpha}&=-\frac{T_0}{\luz}\frac{\qty(\dot X\cdot X^\sidev){X^\sidev}^\alpha-{X^\sidev}^2{\dot X}^\alpha}{\sqrt{\qty(\dot X\cdot X^\sidev)^2- \qty(\dot X\cdot \dot X)\qty(X^\sidev\cdot X^\sidev) }}\label{momentumtau}\numberthis\\
    \mathcal P^{\sigma\alpha}&=-\frac{T_0}{\luz}\frac{\qty(\dot X\cdot X^\sidev){\dot X}^\alpha-{\dot X}^2{X^\sidev}^\alpha}{\sqrt{\qty(\dot X\cdot X^\sidev)^2- \qty(\dot X\cdot \dot X)\qty(X^\sidev\cdot X^\sidev) }}\numberthis\label{momentumsigma}
\end{align*}

So that our variation of the Action is,

\begin{align*}
    \delta S_{\textnormal{NG}}&=\int\dd[2]{\sigma}\qty{\delta{\dot X}_\alpha\mathcal P^{\tau\alpha}+\delta{X^\sidev}_\alpha\mathcal P^{\sigma\alpha}}\\
    \delta S_{\textnormal{NG}}&=\int\dd[2]{\sigma}\qty{\pdv{\tau}\qty[\delta{X}_\alpha\mathcal P^{\tau\alpha}]-\delta{X}_\alpha\pdv{\tau}\mathcal P^{\tau\alpha}+\pdv{\sigma}\qty[\delta{X}_\alpha\mathcal P^{\sigma\alpha}]-\delta{X}_\alpha\pdv{\sigma}\mathcal P^{\sigma\alpha}}\\
    \delta S_{\textnormal{NG}}&=-\int\dd[2]{\sigma}\delta{X}_\alpha\qty{\pdv{\tau}\mathcal P^{\tau\alpha}+\pdv{\sigma}\mathcal P^{\sigma\alpha}}+\int\limits_0^\pi\dd{\sigma}\qty[\delta{X}_\alpha\mathcal P^{\tau\alpha}]\eval_{\tau=-\infty}^{\tau=+\infty}+\int\limits_{-\infty}^{+\infty}\dd{\tau}\qty[\delta{X}_\alpha\mathcal P^{\sigma\alpha}]\eval_{\sigma=0}^{\sigma=\pi}
\end{align*}

From imposing the Stationary Action Principle, we can easily read out both the Equations of Motion,

\begin{align*}
    \pdv{\tau}\mathcal P^{\tau\alpha}+\pdv{\sigma}\mathcal P^{\sigma\alpha}=0\numberthis\label{eom}
\end{align*}

And the Boundary Conditions

\begin{align*}
    \delta{X}_\alpha\mathcal P^{\tau\alpha}\eval_{\tau=-\infty}^{\tau=+\infty}=0=\delta{X}_\alpha\mathcal P^{\sigma\alpha}\eval_{\sigma=0}^{\sigma=\pi}\numberthis\label{bc}
\end{align*}

With this three equations in hand, we just have to compute if the stretched string in the Static Gauge is a solution of them, 
first we calculate the derivatives,

\begin{align*}
    \dot X&=\qty(\luz,0,\cdots,0),\ \ \ X'=\qty(0,f'\qty(\sigma),0,\cdots,0)\numberthis\label{derivatives}
\end{align*}

So now it's trivial that,

\begin{align*}
    \dot X\cdot X'=0,\ \ \ \dot X\cdot\dot X=-\luz^2,\ \ \ X'\cdot X'={f'}^2\numberthis\label{dots}
\end{align*}

Plugging in those in \ref{momentumtau},\ref{momentumsigma}:

\begin{align*}
    \mathcal P^{\tau\alpha}&=-\frac{T_0}{\luz}\frac{-{f'}^2{\dot X}^\alpha}{\sqrt{\luz^2{f'}^2}}=\frac{T_0}{\luz}f'\qty(1,0,\cdots,0)\numberthis\label{mtau}\\
    \mathcal P^{\sigma\alpha}&=-\frac{T_0}{\luz}\frac{\luz^2{X^\sidev}^\alpha}{\sqrt{\luz^2{f'}^2}}=-T_0\qty(0,1,0,\cdots,0)\numberthis\label{msigma}
\end{align*}

From where follows,

\begin{align*}
    \pdv{\tau}\mathcal P^{\tau\alpha}=\frac{T_0}{\luz}\pdv{f'}{\tau}\qty(1,0,\cdots,0)=0\\
    \pdv{\sigma}\mathcal P^{\sigma\alpha}=0
\end{align*}

Hence,

\begin{align*}
    \pdv{\tau}\mathcal P^{\tau\alpha}+\pdv{\sigma}\mathcal P^{\sigma\alpha}=0+0=0
\end{align*}

That is, the Equations of Motion, \ref{eom}, are satisfied for this string configuration! 
Now for the Boundary Conditions --- \ref{bc} ---, the first one, is trivially satisfied, that is due to the variations of the 
target space position, $X$, to which the Action is variated by, because, the initial and final time configuration of $X$ are fixed given conditions, 
to change them would mean to solve another problem of initial conditions, so the variation $\delta X$ 
must be zero at the initial and final times,

\begin{align*}
    \delta X_\alpha\eval_{\tau=-\infty}^{\tau=+\infty}=0\Rightarrow \delta X_\alpha\mathcal P^{\tau\alpha}\eval_{\tau=-\infty}^{\tau=+\infty}=0
\end{align*}

What confirms the first Boundary Condition is true. For the second one, let's write the non null contributions to the 
Boundary Condition,

\begin{align*}
    \delta X_\alpha\mathcal P^{\sigma\alpha}\eval_{\sigma=0}^{\sigma=\pi}=\delta X_1\mathcal P^{\sigma1}\eval_{\sigma=0}^{\sigma=\pi}
\end{align*}

This is the case due to all the $\mathcal P^{\sigma\alpha}$ components being zero, except for $\alpha=1$. But we have completely fixed $X_1$ at the 
endpoints, as know as the Dirichlet Boundary Conditions

\begin{align*}
    X_1\qty(\tau,0)=0,\ \ \ X_1\qty(\tau,\pi)=a\Rightarrow \delta X_1\eval_{\sigma=0}^{\sigma=\pi}
\end{align*}

Hence,

\begin{align*}
    \delta X_\alpha\mathcal P^{\sigma\alpha}\eval_{\sigma=0}^{\sigma=\pi}=\delta X_1\mathcal P^{\sigma1}\eval_{\sigma=0}^{\sigma=\pi}=0
\end{align*}

Showing that our string configuration do satisfy the Boundary Conditions. There are two more constrains we have to verify, which follow from \ref{momentumtau},

\begin{align*}
    \mathcal P^{\tau\alpha}{X'}_\alpha&=0\\
    \mathcal P^{\tau\alpha}\mathcal P^{\tau}_\alpha+\frac{T_0^2}{\luz^2}{X'}^2&=0
\end{align*}

First let us show that these are the right constrains,

\begin{align*}
    \mathcal P^{\tau\alpha}{X'}_\alpha&=-\frac{T_0}{\luz}\frac{\qty(\dot X\cdot X^\sidev){X^\sidev\cdot X^\sidev}-{X^\sidev}^2{\dot X\cdot X^\sidev}}{\sqrt{\qty(\dot X\cdot X^\sidev)^2- \qty(\dot X\cdot \dot X)\qty(X^\sidev\cdot X^\sidev) }}=0
\end{align*}

And,

\begin{align*}
    \mathcal P^{\tau\alpha}\mathcal P^{\tau}_\alpha&=\frac{T_0^2}{\luz^2}\frac{\qty(\dot X\cdot X^\sidev)^2{X^\sidev}^2+{X^\sidev}^4{\dot X}^2-2\qty(\dot X\cdot X^\sidev)^2{X^\sidev}^2}{\qty(\dot X\cdot X^\sidev)^2- \qty(\dot X\cdot \dot X)\qty(X^\sidev\cdot X^\sidev) }\\
    \mathcal P^{\tau\alpha}\mathcal P^{\tau}_\alpha&=\frac{T_0^2}{\luz^2}{X^\sidev}^2\frac{-\qty(\dot X\cdot X^\sidev)^2+{X^\sidev}^2{\dot X}^2}{\qty(\dot X\cdot X^\sidev)^2- \qty(\dot X\cdot \dot X)\qty(X^\sidev\cdot X^\sidev) }\\
    \mathcal P^{\tau\alpha}\mathcal P^{\tau}_\alpha&=-\frac{T_0^2}{\luz^2}{X^\sidev}^2\Rightarrow \mathcal P^{\tau\alpha}\mathcal P^{\tau}_\alpha+\frac{T_0^2}{\luz^2}{X^\sidev}^2=0
\end{align*}

Now we'll prove that these two constrains are true for our string configuration, this is easy, as we already have computed all the 
needed vectors, \ref{derivatives},\ref{mtau},

\begin{align*}
    \mathcal P^{\tau\alpha}{X'}_\alpha&=\frac{T_0}{\luz}f'\qty(1,0,\cdots,0)\cdot\trp{\qty(0,f',0,\cdots,0)}=0
\end{align*}

And,

\begin{align*}
    \mathcal P^{\tau\alpha}\mathcal P^{\tau}_\alpha&=\frac{T_0^2}{\luz^2}{f'}^2\qty(1,0,\cdots,0)\cdot\trp{\qty(-1,0,\cdots,0)}=-\frac{T_0^2}{\luz^2}{f'}^2\\
    \mathcal P^{\tau\alpha}\mathcal P^{\tau}_\alpha&=-\frac{T_0^2}{\luz^2}\qty(0,f',\cdots,0)\cdot\trp{\qty(0,f',\cdots,0)}=-\frac{T_0^2}{\luz^2}{X'}^2\\
    \mathcal P^{\tau\alpha}\mathcal P^{\tau}_\alpha+\frac{T_0^2}{\luz^2}{X'}^2&=0
\end{align*}

This finishes our confirmation that indeed this string configuration is a proper solution.

\subsection{1.B)}

To evaluate the Nambu-Goto Action in this solution, we just have to make use of \ref{dots} in \ref{ngaction},

\begin{align*}
    S_{\textnormal{NG-static}}&=-\frac{T_0}{\luz}\int\limits_{-\infty}^{+\infty}\dd{\tau}\int\limits_0^\pi\dd{\sigma}\sqrt{\qty(\dot X\cdot X^\sidev)^2- \qty(\dot X\cdot \dot X)\qty(X^\sidev\cdot X^\sidev) }\\
    S_{\textnormal{NG-static}}&=-\frac{T_0}{\luz}\int\limits_{-\infty}^{+\infty}\dd{\tau}\int\limits_0^\pi\dd{\sigma}\sqrt{\luz^2{f'}^2}=-T_0\int\limits_{-\infty}^{+\infty}\dd{\tau}\int\limits_0^\pi\dd{\sigma}f'\\
    S_{\textnormal{NG-static}}&=-T_0\int\limits_{-\infty}^{+\infty}\dd{\tau}\qty(f\qty(\pi)-f(0))=-T_0\int\limits_{-\infty}^{+\infty}\dd{\tau}a
\end{align*}

If we argue that the Action is of the form,

\begin{align*}
    S&=\int\dd{t}\qty[K-V]
\end{align*}

Where $K$ is the kinetic energy and $V$ is the potential energy. As in our configuration everything is static, we shouldn't expect any kinetic energy present in the Action/Lagrangian, in other words, all the contribution of the action is solely from 
the potential energy, thus, making this identification,

\begin{align*}
    S_{\textnormal{NG-static}}&=-\int\limits_{-\infty}^{+\infty}\dd{\tau}T_0a=-\int\limits_{-\infty}^{+\infty}\dd{\tau}V\\
    V&=T_0a
\end{align*}

This is a hint that $T_0$ may be interpreted as energy per length, or, the tension of the string.