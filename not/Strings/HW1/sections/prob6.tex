\problem{}
\probitem{}

We define the Parity operator such that,

\begin{align*}
    \Omega X^\mu\qty(\tau,\sigma)\Omega^{-1}&=X^\mu\qty(\tau,l-\sigma)
\end{align*}

Let's take a look how this operator acts on each string mode, case by case,

\begin{itemize}
    \item Action in the Closed string
\end{itemize}
\begin{align*}
    X^\mu\qty(\tau,\sigma)&=x^\mu_0+\sqrt{2\alpha'}\alpha^\mu_0\tau+\im \sqrt{\frac{\alpha'}{2}}\sum\limits_{n\in\mathbb Z^\ast}\frac{\e^{-\im n\tau}}{n}\qty({\bar\alpha}^\mu_n\e^{-\im n\sigma}+{\alpha}^\mu_n\e^{\im n\sigma})\\
    \Omega X^\mu\qty(\tau,\sigma)\Omega^{-1}&=\Omega x^\mu_0\Omega^{-1}+\sqrt{2\alpha'}\Omega\alpha^\mu_0\Omega^{-1}\tau+\im \sqrt{\frac{\alpha'}{2}}\sum\limits_{n\in\mathbb Z^\ast}\frac{\e^{-\im n\tau}}{n}\qty(\Omega{\bar\alpha}^\mu_n\Omega^{-1}\e^{-\im n\sigma}+\Omega{\alpha}^\mu_n\Omega^{-1}\e^{\im n\sigma})\\
    X^\mu\qty(\tau,2\pi-\sigma)&=\Omega x^\mu_0\Omega^{-1}+\sqrt{2\alpha'}\Omega\alpha^\mu_0\Omega^{-1}\tau+\im \sqrt{\frac{\alpha'}{2}}\sum\limits_{n\in\mathbb Z^\ast}\frac{\e^{-\im n\tau}}{n}\qty(\Omega{\bar\alpha}^\mu_n\Omega^{-1}\e^{-\im n\sigma}+\Omega{\alpha}^\mu_n\Omega^{-1}\e^{\im n\sigma})
\end{align*}

But, we also have,

\begin{align*}
    X^\mu\qty(\tau,2\pi-\sigma)&=x^\mu_0+\sqrt{2\alpha'}\alpha^\mu_0\tau+\im \sqrt{\frac{\alpha'}{2}}\sum\limits_{n\in\mathbb Z^\ast}\frac{\e^{-\im n\tau}}{n}\qty({\bar\alpha}^\mu_n\e^{-\im n\qty(2\pi-\sigma)}+{\alpha}^\mu_n\e^{\im n\qty(2\pi-\sigma)})\\
    X^\mu\qty(\tau,2\pi-\sigma)&=x^\mu_0+\sqrt{2\alpha'}\alpha^\mu_0\tau+\im \sqrt{\frac{\alpha'}{2}}\sum\limits_{n\in\mathbb Z^\ast}\frac{\e^{-\im n\tau}}{n}\qty({\bar\alpha}^\mu_n\e^{\im n\sigma}+{\alpha}^\mu_n\e^{-\im n\sigma})
\end{align*}

As the Fourier decomposition is unique, we just have to match term by term of the expansion, which gives,

\begin{align*}
    \Omega x_0^\mu \Omega^{-1}&=x_0^\mu\\
    \Omega \alpha_n^\mu \Omega^{-1}&={\bar\alpha}_n^\mu
\end{align*}

As ${\bar \alpha_0}^\mu=\alpha_0^\mu$.

\begin{itemize}
    \item Action in the NN Open string
\end{itemize}
\begin{align*}
    X^{\mu}\qty(\tau,\sigma)&=x^\mu_0+\sqrt{2\alpha'}\alpha^\mu_0\tau+\im \sqrt{2\alpha'}\sum\limits_{n\in\mathbb Z^\ast}\frac{\e^{-\im n\tau}}{n}{\alpha}^\mu_n\cos\qty( n\sigma)\\
    \Omega X ^{\mu}\qty(\tau,\sigma)\Omega^{-1}&=\Omega x^\mu_0\Omega^{-1}+\sqrt{2\alpha'}\Omega\alpha^\mu_0\Omega^{-1}\tau+\im \sqrt{2\alpha'}\sum\limits_{n\in\mathbb Z^\ast}\frac{\e^{-\im n\tau}}{n}\Omega{\alpha}^\mu_n\Omega^{-1}\cos\qty( n\sigma)\\
    X^{\mu}\qty(\tau,\pi-\sigma)&=\Omega x^\mu_0\Omega^{-1}+\sqrt{2\alpha'}\Omega\alpha^\mu_0\Omega^{-1}\tau+\im \sqrt{2\alpha'}\sum\limits_{n\in\mathbb Z^\ast}\frac{\e^{-\im n\tau}}{n}\Omega{\alpha}^\mu_n\Omega^{-1}\cos\qty( n\sigma)
\end{align*}

But,

\begin{align*}
    X^{\mu}\qty(\tau,\pi-\sigma)&=x^\mu_0+\sqrt{2\alpha'}\alpha^\mu_0\tau+\im \sqrt{2\alpha'}\sum\limits_{n\in\mathbb Z^\ast}\frac{\e^{-\im n\tau}}{n}{\alpha}^\mu_n\cos\qty( n\qty(\pi-\sigma))\\
    X^{\mu}\qty(\tau,\pi-\sigma)&=x^\mu_0+\sqrt{2\alpha'}\alpha^\mu_0\tau+\im \sqrt{2\alpha'}\sum\limits_{n\in\mathbb Z^\ast}\frac{\e^{-\im n\tau}}{n}\qty(-{\alpha}^\mu_n)\cos\qty( n\sigma)
\end{align*}

Matching term by term of the Fourier expansion,

\begin{align*}
    \Omega x_0^\mu\Omega^{-1}=x_0,\ \ \ \Omega\alpha_0^\mu\Omega^{-1}=\alpha_0^\mu,\ \ \ \Omega \alpha_n^\mu\Omega^{-1}=-\alpha_n,\ n\in\mathbb Z^\ast
\end{align*}

\begin{itemize}
    \item Action in the DD Open string
\end{itemize}
We didn't derive the DD Open string solution, but we'll argue here what it should be from our knowledge of the NN Open string, the diference of 
the boundary conditions are instead of imposing $\partial_\sigma X^\mu\qty(\tau,0)=\partial_\sigma X^\mu\qty(\tau,\pi)=0$, we impose $\partial_\tau X^\mu\qty(\tau,0)=\partial_\tau X^\mu\qty(\tau,\pi)$. 
From the same arguments preceding Equation \ref{fulleqom}, we get that our solution should be $X^\mu=\frac12\qty(f\qty(\sigma^+)+g\qty(\sigma^-))$, and imposing the first
boundary condition amounts to fixing, ${g'}^\mu\qty(\tau)=-{f'}^\mu\qty(\tau)\Rightarrow g^\mu\qty(\tau)=-f^\mu\qty(\tau)+2x_0^\mu$, so that our solution is $X^\mu=\frac12\qty(f\qty(\sigma^+)-f\qty(\sigma^-)+2x_0^\mu)$, 
imposing the second boundary condition gives, ${f'}^\mu\qty(\tau+\pi)={f'}^\mu\qty(\tau-\pi)$, which implies that ${f'}^\mu$ is a periodic function with period $2\pi$, and now 
we can just make the usual Fourier mode expansion,

\begin{align*}
    {f'}^\mu\qty(\sigma^+)=f_1^\mu+\sqrt{2\alpha'}\sum\limits_{n\in\mathbb N^\ast}\qty(\alpha_{-n}^\mu\e^{\im n \sigma^+}+\alpha_n^\mu\e^{-\im n \sigma})\\
    f^\mu\qty(\sigma^+)=f_0^\mu+f_1^\mu\sigma^+-\im\sqrt{2\alpha'}\sum\limits_{n\in\mathbb N^\ast}\frac1n\qty(\alpha_{-n}^\mu\e^{\im n \sigma^+}-\alpha_n^\mu\e^{-\im n \sigma^+})\\
    f^\mu\qty(\sigma^+)=f_0^\mu+f_1^\mu\sigma^++\im\sqrt{2\alpha'}\sum\limits_{n\in\mathbb Z^\ast}\frac1n\alpha_n^\mu\e^{-\im n \sigma^+}
\end{align*}

So that our solution is,

\begin{align*}
    X^\mu\qty(\tau,\sigma)&=x_0^\mu+\frac12\qty(f^\mu\qty(\sigma^+)-f^\mu\qty(\sigma^-))\\
    X^\mu\qty(\tau,\sigma)&=x_0^\mu+f_1^\mu\sigma+\frac12\im\sqrt{2\alpha'}\sum\limits_{n\in\mathbb Z^\ast}\frac{\alpha_n^\mu}{n}\qty(\e^{-\im n \sigma^+}-\e^{-\im n \sigma^-})\\
    X^\mu\qty(\tau,\sigma)&=x_0^\mu+f_1^\mu\sigma+\frac12\im\sqrt{2\alpha'}\sum\limits_{n\in\mathbb Z^\ast}\frac{\alpha_n^\mu}{n}\e^{-\im n \tau}\qty(\e^{-\im n \sigma}-\e^{\im n \sigma})\\
    X^\mu\qty(\tau,\sigma)&=x_0^\mu+f_1^\mu\sigma+\sqrt{2\alpha'}\sum\limits_{n\in\mathbb Z^\ast}\frac{\alpha_n^\mu}{n}\e^{-\im n \tau}\sin\qty(n\sigma)\\
    X^\mu\qty(\tau,\sigma)&=x_0^\mu+\sqrt{2\alpha'}\alpha_0^\mu\sigma+\sqrt{2\alpha'}\sum\limits_{n\in\mathbb Z^\ast}\frac{\alpha_n^\mu}{n}\e^{-\im n \tau}\sin\qty(n\sigma)
\end{align*}

We just called $f_1^\mu=\sqrt{2\alpha'}\alpha^\mu$. So now we can continue,

\begin{align*}
    \Omega X^\mu\qty(\tau,\sigma)\Omega^{-1}&=\Omega x_0^\mu\Omega^{-1}+\sqrt{2\alpha'}\Omega\alpha_0^\mu\Omega^{-1}\sigma+\sqrt{2\alpha'}\sum\limits_{n\in\mathbb Z^\ast}\frac{\Omega\alpha_n^\mu\Omega^{-1}}{n}\e^{-\im n \tau}\sin\qty(n\sigma)\\
    X^\mu\qty(\tau,\pi-\sigma)&=\Omega x_0^\mu\Omega^{-1}+\sqrt{2\alpha'}\Omega\alpha_0^\mu\Omega^{-1}\sigma+\sqrt{2\alpha'}\sum\limits_{n\in\mathbb Z^\ast}\frac{\Omega\alpha_n^\mu\Omega^{-1}}{n}\e^{-\im n \tau}\sin\qty(n\sigma)
\end{align*}

But,

\begin{align*}
    X^\mu\qty(\tau,\pi-\sigma)&=x_0^\mu+\sqrt{2\alpha'}\alpha_0^\mu\qty(\pi-\sigma)+\sqrt{2\alpha'}\sum\limits_{n\in\mathbb Z^\ast}\frac{\alpha_n^\mu}{n}\e^{-\im n \tau}\sin\qty(n\qty(\pi-\sigma))\\
    X^\mu\qty(\tau,\pi-\sigma)&=x_0^\mu+\sqrt{2\alpha'}\alpha_0^\mu\pi-\sqrt{2\alpha'}\alpha_0^\mu\sigma+\sqrt{2\alpha'}\sum\limits_{n\in\mathbb Z^\ast}\frac{\alpha_n^\mu}{n}\e^{-\im n \tau}\sin\qty(n\sigma)
\end{align*}

Matching term by term,

\begin{align*}
    \Omega x_0^\mu\Omega^{-1}=x_0^\mu+\sqrt{2\alpha'}\alpha_0^\mu\pi,\ \ \ \Omega \alpha_0^\mu\Omega^{-1}=-\alpha_0^\mu,\ \ \ \Omega \alpha_n^\mu\Omega^{-1}=\alpha_n^\mu,\ n\in\mathbb Z^\ast
\end{align*}

\probitem{}

Let's recall the expression for the Hamiltonian, \ref{openhamilton2} and \ref{closedhamilton2}, respectively for 
the open and closed string,

\begin{align*}
    H&=\begin{cases}
        L_0^\perp-1\\L_0^\perp+{\bar L}_0^\perp-2
    \end{cases}
\end{align*}

And also let us recall the expression for the Virasoro zero mode,

\begin{align*}
    L^\perp_0=\frac12\alpha_0^I\alpha_0^I+\sum\limits_{p>0}\alpha^I_{-p}\alpha^I_p\\
    {\bar L}^\perp_0=\frac12\alpha_0^I\alpha_0^I+\sum\limits_{p>0}{\bar \alpha}^I_{-p}{\bar\alpha}^I_p
\end{align*}

\begin{itemize}
    \item Action on the Closed string
\end{itemize}
\begin{align*}
    \Omega L^\perp_0\Omega^{-1}=\frac12\Omega\alpha_0^I\Omega^{-1}\Omega\alpha_0^I\Omega^{-1}+\sum\limits_{p>0}\Omega\alpha^I_{-p}\Omega^{-1}\Omega\alpha^I_p\Omega^{-1}\\
    \Omega L^\perp_0\Omega^{-1}=\frac12{\bar\alpha}_0^I{\bar\alpha}_0^I+\sum\limits_{p>0}{\bar\alpha}^I_{-p}{\bar\alpha}^I_p\\
    \Omega L^\perp_0\Omega^{-1}={\bar L}_0^\perp
\end{align*}

And,

\begin{align*}
    \Omega {\bar L}^\perp_0\Omega^{-1}=\frac12\Omega{\bar \alpha}_0^I\Omega^{-1}\Omega{\bar \alpha}_0^I\Omega^{-1}+\sum\limits_{p>0}\Omega{\bar \alpha}^I_{-p}\Omega^{-1}\Omega{\bar \alpha}^I_p\Omega^{-1}\\
    \Omega {\bar L}^\perp_0\Omega^{-1}=\frac12{\alpha}_0^I{\alpha}_0^I+\sum\limits_{p>0}{\alpha}^I_{-p}{\alpha}^I_p\\
    \Omega {\bar L}^\perp_0\Omega^{-1}={ L}_0^\perp
\end{align*}

Hence,

\begin{align*}
    \Omega H\Omega^{-1}&=\Omega L^\perp_0\Omega+\Omega{\bar L}^\perp_0\Omega^{-1}-2\\
    \Omega H\Omega^{-1}&={\bar L}^\perp_0+ L^\perp_0-2=H
\end{align*}

That is, the closed string is indeed invariant with respect to this symmetry.

\begin{itemize}
    \item Action on the Open NN string
\end{itemize}
\begin{align*}
    \Omega L^\perp_0\Omega^{-1}=\frac12\Omega\alpha_0^I\Omega^{-1}\Omega\alpha_0^I\Omega^{-1}+\sum\limits_{p>0}\Omega\alpha^I_{-p}\Omega^{-1}\Omega\alpha^I_p\Omega^{-1}\\
    \Omega L^\perp_0\Omega^{-1}=\frac12{\alpha}_0^I{\alpha}_0^I+\sum\limits_{p>0}\qty(-1){\alpha}^I_{-p}\qty(-1){\alpha}^I_p\\
    \Omega L^\perp_0\Omega^{-1}={L}_0^\perp
\end{align*}

Hence,

\begin{align*}
    \Omega H\Omega^{-1}&=\Omega L^\perp_0\Omega-1\\
    \Omega H\Omega^{-1}&=L^\perp_0-1=H
\end{align*}

That is, the open NN string is indeed invariant with respect to this symmetry.

\begin{itemize}
    \item Action on the Open DD string
\end{itemize}
\begin{align*}
    \Omega L^\perp_0\Omega^{-1}=\frac12\Omega\alpha_0^I\Omega^{-1}\Omega\alpha_0^I\Omega^{-1}+\sum\limits_{p>0}\Omega\alpha^I_{-p}\Omega^{-1}\Omega\alpha^I_p\Omega^{-1}\\
    \Omega L^\perp_0\Omega^{-1}=\frac12\qty(-1){\alpha}_0^I\qty(-1){\alpha}_0^I+\sum\limits_{p>0}{\alpha}^I_{-p}{\alpha}^I_p\\
    \Omega L^\perp_0\Omega^{-1}={L}_0^\perp
\end{align*}

Hence,

\begin{align*}
    \Omega H\Omega^{-1}&=\Omega L^\perp_0\Omega-1\\
    \Omega H\Omega^{-1}&=L^\perp_0-1=H
\end{align*}

That is, the open DD string is indeed invariant with respect to this symmetry.

\probitem{}

We assume the vacuum is orientation invariant,

\begin{align*}
    \Omega\ket{q^+,q^I}=\ket{q^+,q^I}
\end{align*}

This is similar to the reasoning done in \ref{5b}, the massless condition is, from \ref{5b}, $N^\perp_0+{\bar N}^\perp_0=2$, but, 
we also have the level matching condition, $N^\perp_0={\bar N}^\perp_0$, which simply implies $N^\perp_0={\bar N}^\perp_0=1$ as the unique 
massless state, which is a combination of the operators $\alpha^I_{-1}{\bar \alpha}^J_{-1}$, that is, the most general massless state is, \ref{5b}, 

\begin{align*}
    \ket{\xi}&=\int\limits_0^\infty\dd{q^+}\int\limits_{\mathbb R^{24}}\dd[24]{q^K}\xi_{IJ}\qty(q^+,q^K)\alpha^I_{-1}{\bar\alpha}^J_{-1}\ket{q^+,q^K}
\end{align*}

Now we impose it should be orientation invariant,

\begin{align*}
    \Omega\ket{\xi}&=\int\limits_0^\infty\dd{q^+}\int\limits_{\mathbb R^{24}}\dd[24]{q^K}\xi_{IJ}\qty(q^+,q^K)\Omega\alpha^I_{-1}\Omega^{-1}\Omega{\bar\alpha}^J_{-1}\Omega^{-1}\Omega\ket{q^+,q^K}\\
    \Omega\ket{\xi}&=\int\limits_0^\infty\dd{q^+}\int\limits_{\mathbb R^{24}}\dd[24]{q^K}\xi_{IJ}\qty(q^+,q^K){\bar\alpha}^I_{-1}{\alpha}^J_{-1}\ket{q^+,q^K}\\
    \Omega\ket{\xi}&=\int\limits_0^\infty\dd{q^+}\int\limits_{\mathbb R^{24}}\dd[24]{q^K}\xi_{IJ}\qty(q^+,q^K){\alpha}^J_{-1}{\bar\alpha}^I_{-1}\ket{q^+,q^K}\\
    \Omega\ket{\xi}&=\int\limits_0^\infty\dd{q^+}\int\limits_{\mathbb R^{24}}\dd[24]{q^K}\xi_{JI}\qty(q^+,q^K){\alpha}^I_{-1}{\bar\alpha}^J_{-1}\ket{q^+,q^K}
\end{align*}

We see here, the state $\ket\xi$ is only orientation invariant if $\xi_{IJ}\qty(q^+,q^K)=\xi_{JI}\qty(q^+,q^K)$. Thus, a possible base of massless invariant states is,

\begin{itemize}
    \item $\frac{24\cdot 23}{2}$ Symmetric traceless states
\end{itemize}
\begin{align*}
    \qty(\alpha^I_{-1}{\bar\alpha}^J_{-1}+\alpha^J_{-1}{\bar\alpha}^I_{-1})\ket{q^+,q^K}
\end{align*}

\begin{itemize}
    \item $24$ Trace states
\end{itemize}
\begin{align*}
    \alpha^I_{-1}{\bar\alpha}^I_{-1}\ket{q^+,q^K},\ \ \ \textnormal{no sum}
\end{align*}

\probitem{}

Now for the open string, the Mass operator is,

\begin{align*}
    M^2&=-p^2=2p^+p^--p^Ip^I=2p^+\frac{1}{\sqrt{2\alpha'}}\alpha_0^--p^Ip^I\\
    M^2&=2p^+\frac{1}{2\alpha'}\frac{1}{p^+}\qty(L^\perp_0+a)-\frac{1}{2\alpha'}\alpha_0^I\alpha_0^I\\
    M^2&=\frac{1}{\alpha'}\qty(\frac12\alpha_0^i\alpha_0^i+\sum\limits_{p> 0}\alpha^I_{-p}\alpha^I_{p}+a)-\frac{1}{2\alpha'}\alpha_0^I\alpha_0^I\\
    \alpha'M^2&=N^\perp+a\\
    \alpha'M^2&=N^\perp-1
\end{align*}

Where we also defined the number operator,

\begin{align*}
    N^\perp&=\sum\limits_{p> 0}\alpha^I_{-p}\alpha^I_{p}
\end{align*}

Hence, for the state to be massless it has to have $N^\perp=1$, which implies having just one $\alpha_{-1}^I$ applied to the vacuum, 
the most generic such state is

\begin{align*}
    \ket{\xi}&=\int\limits_0^\infty\dd{q^+}\int\limits_{\mathbb R^{24}}\dd[24]{q^K}\xi_{I}\qty(q^+,q^K)\alpha^I_{-1}\ket{q^+,q^K}
\end{align*}

Now let's compute the action of the reversing orientation operator in each boundary condition case,

\begin{itemize}
    \item NN Open String
\end{itemize}
\begin{align*}
    \Omega\ket{\xi}&=\int\limits_0^\infty\dd{q^+}\int\limits_{\mathbb R^{24}}\dd[24]{q^K}\xi_{I}\qty(q^+,q^K)\Omega\alpha^I_{-1}\Omega^{-1}\Omega\ket{q^+,q^K}\\
    \Omega\ket{\xi}&=\int\limits_0^\infty\dd{q^+}\int\limits_{\mathbb R^{24}}\dd[24]{q^K}\xi_{I}\qty(q^+,q^K)\qty(-1)\alpha^I_{-1}\ket{q^+,q^K}\\
    \Omega\ket{\xi}&=-\ket\xi
\end{align*}

One could argue that if we're looking to the Projective Hilbert space, equivalence classes of rays of the original Hilbert space, then $\ket \xi$ and $-\ket \xi$ 
are both equally good representatives of the same equivalence class of a single ray, thus, as long as we work with the Projective Hilbert space, 
they should be the same state, and hence all the massless NN open string spectrum is orientation invariant. But, as we're in fact gauging this symmetry, 
we're in fact restricting the original Hilbert space to a set of physical states, characterized by choosing an specific eigenvalue of the 
gauge transformation operator --- in our case, we chose $\Omega\ket{\psi}=+1\ket\psi$---, that is, our Physical Hilbert space don't contain the state $\ket\xi$ and neither $-\ket\xi$, so that the Projective version 
of it won't contain none equivalence classes associated with them. Thus, none of the massless NN open string spectrum is orientation invariant.

\begin{itemize}
    \item DD Open String
\end{itemize}
\begin{align*}
    \Omega\ket{\xi}&=\int\limits_0^\infty\dd{q^+}\int\limits_{\mathbb R^{24}}\dd[24]{q^K}\xi_{I}\qty(q^+,q^K)\Omega\alpha^I_{-1}\Omega^{-1}\Omega\ket{q^+,q^K}\\
    \Omega\ket{\xi}&=\int\limits_0^\infty\dd{q^+}\int\limits_{\mathbb R^{24}}\dd[24]{q^K}\xi_{I}\qty(q^+,q^K)\alpha^I_{-1}\ket{q^+,q^K}\\
    \Omega\ket{\xi}&=\ket\xi
\end{align*}

The result is just trivial, all the massless DD open string spectrum are orientation invariant.

\probitem{}
\label{6e}
Let's consider the most general massless state with Chan-Paton factors,

\begin{align*}
    \ket{\xi}&=\int\limits_0^\infty\dd{q^+}\int\limits_{\mathbb R^{24}}\dd[24]{q^K}\xi_{I,ij}\qty(q^+,q^K)\alpha^I_{-1}\ket{ij,q^+,q^K}\\
    \Omega\ket{\xi}&=\int\limits_0^\infty\dd{q^+}\int\limits_{\mathbb R^{24}}\dd[24]{q^K}\xi_{I,ij}\qty(q^+,q^K)\Omega\alpha^I_{-1}\Omega^{-1}\Omega\ket{ij,q^+,q^K}\\
    \Omega\ket{\xi}&=\int\limits_0^\infty\dd{q^+}\int\limits_{\mathbb R^{24}}\dd[24]{q^K}\xi_{I,ij}\qty(q^+,q^K)\alpha^I_{-1}\ket{ji,q^+,q^K}\\
    \Omega\ket{\xi}&=\int\limits_0^\infty\dd{q^+}\int\limits_{\mathbb R^{24}}\dd[24]{q^K}\xi_{I,ji}\qty(q^+,q^K)\alpha^I_{-1}\ket{ij,q^+,q^K}
\end{align*}

This state is only invariant if $\xi_{I,ij}\qty(q^+,q^K)=\xi_{I,ji}\qty(q^+,q^K)$, this means we have a base of states as,

\begin{itemize}
    \item $24\cdot\frac{2N\cdot \qty(2N-1)}{2}$ Symmetric traceless states
\end{itemize}
\begin{align*}
    \alpha^I_{-1}\qty(\ket{ij,q^+,q^K}+\ket{ji,q^+,q^K})
\end{align*}

\begin{itemize}
    \item $24\cdot 2N$ Trace states
\end{itemize}
\begin{align*}
    \alpha^I_{-1}\ket{ii,q^+,q^K}
\end{align*}

\probitem{}
\label{6f}
We assume now $\Omega\ket{q^+,q^I}=-\ket{q^+,q^I}$, and follow as usual,

\begin{align*}
    \ket{\xi}&=\int\limits_0^\infty\dd{q^+}\int\limits_{\mathbb R^{24}}\dd[24]{q^K}\xi_{I,ij}\qty(q^+,q^K)\alpha^I_{-1}\ket{ij,q^+,q^K}\\
    \Omega\ket{\xi}&=\int\limits_0^\infty\dd{q^+}\int\limits_{\mathbb R^{24}}\dd[24]{q^K}\xi_{I,ij}\qty(q^+,q^K)\Omega\alpha^I_{-1}\Omega^{-1}\Omega\ket{ij,q^+,q^K}\\
    \Omega\ket{\xi}&=\int\limits_0^\infty\dd{q^+}\int\limits_{\mathbb R^{24}}\dd[24]{q^K}\xi_{I,ij}\qty(q^+,q^K)\alpha^I_{-1}\qty(-1)\ket{ji,q^+,q^K}\\
    \Omega\ket{\xi}&=\int\limits_0^\infty\dd{q^+}\int\limits_{\mathbb R^{24}}\dd[24]{q^K}\qty(-1)\xi_{I,ji}\qty(q^+,q^K)\alpha^I_{-1}\ket{ij,q^+,q^K}
\end{align*}

This state is only invariant if $\xi_{I,ij}\qty(q^+,q^K)=-\xi_{I,ji}\qty(q^+,q^K)$, this means we have a base of states as,

\begin{itemize}
    \item $24\cdot\frac{2N\cdot \qty(2N-1)}{2}$ Antisymmetric states
\end{itemize}
\begin{align*}
    \alpha^I_{-1}\qty(\ket{ij,q^+,q^K}-\ket{ji,q^+,q^K})
\end{align*}

\probitem{}

First let us recall why the oriented DD open string in $2N$ parallel D-branes has a $U\qty(2N)$ gauge group, first, 
we can characterize all these states in a basis of all the $\qty(2N)^2$ possible choices of a pair $ij$, that is, 
we choose $\qty(2N)^2$ matrices $\qty(2N)^2$ dimensional, $\lambda^{a}_{ij}$, in this way we can construct the new independent states as,

\begin{align*}
    \ket{a,q^+,q^I}=\sum\limits_{ij}\lambda^{a}_{ij}\ket{ij,q^+,q^I}
\end{align*}

But, these new matrices have reality constrains, namely, ${\qty(\lambda_{ji}^a)}^\ast=\lambda_{ij}^a$. This is the definition of an element of the $\mathfrak u\qty(m)$ algebra, for some $m$. To know what is this dimension we note that we have exactly $\qty(2N)^2$ matrices, which are $\qty(2N)^2$ dimensional, as the $\mathfrak u\qty(m)$ algebra dimension is $m^2$, and the fundamental representation is $m^2$ dimensional matrices, it's obvious that $\lambda^a$ are the fundamental representation of the 
$\mathfrak u\qty(2N)$ algebra, so that we say the gauge group is $U\qty(2N)$. Now back to the \ref{6e}, in this case, apart from $\lambda^a\in \mathfrak u\qty(2N)$, we 
also have the constrain of ${\lambda^a}^{\textnormal{T}}=\lambda^a$, which can be seen from trying to construct a generic state from the basis 
we derived in \ref{6e},

\begin{align*}
    \ket{a,q^+,q^I}=\sum\limits_{ij}\lambda^{a}_{ij}\xi_I\alpha^I_{-1}\qty(\ket{ij,q^+,q^I}+\ket{ji,q^+,q^I})
\end{align*}

Which is the same of

\begin{align*}
    \ket{a,q^+,q^I}=\sum\limits_{ij}\lambda^{a}_{ij}\xi_I\alpha^I_{-1}\ket{ij,q^+,q^I}
\end{align*}

As long we guarantee that $\lambda^{a}_{ij}=\lambda^{a}_{ji}$, that is, ${\lambda^a}^{\textnormal{T}}=\lambda^a$, this is a $N\qty(2N+1)$ dimensional subspace of $\mathfrak u\qty(2N)$, 
as there are just as many symmetrical states, this can be seen from the counting of symmetric traceless $\frac{2N\qty(2N-1)}{2}$, plus pure trace $2N$. The symmetry of the generators is a 
characteristic special to the $\mathfrak {sp}\qty(m)$ algebras, which have dimension $m\qty(2m+1)$, so this is exactly our case, the gauge algebra is $\mathfrak{sp}\qty(N)$ and the gauge group is $Sp\qty(N)$. 

Finally, we have \ref{6f}, the additional constrain is ${\lambda^a}^{\textnormal{T}}=-\lambda^a$, which can be seen from the basis we derived in \ref{6f}, 

\begin{align*}
    \ket{a,q^+,q^I}=\sum\limits_{ij}\lambda^{a}_{ij}\xi_I\alpha^I_{-1}\qty(\ket{ij,q^+,q^I}-\ket{ji,q^+,q^I})
\end{align*}

Which is the same of

\begin{align*}
    \ket{a,q^+,q^I}=\sum\limits_{ij}\lambda^{a}_{ij}\xi_I\alpha^I_{-1}\ket{ij,q^+,q^I}
\end{align*}

As long as we guarantee that $\lambda^a_{ij}=-\lambda^a_{ji}$, that is, ${\lambda^a}^{\textnormal{T}}=-\lambda^a$, this is a $N\qty(2N-1)$ dimensional subspace of $\mathfrak u\qty(2N)$, as 
there are just as many anti-symmetrical states, $\frac{2N\qty(2N-1)}{2}$. Notice that this condition already implies $\Tr\qty[\lambda^a]=0$, the anti-symmetry of the generators is a property of 
only the $\mathfrak o\qty(m)$ algebras, but, together with $\Tr\qty[\lambda ^a]=0$ this reduces to $\mathfrak{so}\qty(m)$, which has dimension $\frac{m\qty(m-1)}{2}$, thus, we conclude that the gauge algebra is $\mathfrak {so}\qty(2N)$ and the gauge group $SO\qty(2N)$.