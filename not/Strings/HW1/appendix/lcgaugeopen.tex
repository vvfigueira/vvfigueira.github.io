\section{Light-Cone Gauge for the Open String}
\label{app-openclass}

Here, we're going to fix the gauge and solve the classical equations of motion for the open string with 
Neumann Boundary Conditions at both ends. To get the classical solution we just have to solve both
\ref{eomx} and \ref{eomh}. Of course, we do know the Polyakov Action has  Diff$\times$Weyl gauge freedom, hence, 
none of the components of the metric $h$ are in fact dynamical, because by fixing this gauge freedom we actually 
fix all the components of $h$, what makes of \ref{eomh} not an equation of motion for $h$, but merely a set of 
constrains regarding the choice of gauge for $h$. But, fixing all three components of the metric isn't the 
only possible choice for picking a gauge, another option is to fix $\tau$, and two of the components 
of $h$, as we have in total 3 gauge degrees of freedom. We could in principle assign any function to $\tau$, but 
what seems to be the optimal choice is to work in the Light-cone gauge. First, for any target space vector we define,

\begin{align*}
    A^\pm=\frac{1}{\sqrt2}\qty(A^0\pm A^1)
\end{align*}

And now we state the Light-cone gauge condition, which fixes the $\tau$ reparametrization, given by\footnote{The factor of $2$ here is just convention, it get along pretty well with the 
domain we'll choose of $\sigma\in\qty[0,\pi]$ as a lot of simplifications happen, $\alpha'$ is here just for making $\tau$ dimensionless, and $p^+$ is 
the only conserved charge with no $\tau,\sigma$ dependence and with the correct target space index.},

\begin{align*}
    X^+\qty(\tau,\sigma)=2\alpha'p^+\tau
\end{align*}

Where $p^+$ is the light-cone component of the conserved charge associated with target space translations --- the momentum, \ref{p} ---. This 
condition do not allow for any more reparametrizations of $\tau$, as,

\begin{align*}
    {X'}^+\qty(\tau',\sigma')&=X^+\qty(\tau,\sigma)\\
    2\alpha'p^+\tau'&=2\alpha'p^+\tau
\end{align*}

Which states the only reparametrization compatible with the light-cone gauge condition is $\tau'\qty(\tau,\sigma)=\tau$. Hence, from our 
initial freedom we can only make $\sigma$ reparametrizations now. One way of fixing this remaining reparametrization, is to choose $\sigma$ proportional 
to the energy carried by the string, actually, we're going to choose a reparametrization such that the energy density in the string, at fixed $\tau$, is independent of 
$\sigma$. One might suspect that the energy density is given by $\mathcal P^{\tau0}$, as in \ref{defcalp}, but, in the light-cone coordinates, the one that has 
the role of energy density is $\mathcal P^{\tau+}$, we'll choose a sigma parametrization such that $\partial_\sigma\mathcal P^{\tau+}=0$, to see how can this 
be done we take the transformation of this object under a reparametrization, $\tau'=\tau,\sigma'=\sigma'\qty(\sigma,\tau)$, as,

\begin{align*}
    \mathcal P^{\tau+}&=-\frac{\sqrt h\partial^\tau X^+}{2\pi\alpha'}=-\frac{\sqrt h h^{\tau\tau}}{2\pi\alpha'}2\alpha'p^+=-\sqrt h h^{\tau\tau}\frac{p^+}{\pi}\\
    {\mathcal P'}^{\tau+}\qty(\tau,\sigma')&=-\pdv{\sigma}{\sigma'}\sqrt h h^{\tau\tau}\frac{p^+}{\pi}=\pdv{\sigma}{\sigma'}\mathcal P^{\tau+}
\end{align*}

Hence, given ${\mathcal P'}^{\tau+}\qty(\tau,\sigma')$, we can choose $\sigma=\sigma\qty(\tau,\sigma')$, so that $\partial_\sigma\mathcal P^{\tau+}=0$, as long as,

\begin{align*}
    \pdv{}{\sigma}\qty[{\pdv{\sigma'}{\sigma}}{\mathcal P'}^{\tau+}]&=0\Rightarrow{\pdv[2]{\sigma'}{\sigma}}{\mathcal P'}^{\tau+}+\qty(\pdv{\sigma'}{\sigma})^2\pdv{{\mathcal P'}^{\tau+}}{\sigma'}=0
\end{align*}

This is simply a differential equation which can be solved. But this really fixes all the $\sigma$ reparametrization? Let's see if it's possible to 
make further changes of $\sigma\rightarrow \sigma''$ preserving this condition,

\begin{align*}
    {\pdv{}{\sigma}}\qty[{\pdv{\sigma''}{\sigma}}{\pdv{\sigma'}{\sigma''}}]{\mathcal P'}^{\tau+}+\qty(\pdv{\sigma'}{\sigma''}\pdv{\sigma''}{\sigma})^2\pdv{{\mathcal P'}^{\tau+}}{\sigma'}&=0\\
    {\pdv[2]{\sigma''}{\sigma}}{\pdv{\sigma'}{\sigma''}}{\mathcal P'}^{\tau+}+\qty(\pdv{\sigma''}{\sigma})^2\qty[{\pdv[2]{{\sigma'}}{{\sigma''}}}{\mathcal P'}^{\tau+}
    +\qty(\pdv{\sigma'}{\sigma''})^2\pdv{{\mathcal P'}^{\tau+}}{\sigma'}
    ]&=0
\end{align*}

Imposing that $\sigma''$ also satisfy the parametrization condition,

\begin{align*}
    {\pdv[2]{\sigma''}{\sigma}}{\pdv{\sigma'}{\sigma''}}{\mathcal P'}^{\tau+}&=0\Rightarrow {\pdv[2]{\sigma''}{\sigma}}=0\Rightarrow \sigma''=a\sigma+b\numberthis\label{sigmaresidual}
\end{align*}

Our condition of constancy of $\mathcal P^{\tau+}$ in sigma, has a residual affine reparametrization, which we can use to set $\sigma=0$ in one of the ends of the 
string, and $\sigma=\pi$ in the other one. This completely fixes the $\sigma$ parametrization. What remains now is to fix the Weyl redundancy, as necessarily 
$h_{ab}$ has a inverse, $\Det\qty[h_{ab}]\neq 0$, thus, by a Weyl transformation is always possible to make $\Det\qty[h_{ab}]=-1$ --- We're using Lorentzian signature ---, which 
will be our choice of fixing the Weyl redundancy. There is 
no more room for any transformation now, hence, the gauge is fully fixed. Notice, as $\mathcal P^{\tau+}$ is constant in $\sigma$, we have the following,

\begin{align*}
    p^+&=\int\limits_0^\pi\dd{\sigma}\mathcal P^{\tau+}=\mathcal P^{\tau+}\pi=-p^+\sqrt h h^{\tau\tau}\\
    -1&=h^{\tau\tau}\numberthis\label{httcond}
\end{align*}

Furthermore, using the equation of motion \ref{eomx} at $\mu=+$, we get,

\begin{align*}
    0&=\partial_\tau\qty[h^{\tau\sigma}\partial_\sigma X^+h^{\tau\tau}\partial_\tau X^+]+\partial_\sigma\qty[h^{\sigma\sigma}\partial_\sigma X^++h^{\sigma\tau}\partial_\tau X^+]\\
    0&=\partial_\tau h^{\tau\tau}+\partial_\sigma h^{\sigma\tau}\\
    0&=\partial_\sigma h^{\sigma\tau}
\end{align*}

But, using the Neumann Boundary conditions \ref{boundaryc} at $\mu=+$,

\begin{align*}
    0&=\partial^\sigma X^+\eval_{\sigma=0}^{\sigma=\pi}\\
    0&=h^{\sigma\tau}\partial_\tau X^++h^{\sigma\sigma}\partial_\sigma X^+\eval_{\sigma=0}^{\sigma=\pi}\\
    0&=h^{\tau\sigma}\eval_{\sigma=0}^{\sigma=\pi}\Rightarrow h^{\tau\sigma}\eval_{\sigma=0}=h^{\tau\sigma}\eval_{\sigma=\pi}=0
\end{align*}

Together with $\partial_\sigma h^{\tau\sigma}=0$, this simply states that $h^{\tau\sigma}=h^{\sigma\tau}\equiv0$. This, with $h^{\tau\tau}=-1$ 
and the determinant condition, says that $h=\textnormal{Diag}\mqty(-1&1)$. That simplifies the equations of motion \ref{eomx} to,

\begin{align*}
    \partial_\tau\partial_\tau X^\mu-\partial_\sigma\partial_\sigma X^\mu={\ddot X}^\mu-{X''}^\mu=0
\end{align*}

And also the consistency conditions \ref{eomh} as,

\begin{align*}
    0=&\partial_a X^\mu \partial_b X_\mu-\frac12 h_{ab}\qty(-{\dot X}^2+{X'}^2)\\
    0=&\begin{cases}
        \partial_\tau X^\mu \partial_\tau X_\mu+\frac12\qty(-{\dot X}^2+{X'}^2)&=\frac12\qty[{\dot X}^2+{X'}^2]\\
        \partial_\tau X^\mu \partial_\sigma X_\mu&=\dot X\cdot X'\\
        \partial_\sigma X^\mu \partial_\sigma X_\mu-\frac12\qty(-{\dot X}^2+{X'}^2)&=\frac12\qty[{\dot X}^2+{X'}^2]
    \end{cases}
\end{align*}

These constrains can be recast as,

\begin{align*}
    {\dot X}^2+{X'}^2\pm&=0\\
    \qty(\dot X\pm X')^2&=0\numberthis\label{constrains}
\end{align*}

Lastly, but not less important, we have the Boundary Conditions, \ref{boundaryc}, which take the form,

\begin{align*}
    \partial_\sigma X^\mu\eval_{\sigma=0}=\partial_\sigma X^\mu\eval_{\sigma=\pi}=0
\end{align*}

We have now to solve the equation of motion with the right Boundary conditions, for this, is easier to change 
coordinates to $\sigma^\pm=\tau\pm\sigma$, in which the new equations of motion read,

\begin{align*}
    0&=\pdv[2]{}{\tau} X^\mu-\pdv[2]{}{\sigma}X^\mu\\
    0&=\pdv{}{\tau}\qty[\pdv{\sigma^+}{\tau}\pdv{}{\sigma^+}X^\mu+\pdv{\sigma^-}{\tau}\pdv{}{\sigma^-}X^\mu]-\pdv{}{\sigma}\qty[\pdv{\sigma^+}{\sigma}\pdv{}{\sigma^+}X^\mu+\pdv{\sigma^-}{\sigma}\pdv{}{\sigma^-}X^\mu]\\
    0&=\pdv{}{\tau}\qty[\pdv{}{\sigma^+}X^\mu+\pdv{}{\sigma^-}X^\mu]-\pdv{}{\sigma}\qty[\pdv{}{\sigma^+}X^\mu-\pdv{}{\sigma^-}X^\mu]\\
    0&=\qty(\pdv{}{\sigma^+}+\pdv{}{\sigma^-})\qty[\pdv{}{\sigma^+}X^\mu+\pdv{}{\sigma^-}X^\mu]-\qty(\pdv{}{\sigma^+}-\pdv{}{\sigma^-})\qty[\pdv{}{\sigma^+}X^\mu-\pdv{}{\sigma^-}X^\mu]\\
    0&=\partial_+\partial_-X^\mu\numberthis\label{fulleqom}
\end{align*}

This is easily solved for,

\begin{align*}
    X^\mu=\frac12\qty(f^\mu\qty(\sigma^+)+g^\mu\qty(\sigma^-))
\end{align*}

Imposing the boundary condition at $\sigma=0$,

\begin{align*}
    {X'}^\mu\qty(\tau,\sigma=0)&=\frac12\qty({f'}^\mu\qty(\tau)-{g'}^\mu\qty(\tau))=0
\end{align*}

This states that $g^\mu$ is equal to $f^\mu$ apart from a constant, which we'll absorb in the definition of $f^\mu$. Now, the boundary condition 
on $\sigma=\pi$,

\begin{align*}
    X^\mu&=\frac12\qty(f^\mu\qty(\sigma^+)+f^\mu\qty(\sigma^-))\\
    {X'}^\mu\qty(\tau,\pi)&=\frac12\qty({f'}^\mu\qty(\tau+\pi)-{f'}^\mu\qty(\tau-\pi))=0
\end{align*}

That is, ${f'}^\mu$ is periodic with period $2\pi$. The most general real function with period $2\pi$ is\footnote{The $\alpha'$ factors here are just to make the $\alpha_n^\mu$ dimensionless.},

\begin{align*}
    {f'}^\mu\qty(u)&=f_1^\mu+\sqrt{2\alpha'}\sum\limits_{n=1}^\infty\qty({\alpha^\ast_n}^\mu\exp\qty(\im n u)+\alpha_n^\mu\exp\qty(-\im n u))\\
    f^\mu\qty(u)&=f_0^\mu+f_1^\mu u-\im\sqrt{2\alpha'}\sum\limits_{n=1}^\infty\frac1n\qty({\alpha^\ast_n}^\mu\exp\qty(\im n u)-\alpha_n^\mu\exp\qty(-\im n u))
\end{align*}

Defining $\alpha_{-n}^\mu={\alpha^\ast_n}^\mu$,

\begin{align*}
    f^\mu\qty(u)&=f_0^\mu+f_1^\mu u+\im\sqrt{2\alpha'}\sum\limits_{n\in\mathbb Z^\ast}\frac{\alpha_n^\mu}{n}\exp\qty(-\im n u)
\end{align*}

Now, back in $X$,

\begin{align*}
    X^\mu&=f_0^\mu+\frac12f_1^\mu\qty(\sigma^++\sigma^-)+\im\sqrt{2\alpha'}\sum\limits_{n\in\mathbb Z^\ast}\frac{\alpha_n^\mu}{n}\frac12\qty[\exp\qty(-\im n \sigma^+)+\exp\qty(-\im n\sigma^-)]\\
    X^\mu&=f_0^\mu+f_1^\mu\tau+\im\sqrt{2\alpha'}\sum\limits_{n\in\mathbb Z^\ast}\frac{\alpha_n^\mu}{n}\exp\qty(-\im n\tau)\frac12\qty[\exp\qty(-\im n \sigma)+\exp\qty(\im n\sigma)]\\
    X^\mu&=f_0^\mu+f_1^\mu\tau+\im\sqrt{2\alpha'}\sum\limits_{n\in\mathbb Z^\ast}\frac{\alpha_n^\mu}{n}\exp\qty(-\im n\tau)\cos\qty(n\sigma)   
\end{align*}

Notice that,

\begin{align*}
    p^\mu&=\int\limits_0^\pi\dd{\sigma}\mathcal P^{\tau\mu}=\int\limits_0^\pi\dd{\sigma}\frac{\partial_\tau X^\mu}{2\pi\alpha'},\ \ \ \textnormal{Because }h=\textnormal{Diag}\mqty(-1&1)\\
    p^\mu&=\frac{1}{2\pi\alpha'}\int\limits_0^\pi\dd{\sigma}\qty[f_1^\mu+\sqrt{2\alpha'}\sum\limits_{n\in\mathbb Z^\ast}\alpha_n^\mu\exp\qty(-\im n\tau)\cos\qty(n\sigma)]=\frac{f_1^\mu}{2\alpha'}   
\end{align*}

So that,

\begin{align*}
    X^\mu&=f_0^\mu+2\alpha' p^\mu\tau+\im\sqrt{2\alpha'}\sum\limits_{n\in\mathbb Z^\ast}\frac{\alpha_n^\mu}{n}\exp\qty(-\im n\tau)\cos\qty(n\sigma)   
\end{align*}

And it's clear that,

\begin{align*}
    \frac1\pi\int\limits_0^{\pi}\dd{\sigma}X^\mu\qty(0,\sigma)&=f_0^\mu
\end{align*}

That is, $f_0^\mu$ is the mean position at $\tau=0$, so we relabel it to be $f_0^\mu=x_0^\mu$. It's also useful to define a additional mode, $\alpha_0^\mu=\sqrt{2\alpha'}p^\mu$,

\begin{align*}
    X^\mu&=x_0^\mu+2\alpha' p^\mu\tau+\im\sqrt{2\alpha'}\sum\limits_{n\in\mathbb Z^\ast}\frac{\alpha_n^\mu}{n}\exp\qty(-\im n\tau)\cos\qty(n\sigma)\\
    X^\mu&=x_0^\mu+\sqrt{2\alpha'} \alpha_0^\mu\tau+\im\sqrt{2\alpha'}\sum\limits_{n\in\mathbb Z^\ast}\frac{\alpha_n^\mu}{n}\exp\qty(-\im n\tau)\cos\qty(n\sigma)   
\end{align*}

This is not the end of it, because we need to make sure the two constrains are satisfied, namely, \ref{constrains}. With the convention of 
uppercase latin index referring to non-light-cone components, $I=2,\cdots,D-1$, the constrains are,

\begin{align*}
    0&=-2\qty({\dot X}^-\pm{X'}^-)\qty({\dot X}^+\pm{X'}^+)+\qty({\dot X}^I\pm{X'}^I)^2\\
    4\alpha'p^+\qty({\dot X}^-\pm{X'}^-)&=\qty({\dot X}^I\pm{X'}^I)^2\\
    {\dot X}^-\pm{X'}^-&=\frac{1}{4\alpha'p^+}\qty({\dot X}^I\pm{X'}^I)^2
\end{align*}

That is, the two constrains implies that $X^-$ is not dynamical, and, the collection of $X^I$ fully determine $X^-$, 
apart from a single integration constant, $x_0^-$. To see this is helpful to note,

\begin{align*}
    {\dot X}^\mu&=\sqrt{2\alpha'} \alpha_0^\mu+\sqrt{2\alpha'}\sum\limits_{n\in\mathbb Z^\ast}\alpha_n^\mu\exp\qty(-\im n\tau)\cos\qty(n\sigma)\\
    {\dot X}^\mu&=\sqrt{2\alpha'} \alpha_0^\mu\exp(-\im 0 \tau)\cos\qty(0\sigma)+\sqrt{2\alpha'}\sum\limits_{n\in\mathbb Z^\ast}\alpha_n^\mu\exp\qty(-\im n\tau)\cos\qty(n\sigma)\\
    {\dot X}^\mu&=\sqrt{2\alpha'}\sum\limits_{n\in\mathbb Z}\alpha_n^\mu\exp\qty(-\im n\tau)\cos\qty(n\sigma)
\end{align*}

And,

\begin{align*}
    {X'}^\mu&=-\im\sqrt{2\alpha'}\sum\limits_{n\in\mathbb Z^\ast}\alpha_n^\mu\exp\qty(-\im n\tau)\sin\qty(n\sigma)\\
    {X'}^\mu&=-\im\sqrt{2\alpha'}\alpha_0^\mu\exp\qty(-\im 0\tau)\sin(0\sigma)-\im\sqrt{2\alpha'}\sum\limits_{n\in\mathbb Z^\ast}\alpha_n^\mu\exp\qty(-\im n\tau)\sin\qty(n\sigma)\\
    {X'}^\mu&=-\im\sqrt{2\alpha'}\sum\limits_{n\in\mathbb Z}\alpha_n^\mu\exp\qty(-\im n\tau)\sin\qty(n\sigma)
\end{align*}

Which implies,

\begin{align*}
    {\dot X}^\mu\pm{X'}^\mu&=\sqrt{2\alpha'}\sum\limits_{n\in\mathbb Z}\alpha_n^\mu\exp\qty(-\im n\tau)\qty(\cos\qty(n\sigma)\mp\im\sin\qty(n\sigma))\\
    {\dot X}^\mu\pm{X'}^\mu&=\sqrt{2\alpha'}\sum\limits_{n\in\mathbb Z}\alpha_n^\mu\exp\qty(-\im n\qty(\tau\pm \sigma))\numberthis\label{pmdot}
\end{align*}

So that,

\begin{align*}
    {\dot X}^-\pm{X'}^-&=\frac{1}{4\alpha'{p^+}}\qty({\dot X}^I\pm{X'}^I)^2\\
    \sqrt{2\alpha'}\sum\limits_{n\in\mathbb Z}\alpha^-_n\exp\qty(-\im n\qty(\tau\pm\sigma))&=\frac{2\alpha'}{4\alpha' p^+}\sum\limits_{p,q\in\mathbb Z}\alpha^I_p\alpha^I_q\exp\qty(-\im\qty(p+q)\qty(\tau\pm\sigma))\\
    \sqrt{2\alpha'}\sum\limits_{n\in\mathbb Z}\alpha^-_n\exp\qty(-\im n\qty(\tau\pm\sigma))&=\frac{1}{2p^+}\sum\limits_{n,p\in\mathbb Z}\alpha^I_p\alpha^I_{n-p}\exp\qty(-\im n\qty(\tau\pm\sigma))\\
    \sqrt{2\alpha'}\sum\limits_{n\in\mathbb Z}\alpha^-_n\exp\qty(-\im n\qty(\tau\pm\sigma))&=\frac{1}{p^+}\sum\limits_{n\in\mathbb Z}\qty(\frac12\sum\limits_{p\in\mathbb Z}\alpha^I_p\alpha^I_{n-p})\exp\qty(-\im n\qty(\tau\pm\sigma))\\
    \sqrt{2\alpha'}\alpha_n^-&=\frac{1}{2p^+}\sum\limits_{p\in\mathbb Z}\alpha_p^I\alpha_{n-p}^I=\frac{1}{p^+}L_n^\perp\numberthis\label{virasorodef}
\end{align*}

Hence, all the fourier modes of $X^-$ are completely determined by the transverse modes ones, apart from of course the integration constant $x^-_0$. In the 
last passage we also defined the Virasoro modes $L_n^\perp$. This is it, we fully solved the equation of motion with all the constrains and boundary 
conditions, let's rewrite them here just for completeness,

\begin{align*}
    X^I&=x_0^I+\sqrt{2\alpha'} \alpha_0^I\tau+\im\sqrt{2\alpha'}\sum\limits_{n\in\mathbb Z^\ast}\frac{\alpha_n^I}{n}\exp\qty(-\im n\tau)\cos\qty(n\sigma)\\
    X^-&=x_0^-+\sqrt{2\alpha'} \alpha_0^-\tau+\im\sqrt{2\alpha'}\sum\limits_{n\in\mathbb Z^\ast}\frac{\alpha_n^-}{n}\exp\qty(-\im n\tau)\cos\qty(n\sigma),\ \ \ \alpha^-_n=\frac{1}{\sqrt{2\alpha'}p^+}L^\perp_n\\
    X^+&=2\alpha'p^+\tau
\end{align*}

The degrees of freedom we found are,

\begin{align*}
    X^I\qty(\tau,\sigma),p^+,x_0^-
\end{align*}

Or,

\begin{align*}
    \alpha_n^I,x_0^I,p^+,x_0^-
\end{align*}