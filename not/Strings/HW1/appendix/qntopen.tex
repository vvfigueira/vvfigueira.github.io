\section{Light-Cone Gauge Quantization of the Open String}
\label{app-openqnt}

To get a grasp on how the quantization should proceed we have to look at the classical Polyakov Action \ref{polyakovaction} in the Light-Cone gauge, using \ref{solopen1},\ref{solopen2} and \ref{solopen3},

\begin{align*}
    S_{\textnormal{P}}&=-\frac{1}{4\pi\alpha'}\int\dd[2]{\sigma}\sqrt{h}h^{ab}g_{\mu\nu}\partial_a X^\mu \partial_b X^\nu\\
    &=-\frac{1}{4\pi\alpha'}\int\dd{\tau}\int\limits_0^\pi\dd{\sigma}\qty(-g_{\mu\nu}{\dot X}^\mu {\dot X}^\nu+g_{\mu\nu}{X'}^\mu { X'}^\nu)\\
    &=-\frac{1}{4\pi\alpha'}\int\dd{\tau}\int\limits_0^\pi\dd{\sigma}\qty(2{\dot X}^- {\dot X}^+-{\dot X}^I{\dot X}^I-2{X'}^- { X'}^++{X'}^I{X'}^I)\\
    &=-\frac{1}{4\pi\alpha'}\int\dd{\tau}\int\limits_0^\pi\dd{\sigma}\qty(4{\dot X}^-\alpha'p^+-{\dot X}^I{\dot X}^I+{X'}^I{X'}^I)\\
    &=-\frac{1}{\pi}\int\dd{\tau}\int\limits_0^\pi\dd{\sigma}{\dot X}^-p^+-\frac{1}{4\pi\alpha'}\int\dd{\tau}\int\limits_0^\pi\dd{\sigma}\qty(-{\dot X}^I{\dot X}^I+{X'}^I{X'}^I)\\
    &=-\int\dd{\tau}{\dot x_0}^-p^+-\frac{1}{4\pi\alpha'}\int\dd{\tau}\int\limits_0^\pi\dd{\sigma}\qty(-{\dot X}^I{\dot X}^I+{X'}^I{X'}^I),
\end{align*}

From this we can read the canonical conjugated momenta,

\begin{align*}
    p_-=-p^+&=\fdv{S_{\textnormal{P}}}{{\dot x}_0^-}=-p^+\\
    \mathcal P^{\tau I}&=\fdv{S_{\textnormal{P}}}{{\dot X}^I}=\frac{1}{2\pi\alpha'}{\dot X}^I
\end{align*}

Hence, the Poisson Brackets in this theory will be,

\begin{align*}
    \acomm{X^I\qty(\tau,\sigma)}{\mathcal P^{\tau J}\qty(\tau,\sigma')}&=g^{IJ}\delta\qty(\sigma-\sigma')\\
    \acomm{x_0^-}{p_-}&=g^{IJ}\Rightarrow\acomm{x_0^-}{p^+}=-g^{IJ}
\end{align*}

From which we quantize following $\comm{.}{.}=\im\acomm{.}{.}$,

\begin{align*}
    \comm{X^I\qty(\tau,\sigma)}{\mathcal P^{\tau J}\qty(\tau,\sigma')}&=\im g^{IJ}\delta\qty(\sigma-\sigma'),\ \ \ \comm{x_0^-}{p^+}=-\im\numberthis\label{qntconditions}
\end{align*}

Any other commutator between these $4$ operators is zero. Of course now, $X^-$ has to be considered as a function 
of the $X^I$ and $\mathcal P^{\tau I}$, so, it has non trivial commutators. The first thing we need to do here is go from these 
canonical commutation relations, to the commutation relations of the modes $\alpha_n^I$. The best way of doing this is working with the 
expression \ref{pmdot}, and computing the following commutator,

\begin{align*}
    \comm{{\dot X}^I\pm{X'}^I}{{\dot X}^I\pm{X'}^I}
\end{align*}

For this is useful to note,

\begin{align*}
    \comm{X^I\qty(\tau,\sigma)}{X^J\qty(\tau,\sigma')}=0&\Rightarrow\comm{{X'}^I\qty(\tau,\sigma)}{{X'}^J\qty(\tau,\sigma')}=0\\
    \comm{X^I\qty(\tau,\sigma)}{\mathcal P^{\tau J}\qty(\tau,\sigma')}=\im g^{IJ}\delta\qty(\sigma-\sigma')&\Rightarrow\comm{{X'}^I\qty(\tau,\sigma)}{{\dot X}^J\qty(\tau,\sigma')}=2\pi\alpha'\im g^{IJ}\dv{}{\sigma}\delta\qty(\sigma-\sigma')\\
    \comm{\mathcal P^{\tau I}\qty(\tau,\sigma)}{\mathcal P^{\tau J}\qty(\tau,\sigma')}=0&\Rightarrow\comm{{\dot X}^I\qty(\tau,\sigma)}{{\dot X}^J\qty(\tau,\sigma')}=0
\end{align*}

Hence, the non-vanishing contributions are,

\begin{align*}
    \comm{\qty({\dot X}^I\pm{X'}^I)\qty(\tau,\sigma)}{\qty({\dot X}^J\pm{X'}^J)\qty(\tau,\sigma')}&=\pm\comm{{\dot X}^I\qty(\tau,\sigma)}{{X'}^J\qty(\tau,\sigma')}\pm\comm{{X'}^I\qty(\tau,\sigma)}{{\dot X}^J\qty(\tau,\sigma')}\\
    \comm{\qty({\dot X}^I\pm{X'}^I)\qty(\tau,\sigma)}{\qty({\dot X}^J\pm{X'}^J)\qty(\tau,\sigma')}&=\mp2\pi\alpha'\im g^{IJ}\dv{}{\sigma'}\delta\qty(\sigma'-\sigma)\pm2\pi\alpha'\im g^{IJ}\dv{}{\sigma}\delta\qty(\sigma-\sigma')\\
    \comm{\qty({\dot X}^I\pm{X'}^I)\qty(\tau,\sigma)}{\qty({\dot X}^J\pm{X'}^J)\qty(\tau,\sigma')}&=\pm4\pi\alpha'\im g^{IJ}\dv{}{\sigma}\delta\qty(\sigma-\sigma')\numberthis\label{commpm}
\end{align*}

And with opposite signs,

\begin{align*}
    \comm{\qty({\dot X}^I\pm{X'}^I)\qty(\tau,\sigma)}{\qty({\dot X}^J\mp{X'}^J)\qty(\tau,\sigma')}&=\mp\comm{{\dot X}^I\qty(\tau,\sigma)}{{X'}^J\qty(\tau,\sigma')}\pm\comm{{X'}^I\qty(\tau,\sigma)}{{\dot X}^J\qty(\tau,\sigma')}\\
    \comm{\qty({\dot X}^I\pm{X'}^I)\qty(\tau,\sigma)}{\qty({\dot X}^J\mp{X'}^J)\qty(\tau,\sigma')}&=\pm2\pi\alpha'\im g^{IJ}\dv{}{\sigma'}\delta\qty(\sigma'-\sigma)\pm2\pi\alpha'\im g^{IJ}\dv{}{\sigma}\delta\qty(\sigma-\sigma')\\
    \comm{\qty({\dot X}^I\pm{X'}^I)\qty(\tau,\sigma)}{\qty({\dot X}^J\mp{X'}^J)\qty(\tau,\sigma')}&=0\numberthis\label{commmp}
\end{align*}

But these expressions, and of course $X$ also, are only defined for $\sigma\in\qty[0,\pi]$, as we want to use \ref{pmdot} to 
isolate the Fourier modes, we'll need to integrate over $2\pi$, the only option here is to find an extension of this expression to the whole interval 
$\qty[0,2\pi]$. As everything is periodic in $\sigma$ with $2\pi$ period, it's sufficient to find an extension to $\sigma\in\qty[-\pi,\pi]$. 
See that,

\begin{align*}
    \qty({\dot X}^I+{X'}^I)\qty(\tau,\sigma)&=\sqrt{2\alpha'}\sum\limits_{n\in\mathbb Z}\alpha_n^I\exp\qty(-\im n\qty(\tau+ \sigma)),\ \ \ \sigma\in\qty[0,\pi]\\
    \qty({\dot X}^I-{X'}^I)\qty(\tau,\sigma)&=\sqrt{2\alpha'}\sum\limits_{n\in\mathbb Z}\alpha_n^I\exp\qty(-\im n\qty(\tau- \sigma)),\ \ \ \sigma\in\qty[0,\pi]
\end{align*}

Performing a change of variables $\sigma\rightarrow-\sigma$ in the second expression,

\begin{align*}
    \qty({\dot X}^I+{X'}^I)\qty(\tau,\sigma)&=\sqrt{2\alpha'}\sum\limits_{n\in\mathbb Z}\alpha_n^I\exp\qty(-\im n\qty(\tau+ \sigma)),\ \ \ \sigma\in\qty[0,\pi]\\
    \qty({\dot X}^I-{X'}^I)\qty(\tau,-\sigma)&=\sqrt{2\alpha'}\sum\limits_{n\in\mathbb Z}\alpha_n^I\exp\qty(-\im n\qty(\tau+\sigma)),\ \ \ \sigma\in\qty[-\pi,0]
\end{align*}

That's interesting, because we found a representation of the modes in the whole domain $\qty[-\pi,\pi]$,

\begin{align*}
    A^I\qty(\tau,\sigma)=\sqrt{2\alpha'}\sum\limits_{n\in\mathbb Z}\alpha_n^I\exp\qty(-\im n\qty(\tau+ \sigma))=\begin{cases}
        \qty({\dot X}^I+{X'}^I)\qty(\tau,\sigma),& \sigma\in\qty[0,\pi]\\
        \qty({\dot X}^I-{X'}^I)\qty(\tau,-\sigma),& \sigma\in\qty[-\pi,0]
    \end{cases}\numberthis\label{atrick}
\end{align*}

And of course, as everything where, we still have $A^I\qty(\tau,\sigma+2\pi)=A^I\qty(\tau,\sigma)$. Also, by \ref{commpm} we do have, using the $+$ sign,

\begin{align*}
    \comm{A^I\qty(\tau,\sigma)}{A^J\qty(\tau,\sigma')}=4\pi\alpha'\im g^{IJ}\dv{}{\sigma}\delta\qty(\sigma-\sigma'),\ \ \ \sigma,\sigma'\in\qty[0,\pi]
\end{align*}

By \ref{commmp}, we get that when $\sigma\in\qty[0,\pi]$ and $\sigma'\in\qty[-\pi,0]$ this commutator is zero, which is consistent as the Dirac delta is 
zero in this domain. At last we use \ref{commpm} with both $-$ sign, and $-\sigma,-\sigma'\in\qty[-\pi,0]$, which get us,

\begin{align*}
    \comm{A^I\qty(\tau,\sigma)}{A^J\qty(\tau,\sigma')}=4\pi\alpha'\im g^{IJ}\dv{}{\sigma}\delta\qty(\sigma-\sigma'),\ \ \ \sigma,\sigma'\in\qty[-\pi,0]
\end{align*}

Putting all together, what we have is,

\begin{align*}
    \comm{A^I\qty(\tau,\sigma)}{A^J\qty(\tau,\sigma')}=4\pi\alpha'\im g^{IJ}\dv{}{\sigma}\delta\qty(\sigma-\sigma'),\ \ \ \sigma,\sigma'\in\qty[-\pi,\pi]
\end{align*}

And, also as everything is periodic in $2\pi$, this fully determines the commutation relation over $\sigma,\sigma'\in\qty[0,2\pi]$. Inserting now the definition of 
$A^I\qty(\tau,\sigma)$,

\begin{align*}
    4\pi\alpha'\im g^{IJ}\dv{}{\sigma}\delta\qty(\sigma-\sigma')&=2\alpha'\sum\limits_{n',m'\in\mathbb Z}\e^{-\im n'\qty(\tau+\sigma')}\e^{-\im m'\qty(\tau+\sigma)}\comm{\alpha_{m'}^I}{\alpha_{n'}^J}\\
    2\pi\im g^{IJ}\int_0^{2\pi}\frac{\dd{\sigma}}{2\pi}\dv{}{\sigma}\delta\qty(\sigma-\sigma')\e^{\im m\sigma}&=\int_0^{2\pi}\frac{\dd{\sigma}}{2\pi}\e^{\im m\sigma}\sum\limits_{m',n'\in\mathbb Z}\e^{-\im \tau\qty(n'+m')}\e^{-\im \sigma' n'-\im\sigma m'}\comm{\alpha_{m'}^I}{\alpha_{n'}^J}\\
    mg^{IJ}\e^{\im m\sigma'}&=\sum\limits_{m',n'\in\mathbb Z}\e^{-\im \tau\qty(n'+m')}\e^{-\im \sigma' n'}\delta_{m,m'}\comm{\alpha_{m'}^I}{\alpha_{n'}^J}\\
    \int_0^{2\pi}\frac{\dd{\sigma}}{2\pi}\e^{\im n\sigma'}mg^{IJ}\e^{\im m\sigma'}&=\int_0^{2\pi}\frac{\dd{\sigma}}{2\pi}\e^{\im n\sigma'}\sum\limits_{n'\in\mathbb Z}\e^{-\im \tau\qty(n'+m)}\e^{-\im \sigma' n'}\comm{\alpha_{m}^I}{\alpha_{n'}^J}\\
    mg^{IJ}\delta_{m+n,0}&=\sum\limits_{n'\in\mathbb Z}\delta_{n,n'}\e^{-\im \tau\qty(n'+m)}\comm{\alpha_{m}^I}{\alpha_{n'}^J}\\
    mg^{IJ}\delta_{m+n,0}\e^{\im \tau\qty(n+m)}&=\comm{\alpha_{m}^I}{\alpha_{n}^J}\\
    \comm{\alpha_{m}^I}{\alpha_{n}^J}&=mg^{IJ}\delta_{m+n,0}\numberthis\label{alphacomm}
\end{align*}

This is not the end, we still have one more commutation relation to get,

\begin{align*}
    2\pi\alpha'\im g^{IJ}\delta\qty(\sigma-\sigma')&=\comm{X^I\qty(\tau,\sigma)}{{\dot X}^J\qty(\tau,\sigma')}\\
    \int\limits_0^\pi\dd{\sigma}2\pi\alpha'\im g^{IJ}\delta\qty(\sigma-\sigma')&=\int\limits_0^\pi\dd{\sigma}\comm{X^I\qty(\tau,\sigma)}{{\dot X}^J\qty(\tau,\sigma')},\ \ \ \int\limits_0^\pi\dd{\sigma}\cos\qty(n\sigma)=0,\ n\in\mathbb Z^\ast\\
    2\pi\alpha'\im g^{IJ}&=\pi\comm{x_0^I+\sqrt{2\alpha'}\alpha_0^I\tau}{{\dot X}^J\qty(\tau,\sigma')}\\
    2\alpha'\im g^{IJ}&=\sqrt{2\alpha'}\sum\limits_{n'\in\mathbb Z}\exp\qty(-\im n'\tau)\cos\qty(n'\sigma')\qty{\comm{x_0^I}{\alpha_{n'}^J}+\sqrt{2\alpha'}\tau\comm{\alpha_0^I}{\alpha_{n'}^J}}\\
    \sqrt{2\alpha'}\im g^{IJ}&=\sum\limits_{n'\in\mathbb Z}\exp\qty(-\im n'\tau)\cos\qty(n'\sigma')\comm{x_0^I}{\alpha_{n'}^J}\\
    \int\limits_0^\pi\frac{\dd{\sigma'}}{\pi}\cos\qty(n\sigma')\sqrt{2\alpha'}\im g^{IJ}&=\int\limits_0^\pi\frac{\dd{\sigma'}}{\pi}\cos\qty(n\sigma')\sum\limits_{n'\in\mathbb Z}\exp\qty(-\im n'\tau)\cos\qty(n'\sigma')\comm{x_0^I}{\alpha_{n'}^J}\\
    \delta_{n,0}\sqrt{2\alpha'}\im g^{IJ}&=\sum\limits_{n'\in\mathbb Z}\exp\qty(-\im n'\tau)\delta_{n',n}\comm{x_0^I}{\alpha_{n'}^J}\\
    \comm{x_0^I}{\alpha_{n}^J}&=\delta_{n,0}\sqrt{2\alpha'}\im g^{IJ}\exp\qty(-\im n\tau)=\delta_{n,0}\sqrt{2\alpha'}\im g^{IJ}\numberthis\label{alphaxcomm}
\end{align*}

The last thing we need to discuss before going to the Lorentz generators is about the Virasoro operators, these were defined as,

\begin{align*}
    L^\perp_n&=\frac12\sum\limits_{p\in\mathbb Z}\alpha_{n-p}^I\alpha_{p}\numberthis\label{nonnormalvirasoro}
\end{align*}

We should be aware of possible ambiguities in the ordering of these, as we now have the commutation relations \ref{alphacomm}, two alphas fail 
to commute only if their mode number sum up to $0$, but, notice that $n-p+p=0\rightarrow n=0$, hence, the only not well defined Virasoro mode is 
$L^\perp_0$. As the difference between any two ordering prescriptions is always proportional to the identity operator, we'll \textbf{define} $L^\perp_0$ 
to be on the \textbf{normal ordered} prescription --- All $\alpha_n^I,\ n\geq 0$ need to be to the right of all $\alpha_n^I,\ n<0$ ---, and wherever there is 
mention to this Virasoro mode we should use the normal ordered one plus an addition undetermined normal ordering constant,

\begin{align*}
    L^\perp_0\rightarrow L_0^\perp+a
\end{align*}

As an example, in \ref{virasorodef}, with $n=0$, the quantum version should read,

\begin{align*}
    2\alpha'p^-=\sqrt{2\alpha'}\alpha^-_0&=\frac{1}{p^+}\qty(L_0^\perp+a)\\
    L_0^\perp&=\frac12\alpha_0^I\alpha_0^I+\sum\limits_{p\in\mathbb N^\ast}\alpha_{-p}^I\alpha_p^I
\end{align*}

We can write a manifestly normal ordered form for all $n$,

\begin{align*}
    L_n^\perp=\frac12\sum\limits_{p\geq 0}\alpha_{n-p}^I\alpha_p^I+\frac12\sum\limits_{p<0}\alpha_p^I\alpha_{n-p}^I\numberthis\label{normalvirasoro}
\end{align*}

Using these, every calculation we do is manifestly normal ordered, which will prevent us from making mistakes. As classically we had, $\qty(\alpha_n^I)^\ast=\alpha_{-n}^I$, in the quantization we have, $\qty(\alpha_n^I)^\dagger=\alpha_{-n}^I$. This 
allows us to conclude that, $\qty(L^\perp_n)^\dagger=L_{-n}^\perp$. A few more properties we'll need are the commutation 
relations of the Virasoro modes with all the other objects, we start with,

\begin{align*}
    \comm{L_m^\perp}{\alpha^J_n}&=\frac12\sum\limits_{p\geq 0}\comm{\alpha_{m-p}^I\alpha_p^I}{\alpha_n^J}+\frac12\sum\limits_{p< 0}\comm{\alpha_{p}^I\alpha_{m-p}^I}{\alpha_n^J}\\
    &=\frac12\sum\limits_{p\geq0}\qty{\alpha_{m-p}^I\comm{\alpha_p^I}{\alpha_n^J}+\comm{\alpha_{m-p}^I}{\alpha_n^J}\alpha_p^I}+\frac12\sum\limits_{p<0}\qty{\alpha_{p}^I\comm{\alpha_{m-p}^I}{\alpha_n^J}+\comm{\alpha_{p}^I}{\alpha_n^J}\alpha_{m-p}^I}\\
    &=\frac12\sum\limits_{p\in\mathbb Z}\qty{\alpha_{m-p}^I g^{IJ}p\delta_{p+n,0}+\qty(m-p)\delta_{m-p+n,0}g^{IJ}\alpha_p^I}\\
    &=\frac12\qty{-\alpha_{m+n}^Jn+\qty(m-m-n)\alpha_{m+n}^J}\\
    \comm{L_m^\perp}{\alpha^J_n}&=-n\alpha_{n+m}^J\numberthis\label{virasoroalpha}
\end{align*}

Now,

\begin{align*}
    \comm{L_m^\perp}{x_0^J}&=\frac12\sum\limits_{p\geq0}\comm{\alpha_{m-p}^I\alpha_p^I}{x_0^J}+\frac12\sum\limits_{p<0}\comm{\alpha_{p}^I\alpha_{m-p}^I}{x_0^J}\\
    &=\frac12\sum\limits_{p\geq0}\qty{\alpha_{m-p}^I\comm{\alpha_p^I}{x_0^J}+\comm{\alpha_{m-p}^I}{x_0^J}\alpha_p^I}+\frac12\sum\limits_{p<0}\qty{\alpha_{p}^I\comm{\alpha_{m-p}^I}{x_0^J}+\comm{\alpha_{p}^I}{x_0^J}\alpha_{m-p}^I}\\
    &=\frac12\sum\limits_{p\in\mathbb Z}\qty{-\im\sqrt{2\alpha'}g^{IJ}\delta_{p,0}\alpha_{m-p}^I-\im\sqrt{2\alpha'}g^{IJ}\delta_{m-p,0}\alpha_p^I}\\
    &=-\im\sqrt{2\alpha'}\frac12\qty{\alpha_{m}^J+\alpha_m^J}\\
    \comm{L_m^\perp}{x_0^J}&=-\im\sqrt{2\alpha'}\alpha_m^J\numberthis\label{virasorox}
\end{align*}

And lastly, but not less important, we have to know the commutation relation between the Virasoro modes themselves, this is more subtle, 
because we defined them being normal ordered, thus, every step of the calculation we have to make sure all terms are normal ordered,

\begin{align*}
    \comm{L_m^\perp}{L_n^\perp}&=\frac12\sum\limits_{p\geq 0}\comm{\alpha_{m-p}^I\alpha_{p}^I}{L^\perp_n}+\frac12\sum\limits_{p< 0}\comm{\alpha_{p}^I\alpha_{m-p}^I}{L^\perp_n}\\
    &=\frac12\sum\limits_{p\geq 0}\qty{\alpha_{m-p}^I\comm{\alpha_{p}^I}{L^\perp_n}+\comm{\alpha_{m-p}^I}{L^\perp_n}\alpha_{p}^I}\\
    &\quad\quad\quad+\frac12\sum\limits_{p< 0}\qty{\alpha_{p}^I\comm{\alpha_{m-p}^I}{L^\perp_n}+\comm{\alpha_{p}^I}{L^\perp_n}\alpha_{m-p}^I}\\
    &=\frac12\sum\limits_{p\geq 0}\qty{p\alpha_{m-p}^I\alpha_{p+n}^I+\qty(m-p)\alpha_{n+m-p}^I\alpha_{p}^I}\\
    &\quad\quad\quad+\frac12\sum\limits_{p< 0}\qty{\qty(m-p)\alpha_{p}^I\alpha_{m+n-p}^I+p\alpha_{p+n}^I\alpha_{m-p}^I}\\
    &=\frac12\sum\limits_{p\geq 0}\qty(m-p)\alpha_{n+m-p}^I\alpha_{p}^I+\frac12\sum\limits_{p< 0}\qty(m-p)\alpha_{p}^I\alpha_{m+n-p}^I\\
    &\quad\quad\quad+\frac12\sum\limits_{p\geq 0}p\alpha_{m-p}^I\alpha_{p+n}^I+\frac12\sum\limits_{p< 0}p\alpha_{p+n}^I\alpha_{m-p}^I\\
    &=\frac12\sum\limits_{p\geq 0}\qty(m-p)\alpha_{n+m-p}^I\alpha_{p}^I+\frac12\sum\limits_{p< 0}\qty(m-p)\alpha_{p}^I\alpha_{m+n-p}^I\\
    &\quad\quad\quad+\frac12\sum\limits_{p\geq n}\qty(p-n)\alpha_{m+n-p}^I\alpha_{p}^I+\frac12\sum\limits_{p< n}\qty(p-n)\alpha_{p}^I\alpha_{m+n-p}^I\\
    &=\frac12\sum\limits_{p\geq 0}\qty(m-p)\alpha_{n+m-p}^I\alpha_{p}^I+\frac12\sum\limits_{p< 0}\qty(m-p)\alpha_{p}^I\alpha_{m+n-p}^I\\
    &\quad\quad\quad+\frac12\sum\limits_{p\geq 0}\qty(p-n)\alpha_{m+n-p}^I\alpha_{p}^I+\frac12\sum\limits_{p< 0}\qty(p-n)\alpha_{p}^I\alpha_{m+n-p}^I\\
    &\quad\quad\quad+\frac12\qty(\frac12-\frac{n}{2\abs{n}})\qty[\sum\limits_{p= n}^{-1}\qty(p-n)\alpha_{m+n-p}^I\alpha_{p}^I-\sum\limits_{p= n}^{-1}\qty(p-n)\alpha_{p}^I\alpha_{m+n-p}^I]\\
    &\quad\quad\quad+\frac12\qty(\frac12+\frac{n}{2\abs{n}})\qty[-\sum\limits_{p= 0}^{n-1}\qty(p-n)\alpha_{m+n-p}^I\alpha_{p}^I+\sum\limits_{p= 0}^{n-1}\qty(p-n)\alpha_{p}^I\alpha_{m+n-p}^I]\\
    &=\frac12\qty(m-n)\sum\limits_{p\geq 0}\alpha_{n+m-p}^I\alpha_{p}^I+\frac12\qty(m-n)\sum\limits_{p< 0}\alpha_{p}^I\alpha_{m+n-p}^I\\
    &\quad\quad\quad+\frac12\qty(\frac12-\frac{n}{2\abs{n}})\qty[\sum\limits_{p= n}^{-1}\qty(p-n)\qty{\comm{\alpha_{m+n-p}^I}{\alpha_{p}^I}+\alpha_{p}^I\alpha_{m+n-p}^I}-\sum\limits_{p= n}^{-1}\qty(p-n)\alpha_{p}^I\alpha_{m+n-p}^I]\\
    &\quad\quad\quad+\frac12\qty(\frac12+\frac{n}{2\abs{n}})\qty[-\sum\limits_{p= 0}^{n-1}\qty(p-n)\alpha_{m+n-p}^I\alpha_{p}^I+\sum\limits_{p= 0}^{n-1}\qty(p-n)\qty{\comm{\alpha_{p}^I}{\alpha_{m+n-p}^I}+\alpha_{m+n-p}^I\alpha_{p}^I}]\\
    &=\qty(m-n)L_{m+n}^\perp\\
    &\quad\quad\quad-\frac{g^{II}}{2}\qty(\frac12-\frac{n}{2\abs{n}})\sum\limits_{p= n}^{-1}\qty(p-n)p\delta_{m+n,0}+\frac{g^{II}}{2}\qty(\frac12+\frac{n}{2\abs{n}})\sum\limits_{p= 0}^{n-1}\qty(p-n)p\delta_{m+n,0}\\
    &=\qty(m-n)L_{m+n}^\perp\\
    &\quad\quad\quad+\frac{D-2}{2}\qty(\frac12-\frac{n}{2\abs{n}})\sum\limits_{p= 0}^{\abs{n}-1}\qty(-p-n)p\delta_{m+n,0}+\frac{D-2}{2}\qty(\frac12+\frac{n}{2\abs{n}})\sum\limits_{p= 0}^{\abs{n}-1}\qty(p-n)p\delta_{m+n,0}\\
    &=\qty(m-n)L_{m+n}^\perp\\
    &\quad\quad\quad+\frac{D-2}{2}\delta_{m+n,0}\sum\limits_{p= 0}^{\abs{n}-1}p\qty[\qty(\frac12-\frac{n}{2\abs{n}})\qty(-p-n)+\qty(\frac12+\frac{n}{2\abs{n}})\qty(p-n)]\\
    &=\qty(m-n)L_{m+n}^\perp\\
    &\quad\quad\quad+\frac{D-2}{2}\delta_{m+n,0}\sum\limits_{p= 0}^{\abs{n}-1}p\qty[-n+\frac{n}{2\abs{n}}\qty(p+n)+\frac{n}{2\abs{n}}\qty(p-n)]\\
    &=\qty(m-n)L_{m+n}^\perp+\frac{D-2}{2}\delta_{m+n,0}\frac{n}{\abs{n}}\sum\limits_{p= 0}^{\abs{n}-1}p\qty[p-\abs{n}]
\end{align*}

In the middle of the calculus we introduced factors of $\frac12\qty(1\pm \frac{n}{\abs{n}})$ just to account for the two possible cases, $n>0$ and $n<0$ --- Of course, the case $n=0$ is trivial 
due to not being necessary to introduce any other factors to ensure the normal ordering of the expression---. 
Now, we're going to prove by induction that the value of the sum is,

\begin{align*}
    \sum\limits_{p=0}^{\abs{n}-1}p\qty(p-\abs{n})=\frac16\qty(\abs{n}-\abs{n}^3),\ \ \ \abs{n}\geq 1
\end{align*}

It's trivial to check it's validity from $\abs{n}=1$, now, suppose it's valid for $\abs{n}=k$,

\begin{align*}
    \sum\limits_{p=0}^{k}p\qty(p-k-1)&=\sum\limits_{p=0}^{k}p\qty(p-k)-\sum\limits_{p=0}^kp\\
    \sum\limits_{p=0}^{k}p\qty(p-k-1)&=\sum\limits_{p=0}^{k-1}p\qty(p-k)-\sum\limits_{p=0}^kp\\
    \sum\limits_{p=0}^{k}p\qty(p-k-1)&=\frac16\qty(k-k^3)-\frac12k\qty(k+1)\\
    \sum\limits_{p=0}^{k}p\qty(p-k-1)&=\frac16\qty(-3k^2-2k-k^3)\\
    \sum\limits_{p=0}^{k}p\qty(p-k-1)&=\frac16\qty(k+1-1-3k-3k^2-k^3)=\frac16\qty(k+1-\qty(k+1)^3)
\end{align*}

Which finishes our proof. Hence,

\begin{align*}
    \comm{L_m^\perp}{L_n^\perp}&=\qty(m-n)L_{m+n}^\perp+\frac{D-2}{2}\delta_{m+n,0}\frac{n}{\abs{n}}\sum\limits_{p= 0}^{\abs{n}-1}p\qty[p-\abs{n}]\\
    \comm{L_m^\perp}{L_n^\perp}&=\qty(m-n)L_{m+n}^\perp+\frac{D-2}{12}\delta_{m+n,0}\frac{n}{\abs{n}}\qty(\abs{n}-\abs{n}^3)\\
    \comm{L_m^\perp}{L_n^\perp}&=\qty(m-n)L_{m+n}^\perp+\frac{D-2}{12}\delta_{m+n,0}\qty(n-n^3)\\
    \comm{L_m^\perp}{L_n^\perp}&=\qty(m-n)L_{m+n}^\perp+\frac{D-2}{12}\qty(m^3-m)\delta_{m+n,0}\numberthis\label{virasorovirasoro}
\end{align*}

As completeness, we're going to derive what should be the Hamiltonian of this theory, of course, the Hamiltonian isn't the $X^0$ 
evolution operator, but rather the $\tau$ evolution, we get simply from the Legendre transform,

\begin{align*}
    H&=-p^+{\dot x}_0^-+\int\limits_0^\pi\dd{\sigma}\mathcal P^{\tau I}{\dot X}^I-L\\
    H&=2\pi\alpha'\int\limits_0^\pi\dd{\sigma}\mathcal P^{\tau I}\mathcal P^{\tau I}-\pi\alpha'\int\limits_0^\pi\dd{\sigma}\mathcal P^{\tau I}\mathcal P^{\tau I}+\frac{1}{4\pi\alpha'}\int\limits_0^\pi\dd{\sigma}{X'}^I{X'}^I\\
    H&=\pi\alpha'\int\limits_0^\pi\dd{\sigma}\qty(\mathcal P^{\tau I}\mathcal P^{\tau I}+\frac{1}{4\pi^2{\alpha'}^2}{X'}^I{X'}^I)
\end{align*}

But, notice, by the constrains,

\begin{align*}
    {\dot X}^-+{X'}^-&=\frac{1}{4\alpha'p^+}\qty({\dot X}^I+{X'}^I)^2\\
    {\dot X}^--{X'}^-&=\frac{1}{4\alpha'p^+}\qty({\dot X}^I-{X'}^I)^2\\
\end{align*}

Implies,

\begin{align*}
    {\dot X}^-&=\frac{1}{4\alpha'p^+}\qty({\dot X}^I{\dot X}^I+{X'}^I{X'}^I)\\
    {\dot X}^-&=\frac{4\pi^2{\alpha'}^2}{4\alpha'p^+}\qty(\mathcal P^{\tau I}\mathcal P^{\tau I}+\frac{1}{4\pi^2{\alpha'}^2}{X'}^I{X'}^I)\\
    \frac{p^+}{\pi}{\dot X}^-&=\pi\alpha'\qty(\mathcal P^{\tau I}\mathcal P^{\tau I}+\frac{1}{4\pi^2{\alpha'}^2}{X'}^I{X'}^I)\\
    \frac{p^+}{\pi}\int\limits_0^\pi\dd{\sigma}{\dot X}^-&=\pi\alpha'\int\limits_0^\pi\dd{\sigma}\qty(\mathcal P^{\tau I}\mathcal P^{\tau I}+\frac{1}{4\pi^2{\alpha'}^2}{X'}^I{X'}^I)
\end{align*}

Thus,

\begin{align*}
    H&=\frac{2\pi\alpha'p^+}{\pi}\int\limits_0^\pi\dd{\sigma}\frac{1}{2\pi\alpha'}{\dot X}^-=2\alpha'p^+p^-\numberthis\label{openhamilton}
\end{align*}

We can put it in a more useful form, 

\begin{align*}
    H&=2\alpha'p^+p^-=2\alpha'p^+\frac{1}{\sqrt{2\alpha'}}\frac{1}{\sqrt{2\alpha'}}\frac{1}{p^+}\qty(L^\perp_0+a)\\
    H&=L^\perp_0+a\numberthis\label{openhamilton2}
\end{align*}