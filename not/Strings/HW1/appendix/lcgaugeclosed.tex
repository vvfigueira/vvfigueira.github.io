\section{Light-Cone Gauge for the Closed String}
\label{app-closedclass}

Here, we're going to fix the gauge and solve the classical equations of motion for the closed string. The procedure here 
is pretty much the same as we already done in the Appendix \ref{app-openclass}, the gauge fixing conditions are very similar 
to the ones used there, that is,

\begin{align*}
    X^+\qty(\tau,\sigma)&=\alpha'p^+\tau\\
    \partial_\sigma\mathcal P^{\tau +}&=0\\
    \Det\qty[h_{ab}]&=-1
\end{align*}

This three conditions almost fully fix the gauge, but, as discussed already in Appendix \ref{app-openclass}, the second condition in this 
list allows for a residual gauge \ref{sigmaresidual},

\begin{align*}
    \sigma'=a\sigma+b\numberthis\label{sresclosed}
\end{align*}

But now we have to impose we're dealing with a closed string, that is, the $\sigma$ parameter lives in a topology of a 
circle, in other words, exists $l\in\mathbb R$, such that,

\begin{align*}
    X\qty(\tau,l+\sigma)=X\qty(\tau,\sigma)
\end{align*}

Or in an even better description, $\sigma\sim \sigma+l\textnormal{ mod }l$. Notice that we can use \ref{sres} partially to 
set $l$ to whatever value we like, we'll choose $l=2\pi$, which facilitates some computations, but, the translational 
part of \ref{sres}, $\sigma'=\sigma+b\qty(\tau)$ is impossible to get fully rid off, we have no privileged point to call $\sigma=0$, but, we can remove the 
$\tau$ dependence in it by choosing a specific family of $\sigma=\textnormal{cte}$ curves, this choice will be of the 
$\sigma=\textnormal{cte}$ lines to be orthogonal to the $\tau=\textnormal{cte}$ lines, this is of course another condition 
on the metric,

\begin{align*}
    h^{\sigma\tau}\qty(\tau,\sigma_0)=0
\end{align*}

Where $\sigma_0$ is just any fixed value. This leaves again a residual reparametrization, a solid translation of the $\sigma$ 
coordinate, $\sigma'=\sigma+a$, this cannot be resolved, and we'll have to live with.
With also the same reasoning of \ref{httcond} we can obtain,

\begin{align*}
    p^+&=\int\limits_{\sigma_0}^{\sigma_0+2\pi}\dd{\sigma}\mathcal P^{\tau+}=\mathcal P^{\tau+}2\pi=-\frac{2\pi\alpha' p^+\sqrt h h^{\tau\tau}}{2\pi\alpha'}\\
    -1&=h^{\tau\tau}\numberthis\label{httcondclosed}
\end{align*}

Furthermore, using the equation of motion \ref{eomx} at $\mu=+$, we get,

\begin{align*}
    0&=\partial_\tau\qty[h^{\tau\sigma}\partial_\sigma X^+h^{\tau\tau}\partial_\tau X^+]+\partial_\sigma\qty[h^{\sigma\sigma}\partial_\sigma X^++h^{\sigma\tau}\partial_\tau X^+]\\
    0&=\partial_\tau h^{\tau\tau}+\partial_\sigma h^{\sigma\tau}\\
    0&=\partial_\sigma h^{\sigma\tau}
\end{align*}

Together with $h^{\tau\sigma}\qty(\tau,\sigma_0)=0$, this simply states that $h^{\tau\sigma}=h^{\sigma\tau}\equiv0$. This, with $h^{\tau\tau}=-1$ 
and the determinant condition, says that $h=\textnormal{Diag}\mqty(-1&1)$. That simplifies the equations of motion \ref{eomx} to,

\begin{align*}
    \partial_\tau\partial_\tau X^\mu-\partial_\sigma\partial_\sigma X^\mu={\ddot X}^\mu-{X''}^\mu=0
\end{align*}

And also the consistency conditions \ref{eomh} as,

\begin{align*}
    0=&\partial_a X^\mu \partial_b X_\mu-\frac12 h_{ab}\qty(-{\dot X}^2+{X'}^2)\\
    0=&\begin{cases}
        \partial_\tau X^\mu \partial_\tau X_\mu+\frac12\qty(-{\dot X}^2+{X'}^2)&=\frac12\qty[{\dot X}^2+{X'}^2]\\
        \partial_\tau X^\mu \partial_\sigma X_\mu&=\dot X\cdot X'\\
        \partial_\sigma X^\mu \partial_\sigma X_\mu-\frac12\qty(-{\dot X}^2+{X'}^2)&=\frac12\qty[{\dot X}^2+{X'}^2]
    \end{cases}
\end{align*}

These constrains can be recast as,

\begin{align*}
    {\dot X}^2+{X'}^2\pm&=0\\
    \qty(\dot X\pm X')^2&=0\numberthis\label{constrains2closed}
\end{align*}

And luckily, we don't need to worry about the boundary condition \ref{boundaryc}, as the $\sigma$ lives in a circle, there is no 
boundary on which we could impose this boundary equations, instead we have another consistency equation, which can be seen as 
an additional boundary condition, as we're trying to solve with $\sigma$ in $\mathbb R$ instead of in a circle,

\begin{align*}
    X^\mu\qty(\tau,\sigma+2\pi)&=X^\mu\qty(\tau,\sigma)
\end{align*}

We have now to solve the equation of motion with the consistency conditions, we already computed the equations of motion 
in the variables $\sigma^\pm=\tau\pm\sigma$, \ref{fulleqom},

\begin{align*}
    \partial_+\partial_-X^\mu&=0
\end{align*}

This is easily solved for,

\begin{align*}
    X^\mu=X_L^\mu\qty(\sigma^+)+X_R^\mu\qty(\sigma^-)
\end{align*}

Imposing the consistency condition,

\begin{align*}
    X^\mu\qty(\tau,\sigma+2\pi)&=X^\mu\qty(\tau,\sigma)\\
    X_L^\mu\qty(\sigma^++2\pi)+X_R^\mu\qty(\sigma^--2\pi)&=X_L^\mu\qty(\sigma^+)+X_R^\mu\qty(\sigma^-)\\
    X_L^\mu\qty(\sigma^++2\pi)-X_L^\mu\qty(\sigma^+)&=X_R^\mu\qty(\sigma^-)-X_R^\mu\qty(\sigma^--2\pi)
\end{align*}

As $\sigma^\pm$ are independent variables,

\begin{align*}
    {X_L'}^\mu\qty(\sigma^++2\pi)-{X_L'}^\mu\qty(\sigma^+)&=0={X_R'}^\mu\qty(\sigma^-)-{X_R'}^\mu\qty(\sigma^--2\pi)
\end{align*}

This states that the two functions $X'_{L,R}$ are both periodic in $2\pi$, hence, we can expand then in Fourier modes, 
we write it as,

\begin{align*}
    {X'_L}^\mu\qty(\sigma^+)&=\sqrt{\frac{\alpha'}{2}}\sum\limits_{n\in\mathbb Z}{\bar\alpha}^\mu_n\exp\qty(-\im n \sigma^+)\\
    {X'_R}^\mu\qty(\sigma^-)&=\sqrt{\frac{\alpha'}{2}}\sum\limits_{n\in\mathbb Z}{\alpha}^\mu_n\exp\qty(-\im n \sigma^-)
\end{align*}

Integrating this equation we get,

\begin{align*}
    X_L^\mu\qty(\sigma^+)&=\frac12 x_{0,L}^\mu+\sqrt{\frac{\alpha'}{2}}{\bar\alpha}_0^\mu\sigma^++\im\sqrt{\frac{\alpha'}{2}}\sum\limits_{n\in\mathbb Z^\ast}\frac{{\bar\alpha}^\mu_n}{n}\exp\qty(-\im n \sigma^+)\\
    X_R^\mu\qty(\sigma^-)&=\frac12 x_{0,R}^\mu+\sqrt{\frac{\alpha'}{2}}\alpha_0^\mu\sigma^-+\im\sqrt{\frac{\alpha'}{2}}\sum\limits_{n\in\mathbb Z^\ast}\frac{{\alpha}^\mu_n}{n}\exp\qty(-\im n \sigma^-)
\end{align*}

Let's see what does the consistency condition of the closed string imply also,

\begin{align*}
    X_L^\mu\qty(\sigma^++2\pi)-X_L^\mu\qty(\sigma^+)&=X_R^\mu\qty(\sigma^-)-X_R^\mu\qty(\sigma^--2\pi)\\
    \sqrt{\frac{\alpha'}{2}}{\bar\alpha}_0^\mu2\pi&=\sqrt{\frac{\alpha'}{2}}{\bar\alpha}_0^\mu2\pi\\
    {\bar\alpha}_0^\mu&=\alpha_0^\mu\numberthis\label{lvmat1}
\end{align*}

This allows us to sum both the parts to obtain the full solution,

\begin{align*}
    X^\mu\qty(\tau,\sigma)&=\frac12\qty(x_{0,L}^\mu+x_{0,R}^\mu)+\sqrt{\frac{\alpha'}{2}}{\bar\alpha}_0^\mu\qty(\sigma^++\sigma^-)+\im\sqrt{\frac{\alpha'}{2}}\sum\limits_{n\in\mathbb Z^\ast}\frac{1}{n}\qty({\bar\alpha}^\mu_n\exp\qty(-\im n \sigma^+)+{\alpha}^\mu_n\exp\qty(-\im n \sigma^-))\\
    X^\mu\qty(\tau,\sigma)&=x_{0}^\mu+\sqrt{2\alpha'}{\bar\alpha}_0^\mu\tau+\im\sqrt{\frac{\alpha'}{2}}\sum\limits_{n\in\mathbb Z^\ast}\frac{\e^{-\im n\tau}}{n}\qty({\bar\alpha}^\mu_n\e^{-\im n\sigma}+{\alpha}^\mu_n\e^{\im n\sigma})
\end{align*}

Notice that,

\begin{align*}
    p^\mu&=\int\limits_{\sigma_0}^{\sigma_0+2\pi}\dd{\sigma}\mathcal P^{\tau\mu}=\int\limits_{\sigma_0}^{\sigma_0+2\pi}\dd{\sigma}\frac{\partial_\tau X^\mu}{2\pi\alpha'},\ \ \ \textnormal{Because }h=\textnormal{Diag}\mqty(-1&1)\\
    p^\mu&=\frac{1}{2\pi\alpha'}\int\limits_{\sigma_0}^{2\pi}\dd{\sigma}\qty[\sqrt{2\alpha'}{\bar\alpha_0}^\mu+\sqrt{\frac{\alpha'}{2}}\sum\limits_{n\in\mathbb Z^\ast}\e^{-\im n\tau}\qty({\bar\alpha}_n^\mu\e^{-\im n\sigma}+{\alpha}_n^\mu\e^{\im n\sigma})]=\sqrt{\frac{2}{\alpha'}}{\bar\alpha}_0^\mu\\
    \alpha_0^\mu&={\bar\alpha}_0^\mu=\sqrt{\frac{\alpha'}{2}}p^\mu
\end{align*}

Rewriting again,

\begin{align*}
    X^\mu\qty(\tau,\sigma)&=x_{0}^\mu+\sqrt{2\alpha'}{\bar\alpha}_0^\mu\tau+\im\sqrt{\frac{\alpha'}{2}}\sum\limits_{n\in\mathbb Z^\ast}\frac{\e^{-\im n\tau}}{n}\qty({\bar\alpha}^\mu_n\e^{-\im n\sigma}+{\alpha}^\mu_n\e^{\im n\sigma})
\end{align*}

This is not the end of it, because we need to make sure the two constrains are satisfied, namely, \ref{constrains2closed}. With the convention of 
uppercase latin index referring to non-light-cone components, $I=2,\cdots,D-1$, the constrains are,

\begin{align*}
    0&=-2\qty({\dot X}^-\pm{X'}^-)\qty({\dot X}^+\pm{X'}^+)+\qty({\dot X}^I\pm{X'}^I)^2\\
    2\alpha'p^+\qty({\dot X}^-\pm{X'}^-)&=\qty({\dot X}^I\pm{X'}^I)^2\\
    {\dot X}^-\pm{X'}^-&=\frac{1}{2\alpha'p^+}\qty({\dot X}^I\pm{X'}^I)^2
\end{align*}

That is, the two constrains implies that $X^-$ is not dynamical, and, the collection of $X^I$ fully determine $X^-$, 
apart from a single integration constant, $x_0^-$. To see this is helpful to note,

\begin{align*}
    {\dot X}^\mu&=\sqrt{2\alpha'}{\bar\alpha}_0^\mu+\sqrt{\frac{\alpha'}{2}}\sum\limits_{n\in\mathbb Z^\ast}\e^{-\im n\tau}\qty({\bar\alpha}^\mu_n\e^{-\im n\sigma}+{\alpha}^\mu_n\e^{\im n\sigma})\\
    {\dot X}^\mu&=\sqrt{\frac{\alpha'}{2}}\qty(\alpha_0^\mu+{\bar\alpha}_0^\mu)+\sqrt{\frac{\alpha'}{2}}\sum\limits_{n\in\mathbb Z^\ast}\e^{-\im n\tau}\qty({\bar\alpha}^\mu_n\e^{-\im n\sigma}+{\alpha}^\mu_n\e^{\im n\sigma})\\
    {\dot X}^\mu&=\sqrt{\frac{\alpha'}{2}}\sum\limits_{n\in\mathbb Z}\e^{-\im n\tau}\qty({\bar\alpha}^\mu_n\e^{-\im n\sigma}+{\alpha}^\mu_n\e^{\im n\sigma})
\end{align*}

And,

\begin{align*}
    {X'}^\mu&=\sqrt{\frac{\alpha'}{2}}\sum\limits_{n\in\mathbb Z^\ast}\e^{-\im n\tau}\qty({\bar\alpha}^\mu_n\e^{-\im n\sigma}-{\alpha}^\mu_n\e^{\im n\sigma})\\
    {X'}^\mu&=\sqrt{\frac{\alpha'}{2}}\sum\limits_{n\in\mathbb Z}\e^{-\im n\tau}\qty({\bar\alpha}^\mu_n\e^{-\im n\sigma}-{\alpha}^\mu_n\e^{\im n\sigma})
\end{align*}

Which implies,

\begin{align*}
    {\dot X}^\mu+{X'}^\mu&=\sqrt{2\alpha'}\sum\limits_{n\in\mathbb Z}{\bar\alpha}_n^\mu\exp\qty(-\im n\sigma^+)\numberthis\label{+dot}\\
    {\dot X}^\mu-{X'}^\mu&=\sqrt{2\alpha'}\sum\limits_{n\in\mathbb Z}{\alpha}_n^\mu\exp\qty(-\im n\sigma^-)\numberthis\label{-dot}
\end{align*}

So that,

\begin{align*}
    {\dot X}^-+{X'}^-&=\frac{1}{2\alpha'{p^+}}\qty({\dot X}^I+{X'}^I)^2\\
    \sqrt{2\alpha'}\sum\limits_{n\in\mathbb Z}{\bar\alpha}^-_n\exp\qty(-\im n\sigma^+)&=\frac{2\alpha'}{2\alpha' p^+}\sum\limits_{p,q\in\mathbb Z}{\bar\alpha}^I_p{\bar\alpha}^I_q\exp\qty(-\im\qty(p+q)\sigma^+)\\
    \sqrt{2\alpha'}\sum\limits_{n\in\mathbb Z}{\bar\alpha}^-_n\exp\qty(-\im n\sigma^+)&=\frac{1}{p^+}\sum\limits_{n,p\in\mathbb Z}{\bar\alpha}^I_p{\bar\alpha}^I_{n-p}\exp\qty(-\im n\qty(\tau\pm\sigma))\\
    \sqrt{2\alpha'}\sum\limits_{n\in\mathbb Z}{\bar\alpha}^-_n\exp\qty(-\im n\sigma^+)&=\frac{2}{p^+}\sum\limits_{n\in\mathbb Z}\qty(\frac12\sum\limits_{p\in\mathbb Z}{\bar\alpha}^I_p{\bar\alpha}^I_{n-p})\exp\qty(-\im n\qty(\tau\pm\sigma))\\
    \sqrt{2\alpha'}{\bar\alpha}_n^-&=\frac{1}{p^+}\sum\limits_{p\in\mathbb Z}\alpha_p^I\alpha_{n-p}^I=\frac{2}{p^+}{\bar L}_n^\perp\numberthis\label{virasorodefbarclosed}
\end{align*}

And,

\begin{align*}
    {\dot X}^-\pm{X'}^-&=\frac{1}{2\alpha'{p^+}}\qty({\dot X}^I\pm{X'}^I)^2\\
    \sqrt{2\alpha'}\sum\limits_{n\in\mathbb Z}\alpha^-_n\exp\qty(-\im n\qty(\tau\pm\sigma))&=\frac{2\alpha'}{2\alpha' p^+}\sum\limits_{p,q\in\mathbb Z}\alpha^I_p\alpha^I_q\exp\qty(-\im\qty(p+q)\qty(\tau\pm\sigma))\\
    \sqrt{2\alpha'}\sum\limits_{n\in\mathbb Z}\alpha^-_n\exp\qty(-\im n\qty(\tau\pm\sigma))&=\frac{1}{p^+}\sum\limits_{n,p\in\mathbb Z}\alpha^I_p\alpha^I_{n-p}\exp\qty(-\im n\qty(\tau\pm\sigma))\\
    \sqrt{2\alpha'}\sum\limits_{n\in\mathbb Z}\alpha^-_n\exp\qty(-\im n\qty(\tau\pm\sigma))&=\frac{2}{p^+}\sum\limits_{n\in\mathbb Z}\qty(\frac12\sum\limits_{p\in\mathbb Z}\alpha^I_p\alpha^I_{n-p})\exp\qty(-\im n\qty(\tau\pm\sigma))\\
    \sqrt{2\alpha'}\alpha_n^-&=\frac{1}{p^+}\sum\limits_{p\in\mathbb Z}\alpha_p^I\alpha_{n-p}^I=\frac{2}{p^+}L_n^\perp\numberthis\label{virasorodefclosed}
\end{align*}

Hence, all the Fourier modes of $X^-$ are completely determined by the transverse modes ones, apart from of course the integration constant $x^-_0$. In the 
last passage we also defined the Virasoro modes $L_n^\perp$. This is it, we fully solved the equation of motion with all the constrains and boundary 
conditions, but there's a catch, imposing $\sigma$ lives in a circle gives \ref{lvmat1}, which for $\mu=-$ implies a non-trivial constrain,

\begin{align*}
    {\bar L}^\perp_0=L^\perp_0\numberthis\label{lvmat2}
\end{align*}

As a consequence we get the full set $\alpha_n^I,{\bar\alpha}_n^I$ cannot be chosen freely. Let's rewrite here the full solution for completeness,

\begin{align*}
    X^I&=x_{0}^I+\sqrt{2\alpha'}{\bar\alpha}_0^I\tau+\im\sqrt{\frac{\alpha'}{2}}\sum\limits_{n\in\mathbb Z^\ast}\frac{\e^{-\im n\tau}}{n}\qty({\bar\alpha}^I_n\e^{-\im n\sigma}+{\alpha}^I_n\e^{\im n\sigma})\\
    X^-&=x_{0}^-+\frac{2}{p^+}L^\perp_0\tau+\frac{\im}{p^+}\sum\limits_{n\in\mathbb Z^\ast}\frac{\e^{-\im n\tau}}{n}\qty({\bar L}^\perp_n\e^{-\im n\sigma}+{L}^\perp_n\e^{\im n\sigma})\\
    X^+&=\alpha'p^+\tau\\
    L^\perp_0&={\bar L}^\perp_0
\end{align*}
