\documentclass[a4paper, 12pt]{article}
\usepackage[a4paper,
    left=2cm,
    right=2cm,
    top=2cm,
    bottom=2cm]{geometry}
\usepackage[font=small,labelfont=bf,
   justification=justified,
   format=plain]{caption}
\makeindex
\usepackage[english]{babel}
\usepackage{amsthm}
\usepackage{graphicx}
\usepackage{setspace}
\usepackage{amsmath}
\usepackage{physics}
\usepackage{amssymb}
\usepackage{hyperref}
\usepackage{cleveref}
\usepackage{float}
\usepackage{mathtools}
\usepackage{enumitem}
\usepackage{slashed}
\usepackage{mathrsfs}
\usepackage[compat=1.1.0]{tikz-feynman}
\usepackage{tensor}
\usepackage{bbm}
\usepackage{simpler-wick}
\usepackage[compact,explicit]{titlesec}
\usepackage{stackengine}


\newtheoremstyle{dotless}% name
    {}% space above
    {}% space below
    {\itshape}% body font
    {}% indent amount
    {\bfseries}% theorem head font
    {}% punctuation after theorem head
    { }% space after theorem head
    {}% theorem head spec

\theoremstyle{dotless}

\newtheorem{p}{Exercício}%[section]

\title{\textbf{Homework II}}

\author{\textbf{Vicente V. Figueira --- NUSP 11809301}}

\date{\today}

\newcommand{\chap}[1]{
    \let\secstore\thesection
    \let\thesection\mythesection
    \section{#1}
    \let\thesection\secstore
}
\newcommand{\insrt}[4]{
    \begin{figure}[H]
        \centering
        \includegraphics[width=#2\linewidth]{#1}
        \caption{#3}
        \label{#4}
    \end{figure}
}
\newcommand{\letra}[1]{\begin{itemize}
    \item{\textbf{Item (#1):}}
    \end{itemize}
}
\newcommand{\prob}[1]{\begin{p}\label{#1}\end{p}}
\newcommand{\cqd}{$\hfill\blacksquare$}

\AtBeginDocument{\renewcommand*{\hbar}{{\mkern-1mu\mathchar'26\mkern-8mu\textnormal{h}}}}
\AtBeginDocument{\newcommand{\e}{\textnormal{e}}}
\AtBeginDocument{\newcommand{\im}{\textnormal{i}}}
\AtBeginDocument{\newcommand{\luz}{\textnormal{c}}}
\AtBeginDocument{\newcommand{\grav}{\textnormal{G}}}
\AtBeginDocument{\newcommand{\kb}{{\textnormal{k}_{\textnormal{B}}}}}
\newcommand{\Dd}[1]{\mathcal D #1}
\newcommand{\trp}[1]{{#1}^{\textnormal{T}}}
\newcommand{\Det}[1]{\textup{Det} #1}
\newcommand{\sign}[1]{\textnormal{sign} #1}
\newcommand{\sidev}{{\raisebox{-0.5ex}{$'$}}}
\newcommand{\cnord}[1]{:#1:}
\newcommand{\snord}[1]{%
  \mathbin{%
    \vcenter{%
      \offinterlineskip
      \halign{##\cr
        \raisebox{.3ex}{\scalebox{0.6}{$\ast$}}\cr
        \raisebox{-0.6ex}{\scalebox{0.6}{$\ast$}}\cr
      }%
    }%
  }\ #1\ \mathbin{%
    \vcenter{%
      \offinterlineskip
      \halign{##\cr
        \raisebox{.3ex}{\scalebox{0.6}{$\ast$}}\cr
        \raisebox{-0.6ex}{\scalebox{0.6}{$\ast$}}\cr
      }%
    }%
  }%
}

\newcommand\numberthis{\addtocounter{equation}{1}\tag{\theequation}}

\numberwithin{equation}{section}

\newtheoremstyle{dotless}% name
    {}% space above
    {}% space below
    {}% body font
    {}% indent amount
    {\bfseries}% theorem head font
    {}% punctuation after theorem head
    { }% space after theorem head
    {}% theorem head spec

\theoremstyle{dotless}

\newtheorem{teo}{Teorema}[section]
\newtheorem{defin}{Definição}[section]
\newtheorem{corl}{Corolário}[teo]
\newtheorem{lemm}{Lema}[defin]
\newtheorem{exem}{Exemplo}[section]
\newtheorem{prov}{Prova:}[section]
\newenvironment{prova}{\paragraph{\nbold{Prova:}}}{\hfill$\blacksquare$\par}
\newcommand{\nbold}[1]{\normalfont{\textbf{#1}}}
\newcommand{\unit}{\mbox{\normalfont{id$_{\vb H}$}}}

%% \doublespacing

%\titleformat{\section}[runin]{\large\bfseries}{}{0pt}{Problem \thesection}

\newcommand{\problem}{%
  \refstepcounter{section}%
  \addcontentsline{toc}{section}{Problem \thesection}%
  \vspace{1.5em}
  \noindent
  {\large\bfseries Problem \thesection}%
  \par\nopagebreak\vspace{0.5em}
}

\newcommand{\probitem}{%
  \refstepcounter{subsection}%
  \addcontentsline{toc}{subsection}{\thesubsection}%
  \vspace{1.5em}
  \noindent
  {\large\bfseries\thesubsection}%
  \par\nopagebreak\vspace{0.5em}
}


\renewcommand{\thesubsection}{\thesection.\Alph{subsection})}

\allowdisplaybreaks

\begin{document}

\maketitle

\tableofcontents

\section{Problem 1}
\subsection{1.A)}
\subsection{1.B)}

\newpage

\section{Problem 2}
\subsection{2.A)}
\subsection{2.B)}

\newpage

\problem{}
\probitem{}

Our action is,

\begin{align*}
    S&=\frac{1}{4\pi}\int\dd[2]{z}\psi\bar\partial\psi
\end{align*}

To obtain the equation of motion is simple, first we set up the path integral of a total derivative, which is zero, the argument of the total derivative 
we set to $\exp\qty(-S)\mathcal O$, where $\mathcal O$ is any combination of local fields that does not contain $\psi\qty(z_1,{\bar z}_1)$, then,

\begin{align*}
    0&=\int\mathcal D\psi\fdv{}{\psi\qty(z_1,\bar z_1)}\qty[\exp\qty(-S)\mathcal O]\\
    0&=\int\mathcal D\psi\fdv{}{\psi\qty(z_1,\bar z_1)}\qty[\exp\qty(-S)]\mathcal O
\end{align*}

As $\fdv{}{\psi\qty(z_1,{\bar z}_1)}\mathcal O$ is zero except for $\mathcal O=\psi\qty(z_1,{\bar z}_1)$, to which it's $\delta^{\qty(2)}\qty(0)$, but, that's not the case,

\begin{align*}
    0&=\int\mathcal D\psi\fdv{}{\psi\qty(z_1,\bar z_1)}\qty[\exp\qty(-S)]\mathcal O\\
    0&=-\frac{1}{4\pi}\int\mathcal D\psi\fdv{}{\psi\qty(z_1,\bar z_1)}\qty[\int\dd[2]{z_2}\psi\qty(z_2,\bar z_2)\partial_{{\bar z}_2}\psi\qty(z_2,{\bar z}_2)]\mathcal O\exp\qty(-S)\\
    0&=-\frac{1}{4\pi}\int\mathcal D\psi\int\dd[2]{z_2}\qty[\delta^{\qty(2)}\qty(z_2-z_1)\partial_{{\bar z}_2}\psi\qty(z_2,{\bar z}_2)-\psi\qty(z_2,\bar z_2)\partial_{{\bar z}_2}\delta^{\qty(2)}\qty(z_2-z_1)]\mathcal O\exp\qty(-S)
\end{align*}

As long as we're dealing with the closed fermion, we need not to worry about the boundary conditions, as there is no boundary, so,

\begin{align*}
    0&=-\frac{1}{4\pi}\int\mathcal D\psi\int\dd[2]{z_2}\qty[\delta^{\qty(2)}\qty(z_2-z_1)\partial_{{\bar z}_2}\psi\qty(z_2,{\bar z}_2)+\partial_{{\bar z}_2}\psi\qty(z_2,\bar z_2)\delta^{\qty(2)}\qty(z_2-z_1)]\mathcal O\exp\qty(-S)\\
    0&=-\frac{1}{2\pi}\int\mathcal D\psi\partial_{{\bar z}_1}\psi\qty(z_1,{\bar z}_1)\mathcal O\exp\qty(-S)
\end{align*}

What in the Operator formalism would account for some radially ordered expectation value under some state, which is specified with the boundary conditions on the path integral itself, 

\begin{align*}
    \partial_{{\bar z}_1}\expval{\psi\qty(z_1,{\bar z}_1)\mathcal O}&=0
\end{align*}

As both the operator $\mathcal O$ and the state are arbitrary, as long as there is no insertion of $\psi\qty(z_1,{\bar z}_1)$, we conclude the 
equation of motion in the operator form is just,

\begin{align*}
    \bar\partial\psi=0\Rightarrow \psi\qty(z,\bar z)\equiv\psi\qty(z)
\end{align*}

That is, $\psi$ is, at least, a meromorphic operator/function. Now, let's repeat this to obtain the two point function, this time we include in 
$\mathcal O$ a single factor of $\psi\qty(z_1,{\bar z}_1)$ disguised as $\psi\qty(z_2,{\bar z}_2)$, this also can be phrased as we being interested 
in the limit $z_2\rightarrow z_1$,

\begin{align*}
    0&=\int\mathcal D\psi\fdv{}{\psi\qty(z_1,\bar z_1)}\qty[\exp\qty(-S)\psi\qty(z_2,{\bar z}_2)\mathcal O]\\
    0&=\int\mathcal D\psi\fdv{}{\psi\qty(z_1,\bar z_1)}\qty[\exp\qty(-S)\psi\qty(z_2,{\bar z}_2)]\mathcal O\\
    0&=\int\mathcal D\psi\qty{\fdv{}{\psi\qty(z_1,\bar z_1)}\qty[\exp\qty(-S)]\psi\qty(z_2,{\bar z}_2)+\fdv{}{\psi\qty(z_1,{\bar z}_1)}\qty[\psi\qty(z_2,{\bar z}_2)]\exp\qty(-S)}\mathcal O\\
    0&=\int\mathcal D\psi\qty{\fdv{}{\psi\qty(z_1,\bar z_1)}\qty[-S]\psi\qty(z_2,{\bar z}_2)+\delta^{\qty(2)}\qty(z_2-z_1)}\exp\qty(-S)\mathcal O
\end{align*}

As we derived before, $\fdv{}{\psi\qty(z_1,\bar z_1)}\qty[-S]=-\frac{1}{2\pi}\partial_{{\bar z}_1}\psi\qty(z_1,{\bar z}_1)$,

\begin{align*}
    0&=\int\mathcal D\psi\qty{-\frac{1}{2\pi}\partial_{{\bar z}_1}\psi\qty(z_1,{\bar z}_1)\psi\qty(z_2,{\bar z}_2)+\delta^{\qty(2)}\qty(z_2-z_1)}\exp\qty(-S)\mathcal O
\end{align*}

Which is, translating to the operator formalism,

\begin{align*}
    \partial_{{\bar z}_1}\expval{\psi\qty(z_1)\psi\qty(z_2)\mathcal O}&=\expval{2\pi\delta^{\qty(2)}\qty(z_2-z_1)\mathcal O}\\
    \partial_{{\bar z}_1}\psi\qty(z_1)\psi\qty(z_2)&=2\pi\delta^{\qty(2)}\qty(z_2-z_1),\ \ \ \abs{z_1}\geq\abs{z_2}
\end{align*}

So that we can interpret this as a operator equality. The condition $\abs{z_1}\geq \abs{z_2}$ is due to the implicit radial ordering in the expectation 
value, and also notice that in the path integral formulation $\psi=\psi\qty(z_1,{\bar z}_1)$, but in the operator formalism $\psi=\psi\qty(z_1)$, as every operator is 
always `\textit{on-shell}', while the path integral integrands aren't. Now we integrate this two point function, pick any compact closed region $R$ in the complex plane which 
contains $z_2$ and not any other of the points of the insertions $\mathcal O$, with also the boundary being a continuous curve, for all $m\in\mathbb N$ the following is true,

\begin{align*}
    \partial_{{\bar z}_1}\expval{\psi\qty(z_1)\psi\qty(z_2)\mathcal O}&=2\pi\delta^{\qty(2)}\qty(z_2-z_1)\expval{\mathcal O}\\
    \qty(z_1-z_2)^m\partial_{{\bar z}_1}\expval{\psi\qty(z_1)\psi\qty(z_2)\mathcal O}&=2\pi\qty(z_1-z_2)^m\delta^{\qty(2)}\qty(z_2-z_1)\expval{\mathcal O},\ \ \ m\in\mathbb N\\
    \int\limits_R\dd[2]{z_1}\qty(z_1-z_2)^m\partial_{{\bar z}_1}\expval{\psi\qty(z_1)\psi\qty(z_2)\mathcal O}&=2\pi\int\limits_R\dd[2]{z_1}\qty(z_1-z_2)^m\delta^{\qty(2)}\qty(z_2-z_1)\expval{\mathcal O},\ \ \ m\in\mathbb N\\
    \int\limits_R\dd[2]{z_1}\partial_{{\bar z}_1}\qty{\qty(z_1-z_2)^m\expval{\psi\qty(z_1)\psi\qty(z_2)\mathcal O}}&=2\pi\qty(z_1-z_2)^m\eval_{z_1=z_2}\expval{\mathcal O},\ \ \ m\in\mathbb N
\end{align*}

In the last line we used the fact `\textit{trivial}' fact that $\partial_{{\bar z}_1}\qty(z_1-z_2)^m=0$, which is valid for $m\in\mathbb N$, but isn't for 
negative non-integer values.

\begin{align*}
    \int\limits_R\dd[2]{z_1}\partial_{{\bar z}_1}\qty{\qty(z_1-z_2)^m\expval{\psi\qty(z_1)\psi\qty(z_2)\mathcal O}}&=2\pi\qty(z_1-z_2)^m\eval_{z_1=z_2}\expval{\mathcal O},\ \ \ m\in\mathbb N\\
    -\im\int\limits_{\partial R}\dd{z_1}\qty(z_1-z_2)^m\expval{\psi\qty(z_1)\psi\qty(z_2)\mathcal O}&=2\pi\delta_{m,0}\expval{\mathcal O},\ \ \ m\in\mathbb N
\end{align*}

This is simply the complex version of the divergence theorem,

\begin{align*}
    -\im\int\limits_{\partial R}\dd{z_1}\qty(z_1-z_2)^m\expval{\psi\qty(z_1)\psi\qty(z_2)\mathcal O}&=2\pi\delta_{m,0}\expval{\mathcal O},\ \ \ m\in\mathbb N\\
    \int\limits_{\partial R}\frac{\dd{z_1}}{2\pi\im}\qty(z_1-z_2)^m\expval{\psi\qty(z_1)\psi\qty(z_2)\mathcal O}&=\delta_{m,0}\expval{\mathcal O},\ \ \ m\in\mathbb N
\end{align*}

Well, the left-hand side of this last equation picks up the pole of $m$-th order of the expression $\expval{\psi\qty(z_1)\psi\qty(z_2)\mathcal O}$ under $z_1\rightarrow z_2$, but, 
the right-hand side is only non-zero for $m=0$, this is telling us that $\expval{\psi\qty(z_1)\psi\qty(z_2)\mathcal O}$ doesn't have any pole besides the first order one with residue $\expval{\mathcal O}$, 
so we can write,

\begin{align*}
    \expval{\psi\qty(z_1)\psi\qty(z_2)\mathcal O}=\frac{1}{z_1-z_2}\expval{\mathcal O}+\textnormal{regular}
\end{align*}

Or, in the operator formalism,

\begin{align*}
    \psi\qty(z_1)\psi\qty(z_2)&=\frac{1}{z_1-z_2}+\textnormal{regular},\ \ \ \abs{z_1}\geq \abs{z_2}\\
    \psi\qty(z_1)\psi\qty(z_2)&=\frac{1}{z_1-z_2}+\cnord{\psi\qty(z_1)\psi\qty(z_2)},\ \ \ \abs{z_1}\geq \abs{z_2}
\end{align*}

Which is the OPE.

\probitem{}

The energy momentum tensor has two components, a meromorphic one $T\qty(z)$, and an anti-meromorphic one $\bar T\qty(\bar z)$, 
as we have at our disposal only a weight $\qty(\frac12,0)$ field, it's impossible to construct a anti-meromorphic energy momentum tensor, 
but, for the meromorphic one, let us see what kind of combinations have the right weights. First, we remember that our energy momentum tensor 
must be normal ordered, so, all the possible normal ordered weight $\qty(2,0)$ combinations of $\psi$ and derivatives are,

\begin{align*}
    \cnord{\psi\partial\psi}\qty(z),\ \cnord{(\partial\psi)\psi}\qty(z), \ \cnord{\partial\qty(\psi\psi)}\qty(z),\ \cnord{\psi\psi\psi\psi}\qty(z)
\end{align*} 

The third and fourth are zero due to the fermionic statistic, and the first and second are linear dependent also because the 
fermionic statistic. Hence, up to a unknown constant $\alpha$, the meromorphic energy momentum tensor is,

\begin{align*}
    T\qty(z)=\alpha\cnord{\psi\partial\psi}\qty(z)
\end{align*}

Let's use this expression to compute the following,

\begin{align*}
    \cnord{T\qty(z_1) T\qty(z_2)}&=T\qty(z_1)T\qty(z_2)+\alpha^2\textnormal{ contractions}\qty{\cnord{\qty(\psi\partial\psi)\qty(z_1)\qty(\psi\partial\psi)\qty(z_2)}},\ \ \ \abs{z_1}\geq \abs{z_2}
\end{align*}

Where the contraction part means exchanging pairs of $\psi$ with different $z$ by $-\qty(z_1-z_2)^{-1}$, as cause of the fermion 
statistic, we have to pay attention to the signs, also, we'll omit for now the $\abs{z_1}\geq\abs{z_2}$, but it will still be imposed, 

\begin{align*}
    \cnord{T\qty(z_1) T\qty(z_2)}&={ T\qty(z_1)T\qty(z_2)}+\alpha^2\cnord{\wick[offset=1.2em]{(\c{\psi}\partial\psi)\qty(z_1)(\c{\psi}\partial\psi)\qty(z_2)}}+\alpha^2\cnord{\wick[offset=1.2em]{(\c{\psi}\partial\psi)\qty(z_1)({\psi}\partial\c{\psi})\qty(z_2)}}\\
    &\quad\quad\quad+\alpha^2\cnord{\wick[offset=1.2em]{({\psi}\partial\c{\psi})\qty(z_1)(\c{\psi}\partial{\psi})\qty(z_2)}}+\alpha^2\cnord{\wick[offset=1.2em]{({\psi}\partial\c{\psi})\qty(z_1)({\psi}\partial\c{\psi})\qty(z_2)}}\\
    &\quad\quad\quad+\alpha^2\cnord{\wick[offset=1.2em]{(\c1{\psi}\partial\c2{\psi})\qty(z_1)(\c1{\psi}\partial\c2{\psi})\qty(z_2)}}+\alpha^2\cnord{\wick[offset=1.2em]{(\c1{\psi}\partial\c2{\psi})\qty(z_1)(\c2{\psi}\partial\c1{\psi})\qty(z_2)}}\\
    \cnord{T\qty(z_1) T\qty(z_2)}&={ T\qty(z_1)T\qty(z_2)}+\frac{1}{z_1-z_2}\alpha^2\cnord{\partial\psi\qty(z_1)\partial\psi\qty(z_2)}-\partial_{z_2}\qty(\frac{1}{z_1-z_2})\alpha^2\cnord{\partial\psi\qty(z_1)\psi\qty(z_2)}\\
    &\quad\quad\quad-\alpha^2\partial_{z_1}\qty(\frac{1}{z_1-z_2})\cnord{{\psi}\qty(z_1)\partial{\psi}\qty(z_2)}+\alpha^2\partial_{z_1}\partial_{z_2}\qty(\frac{1}{z_1-z_2})\cnord{{\psi}\qty(z_1){\psi}\qty(z_2)}\\
    &\quad\quad\quad+\alpha^2\frac{1}{z_1-z_2}\partial_{z_1}\partial_{z_2}\qty(\frac{1}{z_1-z_2})-\alpha^2\partial_{z_2}\qty(\frac{1}{z_1-z_2})\partial_{z_1}\qty(\frac{1}{z_1-z_2})\\
    \cnord{T\qty(z_1) T\qty(z_2)}&={ T\qty(z_1)T\qty(z_2)}+\frac{\alpha^2}{z_1-z_2}\cnord{\partial\psi\qty(z_1)\partial\psi\qty(z_2)}-\frac{\alpha^2}{\qty(z_1-z_2)^2}\cnord{\partial\psi\qty(z_1)\psi\qty(z_2)}\\
    &\quad\quad\quad+\frac{\alpha^2}{\qty(z_1-z_2)^2}\cnord{{\psi}\qty(z_1)\partial{\psi}\qty(z_2)}+\partial_{z_1}\qty(\frac{\alpha^2}{\qty(z_1-z_2)^2})\cnord{{\psi}\qty(z_1){\psi}\qty(z_2)}\\
    &\quad\quad\quad+\frac{\alpha^2}{z_1-z_2}\partial_{z_1}\qty(\frac{1}{\qty(z_1-z_2)^2})-\frac{\alpha^2}{\qty(z_1-z_2)^2}\partial_{z_1}\qty(\frac{1}{z_1-z_2})\\
    \cnord{T\qty(z_1) T\qty(z_2)}&={ T\qty(z_1)T\qty(z_2)}+\frac{\alpha^2}{z_1-z_2}\cnord{\partial\psi\qty(z_1)\partial\psi\qty(z_2)}-\frac{\alpha^2}{\qty(z_1-z_2)^2}\cnord{\partial\psi\qty(z_1)\psi\qty(z_2)}\\
    &\quad\quad\quad+\frac{\alpha^2}{\qty(z_1-z_2)^2}\cnord{{\psi}\qty(z_1)\partial{\psi}\qty(z_2)}-2\frac{\alpha^2}{\qty(z_1-z_2)^3}\cnord{{\psi}\qty(z_1){\psi}\qty(z_2)}\\
    &\quad\quad\quad-2\frac{\alpha^2}{z_1-z_2}\frac{1}{\qty(z_1-z_2)^3}+\frac{\alpha^2}{\qty(z_1-z_2)^2}\frac{1}{\qty(z_1-z_2)^2}\\
    \cnord{T\qty(z_1) T\qty(z_2)}&={ T\qty(z_1)T\qty(z_2)}+\frac{\alpha^2}{z_1-z_2}\cnord{\partial\psi\qty(z_1)\partial\psi\qty(z_2)}-\frac{\alpha^2}{\qty(z_1-z_2)^2}\cnord{\partial\psi\qty(z_1)\psi\qty(z_2)}\\
    &\quad\quad\quad+\frac{\alpha^2}{\qty(z_1-z_2)^2}\cnord{{\psi}\qty(z_1)\partial{\psi}\qty(z_2)}-2\frac{\alpha^2}{\qty(z_1-z_2)^3}\cnord{{\psi}\qty(z_1){\psi}\qty(z_2)}\\
    &\quad\quad\quad-\frac{\alpha^2}{\qty(z_1-z_2)^4}
\end{align*}

To proceed further we have to Taylor expand both $\psi\qty(z_1),\partial\psi\qty(z_1)$, of course this will generate regular terms in our expansion, as for example, the second term in the right-hand side,

\begin{align*}
    \frac{1}{z_1-z_2}\cnord{\partial\psi\qty(z_1)\partial\psi\qty(z_2)}&=\frac{1}{z_1-z_2}\cnord{\partial\psi\qty(z_2)\partial\psi\qty(z_2)}+\sum\limits_{n=1}^\infty\frac{1}{z_1-z_2}\frac{1}{n!}\qty(z_1-z_2)^n\cnord{\partial^n\psi\qty(z_2)\partial\psi\qty(z_2)}\\
    \frac{1}{z_1-z_2}\cnord{\partial\psi\qty(z_1)\partial\psi\qty(z_2)}&=\frac{1}{z_1-z_2}\cnord{\partial\psi\partial\psi}\qty(z_2)+\sum\limits_{n=1}^\infty\frac{\qty(z_1-z_2)^{n-1}}{n!}\cnord{\partial^n\psi\partial\psi}\qty(z_2)
\end{align*}

It's clear that the sum in the right-hand side is of only regular terms, so that,

\begin{align*}
    \frac{1}{z_1-z_2}\cnord{\partial\psi\qty(z_1)\partial\psi\qty(z_2)}&=\frac{1}{z_1-z_2}\cnord{\partial\psi\partial\psi}\qty(z_2)+\textnormal{regular}
\end{align*}

But $\cnord{\partial\psi\partial\psi}$ is zero by statistics, hence, 

\begin{align*}
    \frac{1}{z_1-z_2}\cnord{\partial\psi\qty(z_1)\partial\psi\qty(z_2)}&=\textnormal{regular}
\end{align*}

We do this procedure for all the terms in the expansion,

\begin{align*}
    \cnord{T\qty(z_1) T\qty(z_2)}&={ T\qty(z_1)T\qty(z_2)}-\frac{\alpha^2}{\qty(z_1-z_2)^2}\cnord{\partial\psi\psi}\qty(z_2)-\frac{\alpha^2}{\qty(z_1-z_2)}\cnord{\partial^2\psi\psi}\qty(z_2)\\
    &\quad\quad\quad+\frac{\alpha^2}{\qty(z_1-z_2)^2}\cnord{{\psi}\partial{\psi}}\qty(z_2)+\frac{\alpha^2}{\qty(z_1-z_2)}\cnord{\partial{\psi}\partial{\psi}}\qty(z_2)\\
    &\quad\quad\quad-2\frac{\alpha^2}{\qty(z_1-z_2)^3}\cnord{{\psi}{\psi}}\qty(z_2)-2\frac{\alpha^2}{\qty(z_1-z_2)^2}\cnord{\partial{\psi}{\psi}}\qty(z_2)-\frac{\alpha^2}{\qty(z_1-z_2)}\cnord{\partial^2{\psi}{\psi}}\qty(z_2)\\
    &\quad\quad\quad-\frac{\alpha^2}{\qty(z_1-z_2)^4}+\textnormal{regular}
\end{align*}

Removing the terms which are zero by statistics and grouping the others,

\begin{align*}
    \cnord{T\qty(z_1) T\qty(z_2)}&={ T\qty(z_1)T\qty(z_2)}+\frac{4\alpha^2}{\qty(z_1-z_2)^2}\cnord{\psi\partial\psi}\qty(z_2)+\frac{2\alpha^2}{\qty(z_1-z_2)}\cnord{\psi\partial^2\psi}\qty(z_2)\\
    &\quad\quad\quad-\frac{\alpha^2}{\qty(z_1-z_2)^4}+\textnormal{regular}
\end{align*}

As $\cnord{\partial\psi\partial\psi}\equiv0$ we can add it as we please,

\begin{align*}
    \cnord{T\qty(z_1) T\qty(z_2)}&={ T\qty(z_1)T\qty(z_2)}+\frac{4\alpha^2}{\qty(z_1-z_2)^2}\cnord{\psi\partial\psi}\qty(z_2)+\frac{2\alpha^2}{\qty(z_1-z_2)}\cnord{\psi\partial^2\psi+\partial\psi\partial\psi}\qty(z_2)\\
    &\quad\quad\quad-\frac{\alpha^2}{\qty(z_1-z_2)^4}+\textnormal{regular}\\
    \cnord{T\qty(z_1) T\qty(z_2)}&={ T\qty(z_1)T\qty(z_2)}+\frac{4\alpha^2}{\qty(z_1-z_2)^2}\cnord{\psi\partial\psi}\qty(z_2)+\frac{2\alpha^2}{\qty(z_1-z_2)}\partial\cnord{\psi\partial\psi}\qty(z_2)\\
    &\quad\quad\quad-\frac{\alpha^2}{\qty(z_1-z_2)^4}+\textnormal{regular}\\
    \cnord{T\qty(z_1) T\qty(z_2)}&={ T\qty(z_1)T\qty(z_2)}+\frac{4\alpha}{\qty(z_1-z_2)^2}T\qty(z_2)+\frac{2\alpha}{\qty(z_1-z_2)}\partial T\qty(z_2)\\
    &\quad\quad\quad-\frac{\alpha^2}{\qty(z_1-z_2)^4}+\textnormal{regular}
\end{align*}

Well, $\cnord{T\qty(z_1)T\qty(z_2)}$ itself is regular, then,

\begin{align*}
    { T\qty(z_1)T\qty(z_2)}&=-\frac{4\alpha}{\qty(z_1-z_2)^2}T\qty(z_2)-\frac{2\alpha}{\qty(z_1-z_2)}\partial T\qty(z_2)+\frac{\alpha^2}{\qty(z_1-z_2)^4}+\textnormal{regular}
\end{align*}

But we know the general form of the $TT$ OPE,

\begin{align*}
    { T\qty(z_1)T\qty(z_2)}&=\frac{2}{\qty(z_1-z_2)^2}T\qty(z_2)+\frac{1}{\qty(z_1-z_2)}\partial T\qty(z_2)+\frac{c}{2\qty(z_1-z_2)^4}+\textnormal{regular}
\end{align*}

From where is easy to read $\alpha=-\frac12$, and also, $c=2\alpha^2=2\frac14=\frac12$, that is, the meromorphic component of the 
energy momentum tensor, and the central charge are,

\begin{align*}
    T\qty(z)&=-\frac12\cnord{\psi\partial\psi}\qty(z),\ \ \ c=\frac12
\end{align*}

\probitem{}
\probitem{}
\probitem{}
\probitem{}

\newpage

\problem{}
\probitem{}

First, one remark, we're going to derive everything in this problem for the \textbf{open} string with 
Neumann Boundary Conditions at both ends. The full trip to the classical Light-Cone gauge of the open 
string with NN boundary conditions is done in Appendix \ref{app-openclass}, afterwards the full 
quantization of the same type of string in the same gauge is done in Appendix \ref{app-openqnt}, we're 
just continue here citing the needed results. These are, 

\begin{align*}
    X^I&=x_0^I+\sqrt{2\alpha'} \alpha_0^I\tau+\im\sqrt{2\alpha'}\sum\limits_{n\in\mathbb Z^\ast}\frac{\alpha_n^I}{n}\exp\qty(-\im n\tau)\cos\qty(n\sigma),\ \ \ \alpha_0^\mu=\sqrt{2\alpha'}p^\mu\ \ \ \ref{solopen1}\\
    X^-&=x_0^-+\sqrt{2\alpha'} \alpha_0^-\tau+\im\sqrt{2\alpha'}\sum\limits_{n\in\mathbb Z^\ast}\frac{\alpha_n^-}{n}\exp\qty(-\im n\tau)\cos\qty(n\sigma),\ \ \ \alpha^-_n=\frac{1}{\sqrt{2\alpha'}p^+}L^\perp_n,\ \ \ \ref{solopen2}\\
    X^+&=2\alpha'p^+\tau,\ \ \ \ref{solopen3}\\
    \comm{x_0^-}{p^+}&=-\im,\ \ \ \ref{qntconditions}\\
    \comm{\alpha_{m}^I}{\alpha_{n}^J}&=mg^{IJ}\delta_{m+n,0},\ \ \ \ref{alphacomm}\\
    \comm{x_0^I}{\alpha_{n}^J}&=\delta_{n,0}\sqrt{2\alpha'}\im g^{IJ},\ \ \ \ref{alphaxcomm}\\
    \comm{L_m^\perp}{\alpha^J_n}&=-n\alpha_{n+m}^J,\ \ \ \ref{virasoroalpha}\\
    \comm{L_m^\perp}{x_0^J}&=-\im\sqrt{2\alpha'}\alpha_m^J,\ \ \ \ref{virasorox}\\
    \comm{L_m^\perp}{L_n^\perp}&=\qty(m-n)L_{m+n}^\perp+\frac{D-2}{12}\qty(m^3-m)\delta_{m+n,0},\ \ \ \ref{virasorovirasoro}
\end{align*}

This completes the set of all needed commutation relations. We can now discuss the Lorentz generators in the Light-cone gauge. 
We have already computed them in \ref{m}, but of course we have two remarks, neither they are quantum, nor are in the light-cone gauge, 
about the later, the actual quantum light-cone gauge Lorentz generators \textbf{should} satisfy the same algebra as \ref{lorentzalgebra} in the 
light-cone coordinates. The failure to met this requirement is related to an anomaly in this global symmetry of the Quantum Poincare Action. 
And about the former, we'll change the definition accordingly to ensure the quantum generators do satisfy being Hermitian. With that being said, 
let's evaluate the classical version of \ref{m} in the light-cone gauge,

\begin{align*}
    M^{\mu\nu}&\stackrel{?}{=}\frac{1}{2\pi\alpha'}\int\limits_0^\pi\dd{\sigma}\qty(X^\mu{\dot X}^\nu-X^\nu{\dot X}^\mu)\\
    &\stackrel{?}{=}\frac{1}{2\pi\alpha'}\int\limits_0^\pi\dd{\sigma}\qty(x_0^\mu+2\alpha'p^\mu\tau+\im\sqrt{2\alpha'}\sum\limits_{n\in\mathbb Z^\ast}\frac{\alpha^\mu_n}{n}\exp\qty(-\im n\tau)\cos\qty(n\sigma))\times\\
    &\quad\quad\quad\times\sqrt{2\alpha'}\sum\limits_{m\in\mathbb Z}\alpha^\nu_m\exp\qty(-\im m\tau)\cos\qty(m\sigma)-\qty(\mu\leftrightarrow\nu)\\
    &\stackrel{?}{=}\frac{1}{2\pi\alpha'}\left[\sqrt{2\alpha'}\sum\limits_{m\in\mathbb Z}x_0^\mu\alpha^\nu_m\exp\qty(-\im m\tau)\pi\delta_{m,0}\right.\\
    &\quad\quad\quad+2\alpha'\tau\sqrt{2\alpha'}\sum\limits_{m\in\mathbb Z}p^\mu\alpha^\nu_m\exp\qty(-\im m\tau)\pi\delta_{m,0}\\
    &\quad\quad\quad\left.+\im 2\alpha'\sum\limits_{n\in\mathbb Z^\ast}\sum\limits_{m\in\mathbb Z}\frac{\alpha^\mu_n}{n}\exp\qty(-\im n\tau)\alpha^\nu_m\exp\qty(-\im m\tau)\frac\pi2\qty(\delta_{m,n}+\delta_{m,-n})\right]-\qty(\mu\leftrightarrow\nu)\\
    M^{\mu\nu}&\stackrel{?}{=}\left[\frac{1}{\sqrt{2\alpha'}}x_0^\mu\alpha^\nu_0\right.+\tau\sqrt{2\alpha'}p^\mu\alpha^\nu_0+\frac\im2 \sum\limits_{n\in\mathbb Z^\ast}\frac{1}{n}\alpha^\mu_n\alpha^\nu_n\exp\qty(-\im \qty(n+m)\tau)\left.+\frac\im2\sum\limits_{n\in\mathbb Z^\ast}\frac{1}{n}\alpha^\mu_n\alpha^\nu_{-n}\right]-\qty(\mu\leftrightarrow\nu)
\end{align*}

Employing the before mentioned equality $\alpha^\mu_0=\sqrt{2\alpha'}p^\mu$,

\begin{align*}
    M^{\mu\nu}&\stackrel{?}{=}\left[x_0^\mu p^\nu-x_0^\nu p^\mu+\tau\alpha_0^\mu \alpha_0^\nu-\tau\alpha_0^\nu \alpha_0^\mu+\frac\im2\sum\limits_{n\in\mathbb Z^\ast}\frac{1}{n}\qty(\alpha^\mu_n\alpha^\nu_{-n}-\alpha^\nu_n\alpha^\mu_{-n})\right.\\
    &\quad\quad\quad\left.+\frac\im2 \sum\limits_{n\in\mathbb Z^\ast}\frac{1}{n}\qty(\alpha^\mu_n\alpha^\nu_n-\alpha^\nu_n\alpha^\mu_n)\exp\qty(-\im 2n\tau)\right]\\
    M^{\mu\nu}&\stackrel{?}{=}\left[x_0^\mu p^\nu-x_0^\nu p^\mu+\frac\im2\sum\limits_{n\in\mathbb N^\ast}\frac{1}{n}\qty(\alpha^\mu_n\alpha^\nu_{-n}-\alpha^\nu_n\alpha^\mu_{-n}-\alpha^\mu_{-n}\alpha^\nu_{n}+\alpha^\nu_{-n}\alpha^\mu_{n})\right]\\
    M^{\mu\nu}&\stackrel{?}{=}\left[x_0^\mu p^\nu-x_0^\nu p^\mu-\im\sum\limits_{n\in\mathbb N^\ast}\frac{1}{n}\qty(\alpha^\mu_{-n}\alpha^\nu_{n}-\alpha^\nu_{-n}\alpha^\mu_{n})\right]
\end{align*}

This is the classical version of the generators in the light-cone gauge. Does it is a good Quantum Lorentz Generator Operator? 
For a positive answer, it has to be both Hermitian and normal-ordered, let's consider one by one,

\begin{align*}
    M^{IJ}&\stackrel{?}{=}\left[x_0^I p^J-x_0^J p^I+\im\sum\limits_{n\in\mathbb N^\ast}\frac{1}{n}\qty(\alpha^I_n\alpha^J_{-n}-\alpha^J_n\alpha^I_{-n})\right]\\
    \qty(M^{IJ})^\dagger-M^{IJ}&\stackrel{?}{=}\left[p^Jx_0^I -p^Ix_0^J -x_0^I p^J+x_0^J p^I-\im\sum\limits_{n\in\mathbb N^\ast}\frac{1}{n}\qty(\alpha^J_{n}\alpha^I_{-n}-\alpha^I_{n}\alpha^J_{-n}+\alpha^I_n\alpha^J_{-n}-\alpha^J_n\alpha^I_{-n})\right]\\
    \qty(M^{IJ})^\dagger-M^{IJ}&\stackrel{?}{=}\left[\comm{p^J}{x_0^I} -\comm{p^I}{x_0^J}\right]\\
    \qty(M^{IJ})^\dagger-M^{IJ}&\stackrel{?}{=}\left[-\im g^{IJ} +\im  g^{IJ}\right]=0
\end{align*}

Yes! And about normal-ordered? Yes! Then our quantum operator related to the Lorentz Generators is,

\begin{align*}
    M^{IJ}&=\left[x_0^I p^J-x_0^J p^I-\im\sum\limits_{n\in\mathbb N^\ast}\frac{1}{n}\qty(\alpha^I_{-n}\alpha^J_n-\alpha^J_{-n}\alpha^I_n)\right]
\end{align*}

Now, already seen that it's normal ordered and looking at Hermiticity,

\begin{align*}
    M^{I+}&\stackrel{?}{=}\left[x_0^I p^+-x_0^+ p^I-\im\sum\limits_{n\in\mathbb N^\ast}\frac{1}{n}\qty(\alpha^I_{-n}\alpha^+_{n}-\alpha^+_{-n}\alpha^I_{n})\right]\\
    M^{I+}&\stackrel{?}{=}x_0^I p^+
\end{align*}

Which is of course Hermitian, once $\comm{x_0^I}{p^+=0}$, hence,

\begin{align*}
    M^{I+}&=x_0^I p^+
\end{align*}

By the same reasoning, that is, $x_0^+=0=\alpha_n^+,\ n\neq 0$,

\begin{align*}
    M^{-+}&\stackrel{?}{=}x_0^- p^+
\end{align*}

Which fails to be Hermitian, due the canonical commutation relations. One way to avoid this is to symmetrize it,

\begin{align*}
    M^{-+}&=\frac{1}{2}\qty(x_0^-p^++p^+x_0^-)
\end{align*}

Which now is both normal ordered and hermitian. Now,

\begin{align*}
    M^{++}&\stackrel{?}{=}x_0^+p^+=0
\end{align*}

Which is expected by the anti-symmetry of the generators, we have another which is zero by the anti-symmetry,

\begin{align*}
    M^{--}&\stackrel{?}{=}\qty[x_0^-p^--x_0^-p^--\im\sum\limits_{n\in\mathbb N^\ast}\frac1n\qty(\alpha_{-n}^-\alpha^-_{n}-\alpha^-_{-n}\alpha^-_n)]=0
\end{align*}

At last, we have the most important generator,

\begin{align*}
    M^{-I}&\stackrel{?}{=}\left[x_0^- p^I-x_0^I p^--\im\sum\limits_{n\in\mathbb N^\ast}\frac{1}{n}\qty(\alpha^-_{-n}\alpha^I_{n}-\alpha^I_{-n}\alpha^-_{n})\right]
\end{align*}

Is it Hermitian? No. 

\begin{align*}
    \qty(M^{-I})^\dagger-M^{-I}&\stackrel{?}{=}\left[p^Ix_0^- -p^-x_0^I -x_0^- p^I+x_0^I p^-+\im\sum\limits_{n\in\mathbb N^\ast}\frac{1}{n}\qty(\alpha^I_{-n}\alpha^-_{n}-\alpha^-_{-n}\alpha^I_{n}+\alpha^-_{-n}\alpha^I_{n}-\alpha^I_{-n}\alpha^-_{n})\right]\\
    \qty(M^{-I})^\dagger-M^{-I}&\stackrel{?}{=}\left[\comm{p^I}{x_0^-} -\comm{p^-}{x_0^I}\right]\\
    \qty(M^{-I})^\dagger-M^{-I}&\stackrel{?}{=}-\comm{p^-}{x_0^I}\\
    \qty(M^{-I})^\dagger-M^{-I}&\stackrel{?}{=}-\frac{1}{\sqrt{2\alpha'}}\comm{\alpha_0^-}{x_0^I}\\
    \qty(M^{-I})^\dagger-M^{-I}&\stackrel{?}{=}-\frac{1}{2\alpha'p^+}\comm{L^\perp_0+a}{x_0^I}=\frac{\im}{\sqrt{2\alpha'}p^+}\alpha^I_0
\end{align*}

Again, if we symmetrize the $x_0^Ip^-$ term this issue is resolved,

\begin{align*}
    M^{-I}&\stackrel{?}{=}x_0^- p^I-\frac12\qty(x_0^I p^-+p^-x_0^I)-\im\sum\limits_{n\in\mathbb N^\ast}\frac{1}{n}\qty(\alpha^-_{-n}\alpha^I_{n}-\alpha^I_{-n}\alpha^-_{n})
\end{align*}

Is this expression normal-ordered? Yes! Due to the $\alpha_n^-$ being proportional to the Virasoro modes, which are already normal-ordered. Hence, the expression for this 
Lorentz Generator is, using Virasoro modes,

\begin{align*}
    M^{-I}&=x_0^- p^I-\frac{1}{4\alpha'p^+}\qty(x_0^I\qty(L_0^\perp+a)+\qty(L_0^\perp+a)x_0^I)-\frac{\im}{\sqrt{2\alpha'}p^+}\sum\limits_{n\in\mathbb N^\ast}\frac{1}{n}\qty(L^\perp_{-n}\alpha^I_{n}-\alpha^I_{-n}L^\perp_n)
\end{align*}

Where we made use of our lack of knowledge about the ordering prescription by introducing $L^\perp_0+a$ whenever $\alpha_0^-$ is mentioned.

\probitem{}

This is the real deal, now is the time of truth. We'll \textbf{ensure} the Lorentz invariance of this theory, you may ask how? 
By making sure the Lorentz algebra is satisfied by the Light-cone Lorentz generators. We're not computing all of the commutators in 
the Light-cone gauge, but only $\comm{M^{-I}}{M^{-J}}$, as this is the only one with unknown constants, $a$. The calculus is long and 
cumbersome, hence, we'll split into various parts. Let's compute term by term. We define the auxiliary variables as,

\begin{align*}
    A^I&=x_0^- p^I\\
    B^I&=-\frac{1}{4\alpha'p^+}\qty(x_0^I\qty(L_0^\perp+a)+\qty(L_0^\perp+a)x_0^I)\\
    C^I&=\frac{\im}{\sqrt{2\alpha'}p^+}\sum\limits_{n\in\mathbb N^\ast}\frac{1}{n}\qty(L^\perp_{-n}\alpha^I_{n}-\alpha^I_{-n}L^\perp_n)
\end{align*}

It's clear then that the commutator is,

\begin{align*}
    \comm{M^{-I}}{M^{-J}}&=\comm{A^I}{A^J}+\comm{A^I}{B^J}+\comm{B^I}{A^J}+\comm{B^I}{B^J}\\
    &\quad\quad\quad+\comm{A^I}{C^J}+\comm{C^I}{A^J}+\comm{B^I}{C^J}+\comm{C^I}{B^J}+\comm{C^I}{C^J}
\end{align*}

Which can be rewritten as,

\begin{align*}
    \comm{M^{-I}}{M^{-J}}&=\comm{A^I}{A^J}+\comm{B^I}{B^J}+\comm{C^I}{C^J}\\
    &\quad\quad\quad+\qty{\comm{A^I}{B^J}+\comm{A^I}{C^J}+\comm{B^I}{C^J}-\qty(I\leftrightarrow J)}
\end{align*}

That is, for the mixed commutator, we just need to compute the anti-symmetrical part, a fact that we'll exploit,
$$$$
Now, $\comm{A^I}{A^J}$,

\begin{align*}
    \comm{x^-_0p^I}{x_0^-p^J}&=x^-_0\comm{p^I}{x_0^-p^J}+\comm{x^-_0}{x_0^-p^J}p^I\\
    \comm{x^-_0p^I}{x_0^-p^J}&=x^-_0x_0^-\comm{p^I}{p^J}+x_0^-\comm{x^-_0}{p^J}p^I=0
\end{align*}

That is,

\begin{align*}
    \comm{A^I}{A^J}&=0\numberthis
\end{align*}
$$$$
Now, $\comm{A^I}{B^J}$,

\begin{align*}
    &-\frac{1}{4\alpha'}\comm{x_0^-p^I}{\frac{1}{p^+}\qty(x_0^J\qty(L_0^\perp+a)+\qty(L_0^\perp+a)x_0^J)}+\frac{1}{4\alpha'}\comm{x_0^-p^J}{\frac{1}{p^+}\qty(x_0^I\qty(L_0^\perp+a)+\qty(L_0^\perp+a)x_0^I)}\\
    &=-\frac{1}{4\alpha'}x_0^-\frac{1}{p^+}\qty(\comm{p^I}{\qty(x_0^J\qty(L_0^\perp+a)+\qty(L_0^\perp+a)x_0^J)}-\comm{p^J}{\qty(x_0^I\qty(L_0^\perp+a)+\qty(L_0^\perp+a)x_0^I)})\\
    &\quad\quad\quad-\frac{1}{4\alpha'}\comm{x_0^-}{\frac{1}{p^+}}\qty(\qty(x_0^J\qty(L_0^\perp+a)+\qty(L_0^\perp+a)x_0^J)p^I-\qty(x_0^I\qty(L_0^\perp+a)+\qty(L_0^\perp+a)x_0^I)p^J)\\
    &=-\frac{\im}{4\alpha'{p^+}^2}\qty(\qty(x_0^J\qty(L_0^\perp+a)+\qty(L_0^\perp+a)x_0^J)p^I-\qty(x_0^I\qty(L_0^\perp+a)+\qty(L_0^\perp+a)x_0^I)p^J)\\
    &=-\frac{\im}{4\alpha'{p^+}^2}\qty(\qty(2\qty(L_0^\perp+a)x_0^J+\im 2\alpha' p^J)p^I-\qty(2\qty(L_0^\perp+a)x_0^I+\im 2\alpha'p^I)p^J)\\
    &=-\frac{\im}{2\alpha'{p^+}^2}\qty(L_0^\perp+a)\qty(x_0^Jp^I-x_0^Ip^J)\\
\end{align*}

That is,

\begin{align*}
    \comm{A^I}{B^J}-\comm{A^J}{B^I}&=-\frac{\im}{2\alpha'{p^+}^2}\qty(L_0^\perp+a)\qty(x_0^Jp^I-x_0^Ip^J)\numberthis
\end{align*}
$$$$
Now, $\comm{B^I}{B^J}$,

\begin{align*}
    &\frac{1}{16{\alpha'}^2{p^+}^2}\comm{x_0^I\qty(L_0^\perp+a)+\qty(L_0^\perp+a)x_0^I}{x_0^J\qty(L_0^\perp+a)+\qty(L_0^\perp+a)x_0^J}\\
    &=\frac{1}{16{\alpha'}^2{p^+}^2}\comm{2\qty(L_0^\perp+a)x_0^I+\im2\alpha' p^I}{2\qty(L_0^\perp+a)x_0^J+\im2\alpha'p^J}\\
    &=\frac{1}{4{\alpha'}^2{p^+}^2}\qty{\comm{\qty(L_0^\perp+a)x_0^I}{\qty(L_0^\perp+a)x_0^J}+\im\alpha'\qty(\comm{p^I}{\qty(L_0^\perp+a)x_0^J}-\comm{p^J}{\qty(L_0^\perp+a)x_0^I})}\\
    &=\frac{1}{4{\alpha'}^2{p^+}^2}\qty{\qty(L_0^\perp+a)\comm{x_0^I}{L_0^\perp}x_0^J+\qty(L_0^\perp+a)\comm{L_0^\perp}{x_0^J}x_0^I+\im\alpha'\qty(\comm{p^I}{L_0^\perp}x_0^J-\comm{p^J}{L_0^\perp}x_0^I)}\\
    &=\frac{1}{4{\alpha'}^2{p^+}^2}\qty{\qty(L_0^\perp+a)\im 2\alpha' p^Ix_0^J+\qty(L_0^\perp+a)\qty(-\im 2\alpha' p_0^J)x_0^I}\\
    &=\frac{\im}{2{\alpha'}{p^+}^2}\qty(L_0^\perp+a)\qty(p^Ix_0^J-p_0^Jx_0^I)\\
    &=\frac{\im}{2{\alpha'}{p^+}^2}\qty(L_0^\perp+a)\qty(x_0^Jp^I+\comm{p^I}{x_0^J}-x_0^Ip_0^J-\comm{p_0^J}{x_0^I})\\
    &=\frac{\im}{2{\alpha'}{p^+}^2}\qty(L_0^\perp+a)\qty(x_0^Jp^I-x_0^Ip_0^J)
\end{align*}

That is,

\begin{align*}
    \comm{B^I}{B^J}&=\frac{\im}{2\alpha'{p^+}^2}\qty(L_0^\perp+a)\qty(x_0^Jp^I-x_0^Ip^J)\numberthis
\end{align*}

We neglected the lack of normal order in these expressions because we already have a nice relation, $\comm{A^I}{A^J}+\comm{B^I}{B^J}+\qty(\comm{A^I}{B^J}-\comm{B^I}{A^J})=0$.
$$$$ 
Now, $\comm{A^I}{C^J}$,

\begin{align*}
    &-\frac{\im}{\sqrt{2\alpha'}}\sum\limits_{n\in\mathbb N^\ast}\frac1n\comm{x_0^-p^I}{\frac{1}{p^+}\qty(L^\perp_{-n}\alpha_n^J-\alpha_{-n}^JL^\perp_{n})}\\
    &=-\frac{\im}{\sqrt{2\alpha'}}\sum\limits_{n\in\mathbb N^\ast}\frac1n\qty{x_0^-\comm{p^I}{\frac{1}{p^+}\qty(L^\perp_{-n}\alpha_n^J-\alpha_{-n}^JL^\perp_{n})}+\comm{x_0^-}{\frac{1}{p^+}\qty(L^\perp_{-n}\alpha_n^J-\alpha_{-n}^JL^\perp_{n})}p^I}\\
    &=-\frac{\im}{\sqrt{2\alpha'}}\sum\limits_{n\in\mathbb N^\ast}\frac1n\qty{x_0^-\frac{1}{p^+}\comm{p^I}{L^\perp_{-n}\alpha_n^J-\alpha_{-n}^JL^\perp_{n}}+\comm{x_0^-}{\frac{1}{p^+}}\qty(L^\perp_{-n}\alpha_n^J-\alpha_{-n}^JL^\perp_{n})p^I}\\
    &=-\frac{\im}{\sqrt{2\alpha'}}\sum\limits_{n\in\mathbb N^\ast}\frac1n\qty{x_0^-\frac{1}{p^+}\qty(\comm{p^I}{L^\perp_{-n}\alpha_n^J}-\comm{p^I}{\alpha_{-n}^JL^\perp_{n}})+\frac{\im}{{p^+}^2}\qty(L^\perp_{-n}\alpha_n^J-\alpha_{-n}^JL^\perp_{n})p^I}\\
    &=-\frac{\im}{\sqrt{2\alpha'}}\sum\limits_{n\in\mathbb N^\ast}\frac1n\left\{x_0^-\frac{1}{p^+}\qty(\comm{p^I}{L^\perp_{-n}}\alpha_n^J+L^\perp_{-n}\comm{p^I}{\alpha_n^J}-\alpha_{-n}^J\comm{p^I}{L^\perp_{n}}-\comm{p^I}{\alpha_{-n}^J}L^\perp_{n})\right.\\
    &\quad\quad\quad\left.+\frac{\im}{{p^+}^2}\qty(L^\perp_{-n}\alpha_n^J-\alpha_{-n}^JL^\perp_{n})p^I\right\}\\
    &=\frac{1}{\sqrt{2\alpha'}{p^+}^2}\sum\limits_{n\in\mathbb N^\ast}\frac1n\qty(L^\perp_{-n}\alpha_n^J-\alpha_{-n}^JL^\perp_{n})p^I\\
    &=\frac{1}{2\alpha'{p^+}^2}\sum\limits_{n\in\mathbb N^\ast}\frac1n\qty(L^\perp_{-n}\alpha_n^J-\alpha_{-n}^JL^\perp_{n})\alpha^I_0
\end{align*}

That is,

\begin{align*}
    \comm{A^I}{C^J}-\comm{A^J}{C^I}&=\frac{1}{2\alpha'{p^+}^2}\sum\limits_{n\in\mathbb N^\ast}\frac1n\qty[\qty(L^\perp_{-n}\alpha_n^J-\alpha_{-n}^JL^\perp_{n})\alpha^I_0-\qty(L^\perp_{-n}\alpha_n^I-\alpha_{-n}^IL^\perp_{n})\alpha^J_0]\numberthis\label{aicj}
\end{align*}
$$$$
Now, $\comm{B^I}{C^J}$,

\begin{align*}
    &\frac{\im}{4\alpha'\sqrt{2\alpha'}{p^+}^2}\sum\limits_{n\in\mathbb N^\ast}\frac1n\comm{x_0^I\qty(L_0^\perp+a)+\qty(L_0^\perp+a)x_0^I}{L^\perp_{-n}\alpha_n^J-\alpha^J_{-n}L^\perp_n}\\
    &=\frac{\im}{4\alpha'\sqrt{2\alpha'}{p^+}^2}\sum\limits_{n\in\mathbb N^\ast}\frac1n\comm{\comm{x_0^I}{\qty(L_0^\perp+a)}+\qty(L^\perp+a)x_0^I+\qty(L_0^\perp+a)x_0^I}{L^\perp_{-n}\alpha_n^J-\alpha^J_{-n}L^\perp_n}\\
    &=\frac{\im}{4\alpha'\sqrt{2\alpha'}{p^+}^2}\sum\limits_{n\in\mathbb N^\ast}\frac1n\comm{2\qty(L_0^\perp+a)x_0^I+\im\sqrt{2\alpha'} \alpha_0^I}{L^\perp_{-n}\alpha_n^J-\alpha^J_{-n}L^\perp_n}\\
    &=\frac{\im}{2\alpha'\sqrt{2\alpha'}{p^+}^2}\sum\limits_{n\in\mathbb N^\ast}\frac1n\comm{\qty(L_0^\perp+a)x_0^I}{L^\perp_{-n}\alpha_n^J-\alpha^J_{-n}L^\perp_n}\\
    &=\frac{\im}{2\alpha'\sqrt{2\alpha'}{p^+}^2}\sum\limits_{n\in\mathbb N^\ast}\frac1n\qty{\comm{L_0^\perp}{L^\perp_{-n}\alpha_n^J-\alpha^J_{-n}L^\perp_n}x_0^I+\qty(L_0^\perp+a)\comm{x_0^I}{L^\perp_{-n}}\alpha_n^J-\qty(L_0^\perp+a)\alpha^J_{-n}\comm{x_0^I}{L^\perp_n}}\\
    &=\frac{\im}{2\alpha'\sqrt{2\alpha'}{p^+}^2}\sum\limits_{n\in\mathbb N^\ast}\frac1n\qty{\comm{L_0^\perp}{L^\perp_{-n}\alpha_n^J-\alpha^J_{-n}L^\perp_n}x_0^I+\im\sqrt{2\alpha'}\qty(L_0^\perp+a)\qty(\alpha_{-n}^I\alpha_n^J-\alpha^J_{-n}\alpha_n^I)}\\
    &=\frac{\im}{2\alpha'\sqrt{2\alpha'}{p^+}^2}\sum\limits_{n\in\mathbb N^\ast}\frac1n\left\{\im\sqrt{2\alpha'}\qty(L_0^\perp+a)\qty(\alpha_{-n}^I\alpha_n^J-\alpha^J_{-n}\alpha_n^I)\right.\\
    &\quad\quad\quad\left.+\comm{L_0^\perp}{L^\perp_{-n}\alpha_n^J}x_0^I-\comm{L_0^\perp}{\alpha^J_{-n}L^\perp_n}x_0^I\right\}\\
    &=\frac{\im}{2\alpha'\sqrt{2\alpha'}{p^+}^2}\sum\limits_{n\in\mathbb N^\ast}\frac1n\left\{\im\sqrt{2\alpha'}\qty(L_0^\perp+a)\qty(\alpha_{-n}^I\alpha_n^J-\alpha^J_{-n}\alpha_n^I)\right.\\
    &\quad\quad\quad\left.+L^\perp_{-n}\comm{L_0^\perp}{\alpha_n^J}x_0^I+\comm{L_0^\perp}{L^\perp_{-n}}\alpha_n^Jx_0^I-\alpha^J_{-n}\comm{L_0^\perp}{L^\perp_n}x_0^I-\comm{L_0^\perp}{\alpha^J_{-n}}L^\perp_nx_0^I\right\}\\
    &=\frac{\im}{2\alpha'\sqrt{2\alpha'}{p^+}^2}\sum\limits_{n\in\mathbb N^\ast}\frac1n\left\{\im\sqrt{2\alpha'}\qty(L_0^\perp+a)\qty(\alpha_{-n}^I\alpha_n^J-\alpha^J_{-n}\alpha_n^I)\right.\\
    &\quad\quad\quad\left.-nL^\perp_{-n}\alpha_n^Jx_0^I+nL^\perp_{-n}\alpha_n^Jx_0^I+\alpha^J_{-n}nL_{n}^\perp x_0^I-n\alpha^J_{-n}L^\perp_nx_0^I\right\}\\
    &=\frac{\im}{2\alpha'\sqrt{2\alpha'}{p^+}^2}\sum\limits_{n\in\mathbb N^\ast}\frac1n\left\{\im\sqrt{2\alpha'}\qty(L_0^\perp+a)\qty(\alpha_{-n}^I\alpha_n^J-\alpha^J_{-n}\alpha_n^I)\right\}\\
    &=-\frac{1}{2\alpha'{p^+}^2}\sum\limits_{n\in\mathbb N^\ast}\frac1n\qty(L_0^\perp+a)\qty(\alpha_{-n}^I\alpha_n^J-\alpha^J_{-n}\alpha_n^J)\\
    &=-\frac{1}{2\alpha'{p^+}^2}\sum\limits_{n\in\mathbb N^\ast}\frac1n\qty(\comm{L_0^\perp+a}{\alpha_{-n}^I}\alpha_n^J+\alpha_{-n}^I\qty(L_0^\perp +a)\alpha_n^J-\comm{L_0^\perp+a}{\alpha_{-n}^J}\alpha_n^I-\alpha_{-n}^J\qty(L^\perp_0+a)\alpha_n^I)\\
    &=-\frac{1}{2\alpha'{p^+}^2}\sum\limits_{n\in\mathbb N^\ast}\frac1n\qty(n\alpha_{-n}^I\alpha_n^J+\alpha_{-n}^I\qty(L_0^\perp +a)\alpha_n^J-n\alpha_{-n}^J\alpha_n^I-\alpha_{-n}^J\qty(L^\perp_0+a)\alpha_n^I)\\
    &=-\frac{1}{2\alpha'{p^+}^2}\sum\limits_{n\in\mathbb N^\ast}\frac1n\qty(\alpha_{-n}^I\qty(L_0^\perp +a+n)\alpha_n^J-\alpha_{-n}^J\qty(L^\perp_0+a+n)\alpha_n^I)
\end{align*}

Where at the end we made some tweaks for the whole expression stay normal-ordered. Summing the anti-symmetric part,

\begin{align*}
    \comm{B^I}{C^J}-\comm{B^J}{C^I}&=-\frac{1}{\alpha'{p^+}^2}\sum\limits_{n\in\mathbb N^\ast}\frac1n\qty(\alpha_{-n}^I\qty(L_0^\perp +a+n)\alpha_n^J-\alpha_{-n}^J\qty(L^\perp_0+a+n)\alpha_n^I)\numberthis\label{bicj}
\end{align*}
$$$$

At last, $\comm{C^I}{C^J}$, this is the hardest one. So we'll need to split again into even more parts to make the result intelligible, first, notice that 
the full commutator is,

\begin{align*}
    \comm{C^I}{C^J}=-\frac{1}{2\alpha'{p^+}^2}\sum\limits_{n,m\in\mathbb N^\ast}\frac{1}{nm}\comm{L^\perp_{-n}\alpha^I_n-\alpha^I_{-n}L^\perp_n}{L^\perp_{-m}\alpha^J_m-\alpha^J_{-m}L^\perp_m}\numberthis\label{cicj}
\end{align*}

From which we subdivide as,

\begin{align*}
    C_{11}^{IJ}&=-\frac{1}{2\alpha'{p^+}^2}\sum\limits_{n,m\in\mathbb N^\ast}\frac{1}{nm}\comm{L^\perp_{-n}\alpha^I_n}{L^\perp_{-m}\alpha^J_m}\\
    C_{21}^{IJ}&=\frac{1}{2\alpha'{p^+}^2}\sum\limits_{n,m\in\mathbb N^\ast}\frac{1}{nm}\comm{\alpha^I_{-n}L^\perp_n}{L^\perp_{-m}\alpha^J_m}\\
    C_{12}^{IJ}&=\frac{1}{2\alpha'{p^+}^2}\sum\limits_{n,m\in\mathbb N^\ast}\frac{1}{nm}\comm{L^\perp_{-n}\alpha^I_n}{\alpha^J_{-m}L^\perp_m}\\
    C_{22}^{IJ}&=-\frac{1}{2\alpha'{p^+}^2}\sum\limits_{n,m\in\mathbb N^\ast}\frac{1}{nm}\comm{\alpha^I_{-n}L^\perp_n}{\alpha^J_{-m}L^\perp_m}\numberthis\label{c11def}
\end{align*}

Starting by $C_{11}^{IJ}$,

\begin{align*}
    &-\frac{1}{2\alpha'{p^+}^2}\sum\limits_{n,m\in\mathbb N^\ast}\frac{1}{nm}\comm{L^\perp_{-n}\alpha^I_n}{L^\perp_{-m}\alpha^J_m}\\
    &=-\frac{1}{2\alpha'{p^+}^2}\sum\limits_{n,m\in\mathbb N^\ast}\frac{1}{nm}\qty(L^\perp_{-n}\comm{\alpha^I_n}{L^\perp_{-m}\alpha^J_m}+\comm{L^\perp_{-n}}{L^\perp_{-m}\alpha^J_m}\alpha^I_n)\\
    &=-\frac{1}{2\alpha'{p^+}^2}\sum\limits_{n,m\in\mathbb N^\ast}\frac{1}{nm}\left(L^\perp_{-n}L^\perp_{-m}\comm{\alpha^I_n}{\alpha^J_m}+L^\perp_{-n}\comm{\alpha^I_n}{L^\perp_{-m}}\alpha^J_m+L^\perp_{-m}\comm{L^\perp_{-n}}{\alpha^J_m}\alpha^I_n\right.\\
    &\quad\quad\quad\left.+\comm{L^\perp_{-n}}{L^\perp_{-m}}\alpha^J_m\alpha^I_n\right)\\
    &=-\frac{1}{2\alpha'{p^+}^2}\sum\limits_{n,m\in\mathbb N^\ast}\frac{1}{nm}\left(L^\perp_{-n}L^\perp_{-m}ng^{IJ}\delta_{n+m,0}+L^\perp_{-n}n\alpha^I_{n-m}\alpha^J_m-L^\perp_{-m}m\alpha^J_{m-n}\alpha^I_n\right.\\
    &\quad\quad\quad\left.+\qty(\qty(-n+m)L^\perp_{-n-m}-\frac{D-2}{12}\qty(n^3-n)\delta_{-n-m,0})\alpha^J_m\alpha^I_n\right)\\
    &=-\frac{1}{2\alpha'{p^+}^2}\sum\limits_{n,m\in\mathbb N^\ast}\frac{1}{nm}\left(L^\perp_{-n}n\alpha^I_{n-m}\alpha^J_m-L^\perp_{-m}m\alpha^J_{m-n}\alpha^I_n+\qty(-n+m)L^\perp_{-n-m}\alpha^J_m\alpha^I_n\right)
\end{align*}

Notice that the last term is already normal-ordered, but, the two first ones aren't. We'll rewrite them in normal ordered form. First, 
change labels $n\leftrightarrow m$ in the first term. Then,

\begin{align*}
    C_{11}^{IJ}&=-\frac{1}{2\alpha'{p^+}^2}\sum\limits_{n,m\in\mathbb N^\ast}\frac{1}{nm}\qty(m-n)L^\perp_{-n-m}\alpha^J_m\alpha^I_n-\frac{1}{2\alpha'{p^+}^2}\sum\limits_{n,m\in\mathbb N^\ast}\frac{1}{n}\left(L^\perp_{-m}\alpha^I_{m-n}\alpha^J_n-L^\perp_{-m}\alpha^J_{m-n}\alpha^I_n\right)\\
    C_{11}^{IJ}&=-\frac{1}{2\alpha'{p^+}^2}\sum\limits_{n,m\in\mathbb N^\ast}\frac{1}{nm}\qty(m-n)L^\perp_{-n-m}\alpha^J_m\alpha^I_n\\
    &\quad\quad\quad-\frac{1}{2\alpha'{p^+}^2}\sum\limits_{n\in\mathbb N^\ast}\frac{1}{n}\qty(\sum\limits_{m>n}^\infty+\sum\limits_{m>0}^{n})\left(L^\perp_{-m}\alpha^I_{m-n}\alpha^J_n-L^\perp_{-m}\alpha^J_{m-n}\alpha^I_n\right)\\
    C_{11}^{IJ}&=-\frac{1}{2\alpha'{p^+}^2}\sum\limits_{n,m\in\mathbb N^\ast}\frac{1}{nm}\qty(m-n)L^\perp_{-n-m}\alpha^J_m\alpha^I_n\\
    &\quad\quad\quad-\frac{1}{2\alpha'{p^+}^2}\sum\limits_{n\in\mathbb N^\ast}\frac{1}{n}\sum\limits_{m>n}^\infty\left(L^\perp_{-m}\alpha^I_{m-n}\alpha^J_n-L^\perp_{-m}\alpha^J_{m-n}\alpha^I_n\right)\\
    &\quad\quad\quad-\frac{1}{2\alpha'{p^+}^2}\sum\limits_{n\in\mathbb N^\ast}\frac{1}{n}\sum\limits_{m>0}^{n}\left(L^\perp_{-m}\alpha^I_{m-n}\alpha^J_n-L^\perp_{-m}\alpha^J_{m-n}\alpha^I_n\right)
\end{align*}

It's clear in this form that the first and second ine are normal-ordered, but the third isn't, let's fix this. Also, we relabel the sum over $m$ in the second line, 
we make $m\rightarrow m+n$, this makes the sum over $m$ to be over $m>0$,

\begin{align*}
    C_{11}^{IJ}&=-\frac{1}{2\alpha'{p^+}^2}\sum\limits_{n,m\in\mathbb N^\ast}\frac{1}{nm}\qty(m-n)L^\perp_{-n-m}\alpha^J_m\alpha^I_n\\
    &\quad\quad\quad-\frac{1}{2\alpha'{p^+}^2}\sum\limits_{n\in\mathbb N^\ast}\frac{1}{n}\sum\limits_{m>n}^\infty\left(L^\perp_{-m}\alpha^I_{m-n}\alpha^J_n-L^\perp_{-m}\alpha^J_{m-n}\alpha^I_n\right)\\
    &\quad\quad\quad-\frac{1}{2\alpha'{p^+}^2}\sum\limits_{n\in\mathbb N^\ast}\frac{1}{n}\sum\limits_{m>0}^{n}\left(\qty(\comm{L^\perp_{-m}}{\alpha^I_{m-n}}+\alpha_{m-n}^IL^\perp_{-m})\alpha^J_n-\qty(\comm{L^\perp_{-m}}{\alpha^J_{m-n}}+\alpha^J_{m-n}L^\perp_{-m})\alpha^I_n\right)\\
    C_{11}^{IJ}&=-\frac{1}{2\alpha'{p^+}^2}\sum\limits_{n,m\in\mathbb N^\ast}\frac{1}{nm}\qty(m-n)L^\perp_{-n-m}\alpha^J_m\alpha^I_n\\
    &\quad\quad\quad-\frac{1}{2\alpha'{p^+}^2}\sum\limits_{n\in\mathbb N^\ast}\frac{1}{n}\sum\limits_{m>0}^\infty\left(L^\perp_{-m-n}\alpha^I_{m}\alpha^J_n-L^\perp_{-m-n}\alpha^J_{m}\alpha^I_n\right)\\
    &\quad\quad\quad-\frac{1}{2\alpha'{p^+}^2}\sum\limits_{n\in\mathbb N^\ast}\frac{1}{n}\sum\limits_{m>0}^{n}\left(\qty(\qty(n-m)\alpha^I_{-n}+\alpha_{m-n}^IL^\perp_{-m})\alpha^J_n-\qty(\qty(n-m)\alpha^J_{-n}+\alpha^J_{m-n}L^\perp_{-m})\alpha^I_n\right)\\
    C_{11}^{IJ}&=-\frac{1}{2\alpha'{p^+}^2}\sum\limits_{n,m\in\mathbb N^\ast}\frac{1}{nm}\qty(m-n)L^\perp_{-n-m}\alpha^J_m\alpha^I_n\\
    &\quad\quad\quad-\frac{1}{2\alpha'{p^+}^2}\sum\limits_{n,m\in\mathbb N^\ast}\frac{1}{nm}\left(mL^\perp_{-m-n}\alpha^I_{m}\alpha^J_n-mL^\perp_{-m-n}\alpha^J_{m}\alpha^I_n\right)\\
    &\quad\quad\quad-\frac{1}{2\alpha'{p^+}^2}\sum\limits_{n\in\mathbb N^\ast}\frac{1}{n}\sum\limits_{m>0}^{n}\left(\qty(\qty(n-m)\alpha^I_{-n}+\alpha_{m-n}^IL^\perp_{-m})\alpha^J_n-\qty(\qty(n-m)\alpha^J_{-n}+\alpha^J_{m-n}L^\perp_{-m})\alpha^I_n\right)
\end{align*}

As the $\alpha$s commute in the second line, due to $n,m>0$, we exchange them in the first term and them relabel $n\leftrightarrow m$ in it,

\begin{align*}
    C_{11}^{IJ}&=-\frac{1}{2\alpha'{p^+}^2}\sum\limits_{n,m\in\mathbb N^\ast}\frac{1}{nm}\qty(m-n)L^\perp_{-n-m}\alpha^J_m\alpha^I_n\\
    &\quad\quad\quad-\frac{1}{2\alpha'{p^+}^2}\sum\limits_{n,m\in\mathbb N^\ast}\frac{1}{nm}\left(nL^\perp_{-m-n}\alpha^J_m\alpha^I_{n}-mL^\perp_{-m-n}\alpha^J_{m}\alpha^I_n\right)\\
    &\quad\quad\quad-\frac{1}{2\alpha'{p^+}^2}\sum\limits_{n\in\mathbb N^\ast}\frac{1}{n}\sum\limits_{m>0}^{n}\left(\qty(\qty(n-m)\alpha^I_{-n}+\alpha_{m-n}^IL^\perp_{-m})\alpha^J_n-\qty(\qty(n-m)\alpha^J_{-n}+\alpha^J_{m-n}L^\perp_{-m})\alpha^I_n\right)\\
    C_{11}^{IJ}&=-\frac{1}{2\alpha'{p^+}^2}\sum\limits_{n,m\in\mathbb N^\ast}\frac{1}{nm}\qty(m-n)L^\perp_{-n-m}\alpha^J_m\alpha^I_n-\frac{1}{2\alpha'{p^+}^2}\sum\limits_{n,m\in\mathbb N^\ast}\frac{1}{nm}\qty(n-m)L^\perp_{-m-n}\alpha^J_m\alpha^I_{n}\\
    &\quad\quad\quad-\frac{1}{2\alpha'{p^+}^2}\sum\limits_{n\in\mathbb N^\ast}\frac{1}{n}\sum\limits_{m>0}^{n}\left(\qty(\qty(n-m)\alpha^I_{-n}+\alpha_{m-n}^IL^\perp_{-m})\alpha^J_n-\qty(\qty(n-m)\alpha^J_{-n}+\alpha^J_{m-n}L^\perp_{-m})\alpha^I_n\right)\\
    C_{11}^{IJ}&=-\frac{1}{2\alpha'{p^+}^2}\sum\limits_{n\in\mathbb N^\ast}\frac{1}{n}\sum\limits_{m>0}^{n}\left(\qty(\qty(n-m)\alpha^I_{-n}+\alpha_{m-n}^IL^\perp_{-m})\alpha^J_n-\qty(\qty(n-m)\alpha^J_{-n}+\alpha^J_{m-n}L^\perp_{-m})\alpha^I_n\right)\\
    C_{11}^{IJ}&=-\frac{1}{2\alpha'{p^+}^2}\sum\limits_{n\in\mathbb N^\ast}\frac{1}{n}\qty(\alpha^I_{-n}\alpha^J_n-\alpha^J_{-n}\alpha^I_n)\sum\limits_{m>0}^{n}\qty(n-m)\\
    &\quad\quad\quad-\frac{1}{2\alpha'{p^+}^2}\sum\limits_{n\in\mathbb N^\ast}\frac{1}{n}\sum\limits_{m>0}^{n}\qty(\alpha_{m-n}^IL^\perp_{-m}\alpha^J_n-\alpha^J_{m-n}L^\perp_{-m}\alpha^I_n)
\end{align*}

The sum over $m$ in the first line is trivial, $\sum\limits_{m>0}^nn=n^2$ and, $\sum\limits_{m>0}^nm=\frac12n\qty(n+1)$. In the second line we make a 
change of summation variables of $m\rightarrow n-m$, this makes the sum range go to $0\leq m\leq n-1$,

\begin{align*}
    C_{11}^{IJ}&=-\frac{1}{2\alpha'{p^+}^2}\sum\limits_{n\in\mathbb N^\ast}\frac{1}{n}\qty(\alpha^I_{-n}\alpha^J_n-\alpha^J_{-n}\alpha^I_n)\qty(n^2-\frac12n\qty(n+1))\\
    &\quad\quad\quad-\frac{1}{2\alpha'{p^+}^2}\sum\limits_{n\in\mathbb N^\ast}\frac{1}{n}\sum\limits_{m\geq0}^{n-1}\qty(\alpha_{-m}^IL^\perp_{m-n}\alpha^J_n-\alpha^J_{-m}L^\perp_{m-n}\alpha^I_n)\\
    C_{11}^{IJ}&=-\frac{1}{4\alpha'{p^+}^2}\sum\limits_{n\in\mathbb N^\ast}\qty(n-1)\qty(\alpha^I_{-n}\alpha^J_n-\alpha^J_{-n}\alpha^I_n)\\
    &\quad\quad\quad-\frac{1}{2\alpha'{p^+}^2}\sum\limits_{n\in\mathbb N^\ast}\frac{1}{n}\sum\limits_{m\geq0}^{n-1}\qty(\alpha_{-m}^IL^\perp_{m-n}\alpha^J_n-\alpha^J_{-m}L^\perp_{m-n}\alpha^I_n)\numberthis\label{c11}
\end{align*}

Now, the next one, $C_{12}^{IJ}$, notice that $C_{21}^{IJ}=-C_{12}^{JI}$. That is, we just need to compute $C_{12}^{IJ}$ 
and sum with it's anti-symmetric part.

\begin{align*}
    C_{12}^{IJ}&=\frac{1}{2\alpha'{p^+}^2}\sum\limits_{n,m\in\mathbb N^\ast}\frac{1}{nm}\comm{L^\perp_{-n}\alpha_n^I}{\alpha^J_{-m}L^\perp_m}\\
    &=\frac{1}{2\alpha'{p^+}^2}\sum\limits_{n,m\in\mathbb N^\ast}\frac{1}{nm}\qty(L^\perp_{-n}\comm{\alpha_n^I}{\alpha^J_{-m}L^\perp_m}+\comm{L^\perp_{-n}}{\alpha^J_{-m}L^\perp_m}\alpha_n^I)\\
    &=\frac{1}{2\alpha'{p^+}^2}\sum\limits_{n,m\in\mathbb N^\ast}\frac{1}{nm}\left(L^\perp_{-n}\alpha^J_{-m}\comm{\alpha_n^I}{L^\perp_m}+L^\perp_{-n}\comm{\alpha_n^I}{\alpha^J_{-m}}L^\perp_m+\comm{L^\perp_{-n}}{\alpha^J_{-m}}L^\perp_m\alpha_n^I\right.\\
    &\quad\quad\quad\left.+\alpha^J_{-m}\comm{L^\perp_{-n}}{L^\perp_m}\alpha_n^I\right)\\
    &=\frac{1}{2\alpha'{p^+}^2}\sum\limits_{n,m\in\mathbb N^\ast}\frac{1}{nm}\left(L^\perp_{-n}\alpha^J_{-m}n\alpha^I_{n+m}+L^\perp_{-n}ng^{IJ}\delta_{n-m,0}L^\perp_m+m\alpha^J_{-m-n}L^\perp_m\alpha_n^I\right.\\
    &\quad\quad\quad\left.+\alpha^J_{-m}\qty(\qty(-n-m)L^\perp_{-n+m}-\frac{D-2}{12}\qty(n^3-n)\delta_{-n+m,0})\alpha_n^I\right)
\end{align*}

It's obvious that when we anti-symmetrize the $g^{IJ}$ term will drop out, so we'll not even attempt to normal-order it, but we won't 
omit it either for sake of completeness. The only relevant non-normal-ordered term is the first one,

\begin{align*}
    &=\frac{1}{2\alpha'{p^+}^2}\sum\limits_{n,m\in\mathbb N^\ast}\frac{1}{nm}\left(n\qty(\comm{L^\perp_{-n}}{\alpha^J_{-m}}+\alpha^J_{-m}L^\perp_{-n})\alpha^I_{n+m}+m\alpha^J_{-m-n}L^\perp_m\alpha_n^I-\qty(n+m)\alpha^J_{-m}L^\perp_{-n+m}\alpha_n^I\right.\\
    &\quad\quad\quad\left.+L^\perp_{-n}ng^{IJ}\delta_{n-m,0}L^\perp_m\right)-\frac{1}{2\alpha'{p^+}^2}\frac{D-2}{12}\sum\limits_{n\in\mathbb N^\ast}\frac{n^2-1}{n}\alpha^J_{-n}\alpha_n^I\\
    &=\frac{1}{2\alpha'{p^+}^2}\sum\limits_{n,m\in\mathbb N^\ast}\frac{1}{nm}\left(nm\alpha^J_{-m-n}\alpha^I_{n+m}+n\alpha^J_{-m}L^\perp_{-n}\alpha^I_{n+m}+m\alpha^J_{-m-n}L^\perp_m\alpha_n^I-\qty(n+m)\alpha^J_{-m}L^\perp_{-n+m}\alpha_n^I\right.\\
    &\quad\quad\quad\left.+L^\perp_{-n}ng^{IJ}\delta_{n-m,0}L^\perp_m\right)-\frac{1}{2\alpha'{p^+}^2}\frac{D-2}{12}\sum\limits_{n\in\mathbb N^\ast}\frac{n^2-1}{n}\alpha^J_{-n}\alpha_n^I\\
    &=\frac{1}{2\alpha'{p^+}^2}\sum\limits_{n,m\in\mathbb N^\ast}\frac{1}{m}\alpha^J_{-m}L^\perp_{-n}\alpha^I_{n+m}\\
    &\quad\quad\quad+\frac{1}{2\alpha'{p^+}^2}\sum\limits_{n,m\in\mathbb N^\ast}\frac{1}{n}\alpha^J_{-m-n}L^\perp_m\alpha_n^I\\
    &\quad\quad\quad-\frac{1}{2\alpha'{p^+}^2}\sum\limits_{n,m\in\mathbb N^\ast}\qty[\frac1n\alpha^J_{-m}L^\perp_{-n+m}\alpha_n^I+\frac1m\alpha^J_{-m}L^\perp_{-n+m}\alpha_n^I]\\
    &\quad\quad\quad+\frac{1}{2\alpha'{p^+}^2}\sum\limits_{n,m\in\mathbb N^\ast}\frac{1}{nm}\left(nm\alpha^J_{-m-n}\alpha^I_{n+m}+L^\perp_{-n}ng^{IJ}\delta_{n-m,0}L^\perp_m\right)\\
    &\quad\quad\quad-\frac{1}{2\alpha'{p^+}^2}\frac{D-2}{12}\sum\limits_{n\in\mathbb N^\ast}\frac{n^2-1}{n}\alpha^J_{-n}\alpha_n^I
\end{align*}

We'll relabel in the first line $n\rightarrow n+m$, which change the range of the sum to $n>m$, and in the second line we make $m\rightarrow m+n$, and the new 
range will be $m>n$,

\begin{align*}
    C_{12}^{IJ}&=\frac{1}{2\alpha'{p^+}^2}\sum\limits_{m\in\mathbb N^\ast}\frac{1}{m}\sum\limits_{n>m}^\infty\alpha^J_{-m}L^\perp_{-n+m}\alpha^I_{n}\\
    &\quad\quad\quad+\frac{1}{2\alpha'{p^+}^2}\sum\limits_{n\in\mathbb N^\ast}\frac{1}{n}\sum\limits_{m>n}\alpha^J_{-m}L^\perp_{m-n}\alpha_n^I\\
    &\quad\quad\quad-\frac{1}{2\alpha'{p^+}^2}\sum\limits_{n,m\in\mathbb N^\ast}\qty[\frac1n\alpha^J_{-m}L^\perp_{-n+m}\alpha_n^I+\frac1m\alpha^J_{-m}L^\perp_{-n+m}\alpha_n^I]\\
    &\quad\quad\quad+\frac{1}{2\alpha'{p^+}^2}\sum\limits_{n,m\in\mathbb N^\ast}\frac{1}{nm}\left(nm\alpha^J_{-m-n}\alpha^I_{n+m}+L^\perp_{-n}ng^{IJ}\delta_{n-m,0}L^\perp_m\right)\\
    &\quad\quad\quad-\frac{1}{2\alpha'{p^+}^2}\frac{D-2}{12}\sum\limits_{n\in\mathbb N^\ast}\frac{n^2-1}{n}\alpha^J_{-n}\alpha_n^I\\
    C_{12}^{IJ}&=-\frac{1}{2\alpha'{p^+}^2}\qty[\sum\limits_{n\in\mathbb N^\ast}\frac1n\sum\limits_{m>0}^n\alpha^J_{-m}L^\perp_{-n+m}\alpha_n^I+\sum\limits_{m\in\mathbb N^\ast}\frac1m\sum\limits_{n>0}^m\alpha^J_{-m}L^\perp_{-n+m}\alpha_n^I]\\
    &\quad\quad\quad+\frac{1}{2\alpha'{p^+}^2}\sum\limits_{n,m\in\mathbb N^\ast}\frac{1}{nm}\left(nm\alpha^J_{-m-n}\alpha^I_{n+m}+L^\perp_{-n}ng^{IJ}\delta_{n-m,0}L^\perp_m\right)\\
    &\quad\quad\quad-\frac{1}{2\alpha'{p^+}^2}\frac{D-2}{12}\sum\limits_{n\in\mathbb N^\ast}\frac{n^2-1}{n}\alpha^J_{-n}\alpha_n^I
\end{align*}

So that,

\begin{align*}
    C_{12}^{IJ}-C_{12}^{JI}&=\frac{1}{2\alpha'{p^+}^2}\sum\limits_{n\in\mathbb N^\ast}\frac1n\sum\limits_{m>0}^n\qty[\alpha^I_{-m}L^\perp_{-n+m}\alpha_n^J-\alpha^J_{-m}L^\perp_{-n+m}\alpha_n^I]\\
    &\quad\quad\quad+\frac{1}{2\alpha'{p^+}^2}\sum\limits_{m\in\mathbb N^\ast}\frac1m\sum\limits_{n>0}^m\qty[\alpha^I_{-m}L^\perp_{-n+m}\alpha_n^J-\alpha^J_{-m}L^\perp_{-n+m}\alpha_n^I]\\
    &\quad\quad\quad-\frac{1}{2\alpha'{p^+}^2}\sum\limits_{n,m\in\mathbb N^\ast}\left(\alpha^I_{-m-n}\alpha^J_{n+m}-\alpha^J_{-m-n}\alpha^I_{n+m}\right)\\
    &\quad\quad\quad+\frac{1}{2\alpha'{p^+}^2}\frac{D-2}{12}\sum\limits_{n\in\mathbb N^\ast}\frac{n^2-1}{n}\qty(\alpha^I_{-n}\alpha_n^J-\alpha^J_{-n}\alpha_n^I)\numberthis\label{c12}
\end{align*}

At last, we compute $C_{22}^{IJ}$,

\begin{align*}
    C_{22}^{IJ}&=-\frac{1}{2\alpha'{p^+}^2}\sum\limits_{n,m\in\mathbb N^\ast}\frac{1}{nm}\comm{\alpha_{-n}^IL^\perp_n}{\alpha^J_{-m}L^\perp_m}\\
    &=-\frac{1}{2\alpha'{p^+}^2}\sum\limits_{n,m\in\mathbb N^\ast}\frac{1}{nm}\qty(\alpha_{-n}^I\comm{L^\perp_n}{\alpha^J_{-m}L^\perp_m}+\comm{\alpha_{-n}^I}{\alpha^J_{-m}L^\perp_m}L^\perp_n)\\
    &=-\frac{1}{2\alpha'{p^+}^2}\sum\limits_{n,m\in\mathbb N^\ast}\frac{1}{nm}\left(\alpha_{-n}^I\comm{L^\perp_n}{\alpha^J_{-m}}L^\perp_m+\alpha^J_{-m}\comm{\alpha_{-n}^I}{L^\perp_m}L^\perp_n+\comm{\alpha_{-n}^I}{\alpha^J_{-m}}L^\perp_mL^\perp_n\right.\\
    &\quad\quad\quad\left.+\alpha_{-n}^I\alpha^J_{-m}\comm{L^\perp_n}{L^\perp_m}\right)\\
    &=-\frac{1}{2\alpha'{p^+}^2}\sum\limits_{n,m\in\mathbb N^\ast}\frac{1}{nm}\left(\alpha_{-n}^I\comm{L^\perp_n}{\alpha^J_{-m}}L^\perp_m+\alpha^J_{-m}\comm{\alpha_{-n}^I}{L^\perp_m}L^\perp_n+\comm{\alpha_{-n}^I}{\alpha^J_{-m}}L^\perp_mL^\perp_n\right.\\
    &\quad\quad\quad\left.+\alpha_{-n}^I\alpha^J_{-m}\comm{L^\perp_n}{L^\perp_m}\right)\\
    &=-\frac{1}{2\alpha'{p^+}^2}\sum\limits_{n,m\in\mathbb N^\ast}\frac{1}{nm}\left(\alpha_{-n}^Im\alpha^J_{n-m}L^\perp_m-\alpha^J_{-m}n\alpha^I_{-n+m}L^\perp_n-ng^{IJ}\delta_{-n-m,0}L^\perp_mL^\perp_n\right.\\
    &\quad\quad\quad\left.+\alpha_{-n}^I\alpha^J_{-m}\qty(\qty(n-m)L^\perp_{n+m}+\frac{D-2}{12}\qty(n^3-n)\delta_{n+m,0})\right)\\
    &=-\frac{1}{2\alpha'{p^+}^2}\sum\limits_{n,m\in\mathbb N^\ast}\frac{1}{nm}\left(m\alpha_{-n}^I\alpha^J_{n-m}L^\perp_m-n\alpha^J_{-m}\alpha^I_{-n+m}L^\perp_n+\qty(n-m)\alpha_{-n}^I\alpha^J_{-m}L^\perp_{n+m}\right)\\
    &=-\frac{1}{2\alpha'{p^+}^2}\sum\limits_{n,m\in\mathbb N^\ast}\frac{1}{nm}\qty(n-m)\alpha_{-n}^I\alpha^J_{-m}L^\perp_{n+m}\\
    &\quad\quad\quad-\frac{1}{2\alpha'{p^+}^2}\sum\limits_{n,m\in\mathbb N^\ast}\frac{1}{n}\alpha_{-n}^I\alpha^J_{n-m}L^\perp_m\\
    &\quad\quad\quad+\frac{1}{2\alpha'{p^+}^2}\sum\limits_{n,m\in\mathbb N^\ast}\frac{1}{m}\alpha^J_{-m}\alpha^I_{-n+m}L^\perp_n\\
    &=-\frac{1}{2\alpha'{p^+}^2}\sum\limits_{n,m\in\mathbb N^\ast}\frac{1}{nm}\qty(n-m)\alpha_{-n}^I\alpha^J_{-m}L^\perp_{n+m}\\
    &\quad\quad\quad-\frac{1}{2\alpha'{p^+}^2}\sum\limits_{n,m\in\mathbb N^\ast}\frac{1}{n}\alpha_{-n}^I\alpha^J_{n-m}L^\perp_m\\
    &\quad\quad\quad+\frac{1}{2\alpha'{p^+}^2}\sum\limits_{n,m\in\mathbb N^\ast}\frac{1}{n}\alpha^J_{-n}\alpha^I_{n-m}L^\perp_m\\
    &=-\frac{1}{2\alpha'{p^+}^2}\sum\limits_{n,m\in\mathbb N^\ast}\frac{1}{nm}\qty(n-m)\alpha_{-n}^I\alpha^J_{-m}L^\perp_{n+m}\\
    &\quad\quad\quad-\frac{1}{2\alpha'{p^+}^2}\sum\limits_{n\in\mathbb N^\ast}\frac{1}{n}\qty[\sum\limits_{m>0}^n+\sum\limits_{m>n}^\infty]\qty(\alpha_{-n}^I\alpha^J_{n-m}-\alpha_{-n}^J\alpha^I_{n-m})L^\perp_m\\
    &=-\frac{1}{2\alpha'{p^+}^2}\sum\limits_{n,m\in\mathbb N^\ast}\frac{1}{nm}\qty(n-m)\alpha_{-n}^I\alpha^J_{-m}L^\perp_{n+m}-\frac{1}{2\alpha'{p^+}^2}\sum\limits_{n\in\mathbb N^\ast}\frac{1}{n}\sum\limits_{m>n}^\infty\qty(\alpha_{-n}^I\alpha^J_{n-m}-\alpha_{-n}^J\alpha^I_{n-m})L^\perp_m\\
    &\quad\quad\quad-\frac{1}{2\alpha'{p^+}^2}\sum\limits_{n\in\mathbb N^\ast}\frac{1}{n}\sum\limits_{m>0}^n\qty(\alpha_{-n}^I\alpha^J_{n-m}-\alpha_{-n}^J\alpha^I_{n-m})L^\perp_m\\
    &=-\frac{1}{2\alpha'{p^+}^2}\sum\limits_{n,m\in\mathbb N^\ast}\frac{1}{nm}\qty(n-m)\alpha_{-n}^I\alpha^J_{-m}L^\perp_{n+m}-\frac{1}{2\alpha'{p^+}^2}\sum\limits_{n\in\mathbb N^\ast}\frac{1}{n}\sum\limits_{m>0}^\infty\qty(\alpha_{-n}^I\alpha^J_{-m}-\alpha_{-n}^J\alpha^I_{-m})L^\perp_{m+n}\\
    &\quad\quad\quad-\frac{1}{2\alpha'{p^+}^2}\sum\limits_{n\in\mathbb N^\ast}\frac{1}{n}\sum\limits_{m>0}^n\qty(\alpha_{-n}^I\alpha^J_{n-m}-\alpha_{-n}^J\alpha^I_{n-m})L^\perp_m\\
    &=-\frac{1}{2\alpha'{p^+}^2}\sum\limits_{n,m\in\mathbb N^\ast}\frac{1}{nm}\qty(n-m)\alpha_{-n}^I\alpha^J_{-m}L^\perp_{n+m}-\frac{1}{2\alpha'{p^+}^2}\sum\limits_{n,m\in\mathbb N^\ast}\frac{1}{nm}\qty(m\alpha_{-n}^I\alpha^J_{-m}-m\alpha_{-n}^J\alpha^I_{-m})L^\perp_{m+n}\\
    &\quad\quad\quad-\frac{1}{2\alpha'{p^+}^2}\sum\limits_{n\in\mathbb N^\ast}\frac{1}{n}\sum\limits_{m>0}^n\qty(\alpha_{-n}^I\alpha^J_{n-m}-\alpha_{-n}^J\alpha^I_{n-m})L^\perp_m\\
    &=-\frac{1}{2\alpha'{p^+}^2}\sum\limits_{n,m\in\mathbb N^\ast}\frac{1}{nm}\qty(n-m)\alpha_{-n}^I\alpha^J_{-m}L^\perp_{n+m}-\frac{1}{2\alpha'{p^+}^2}\sum\limits_{n,m\in\mathbb N^\ast}\frac{1}{nm}\qty(m\alpha_{-n}^I\alpha^J_{-m}-n\alpha_{-m}^J\alpha^I_{-n})L^\perp_{m+n}\\
    &\quad\quad\quad-\frac{1}{2\alpha'{p^+}^2}\sum\limits_{n\in\mathbb N^\ast}\frac{1}{n}\sum\limits_{m>0}^n\qty(\alpha_{-n}^I\alpha^J_{n-m}-\alpha_{-n}^J\alpha^I_{n-m})L^\perp_m\\
    &=-\frac{1}{2\alpha'{p^+}^2}\sum\limits_{n,m\in\mathbb N^\ast}\frac{1}{nm}\qty(n-m)\alpha_{-n}^I\alpha^J_{-m}L^\perp_{n+m}-\frac{1}{2\alpha'{p^+}^2}\sum\limits_{n,m\in\mathbb N^\ast}\frac{1}{nm}\qty(m-n)\alpha_{-n}^I\alpha^J_{-m}L^\perp_{m+n}\\
    &\quad\quad\quad-\frac{1}{2\alpha'{p^+}^2}\sum\limits_{n\in\mathbb N^\ast}\frac{1}{n}\sum\limits_{m>0}^n\qty(\alpha_{-n}^I\alpha^J_{n-m}-\alpha_{-n}^J\alpha^I_{n-m})L^\perp_m\\
    &=-\frac{1}{2\alpha'{p^+}^2}\sum\limits_{n\in\mathbb N^\ast}\frac{1}{n}\sum\limits_{m>0}^n\qty(\alpha_{-n}^I\alpha^J_{n-m}-\alpha_{-n}^J\alpha^I_{n-m})L^\perp_m\\
    &=-\frac{1}{2\alpha'{p^+}^2}\sum\limits_{n\in\mathbb N^\ast}\frac{1}{n}\sum\limits_{m>0}^n\qty(\alpha_{-n}^I\comm{\alpha^J_{n-m}}{L^\perp_m}+\alpha_{-n}^IL^\perp_m\alpha^J_{n-m}-\alpha_{-n}^J\comm{\alpha^I_{n-m}}{L^\perp_m}-\alpha_{-n}^JL^\perp_m\alpha^I_{n-m})\\
    &=-\frac{1}{2\alpha'{p^+}^2}\sum\limits_{n\in\mathbb N^\ast}\frac{1}{n}\sum\limits_{m>0}^n\qty(\alpha_{-n}^I\qty(n-m)\alpha^J_{n}+\alpha_{-n}^IL^\perp_m\alpha^J_{n-m}-\alpha_{-n}^J\qty(n-m)\alpha^I_n-\alpha_{-n}^JL^\perp_m\alpha^I_{n-m})\\
    &=-\frac{1}{2\alpha'{p^+}^2}\sum\limits_{n\in\mathbb N^\ast}\frac{1}{n}\qty(\alpha_{-n}^I\alpha^J_{n}-\alpha_{-n}^J\alpha^I_n)\sum\limits_{m>0}^n\qty(n-m)\\
    &\quad\quad\quad-\frac{1}{2\alpha'{p^+}^2}\sum\limits_{n\in\mathbb N^\ast}\frac{1}{n}\sum\limits_{m>0}^n\qty(\alpha_{-n}^IL^\perp_m\alpha^J_{n-m}-\alpha_{-n}^JL^\perp_m\alpha^I_{n-m})\\
    &=-\frac{1}{4\alpha'{p^+}^2}\sum\limits_{n\in\mathbb N^\ast}\qty(n-1)\qty(\alpha_{-n}^I\alpha^J_{n}-\alpha_{-n}^J\alpha^I_n)\\
    &\quad\quad\quad-\frac{1}{2\alpha'{p^+}^2}\sum\limits_{n\in\mathbb N^\ast}\frac{1}{n}\sum\limits_{m>0}^n\qty(\alpha_{-n}^IL^\perp_m\alpha^J_{n-m}-\alpha_{-n}^JL^\perp_m\alpha^I_{n-m})\\
    C_{22}^{IJ}&=-\frac{1}{4\alpha'{p^+}^2}\sum\limits_{n\in\mathbb N^\ast}\qty(n-1)\qty(\alpha_{-n}^I\alpha^J_{n}-\alpha_{-n}^J\alpha^I_n)\\
    &\quad\quad\quad-\frac{1}{2\alpha'{p^+}^2}\sum\limits_{n\in\mathbb N^\ast}\frac{1}{n}\sum\limits_{m\geq0}^{n-1}\qty(\alpha_{-n}^IL^\perp_{-m+n}\alpha^J_{m}-\alpha_{-n}^JL^\perp_{-m+n}\alpha^I_{m})\numberthis\label{c22}
\end{align*}

Ok. Let's sum all the three expressions, they are \ref{c11}, \ref{c12} and \ref{c22}, which by the definition \ref{c11def}, when summed should recover $\comm{C^I}{C^J}$, \ref{cicj},

\begin{align*}
    \comm{C^{I}}{C^{J}}&=-\frac{1}{4\alpha'{p^+}^2}\sum\limits_{n\in\mathbb N^\ast}\qty(n-1)\qty(\alpha^I_{-n}\alpha^J_n-\alpha^J_{-n}\alpha^I_n)\\
    &\quad\quad\quad-\frac{1}{2\alpha'{p^+}^2}\sum\limits_{n\in\mathbb N^\ast}\frac{1}{n}\sum\limits_{m\geq0}^{n-1}\qty(\alpha_{-m}^IL^\perp_{m-n}\alpha^J_n-\alpha^J_{-m}L^\perp_{m-n}\alpha^I_n)\\
    &\quad\quad\quad+\frac{1}{2\alpha'{p^+}^2}\sum\limits_{n\in\mathbb N^\ast}\frac1n\sum\limits_{m>0}^n\qty[\alpha^I_{-m}L^\perp_{-n+m}\alpha_n^J-\alpha^J_{-m}L^\perp_{-n+m}\alpha_n^I]\\
    &\quad\quad\quad+\frac{1}{2\alpha'{p^+}^2}\sum\limits_{m\in\mathbb N^\ast}\frac1m\sum\limits_{n>0}^m\qty[\alpha^I_{-m}L^\perp_{-n+m}\alpha_n^J-\alpha^J_{-m}L^\perp_{-n+m}\alpha_n^I]\\
    &\quad\quad\quad-\frac{1}{2\alpha'{p^+}^2}\sum\limits_{n,m\in\mathbb N^\ast}\left(\alpha^I_{-m-n}\alpha^J_{n+m}-\alpha^J_{-m-n}\alpha^I_{n+m}\right)\\
    &\quad\quad\quad+\frac{1}{2\alpha'{p^+}^2}\frac{D-2}{12}\sum\limits_{n\in\mathbb N^\ast}\frac{n^2-1}{n}\qty(\alpha^I_{-n}\alpha_n^J-\alpha^J_{-n}\alpha_n^I)\\
    &\quad\quad\quad-\frac{1}{4\alpha'{p^+}^2}\sum\limits_{n\in\mathbb N^\ast}\qty(n-1)\qty(\alpha_{-n}^I\alpha^J_{n}-\alpha_{-n}^J\alpha^I_n)\\
    &\quad\quad\quad-\frac{1}{2\alpha'{p^+}^2}\sum\limits_{n\in\mathbb N^\ast}\frac{1}{n}\sum\limits_{m\geq0}^{n-1}\qty(\alpha_{-n}^IL^\perp_{-m+n}\alpha^J_{m}-\alpha_{-n}^JL^\perp_{-m+n}\alpha^I_{m})
\end{align*}

The second and third line cancel each other apart from the $m=0$ and $m=n$ contributions, the same is true for the fourth and eight lines but now leaving the $m=0$ and $n=m$ terms. We also sum the 
first and seventh lines,

\begin{align*}
    \comm{C^{I}}{C^{J}}&=-\frac{1}{2\alpha'{p^+}^2}\sum\limits_{n\in\mathbb N^\ast}\qty(n-1)\qty(\alpha^I_{-n}\alpha^J_n-\alpha^J_{-n}\alpha^I_n)\\
    &\quad\quad\quad-\frac{1}{2\alpha'{p^+}^2}\sum\limits_{n\in\mathbb N^\ast}\frac{1}{n}\qty(\alpha_{0}^IL^\perp_{-n}\alpha^J_n-\alpha^J_{0}L^\perp_{0-n}\alpha^I_n)\\
    &\quad\quad\quad+\frac{1}{2\alpha'{p^+}^2}\sum\limits_{n\in\mathbb N^\ast}\frac1n\qty[\alpha^I_{-n}L^\perp_{0}\alpha_n^J-\alpha^J_{-n}L^\perp_{0}\alpha_n^I]\\
    &\quad\quad\quad+\frac{1}{2\alpha'{p^+}^2}\sum\limits_{m\in\mathbb N^\ast}\frac1m\qty[\alpha^I_{-m}L^\perp_{0}\alpha_m^J-\alpha^J_{-m}L^\perp_{0}\alpha_m^I]\\
    &\quad\quad\quad-\frac{1}{2\alpha'{p^+}^2}\sum\limits_{n,m\in\mathbb N^\ast}\left(\alpha^I_{-m-n}\alpha^J_{n+m}-\alpha^J_{-m-n}\alpha^I_{n+m}\right)\\
    &\quad\quad\quad+\frac{1}{2\alpha'{p^+}^2}\frac{D-2}{12}\sum\limits_{n\in\mathbb N^\ast}\frac{n^2-1}{n}\qty(\alpha^I_{-n}\alpha_n^J-\alpha^J_{-n}\alpha_n^I)\\
    &\quad\quad\quad-\frac{1}{2\alpha'{p^+}^2}\sum\limits_{n\in\mathbb N^\ast}\frac{1}{n}\qty(\alpha_{-n}^IL^\perp_{n}\alpha^J_{0}-\alpha_{-n}^JL^\perp_{n}\alpha^I_{0})
\end{align*}

Relabeling the forth line to $m\rightarrow n$, we can sum most of the lines,

\begin{align*}
    \comm{C^{I}}{C^{J}}&=-\frac{1}{2\alpha'{p^+}^2}\sum\limits_{n\in\mathbb N^\ast}\qty(n-1)\qty(\alpha^I_{-n}\alpha^J_n-\alpha^J_{-n}\alpha^I_n)\\
    &\quad\quad\quad-\frac{1}{2\alpha'{p^+}^2}\sum\limits_{n\in\mathbb N^\ast}\frac{1}{n}\qty(\alpha_{0}^IL^\perp_{-n}\alpha^J_n-\alpha^J_{0}L^\perp_{0-n}\alpha^I_n)\\
    &\quad\quad\quad+\frac{1}{ \alpha'{p^+}^2}\sum\limits_{n\in\mathbb N^\ast}\frac1n\qty[\alpha^I_{-n}L^\perp_{0}\alpha_n^J-\alpha^J_{-n}L^\perp_{0}\alpha_n^I]\\
    &\quad\quad\quad-\frac{1}{2\alpha'{p^+}^2}\sum\limits_{n,m\in\mathbb N^\ast}\left(\alpha^I_{-m-n}\alpha^J_{n+m}-\alpha^J_{-m-n}\alpha^I_{n+m}\right)\\
    &\quad\quad\quad+\frac{1}{2\alpha'{p^+}^2}\frac{D-2}{12}\sum\limits_{n\in\mathbb N^\ast}\frac{n^2-1}{n}\qty(\alpha^I_{-n}\alpha_n^J-\alpha^J_{-n}\alpha_n^I)\\
    &\quad\quad\quad-\frac{1}{2\alpha'{p^+}^2}\sum\limits_{n\in\mathbb N^\ast}\frac{1}{n}\qty(\alpha_{-n}^IL^\perp_{n}\alpha^J_{0}-\alpha_{-n}^JL^\perp_{n}\alpha^I_{0})\\
    \comm{C^{I}}{C^{J}}&=-\frac{1}{2\alpha'{p^+}^2}\sum\limits_{n\in\mathbb N^\ast}\frac{1}{n}\qty[\qty(L^\perp_{-n}\alpha^J_n-\alpha_{-n}L^\perp_n)\alpha_0^I-\qty(L^\perp_{0-n}\alpha^I_n-\alpha^I_{-n}L^\perp_n)\alpha^J_{0}]\\
    &\quad\quad\quad+\frac{1}{ \alpha'{p^+}^2}\sum\limits_{n\in\mathbb N^\ast}\frac1n\qty[\alpha^I_{-n}L^\perp_{0}\alpha_n^J-\alpha^J_{-n}L^\perp_{0}\alpha_n^I]\\
    &\quad\quad\quad-\frac{1}{2\alpha'{p^+}^2}\sum\limits_{n,m\in\mathbb N^\ast}\left(\alpha^I_{-m-n}\alpha^J_{n+m}-\alpha^J_{-m-n}\alpha^I_{n+m}\right)\\
    &\quad\quad\quad-\frac{1}{2\alpha'{p^+}^2}\sum\limits_{n\in\mathbb N^\ast}\qty(\alpha^I_{-n}\alpha^J_n-\alpha^J_{-n}\alpha^I_n)\qty[-\frac{D-2}{12}\frac{n^2-1}{n}+\qty(n-1)]\\
    \comm{C^{I}}{C^{J}}&=-\frac{1}{2\alpha'{p^+}^2}\sum\limits_{n\in\mathbb N^\ast}\frac{1}{n}\qty[\qty(L^\perp_{-n}\alpha^J_n-\alpha_{-n}L^\perp_n)\alpha_0^I-\qty(L^\perp_{0-n}\alpha^I_n-\alpha^I_{-n}L^\perp_n)\alpha^J_{0}]\\
    &\quad\quad\quad+\frac{1}{ \alpha'{p^+}^2}\sum\limits_{n\in\mathbb N^\ast}\frac1n\qty[\alpha^I_{-n}L^\perp_{0}\alpha_n^J-\alpha^J_{-n}L^\perp_{0}\alpha_n^I]\\
    &\quad\quad\quad-\frac{1}{2\alpha'{p^+}^2}\sum\limits_{n,m\in\mathbb N^\ast}\left(\alpha^I_{-m-n}\alpha^J_{n+m}-\alpha^J_{-m-n}\alpha^I_{n+m}\right)\\
    &\quad\quad\quad-\frac{1}{\alpha'{p^+}^2}\sum\limits_{n\in\mathbb N^\ast}\qty(\alpha^I_{-n}\alpha^J_n-\alpha^J_{-n}\alpha^I_n)\qty[n\qty(\frac12-\frac{D-2}{24})+\frac1n\qty(\frac{D-2}{24})-\frac12]
\end{align*}

The third line in this expression is a little bit off, it depends only in $n+m$, hence, we can change the two sums to just one sum over $u=n+m>1$, of course just doing this 
is not enough, because we would be undercounting as there are various $n,m$ such that $n+m=u>1$, in fact, this is a not hard to solve combinatorial problem, 
we have $n+m=u$ `objects', to subdivide into two, $n,m$. As each one has to be $n,m>0$, the number of `objects' are in fact $n+m-2$. It's simpler to solve visually 
by imagining we have a line of $n+m-2$ balls and some kind of wall separating them into two sets, the possibles arrangements are $\frac{\qty(n+m-2+1)!}{\qty(n+m-2)!1!}=n+m-1$, hence, 
we have the equality,

\begin{align*}
    \sum\limits_{n,m\in\mathbb N^\ast}f(n+m)=\sum\limits_{n>1}^\infty\qty(n-1)f(n)\equiv \sum\limits_{n\in\mathbb N^\ast}\qty(n-1)f(n)
\end{align*}

Using this fact in our expression get to us,

\begin{align*}
    \comm{C^{I}}{C^{J}}&=-\frac{1}{2\alpha'{p^+}^2}\sum\limits_{n\in\mathbb N^\ast}\frac{1}{n}\qty[\qty(L^\perp_{-n}\alpha^J_n-\alpha_{-n}L^\perp_n)\alpha_0^I-\qty(L^\perp_{0-n}\alpha^I_n-\alpha^I_{-n}L^\perp_n)\alpha^J_{0}]\\
    &\quad\quad\quad+\frac{1}{ \alpha'{p^+}^2}\sum\limits_{n\in\mathbb N^\ast}\frac1n\qty[\alpha^I_{-n}L^\perp_{0}\alpha_n^J-\alpha^J_{-n}L^\perp_{0}\alpha_n^I]\\
    &\quad\quad\quad-\frac{1}{2\alpha'{p^+}^2}\sum\limits_{n\in\mathbb N^\ast}\qty(n-1)\left(\alpha^I_{-n}\alpha^J_{n}-\alpha^J_{-n}\alpha^I_{n}\right)\\
    &\quad\quad\quad-\frac{1}{\alpha'{p^+}^2}\sum\limits_{n\in\mathbb N^\ast}\qty(\alpha^I_{-n}\alpha^J_n-\alpha^J_{-n}\alpha^I_n)\qty[n\qty(\frac12-\frac{D-2}{24})+\frac1n\qty(\frac{D-2}{24})-\frac12]\\
    \comm{C^{I}}{C^{J}}&=-\frac{1}{2\alpha'{p^+}^2}\sum\limits_{n\in\mathbb N^\ast}\frac{1}{n}\qty[\qty(L^\perp_{-n}\alpha^J_n-\alpha_{-n}L^\perp_n)\alpha_0^I-\qty(L^\perp_{0-n}\alpha^I_n-\alpha^I_{-n}L^\perp_n)\alpha^J_{0}]\\
    &\quad\quad\quad+\frac{1}{ \alpha'{p^+}^2}\sum\limits_{n\in\mathbb N^\ast}\frac1n\qty[\alpha^I_{-n}L^\perp_{0}\alpha_n^J-\alpha^J_{-n}L^\perp_{0}\alpha_n^I]\\
    &\quad\quad\quad-\frac{1}{\alpha'{p^+}^2}\sum\limits_{n\in\mathbb N^\ast}\qty(\alpha^I_{-n}\alpha^J_n-\alpha^J_{-n}\alpha^I_n)\qty[n\qty(1-\frac{D-2}{24})+\frac1n\qty(\frac{D-2}{24})-1]\numberthis\label{cicjfinal}
\end{align*}

Now, what remains is to sum all the three terms, namely, \ref{aicj}, \ref{bicj} and \ref{cicjfinal}, which, as we already had $\comm{A^I}{A^J}+\comm{B^I}{B^J}+\qty(\comm{A^I}{B^J}-\comm{B^I}{A^J})=0$, should 
recover the full commutator $\comm{M^{-I}}{M^{-J}}$,

\begin{align*}
    \comm{M^{-I}}{M^{-J}}&=\comm{A^I}{C^J}-\comm{A^J}{C^I}+\comm{B^I}{C^J}-\comm{C^I}{B^J}+\comm{C^I}{C^J}\\
    \comm{M^{-I}}{M^{-J}}&=\frac{1}{2\alpha'{p^+}^2}\sum\limits_{n\in\mathbb N^\ast}\frac1n\qty[\qty(L^\perp_{-n}\alpha_n^J-\alpha_{-n}^JL^\perp_{n})\alpha^I_0-\qty(L^\perp_{-n}\alpha_n^I-\alpha_{-n}^IL^\perp_{n})\alpha^J_0]\\
    &\quad\quad\quad-\frac{1}{\alpha'{p^+}^2}\sum\limits_{n\in\mathbb N^\ast}\frac1n\qty(\alpha_{-n}^I\qty(L_0^\perp +a+n)\alpha_n^J-\alpha_{-n}^J\qty(L^\perp_0+a+n)\alpha_n^I)\\
    &\quad\quad\quad-\frac{1}{2\alpha'{p^+}^2}\sum\limits_{n\in\mathbb N^\ast}\frac{1}{n}\qty[\qty(L^\perp_{-n}\alpha^J_n-\alpha_{-n}L^\perp_n)\alpha_0^I-\qty(L^\perp_{0-n}\alpha^I_n-\alpha^I_{-n}L^\perp_n)\alpha^J_{0}]\\
    &\quad\quad\quad+\frac{1}{ \alpha'{p^+}^2}\sum\limits_{n\in\mathbb N^\ast}\frac1n\qty[\alpha^I_{-n}L^\perp_{0}\alpha_n^J-\alpha^J_{-n}L^\perp_{0}\alpha_n^I]\\
    &\quad\quad\quad-\frac{1}{\alpha'{p^+}^2}\sum\limits_{n\in\mathbb N^\ast}\qty(\alpha^I_{-n}\alpha^J_n-\alpha^J_{-n}\alpha^I_n)\qty[n\qty(1-\frac{D-2}{24})+\frac1n\qty(\frac{D-2}{24})-1]\\
    \comm{M^{-I}}{M^{-J}}&=-\frac{1}{\alpha'{p^+}^2}\sum\limits_{n\in\mathbb N^\ast}\frac1n\qty(\alpha_{-n}^I\qty(a+n)\alpha_n^J-\alpha_{-n}^J\qty(a+n)\alpha_n^I)\\
    &\quad\quad\quad-\frac{1}{\alpha'{p^+}^2}\sum\limits_{n\in\mathbb N^\ast}\qty(\alpha^I_{-n}\alpha^J_n-\alpha^J_{-n}\alpha^I_n)\qty[n\qty(1-\frac{D-2}{24})+\frac1n\qty(\frac{D-2}{24})-1]\\
    \comm{M^{-I}}{M^{-J}}&=-\frac{1}{\alpha'{p^+}^2}\sum\limits_{n\in\mathbb N^\ast}\qty(\alpha^I_{-n}\alpha^J_n-\alpha^J_{-n}\alpha^I_n)\qty[n\qty(1-\frac{D-2}{24})+\frac1n\qty(\frac{D-2}{24}+a)]
\end{align*}

To impose $\comm{M^{-I}}{M^{-J}}=0$ is to impose the charges in the light-cone gauge satisfy the Lorentz algebra --- Although we didn't computed all the commutators, granted that this one is zero, all the others will be ---, that is,
the quantum theory possesses a global Lorentz invariance. Let's see what this constrain implies, first, all the $\alpha$s are independent operators, this means the expression has to vanish 
term by term, also, the combination of $\alpha$s in each term cannot be zero for all states of the Hilbert space, hence, we're forced to conclude that what should be zero is the prefactor,

\begin{align*}
    0&=n\qty(1-\frac{D-2}{24})+\frac1n\qty(\frac{D-2}{24}+a)\\
    0&=n^2\qty(1-\frac{D-2}{24})+\qty(a+\frac{D-2}{24})
\end{align*}

This has to vanish for every $n$, and both $D$ and $a$ cannot depend on $n$, thus, the coefficient of $n^2$ 
has to vanish,

\begin{align*}
    1-\frac{D-2}{24}&=0\\
    D&=26
\end{align*}

This implies,

\begin{align*}
    0&=a+\frac{D-2}{24}\\
    a&=-1
\end{align*}

That is, imposing Lorentz invariance fixes the normal-ordering constant and the space-time dimension!

\newpage

\section{Problem 5}
\subsection{5.A)}

We're assuming the dimension of the theory in consideration is $D=26$, so we'll assume the compactification is done in the $X^{D-1}=X^{25}$ target 
space direction. The usual mode expansion is,

\begin{align*}
    X^{25}\qty(\tau,\sigma)&=x_0^{25}+\sqrt{2\alpha'}\alpha_0^{25}\tau+\im\sqrt{\frac{\alpha'}{2}}\sum\limits_{n\in\mathbb Z^\ast}\frac{\e^{-\im n \tau}}{n}\qty(\alpha_n^{25}\e^{\im n\sigma}+{\bar\alpha}_n^{25}\e^{-\im n\sigma})\\
    UX^{25}\qty(\tau,\sigma)U^{-1}&=Ux_0^{25}U^{-1}+\sqrt{2\alpha'}U\alpha_0^{25}U^{-1}\tau+\im\sqrt{\frac{\alpha'}{2}}\sum\limits_{n\in\mathbb Z^\ast}\frac{\e^{-\im n \tau}}{n}\qty(U\alpha_n^{25}U^{-1}\e^{\im n\sigma}+U{\bar\alpha}_n^{25}U^{-1}\e^{-\im n\sigma})\\
    -X^{25}\qty(\tau,\sigma)&=Ux_0^{25}U^{-1}+\sqrt{2\alpha'}U\alpha_0^{25}U^{-1}\tau+\im\sqrt{\frac{\alpha'}{2}}\sum\limits_{n\in\mathbb Z^\ast}\frac{\e^{-\im n \tau}}{n}\qty(U\alpha_n^{25}U^{-1}\e^{\im n\sigma}+U{\bar\alpha}_n^{25}U^{-1}\e^{-\im n\sigma})\numberthis\label{ux}
\end{align*}

But, what is also true is,

\begin{align*}
    -X^{25}\qty(\tau,\sigma)&=-x_0^{25}+\sqrt{2\alpha'}\qty(-\alpha_0^{25})\tau+\im\sqrt{\frac{\alpha'}{2}}\sum\limits_{n\in\mathbb Z^\ast}\frac{\e^{-\im n \tau}}{n}\qty(-\alpha_n^{25}\e^{\im n\sigma}-{\bar\alpha}_n^{25}\e^{-\im n\sigma})\numberthis\label{-x}
\end{align*}

But as both \ref{ux} and \ref{-x} are different representations of the same operator, and, the decomposition in Fourier modes is unique, we 
must have an equality of the two equations term by term, that is,

\begin{align*}
    Ux_0^{25}U^{-1}=-x_0^{25},\ \ \ U{\bar\alpha}_n^{25}U^{-1}=-{\bar\alpha}_n^{25},\ \ \ U\alpha_n^{25}U^{-1}=-\alpha_n^{25}
\end{align*}

We can summarize the action of $U$ in all operators as,

\begin{align*}
    Ux_0^\mu U^{-1}=\qty(-1)^{g^{25 I}}x_0^\mu,\ \ \ U{\bar\alpha}_n^\mu U^{-1}=\qty(-1)^{g^{25 I}}{\bar\alpha}_n^\mu,\ \ \ U\alpha_n^\mu U^{-1}=\qty(-1)^{g^{25 I}}\alpha_n^\mu
\end{align*}

So that we can study whether of not the light-cone gauge Hamiltonian is invariant under this transformation,

\begin{align*}
    H&={\bar L}_0^\perp+L_0^\perp-2=\frac12{\bar\alpha}_0^I{\bar\alpha}_0^I+\frac12{\alpha}_0^I{\alpha}_0^I+\sum\limits_{n\in\mathbb N^\ast}{\bar\alpha}_{-n}^I{\bar\alpha}_{n}^I+\sum\limits_{n\in\mathbb N^\ast}{\alpha}_{-n}^I{\alpha}_{n}^I-2\\
    UHU^{-1}&=\frac12U{\bar\alpha}_0^IU^{-1}U{\bar\alpha}_0^IU^{-1}+\frac12U{\alpha}_0^IU^{-1}U{\alpha}_0^IU^{-1}\\
    &\quad\quad\quad+\sum\limits_{n\in\mathbb N^\ast}U{\bar\alpha}_{-n}^IU^{-1}U{\bar\alpha}_{n}^IU^{-1}+\sum\limits_{n\in\mathbb N^\ast}U{\alpha}_{-n}^IU^{-1}U{\alpha}_{n}^IU^{-1}-2\\
    UHU^{-1}&=\frac12\qty[(-1)^{g^{25I}}{\bar\alpha}_0^I]\qty[(-1)^{g^{25I}}{\bar\alpha}_0^I]+\frac12\qty[(-1)^{g^{25I}}{\alpha}_0^I]\qty[(-1)^{g^{25I}}{\alpha}_0^I]\\
    &\quad\quad\quad+\sum\limits_{n\in\mathbb N^\ast}\qty[(-1)^{g^{25I}}{\bar\alpha}_{-n}^I]\qty[(-1)^{g^{25I}}{\bar\alpha}_{n}^I]+\sum\limits_{n\in\mathbb N^\ast}\qty[(-1)^{g^{25I}}{\alpha}_{-n}^I]\qty[(-1)^{g^{25I}}{\alpha}_{n}^I]-2\\
    UHU^{-1}&=\frac12(-1)^{2g^{25I}}{\bar\alpha}_0^I{\bar\alpha}_0^I+\frac12(-1)^{2g^{25I}}{\alpha}_0^I{\alpha}_0^I\\
    &\quad\quad\quad+\sum\limits_{n\in\mathbb N^\ast}(-1)^{2g^{25I}}{\bar\alpha}_{-n}^I{\bar\alpha}_{n}^I+\sum\limits_{n\in\mathbb N^\ast}(-1)^{2g^{25I}}{\alpha}_{-n}^I{\alpha}_{n}^I-2\\
    UHU^{-1}&=\frac12(-1)^{2g^{25I}}{\bar\alpha}_0^I{\bar\alpha}_0^I+\frac12(-1)^{2g^{25I}}{\alpha}_0^I{\alpha}_0^I\\
    &\quad\quad\quad+\sum\limits_{n\in\mathbb N^\ast}(-1)^{2g^{25I}}{\bar\alpha}_{-n}^I{\bar\alpha}_{n}^I+\sum\limits_{n\in\mathbb N^\ast}(-1)^{2g^{25I}}{\alpha}_{-n}^I{\alpha}_{n}^I-2,\ \ \ 2g^{25 I}=\qty{0,2}\rightarrow \qty(-1)^{2g^{25I}}=1\\
    UHU^{-1}&=\frac12{\bar\alpha}_0^I{\bar\alpha}_0^I+\frac12{\alpha}_0^I{\alpha}_0^I+\sum\limits_{n\in\mathbb N^\ast}{\bar\alpha}_{-n}^I{\bar\alpha}_{n}^I+\sum\limits_{n\in\mathbb N^\ast}{\alpha}_{-n}^I{\alpha}_{n}^I-2\\
    UHU^{-1}&=H
\end{align*}

Hence, the Hamiltonian is invariant.

\subsection{5.B)}

The closed string vacuum is defined by $25$ numbers, namely, $q^+,q^I$, we'll reserve the name $p$ for the momentum operators, and use $q$ 
for the eigenvalues of those. In this way the vacuum satisfy,

\begin{align*}
    p^+\ket{q^+,q^I}=q^+\ket{q^+,q^I},\ \ \ p^J\ket{q^+,q^I}=q^J\ket{q^+,q^I},\ \ \ \alpha_{-n}^I\ket{q^+,q^I}=0,\ n\geq 1
\end{align*}

Let's look at which of those properties does the new state defined by the action of $U$ over the vacuum fails to met,

\begin{align*}
    p^+U\ket{q^+,q^I}=UU^{-1}p^+U\ket{q^+,q^I}=U\qty(-1)^{g25+}p^+\ket{q^+,q^I}=Uq^+\ket{q^+,q^I}=q^+U\ket{q^+,q^I}
\end{align*}

So it still an eigenvector of $p^+$ with eigenvalue $q^+$,

\begin{align*}
    p^JU\ket{q^+,q^I}&=UU^{-1}p^JU\ket{q^+,q^I}=U\qty(-1)^{g^{25J}}p^J\ket{q^+,q^I}\\
    p^JU\ket{q^+,q^I}&=\qty(-1)^{g^{25J}}Uq^J\ket{q^+,q^I}=\qty(-1)^{g^{25J}}q^JU\ket{q^+,q^I}
\end{align*}

That is, the state still a eigenvector of $p^J$, but, the presence of $\qty(-1)^{g^{25J}}$ makes the eigenvalue of the $p^{25}$ momentum operator to 
have the opposite sign then before, before concluding that this state is also a vacuum, we have to ensure that all the annihilation operators 
still annihilate it,

\begin{align*}
    \alpha_{-n}^JU\ket{q^+,q^I}=UU^{-1}\alpha_{-n}^JU\ket{q^+,q^I}=U\qty(-1)^{g^{25J}}\alpha_{-n}^J\ket{q^+,q^I}=0
\end{align*}

So now we can be sure that $U\ket{q^+,q^I}$ is a closed string vacuum with the same eigenvalues as the former, with exception of the $q^{25}$. This 
can be written in a simplified manner using the convention of upper case roman letters to non-light-cone gauge target space index $I=2,\cdots,25$, 
and lower case roman letters to non-compactified non-light-cone gauge target space index $i=2,\cdots,24$,

\begin{align*}
    U\ket{q^+,q^i,q^{25}}=\ket{q^+,q^i,-q^{25}}\numberthis\label{uvac}
\end{align*}

This has to be true as we shown that this new state has all the correct eigenvalues.

\subsection{5.C)}

We know the mass operator is,

\begin{align*}
    M^2&=\frac{2}{\alpha'}\qty(N^\perp+{\bar N}^\perp-2)
\end{align*}

Hence, for a state to have zero mass, it has to have eigenvalues $N^\perp+{\bar N}^\perp=2$, but, by the level matching, $N^\perp={\bar N}^\perp$, 
this fixes the spectrum as states with eigenvalues $N^\perp={\bar N}^\perp=1$, that is, is necessary to have exactly one $\alpha^I_{-1}$ and one 
${\bar \alpha}_{-1}^J$, so the most general state is given by a sum over linear combination of those,

\begin{align*}
    \ket{\xi}&=\int\limits_0^\infty\dd{q^+}\int\limits_{-\infty}^{+\infty}\dd[23]{q^i}\int\limits_{-\infty}^{+\infty}\dd{q^{25}}\xi_{IJ}\qty(q^+,q^i,q^{25})\alpha^I_{-1}{\bar\alpha}^J_{-1}\ket{q^+,q^i,q^{25}}\\
    U\ket{\xi}&=\int\limits_0^\infty\dd{q^+}\int\limits_{-\infty}^{+\infty}\dd[23]{q^i}\int\limits_{-\infty}^{+\infty}\dd{q^{25}}\xi_{IJ}\qty(q^+,q^i,q^{25})U\alpha^I_{-1}U^{-1}U{\bar\alpha}^J_{-1}U^{-1}U\ket{q^+,q^i,q^{25}}\\
    U\ket{\xi}&=\int\limits_0^\infty\dd{q^+}\int\limits_{-\infty}^{+\infty}\dd[23]{q^i}\int\limits_{-\infty}^{+\infty}\dd{q^{25}}\xi_{IJ}\qty(q^+,q^i,q^{25})\qty(-1)^{g^{25I}+g^{25J}}\alpha^I_{-1}{\bar\alpha}^J_{-1}\ket{q^+,q^i,-q^{25}}\\
    U\ket{\xi}&=\int\limits_0^\infty\dd{q^+}\int\limits_{-\infty}^{+\infty}\dd[23]{q^i}\int\limits_{-\infty}^{+\infty}\dd{\qty(-q^{25})}\xi_{IJ}\qty(q^+,q^i,q^{25})\qty(-1)^{g^{25I}+g^{25J}+1}\alpha^I_{-1}{\bar\alpha}^J_{-1}\ket{q^+,q^i,-q^{25}}\\
    U\ket{\xi}&=\int\limits_0^\infty\dd{q^+}\int\limits_{-\infty}^{+\infty}\dd[23]{q^i}\int\limits_{-\infty}^{+\infty}\dd{q^{25}}\xi_{IJ}\qty(q^+,q^i,-q^{25})\qty(-1)^{g^{25I}+g^{25J}+1}\alpha^I_{-1}{\bar\alpha}^J_{-1}\ket{q^+,q^i,q^{25}}
\end{align*}

\subsection{5.D)}
\subsection{5.E)}

\newpage

\problem{}
\probitem{}

We have already done this for general $N$ real fermions in problem \ref{3d}. Let us just cite here the 
results,

\begin{align*}
    j^{\qty[ab]}\qty(z)&=\frac1 2\tensor{T}{^{\qty[ab]}_k_l}\cnord{\psi^k\psi^l}\qty(z)\\
    j^A\qty(z)&=\frac1 2\tensor{T}{^A_k_l}\cnord{\psi^k\psi^l}\qty(z)\\
    j^A\qty(z_1)j ^B\qty(z_2)&=\frac{\im\tensor{f}{^A^B_C}}{\qty(z_1-z_2)}j^C\qty(z_2)+\frac{\Tr\qty[T^A T^B]}{2\qty(z_1-z_2)^2}+\textnormal{regular}
\end{align*}

Where in this case $\tensor{T}{^{\qty[ab]}_k_l}=\tensor{T}{^A_k_l}$ are the generators of $SO\qty(4)$, and $\tensor{f}{^A^B_C}$ are the structure constants such 
that $a,b,k,l=1,2,3,4$ and $A,B,C=1,2,3,4,5,6$.

\probitem{}

To keep the conventions already established in problems \ref{3c},\ref{4c}, we're going to choose the following 
for grouping the four fermions into a pair of complex ones,

\begin{align*}
    {\Psi}^1=\frac{1}{\sqrt{2}}\psi^1-\frac{\im}{\sqrt2}\psi^2\\
    {\tilde\Psi}^1=\frac{1}{\sqrt{2}}\psi^1+\frac{\im}{\sqrt2}\psi^2\\
    {\Psi}^2=\frac{1}{\sqrt{2}}\psi^3-\frac{\im}{\sqrt2}\psi^4\\
    {\tilde\Psi}^2=\frac{1}{\sqrt{2}}\psi^3+\frac{\im}{\sqrt2}\psi^4
\end{align*}

This choice makes the following,


\begin{align*}
    -\cnord{{\tilde\Psi}^1\Psi^1}&=-\frac12\cnord{\psi^1\psi^1+\im\psi^2\psi^1-\psi^1\im\psi^2+\psi^2\psi^2}\\
    -\cnord{{\tilde\Psi}^1\Psi^1}&=\im\cnord{\psi^1\psi^2}
\end{align*}

As outcome of problems \ref{3c},\ref{4c} this should be seen as a good relation to hold. To bosonize each pair ${\tilde\Psi}^{\dot a}\Psi^{\dot a}$ --- We'll use $\dot a,\dot b=1,2$ for the complex fermions we 
leave $j,k,l=1,2,3,4$ for the real ones ---, 
we follow the ideas of problem \ref{4}, to each pair of complex fermions we attribute a chiral boson $X^{\dot a}\qty(z)$, then 
we should have, as was already argued in problem \ref{4d}, the following correspondence,

\begin{subequations}\label{teste1}
\begin{align}
    \cnord{\exp\qty(\im X^{\dot a})}&=\im \Psi^{\dot a}\\
    \cnord{\exp\qty(-\im X^{\dot a})}&=\im {\tilde\Psi}^{\dot a}
\end{align}
\end{subequations}

We choose a particular way of bosonizing, a more generic choice could be $\cnord{\exp\qty(\im X^{\dot a})}=\alpha \Psi^{\dot a},\cnord{\exp\qty(-\im X^{\dot a})}=-\alpha^{-1} {\tilde\Psi}^{\dot a}$. 
But what matters is that now we have a description of the fermionic theory by a bosonic one, consisting of two compactified chiral bosons $X^{\dot a}\qty(z)=X^{\dot a}\qty(z)+2\pi$. 

\probitem{}

The symmetry of the bosonic theory is,

\begin{align*}
    X^{\dot a}\qty(z)\rightarrow X^{\dot a}\qty(z)+t^{\dot a}\ \qty(\textnormal{mod }2\pi)
\end{align*}

What from \ref{teste1} can be seen as a realization of a $U\qty(1)$ symmetry for each index $\dot a$, as there is 
two of them, and the symmetries are disconnected, this is a $U\qty(1)\times U\qty(1)$,

\begin{align*}
    \im\Psi^{\dot a}=\cnord{\exp\qty(\im X^{\dot a})}&\rightarrow\cnord{\exp\qty(\im X^{\dot a}+\im t^{\dot a})},\ \ \ t^{\dot a}\in[0,2\pi)\\
    \im\Psi^{\dot a}&\rightarrow\exp\qty(\im t^{\dot a})\cnord{\exp\qty(\im X^{\dot a})},\ \ \ t^{\dot a}\in[0,2\pi)\\
    \im\Psi^{\dot a}&\rightarrow\exp\qty(\im t^{\dot a})\im\Psi^{\dot a},\ \ \ t^{\dot a}\in[0,2\pi)\\
    \Psi^{\dot a}&\rightarrow\exp\qty(\im t^{\dot a})\Psi^{\dot a},\ \ \ t^{\dot a}\in[0,2\pi)
\end{align*}

From the last line is clear that we have a $U\qty(1)$ symmetry for each index, specially due to the chiral boson being 
compactified. Of course, this transformation above implies the transformation of the complex one also,

\begin{align*}
    \im{\tilde\Psi}^{\dot a}=\cnord{\exp\qty(-\im X^{\dot a})}&\rightarrow\cnord{\exp\qty(-\im X^{\dot a}+\im t^{\dot a})},\ \ \ t^{\dot a}\in[0,2\pi)\\
    \im{\tilde\Psi}^{\dot a}&\rightarrow\exp\qty(-\im t^{\dot a})\cnord{\exp\qty(-\im X^{\dot a})},\ \ \ t^{\dot a}\in[0,2\pi)\\
    \im{\tilde\Psi}^{\dot a}&\rightarrow\exp\qty(-\im t^{\dot a})\im{\tilde\Psi}^{\dot a},\ \ \ t^{\dot a}\in[0,2\pi)\\
    {\tilde\Psi}^{\dot a}&\rightarrow\exp\qty(-\im t^{\dot a}){\tilde\Psi}^{\dot a},\ \ \ t^{\dot a}\in[0,2\pi)
\end{align*}

So now it's clear that ${\tilde\Psi}^{\dot a}$ have charge $-1$ with the respective $U\qty(1)$'s, and $\Psi^{\dot a}$ have charge $+1$ with the 
respective $U\qty(1)$'s. What we have shown is that $U\qty(1)\times U\qty(1)\subset SO\qty(4)$, so it's possible for us to work out the action of each of these $U\qty(1)$'s over the vector representation, 
let's name $U\qty(1)\times U\qty(1)$ by $U\qty(1)_{\dot 1}\times U\qty(1)_{\dot 2}$, the nomenclature should be self evident.

\newpage

\appendix

\section{Bosonic CFT}

Here we're going to derive some results of a bosonic CFT of $X^\mu$,

\begin{align*}
    S&=\frac{1}{4\pi\alpha'}\int\dd[2]{\sigma}\partial_a X^\mu\partial^a X_\mu
\end{align*}

Where it's understood a flat `\textit{world-sheet}' metric $\delta_{ab}$, we make a change of variables from $\sigma^1,\sigma^2$ to 
$\sigma^z\equiv z=\sigma^1+\im \sigma^2,\sigma^{\bar z}\equiv \bar z=\sigma^1-\im \sigma^2$, so that the new metric reads,

\begin{align*}
    g^{zz}&=\pdv{z}{\sigma^1}\pdv{z}{\sigma^1}\delta^{11}+\pdv{z}{\sigma^2}\pdv{z}{\sigma^2}\delta^{22}\\
    g^{zz}&=1-1=0\\
    g^{z\bar z}&=\pdv{z}{\sigma^1}\pdv{\bar z}{\sigma^1}\delta^{11}+\pdv{z}{\sigma^2}\pdv{\bar z}{\sigma^2}\delta^{22}\\
    g^{z\bar z}&=1+1=2=g^{\bar zz}\\
    g^{\bar z \bar z}&=\pdv{\bar z}{\sigma^1}\pdv{\bar z}{\sigma^1}\delta^{11}+\pdv{\bar z}{\sigma^2}\pdv{\bar z}{\sigma^2}\delta^{22}\\
    g^{\bar z \bar z}&=1-1=0
\end{align*}

So that or action in these new coordinates is,

\begin{align*}
    S&=\frac{1}{4\pi\alpha'}\int\dd{\sigma^1}\dd{\sigma^2}\sqrt{\abs{\Det\qty[\delta]}}\partial_a X^\mu\partial^a X_\mu\\
    S&=\frac{1}{4\pi\alpha'}\int\dd{z}\dd{\bar z}\sqrt{\abs{\Det\qty[g]}}\partial_a X^\mu\partial^a X_\mu\\
    S&=\frac{1}{4\pi\alpha'}\frac12\int\dd{z}\dd{\bar z}\qty(g^{z\bar z}\partial_z X^\mu\partial_{\bar z} X_\mu+g^{z\bar z}\partial_{\bar z} X^\mu\partial_{z} X_\mu)\\
    S&=\frac{1}{4\pi\alpha'}\frac42\int\dd{z}\dd{\bar z}\partial_z X^\mu\partial_{\bar z} X_\mu\\
    S&=\frac{1}{2\pi\alpha'}\int\dd[2]{z}\partial X^\mu\bar\partial X_\mu
\end{align*}

Where we defined $\partial_z=\partial,\partial_{\bar z}=\bar \partial$ and $\dd[2]{z}=\dd{z}\dd{\bar z}$

\end{document}