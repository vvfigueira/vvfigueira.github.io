\problem{}
\probitem{}

Until now we have been using a notion of normal ordering, $\cnord{A}$ which is heavily dependent on knowing the equation of motion 
of the operators to construct the normal ordered form, but, of course, any ordering scheme which gives a finite operator is equally good, and 
as all of them give finite operators they differ only by finite parts. Another possible notion of ordering, different from what we have been 
using, is a definition which makes no reference of the equation of motion, and relies solely in removing the divergence from the operator,

\begin{align*}
    \snord{\mathcal O_1\mathcal O_2}\qty(z)&=\oint\frac{\dd{w}}{2\pi\im}\frac{\mathcal O_1\qty(w)\mathcal O_2\qty(z)}{w-z}
\end{align*}

It's clear that the contour integral picks up just the finite part of $\mathcal O_1\qty(w)\mathcal O_2\qty(z)$ evaluated at $z=w$. But, it has 
two drawbacks, it can only define the normal ordering at equal points, while our other definition can do it at different points, and it requires 
the knowledge of another OPE $\mathcal O_1\qty(w)\mathcal O_2\qty(z)$ to be computed, which has to be calculated from other arguments. But still 
it provides a classy way of writing the normal ordering. Now focusing in the problem at hand, we have already constructed the current for 
general $SO\qty(N)$ groups, \ref{jabcurrent}, which we recall here,

\begin{align*}
    j^{\qty[ab]}\qty(z)&=\frac1 2\tensor{T}{^{\qty[ab]}_k_l}\cnord{\psi^k\psi^l}\qty(z)
\end{align*}

For the $N=3$, $SO\qty(3)$ case we have only three currents,

\begin{align*}
    j^{12}=j^{\qty[12]}\qty(z)&=\frac1 2\tensor{T}{^{\qty[12]}_k_l}\cnord{\psi^k\psi^l}\qty(z)\\
    j^{23}=j^{\qty[23]}\qty(z)&=\frac1 2\tensor{T}{^{\qty[23]}_k_l}\cnord{\psi^k\psi^l}\qty(z)\\
    j^{31}=j^{\qty[31]}\qty(z)&=\frac1 2\tensor{T}{^{\qty[31]}_k_l}\cnord{\psi^k\psi^l}\qty(z)
\end{align*}

The generators in the fundamental representation are easily written in terms of the Levi-Civita, $\epsilon_{123}=1$,

\begin{align*}
    \tensor{T}{^{\qty[ab]}_k_l}&=\im\tensor{\epsilon}{^a^b^c}\tensor{\epsilon}{_c_k_l}\Rightarrow\begin{cases}
        \tensor{T}{^{\qty[12]}_k_l}&=\im\epsilon_{3kl}\\
        \tensor{T}{^{\qty[23]}_k_l}&=\im\epsilon_{1kl}\\
        \tensor{T}{^{\qty[31]}_k_l}&=\im\epsilon_{2kl}
    \end{cases}
\end{align*}

So that,

\begin{align*}
    j^{12}&=\im\frac1 2\epsilon_{3kl}\cnord{\psi^k\psi^l}\qty(z)=\im\cnord{\psi^1\psi^2}\qty(z)\\
    j^{23}&=\im\frac1 2\epsilon_{1kl}\cnord{\psi^k\psi^l}\qty(z)=\im\cnord{\psi^2\psi^3}\qty(z)\\
    j^{31}&=\im\frac1 2\epsilon_{2kl}\cnord{\psi^k\psi^l}\qty(z)=\im\cnord{\psi^3\psi^1}\qty(z)
\end{align*}

One might worry about these currents being defined through the \textit{old} normal ordering $\cnord{}$, but actually, the expression 
for the currents need not ordering prescription, as they're totally finite without normal ordering due to the OPE of two different fermions 
not having poles, as they do not mix kinetic terms, so there is no problem here in mixing the two kinds of orderings.

\probitem{}

With the remarks being made, we make use of the full OPE,

\begin{align*}
    \psi^i\qty(z)\psi^j\qty(w)=\frac{\delta^{ij}}{z-w}+\cnord{\psi^i\qty(z)\psi^j\qty(w)}
\end{align*}

And star by computing just $\snord{j^{12}j^{23}}\qty(z)$,

\begin{align*}
    \snord{j^{12}j^{23}}\qty(z)&=\oint\frac{\dd{w}}{2\pi\im}\frac{j^{12}\qty(w)j^{23}\qty(z)}{w-z}\\
    \snord{j^{12}j^{23}}\qty(z)&=-\oint\frac{\dd{w}}{2\pi\im}\frac{\psi^1\qty(w)\psi^2\qty(w)\psi^2\qty(z)\psi^3\qty(z)}{w-z}\\
    \snord{j^{12}j^{23}}\qty(z)&=-\oint\frac{\dd{w}}{2\pi\im}\frac{\psi^1\qty(w)\psi^3\qty(z)\psi^2\qty(w)\psi^2\qty(z)}{w-z}
\end{align*}

Where we used that for $i\neq j,\ \acomm{\psi^i\qty(z)}{\psi^j\qty(w)}=0$. Now we expand in Taylor the regular term $\psi^1\qty(w)\psi^3\qty(z)$, 

\begin{align*}
    \snord{j^{12}j^{23}}\qty(z)&=-\oint\frac{\dd{w}}{2\pi\im}\frac{\psi^1\qty(z)\psi^3\qty(z)+\qty(w-z)\partial\psi^1\qty(z)\psi^3\qty(z)+\mathcal O\qty(\qty(w-z)^2)}{w-z}\psi^2\qty(w)\psi^2\qty(z)
\end{align*}

As the $\psi^2\qty(w)\psi^2\qty(z)$ OPE has at most a single pole, the contour integral of the $\mathcal O\qty(\qty(w-z)^2)$ contribution 
gives identically zero, as we have,

\begin{align*}
    \oint\frac{\dd{w}}{2\pi\im}\frac{\mathcal O\qty(\qty(w-z) ^2)}{\qty(w-z)^2}&=\oint\frac{\dd{w}}{2\pi\im}\mathcal O\qty(1)=0
\end{align*}

As it has no pole, so, we can in fact neglect it, rewriting the integral without this term and already opening the OPE,

\begin{align*}
    \snord{j^{12}j^{23}}\qty(z)&=-\oint\frac{\dd{w}}{2\pi\im}\frac{\psi^1\qty(z)\psi^3\qty(z)+\qty(w-z)\partial\psi^1\qty(z)\psi^3\qty(z)}{w-z}\qty(\frac{1}{w-z}+\cnord{\psi^2\qty(w)\psi^2\qty(z)})\\
    \snord{j^{12}j^{23}}\qty(z)&=-\oint\frac{\dd{w}}{2\pi\im}\qty{\frac{\psi^1\qty(z)\psi^3\qty(z)}{\qty(w-z)^2}+\frac{\partial\psi^1\qty(z)\psi^3\qty(z)}{w-z}+\frac{\psi^1\qty(z)\psi^3\qty(z)\cnord{\psi^2\qty(w)\psi^2\qty(z)}}{w-z}}\\
    &\quad\quad\quad-\oint\frac{\dd{w}}{2\pi\im}\partial\psi^1\qty(z)\psi^3\qty(z)\cnord{\psi^2\qty(w)\psi^2\qty(z)}
\end{align*}

The first term in the right-hand side has only a double pole, so, the integral vanishes, as the integral captures just single poles, so do the 
last term in the right-hand side, as it's regular at $w=z$, the second term has a single pole, which contributes as $1$, and for the third term we have to expand it in Taylor,

\begin{align*}
    \snord{j^{12}j^{23}}\qty(z)&=-\partial\psi^1\qty(z)\psi^3\qty(z)-\psi^1\qty(z)\psi^3\qty(z)\oint\frac{\dd{w}}{2\pi\im}\frac{\cnord{\psi^2\qty(z)\psi^2\qty(z)}+\qty(w-z)\cnord{\partial\psi^2\qty(z)\psi^2\qty(z)}}{w-z}
\end{align*}

The first term of the integral is zero by statistics, $\cnord{\psi^2\qty(z)\psi^2\qty(z)}=0$, and the second one is zero due to being regular,

\begin{align*}
    \snord{j^{12}j^{23}}\qty(z)&=-\partial\psi^1\qty(z)\psi^3\qty(z)
\end{align*}

Now for the other component,

\begin{align*}
    \snord{j^{23}j^{12}}\qty(z)&=\oint\frac{\dd{w}}{2\pi\im}\frac{j^{23}\qty(w)j^{12}\qty(z)}{w-z}\\
    \snord{j^{23}j^{12}}\qty(z)&=-\oint\frac{\dd{w}}{2\pi\im}\frac{\psi^2\qty(w)\psi^3\qty(w)\psi^1\qty(z)\psi^2\qty(z)}{w-z}\\
    \snord{j^{23}j^{12}}\qty(z)&=-\oint\frac{\dd{w}}{2\pi\im}\frac{\psi^3\qty(w)\psi^1\qty(z)}{w-z}\psi^2\qty(w)\psi^2\qty(z)\\
    \snord{j^{23}j^{12}}\qty(z)&=-\oint\frac{\dd{w}}{2\pi\im}\frac{\psi^3\qty(z)\psi^1\qty(z)+\qty(w-z)\partial\psi^3\qty(z)\psi^1\qty(z)}{w-z}\qty(\frac{1}{w-z}+\cnord{\psi^2\qty(w)\psi^2\qty(z)})\\
    \snord{j^{23}j^{12}}\qty(z)&=-\oint\frac{\dd{w}}{2\pi\im}\partial\psi^3\qty(z)\psi^1\qty(z)\qty(\frac{1}{w-z}+\cnord{\psi^2\qty(w)\psi^2\qty(z)})\\
    \snord{j^{23}j^{12}}\qty(z)&=-\partial\psi^3\qty(z)\psi^1\qty(z)
\end{align*}

Where we used the same arguments of before, keeping only the single poles, and the terms which does not vanishes by statistics. Hence,

\begin{align*}
    \snord{j^{12}j^{23}}\qty(z)-\snord{j^{23}j^{12}}\qty(z)&=-\partial\psi^1\qty(z)\psi^3\qty(z)+\partial\psi^3\qty(z)\psi^1\qty(z)\\
    \snord{j^{12}j^{23}}\qty(z)-\snord{j^{23}j^{12}}\qty(z)&=\psi^3\qty(z)\partial\psi^1\qty(z)+\partial\psi^3\qty(z)\psi^1\qty(z)\\
    \snord{j^{12}j^{23}}\qty(z)-\snord{j^{23}j^{12}}\qty(z)&=\partial\qty[\psi^3\qty(z)\psi^1\qty(z)]\\
    \snord{j^{12}j^{23}}\qty(z)-\snord{j^{23}j^{12}}\qty(z)&=-\im\partial j^{31}\qty(z)
\end{align*}