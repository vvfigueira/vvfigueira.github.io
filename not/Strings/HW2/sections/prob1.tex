\problem{}
\probitem{}

Let $M$ be our $D>2$ dimensional $C^\infty$ manifold and $\phi:\mathbb R\times M\rightarrow M$ be a one parameter family of diffeomorphisms, that is, $\forall t\in\mathbb R|\ \phi_t:M\rightarrow M$ is a diffeomorphism such that, 

\begin{itemize}
    \item $\forall p\in M|\ \phi_0\qty(p)=p$
    \item $\forall p\in M; \forall t,s\in\mathbb R|\ \phi_{t+s}\qty(p)=\qty(\phi_t\circ\phi_s)\qty(p)$
    \item $\forall p\in M|\ \phi\qty(p):\mathbb R\rightarrow M$ is at least $C^1$
\end{itemize}

Then this family of diffeomorphisms define in a natural manner a vector which generate these transformations, 
we define at each point $p\in M$ this vector by it's action in a function $f:M\rightarrow \mathbb R$,

\begin{align*}
    \xi_p\qty(f)&=\dv{}{t}\qty(\qty(f\circ\phi_t)\qty(p))\eval_{t=0}
\end{align*}

And from this we define $\xi$ as a vector field in $M$, this vector field has as integral curves exactly $\phi$. Let's 
open a little bit more in some chart $x:M\rightarrow \mathbb R^D$,

\begin{align*}
    \xi_p\qty(f)&=\dv{}{t}\qty(\qty(f\circ x^{-1}\circ x\circ\phi_t)\qty(p))\eval_{t=0}\\
    \xi_p\qty(f)&=\dv{}{t}\qty(\qty(f\circ x^{-1})\circ\qty(x\circ\phi_t)\qty(p))\eval_{t=0}\\
    \xi_p\qty(f)&=\partial_\mu\qty(f\circ x^{-1})\eval_{x\circ\phi_0\qty(p)}\dv{}{t}\qty(\qty(x\circ\phi_t)^\mu\qty(p))\eval_{t=0}\\
    \xi_p^\mu\partial_\mu\qty(f\circ x^{-1})\eval_{x\qty(p)}&=\partial_\mu\qty(f\circ x^{-1})\eval_{x\qty(p)}\dv{}{t}\qty(\qty(x\circ\phi_t)^\mu\qty(p))\eval_{t=0}
\end{align*}

Where of curse $\partial_\mu$ is to be interpreted as the derivative of the $\mu$-th component in the chart $x$. Here we have a clear definition of the 
values of the $\xi$ vector field in a chart $x$,

\begin{align*}
    \xi_p^\mu&=\dv{}{t}\qty(\qty(x\circ\phi_t)^\mu\qty(p))\eval_{t=0}
\end{align*}

The term inside the derivative is just the pullback of the chart $x$ --- the chart can be seen as a $\mathbb R^D$-valued function ---, which 
in it's own can be seen as a new chart $x'_t$ defined by the transformations of the diffeomorphism family $\phi$, that is,

\begin{align*}
    x'_t=\phi^\ast_tx=x\circ\phi_t:M\rightarrow \mathbb R^D
\end{align*}

All of this is consistent with our interpretation of the diffeomorphisms being a `\textit{coordinate change}', in principle, with 
enough derivability of $\phi$ we can actually write,

\begin{align*}
    x'_t&=x'_0+t\dv{}{t}\qty(x'_t)\eval_{t=0}+\mathcal O\qty(t^2)\\
    {x'_1}^\mu\eqcolon {x'}^\mu&={x}^\mu+\xi^\mu+\cdots
\end{align*}

We just restored the index to not confuse the components of the vector field $\xi$ in the basis $x$ with the vector field itself. 
That is, we showed that the transformation done by $\phi_1$ is equivalent to a `\textit{infinitesimal coordinate change}' by $\xi^\mu$. 
Actually, all this we did is the special case of a more general type of derivative, the Lie Derivative, given a vector field $\xi$ and it's 
family of integral curves $\phi$, it's defined in terms of the pushforward of the object under analysis,

\begin{align*}
    \pounds_\xi T=\dv{}{t}\qty(\phi_{-t\ast}T)\eval_{t=0}
\end{align*}

For a $\qty(0,2)$ tensor, that is, for the metric,

\begin{align*}
    \pounds_\xi g_{\mu\nu}=\nabla_\xi g_{\mu\nu}+g_{\mu\alpha}\nabla_\nu\xi^\alpha+g_{\alpha\nu}\nabla_\mu\xi^\alpha
\end{align*}

And as the connection is metric compatible,

\begin{align*}
    \pounds_\xi g_{\mu\nu}=\nabla_\nu\xi_\mu+\nabla_\mu\xi_\nu=2\nabla_{(\mu}\xi_{\nu)}
\end{align*}

This amounts for the first term in an expansion in $t$ of the diffeomorphism transformed metric $\phi_{-t\ast}g=g'_{t\mu\nu}$, that is, 

\begin{align*}
    \phi_{-t\ast}g_{\mu\nu}\eqcolon g'_{t\mu\nu}&=g_{\mu\nu}+t\pounds _\xi g_{\mu\nu}+\mathcal O\qty(t^2)\\
    g'_{t\mu\nu}&=g_{\mu\nu}+2t\nabla_{(\mu}\xi_{\nu)}+\mathcal O\qty(t^2)
\end{align*}

Which also can be interpreted as the `\textit{infinitesimal transformation}' of the metric. Imposing that the initial and transformed metric are conformally flat,

\begin{align*}
    \exp\qty(2\omega'_t)\eta_{\mu\nu}&=\exp\qty(2\omega)\eta_{\mu\nu}+2t\nabla_{(\mu}\xi_{\nu)}+\mathcal O\qty(t^2)\\
    \exp\qty(2\omega'_0+2t\dv{}{t}\qty(\omega'_t)\eval_{t=0}+\mathcal O\qty(t^2))\eta_{\mu\nu}&=\exp\qty(2\omega)\eta_{\mu\nu}+2t\nabla_{(\mu}\xi_{\nu)}+\mathcal O\qty(t^2)\\
    \exp\qty(2\omega)\exp\qty(2t\dv{}{t}\qty(\omega'_t)\eval_{t=0}+\mathcal O\qty(t^2))\eta_{\mu\nu}&=\exp\qty(2\omega)\eta_{\mu\nu}+2t\nabla_{(\mu}\xi_{\nu)}+\mathcal O\qty(t^2)\\
    \exp\qty(2\omega)\qty(1+2t\dv{}{t}\qty(\omega'_t)\eval_{t=0}+\mathcal O\qty(t^2))\eta_{\mu\nu}&=\exp\qty(2\omega)\eta_{\mu\nu}+2t\nabla_{(\mu}\xi_{\nu)}+\mathcal O\qty(t^2)\\
    \exp\qty(2\omega)2t\dv{}{t}\qty(\omega'_t)\eval_{t=0}\eta_{\mu\nu}&=2t\nabla_{(\mu}\xi_{\nu)}\\
    \dv{}{t}\qty(\omega'_t)\eval_{t=0}g_{\mu\nu}&=\nabla_{(\mu}\xi_{\nu)}
\end{align*}

The term $\dv{}{t}\qty(\omega '_t)\eval_{t=0}$ is fully determined by $\xi$, to see this just contract both sides with the metric,

\begin{align*}
    \dv{}{t}\qty(\omega'_t)\eval_{t=0}g^{\mu\nu}g_{\mu\nu}&=g^{\mu\nu}\nabla_{(\mu}\xi_{\nu)}\\
    \dv{}{t}\qty(\omega'_t)\eval_{t=0}D&=\nabla_{\mu}\xi^{\mu}
\end{align*}

Substituting this back in the original equation,

\begin{align*}
    \nabla_\alpha \xi^\alpha g_{\mu\nu}&=D\nabla_{(\mu}\xi_{\nu)}\numberthis\label{conformaldefeq}
\end{align*}

This is the condition upon $\xi$ that ensures the diffeomorphism maintain the conformally flatness of the metric. To solve it 
in it's full generality is hard, so, we'll make some use of symmetry, first, let's write it again in a coordinate free form,

\begin{align*}
    \textnormal{Div}_g\qty(\xi)g=\frac D2\pounds_\xi g\numberthis\label{confnocoor}
\end{align*}

Where $\textnormal{Div}_g\qty(\xi)$ is just $\nabla_\alpha\xi^\alpha$, but with the connection $\nabla$ defined with respect with the metric $g$, now, suppose we have a function $f:M\rightarrow \mathbb R$, 
define $\tilde g =\exp\qty(2f)g$, thus,

\begin{align*}
    \pounds_\xi \tilde g&=\pounds _\xi\qty(\exp\qty(2f)g)\\
    \pounds_\xi \tilde g&=\exp\qty(2f)\pounds _\xi g+g\pounds _\xi \exp\qty(2f)\\
    \pounds_\xi \tilde g&=\exp\qty(2f)\pounds _\xi g+g\xi\qty(\exp\qty(2f))\\
    \pounds_\xi \tilde g&=\exp\qty(2f)\pounds _\xi g+g\xi\qty(f)\exp\qty(2f)\\
    \pounds_\xi \tilde g&=\exp\qty(2f)\pounds _\xi g+\tilde g\xi\qty(2f)\numberthis\label{lievers}
\end{align*}

Also,

\begin{align*}
    \textnormal{Div}_g\qty(\xi)&=\partial_\alpha\xi^\alpha+\Gamma^\alpha_{\alpha\lambda}\xi^\lambda\\
    \textnormal{Div}_g\qty(\xi)&=\partial_\alpha\xi^\alpha+\frac12g^{\alpha\beta}\qty(\partial_\alpha g_{\beta\lambda}+\partial_\lambda g_{\beta\alpha}-\partial_\beta g_{\alpha\lambda})\xi^\lambda\\
    \textnormal{Div}_g\qty(\xi)&=\partial_\alpha\xi^\alpha+g^{\alpha\beta}\qty(\partial_\alpha \omega g_{\beta\lambda}+\partial_\lambda \omega g_{\beta\alpha}-\partial_\beta \omega g_{\alpha\lambda})\xi^\lambda\\
    \textnormal{Div}_g\qty(\xi)&=\partial_\alpha\xi^\alpha+\qty(\partial_\lambda \omega +D\partial_\lambda \omega-\partial_\lambda \omega )\xi^\lambda\\
    \textnormal{Div}_g\qty(\xi)&=\partial_\alpha\xi^\alpha+D\partial_\lambda \omega\xi^\lambda
\end{align*}

And as $\tilde g= \exp\qty(2f)g=\exp\qty(2\qty(f+w))\eta$,

\begin{align*}
    \textnormal{Div}_{\tilde g}\qty(\xi)&=\partial_\alpha\xi^\alpha+D\partial_\lambda \qty(f+\omega)\xi^\lambda\\
    \textnormal{Div}_{\tilde g}\qty(\xi)&=\partial_\alpha\xi^\alpha+D\partial_\lambda \omega\xi^\lambda+D\xi^\lambda\partial_\lambda f\\
    \textnormal{Div}_{\tilde g}\qty(\xi)&=\textnormal{Div}_{g}\qty(\xi)+D\xi\qty(f)\numberthis\label{divrel}
\end{align*}

Multiplying \ref{divrel} by $g$ and subtracting \ref{lievers},

\begin{align*}
    \textnormal{Div}_{\tilde g}\qty(\xi)\tilde g-\frac D2\pounds_\xi\tilde g&=\textnormal{Div}_{g}\qty(\xi)\tilde g+D\tilde g\xi\qty(f)-\frac D2\exp\qty(2f)\pounds_\xi g-\frac D2\tilde g \xi\qty(2f)\\
    \textnormal{Div}_{\tilde g}\qty(\xi)\tilde g-\frac D2\pounds_\xi\tilde g&=\exp\qty(2f)\textnormal{Div}_{g}\qty(\xi)g-\exp\qty(2f)\frac D2\pounds_\xi g+D\tilde g\xi\qty(f)-D\tilde g \xi\qty(f)\\
    \textnormal{Div}_{\tilde g}\qty(\xi)\tilde g-\frac D2\pounds_\xi\tilde g&=\exp\qty(2f)\qty[\textnormal{Div}_{g}\qty(\xi)g-\frac D2\pounds_\xi g]
\end{align*}

That is, as long as $f$ is sufficiently well behaved, we have,

\begin{align*}
    \textnormal{Div}_g\qty(\xi)g=\frac D2\pounds_\xi g\Leftrightarrow \textnormal{Div}_{\tilde g}\qty(\xi)\tilde g=\frac D2\pounds_\xi \tilde g\numberthis\label{eqcond}
\end{align*}

In other words, the vector field $\xi$ which generates diffeomorphisms that preserves the conformally flat condition, does not depend on $\omega$ of our 
conformally flat metric $g=\exp\qty(2\omega)\eta$, so, we can choose $f=-\omega$, such that $\tilde g=\eta$, and by \ref{eqcond}, we just have to solve for,

\begin{align*}
    \partial_\alpha\xi^\alpha\eta_{\mu\nu}&=D\partial_{(\mu}\xi_{\nu)}\numberthis\label{theeq}
\end{align*}

Which is a lot easier then \ref{conformaldefeq}, first, we apply $\partial^\nu$ to the both sides, relabel the index, apply $\partial_\nu$, symmetrize $\mu\leftrightarrow \nu$ and use \ref{theeq},

\begin{align*}
    \partial^\nu\partial_\alpha\xi^\alpha\eta_{\mu\nu}&=D\partial^\nu\partial_{(\mu}\xi_{\nu)}\\
    \frac2D\partial_\mu\partial_\alpha\xi^\alpha&=\partial^\nu\partial_{\mu}\xi_{\nu}+\partial^\nu\partial_{\nu}\xi_{\mu}\\
    \frac2D\partial_\mu\partial_\alpha\xi^\alpha&=\partial^\alpha\partial_{\mu}\xi_{\alpha}+\partial^\alpha\partial_{\alpha}\xi_{\mu}\\
    \frac2D\partial_\nu\partial_\mu\partial_\alpha\xi^\alpha&=\partial_\nu\partial^\alpha\partial_{\mu}\xi_{\alpha}+\partial_\nu\partial^\alpha\partial_{\alpha}\xi_{\mu}\\
    \frac2D\partial_{(\nu}\partial_{\mu)}\partial_\alpha\xi^\alpha&=\partial_{(\nu}\partial_{\mu)}\partial^\alpha\xi_{\alpha}+\partial^\alpha\partial_{\alpha}\partial_{(\nu}\xi_{\mu)}\\
    \frac2D\partial_{\nu}\partial_{\mu}\partial_\alpha\xi^\alpha&=\partial_{\nu}\partial_{\mu}\partial_\alpha\xi^{\alpha}+\frac{\eta_{\mu\nu}}{D}\partial^\alpha\partial_{\alpha}\partial_{\beta}\xi^\beta\\
    \qty(2-D)\partial_{\nu}\partial_{\mu}\partial_\alpha\xi^\alpha&=\eta_{\mu\nu}\partial^\alpha\partial_{\alpha}\partial_{\beta}\xi^\beta
\end{align*}

Here enters the hypothesis of $D>2$, if $D=2$ the left-hand side of this equation would vanish, but in $D>2$, contracting with $\eta^{\mu\nu}$ gives,

\begin{align*}
    \qty(2-D)\partial^{\mu}\partial_{\mu}\partial_\alpha\xi^\alpha&=D\partial^\alpha\partial_{\alpha}\partial_{\beta}\xi^\beta\\
    2\qty(1-D)\partial^{\mu}\partial_{\mu}\partial_\alpha\xi^\alpha&=0
\end{align*}

\probitem{}