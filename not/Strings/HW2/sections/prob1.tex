\problem{}
\probitem{}

Let $M$ be our $D>2$ dimensional $C^\infty$ manifold and $\phi:\mathbb R\times M\rightarrow M$ be a one parameter family of diffeomorphisms, that is, $\forall t\in\mathbb R|\ \phi_t:M\rightarrow M$ is a diffeomorphism such that, 

\begin{itemize}
    \item $\forall p\in M|\ \phi_0\qty(p)=p$
    \item $\forall p\in M; \forall t,s\in\mathbb R|\ \phi_{t+s}\qty(p)=\qty(\phi_t\circ\phi_s)\qty(p)$
    \item $\forall p\in M|\ \phi\qty(p):\mathbb R\rightarrow M$ is at least $C^1$
\end{itemize}

Then this family of diffeomorphisms define in a natural manner a vector which generate these transformations, 
we define at each point $p\in M$ this vector by it's action in a function $f:M\rightarrow \mathbb R$,

\begin{align*}
    \xi_p\qty(f)&=\dv{}{t}\qty(\qty(f\circ\phi_t)\qty(p))\eval_{t=0}
\end{align*}

And from this we define $\xi$ as a vector field in $M$, this vector field has as integral curves exactly $\phi$. Let's 
open a little bit more in some chart $x:M\rightarrow \mathbb R^D$,

\begin{align*}
    \xi_p\qty(f)&=\dv{}{t}\qty(\qty(f\circ x^{-1}\circ x\circ\phi_t)\qty(p))\eval_{t=0}\\
    \xi_p\qty(f)&=\dv{}{t}\qty(\qty(f\circ x^{-1})\circ\qty(x\circ\phi_t)\qty(p))\eval_{t=0}\\
    \xi_p\qty(f)&=\partial_\mu\qty(f\circ x^{-1})\eval_{x\circ\phi_0\qty(p)}\dv{}{t}\qty(\qty(x\circ\phi_t)^\mu\qty(p))\eval_{t=0}\\
    \xi_p^\mu\partial_\mu\qty(f\circ x^{-1})\eval_{x\qty(p)}&=\partial_\mu\qty(f\circ x^{-1})\eval_{x\qty(p)}\dv{}{t}\qty(\qty(x\circ\phi_t)^\mu\qty(p))\eval_{t=0}
\end{align*}

Where of curse $\partial_\mu$ is to be interpreted as the derivative of the $\mu$-th component in the chart $x$. Here we have a clear definition of the 
values of the $\xi$ vector field in a chart $x$,

\begin{align*}
    \xi_p^\mu&=\dv{}{t}\qty(\qty(x\circ\phi_t)^\mu\qty(p))\eval_{t=0}
\end{align*}

The term inside the derivative is just the pullback of the chart $x$ --- the chart can be seen as a $\mathbb R^D$-valued function ---, which 
in it's own can be seen as a new chart $x'_t$ defined by the transformations of the diffeomorphism family $\phi$, that is,

\begin{align*}
    x'_t=\phi^\ast_tx=x\circ\phi_t:M\rightarrow \mathbb R^D
\end{align*}

All of this is consistent with our interpretation of the diffeomorphisms being a `\textit{coordinate change}', in principle, with 
enough derivability of $\phi$ we can actually write,

\begin{align*}
    x'_t&=x'_0+t\dv{}{t}\qty(x'_t)\eval_{t=0}+\mathcal O\qty(t^2)\\
    {x'_1}^\mu\eqcolon {x'}^\mu&={x}^\mu+\xi^\mu+\cdots
\end{align*}

We just restored the index to not confuse the components of the vector field $\xi$ in the basis $x$ with the vector field itself. 
That is, we showed that the transformation done by $\phi_1$ is equivalent to a `\textit{infinitesimal coordinate change}' by $\xi^\mu$. 
Actually, all this we did is the special case of a more general type of derivative, the Lie Derivative, given a vector field $\xi$ and it's 
family of integral curves $\phi$, it's defined in terms of the pushforward of the object under analysis,

\begin{align*}
    \pounds_\xi T=\dv{}{t}\qty(\phi_{-t\ast}T)\eval_{t=0}
\end{align*}

For a $\qty(0,2)$ tensor, that is, for the metric,

\begin{align*}
    \pounds_\xi g_{\mu\nu}=\nabla_\xi g_{\mu\nu}+g_{\mu\alpha}\nabla_\nu\xi^\alpha+g_{\alpha\nu}\nabla_\mu\xi^\alpha
\end{align*}

And as the connection is metric compatible,

\begin{align*}
    \pounds_\xi g_{\mu\nu}=\nabla_\nu\xi_\mu+\nabla_\mu\xi_\nu=2\nabla_{(\mu}\xi_{\nu)}
\end{align*}

This amounts for the first term in an expansion in $t$ of the diffeomorphism transformed metric $\phi_{-t\ast}g=g'_{t\mu\nu}$, that is, 

\begin{align*}
    \phi_{-t\ast}g_{\mu\nu}\eqcolon g'_{t\mu\nu}&=g_{\mu\nu}+t\pounds _\xi g_{\mu\nu}+\mathcal O\qty(t^2)\\
    g'_{t\mu\nu}&=g_{\mu\nu}+2t\nabla_{(\mu}\xi_{\nu)}+\mathcal O\qty(t^2)
\end{align*}

Which also can be interpreted as the `\textit{infinitesimal transformation}' of the metric. Imposing that the initial and transformed metric are conformally flat,

\begin{align*}
    \exp\qty(2\omega'_t)\eta_{\mu\nu}&=\exp\qty(2\omega)\eta_{\mu\nu}+2t\nabla_{(\mu}\xi_{\nu)}+\mathcal O\qty(t^2)\\
    \exp\qty(2\omega'_0+2t\dv{}{t}\qty(\omega'_t)\eval_{t=0}+\mathcal O\qty(t^2))\eta_{\mu\nu}&=\exp\qty(2\omega)\eta_{\mu\nu}+2t\nabla_{(\mu}\xi_{\nu)}+\mathcal O\qty(t^2)\\
    \exp\qty(2\omega)\exp\qty(2t\dv{}{t}\qty(\omega'_t)\eval_{t=0}+\mathcal O\qty(t^2))\eta_{\mu\nu}&=\exp\qty(2\omega)\eta_{\mu\nu}+2t\nabla_{(\mu}\xi_{\nu)}+\mathcal O\qty(t^2)\\
    \exp\qty(2\omega)\qty(1+2t\dv{}{t}\qty(\omega'_t)\eval_{t=0}+\mathcal O\qty(t^2))\eta_{\mu\nu}&=\exp\qty(2\omega)\eta_{\mu\nu}+2t\nabla_{(\mu}\xi_{\nu)}+\mathcal O\qty(t^2)\\
    \exp\qty(2\omega)2t\dv{}{t}\qty(\omega'_t)\eval_{t=0}\eta_{\mu\nu}&=2t\nabla_{(\mu}\xi_{\nu)}\\
    \dv{}{t}\qty(\omega'_t)\eval_{t=0}g_{\mu\nu}&=\nabla_{(\mu}\xi_{\nu)}
\end{align*}

The term $\dv{}{t}\qty(\omega '_t)\eval_{t=0}$ is fully determined by $\xi$, to see this just contract both sides with the metric,

\begin{align*}
    \dv{}{t}\qty(\omega'_t)\eval_{t=0}g^{\mu\nu}g_{\mu\nu}&=g^{\mu\nu}\nabla_{(\mu}\xi_{\nu)}\\
    \dv{}{t}\qty(\omega'_t)\eval_{t=0}D&=\nabla_{\mu}\xi^{\mu}
\end{align*}

Substituting this back in the original equation,

\begin{align*}
    \nabla_\alpha \xi^\alpha g_{\mu\nu}&=D\nabla_{(\mu}\xi_{\nu)}\numberthis\label{conformaldefeq}
\end{align*}

This is the condition upon $\xi$ that ensures the diffeomorphism maintain the conformally flatness of the metric. To solve it 
in it's full generality is hard, so, we'll make some use of symmetry, first, let's write it again in a coordinate free form,

\begin{align*}
    \textnormal{Div}_g\qty(\xi)g=\frac D2\pounds_\xi g\numberthis\label{confnocoor}
\end{align*}

Where $\textnormal{Div}_g\qty(\xi)$ is just $\nabla_\alpha\xi^\alpha$, but with the connection $\nabla$ defined with respect with the metric $g$, now, suppose we have a function $f:M\rightarrow \mathbb R$, 
define $\tilde g =\exp\qty(2f)g$, thus,

\begin{align*}
    \pounds_\xi \tilde g&=\pounds _\xi\qty(\exp\qty(2f)g)\\
    \pounds_\xi \tilde g&=\exp\qty(2f)\pounds _\xi g+g\pounds _\xi \exp\qty(2f)\\
    \pounds_\xi \tilde g&=\exp\qty(2f)\pounds _\xi g+g\xi\qty(\exp\qty(2f))\\
    \pounds_\xi \tilde g&=\exp\qty(2f)\pounds _\xi g+g\xi\qty(2f)\exp\qty(2f)\\
    \pounds_\xi \tilde g&=\exp\qty(2f)\pounds _\xi g+\tilde g\xi\qty(2f)\numberthis\label{lievers}
\end{align*}

Also,

\begin{align*}
    \textnormal{Div}_g\qty(\xi)&=\partial_\alpha\xi^\alpha+\Gamma^\alpha_{\alpha\lambda}\xi^\lambda\\
    \textnormal{Div}_g\qty(\xi)&=\partial_\alpha\xi^\alpha+\frac12g^{\alpha\beta}\qty(\partial_\alpha g_{\beta\lambda}+\partial_\lambda g_{\beta\alpha}-\partial_\beta g_{\alpha\lambda})\xi^\lambda\\
    \textnormal{Div}_g\qty(\xi)&=\partial_\alpha\xi^\alpha+g^{\alpha\beta}\qty(\partial_\alpha \omega g_{\beta\lambda}+\partial_\lambda \omega g_{\beta\alpha}-\partial_\beta \omega g_{\alpha\lambda})\xi^\lambda\\
    \textnormal{Div}_g\qty(\xi)&=\partial_\alpha\xi^\alpha+\qty(\partial_\lambda \omega +D\partial_\lambda \omega-\partial_\lambda \omega )\xi^\lambda\\
    \textnormal{Div}_g\qty(\xi)&=\partial_\alpha\xi^\alpha+D\partial_\lambda \omega\xi^\lambda
\end{align*}

And as $\tilde g= \exp\qty(2f)g=\exp\qty(2\qty(f+w))\eta$,

\begin{align*}
    \textnormal{Div}_{\tilde g}\qty(\xi)&=\partial_\alpha\xi^\alpha+D\partial_\lambda \qty(f+\omega)\xi^\lambda\\
    \textnormal{Div}_{\tilde g}\qty(\xi)&=\partial_\alpha\xi^\alpha+D\partial_\lambda \omega\xi^\lambda+D\xi^\lambda\partial_\lambda f\\
    \textnormal{Div}_{\tilde g}\qty(\xi)&=\textnormal{Div}_{g}\qty(\xi)+D\xi\qty(f)\numberthis\label{divrel}
\end{align*}

Multiplying \ref{divrel} by $\tilde g$ and subtracting \ref{lievers},

\begin{align*}
    \textnormal{Div}_{\tilde g}\qty(\xi)\tilde g-\frac D2\pounds_\xi\tilde g&=\textnormal{Div}_{g}\qty(\xi)\tilde g+D\tilde g\xi\qty(f)-\frac D2\exp\qty(2f)\pounds_\xi g-\frac D2\tilde g \xi\qty(2f)\\
    \textnormal{Div}_{\tilde g}\qty(\xi)\tilde g-\frac D2\pounds_\xi\tilde g&=\exp\qty(2f)\textnormal{Div}_{g}\qty(\xi)g-\exp\qty(2f)\frac D2\pounds_\xi g+D\tilde g\xi\qty(f)-D\tilde g \xi\qty(f)\\
    \textnormal{Div}_{\tilde g}\qty(\xi)\tilde g-\frac D2\pounds_\xi\tilde g&=\exp\qty(2f)\qty[\textnormal{Div}_{g}\qty(\xi)g-\frac D2\pounds_\xi g]
\end{align*}

That is, as long as $f$ is sufficiently well behaved, we have,

\begin{align*}
    \textnormal{Div}_g\qty(\xi)g=\frac D2\pounds_\xi g\Leftrightarrow \textnormal{Div}_{\tilde g}\qty(\xi)\tilde g=\frac D2\pounds_\xi \tilde g\numberthis\label{eqcond}
\end{align*}

In other words, the vector field $\xi$ which generates diffeomorphisms that preserves the conformally flat condition, does not depend on $\omega$ of our 
conformally flat metric $g=\exp\qty(2\omega)\eta$, so, we can choose $f=-\omega$, such that $\tilde g=\eta$, and by \ref{eqcond}, we just have to solve for,

\begin{align*}
    \partial_\alpha\xi^\alpha\eta_{\mu\nu}&=D\partial_{(\mu}\xi_{\nu)}\numberthis\label{theeq}
\end{align*}

Which is a lot easier then \ref{conformaldefeq}, first, we apply $\partial^\nu$ to the both sides, relabel the index, apply $\partial_\nu$, symmetrize $\mu\leftrightarrow \nu$ and use \ref{theeq},

\begin{align*}
    \partial^\nu\partial_\alpha\xi^\alpha\eta_{\mu\nu}&=D\partial^\nu\partial_{(\mu}\xi_{\nu)}\\
    \frac2D\partial_\mu\partial_\alpha\xi^\alpha&=\partial^\nu\partial_{\mu}\xi_{\nu}+\partial^\nu\partial_{\nu}\xi_{\mu}\\
    \frac2D\partial_\mu\partial_\alpha\xi^\alpha&=\partial^\alpha\partial_{\mu}\xi_{\alpha}+\partial^\alpha\partial_{\alpha}\xi_{\mu}\\
    \frac2D\partial_\nu\partial_\mu\partial_\alpha\xi^\alpha&=\partial_\nu\partial^\alpha\partial_{\mu}\xi_{\alpha}+\partial_\nu\partial^\alpha\partial_{\alpha}\xi_{\mu}\\
    \frac2D\partial_{(\nu}\partial_{\mu)}\partial_\alpha\xi^\alpha&=\partial_{(\nu}\partial_{\mu)}\partial^\alpha\xi_{\alpha}+\partial^\alpha\partial_{\alpha}\partial_{(\nu}\xi_{\mu)}\\
    \frac2D\partial_{\nu}\partial_{\mu}\partial_\alpha\xi^\alpha&=\partial_{\nu}\partial_{\mu}\partial_\alpha\xi^{\alpha}+\frac{\eta_{\mu\nu}}{D}\partial^\alpha\partial_{\alpha}\partial_{\beta}\xi^\beta\\
    \qty(2-D)\partial_{\nu}\partial_{\mu}\partial_\alpha\xi^\alpha&=\eta_{\mu\nu}\partial^\alpha\partial_{\alpha}\partial_{\beta}\xi^\beta\numberthis\label{d2}
\end{align*}

Contracting with $\eta^{\mu\nu}$ gives,

\begin{align*}
    \qty(2-D)\partial^{\mu}\partial_{\mu}\partial_\alpha\xi^\alpha&=D\partial^\alpha\partial_{\alpha}\partial_{\beta}\xi^\beta\\
    2\qty(1-D)\partial^{\mu}\partial_{\mu}\partial_\alpha\xi^\alpha&=0\\
    \partial^{\mu}\partial_{\mu}\partial_\alpha\xi^\alpha&=0\numberthis\label{d2next}
\end{align*}

The last step is justified by merely $D\geq2$, so that up to now, we haven't fully used the hypothesis of being in $D>2$. So, we'll invoke it now, by using \ref{d2next} in equation \ref{d2}, this is only justified if $D\neq2$, 
because, for $D=2$ the left-hand side of \ref{d2} is identically zero, but, for $D> 2$ the use of \ref{d2next} in \ref{d2} results in 
another constraint,

\begin{align*}
    \qty(2-D)\partial_\mu\partial_\nu\partial_\alpha\xi^\alpha&=0\\
    \partial_\mu\partial_\nu\partial_\alpha\xi^\alpha&=0\numberthis\label{d2nn}
\end{align*}

It's clear that equation \ref{d2} allows for much more solutions than \ref{d2nn}, for example, $\partial\cdot\xi=\cos\qty(k_\mu x^\mu)$ with $k_\mu k^\mu=0$ is a 
solution for \ref{d2next}, but, isn't for \ref{d2nn}. To integrate \ref{d2nn} is also a lot easier than \ref{d2next},

\begin{align*}
    \partial_\mu\partial_\nu\partial_\alpha\xi^\alpha&=0\\
    \partial_\nu\partial_\alpha\xi^\alpha&=a_\nu\\
    \partial_\alpha\xi^\alpha&=a_\nu x^\nu +b\numberthis\label{divxi}
\end{align*}

With $a_\nu,b$ arbitrary constants, this is also automatically a solution of \ref{d2next}. To integrate \ref{divxi} is not trivial, as there might be some divergenceless current contributing to $\xi^\alpha$, 
that is, the integration gives,

\begin{align*}
    \partial_\alpha\xi^\alpha&=a_\nu x^\nu +b\\
    \xi^\alpha&=\frac12\tensor{a}{^\alpha_\mu_\nu}x^\mu x^\nu+\tensor{b}{^\alpha_\mu}x^\mu+c^\alpha+f^\alpha\numberthis\label{xisol}
\end{align*}
    
With of course $\tensor{a}{^\alpha_\mu_\nu}=\tensor{a}{^\alpha_{(\mu}_{\nu)}},\tensor{a}{^\alpha_\alpha_\nu}=\tensor{a}{_\nu},\tensor{b}{^\alpha_\alpha}=b,c^\alpha$ being constants, with exception of $f^\alpha$, which is some non constant divergenceless vector field $\partial_\alpha f^\alpha=0$. To know what choice of $f^\alpha$ is the correct one 
we have to go back to equation \ref{theeq}, and apply to it $\partial_\lambda$,

\begin{align*}
    \frac2D\partial_\lambda\partial_\alpha\xi^\alpha\eta_{\mu\nu}&=\partial_\lambda\partial_{\mu}\xi_{\nu}+\partial_\lambda\partial_{\nu}\xi_{\mu}
\end{align*}

Now we permute the index,

\begin{align*}
    \frac2D\partial_\lambda\partial_\alpha\xi^\alpha\eta_{\mu\nu}&=\partial_\lambda\partial_{\mu}\xi_{\nu}+\partial_{\nu}\partial_\lambda\xi_{\mu}\\
    \frac2D\partial_\mu\partial_\alpha\xi^\alpha\eta_{\nu\lambda}&=\partial_\mu\partial_{\nu}\xi_{\lambda}+\partial_\lambda\partial_{\mu}\xi_{\nu}\\
    \frac2D\partial_\nu\partial_\alpha\xi^\alpha\eta_{\lambda\mu}&=\partial_\nu\partial_{\lambda}\xi_{\mu}+\partial_\mu\partial_{\nu}\xi_{\lambda}
\end{align*}

Sum the two first equations and subtract the third one,

\begin{align*}
    \frac2D\qty(\partial_\lambda\partial_\alpha\xi^\alpha\eta_{\mu\nu}+\partial_\mu\partial_\alpha\xi^\alpha\eta_{\nu\lambda}-\partial_\nu\partial_\alpha\xi^\alpha\eta_{\lambda\mu})&=\partial_\lambda\partial_{\mu}\xi_{\nu}+\partial_{\nu}\partial_\lambda\xi_{\mu}+\partial_\mu\partial_{\nu}\xi_{\lambda}+\partial_\lambda\partial_{\mu}\xi_{\nu}-\partial_\nu\partial_{\lambda}\xi_{\mu}-\partial_\mu\partial_{\nu}\xi_{\lambda}
\end{align*}

By use of \ref{divxi},

\begin{align*}
    \frac2D\qty(a_\lambda\eta_{\mu\nu}+a_\mu\eta_{\nu\lambda}-a_\nu\eta_{\lambda\mu})&=2\partial_\lambda\partial_{\mu}\xi_{\nu}
\end{align*}

And substituting \ref{xisol} in the right-hand side,

\begin{align*}
    \frac2D\qty(a_\lambda\eta_{\mu\nu}+a_\mu\eta_{\nu\lambda}-a_\nu\eta_{\lambda\mu})&=2a_{\nu\mu\lambda}+2\partial_\lambda\partial_\mu f_\nu
\end{align*}

We can thus use this to fix $f_\lambda$, it's trivial to carry out the integration,

\begin{align*}
    \partial_\lambda\partial_\mu f_\nu&=\frac1D\qty(a_\lambda\eta_{\mu\nu}+a_\mu\eta_{\nu\lambda}-a_\nu\eta_{\lambda\mu})-a_{\nu\mu\lambda}\numberthis\label{fdef}\\
    \partial_\mu f_\nu&=\frac1D\qty(a\cdot x\eta_{\mu\nu}+a_\mu x_\nu-a_\nu x_\mu)-a_{\nu\mu\lambda}x^\lambda+A_{\nu\mu}\\
    f_\nu&=\frac1D\qty(a\cdot x x_\nu-\frac12a_\nu x\cdot x)-\frac12a_{\nu\mu\lambda}x^\lambda x^\mu+A_{\nu\mu}x^\mu+B_{\nu}\\
    \partial_\mu f^\mu&=\frac1D\qty(a\cdot x D+a\cdot x-a\cdot x)-a\cdot x+\tensor{A}{^\mu_\mu}=\tensor{A}{^\mu_\mu}
\end{align*}

Where we just computed also the divergence, which implies the constraint $\tensor{A}{^\mu_\mu}=0$. Substituting this back in \ref{xisol} get us, 
% the most general solution we found to equation \ref{d2nn} is,

\begin{align*}
    \xi^\alpha&=\frac12\tensor{a}{^\alpha_\mu_\nu}x^\mu x^\nu+\tensor{b}{^\alpha_\mu}x^\mu+c^\alpha+\frac1D\qty(a\cdot x x^\alpha-\frac12a^\alpha x\cdot x)-\frac12\tensor{a}{^\alpha_\mu_\lambda}x^\lambda x^\mu+\tensor{A}{^\alpha_\mu}x^\mu+B^{\alpha}\\
    \xi^\alpha&=\frac1D\qty(a\cdot x x^\alpha-\frac12a^\alpha x\cdot x)+\qty(\tensor{A}{^\alpha_\mu}+\tensor{b}{^\alpha_\mu})x^\mu+\qty(B^{\alpha}+c^\alpha)\\
    \xi^\alpha&=\frac1D\qty(a\cdot x x^\alpha-\frac12a^\alpha x\cdot x)+\tensor{b}{^\alpha_\mu}x^\mu+c^\alpha
\end{align*}

In the last line we just redefined the tensors. All of this with $a^\alpha,\tensor{b}{^\alpha_\mu},c^\alpha$ arbitrary constants, but, we didn't really 
confirmed this solution of \ref{d2nn} is a fully compatible with \ref{theeq}, we just checked it satisfy some derived equations from \ref{theeq}, now, let's put it to 
the real test,

\begin{align*}
    \partial_\alpha\xi^\alpha\eta_{\mu\nu}&=D\partial_{(\mu}\xi_{\nu)}\\
    \qty[\frac1D\qty(a\cdot x D+a\cdot x-2\frac12a\cdot x)+\tensor{b}{^\alpha_\alpha}]\eta_{\mu\nu}&=D\qty[\frac1D\qty(a\cdot x \eta_{\mu\nu}+a_{(\mu} x_{\nu)}-a_{(\nu} x_{\mu)})+\tensor{b}{_{(\nu}_{\mu)}}]\\
    a\cdot x\eta_{\mu\nu} +\tensor{b}{^\alpha_\alpha}\eta_{\mu\nu}&=a\cdot x \eta_{\mu\nu}+D\tensor{b}{_{(\nu}_{\mu)}}\\
    \tensor{b}{^\alpha_\alpha}\eta_{\mu\nu}&=D\tensor{b}{_{(\nu}_{\mu)}}\\
    \tensor{b}{_{(\nu}_{\mu)}}&=\frac1D\tensor{b}{^\alpha_\alpha}\eta_{\mu\nu}
\end{align*}

This constrains the symmetric part of $b_{\mu\nu}$ being a pure trace, so that the degrees of freedom reduces from $b_{\mu\nu}$ to $b_{[\mu\nu]}, \tensor{b}{^\alpha_\alpha}$. This is not the end! 
We haven't show all the solutions we found are in fact all the solutions from \ref{theeq}, as we obtained them from a different approach than direct integration of the equation \ref{theeq}, so let's show this, 
first, \ref{theeq} is a set of first order partial differential equations, there are $D^2$ such equation, but, they're symmetric in exchange $\mu\leftrightarrow\nu$, so we have to account only for $\frac{D^2+D}{2}$ of them, still, we have to account for 
the possible constants which can be added to $\xi$ without changing the equation, $D$ of them, and also for another boundary condition on the divergence, $1$, so that the full number of constants is,

\begin{align*}
    \frac{D^2+D}{2}+D+1=\frac12\qty(D+1)\qty(D+2)
\end{align*}

Now let's count the number of constants in the solution we found, namely,

\begin{align*}
    \xi_\alpha&=\frac1D\qty(a\cdot x x_\alpha-\frac12a_\alpha x\cdot x)+\tensor{b}{_{[\alpha}_{\mu]}}x^\mu+\frac1D\tensor{b}{^\mu_\mu}x_\alpha+c_\alpha
\end{align*}

Both $c^\alpha,a^\alpha$ contributes with $D$, $\tensor{b}{^\mu_\mu}$ contributes with $1$, and lastly $\tensor{b}{_{[\alpha}_{\mu]}}$ contributes with $\frac{D^2-D}{2}$, hence, the total number of constants is,

\begin{align*}
    \frac{D^2-D}{2}+2D+1=\frac12\qty(D+1)\qty(D+2)
\end{align*}

Matching exactly the number from the original equation, that is, we already got all the solutions, hence, the most general solution of \ref{theeq}, for $D>2$, is given by, just for better reading we make a change of variables that does not 
affect at all the number of free parameters,

\begin{align*}
    \xi_\alpha&=2a\cdot x x_\alpha-a_\alpha x\cdot x+\tensor{b}{_{[\alpha}_{\mu]}}x^\mu+x_\alpha d+c_\alpha
\end{align*}

\probitem{}

We have already counted for the number of independent parameters in the last item as an way of making sure we got all the solutions to the equation, so 
now it remains to get what is the geometric picture of the transformations $\phi$ generated by this $\xi$, first, notice that we have four different kinds of terms 
in our expression for $\xi$,

\begin{align*}
    \xi_\alpha&=2a\cdot x x_\alpha-a_\alpha x\cdot x+\tensor{b}{_{[\alpha}_{\mu]}}x^\mu+x_\alpha d+c_\alpha
\end{align*}

A quadratic, a two linear and a constant in $x$. We'll treat one by one, first, let us discuss what is to get $\phi$ from $\xi$. As we pointed before, 

\begin{align*}
    \xi_p^\mu&=\dv{}{t}\qty(\qty(x\circ\phi_t)^\mu(p))\eval_{t=0}
\end{align*}

We can deform this definition a little by,

\begin{align*}
    \xi_{\phi_s\qty(p)}^\mu\eval_{s=0}&=\dv{}{t}\qty(\qty(x\circ\phi_t)^\mu(\phi_s\qty(p)))\eval_{t=0}\eval_{s=0}\\
    \xi_{\phi_s\qty(p)}^\mu&=\dv{}{t}\qty(\qty(x\circ\phi_{t+s})^\mu(p))\eval_{t=0}\\
    \xi_{\phi_s\qty(p)}^\mu&=\dv{}{\qty(t+s)}\qty(\qty(x\circ\phi_{t+s})^\mu(p))\eval_{t+s=s}\dv{\qty(t+s)}{t}\eval_{t=0}\\
    \xi_{\phi_s\qty(p)}^\mu&=\dv{}{s}\qty(\qty(x\circ\phi_{s})^\mu(p))
\end{align*}

Seeing $\xi_{\phi_s\qty(p)}$ as a function of the coordinates of the point $\phi_s\qty(p)$,

\begin{align*}
    \xi^\mu\qty(\qty(x\circ\phi_s)\qty(p))&=\dv{}{s}\qty(\qty(x\circ\phi_{s})^\mu(p))\\
    \xi^\mu\qty(x\circ\phi_s)&=\dv{}{s}\qty(x\circ\phi_{s})^\mu\\
    \xi^\mu\qty(x'_s)&=\dv{}{s}\qty(x'_s)^\mu
\end{align*}

That is, to solve for $\phi$, we just have to solve this first order differential equation for the coordinates of $\phi_s\qty(p)$, and 
as we had already defined, $x\circ\phi_s\qty(p)\coloneq x'_s$, so that, we have to solve for,

\begin{align*}
    \dv{}{s}x'_{s\alpha}={\dot x}'_{s\alpha}&=2a\cdot x'_s x'_{s\alpha}-a_\alpha x'_s\cdot x'_s+\tensor{b}{_{[\alpha}_{\mu]}}x'^\mu_s+x'_{s\alpha} d+c_\alpha
\end{align*}

In this form it's obvious that it is too hard to solve, so we rather solve for each parameter individually, starting by,

\begin{itemize}
    \item $c^\alpha$
\end{itemize}

\begin{align*}
    \dv{}{s}x'_{s\alpha}&=c_\alpha\\
    x'_{s\alpha}&=sc_\alpha+x'_{0\alpha}\\
    x'_{s\alpha}&=sc_\alpha+x_\alpha
\end{align*}

In other words, the transformation parametrized by $c^\alpha$ describe a translation of the coordinate $x^\alpha$,

\begin{itemize}
    \item $d$
\end{itemize}

\begin{align*}
    \dv{}{s}x'_{s\alpha}&=x'_{s\alpha} d\\
    x'_{s\alpha}&=A_\alpha\exp\qty(sd)
\end{align*}

Imposing $x'_{0\alpha}=x_\alpha$,

\begin{align*}
    x'_{s\alpha}&=x_\alpha\exp\qty(sd)
\end{align*}

Which accounts for dilatations --- rescaling --- of the coordinate $x_\alpha$,

\begin{itemize}
    \item $b_{[\alpha\mu]}$
\end{itemize}

\begin{align*}
    \dv{}{s}x'_{s\alpha}&=\tensor{b}{_{[\alpha}_{\mu]}}x'^\mu_s\\
    x'_{s\alpha}&=\tensor{\exp\qty(s\frac12b_{[\rho\sigma]}\tensor{M}{^{[\rho}^{\sigma]}})}{_\alpha_\mu}x^\mu
\end{align*}

This solution may seem taken directly from a hat, but, let's record that this equation we're solving is a set of first order linear ordinary differential equations, 
and the solution to one of the kind $\dot{\vb x}\qty(s)=\vb A\cdot \vb x\qty(s)$ is always of the form $\vb x\qty(s)=\exp\qty(s\vb A)\vb x\qty(0)$, which is exactly what we wrote, apart 
from introducing a new tensor $M$ and a $\frac12$ factor, why you may ask? Let us be honest, we all know what is happening here... But, we have explicit written just two of the four index in $M$, the other two 
stands outside the exponential, $M$ is not at all a free parameter, it has to satisfy the consistency condition, 

\begin{align*}
    \dv{}{s}x'_{s\alpha}&=\frac12b_{[\rho\sigma]}\tensor{M}{^{[\rho}^{\sigma]}_\alpha^\nu}\tensor{\exp\qty(s\frac12b_{[\rho\sigma]}\tensor{M}{^{[\rho}^{\sigma]}})}{_\nu_\mu}x^\mu\\
    \dv{}{s}x'_{s\alpha}&=\frac12b_{[\rho\sigma]}\tensor{M}{^{[\rho}^{\sigma]}_\alpha^\nu}x'_{s\nu}\\
    \dv{}{s}x'_{s\alpha}&=\frac12b_{[\rho\sigma]}\tensor{M}{^{[\rho}^{\sigma]}_\alpha_\nu}x'^\nu_{s}\Rightarrow \frac12b_{[\rho\sigma]}\tensor{M}{^{[\rho}^{\sigma]}_\alpha_\nu}=b_{[\alpha\nu]}
\end{align*}

This is easy to solve,

\begin{align*}
    \frac12b_{[\rho\sigma]}\tensor{M}{^{[\rho}^{\sigma]}_\alpha_\nu}&=b_{[\alpha\nu]}\\
    \frac12b_{[\rho\sigma]}\tensor{M}{^{[\rho}^{\sigma]}_\alpha_\nu}&=\frac12b_{[\rho\sigma]}\qty(\tensor{\eta}{_\alpha^\rho}\tensor{\eta}{_\nu^\sigma}-\tensor{\eta}{_\nu^\rho}\tensor{\eta}{_\alpha^\sigma})\\
    \tensor{M}{^{[\rho}^{\sigma]}_{[\alpha}_{\nu]}}&=\qty(\tensor{\eta}{_\alpha^\rho}\tensor{\eta}{_\nu^\sigma}-\tensor{\eta}{_\nu^\rho}\tensor{\eta}{_\alpha^\sigma})
\end{align*}

It's clear that these transformations are just the usual Lorentz ones, boosts plus rotations, as $M$ is exactly the generators of such transformations --- in a real form ---, just 
for completeness, we compute,

\begin{align*}
    x'^\alpha_{s}x'_{s\alpha}&=\tensor{\exp\qty(s\frac12b_{[\rho\sigma]}\tensor{M}{^{[\rho}^{\sigma]}})}{^{[\alpha}^{\nu]}}x_\nu\tensor{\exp\qty(s\frac12b_{[\rho\sigma]}\tensor{M}{^{[\rho}^{\sigma]}})}{_{[\alpha}_{\mu]}}x^\mu\\
    x'^\alpha_{s}x'_{s\alpha}&=\tensor{\exp\qty(-s\frac12b_{[\rho\sigma]}\tensor{M}{^{[\rho}^{\sigma]}})}{^{\nu]}^{[\alpha}}x_\nu\tensor{\exp\qty(s\frac12b_{[\rho\sigma]}\tensor{M}{^{[\rho}^{\sigma]}})}{_{[\alpha}_{\mu]}}x^\mu\\
    x'^\alpha_{s}x'_{s\alpha}&=x_\nu\tensor{\exp\qty(-s\frac12b_{[\rho\sigma]}\tensor{M}{^{[\rho}^{\sigma]}})}{^{[\nu}^{\alpha]}}\tensor{\exp\qty(s\frac12b_{[\rho\sigma]}\tensor{M}{^{[\rho}^{\sigma]}})}{_{[\alpha}_{\mu]}}x^\mu
\end{align*}

As trivially $b_{[\rho\sigma]}\tensor{M}{^{[\rho}^{\sigma]}}$ commute with itself --- everything commutes with itself, as long as it doesn't have characteristic $2$ ---, so that BCH gives trivially,

\begin{align*}
    x'^\alpha_{s}x'_{s\alpha}&=x_\nu\tensor{\exp\qty(-s\frac12b_{[\rho\sigma]}\tensor{M}{^{[\rho}^{\sigma]}}+s\frac12b_{[\rho\sigma]}\tensor{M}{^{[\rho}^{\sigma]}})}{^{\nu}_{\mu}}x^\mu\\
    x'^\alpha_{s}x'_{s\alpha}&=x_\nu\tensor{\exp\qty(\vb 0)}{^{\nu}_{\mu}}x^\mu\\
    x'^\alpha_{s}x'_{s\alpha}&=x_\nu\tensor{\eta}{^{\nu}_{\mu}}x^\mu\\
    x'^\alpha_{s}x'_{s\alpha}&=x_\mu x^\mu
\end{align*}

Just exactly what a Lorentz transformation should do. Then, the $b_{[\alpha\mu]}$ generates the Lorentz transformations.

\begin{itemize}
    \item $a_{\alpha}$
\end{itemize}

\begin{align*}
    {\dot x}'_{s\alpha}&=2a\cdot x'_s x'_{s\alpha}-a_\alpha x'_s\cdot x'_s\numberthis\label{eqtosol}
\end{align*}

This is the last, hardest and most interesting one to solve, we proceed with the usual magic hat tricks and divine inspiration/intervention, 

\begin{align*}
    \frac{{\dot x}'_{s\alpha}}{x'_s\cdot x'_s}&=\frac{2\qty(a\cdot x'_s) x'_{s\alpha}}{x'_s\cdot x'_s}-a_\alpha\\
    \frac{{\dot x}'_{s\alpha}}{x'_s\cdot x'_s}-\frac{2\qty(a\cdot x'_s) x'_{s\alpha}}{x'_s\cdot x'_s}&=-a_\alpha\\
    \frac{{\dot x}'_{s\alpha}}{x'_s\cdot x'_s}-\frac{2\qty(a\cdot x'_s) \qty(x'_s\cdot x'_s) x'_{s\alpha}}{\qty(x'_s\cdot x'_s)^2}&=-a_\alpha\numberthis\label{progress1} 
\end{align*}

Now multiply equation \ref{eqtosol} by $x'^\alpha_s$, which gives,

\begin{align*}
    x'_s\cdot{\dot x}'_{s}&=2\qty(a\cdot x'_s) \qty(x'_s\cdot x'_{s})-\qty(a\cdot x'_s) \qty(x'_s\cdot x'_s)\\
    x'_s\cdot{\dot x}'_{s}&=\qty(a\cdot x'_s) \qty(x'_s\cdot x'_{s})
\end{align*}

Substituting this back in equation \ref{progress1},

\begin{align*}
    \frac{{\dot x}'_{s\alpha}}{x'_s\cdot x'_s}-\frac{2\qty(x'_s\cdot {\dot x}'_s) x'_{s\alpha}}{\qty(x'_s\cdot x'_s)^2}&=-a_\alpha\\
    \frac{1}{x'_s\cdot x'_s}\dv{}{s}{x}'_{s\alpha}-x'_{s\alpha}\frac{1}{\qty(x'_s\cdot x'_s)^2}\dv{}{s}\qty(x'_s\cdot {x}'_s) &=-a_\alpha\\
    \frac{1}{x'_s\cdot x'_s}\dv{}{s}{x}'_{s\alpha}+x'_{s\alpha}\dv{}{s}\frac{1}{x'_s\cdot x'_s} &=-a_\alpha\\
    \dv{}{s}\qty(\frac{{x}'_{s\alpha}}{x'_s\cdot x'_s})&=-a_\alpha\\
    \frac{{x}'_{s\alpha}}{x'_s\cdot x'_s}&=-sa_\alpha+\frac{x_\alpha}{x\cdot x}
\end{align*}

What remains is a mere algebraic equation which is straightforward to solve, first, contract with $x'^\alpha_s$,

\begin{align*}
    \frac{x'_s\cdot {x}'_{s}}{x'_s\cdot x'_s}=1&=-sx'_s\cdot a+\frac{x'_s\cdot x}{x\cdot x}\\
    -sx'_s\cdot a+\frac{x'_s\cdot x}{x\cdot x}&=1
\end{align*}

Which gives one constraint, now, contract with $-sa^\alpha+\frac{x^\alpha}{x\cdot x}$,

\begin{align*}
    \frac{1}{x'_s\cdot x'_s}\qty(-sx'_s\cdot a+\frac{x'_s\cdot x}{x\cdot x})&=\qty(-sa^\alpha+\frac{x^\alpha}{x\cdot x})\qty(-sa_\alpha+\frac{x_\alpha}{x\cdot x})
\end{align*}

Using the constraint derived,

\begin{align*}
    \frac{1}{x'_s\cdot x'_s}&=\qty(-sa^\alpha+\frac{x^\alpha}{x\cdot x})\qty(-sa_\alpha+\frac{x_\alpha}{x\cdot x})
\end{align*}

Which gives the final constraint, substituting this back on our original equation gives,

\begin{align*}
    \frac{{x}'_{s\alpha}}{x'_s\cdot x'_s}&=-sa_\alpha+\frac{x_\alpha}{x\cdot x}\numberthis\label{sct}\\
    {x}'_{s\alpha}\qty(-sa^\alpha+\frac{x^\alpha}{x\cdot x})\qty(-sa_\alpha+\frac{x_\alpha}{x\cdot x})&=-sa_\alpha+\frac{x_\alpha}{x\cdot x}\\
    {x}'_{s\alpha}&=\frac{-sa_\alpha+\frac{x_\alpha}{x\cdot x}}{\qty(-sa^\alpha+\frac{x^\alpha}{x\cdot x})\qty(-sa_\alpha+\frac{x_\alpha}{x\cdot x})}\\
    {x}'_{s\alpha}&=\frac{-sa_\alpha\qty(x\cdot x)+x_\alpha}{\qty(-sa^\alpha\qty(x\cdot x)+x^\alpha)\qty(-sa_\alpha+\frac{x_\alpha}{x\cdot x})}\\
    {x}'_{s\alpha}&=\frac{x_\alpha-sa_\alpha\qty(x\cdot x)}{1-2sa\cdot x+s^2a\cdot a\qty(x\cdot x)}\numberthis\label{sct2}
\end{align*}

These are the special conformal transformations, from \ref{sct} it's easy to interpret geometrically what they do. First, 
apply a inversion, $x^\mu\rightarrow \frac{x^\mu}{x\cdot x}$, second, do a translation $\frac{x^\mu}{x\cdot x}\rightarrow\frac{x^\mu}{x\cdot x}-sa^\mu$, and lastly perform a second 
inversion, $\frac{x^\mu}{x\cdot x}-sa^\mu\rightarrow \frac{\frac{x^\mu}{x\cdot x}-sa^\mu}{\qty(\frac{x^\mu}{x\cdot x}-sa^\mu)\qty(\frac{x_\mu}{x\cdot x}-sa_\mu)}$. As opposed to a first impression, 
they're indeed well behaved at $x\cdot x=0$, as can be seen from \ref{sct2}, but sadly, they aren't well behaved for the denominator in \ref{sct2} be zero, that is,

\begin{align*}
    1-2\tilde s a\cdot x+{\tilde s}^2\qty(a\cdot a)\qty(x\cdot x)&=0\Rightarrow \tilde s=\frac{2\qty(a\cdot x)\pm\sqrt{4\qty(a\cdot x)^2-4\qty(a\cdot a)\qty(x\cdot x)}}{2\qty(a\cdot a)\qty(x\cdot x)}
\end{align*}

Which only has a real solution for $x^\mu\propto a^\mu$, but, for this point away from the origin the transformation needs not to be well behaved, as the charts $x,x'_s$ themselves doesn't need to be 
defined from the whole manifold, nevertheless, if we're working in a compact manifold we can make sense of this divergence $x'^\mu_s\rightarrow\infty$ as this is indeed a point of a compact manifold.