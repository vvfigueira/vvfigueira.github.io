\problem{}
\probitem{}

First, let us go back a little to the Polyakov Action,

\begin{align*}
    S_{\textnormal P}&=\frac{1}{4\pi\alpha'}\int\dd[2]{\sigma}\sqrt{\abs{\Det\qty[h]}}h^{ab}g_{\mu\nu}\partial^a X^\mu\partial^b X^\nu\numberthis\label{polyac}
\end{align*}

We wrote in the Euclidean signature, as this is the one we're interested. At this point we already know by heart 
that we have Diff$\times$Weyl gauge redundancy, and also that we can extract two equations of motion,

\begin{align*}
    \fdv{S_{\textnormal P}}{X_\mu}&=0\Rightarrow \partial_a\qty[\sqrt{h}\partial^a X^\mu]=0\numberthis\label{eomx}\\
    \fdv{S_{\textnormal P}}{h^{ab}}&=0\Rightarrow T^{ab}=-\frac{4\pi}{\sqrt{h}}\fdv{S_{\textnormal P}}{h^{ab}}=0\Rightarrow T_{ab}=-\frac{1}{\alpha'}\qty(\partial_aX^\mu\partial_bX_\mu-\frac12h_{ab}\partial_cX^\mu\partial^cX_\mu)=0\numberthis\label{eomt}
\end{align*}

There are two main ways we could procede, one option is to \textbf{fully fix} the gauge redundancy by making some specific choice of 
coordinates and/or metric, as is done in the light-cone gauge, in this case \ref{eomt} hold trivially as consequence of the choice 
of gauge, and need not to be imposed as an equation of motion, this is consistent in a sense that after the \textbf{full fixing} of the 
gauge redundancy, there should not be any metric in the Action, and hence no equation of motion for it. In other words, to solve the equation of 
motion for the metric is sort of equivalent to fully fixing the gauge. This is the first option, fully fixing the gauge so that is not 
necessary to impose an equation of motion for the metric. Now the second option, we don't fully fix the gauge a priori, and actually try 
to solve for both \ref{eomx}, \ref{eomt}.

Of course the light-cone gauge is all fun and games, but, the real deal is in the conformal gauge, let's rewrite the Polyakov Action in 
the conformal gauge,

\begin{align*}
    S_{\textnormal {Pc}}&=\frac{1}{2\pi\alpha'}\int\dd[2]{\sigma}g_{\mu\nu}\partial X^\mu\bar\partial X^\nu\numberthis\label{polyacconf}
\end{align*}

There is no metric in sight here, so we just have one equation of motion,

\begin{align*}
    \bar\partial\partial X^\mu&=0\numberthis\label{eomxconf}
\end{align*}

Until now there is no problem, it's ok to have just one equation of motion --- as in the case in the light-cone gauge ---, but, as long 
as the gauge fully fixes the Diff$\times$Weyl, so that equation \ref{eomt} holds trivially, is this the case in the conformal gauge? No. 
There are for sure \textit{some} of the Weyl transformations which can be undone by Diffeomorphisms, but if the gauge was really fully fixed 
there should be none of available Weyl and Diffeomorphisms for us to use, but there are. This is not really a problem, the thing is, as 
there is no metric in sight in \ref{polyacconf} we can naively believe that we only have to solve for \ref{eomxconf}, but, as the original 
equation is \ref{polyac}, if we don't fully fix the gauge we have to also solve \ref{eomt}, which in the conformal gauge is,

\begin{align*}
    T_{ab}=-\frac{1}{\alpha'}\qty(\partial_aX^\mu\partial_bX_\mu-\qty(1-\delta_{ab})\partial X^\mu\bar \partial X_\mu)=0
\end{align*}

Which gives two linearly independent equations,

\begin{subequations}\label{tconstraint}
\begin{align}
    T_{zz}&=-\frac{1}{\alpha'}\partial X^\mu\partial X_\mu=0\\
    T_{\bar z\bar z}&=-\frac{1}{\alpha'}\bar\partial X^\mu\bar\partial X_\mu=0
\end{align}
\end{subequations}

Ok. All we did this for is: No matter what kind of gauge choice is done, we have to guarantee that $T_{ab}=0$, if the gauge 
is fully fixed this is trivial to hold, if not this condition ensures that we're not doing something wrong.

Now let's go to the quantum theory, to solve for \ref{eomxconf} is trivial, the problem is to solve for \ref{tconstraint}, which in the quantum 
theory read,

\begin{subequations}\label{tqconstraint}
\begin{align}
    T\qty(z)&=-\frac{1}{\alpha'}\cnord{\partial X^\mu\partial X_\mu}\qty(z)=0\\
    \bar T\qty(\bar z)&=-\frac{1}{\alpha'}\cnord{\bar\partial X^\mu\bar\partial X_\mu}\qty(z)=0
\end{align}
\end{subequations}

To this to hold as an operator equation over the whole Hilbert space is actually not possible, to see this expand in modes $T$, 
\begin{align*}
    T\qty(z)&=\sum\limits_{m\in\mathbb Z}\frac{L_m}{z^{m+2}}
\end{align*}
where $L_m$ are the Virasoro generators, satisfying 
\begin{align*}
    \comm{L_m}{L_n}&=\qty(m-n)L_{n+m}+\frac{c}{12}m\qty(m^2-1)\delta_{m,-n}
\end{align*}

It's clear that if $T\qty(z)=0$ is to hold as an operator equation, then, given any state $\Psi$ in our Hilbert space $\mathscr H$, we would need,
\begin{align*}
    T\qty(z)\Psi=0\Rightarrow L_m\Psi=0,\ \ \ \forall m\in\mathbb Z
\end{align*}

To see this is not consistent act with $L_n$,
\begin{align*}
    0&=L_m\Psi\\
    0&=L_nL_m\Psi=\comm{L_n}{L_m}\Psi\\
    0&=(\qty(n-m)L_{n+m}+\frac{c}{12}n\qty(n^2-1)\delta_{n,-m})\Psi\\
    0&=\qty(n-m)L_{n+m}\Psi+\frac{c}{12}n\qty(n^2-1)\delta_{n,-m}\Psi\\
    0&=\frac{c}{12}n\qty(n^2-1)\delta_{n,-m}\Psi\\
\end{align*}

The right-hand side is not zero for arbitrary values of $n,m,c$ and arbitrary $\Psi$, hence, we cannot look forward to 
satisfy the equation of motion \ref{tqconstraint} as an operator equation through the whole Hilbert space $\mathscr H$, 
what we do is to select a few states to compose a subspace of it, and call it the Physical Hilbert space $\mathscr H_{\textnormal{phys}}$, such that,

\begin{align*}
    \qty(L_n-A\delta_{n,0})\Psi&=0,\ \ \ \Psi\in\mathscr H_{\textnormal{phys}},\ n\geq0
\end{align*}





we have to 
use some general facts about $T,\bar T$, first, it satisfies the following OPE,

\begin{align*}
    T\qty(z_1)T\qty(z_2)&=\frac{c}{2\qty(z_1-z_2)^4}+\frac{2}{\qty(z_1-z_2)^2}T\qty(z_2)+\frac{1}{\qty(z_2-z_2)}\partial T\qty(z_2)
\end{align*}

And for any primary operator $\mathcal O\qty(z)$ with conformal weight $h_{\mathcal O}$, which allows for an expansion 
in modes as, 

\begin{align*}
    \mathcal O\qty(z)=\sum\limits_{m\in\mathbb Z}\frac{\mathcal O_m}{z^{m+h_{\mathcal O}}}
\end{align*}

We have the OPE,

\begin{align*}
    T\qty(z_1)\mathcal O\qty(z_2)&=\frac{h_{\mathcal O}}{\qty(z_1-z_2)^2}\mathcal O\qty(z_2)+\frac{1}{\qty(z_2-z_2)}\partial \mathcal O\qty(z_2)
\end{align*}

Where $h=\sqrt{\abs{\Det[h]}}$. We usually fix the Diff$\times$Weyl gauge redundancy by going to the conformal gauge, 
which is equivalent to the choice of complex coordinates, $\sigma^z=z=\sigma^1+\im\sigma^2,\sigma^{\bar z}=\bar z=\sigma^1-\im\sigma^2\rightarrow h_{zz}=h_{\bar z\bar z}=0, h_{z\bar z}=h_{\bar zz}=\frac12$, 
so that the equations of motion take the form,

\begin{align*}
    \fdv{S_{\textnormal P}}{X_\mu}&=0\Rightarrow \bar\partial\partial X^\mu=0\\
    \fdv{S_{\textnormal P}}{h^{ab}}&=0\Rightarrow \partial_aX^\mu\partial_bX_\mu-\qty(1-\delta_{ab})\partial X^\mu\bar\partial X_\mu=0
\end{align*}

About the first equation, the equation of motion of $X$, it's easy to solve it, it's just saying that $X\qty(z,\bar z)=X_L\qty(z)+X_R\qty(\bar z)$. 
But what about the second equation? For $a\neq b$ it's identically zero, but, for $a=b$ we have two linear independent equations,

\begin{align*}
    \partial X^\mu\partial X^\mu=0=\bar\partial X^\mu\bar\partial X_\mu
\end{align*}

Classically the only solution to this is $X\qty(z\bar z)=\textnormal{cte}$. Ok, what about the quantum theory? What happens is, after 
the gauge fixing, we work with the Action,

\begin{align*}
    S_{\textnormal P}&=\frac{1}{2\pi\alpha'}\int\dd[2]{z}g_{\mu\nu}\partial X^\mu\bar\partial X^\nu
\end{align*}

Which gives just \textbf{one} equation of motion,

\begin{align*}
    \bar\partial\partial X^\mu&=0
\end{align*}

Which again, as an operator equation has a trivial solution, $X\qty(z,\bar z)=X_L\qty(z)+X_R\qty(\bar z)$, of course, we have 
the fuzz about the equation of motion not being valid when there are insertions of other operators, but let's just forget this 
now. The great question here is, where did the other equation of motion went? Of course as we first fixed the gauge, to after 
obtain the equations of motion, there's no metric around for us to vary with respect to in order to obtain other equation of motion, and, in 
the canonical procedure it's actually hard to see what is going wrong, but, all lays down to the conformal gauge not being a full gauge fix. It's a 
lot easier to understand why is a equation of motion missing if the gauge is not fully fixed in the path integral formalism, after gauge fixing the 
metric $h_{ab}$ to a fixed metric ${\hat h}_{ab}$, the path integral is,

\begin{align*}
    \mathcal Z&=\int\mathcal D{X}\mathcal D{b}\mathcal D{c}\exp\qty[-\frac{1}{4\pi\alpha'}\int\dd[2]{\sigma}\hat h{\hat h}^{ab}\partial^aX^\mu \partial^bX_\mu-\frac{1}{2\pi}\int\dd[2]{\sigma}\hat hb^{ab}{\hat \nabla}_ac_b]
\end{align*}

what we'll do is to obtain the equations of motion first, then after fix the gauge\footnote{This isn't exactly the right thing to do, what should be done is 
gauge fixing in the path integral formalism and after the introduction of the Faddeev-Popov determinant/ghosts variating with respect to the gauge 
fixed metric, which will give an additional equation of motion $T^m_{ab}+T^g_{ab}=0$, what we'll do here is just }, just a quick reminder, by definition,

\begin{align*}
    T^{ab}&\coloneq-4\pi\frac{1}{\sqrt{h}}\fdv{S_{\textnormal{P}}}{h_{ab}}=-\frac{1}{\alpha'}\qty(\partial^aX^\mu\partial^b X_\mu-\frac12h^{ab}\partial_cX^\mu\partial ^cX_\mu)
\end{align*}

First, let's state a few remarks, we have at our disposal the following algebras,

\begin{align*}
    \comm{L_m}{L_n}&=\qty(m-n)L_{n+m}+\frac{c}{12}m\qty(m^2-1)\delta_{m,-n}\\
    \comm{\alpha^\mu_m}{\alpha^\nu_n}&=mg^{\mu\nu}\delta_{m,-n}\\
    \comm{L_m}{\alpha^\nu_n}&=-n\alpha^\nu_{n+m}
\end{align*}

Which should be interpreted only as elements of a Lie Algebra. But, we seek representations of this algebra as linear 
maps in our physical Hilbert space, as any Hilbert space, it'll contain a notion of inner product, to which can be defined 
adjoint of linear maps. Let's formally manipulate the adjoints of the algebra,

\begin{align*}
    &\begin{cases}
        \comm{L_m}{L_n}^\dagger&=\qty(m-n)L^\dagger_{n+m}+\frac{c}{12}m\qty(m^2-1)\delta_{m,-n}\\
        \comm{\alpha^\mu_m}{\alpha^\nu_n}^\dagger&=mg^{\mu\nu}\delta_{m,-n}\\
        \comm{L_m}{\alpha^\nu_n}^\dagger&=-n\qty(\alpha^\nu_{n+m})^\dagger
    \end{cases}\\
    &\begin{cases}
        \comm{L^\dagger_n}{L^\dagger_m}&=\qty(-n-\qty(-m))L^\dagger_{n+m}+\frac{c}{12}\qty(-n)\qty(\qty(-n)^2-1)\delta_{-n,-\qty(-m)}\\
        \comm{\qty(\alpha^\nu_n)^\dagger}{\qty(\alpha^\mu_m)^\dagger}&=-ng^{\mu\nu}\delta_{-n,-\qty(-m)}\\
        \comm{\qty(\alpha^\nu_n)^\dagger}{L^\dagger_m}=-\comm{L^\dagger_m}{\qty(\alpha^\nu_n)^\dagger}&=-n\qty(\alpha^\nu_{n+m})^\dagger
    \end{cases}
\end{align*}

In this format it's clear to see that the algebra span by $L^\dagger_n,\qty(\alpha_m^\mu)^\dagger$ is isomorphic to the algebra span by $L_n,\alpha_m^\mu$, 
in particular, the isomorphism is given by, $L^\dagger_n\rightarrow L_{-n},\qty(\alpha^\mu_m)\rightarrow\alpha^\mu_{-m}$, as can be seen directly from 
the algebra by making the substitutions,

\begin{align*}
    \comm{L_{-n}}{L_{-m}}&=\qty(-n-\qty(-m))L_{-n-m}+\frac{c}{12}\qty(-n)\qty(\qty(-n)^2-1)\delta_{-n,-\qty(-m)}\\
    \comm{\alpha^\nu_{-n}}{\alpha^\mu_{-m}}&=-ng^{\mu\nu}\delta_{-n,-\qty(-m)}\\
    \comm{L_{-m}}{\alpha^\nu_{-n}}&=n\alpha^\nu_{-n-m}
\end{align*}

Which is exactly the right algebra. We'll suppose that in our representation we have exactly $L^\dagger_n=L_{-n},\qty(\alpha^\mu_n)^\dagger=\alpha^\mu_{-n}$ as hinted 
above. A few comments on the inner product of our Hilbert space, it is non degenerate and non negative. About the non 
negative part, as we have a Lorentzian metric in the target space, at least one of the components of $X$ --- the time-like one --- has a wrong sign in the kinetic term in the 
action, that is, some of the modes will inevitably `\textit{create}' negative norm states, but, these cannot be physical, and therefore must/can 
be removed. One way of doing such is fixing the gauge a priori, which isn't the case, the other way is restricting the polarization vectors 
to which the modes, of $\partial X$ can couple to. That is, as example, if $\alpha^\mu_n$ when acting in the vacuum creates a negative norm state, 
this by itself is not enough to guarantee that it annihilate the vacuum, we have to show that no matter what choice of $\xi_\mu$ is done, $\xi_\mu \alpha^\mu_n$ 
always is a negative norm state, then we can conclude $\alpha^\mu_n$ annihilate the vacuum. Closing these initial remarks, we start by defining the vacuum state, $\Psi_0$, as the one for which,

\begin{align*}
    L_m\Psi_0=0,\ \ \ m\geq -1
\end{align*}

Let's analyze the action of $\xi_\mu\alpha^\mu_n$ in it,

\begin{align*}
    L_m\xi_\mu\alpha_n^\mu\Psi_0&=\xi_\mu\qty(L_m\alpha_n^\mu-\alpha^\mu_n L_m)\Psi_0,\ \ \ m\geq -1\\
    L_m\xi_\mu\alpha_n^\mu\Psi_0&=\xi_\mu\comm{L_m}{\alpha_n^\mu}\Psi_0,\ \ \ m\geq -1\\
    L_m\xi_\mu\alpha_n^\mu\Psi_0&=-n\xi_\mu\alpha_{n+m}^\mu\Psi_0,\ \ \ m\geq -1\numberthis\label{2prim}
\end{align*}

The first step was add $0=\xi_\mu\alpha^\mu_nL_m\Psi_0$, as $L_m,\ m\geq -1$ annihilate the vacuum. In the special case of $n=0$, we have,

\begin{align*}
    L_m\xi_\mu\alpha_0^\mu\Psi_0&=0,\ \ \ m\geq -1
\end{align*}

That implies $\xi_\mu\alpha_0^\mu\Psi_0$ has exactly all the symmetries of the vacuum, hence, it must be proportional to it, $\xi_\mu\alpha_0^\mu\Psi_0=\xi\Psi_0$ for some constant $\xi$, now we have to 
evoke that we do have an inner product $\qty(\cdot,\cdot)$, so that the special case of $n=-m$ in \ref{2prim} is,

\begin{align*}
    L_m\xi_\mu\alpha_{-m}^\mu\Psi_0&=m\xi_\mu\alpha_{0}^\mu\Psi_0,\ \ \ m\geq -1\\
    L_m\xi_\mu\alpha_{-m}^\mu\Psi_0&=m\xi\Psi_0,\ \ \ m\geq -1\\
    \qty(\Psi_0,L_m\xi_\mu\alpha_{-m}^\mu\Psi_0)&=\qty(\Psi_0,m\xi\Psi_0),\ \ \ m\geq -1\\
    \qty(L_m^\dagger\Psi_0,\xi_\mu\alpha_{-m}^\mu\Psi_0)&=m\xi\qty(\Psi_0,\Psi_0),\ \ \ m\geq -1\\
    \qty(L_{-m}\Psi_0,\xi_\mu\alpha_{-m}^\mu\Psi_0)&=m\xi\qty(\Psi_0,\Psi_0),\ \ \ m\geq -1
\end{align*}

For $m=\pm 1$,

\begin{align*}
    \qty(L_{\mp1}\Psi_0,\xi_\mu\alpha_{\mp1}^\mu\Psi_0)&=\pm \xi\qty(\Psi_0,\Psi_0)\\
    \qty(0,\xi_\mu\alpha_{\mp1}^\mu\Psi_0)&=\pm \xi\qty(\Psi_0,\Psi_0)\\
    0&=\pm \xi\qty(\Psi_0,\Psi_0)
\end{align*}

Of course $\qty(\Psi_0,\Psi_0)\neq 0$ otherwise it would be the null state, hence, $\xi=0\Rightarrow \xi_\mu\alpha_0^\mu\Psi_0=\xi\Psi_0=0$. As this 
is valid for any choice of $\xi_\mu$, we just proved that $\alpha^\mu_0\Psi_0=0$.

% Let $\ket 0$ be the vacuum, we act with $\partial X^\mu\qty(z)$ in it, and analyze the following,

% \begin{align*}
%     L_m\partial X^\mu\qty(z)\ket 0&=\oint\limits_{C_z}\frac{\dd{z_1}}{2\pi \im}z_1^{m+1}T\qty(z_1)\partial X^\mu\qty(z)\ket 0
% \end{align*}

% In principle $C_z$ could be any contour encircling the origin, but, as we want to use the $T\partial X$ OPE, we'll choose a contour such that 
% $\norm{z_1}>\norm{z}$, so now,

% \begin{align*}
%     L_m\partial X^\mu\qty(z)\ket 0&=\oint\limits_{C_z}\frac{\dd{z_1}}{2\pi \im}z_1^{m+1}\qty(\frac{\partial X^\mu\qty(z)}{\qty(z_1-z)^2}+\frac{\partial^2 X^\mu\qty(z)}{\qty(z_1-z)}+\textnormal{regular})\ket 0
% \end{align*}

% The integrand clearly has only poles at $z_1=0,z$, so, we just have to compute these two residi and this will be the result of the integral. Working with $m\geq-1$ makes the integral of the 
% regular terms vanish, and also removes all the $z_1=0$ poles,

% \begin{align*}
%     L_m\partial X^\mu\qty(z)\ket 0&=\oint\limits_{C_z}\frac{\dd{z_1}}{2\pi \im}z_1^{m+1}\qty(\frac{\partial X^\mu\qty(z)}{\qty(z_1-z)^2}+\frac{\partial^2 X^\mu\qty(z)}{\qty(z_1-z)})\ket 0,\ \ \ m\geq -1\\
%     L_m\partial X^\mu\qty(z)\ket 0&=\textnormal{Res}_{z_1=z}\frac{z_1^{m+1}}{\qty(z_1-z)^2}\partial X^\mu\qty(z)\ket 0+\textnormal{Res}_{z_1=z}\frac{z_1^{m+1}}{\qty(z_1-z)}\partial^2 X^\mu\qty(z)\ket 0,\ \ \ m\geq -1\\
%     L_m\partial X^\mu\qty(z)\ket 0&=\pdv{}{z_1}\qty(\frac{z_1^{m+1}\qty(z_1-z)^2}{\qty(z_1-z)^2})\eval_{z_1=z}\partial X^\mu\qty(z)\ket 0+\frac{z_1^{m+1}\qty(z_1-z)}{\qty(z_1-z)}\eval_{z_1=z}\partial^2 X^\mu\qty(z)\ket 0,\ \ \ m\geq -1\\
%     L_m\partial X^\mu\qty(z)\ket 0&=\qty(m+1)z^m\partial X^\mu\qty(z)\ket 0+z^{m+1}\partial^2 X^\mu\qty(z)\ket 0,\ \ \ m\geq -1
% \end{align*}

% Writing all of this with the mode expansion $\partial X^\mu\qty(z)=-\im\sqrt{\frac{\alpha'}{2}}\sum\limits_{n\in\mathbb Z}\alpha^\mu_nz^{-n-1}$,

% \begin{align*}
%     \sum\limits_{n\in\mathbb Z}z^{-n-1}L_m \alpha_n^\mu\ket 0&=\sum\limits_{n\in\mathbb Z}z^{-n-1}\qty(m+1)z^m\alpha_n^\mu\ket 0+\sum\limits_{n\in\mathbb Z}\qty(-n-1)z^{-n-2}z^{m+1}\alpha_n^\mu\ket 0,\ \ \ m\geq -1\\
%     \sum\limits_{n\in\mathbb Z}z^{-n-1}L_m \alpha_n^\mu\ket 0&=\sum\limits_{n\in\mathbb Z}z^{m-n-1}\qty(m+1)\alpha_n^\mu\ket 0+\sum\limits_{n\in\mathbb Z}\qty(-n-1)z^{m-n-1}\alpha_n^\mu\ket 0,\ \ \ m\geq -1
% \end{align*}

% Relabeling the $n$ index in the right-hand side,

% \begin{align*}
%     \sum\limits_{n\in\mathbb Z}z^{-n-1}L_m \alpha_n^\mu\ket 0&=\sum\limits_{n\in\mathbb Z}z^{-n-1}\qty(m+1)\alpha_{n+m}^\mu\ket 0+\sum\limits_{n\in\mathbb Z}\qty(-n-m-1)z^{-n-1}\alpha_{n+m}^\mu\ket 0,\ \ \ m\geq -1\\
%     L_m \alpha_n^\mu\ket 0&=\qty(m+1)\alpha_{n+m}^\mu\ket 0+\qty(-n-m-1)\alpha_{n+m}^\mu\ket 0,\ \ \ m\geq -1
% \end{align*}

\probitem{}