\problem{}
\probitem{}

First, let's state a few remarks, we have at our disposal the following algebras,

\begin{align*}
    \comm{L_m}{L_n}&=\qty(m-n)L_{n+m}+\frac{c}{12}m\qty(m^2-1)\delta_{m,-n}\\
    \comm{\alpha^\mu_m}{\alpha^\nu_n}&=mg^{\mu\nu}\delta_{m,-n}\\
    \comm{L_m}{\alpha^\nu_n}&=-n\alpha^\nu_{n+m}
\end{align*}

Which should be interpreted only as elements of a Lie Algebra. But, we seek representations of this algebra as linear 
maps in our physical Hilbert space, as any Hilbert space, it'll contain a notion of inner product, to which can be defined 
adjoint of linear maps. Let's formally manipulate the adjoints of the algebra,

\begin{align*}
    &\begin{cases}
        \comm{L_m}{L_n}^\dagger&=\qty(m-n)L^\dagger_{n+m}+\frac{c}{12}m\qty(m^2-1)\delta_{m,-n}\\
        \comm{\alpha^\mu_m}{\alpha^\nu_n}^\dagger&=mg^{\mu\nu}\delta_{m,-n}\\
        \comm{L_m}{\alpha^\nu_n}^\dagger&=-n\qty(\alpha^\nu_{n+m})^\dagger
    \end{cases}\\
    &\begin{cases}
        \comm{L^\dagger_n}{L^\dagger_m}&=\qty(-n-\qty(-m))L^\dagger_{n+m}+\frac{c}{12}\qty(-n)\qty(\qty(-n)^2-1)\delta_{-n,-\qty(-m)}\\
        \comm{\qty(\alpha^\nu_n)^\dagger}{\qty(\alpha^\mu_m)^\dagger}&=-ng^{\mu\nu}\delta_{-n,-\qty(-m)}\\
        \comm{\qty(\alpha^\nu_n)^\dagger}{L^\dagger_m}=-\comm{L^\dagger_m}{\qty(\alpha^\nu_n)^\dagger}&=-n\qty(\alpha^\nu_{n+m})^\dagger
    \end{cases}
\end{align*}

In this format it's clear to see that the algebra span by $L^\dagger_n,\qty(\alpha_m^\mu)^\dagger$ is isomorphic to the algebra span by $L_n,\alpha_m^\mu$, 
in particular, the isomorphism is given by, $L^\dagger_n\rightarrow L_{-n},\qty(\alpha^\mu_m)\rightarrow\alpha^\mu_{-m}$, as can be seen directly from 
the algebra by making the substitutions,

\begin{align*}
    \comm{L_{-n}}{L_{-m}}&=\qty(-n-\qty(-m))L_{-n-m}+\frac{c}{12}\qty(-n)\qty(\qty(-n)^2-1)\delta_{-n,-\qty(-m)}\\
    \comm{\alpha^\nu_{-n}}{\alpha^\mu_{-m}}&=-ng^{\mu\nu}\delta_{-n,-\qty(-m)}\\
    \comm{L_{-m}}{\alpha^\nu_{-n}}&=n\alpha^\nu_{-n-m}
\end{align*}

Which is exactly the right algebra. We'll suppose that in our representation we have exactly $L^\dagger_n=L_{-n},\qty(\alpha^\mu_n)^\dagger=\alpha^\mu_{-n}$ as hinted 
above. A few comments on the inner product of our Hilbert space, it is non degenerate and non negative. About the non 
negative part, as we have a Lorentzian metric in the target space, at least one of the components of $X$ --- the time-like one --- has a wrong sign in the kinetic term in the 
action, that is, some of the modes will inevitably `\textit{create}' negative norm states, but, these cannot be physical, and therefore must/can 
be removed. One way of doing such is fixing the gauge a priori, which isn't the case, the other way is restricting the polarization vectors 
to which the modes, of $\partial X$ can couple to. That is, as example, if $\alpha^\mu_n$ when acting in the vacuum creates a negative norm state, 
this by itself is not enough to guarantee that it annihilate the vacuum, we have to show that no matter what choice of $\xi_\mu$ is done, $\xi_\mu \alpha^\mu_n$ 
always is a negative norm state, then we can conclude $\alpha^\mu_n$ annihilate the vacuum. Closing these initial remarks, we start by defining the vacuum state, $\Psi_0$, as the one for which,

\begin{align*}
    L_m\Psi_0=0,\ \ \ m\geq -1
\end{align*}

Let's analyze the action of $\xi_\mu\alpha^\mu_n$ in it,

\begin{align*}
    L_m\xi_\mu\alpha_n^\mu\Psi_0&=\xi_\mu\qty(L_m\alpha_n^\mu-\alpha^\mu_n L_m)\Psi_0,\ \ \ m\geq -1\\
    L_m\xi_\mu\alpha_n^\mu\Psi_0&=\xi_\mu\comm{L_m}{\alpha_n^\mu}\Psi_0,\ \ \ m\geq -1\\
    L_m\xi_\mu\alpha_n^\mu\Psi_0&=-n\xi_\mu\alpha_{n+m}^\mu\Psi_0,\ \ \ m\geq -1\numberthis\label{2prim}
\end{align*}

The first step was add $0=\xi_\mu\alpha^\mu_nL_m\Psi_0$, as $L_m,\ m\geq -1$ annihilate the vacuum. In the special case of $n=0$, we have,

\begin{align*}
    L_m\xi_\mu\alpha_0^\mu\Psi_0&=0,\ \ \ m\geq -1
\end{align*}

That implies $\xi_\mu\alpha_0^\mu\Psi_0$ has exactly all the symmetries of the vacuum, hence, it must be proportional to it, $\xi_\mu\alpha_0^\mu\Psi_0=\xi\Psi_0$ for some constant $\xi$, now we have to 
evoke that we do have an inner product $\qty(\cdot,\cdot)$, so that the special case of $n=-m$ in \ref{2prim} is,

\begin{align*}
    L_m\xi_\mu\alpha_{-m}^\mu\Psi_0&=m\xi_\mu\alpha_{0}^\mu\Psi_0,\ \ \ m\geq -1\\
    L_m\xi_\mu\alpha_{-m}^\mu\Psi_0&=m\xi\Psi_0,\ \ \ m\geq -1\\
    \qty(\Psi_0,L_m\xi_\mu\alpha_{-m}^\mu\Psi_0)&=\qty(\Psi_0,m\xi\Psi_0),\ \ \ m\geq -1\\
    \qty(L_m^\dagger\Psi_0,\xi_\mu\alpha_{-m}^\mu\Psi_0)&=m\xi\qty(\Psi_0,\Psi_0),\ \ \ m\geq -1\\
    \qty(L_{-m}\Psi_0,\xi_\mu\alpha_{-m}^\mu\Psi_0)&=m\xi\qty(\Psi_0,\Psi_0),\ \ \ m\geq -1
\end{align*}

For $m=\pm 1$,

\begin{align*}
    \qty(L_{\mp1}\Psi_0,\xi_\mu\alpha_{\mp1}^\mu\Psi_0)&=\pm \xi\qty(\Psi_0,\Psi_0)\\
    \qty(0,\xi_\mu\alpha_{\mp1}^\mu\Psi_0)&=\pm \xi\qty(\Psi_0,\Psi_0)\\
    0&=\pm \xi\qty(\Psi_0,\Psi_0)
\end{align*}

Of course $\qty(\Psi_0,\Psi_0)\neq 0$ otherwise it would be the null state, hence, $\xi=0\Rightarrow \xi_\mu\alpha_0^\mu\Psi_0=\xi\Psi_0=0$. As this 
is valid for any choice of $\xi_\mu$, we just proved that $\alpha^\mu_0\Psi_0=0$.

% Let $\ket 0$ be the vacuum, we act with $\partial X^\mu\qty(z)$ in it, and analyze the following,

% \begin{align*}
%     L_m\partial X^\mu\qty(z)\ket 0&=\oint\limits_{C_z}\frac{\dd{z_1}}{2\pi \im}z_1^{m+1}T\qty(z_1)\partial X^\mu\qty(z)\ket 0
% \end{align*}

% In principle $C_z$ could be any contour encircling the origin, but, as we want to use the $T\partial X$ OPE, we'll choose a contour such that 
% $\norm{z_1}>\norm{z}$, so now,

% \begin{align*}
%     L_m\partial X^\mu\qty(z)\ket 0&=\oint\limits_{C_z}\frac{\dd{z_1}}{2\pi \im}z_1^{m+1}\qty(\frac{\partial X^\mu\qty(z)}{\qty(z_1-z)^2}+\frac{\partial^2 X^\mu\qty(z)}{\qty(z_1-z)}+\textnormal{regular})\ket 0
% \end{align*}

% The integrand clearly has only poles at $z_1=0,z$, so, we just have to compute these two residi and this will be the result of the integral. Working with $m\geq-1$ makes the integral of the 
% regular terms vanish, and also removes all the $z_1=0$ poles,

% \begin{align*}
%     L_m\partial X^\mu\qty(z)\ket 0&=\oint\limits_{C_z}\frac{\dd{z_1}}{2\pi \im}z_1^{m+1}\qty(\frac{\partial X^\mu\qty(z)}{\qty(z_1-z)^2}+\frac{\partial^2 X^\mu\qty(z)}{\qty(z_1-z)})\ket 0,\ \ \ m\geq -1\\
%     L_m\partial X^\mu\qty(z)\ket 0&=\textnormal{Res}_{z_1=z}\frac{z_1^{m+1}}{\qty(z_1-z)^2}\partial X^\mu\qty(z)\ket 0+\textnormal{Res}_{z_1=z}\frac{z_1^{m+1}}{\qty(z_1-z)}\partial^2 X^\mu\qty(z)\ket 0,\ \ \ m\geq -1\\
%     L_m\partial X^\mu\qty(z)\ket 0&=\pdv{}{z_1}\qty(\frac{z_1^{m+1}\qty(z_1-z)^2}{\qty(z_1-z)^2})\eval_{z_1=z}\partial X^\mu\qty(z)\ket 0+\frac{z_1^{m+1}\qty(z_1-z)}{\qty(z_1-z)}\eval_{z_1=z}\partial^2 X^\mu\qty(z)\ket 0,\ \ \ m\geq -1\\
%     L_m\partial X^\mu\qty(z)\ket 0&=\qty(m+1)z^m\partial X^\mu\qty(z)\ket 0+z^{m+1}\partial^2 X^\mu\qty(z)\ket 0,\ \ \ m\geq -1
% \end{align*}

% Writing all of this with the mode expansion $\partial X^\mu\qty(z)=-\im\sqrt{\frac{\alpha'}{2}}\sum\limits_{n\in\mathbb Z}\alpha^\mu_nz^{-n-1}$,

% \begin{align*}
%     \sum\limits_{n\in\mathbb Z}z^{-n-1}L_m \alpha_n^\mu\ket 0&=\sum\limits_{n\in\mathbb Z}z^{-n-1}\qty(m+1)z^m\alpha_n^\mu\ket 0+\sum\limits_{n\in\mathbb Z}\qty(-n-1)z^{-n-2}z^{m+1}\alpha_n^\mu\ket 0,\ \ \ m\geq -1\\
%     \sum\limits_{n\in\mathbb Z}z^{-n-1}L_m \alpha_n^\mu\ket 0&=\sum\limits_{n\in\mathbb Z}z^{m-n-1}\qty(m+1)\alpha_n^\mu\ket 0+\sum\limits_{n\in\mathbb Z}\qty(-n-1)z^{m-n-1}\alpha_n^\mu\ket 0,\ \ \ m\geq -1
% \end{align*}

% Relabeling the $n$ index in the right-hand side,

% \begin{align*}
%     \sum\limits_{n\in\mathbb Z}z^{-n-1}L_m \alpha_n^\mu\ket 0&=\sum\limits_{n\in\mathbb Z}z^{-n-1}\qty(m+1)\alpha_{n+m}^\mu\ket 0+\sum\limits_{n\in\mathbb Z}\qty(-n-m-1)z^{-n-1}\alpha_{n+m}^\mu\ket 0,\ \ \ m\geq -1\\
%     L_m \alpha_n^\mu\ket 0&=\qty(m+1)\alpha_{n+m}^\mu\ket 0+\qty(-n-m-1)\alpha_{n+m}^\mu\ket 0,\ \ \ m\geq -1
% \end{align*}

\probitem{}