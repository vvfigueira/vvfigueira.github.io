\problem{}
\probitem{}

We have already done this for general $N$ real fermions in problem \ref{3d}. Let us just cite here the 
results,

\begin{align*}
    j^{\qty[ab]}\qty(z)&=\frac1 2\tensor{T}{^{\qty[ab]}_k_l}\cnord{\psi^k\psi^l}\qty(z)\\
    j^A\qty(z)&=\frac1 2\tensor{T}{^A_k_l}\cnord{\psi^k\psi^l}\qty(z)\\
    j^A\qty(z_1)j ^B\qty(z_2)&=\frac{\im\tensor{f}{^A^B_C}}{\qty(z_1-z_2)}j^C\qty(z_2)+\frac{\Tr\qty[T^A T^B]}{2\qty(z_1-z_2)^2}+\textnormal{regular}
\end{align*}

Where in this case $\tensor{T}{^{\qty[ab]}_k_l}=\tensor{T}{^A_k_l}$ are the generators of $SO\qty(4)$, and $\tensor{f}{^A^B_C}$ are the structure constants such 
that $a,b,k,l=1,2,3,4$ and $A,B,C=1,2,3,4,5,6$. That $\psi^k$ transforms in the fundamental representation is a 
trivial fact, due to they consisting of a real --- the Grassmannian nature of them do not interfere --- vector of four components, exactly the 
representation on which the $4x4$ orthogonal, determinant one matrices act. If it form a $SO\qty(4)$ Kac-Moody algebra can 
be already seem from our derivation of a generic $SO\qty(N)$ Kac-Moody algebra, which we cited above.

\probitem{}

To keep the conventions already established in problems \ref{3c},\ref{4c}, we're going to choose the following 
for grouping the four fermions into a pair of complex ones,

\begin{align*}
    {\Psi}^{\dot 1}=\frac{1}{\sqrt{2}}\psi^1-\frac{\im}{\sqrt2}\psi^2\\
    {\tilde\Psi}^{\dot 1}=\frac{1}{\sqrt{2}}\psi^1+\frac{\im}{\sqrt2}\psi^2\\
    {\Psi}^{\dot 2}=\frac{1}{\sqrt{2}}\psi^3-\frac{\im}{\sqrt2}\psi^4\\
    {\tilde\Psi}^{\dot 2}=\frac{1}{\sqrt{2}}\psi^3+\frac{\im}{\sqrt2}\psi^4
\end{align*}

This choice makes the following,


\begin{align*}
    -\cnord{{\tilde\Psi}^{\dot 1}\Psi^{\dot 1}}&=-\frac12\cnord{\psi^1\psi^1+\im\psi^2\psi^1-\psi^1\im\psi^2+\psi^2\psi^2}\\
    -\cnord{{\tilde\Psi}^{\dot 1}\Psi^{\dot 1}}&=\im\cnord{\psi^1\psi^2}
\end{align*}

As outcome of problems \ref{3c},\ref{4c} this should be seen as a good relation to hold. To bosonize each pair ${\tilde\Psi}^{\dot a}\Psi^{\dot a}$ --- We'll use $\dot a,\dot b=1,2$ for the complex fermions we 
leave $j,k,l=1,2,3,4$ for the real ones ---, 
we follow the ideas of problem \ref{4}, to each pair of complex fermions we attribute a chiral boson $X^{\dot a}\qty(z)$, then 
we should have, as was already argued in problem \ref{4d}, the following correspondence,

\begin{subequations}\label{teste1}
\begin{align}
    \cnord{\exp\qty(\im X^{\dot a})}&= \Psi^{\dot a}\\
    \cnord{\exp\qty(-\im X^{\dot a})}&= {\tilde\Psi}^{\dot a}
\end{align}
\end{subequations}

We choose a particular way of bosonizing, a more generic choice could be $\cnord{\exp\qty(\im X^{\dot a})}=\alpha \Psi^{\dot a},\cnord{\exp\qty(-\im X^{\dot a})}=\alpha^{-1} {\tilde\Psi}^{\dot a}$. 
But what matters is that now we have a description of the fermionic theory by a bosonic one, consisting of two compactified chiral bosons $X^{\dot a}\qty(z)=X^{\dot a}\qty(z)+2\pi$. 

\probitem{}

The symmetry of the bosonic theory is,

\begin{align*}
    X^{\dot a}\qty(z)\rightarrow X^{\dot a}\qty(z)+t^{\dot a}\ \qty(\textnormal{mod }2\pi)
\end{align*}

What from \ref{teste1} can be seen as a realization of a $U\qty(1)$ symmetry for each index $\dot a$, as there is 
two of them, and the symmetries are disconnected, this means a $U\qty(1)\times U\qty(1)$ symmetry,

\begin{align*}
    \Psi^{\dot a}=\cnord{\exp\qty(\im X^{\dot a})}&\rightarrow\cnord{\exp\qty(\im X^{\dot a}+\im t^{\dot a})},\ \ \ t^{\dot a}\in[0,2\pi)\\
    \Psi^{\dot a}&\rightarrow\exp\qty(\im t^{\dot a})\cnord{\exp\qty(\im X^{\dot a})},\ \ \ t^{\dot a}\in[0,2\pi)\\
    \Psi^{\dot a}&\rightarrow\exp\qty(\im t^{\dot a})\Psi^{\dot a},\ \ \ t^{\dot a}\in[0,2\pi)
\end{align*}

From the last line is clear that we have a $U\qty(1)$ symmetry for each index, specially due to the chiral boson being 
compactified. Of course, this transformation above implies the transformation of the complex one also,

\begin{align*}
    {\tilde\Psi}^{\dot a}=\cnord{\exp\qty(-\im X^{\dot a})}&\rightarrow\cnord{\exp\qty(-\im X^{\dot a}-\im t^{\dot a})},\ \ \ t^{\dot a}\in[0,2\pi)\\
    {\tilde\Psi}^{\dot a}&\rightarrow\exp\qty(-\im t^{\dot a})\cnord{\exp\qty(-\im X^{\dot a})},\ \ \ t^{\dot a}\in[0,2\pi)\\
    {\tilde\Psi}^{\dot a}&\rightarrow\exp\qty(-\im t^{\dot a}){\tilde\Psi}^{\dot a},\ \ \ t^{\dot a}\in[0,2\pi)
\end{align*}

So now it's clear that ${\tilde\Psi}^{\dot a}$ have charge $-1$ with the respective $U\qty(1)$'s, and $\Psi^{\dot a}$ have charge $+1$ with the 
respective $U\qty(1)$'s. What we have shown is that $U\qty(1)\times U\qty(1)\subset SO\qty(4)$, so it's possible for us to work out the action of each of these $U\qty(1)$'s over the vector representation, 
let's name $U\qty(1)\times U\qty(1)$ by $U\qty(1)_{\dot 1}\times U\qty(1)_{\dot 2}$, the nomenclature should be self evident. With a little 
help from group theory, it's known that \textbf{locally} $SO\qty(4)$ is isomorphic to $SU\qty(2)\times SU\qty(2)$, that is not true globally, as the 
two groups have different topologies. But nevertheless, under a quantization, any global classical symmetry has to under go a double cover, here is no 
different, our symmetry group $SO\qty(4)$ has to under go to a double cover, which is by definition the group $Spin\qty(4)$, which luckily is 
globally isomorphic to $SU\qty(2)\times SU\qty(2)$. Just recalling everything, our classical theory of four fermions 
enjoyed a $SO\qty(4)$ symmetry, that is, generically if we stayed at classical analysis, we would be interested in 
operators which transform under representations of $SO\qty(4)$, but, as ultimately we're interested in the quantum theory, 
this enlargers the symmetry group, and we actually have to search for operators which transform in representations of the double cover of $SO\qty(4)$, 
that is $Spin\qty(2)\simeq SU\qty(2)\times SU\qty(2)$, while $SO\qty(4)$ just allows for integer charges/spins, it's double cover 
allows for half-integers charges/spins, this is a well known fact, as $SU\qty(2)$ possesses representations which have a half integer 
quadratic Casimir operator, the fundamental representation as example. But, we just found an explicit representation of a subgroup of $SO\qty(4)$ by the 
bosonization of the fermionic theory, $U\qty(1)_{\dot 1}\times U\qty(1)_{\dot 2}\subset SO\qty(4)\Rightarrow U\qty(1)_{\dot 1}\times U\qty(1)_{\dot 2}\subset SU\qty(2)\times SU\qty(2)$, 
let us just stress some facts here, we have found \textbf{two abelian }$\vb{U\qty(1)}$ \textbf{subgroups of }$\vb{SU\qty(2)\times SU\qty(2)}$, of course $U\qty(1)$ 
by itself is an abelian group, but, they're abelian \textbf{among themselves}, this is what the $\times$ is saying, and this can be seen both from the bosonization 
as well from the fermionic representation, the transformations for $X^{\dot a}\rightarrow X^{\dot a}+t^{\dot a}\ \qty(\textnormal{mod }2\pi)$ are totally independent 
for each index, as well as,
\begin{align*}
    \begin{cases}
        \Psi^{\dot a}&\rightarrow\exp\qty(\im t^{\dot a})\Psi^{\dot a}\\
        {\tilde\Psi}^{\dot a}&\rightarrow\exp\qty(-\im t^{\dot a}){\tilde\Psi}^{\dot a}
    \end{cases}
\end{align*}
are independent for each index. This raises an eyebrow, as certainly $SO\qty(4)$ is not abelian, an neither is $SU\qty(2)\times SU\qty(2)$, but, 
the double cover has naturally an $\times$ in the definition, this means it's composed of two copies of $SU\qty(2)$ which are abelian among themselves! 
It's clear that $SU\qty(2)$ itself is not abelian, and also is clear that $U\qty(1)\subset SU\qty(2)$, thus, the only possible way we could extract two abelian 
copies of $U\qty(1)$ from $SU\qty(2)\times SU\qty(2)$ is if we extracted one copy of each, this means we have a bigger identification of the symmetry $U\qty(1)_{\dot1}\times U\qty(1)_{\dot 2}$ as being,

\begin{align*}
    U\qty(1)_{\dot 1}\times U\qty(2)_{\dot 2}\subset SU\qty(2)_{\dot 1}\times SU\qty(2)_{\dot 2}
\end{align*}

We just identified which $U\qty(1)$ came from which $SU\qty(1)$, this particular $U\qty(1)_{\dot a}\subset SU\qty(2)_{\dot a}$ we picked up 
can be seen to be represented in a diagonal form in $\Psi^{\dot a},{\tilde\Psi}^{\dot a}$, as at in each $\mathfrak{su}\qty(2)_{\dot a}$ we 
can diagonalize at maximum one generator and the single Casimir --- which will have positive half-integer eigenvalues for the Casimir $j$, 
and $m=-j,-j+1,\cdots,j$ for the chosen generator ---, it's clear that 
\begin{align*}
    \begin{cases}
        \Psi^{\dot a}&\rightarrow\exp\qty(\im j t^{\dot a})\Psi^{\dot a},\ \ \ j=1\\
        {\tilde\Psi}^{\dot a}&\rightarrow\exp\qty(-\im j t^{\dot a}){\tilde\Psi}^{\dot a},\ \ \ j=1
    \end{cases}
\end{align*}
implies they transform in the spin $1$ representation of each particular $SU\qty(2)$. After all these remarks, 
we finally know how to ask what we want, we want operators that transform under the spin $\frac12$ representation 
of the double cover of $SO\qty(4)$, but now we know what this means, this means we want operators that transform as,
\begin{align*}
    \begin{cases}
        \Phi^{\dot a}&\rightarrow\exp\qty(\im j t^{\dot a})\Phi^{\dot a}\\
        {\tilde\Phi}^{\dot a}&\rightarrow\exp\qty(-\im j t^{\dot a}){\tilde\Phi}^{\dot a}
    \end{cases}
\end{align*}
for $j=\frac12$. As for each index this is a spin $\frac12$ representation of $SU\qty(2)_{\dot a}$. But we have a natural way of 
analyzing the transformations being done by the diagonal generator of $SU\qty(2)_{\dot a}$, the symmetry in the bosonic theory! 
$X^{\dot a}\rightarrow X^{\dot a}+t^{\dot a}\ \qty(\textnormal{mod }2\pi)$ correspond to a transformation generated by the diagonal 
generator of each $SU\qty(2)_{\dot a}$, thus, is easy to get the desired result, consider,
\begin{align*}
    \Phi^{\dot a}=\cnord{\exp\qty(\frac\im2 X^{\dot a})}&\rightarrow\cnord{\exp\qty(\frac\im2 X^{\dot a}+\frac\im2 t^{\dot a})}\\
    \Phi^{\dot a}&\rightarrow\exp\qty(\frac\im2 t^{\dot a})\cnord{\exp\qty(\frac12\im X^{\dot a})}\\
    \Phi^{\dot a}&\rightarrow\exp\qty(\frac\im2 t^{\dot a})\Phi^{\dot a}
\end{align*}
This is exactly the desired transformation with $j=\frac12$, what means $\Phi^{\dot a}$ transform as the spin $\frac12$ representation 
of the double cover of $SO\qty(4)$, the same can be done for,
\begin{align*}
    {\tilde\Phi}^{\dot a}=\cnord{\exp\qty(-\frac\im2 X^{\dot a})}&\rightarrow\cnord{\exp\qty(-\frac\im2 X^{\dot a}-\frac\im2 t^{\dot a})}\\
    {\tilde\Phi}^{\dot a}&\rightarrow\exp\qty(-\frac\im2 t^{\dot a})\cnord{\exp\qty(-\frac\im2 X^{\dot a})}\\
    {\tilde\Phi}^{\dot a}&\rightarrow\exp\qty(-\frac\im2 t^{\dot a}){\tilde\Phi}^{\dot a}
\end{align*}

These have not a polinomial definition on the fermionic theory, but, the bosonic theory provides a good representation of these spin $\frac12$ operators,

\begin{align*}
    \cnord{\exp\qty(\frac\im2 X^{\dot a})},\ \ \ \cnord{\exp\qty(-\frac\im2 X^{\dot a})}
\end{align*}

\textbf{BUT}, one might protest against these operators: ``\textit{They aren't well defined! Notice the under the symmetry }$X^{\dot a}\equiv X^{\dot a}+2\pi$\textit{, they adquire an extra minus sign!}''. 
Well, this is an honest worry, which happily has a solution, when the double cover is done in the fermionic symmetry $SO\qty(4)$, 
it has to be done also in the bosonic symmetry $X^{\dot a}\equiv X^{\dot a}+2\pi$, this is equivalent to the well known fact that $2\pi$ is a full 
rotation in $SO\qty(3)$ but isn't in $SU\qty(2)$, despite the two being locally isomorphic. The double cover in the bosonic 
symmetry is rather simple, $X^{\dot a}+2\pi\equiv\pm X^{\dot a}\Rightarrow X^{\dot a}+4\pi\equiv X^{\dot a}$, exactly happens in $SO\qty(3)\rightarrow SU\qty(2)$, 
as now our quantum symmetry of the bosonic theory is $X^{\dot a}+2\pi\equiv\pm X^{\dot a}$, the operators of half spin are indeed well defined!