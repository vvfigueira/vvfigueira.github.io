\problem{}
\probitem{}

Our action is,

\begin{align*}
    S&=\frac{1}{4\pi}\int\dd[2]{z}\psi\bar\partial\psi
\end{align*}

To obtain the equation of motion is simple, first we set up the path integral of a total derivative, which is zero, the argument of the total derivative 
we set to $\exp\qty(-S)\mathcal O$, where $\mathcal O$ is any combination of local fields that does not contain $\psi\qty(z_1,{\bar z}_1)$, then,

\begin{align*}
    0&=\int\mathcal D\psi\fdv{}{\psi\qty(z_1,\bar z_1)}\qty[\exp\qty(-S)\mathcal O]\\
    0&=\int\mathcal D\psi\fdv{}{\psi\qty(z_1,\bar z_1)}\qty[\exp\qty(-S)]\mathcal O
\end{align*}

As $\fdv{}{\psi\qty(z_1,{\bar z}_1)}\mathcal O$ is zero except for $\mathcal O=\psi\qty(z_1,{\bar z}_1)$, to which it's $\delta^{\qty(2)}\qty(0)$, but, that's not the case,

\begin{align*}
    0&=\int\mathcal D\psi\fdv{}{\psi\qty(z_1,\bar z_1)}\qty[\exp\qty(-S)]\mathcal O\\
    0&=-\frac{1}{4\pi}\int\mathcal D\psi\fdv{}{\psi\qty(z_1,\bar z_1)}\qty[\int\dd[2]{z_2}\psi\qty(z_2,\bar z_2)\partial_{{\bar z}_2}\psi\qty(z_2,{\bar z}_2)]\mathcal O\exp\qty(-S)\\
    0&=-\frac{1}{4\pi}\int\mathcal D\psi\int\dd[2]{z_2}\qty[\delta^{\qty(2)}\qty(z_2-z_1)\partial_{{\bar z}_2}\psi\qty(z_2,{\bar z}_2)-\psi\qty(z_2,\bar z_2)\partial_{{\bar z}_2}\delta^{\qty(2)}\qty(z_2-z_1)]\mathcal O\exp\qty(-S)
\end{align*}

As long as we're dealing with the closed fermion, we need not to worry about the boundary conditions, as there is no boundary, so,

\begin{align*}
    0&=-\frac{1}{4\pi}\int\mathcal D\psi\int\dd[2]{z_2}\qty[\delta^{\qty(2)}\qty(z_2-z_1)\partial_{{\bar z}_2}\psi\qty(z_2,{\bar z}_2)+\partial_{{\bar z}_2}\psi\qty(z_2,\bar z_2)\delta^{\qty(2)}\qty(z_2-z_1)]\mathcal O\exp\qty(-S)\\
    0&=-\frac{1}{2\pi}\int\mathcal D\psi\partial_{{\bar z}_1}\psi\qty(z_1,{\bar z}_1)\mathcal O\exp\qty(-S)
\end{align*}

What in the Operator formalism would account for some radially ordered expectation value under some state, which is specified with the boundary conditions on the path integral itself, 

\begin{align*}
    \partial_{{\bar z}_1}\expval{\psi\qty(z_1,{\bar z}_1)\mathcal O}&=0
\end{align*}

As both the operator $\mathcal O$ and the state are arbitrary, as long as there is no insertion of $\psi\qty(z_1,{\bar z}_1)$, we conclude the 
equation of motion in the operator form is just,

\begin{align*}
    \bar\partial\psi=0\Rightarrow \psi\qty(z,\bar z)\equiv\psi\qty(z)
\end{align*}

That is, $\psi$ is, at least, a meromorphic operator/function. Now, let's repeat this to obtain the two point function, this time we include in 
$\mathcal O$ a single factor of $\psi\qty(z_1,{\bar z}_1)$ disguised as $\psi\qty(z_2,{\bar z}_2)$, this also can be phrased as we being interested 
in the limit $z_2\rightarrow z_1$,

\begin{align*}
    0&=\int\mathcal D\psi\fdv{}{\psi\qty(z_1,\bar z_1)}\qty[\exp\qty(-S)\psi\qty(z_2,{\bar z}_2)\mathcal O]\\
    0&=\int\mathcal D\psi\fdv{}{\psi\qty(z_1,\bar z_1)}\qty[\exp\qty(-S)\psi\qty(z_2,{\bar z}_2)]\mathcal O\\
    0&=\int\mathcal D\psi\qty{\fdv{}{\psi\qty(z_1,\bar z_1)}\qty[\exp\qty(-S)]\psi\qty(z_2,{\bar z}_2)+\fdv{}{\psi\qty(z_1,{\bar z}_1)}\qty[\psi\qty(z_2,{\bar z}_2)]\exp\qty(-S)}\mathcal O\\
    0&=\int\mathcal D\psi\qty{\fdv{}{\psi\qty(z_1,\bar z_1)}\qty[-S]\psi\qty(z_2,{\bar z}_2)+\delta^{\qty(2)}\qty(z_2-z_1)}\exp\qty(-S)\mathcal O
\end{align*}

As we derived before, $\fdv{}{\psi\qty(z_1,\bar z_1)}\qty[-S]=-\frac{1}{2\pi}\partial_{{\bar z}_1}\psi\qty(z_1,{\bar z}_1)$,

\begin{align*}
    0&=\int\mathcal D\psi\qty{-\frac{1}{2\pi}\partial_{{\bar z}_1}\psi\qty(z_1,{\bar z}_1)\psi\qty(z_2,{\bar z}_2)+\delta^{\qty(2)}\qty(z_2-z_1)}\exp\qty(-S)\mathcal O
\end{align*}

Which is, translating to the operator formalism,

\begin{align*}
    \partial_{{\bar z}_1}\expval{\psi\qty(z_1)\psi\qty(z_2)\mathcal O}&=\expval{2\pi\delta^{\qty(2)}\qty(z_2-z_1)\mathcal O}\\
    \partial_{{\bar z}_1}\psi\qty(z_1)\psi\qty(z_2)&=2\pi\delta^{\qty(2)}\qty(z_2-z_1),\ \ \ \abs{z_1}\geq\abs{z_2}
\end{align*}

So that we can interpret this as a operator equality. The condition $\abs{z_1}\geq \abs{z_2}$ is due to the implicit radial ordering in the expectation 
value, and also notice that in the path integral formulation $\psi=\psi\qty(z_1,{\bar z}_1)$, but in the operator formalism $\psi=\psi\qty(z_1)$, as every operator is 
always `\textit{on-shell}', while the path integral integrands aren't. Now we integrate this two point function, pick any compact closed region $R$ in the complex plane which 
contains $z_2$ and not any other of the points of the insertions $\mathcal O$, with also the boundary being a continuous curve, for all $m\in\mathbb N$ the following is true,

\begin{align*}
    \partial_{{\bar z}_1}\expval{\psi\qty(z_1)\psi\qty(z_2)\mathcal O}&=2\pi\delta^{\qty(2)}\qty(z_2-z_1)\expval{\mathcal O}\\
    \qty(z_1-z_2)^m\partial_{{\bar z}_1}\expval{\psi\qty(z_1)\psi\qty(z_2)\mathcal O}&=2\pi\qty(z_1-z_2)^m\delta^{\qty(2)}\qty(z_2-z_1)\expval{\mathcal O},\ \ \ m\in\mathbb N\\
    \int\limits_R\dd[2]{z_1}\qty(z_1-z_2)^m\partial_{{\bar z}_1}\expval{\psi\qty(z_1)\psi\qty(z_2)\mathcal O}&=2\pi\int\limits_R\dd[2]{z_1}\qty(z_1-z_2)^m\delta^{\qty(2)}\qty(z_2-z_1)\expval{\mathcal O},\ \ \ m\in\mathbb N\\
    \int\limits_R\dd[2]{z_1}\partial_{{\bar z}_1}\qty{\qty(z_1-z_2)^m\expval{\psi\qty(z_1)\psi\qty(z_2)\mathcal O}}&=2\pi\qty(z_1-z_2)^m\eval_{z_1=z_2}\expval{\mathcal O},\ \ \ m\in\mathbb N
\end{align*}

In the last line we used the fact `\textit{trivial}' fact that $\partial_{{\bar z}_1}\qty(z_1-z_2)^m=0$, which is valid for $m\in\mathbb N$, but isn't for 
negative non-integer values.

\begin{align*}
    \int\limits_R\dd[2]{z_1}\partial_{{\bar z}_1}\qty{\qty(z_1-z_2)^m\expval{\psi\qty(z_1)\psi\qty(z_2)\mathcal O}}&=2\pi\qty(z_1-z_2)^m\eval_{z_1=z_2}\expval{\mathcal O},\ \ \ m\in\mathbb N\\
    -\im\int\limits_{\partial R}\dd{z_1}\qty(z_1-z_2)^m\expval{\psi\qty(z_1)\psi\qty(z_2)\mathcal O}&=2\pi\delta_{m,0}\expval{\mathcal O},\ \ \ m\in\mathbb N
\end{align*}

This is simply the complex version of the divergence theorem,

\begin{align*}
    -\im\int\limits_{\partial R}\dd{z_1}\qty(z_1-z_2)^m\expval{\psi\qty(z_1)\psi\qty(z_2)\mathcal O}&=2\pi\delta_{m,0}\expval{\mathcal O},\ \ \ m\in\mathbb N\\
    \int\limits_{\partial R}\frac{\dd{z_1}}{2\pi\im}\qty(z_1-z_2)^m\expval{\psi\qty(z_1)\psi\qty(z_2)\mathcal O}&=\delta_{m,0}\expval{\mathcal O},\ \ \ m\in\mathbb N
\end{align*}

Well, the left-hand side of this last equation picks up the pole of $m$-th order of the expression $\expval{\psi\qty(z_1)\psi\qty(z_2)\mathcal O}$ under $z_1\rightarrow z_2$, but, 
the right-hand side is only non-zero for $m=0$, this is telling us that $\expval{\psi\qty(z_1)\psi\qty(z_2)\mathcal O}$ doesn't have any pole besides the first order one with residue $\expval{\mathcal O}$, 
so we can write,

\begin{align*}
    \expval{\psi\qty(z_1)\psi\qty(z_2)\mathcal O}=\frac{1}{z_1-z_2}\expval{\mathcal O}+\textnormal{regular}
\end{align*}

Or, in the operator formalism,

\begin{align*}
    \psi\qty(z_1)\psi\qty(z_2)&=\frac{1}{z_1-z_2}+\textnormal{regular},\ \ \ \abs{z_1}\geq \abs{z_2}\\
    \psi\qty(z_1)\psi\qty(z_2)&=\frac{1}{z_1-z_2}+\cnord{\psi\qty(z_1)\psi\qty(z_2)},\ \ \ \abs{z_1}\geq \abs{z_2}
\end{align*}

Which is the OPE.

\probitem{}

The energy momentum tensor has two components, a meromorphic one $T\qty(z)$, and an anti-meromorphic one $\bar T\qty(\bar z)$, 
as we have at our disposal only a weight $\qty(\frac12,0)$ field, it's impossible to construct a anti-meromorphic energy momentum tensor, 
but, for the meromorphic one, let us see what kind of combinations have the right weights. First, we remember that our energy momentum tensor 
must be normal ordered, so, all the possible normal ordered weight $\qty(2,0)$ combinations of $\psi$ and derivatives are,

\begin{align*}
    \cnord{\psi\partial\psi}\qty(z),\ \cnord{(\partial\psi)\psi}\qty(z), \ \cnord{\partial\qty(\psi\psi)}\qty(z),\ \cnord{\psi\psi\psi\psi}\qty(z)
\end{align*} 

The third and fourth are zero due to the fermionic statistic, and the first and second are linear dependent also because the 
fermionic statistic. Hence, up to a unknown constant $\alpha$, the meromorphic energy momentum tensor is,

\begin{align*}
    T\qty(z)=\alpha\cnord{\psi\partial\psi}\qty(z)
\end{align*}

Let's use this expression to compute the following,

\begin{align*}
    \cnord{T\qty(z_1) T\qty(z_2)}&=T\qty(z_1)T\qty(z_2)+\alpha^2\textnormal{ contractions}\qty{\cnord{\qty(\psi\partial\psi)\qty(z_1)\qty(\psi\partial\psi)\qty(z_2)}},\ \ \ \abs{z_1}\geq \abs{z_2}
\end{align*}

Where the contraction part means exchanging pairs of $\psi$ with different $z$ by $-\qty(z_1-z_2)^{-1}$, as cause of the fermion 
statistic, we have to pay attention to the signs, also, we'll omit for now the $\abs{z_1}\geq\abs{z_2}$, but it will still be imposed, 

\begin{align*}
    \cnord{T\qty(z_1) T\qty(z_2)}&={ T\qty(z_1)T\qty(z_2)}+\alpha^2\cnord{\wick[offset=1.2em]{(\c{\psi}\partial\psi)\qty(z_1)(\c{\psi}\partial\psi)\qty(z_2)}}+\alpha^2\cnord{\wick[offset=1.2em]{(\c{\psi}\partial\psi)\qty(z_1)({\psi}\partial\c{\psi})\qty(z_2)}}\\
    &\quad\quad\quad+\alpha^2\cnord{\wick[offset=1.2em]{({\psi}\partial\c{\psi})\qty(z_1)(\c{\psi}\partial{\psi})\qty(z_2)}}+\alpha^2\cnord{\wick[offset=1.2em]{({\psi}\partial\c{\psi})\qty(z_1)({\psi}\partial\c{\psi})\qty(z_2)}}\\
    &\quad\quad\quad+\alpha^2\cnord{\wick[offset=1.2em]{(\c1{\psi}\partial\c2{\psi})\qty(z_1)(\c1{\psi}\partial\c2{\psi})\qty(z_2)}}+\alpha^2\cnord{\wick[offset=1.2em]{(\c1{\psi}\partial\c2{\psi})\qty(z_1)(\c2{\psi}\partial\c1{\psi})\qty(z_2)}}\\
    \cnord{T\qty(z_1) T\qty(z_2)}&={ T\qty(z_1)T\qty(z_2)}+\frac{1}{z_1-z_2}\alpha^2\cnord{\partial\psi\qty(z_1)\partial\psi\qty(z_2)}-\partial_{z_2}\qty(\frac{1}{z_1-z_2})\alpha^2\cnord{\partial\psi\qty(z_1)\psi\qty(z_2)}\\
    &\quad\quad\quad-\alpha^2\partial_{z_1}\qty(\frac{1}{z_1-z_2})\cnord{{\psi}\qty(z_1)\partial{\psi}\qty(z_2)}+\alpha^2\partial_{z_1}\partial_{z_2}\qty(\frac{1}{z_1-z_2})\cnord{{\psi}\qty(z_1){\psi}\qty(z_2)}\\
    &\quad\quad\quad+\alpha^2\frac{1}{z_1-z_2}\partial_{z_1}\partial_{z_2}\qty(\frac{1}{z_1-z_2})-\alpha^2\partial_{z_2}\qty(\frac{1}{z_1-z_2})\partial_{z_1}\qty(\frac{1}{z_1-z_2})\\
    \cnord{T\qty(z_1) T\qty(z_2)}&={ T\qty(z_1)T\qty(z_2)}+\frac{\alpha^2}{z_1-z_2}\cnord{\partial\psi\qty(z_1)\partial\psi\qty(z_2)}-\frac{\alpha^2}{\qty(z_1-z_2)^2}\cnord{\partial\psi\qty(z_1)\psi\qty(z_2)}\\
    &\quad\quad\quad+\frac{\alpha^2}{\qty(z_1-z_2)^2}\cnord{{\psi}\qty(z_1)\partial{\psi}\qty(z_2)}+\partial_{z_1}\qty(\frac{\alpha^2}{\qty(z_1-z_2)^2})\cnord{{\psi}\qty(z_1){\psi}\qty(z_2)}\\
    &\quad\quad\quad+\frac{\alpha^2}{z_1-z_2}\partial_{z_1}\qty(\frac{1}{\qty(z_1-z_2)^2})-\frac{\alpha^2}{\qty(z_1-z_2)^2}\partial_{z_1}\qty(\frac{1}{z_1-z_2})\\
    \cnord{T\qty(z_1) T\qty(z_2)}&={ T\qty(z_1)T\qty(z_2)}+\frac{\alpha^2}{z_1-z_2}\cnord{\partial\psi\qty(z_1)\partial\psi\qty(z_2)}-\frac{\alpha^2}{\qty(z_1-z_2)^2}\cnord{\partial\psi\qty(z_1)\psi\qty(z_2)}\\
    &\quad\quad\quad+\frac{\alpha^2}{\qty(z_1-z_2)^2}\cnord{{\psi}\qty(z_1)\partial{\psi}\qty(z_2)}-2\frac{\alpha^2}{\qty(z_1-z_2)^3}\cnord{{\psi}\qty(z_1){\psi}\qty(z_2)}\\
    &\quad\quad\quad-2\frac{\alpha^2}{z_1-z_2}\frac{1}{\qty(z_1-z_2)^3}+\frac{\alpha^2}{\qty(z_1-z_2)^2}\frac{1}{\qty(z_1-z_2)^2}\\
    \cnord{T\qty(z_1) T\qty(z_2)}&={ T\qty(z_1)T\qty(z_2)}+\frac{\alpha^2}{z_1-z_2}\cnord{\partial\psi\qty(z_1)\partial\psi\qty(z_2)}-\frac{\alpha^2}{\qty(z_1-z_2)^2}\cnord{\partial\psi\qty(z_1)\psi\qty(z_2)}\\
    &\quad\quad\quad+\frac{\alpha^2}{\qty(z_1-z_2)^2}\cnord{{\psi}\qty(z_1)\partial{\psi}\qty(z_2)}-2\frac{\alpha^2}{\qty(z_1-z_2)^3}\cnord{{\psi}\qty(z_1){\psi}\qty(z_2)}\\
    &\quad\quad\quad-\frac{\alpha^2}{\qty(z_1-z_2)^4}
\end{align*}

To proceed further we have to Taylor expand both $\psi\qty(z_1),\partial\psi\qty(z_1)$, of course this will generate regular terms in our expansion, as for example, the second term in the right-hand side,

\begin{align*}
    \frac{1}{z_1-z_2}\cnord{\partial\psi\qty(z_1)\partial\psi\qty(z_2)}&=\frac{1}{z_1-z_2}\cnord{\partial\psi\qty(z_2)\partial\psi\qty(z_2)}+\sum\limits_{n=1}^\infty\frac{1}{z_1-z_2}\frac{1}{n!}\qty(z_1-z_2)^n\cnord{\partial^n\psi\qty(z_2)\partial\psi\qty(z_2)}\\
    \frac{1}{z_1-z_2}\cnord{\partial\psi\qty(z_1)\partial\psi\qty(z_2)}&=\frac{1}{z_1-z_2}\cnord{\partial\psi\partial\psi}\qty(z_2)+\sum\limits_{n=1}^\infty\frac{\qty(z_1-z_2)^{n-1}}{n!}\cnord{\partial^n\psi\partial\psi}\qty(z_2)
\end{align*}

It's clear that the sum in the right-hand side is of only regular terms, so that,

\begin{align*}
    \frac{1}{z_1-z_2}\cnord{\partial\psi\qty(z_1)\partial\psi\qty(z_2)}&=\frac{1}{z_1-z_2}\cnord{\partial\psi\partial\psi}\qty(z_2)+\textnormal{regular}
\end{align*}

But $\cnord{\partial\psi\partial\psi}$ is zero by statistics, hence, 

\begin{align*}
    \frac{1}{z_1-z_2}\cnord{\partial\psi\qty(z_1)\partial\psi\qty(z_2)}&=\textnormal{regular}
\end{align*}

We do this procedure for all the terms in the expansion,

\begin{align*}
    \cnord{T\qty(z_1) T\qty(z_2)}&={ T\qty(z_1)T\qty(z_2)}-\frac{\alpha^2}{\qty(z_1-z_2)^2}\cnord{\partial\psi\psi}\qty(z_2)-\frac{\alpha^2}{\qty(z_1-z_2)}\cnord{\partial^2\psi\psi}\qty(z_2)\\
    &\quad\quad\quad+\frac{\alpha^2}{\qty(z_1-z_2)^2}\cnord{{\psi}\partial{\psi}}\qty(z_2)+\frac{\alpha^2}{\qty(z_1-z_2)}\cnord{\partial{\psi}\partial{\psi}}\qty(z_2)\\
    &\quad\quad\quad-2\frac{\alpha^2}{\qty(z_1-z_2)^3}\cnord{{\psi}{\psi}}\qty(z_2)-2\frac{\alpha^2}{\qty(z_1-z_2)^2}\cnord{\partial{\psi}{\psi}}\qty(z_2)-\frac{\alpha^2}{\qty(z_1-z_2)}\cnord{\partial^2{\psi}{\psi}}\qty(z_2)\\
    &\quad\quad\quad-\frac{\alpha^2}{\qty(z_1-z_2)^4}+\textnormal{regular}
\end{align*}

Removing the terms which are zero by statistics and grouping the others,

\begin{align*}
    \cnord{T\qty(z_1) T\qty(z_2)}&={ T\qty(z_1)T\qty(z_2)}+\frac{4\alpha^2}{\qty(z_1-z_2)^2}\cnord{\psi\partial\psi}\qty(z_2)+\frac{2\alpha^2}{\qty(z_1-z_2)}\cnord{\psi\partial^2\psi}\qty(z_2)\\
    &\quad\quad\quad-\frac{\alpha^2}{\qty(z_1-z_2)^4}+\textnormal{regular}
\end{align*}

As $\cnord{\partial\psi\partial\psi}\equiv0$ we can add it as we please,

\begin{align*}
    \cnord{T\qty(z_1) T\qty(z_2)}&={ T\qty(z_1)T\qty(z_2)}+\frac{4\alpha^2}{\qty(z_1-z_2)^2}\cnord{\psi\partial\psi}\qty(z_2)+\frac{2\alpha^2}{\qty(z_1-z_2)}\cnord{\psi\partial^2\psi+\partial\psi\partial\psi}\qty(z_2)\\
    &\quad\quad\quad-\frac{\alpha^2}{\qty(z_1-z_2)^4}+\textnormal{regular}\\
    \cnord{T\qty(z_1) T\qty(z_2)}&={ T\qty(z_1)T\qty(z_2)}+\frac{4\alpha^2}{\qty(z_1-z_2)^2}\cnord{\psi\partial\psi}\qty(z_2)+\frac{2\alpha^2}{\qty(z_1-z_2)}\partial\cnord{\psi\partial\psi}\qty(z_2)\\
    &\quad\quad\quad-\frac{\alpha^2}{\qty(z_1-z_2)^4}+\textnormal{regular}\\
    \cnord{T\qty(z_1) T\qty(z_2)}&={ T\qty(z_1)T\qty(z_2)}+\frac{4\alpha}{\qty(z_1-z_2)^2}T\qty(z_2)+\frac{2\alpha}{\qty(z_1-z_2)}\partial T\qty(z_2)\\
    &\quad\quad\quad-\frac{\alpha^2}{\qty(z_1-z_2)^4}+\textnormal{regular}
\end{align*}

Well, $\cnord{T\qty(z_1)T\qty(z_2)}$ itself is regular, then,

\begin{align*}
    { T\qty(z_1)T\qty(z_2)}&=-\frac{4\alpha}{\qty(z_1-z_2)^2}T\qty(z_2)-\frac{2\alpha}{\qty(z_1-z_2)}\partial T\qty(z_2)+\frac{\alpha^2}{\qty(z_1-z_2)^4}+\textnormal{regular}
\end{align*}

But we know the general form of the $TT$ OPE,

\begin{align*}
    { T\qty(z_1)T\qty(z_2)}&=\frac{2}{\qty(z_1-z_2)^2}T\qty(z_2)+\frac{1}{\qty(z_1-z_2)}\partial T\qty(z_2)+\frac{c}{2\qty(z_1-z_2)^4}+\textnormal{regular}
\end{align*}

From where is easy to read $\alpha=-\frac12$, and also, $c=2\alpha^2=2\frac14=\frac12$, that is, the meromorphic component of the 
energy momentum tensor, and the central charge are,

\begin{align*}
    T\qty(z)&=-\frac12\cnord{\psi\partial\psi}\qty(z),\ \ \ c=\frac12
\end{align*}

\probitem{}
\label{3c}
Now let $\psi$ have an additional index, $\psi^i$ which takes two values $i=1,2$. This is a realization of a internal $SO\qty(2)$ symmetry, to why, 
let $\psi'^i=\tensor{\Lambda}{^i_j}\psi^j$, the change in the action is,

\begin{align*}
    S&=\frac{1}{4\pi}\int\dd[2]{z}\delta_{ij}\psi^i\bar\partial\psi^j\\
    S'&=\frac{1}{4\pi}\int\dd[2]{z}\delta_{ij}\psi'^i\bar\partial\psi'^j\\
    S'&=\frac{1}{4\pi}\int\dd[2]{z}\delta_{ij}\tensor{\Lambda}{^i_k}\tensor{\Lambda}{^j_l}\psi^k\bar\partial\psi^l
\end{align*}

As long as $\delta_{ij}\tensor{\Lambda}{^i_k}\tensor{\Lambda}{^j_l}=\delta_{kl}$, this transformation is a symmetry. But that's exactly the 
definition of $\Lambda$ belonging to the $O\qty(2)$ group, but as we're just interested in elements continuously connected to the identity, 
we set $SO\qty(2)$. The group definition under an infinitesimal transformation,

\begin{align*}
    \delta_{kl}&=\delta_{ij}\tensor{\Lambda}{^i_k}\tensor{\Lambda}{^j_l}\\
    \delta_{kl}&=\delta_{ij}\qty(\tensor{\delta}{^i_k}+\tensor{\omega}{^i_k}+\mathcal O\qty(\omega^2))\qty(\tensor{\delta}{^j_l}+\tensor{\omega}{^j_l}+\mathcal O\qty(\omega^2))\\
    \delta_{kl}&=\delta_{kl}+\omega_{lk}+\omega_{kl}+\mathcal O\qty(\omega^2)\\
    \omega_{lk}&=-\omega_{kl}
\end{align*}

We can use this information to write,

\begin{align*}
    \tensor{\Lambda}{^i_j}&=\tensor{\delta}{^i_j}+\tensor{\omega}{^i_j}+\mathcal O\qty(\omega^2)\\
    \tensor{\Lambda}{^i_j}&=\tensor{\delta}{^i_j}+\frac12\tensor{\omega}{_l_k}\qty(\tensor{\delta}{^i^l}\tensor{\delta}{^k_j}-\tensor{\delta}{^i^k}\tensor{\delta}{^l_j})+\mathcal O\qty(\omega^2)\\
    \tensor{\Lambda}{^i_j}&=\tensor{\delta}{^i_j}-\frac\im2\tensor{\omega}{_l_k}\im\qty(\tensor{\delta}{^i^l}\tensor{\delta}{^k_j}-\tensor{\delta}{^i^k}\tensor{\delta}{^l_j})+\mathcal O\qty(\omega^2)\\
    \tensor{\Lambda}{^i_j}&=\tensor{\delta}{^i_j}-\frac\im2\tensor{\omega}{_{\qty[lk]}}\tensor{T}{^{\qty[lk]}^i_j}+\mathcal O\qty(\omega^2)
\end{align*}

Where we defined the generator of the group, $\tensor{T}{^{\qty[lk]}^i_j}=\im\qty(\tensor{\delta}{^i^l}\tensor{\delta}{^k_j}-\tensor{\delta}{^i^k}\tensor{\delta}{^l_j})$. The sign of the 
definition is arbitrary. A finite transformation is then,

\begin{align*}
    \psi ^i\rightarrow\tensor{\Lambda}{^i_j}\psi^j=\tensor{\exp\qty(-\frac\im2\tensor{\omega}{_{\qty[lk]}}\tensor{T}{^{\qty[lk]}})}{^i_j}\psi^j
\end{align*}

To obtain the conserved current of this symmetry we make a little trick, set $\omega$ as a function $\omega\qty(z,\bar z)$, so that the infinitesimal variation of the action is,

\begin{align*}
    S&=\frac{1}{4\pi}\int\dd[2]{z}\delta_{ij}\psi^i\bar\partial\psi^j\\
    S'&=\frac{1}{4\pi}\int\dd[2]{z}\delta_{ij}\psi'^i\bar\partial\psi'^j\\
    S'&=\frac{1}{4\pi}\int\dd[2]{z}\delta_{ij}\tensor{\Lambda}{^i_k}\psi^k\bar\partial\qty(\tensor{\Lambda}{^j_l}\psi^l)\\
    S'&=\frac{1}{4\pi}\int\dd[2]{z}\delta_{ij}\tensor{\Lambda}{^i_k}\psi^k\qty{\tensor{\Lambda}{^j_l}\bar\partial\psi^l+\psi^l\bar\partial\tensor{\Lambda}{^j_l}}\\
    S'&=\frac{1}{4\pi}\int\dd[2]{z}\delta_{ij}\tensor{\Lambda}{^i_k}\psi^k\qty{\tensor{\Lambda}{^j_l}\bar\partial\psi^l-\frac\im 2\tensor{T}{^{\qty[ab]}^m_l}\psi^l\tensor{\Lambda}{^j_m}\bar\partial\omega_{\qty[ab]}}\\
    S'&=\frac{1}{4\pi}\int\dd[2]{z}\tensor{\Lambda}{_j_k}\psi^k\tensor{\Lambda}{^j_l}\bar\partial\psi^l-\frac{1}{4\pi}\frac\im 2\int\dd[2]{z}\tensor{\Lambda}{_j_k}\psi^k\tensor{T}{^{\qty[ab]}^m_l}\psi^l\tensor{\Lambda}{^j_m}\bar\partial\omega_{\qty[ab]}\\
    S'&=\frac{1}{4\pi}\int\dd[2]{z}\delta_{kl}\psi^k\bar\partial\psi^l-\frac{1}{4\pi}\frac\im 2\int\dd[2]{z}\delta_{km}\psi^k\tensor{T}{^{\qty[ab]}^m_l}\psi^l\bar\partial\omega_{\qty[ab]}\\
    S'&=S-\frac{1}{4\pi}\frac\im 2\int\dd[2]{z}\psi^k\tensor{T}{^{\qty[ab]}_k_l}\psi^l\bar\partial\omega_{\qty[ab]}\\
    \delta S&=-\frac{1}{4\pi}\frac\im 2\int\dd[2]{z}\bar\partial\qty{\psi^k\tensor{T}{^{\qty[ab]}_k_l}\psi^l\omega_{\qty[ab]}}+\frac{1}{4\pi}\frac\im 2\int\dd[2]{z}\bar\partial\qty{\psi^k\tensor{T}{^{\qty[ab]}_k_l}\psi^l}\omega_{\qty[ab]}
\end{align*}

With suitable boundary conditions on $\omega_{\qty[ab]}$, or, with no boundaries, we conclude,

\begin{align*}
    \delta S&=\frac{1}{4\pi}\frac\im 2\int\dd[2]{z}\bar\partial\qty{\psi^k\tensor{T}{^{\qty[ab]}_k_l}\psi^l}\omega_{\qty[ab]}
\end{align*}

Now back to $\omega_{\qty[ab]}$ being a constant --- so that $\delta S=0$ ---, using also that there is just one two index anti-symmetric tensor in two dimensions, the Levi-Civita $\epsilon_{ab},\ \epsilon_{12}=1$, 
we conclude that $\omega_{\qty[ab]}=\omega\epsilon_{ab}$,

\begin{align*}
    0&=\frac{1}{4\pi}\frac\im 2\omega\epsilon_{ab}\tensor{T}{^{\qty[ab]}_k_l}\int\dd[2]{z}\bar\partial\qty{\psi^k\psi^l}\\
    0&=-\frac{\omega\epsilon_{12}}{2\pi\im}\frac12\tensor{T}{^{12}_k_l}\int\dd[2]{z}\bar\partial\qty{\psi^k\psi^l}
\end{align*}

So there is just one current, as there is just one linear independent generator. The classical current can be read from the last equation as,

\begin{align*}
    j\qty(z)&=\frac12\tensor{T}{^{12}_k_l}{\psi^k\psi^l}\qty(z)
\end{align*}

We factored out the $-\frac{\omega\epsilon_{12}}{2\pi\im}$ as is usual to obtain an interpretation of the residue of the $j$ OPE. 
The quantum version of it will require normal ordering,

\begin{align*}
    j\qty(z)&=\frac12\tensor{T}{^{12}_k_l}\cnord{\psi^k\psi^l}\qty(z)\\
    j\qty(z)&=\frac12\im\qty(\tensor{\delta}{_k^1}\tensor{\delta}{^2_l}-\tensor{\delta}{_k^2}\tensor{\delta}{^1_l})\cnord{\psi^k\psi^l}\qty(z)\\
    j\qty(z)&=\frac\im2\qty(\cnord{\psi^1\psi^2}\qty(z)-\cnord{\psi^2\psi^1}\qty(z))\\
    j\qty(z)&=\im\cnord{\psi^1\psi^2}\qty(z)
\end{align*}

Now, to get it's OPE we proceed as usual, computing the following normal ordered operator,

\begin{align*}
    \cnord{j\qty(z_1)j\qty(z_2)}&=j\qty(z_1)j\qty(z_2)-\wick{\cnord{(\c1\psi^1\psi^2)\qty(z_1)(\c1\psi^1\psi^2)\qty(z_2)}}-\wick{\cnord{(\psi^1\c1\psi^2)\qty(z_1)(\psi^1\c1\psi^2)\qty(z_2)}}\\
    &\quad\quad\quad-\wick{\cnord{(\c1\psi^1\c2\psi^2)\qty(z_1)(\c1\psi^1\c2\psi^2)\qty(z_2)}}\\
    \cnord{j\qty(z_1)j\qty(z_2)}&=j\qty(z_1)j\qty(z_2)-\frac{1}{z_1-z_2}\cnord{\psi^2\qty(z_1)\psi^2\qty(z_2)}-\frac{1}{z_1-z_2}\cnord{\psi^1\qty(z_1)\psi^1\qty(z_2)}\\
    &\quad\quad\quad-\frac{1}{\qty(z_1-z_2)^2}\\
    \cnord{j\qty(z_1)j\qty(z_2)}&=j\qty(z_1)j\qty(z_2)-\frac{1}{z_1-z_2}\cnord{\psi^2\qty(z_2)\psi^2\qty(z_2)}-\frac{1}{z_1-z_2}\cnord{\psi^1\qty(z_1)\psi^1\qty(z_1)}\\
    &\quad\quad\quad-\frac{1}{\qty(z_1-z_2)^2}+\textnormal{regular}
\end{align*}

In the first to second like we used the $\psi\psi$ OPE, remembering that the $\psi^1\psi^2$ OPE has no poles, from the second line to third, we expanded 
every term around $z_2$. And now we set the terms which are zero by statistics,

\begin{align*}
    \cnord{j\qty(z_1)j\qty(z_2)}&=j\qty(z_1)j\qty(z_2)-\frac{1}{\qty(z_1-z_2)^2}+\textnormal{regular}\\
    j\qty(z_1)j\qty(z_2)&=\frac{1}{\qty(z_1-z_2)^2}+\textnormal{regular}
\end{align*}

A rather simple OPE.

\probitem{}
\label{3d}
Much of what we did can be recycled here. First, we start with $N$ real fermions $\psi^i,\ i=1,\cdots, N$, and by similar arguments analyzing the action,

\begin{align*}
    S&=\frac{1}{4\pi}\int\dd[2]{z}\delta_{ij}\psi^i\bar\partial\psi^j\\
    S'&=\frac{1}{4\pi}\int\dd[2]{z}\delta_{ij}\psi'^i\bar\partial\psi'^j\\
    S'&=\frac{1}{4\pi}\int\dd[2]{z}\delta_{ij}\tensor{\Lambda}{^i_k}\tensor{\Lambda}{^j_l}\psi^k\bar\partial\psi^l
\end{align*}

We obtain the constraint of this being a symmetry, $\delta_{ij}\tensor{\Lambda}{^i_k}\tensor{\Lambda}{^j_l}=\delta_{kl}$, which is the defining 
property of the $O\qty(N)$ group. Of course, the relevant part for generators and Lie Algebras is just the part connected to the identity, $SO\qty(N)$, 
the same reasoning of infinitesimal transformations allows us to write,

\begin{align*}
    \delta_{kl}&=\delta_{ij}\tensor{\Lambda}{^i_k}\tensor{\Lambda}{^j_l}\\
    \delta_{kl}&=\delta_{ij}\qty(\tensor{\delta}{^i_k}+\tensor{\omega}{^i_k}+\mathcal O\qty(\omega^2))\qty(\tensor{\delta}{^j_l}+\tensor{\omega}{^j_l}+\mathcal O\qty(\omega^2))\\
    \delta_{kl}&=\delta_{kl}+\omega_{lk}+\omega_{kl}+\mathcal O\qty(\omega^2)\\
    \omega_{lk}&=-\omega_{kl}
\end{align*}

We can use this information to write,

\begin{align*}
    \tensor{\Lambda}{^i_j}&=\tensor{\delta}{^i_j}+\tensor{\omega}{^i_j}+\mathcal O\qty(\omega^2)\\
    \tensor{\Lambda}{^i_j}&=\tensor{\delta}{^i_j}+\frac12\tensor{\omega}{_l_k}\qty(\tensor{\delta}{^i^l}\tensor{\delta}{^k_j}-\tensor{\delta}{^i^k}\tensor{\delta}{^l_j})+\mathcal O\qty(\omega^2)\\
    \tensor{\Lambda}{^i_j}&=\tensor{\delta}{^i_j}-\frac\im2\tensor{\omega}{_l_k}\im\qty(\tensor{\delta}{^i^l}\tensor{\delta}{^k_j}-\tensor{\delta}{^i^k}\tensor{\delta}{^l_j})+\mathcal O\qty(\omega^2)\\
    \tensor{\Lambda}{^i_j}&=\tensor{\delta}{^i_j}-\frac\im2\tensor{\omega}{_{\qty[lk]}}\tensor{T}{^{\qty[lk]}^i_j}+\mathcal O\qty(\omega^2)\\
    \tensor{\Lambda}{^i_j}&=\tensor{\exp\qty(-\frac\im 2\omega_{\qty[lk]}\tensor{T}{^{\qty[lk]}})}{^i_j}
\end{align*}

To obtain the current we do again the same trick of promoting $\omega_{\qty[lk]}$ to a function and computing the change in the action,

\begin{align*}
    S&=\frac{1}{4\pi}\int\dd[2]{z}\delta_{ij}\psi^i\bar\partial\psi^j\\
    S'&=\frac{1}{4\pi}\int\dd[2]{z}\delta_{ij}\psi'^i\bar\partial\psi'^j\\
    S'&=\frac{1}{4\pi}\int\dd[2]{z}\delta_{ij}\tensor{\Lambda}{^i_k}\psi^k\bar\partial\qty(\tensor{\Lambda}{^j_l}\psi^l)\\
    S'&=\frac{1}{4\pi}\int\dd[2]{z}\delta_{ij}\tensor{\Lambda}{^i_k}\psi^k\qty{\tensor{\Lambda}{^j_l}\bar\partial\psi^l+\psi^l\bar\partial\tensor{\Lambda}{^j_l}}\\
    S'&=\frac{1}{4\pi}\int\dd[2]{z}\delta_{ij}\tensor{\Lambda}{^i_k}\psi^k\qty{\tensor{\Lambda}{^j_l}\bar\partial\psi^l-\frac\im 2\tensor{T}{^{\qty[ab]}^m_l}\psi^l\tensor{\Lambda}{^j_m}\bar\partial\omega_{\qty[ab]}}\\
    S'&=\frac{1}{4\pi}\int\dd[2]{z}\tensor{\Lambda}{_j_k}\psi^k\tensor{\Lambda}{^j_l}\bar\partial\psi^l-\frac{1}{4\pi}\frac\im 2\int\dd[2]{z}\tensor{\Lambda}{_j_k}\psi^k\tensor{T}{^{\qty[ab]}^m_l}\psi^l\tensor{\Lambda}{^j_m}\bar\partial\omega_{\qty[ab]}\\
    S'&=\frac{1}{4\pi}\int\dd[2]{z}\delta_{kl}\psi^k\bar\partial\psi^l-\frac{1}{4\pi}\frac\im 2\int\dd[2]{z}\delta_{km}\psi^k\tensor{T}{^{\qty[ab]}^m_l}\psi^l\bar\partial\omega_{\qty[ab]}\\
    S'&=S-\frac{1}{4\pi}\frac\im 2\int\dd[2]{z}\psi^k\tensor{T}{^{\qty[ab]}_k_l}\psi^l\bar\partial\omega_{\qty[ab]}\\
    \delta S&=-\frac{1}{4\pi}\frac\im 2\int\dd[2]{z}\bar\partial\qty{\psi^k\tensor{T}{^{\qty[ab]}_k_l}\psi^l\omega_{\qty[ab]}}+\frac{1}{4\pi}\frac\im 2\int\dd[2]{z}\bar\partial\qty{\psi^k\tensor{T}{^{\qty[ab]}_k_l}\psi^l}\omega_{\qty[ab]}
\end{align*}

With suitable boundary conditions on $\omega_{\qty[ab]}$, or, with no boundaries, we conclude,

\begin{align*}
    \delta S&=\frac{1}{4\pi}\frac\im 2\int\dd[2]{z}\bar\partial\qty{\psi^k\tensor{T}{^{\qty[ab]}_k_l}\psi^l}\omega_{\qty[ab]}
\end{align*}

Demoting $\omega_{\qty[ab]}$ back to being a constant --- so that $\delta S=0$ ---, we obtain,

\begin{align*}
    0&=-\frac{\omega_{\qty[ab]}}{2\pi\im}\frac1 2\tensor{T}{^{\qty[ab]}_k_l}\int\dd[2]{z}\bar\partial\qty{\psi^k\psi^l}
\end{align*}

Again, the usual normalization is done by removing $-\frac{\omega_{\qty[ab]}}{2\pi\im}$, so that the current is,

\begin{align*}
    j^{\qty[ab]}\qty(z)&=\frac1 2\tensor{T}{^{\qty[ab]}_k_l}\cnord{\psi^k\psi^l}\qty(z)\numberthis\label{jabcurrent}
\end{align*}

This is not a very orthodox way of displaying, all of our index here have the range $1,\cdots, N$, but notice the generators are indexed by an 
anti-symmetric pair of index, $\qty[ab]$, so there are $\frac{N\qty(N-1)}{2}$ such, this is only useful if we work in the fundamental representation, 
so, we'll change now to a more common notation, instead of labeling the generators by $\frac{N\qty(N-1)}{2}$ anti-symmetric pairs of index, we'll 
label them by just one index with range $A=1,\cdots , \frac{N\qty(N-1)}{2}$, so that the same equation reads,

\begin{align*}
    j^A\qty(z)&=\frac1 2\tensor{T}{^A_k_l}\cnord{\psi^k\psi^l}\qty(z)
\end{align*}

This is more aesthetic for Lie Algebras in general, but not always desired --- Lorentz group as example ---. Now we proceed by computing the 
OPE associated with this collection of currents,

\begin{align*}
    \cnord{j^A\qty(z_1)j^B\qty(z_2)}&=j^A\qty(z_1)j ^B\qty(z_2)+\frac14\tensor{T}{^A_{ij}}\tensor{T}{^B_{kl}}\wick{\cnord{(\c1\psi^i\psi^j)\qty(z_1)(\c1\psi^k\psi^l)\qty(z_2)}}\\
    &\quad\quad\quad+\frac14\tensor{T}{^A_{ij}}\tensor{T}{^B_{kl}}\wick{\cnord{(\c1\psi^i\psi^j)\qty(z_1)(\psi^k\c1\psi^l)\qty(z_2)}}+\frac14\tensor{T}{^A_{ij}}\tensor{T}{^B_{kl}}\wick{\cnord{(\psi^i\c1\psi^j)\qty(z_1)(\c1\psi^k\psi^l)\qty(z_2)}}\\
    &\quad\quad\quad+\frac14\tensor{T}{^A_{ij}}\tensor{T}{^B_{kl}}\wick{\cnord{(\psi^i\c1\psi^j)\qty(z_1)(\psi^k\c1\psi^l)\qty(z_2)}}+\frac14\tensor{T}{^A_{ij}}\tensor{T}{^B_{kl}}\wick{\cnord{(\c1\psi^i\c2\psi^j)\qty(z_1)(\c1\psi^k\c2\psi^l)\qty(z_2)}}\\
    &\quad\quad\quad+\frac14\tensor{T}{^A_{ij}}\tensor{T}{^B_{kl}}\wick{\cnord{(\c1\psi^i\c2\psi^j)\qty(z_1)(\c2\psi^k\c1\psi^l)\qty(z_2)}}
\end{align*}

As we stated before, the contraction only is non zero for equal index $\psi$,

\begin{align*}
    \cnord{j^A\qty(z_1)j^B\qty(z_2)}&=j^A\qty(z_1)j ^B\qty(z_2)+\frac{\delta^{ik}}{4\qty(z_1-z_2)}\tensor{T}{^A_{ij}}\tensor{T}{^B_{kl}}\cnord{\psi^j\qty(z_1)\psi^l\qty(z_2)}\\
    &\quad\quad\quad-\frac{\delta^{il}}{4\qty(z_1-z_2)}\tensor{T}{^A_{ij}}\tensor{T}{^B_{kl}}\cnord{\psi^j\qty(z_1)\psi^k\qty(z_2)}-\frac{\delta^{jk}}{4\qty(z_1-z_2)}\tensor{T}{^A_{ij}}\tensor{T}{^B_{kl}}\cnord{\psi^i\qty(z_1)\psi^l\qty(z_2)}\\
    &\quad\quad\quad+\frac{\delta^{jl}}{4\qty(z_1-z_2)}\tensor{T}{^A_{ij}}\tensor{T}{^B_{kl}}\cnord{\psi^i\qty(z_1)\psi^k\qty(z_2)}+\frac{\delta^{ik}\delta^{jl}}{4\qty(z_1-z_2)^2}\tensor{T}{^A_{ij}}\tensor{T}{^B_{kl}}\\
    &\quad\quad\quad-\frac{\delta^{il}\delta^{jk}}{4\qty(z_1-z_2)}\tensor{T}{^A_{ij}}\tensor{T}{^B_{kl}}
\end{align*}

Expanding in Taylor everything around $z_2$,

\begin{align*}
    \cnord{j^A\qty(z_1)j^B\qty(z_2)}&=j^A\qty(z_1)j ^B\qty(z_2)+\frac{\delta^{ik}}{4\qty(z_1-z_2)}\tensor{T}{^A_{ij}}\tensor{T}{^B_{kl}}\cnord{\psi^j\qty(z_2)\psi^l\qty(z_2)}\\
    &\quad\quad\quad-\frac{\delta^{il}}{4\qty(z_1-z_2)}\tensor{T}{^A_{ij}}\tensor{T}{^B_{kl}}\cnord{\psi^j\qty(z_2)\psi^k\qty(z_2)}-\frac{\delta^{jk}}{4\qty(z_1-z_2)}\tensor{T}{^A_{ij}}\tensor{T}{^B_{kl}}\cnord{\psi^i\qty(z_2)\psi^l\qty(z_2)}\\
    &\quad\quad\quad+\frac{\delta^{jl}}{4\qty(z_1-z_2)}\tensor{T}{^A_{ij}}\tensor{T}{^B_{kl}}\cnord{\psi^i\qty(z_2)\psi^k\qty(z_2)}+\frac{\delta^{ik}\delta^{jl}}{4\qty(z_1-z_2)^2}\tensor{T}{^A_{ij}}\tensor{T}{^B_{kl}}\\
    &\quad\quad\quad-\frac{\delta^{il}\delta^{jk}}{4\qty(z_1-z_2)}\tensor{T}{^A_{ij}}\tensor{T}{^B_{kl}}+\textnormal{regular}
\end{align*}

It's clear we have two kinds of terms, with $\cnord{\psi\psi}$ and without, to group together all of them, we'll go under a massive round of relabeling, 
first, relabel all $\psi$ index to be $b$ in the right one and $a$ in the left one,

\begin{align*}
    \cnord{j^A\qty(z_1)j^B\qty(z_2)}&=j^A\qty(z_1)j ^B\qty(z_2)+\frac{\delta^{ik}}{4\qty(z_1-z_2)}\tensor{T}{^A_{ia}}\tensor{T}{^B_{kb}}\cnord{\psi^a\psi^b}\qty(z_2)\\
    &\quad\quad\quad-\frac{\delta^{il}}{4\qty(z_1-z_2)}\tensor{T}{^A_{ia}}\tensor{T}{^B_{bl}}\cnord{\psi^a\psi^b}\qty(z_2)-\frac{\delta^{jk}}{4\qty(z_1-z_2)}\tensor{T}{^A_{aj}}\tensor{T}{^B_{kb}}\cnord{\psi^a\psi^b}\qty(z_2)\\
    &\quad\quad\quad+\frac{\delta^{jl}}{4\qty(z_1-z_2)}\tensor{T}{^A_{aj}}\tensor{T}{^B_{bl}}\cnord{\psi^a\psi^b}\qty(z_2)+\frac{\delta^{ik}\delta^{jl}}{4\qty(z_1-z_2)^2}\tensor{T}{^A_{ij}}\tensor{T}{^B_{kl}}\\
    &\quad\quad\quad-\frac{\delta^{il}\delta^{jk}}{4\qty(z_1-z_2)}\tensor{T}{^A_{ij}}\tensor{T}{^B_{kl}}+\textnormal{regular}
\end{align*}

Notice that $\tensor{T}{^A_{ij}}=\tensor{T}{^{\qty[ab]}_{ij}}=\im\qty(\tensor{\delta}{_i^a}\tensor{\delta}{^b_j}-\tensor{\delta}{_i^b}\tensor{\delta}{^a_j})$ is antisymmetric in $ij$, so we reverse the order of every pair of $TT$ to be with 
$T^A$ with the index $a$ at left and $T^B$ with index $b$ in the right,

\begin{align*}
    \cnord{j^A\qty(z_1)j^B\qty(z_2)}&=j^A\qty(z_1)j ^B\qty(z_2)-\frac{\delta^{ik}}{4\qty(z_1-z_2)}\tensor{T}{^A_{ai}}\tensor{T}{^B_{kb}}\cnord{\psi^a\psi^b}\qty(z_2)\\
    &\quad\quad\quad-\frac{\delta^{il}}{4\qty(z_1-z_2)}\tensor{T}{^A_{ai}}\tensor{T}{^B_{lb}}\cnord{\psi^a\psi^b}\qty(z_2)-\frac{\delta^{jk}}{4\qty(z_1-z_2)}\tensor{T}{^A_{aj}}\tensor{T}{^B_{kb}}\cnord{\psi^a\psi^b}\qty(z_2)\\
    &\quad\quad\quad-\frac{\delta^{jl}}{4\qty(z_1-z_2)}\tensor{T}{^A_{aj}}\tensor{T}{^B_{lb}}\cnord{\psi^a\psi^b}\qty(z_2)-\frac{\delta^{ik}\delta^{jl}}{4\qty(z_1-z_2)^2}\tensor{T}{^A_{ji}}\tensor{T}{^B_{kl}}\\
    &\quad\quad\quad-\frac{\delta^{il}\delta^{jk}}{4\qty(z_1-z_2)}\tensor{T}{^A_{ij}}\tensor{T}{^B_{kl}}+\textnormal{regular}
\end{align*}

Now it's trivial to sum the terms,

\begin{align*}
    \cnord{j^A\qty(z_1)j^B\qty(z_2)}&=j^A\qty(z_1)j ^B\qty(z_2)-\frac{\delta^{ik}\tensor{T}{^A_{ai}}\tensor{T}{^B_{kb}}}{\qty(z_1-z_2)}\cnord{\psi^a\psi^b}\qty(z_2)-\frac{\delta^{ik}\delta^{jl}}{2\qty(z_1-z_2)^2}\tensor{T}{^A_{ji}}\tensor{T}{^B_{kl}}\\
    &\quad\quad\quad+\textnormal{regular}
\end{align*}

Due to the statistic we have to anti-symmetrize the second term in the right-hand side in the index $ab$

\begin{align*}
    \cnord{j^A\qty(z_1)j^B\qty(z_2)}&=j^A\qty(z_1)j ^B\qty(z_2)-\frac{\tensor{T}{^A_{a}^i}\tensor{T}{^B_{ib}}-\tensor{T}{^A_{b}^i}\tensor{T}{^B_{ia}}}{2\qty(z_1-z_2)}\cnord{\psi^a\psi^b}\qty(z_2)-\frac{\Tr\qty[T^A T^B]}{2\qty(z_1-z_2)^2}\\
    &\quad\quad\quad+\textnormal{regular}\\
    \cnord{j^A\qty(z_1)j^B\qty(z_2)}&=j^A\qty(z_1)j ^B\qty(z_2)-\frac{\tensor{T}{^A_{a}^i}\tensor{T}{^B_{ib}}-\tensor{T}{^B_{ai}}\tensor{T}{^A^i_{b}}}{2\qty(z_1-z_2)}\cnord{\psi^a\psi^b}\qty(z_2)-\frac{\Tr\qty[T^A T^B]}{2\qty(z_1-z_2)^2}\\
    &\quad\quad\quad+\textnormal{regular}
\end{align*}

Well the numerator in the second term in the right-hand side is just the commutator of the $T$'s, which is of course written in terms of the 
structure constants,

\begin{align*}
    \cnord{j^A\qty(z_1)j^B\qty(z_2)}&=j^A\qty(z_1)j ^B\qty(z_2)-\frac{\im\tensor{f}{^A^B_C}\tensor{T}{^C_a_b}}{2\qty(z_1-z_2)}\cnord{\psi^a\psi^b}\qty(z_2)-\frac{\Tr\qty[T^A T^B]}{2\qty(z_1-z_2)^2}\\
    &\quad\quad\quad+\textnormal{regular}\\
    \cnord{j^A\qty(z_1)j^B\qty(z_2)}&=j^A\qty(z_1)j ^B\qty(z_2)-\frac{\im\tensor{f}{^A^B_C}}{\qty(z_1-z_2)}j^C\qty(z_2)-\frac{\Tr\qty[T^A T^B]}{2\qty(z_1-z_2)^2}\\
    &\quad\quad\quad+\textnormal{regular}
\end{align*}

So that, finally,

\begin{align*}
    j^A\qty(z_1)j ^B\qty(z_2)&=\frac{\im\tensor{f}{^A^B_C}}{\qty(z_1-z_2)}j^C\qty(z_2)+\frac{\Tr\qty[T^A T^B]}{2\qty(z_1-z_2)^2}+\textnormal{regular}\numberthis\label{jope}
\end{align*}

\probitem{}

As $\psi$ has conformal weight $\qty(\frac12,0)$ it's clear that $j^A$ will have conformal weight $\qty(1,0)$, so that a mode expansion is,

\begin{align*}
    j^A\qty(z)&=\sum\limits_{n\in\mathbb Z}\frac{j^A_n}{z^{n+1}}\\
    j^A_n&=\oint\limits_C\frac{\dd{z}}{2\pi\im}z^nj^A\qty(z)
\end{align*}

Where $C$ is any sufficiently well behaved closed curve around the origin. To obtain the algebra of the current modes, 
is easier to rewrite \ref{jope} back in the path integral formalism, from which the radial order is always manifest,

\begin{align*}
    \expval{j^A\qty(z_1)j ^B\qty(z_2)\mathcal O}&=\frac{\im\tensor{f}{^A^B_C}}{\qty(z_1-z_2)}\expval{j^C\qty(z_2)\mathcal O}+\frac{\Tr\qty[T^A T^B]}{2\qty(z_1-z_2)^2}\expval{\mathcal O}+\textnormal{regular}
\end{align*}

We now integrate this expression with two different contours, $C_1$ and $C_2$, $C_1$ is a contour around the origin which also 
encloses the point $z_2$, and $C_2$ is a contour around the origin that does not encloses $z_2$, hence,

\begin{align*}
    \oint\limits_{C_1}\frac{\dd{{z_1}}}{2\pi\im}z_1^n\expval{j^A\qty(z_1)j ^B\qty(z_2)\mathcal O}&=\oint\limits_{C_1}\frac{\dd{{z_1}}}{2\pi\im}z_1^n\frac{\im\tensor{f}{^A^B_C}}{\qty(z_1-z_2)}\expval{j^C\qty(z_2)\mathcal O}+\oint\limits_{C_1}\frac{\dd{{z_1}}}{2\pi\im}z_1^n\frac{\Tr\qty[T^A T^B]}{2\qty(z_1-z_2)^2}\expval{\mathcal O}+\textnormal{regular}\\
    \oint\limits_{C_2}\frac{\dd{{z_1}}}{2\pi\im}z_1^n\expval{j^A\qty(z_1)j ^B\qty(z_2)\mathcal O}&=\oint\limits_{C_2}\frac{\dd{{z_1}}}{2\pi\im}z_1^n\frac{\im\tensor{f}{^A^B_C}}{\qty(z_1-z_2)}\expval{j^C\qty(z_2)\mathcal O}+\oint\limits_{C_2}\frac{\dd{{z_1}}}{2\pi\im}z_1^n\frac{\Tr\qty[T^A T^B]}{2\qty(z_1-z_2)^2}\expval{\mathcal O}+\textnormal{regular}
\end{align*}

Notice that for the second contour, the order of the $j$ inside the expectation value will change once we go to the operator formalism, as for $C_2$, $\abs{z_2}>\abs{z_1}$, 
we'll use this to our advantage, first, as for the expectation value the order is irrelevant, due to the implicit radial ordering, we exchange the order of the two $j$, as we're 
in the region $\abs{z_2}>\abs{z_1}$,

\begin{align*}
    \oint\limits_{C_1}\frac{\dd{{z_1}}}{2\pi\im}z_1^n\expval{j^A\qty(z_1)j ^B\qty(z_2)\mathcal O}&=\oint\limits_{C_1}\frac{\dd{{z_1}}}{2\pi\im}z_1^n\frac{\im\tensor{f}{^A^B_C}}{\qty(z_1-z_2)}\expval{j^C\qty(z_2)\mathcal O}+\oint\limits_{C_1}\frac{\dd{{z_1}}}{2\pi\im}z_1^n\frac{\Tr\qty[T^A T^B]}{2\qty(z_1-z_2)^2}\expval{\mathcal O}+\textnormal{regular}\\
    \oint\limits_{C_2}\frac{\dd{{z_1}}}{2\pi\im}z_1^n\expval{j ^B\qty(z_2)j^A\qty(z_1)\mathcal O}&=\oint\limits_{C_2}\frac{\dd{{z_1}}}{2\pi\im}z_1^n\frac{\im\tensor{f}{^A^B_C}}{\qty(z_1-z_2)}\expval{j^C\qty(z_2)\mathcal O}+\oint\limits_{C_2}\frac{\dd{{z_1}}}{2\pi\im}z_1^n\frac{\Tr\qty[T^A T^B]}{2\qty(z_1-z_2)^2}\expval{\mathcal O}+\textnormal{regular}
\end{align*}

Now we subtract the second equation from the first,

\begin{align*}
    \oint\limits_{C_1}\frac{\dd{{z_1}}}{2\pi\im}z_1^n\expval{j^A\qty(z_1)j ^B\qty(z_2)\mathcal O}&-\oint\limits_{C_2}\frac{\dd{{z_1}}}{2\pi\im}z_1^n\expval{j ^B\qty(z_2)j^A\qty(z_1)\mathcal O}=\\
    &\quad\quad\quad+\oint\limits_{C_1}\frac{\dd{{z_1}}}{2\pi\im}z_1^n\frac{\im\tensor{f}{^A^B_C}}{\qty(z_1-z_2)}\expval{j^C\qty(z_2)\mathcal O}-\oint\limits_{C_2}\frac{\dd{{z_1}}}{2\pi\im}z_1^n\frac{\im\tensor{f}{^A^B_C}}{\qty(z_1-z_2)}\expval{j^C\qty(z_2)\mathcal O}\\
    &\quad\quad\quad+\oint\limits_{C_1}\frac{\dd{{z_1}}}{2\pi\im}z_1^n\frac{\Tr\qty[T^A T^B]}{2\qty(z_1-z_2)^2}\expval{\mathcal O}-\oint\limits_{C_2}\frac{\dd{{z_1}}}{2\pi\im}z_1^n\frac{\Tr\qty[T^A T^B]}{2\qty(z_1-z_2)^2}\expval{\mathcal O}\\
    &\quad\quad\quad+\oint\limits_{C_1}\frac{\dd{{z_1}}}{2\pi\im}z_1^n\textnormal{regular}-\oint\limits_{C_2}\frac{\dd{{z_1}}}{2\pi\im}z_1^n\textnormal{regular}\\
    \oint\limits_{C_1}\frac{\dd{{z_1}}}{2\pi\im}z_1^n\expval{j^A\qty(z_1)j ^B\qty(z_2)\mathcal O}&-\oint\limits_{C_2}\frac{\dd{{z_1}}}{2\pi\im}z_1^n\expval{j ^B\qty(z_2)j^A\qty(z_1)\mathcal O}=\\
    &\quad\quad\quad+\qty[\oint\limits_{C_1}-\oint\limits_{C_2}]\frac{\dd{{z_1}}}{2\pi\im}z_1^n\frac{\im\tensor{f}{^A^B_C}}{\qty(z_1-z_2)}\expval{j^C\qty(z_2)\mathcal O}\\
    &\quad\quad\quad+\qty[\oint\limits_{C_1}-\oint\limits_{C_2}]\frac{\dd{{z_1}}}{2\pi\im}z_1^n\frac{\Tr\qty[T^A T^B]}{2\qty(z_1-z_2)^2}\expval{\mathcal O}\\
    &\quad\quad\quad+\qty[\oint\limits_{C_1}-\oint\limits_{C_2}]\frac{\dd{{z_1}}}{2\pi\im}z_1^n\textnormal{regular}
\end{align*}

The subtraction of two integrals, one contouring the origin and $z_2$, $C_1$, by other just contouring the origin, $C_2$, is given by an integral 
over a contour of $z_2$ but not of the origin $C_3$,

\begin{align*}
    \oint\limits_{C_1}\frac{\dd{{z_1}}}{2\pi\im}z_1^n\expval{j^A\qty(z_1)j ^B\qty(z_2)\mathcal O}&-\oint\limits_{C_2}\frac{\dd{{z_1}}}{2\pi\im}z_1^n\expval{j ^B\qty(z_2)j^A\qty(z_1)\mathcal O}=\\
    &\quad\quad\quad+\oint\limits_{C_3}\frac{\dd{{z_1}}}{2\pi\im}z_1^n\frac{\im\tensor{f}{^A^B_C}}{\qty(z_1-z_2)}\expval{j^C\qty(z_2)\mathcal O}\\
    &\quad\quad\quad+\oint\limits_{C_3}\frac{\dd{{z_1}}}{2\pi\im}z_1^n\frac{\Tr\qty[T^A T^B]}{2\qty(z_1-z_2)^2}\expval{\mathcal O}\\
    &\quad\quad\quad+\oint\limits_{C_3}\frac{\dd{{z_1}}}{2\pi\im}z_1^n\textnormal{regular}
\end{align*}

As the regular part doesn't have poles at $z_1=z_2$, 

\begin{align*}
    \oint\limits_{C_1}\frac{\dd{{z_1}}}{2\pi\im}z_1^n\expval{j^A\qty(z_1)j ^B\qty(z_2)\mathcal O}&-\oint\limits_{C_2}\frac{\dd{{z_1}}}{2\pi\im}z_1^n\expval{j ^B\qty(z_2)j^A\qty(z_1)\mathcal O}=\\
    &\quad\quad\quad+\oint\limits_{C_3}\frac{\dd{{z_1}}}{2\pi\im}z_1^n\frac{\im\tensor{f}{^A^B_C}}{\qty(z_1-z_2)}\expval{j^C\qty(z_2)\mathcal O}\\
    &\quad\quad\quad+\oint\limits_{C_3}\frac{\dd{{z_1}}}{2\pi\im}z_1^n\frac{\Tr\qty[T^A T^B]}{2\qty(z_1-z_2)^2}\expval{\mathcal O}
\end{align*}

We now can go back to the operator formalism, and, as $z_1$ is integrated, there is no ordering ambiguities,

\begin{align*}
    \oint\limits_{C_1}\frac{\dd{{z_1}}}{2\pi\im}z_1^nj^A\qty(z_1)j ^B\qty(z_2)&-\oint\limits_{C_2}\frac{\dd{{z_1}}}{2\pi\im}z_1^nj ^B\qty(z_2)j^A\qty(z_1)=\\
    &\quad\quad\quad+\oint\limits_{C_3}\frac{\dd{{z_1}}}{2\pi\im}z_1^n\frac{\im\tensor{f}{^A^B_C}}{\qty(z_1-z_2)}j^C\qty(z_2)\\
    &\quad\quad\quad+\oint\limits_{C_3}\frac{\dd{{z_1}}}{2\pi\im}z_1^n\frac{\Tr\qty[T^A T^B]}{2\qty(z_1-z_2)^2}\\
    j_n^Aj ^B\qty(z_2)&-j ^B\qty(z_2)j^A_n=z_2^n\im\tensor{f}{^A^B_C}j^C\qty(z_2)+\qty(\partial_{z_1}z_1^n)\eval_{z_1=z_2}\frac{\Tr\qty[T^A T^B]}{2}\\
    j_n^Aj ^B\qty(z_2)&-j ^B\qty(z_2)j^A_n=z_2^n\im\tensor{f}{^A^B_C}j^C\qty(z_2)+nz_2^{n-1}\frac{\Tr\qty[T^A T^B]}{2}
\end{align*}

Now integrating over a contour $C$ around the origin to get the modes of $j ^B$,

\begin{align*}
    \oint\limits_C\frac{\dd{z_2}}{2\pi\im}z_2^m \qty(j_n^Aj ^B\qty(z_2)-j ^B\qty(z_2)j^A_n)&=\oint\limits_C\frac{\dd{z_2}}{2\pi\im}z_2^m z_2^n\im\tensor{f}{^A^B_C}j^C\qty(z_2)+\oint\limits_C\frac{\dd{z_2}}{2\pi\im}z_2^m nz_2^{n-1}\frac{\Tr\qty[T^A T^B]}{2}\\
    j_n^Aj ^B_m-j ^B_mj^A_n&=\im\tensor{f}{^A^B_C}j^C_{n+m}+\frac{n}{2}\delta_{m+n,0} \Tr\qty[T^A T^B]\\
    \comm{j_n^A}{j ^B_m}&=\im\tensor{f}{^A^B_C}j^C_{n+m}+\frac{n}{2}\delta_{m+n,0} \Tr\qty[T^A T^B]
\end{align*}

This is the algebra of the modes! We used that $z_2^{n+m-1}$ only has a residue at $z_2=0$ for $n+m=0$. The value $\Tr\qty[T^A T^B]$ 
is dependent of the normalization of the algebra and the representation, but usually it can be taken to be $\Tr\qty[T^A T^B]=k\delta^{AB}$.

\probitem{}

To have a $SU\qty(N)$ group, it's obvious that we need a transformation law $\psi'^i=\tensor{\Lambda}{^i_ j}\psi^j$, such that, 

\begin{align*}
    \delta_{ij}&=\delta_{kl}\tensor{{\Lambda^\ast}}{^i_k}\tensor{\Lambda}{^j_l}\Leftrightarrow \Lambda^\dagger\Lambda=\mathbbm 1
\end{align*}

This is only possible if we include in the action a complex fermion, and, one of them being a complex conjugate, that is, now we have a collection of 
$2N$ real fermions,

\begin{align*}
    \psi^i=\textnormal{Re}\qty[\psi^i]+\im \textnormal{Im}\qty[\psi^i]
\end{align*}

Which compose a $N$ dimensional vector, while seems natural to express a complex field as a pair $\psi,\bar\psi$, 
the notation $\bar\psi$ is kind of misleading, does it is a meromorphic or anti-meromorphic function/operator? So, we will 
choose do denote the complex fermion, and it's complex conjugate by,

\begin{align*}
    \psi^i&=\textnormal{Re}\qty[\psi^i]+\im \textnormal{Im}\qty[\psi^i]\\
    {\tilde\psi}^i&=\textnormal{Re}\qty[\psi^i]-\im \textnormal{Im}\qty[\psi^i]
\end{align*}

This could be seen as a change of variables from a model of $2N$ free real fermions, which of course enjoy a $O\qty(2N)$ 
symmetry, and as $U\qty(N)\subset O\qty(2N)$, it's possible for us to have a $SU\qty(N)$ symmetry. We impose the 
following transformations,

\begin{align*}
    {\psi}'^i=\tensor{\Lambda}{^i_j}\psi^j,\ \ \ {\tilde\psi}'^i=\tensor{{\Lambda^\ast}}{^i_j}{\tilde \psi}^j
\end{align*}

So that the action is,

\begin{align*}
    S&=\frac{1}{4\pi}\int\dd[2]{z}\delta_{ij}{\tilde\psi}^i\bar\partial\psi^j\\
    S'&=\frac{1}{4\pi}\int\dd[2]{z}\delta_{ij}{\tilde\psi}'^i\bar\partial\psi'^j\\
    S'&=\frac{1}{4\pi}\int\dd[2]{z}\delta_{ij}\tensor{{\Lambda^\ast}}{^i_k}\tensor{\Lambda}{^j_l}{\tilde\psi}^k\bar\partial\psi^l
\end{align*}

From where we get our long waited $U\qty(N)$ constraint, $\delta_{ij}\tensor{{\Lambda^\ast}}{^i_k}\tensor{\Lambda}{^j_l}=\delta_{kl}$, but, 
no longer we have any problem requiring the group to be connected, as $U\qty(N)$ is already connected, but nevertheless, there is a number of 
reasons why we would prefer to have only $SU\qty(N)$, to go from $U\qty(N)$ to $SU\qty(N)$ is simple, just impose $\Det\qty[\Lambda]=1$.

% \begin{align*}
%     \delta_{kl}&=\delta_{ij}\tensor{\Lambda}{^i_k}\tensor{\Lambda}{^j_l}\\
%     \delta_{kl}&=\delta_{ij}\qty(\tensor{\delta}{^i_k}+\tensor{\omega}{^i_k}+\mathcal O\qty(\omega^2))\qty(\tensor{\delta}{^j_l}+\tensor{\omega}{^j_l}+\mathcal O\qty(\omega^2))\\
%     \delta_{kl}&=\delta_{kl}+\omega_{lk}+\omega_{kl}+\mathcal O\qty(\omega^2)\\
%     \omega_{lk}&=-\omega_{kl}
% \end{align*}

% We can use this information to write,

% \begin{align*}
%     \tensor{\Lambda}{^i_j}&=\tensor{\delta}{^i_j}+\tensor{\omega}{^i_j}+\mathcal O\qty(\omega^2)\\
%     \tensor{\Lambda}{^i_j}&=\tensor{\delta}{^i_j}+\frac12\tensor{\omega}{_l_k}\qty(\tensor{\delta}{^i^l}\tensor{\delta}{^k_j}-\tensor{\delta}{^i^k}\tensor{\delta}{^l_j})+\mathcal O\qty(\omega^2)\\
%     \tensor{\Lambda}{^i_j}&=\tensor{\delta}{^i_j}-\frac\im2\tensor{\omega}{_l_k}\im\qty(\tensor{\delta}{^i^l}\tensor{\delta}{^k_j}-\tensor{\delta}{^i^k}\tensor{\delta}{^l_j})+\mathcal O\qty(\omega^2)\\
%     \tensor{\Lambda}{^i_j}&=\tensor{\delta}{^i_j}-\frac\im2\tensor{\omega}{_{\qty[lk]}}\tensor{T}{^{\qty[lk]}^i_j}+\mathcal O\qty(\omega^2)\\
%     \tensor{\Lambda}{^i_j}&=\tensor{\exp\qty(-\frac\im 2\omega_{\qty[lk]}\tensor{T}{^{\qty[lk]}})}{^i_j}
% \end{align*}

% To obtain the current we do again the same trick of promoting $\omega_{\qty[lk]}$ to a function and computing the change in the action,

% \begin{align*}
%     S&=\frac{1}{4\pi}\int\dd[2]{z}\delta_{ij}\psi^i\bar\partial\psi^j\\
%     S'&=\frac{1}{4\pi}\int\dd[2]{z}\delta_{ij}\psi'^i\bar\partial\psi'^j\\
%     S'&=\frac{1}{4\pi}\int\dd[2]{z}\delta_{ij}\tensor{\Lambda}{^i_k}\psi^k\bar\partial\qty(\tensor{\Lambda}{^j_l}\psi^l)\\
%     S'&=\frac{1}{4\pi}\int\dd[2]{z}\delta_{ij}\tensor{\Lambda}{^i_k}\psi^k\qty{\tensor{\Lambda}{^j_l}\bar\partial\psi^l+\psi^l\bar\partial\tensor{\Lambda}{^j_l}}\\
%     S'&=\frac{1}{4\pi}\int\dd[2]{z}\delta_{ij}\tensor{\Lambda}{^i_k}\psi^k\qty{\tensor{\Lambda}{^j_l}\bar\partial\psi^l-\frac\im 2\tensor{T}{^{\qty[ab]}^m_l}\psi^l\tensor{\Lambda}{^j_m}\bar\partial\omega_{\qty[ab]}}\\
%     S'&=\frac{1}{4\pi}\int\dd[2]{z}\tensor{\Lambda}{_j_k}\psi^k\tensor{\Lambda}{^j_l}\bar\partial\psi^l-\frac{1}{4\pi}\frac\im 2\int\dd[2]{z}\tensor{\Lambda}{_j_k}\psi^k\tensor{T}{^{\qty[ab]}^m_l}\psi^l\tensor{\Lambda}{^j_m}\bar\partial\omega_{\qty[ab]}\\
%     S'&=\frac{1}{4\pi}\int\dd[2]{z}\delta_{kl}\psi^k\bar\partial\psi^l-\frac{1}{4\pi}\frac\im 2\int\dd[2]{z}\delta_{km}\psi^k\tensor{T}{^{\qty[ab]}^m_l}\psi^l\bar\partial\omega_{\qty[ab]}\\
%     S'&=S-\frac{1}{4\pi}\frac\im 2\int\dd[2]{z}\psi^k\tensor{T}{^{\qty[ab]}_k_l}\psi^l\bar\partial\omega_{\qty[ab]}\\
%     \delta S&=-\frac{1}{4\pi}\frac\im 2\int\dd[2]{z}\bar\partial\qty{\psi^k\tensor{T}{^{\qty[ab]}_k_l}\psi^l\omega_{\qty[ab]}}+\frac{1}{4\pi}\frac\im 2\int\dd[2]{z}\bar\partial\qty{\psi^k\tensor{T}{^{\qty[ab]}_k_l}\psi^l}\omega_{\qty[ab]}
% \end{align*}

% With suitable boundary conditions on $\omega_{\qty[ab]}$, or, with no boundaries, we conclude,

% \begin{align*}
%     \delta S&=\frac{1}{4\pi}\frac\im 2\int\dd[2]{z}\bar\partial\qty{\psi^k\tensor{T}{^{\qty[ab]}_k_l}\psi^l}\omega_{\qty[ab]}
% \end{align*}

% Demoting $\omega_{\qty[ab]}$ back to being a constant --- so that $\delta S=0$ ---, we obtain,

% \begin{align*}
%     0&=-\frac{\omega_{\qty[ab]}}{2\pi\im}\frac1 2\tensor{T}{^{\qty[ab]}_k_l}\int\dd[2]{z}\bar\partial\qty{\psi^k\psi^l}
% \end{align*}

% Again, the usual normalization is done by removing $-\frac{\omega_{\qty[ab]}}{2\pi\im}$, so that the current is,

% \begin{align*}
%     j^{\qty[ab]}\qty(z)&=\frac1 2\tensor{T}{^{\qty[ab]}_k_l}\cnord{\psi^k\psi^l}\qty(z)
% \end{align*}