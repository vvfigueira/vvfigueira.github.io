\problem{}
\probitem{}

Our action is,

\begin{align*}
    S&=\frac{1}{4\pi}\int\dd[2]{z}\psi\bar\partial\psi
\end{align*}

To obtain the equation of motion is simple, first we set up the path integral of a total derivative, which is zero, the argument of the total derivative 
we set to $\exp\qty(-S)\mathcal O$, where $\mathcal O$ is any combination of local fields that does not contain $\psi\qty(z_1,{\bar z}_1)$, then,

\begin{align*}
    0&=\int\mathcal D\psi\fdv{}{\psi\qty(z_1,\bar z_1)}\qty[\exp\qty(-S)\mathcal O]\\
    0&=\int\mathcal D\psi\fdv{}{\psi\qty(z_1,\bar z_1)}\qty[\exp\qty(-S)]\mathcal O
\end{align*}

As $\fdv{}{\psi\qty(z_1,{\bar z}_1)}\mathcal O$ is zero except for $\mathcal O=\psi\qty(z_1,{\bar z}_1)$, to which it's $\delta^{\qty(2)}\qty(0)$, but, that's not the case,

\begin{align*}
    0&=\int\mathcal D\psi\fdv{}{\psi\qty(z_1,\bar z_1)}\qty[\exp\qty(-S)]\mathcal O\\
    0&=-\frac{1}{4\pi}\int\mathcal D\psi\fdv{}{\psi\qty(z_1,\bar z_1)}\qty[\int\dd[2]{z_2}\psi\qty(z_2,\bar z_2)\partial_{{\bar z}_2}\psi\qty(z_2,{\bar z}_2)]\mathcal O\exp\qty(-S)\\
    0&=-\frac{1}{4\pi}\int\mathcal D\psi\int\dd[2]{z_2}\qty[\delta^{\qty(2)}\qty(z_2-z_1)\partial_{{\bar z}_2}\psi\qty(z_2,{\bar z}_2)-\psi\qty(z_2,\bar z_2)\partial_{{\bar z}_2}\delta^{\qty(2)}\qty(z_2-z_1)]\mathcal O\exp\qty(-S)
\end{align*}

As long as we're dealing with the closed fermion, we need not to worry about the boundary conditions, as there is no boundary, so,

\begin{align*}
    0&=-\frac{1}{4\pi}\int\mathcal D\psi\int\dd[2]{z_2}\qty[\delta^{\qty(2)}\qty(z_2-z_1)\partial_{{\bar z}_2}\psi\qty(z_2,{\bar z}_2)+\partial_{{\bar z}_2}\psi\qty(z_2,\bar z_2)\delta^{\qty(2)}\qty(z_2-z_1)]\mathcal O\exp\qty(-S)\\
    0&=-\frac{1}{2\pi}\int\mathcal D\psi\partial_{{\bar z}_1}\psi\qty(z_1,{\bar z}_1)\mathcal O\exp\qty(-S)
\end{align*}

What in the Operator formalism would account for some radially ordered expectation value under some state, which is specified with the boundary conditions on the path integral itself, 

\begin{align*}
    \partial_{{\bar z}_1}\expval{\psi\qty(z_1,{\bar z}_1)\mathcal O}&=0
\end{align*}

As both the operator $\mathcal O$ and the state are arbitrary, as long as there is no insertion of $\psi\qty(z_1,{\bar z}_1)$, we conclude the 
equation of motion in the operator form is just,

\begin{align*}
    \bar\partial\psi=0\Rightarrow \psi\qty(z,\bar z)\equiv\psi\qty(z)
\end{align*}

That is, $\psi$ is, at least, a meromorphic operator/function. Now, let's repeat this to obtain the two point function, this time we include in 
$\mathcal O$ a single factor of $\psi\qty(z_1,{\bar z}_1)$ disguised as $\psi\qty(z_2,{\bar z}_2)$, this also can be phrased as we being interested 
in the limit $z_2\rightarrow z_1$,

\begin{align*}
    0&=\int\mathcal D\psi\fdv{}{\psi\qty(z_1,\bar z_1)}\qty[\exp\qty(-S)\psi\qty(z_2,{\bar z}_2)\mathcal O]\\
    0&=\int\mathcal D\psi\fdv{}{\psi\qty(z_1,\bar z_1)}\qty[\exp\qty(-S)\psi\qty(z_2,{\bar z}_2)]\mathcal O\\
    0&=\int\mathcal D\psi\qty{\fdv{}{\psi\qty(z_1,\bar z_1)}\qty[\exp\qty(-S)]\psi\qty(z_2,{\bar z}_2)+\fdv{}{\psi\qty(z_1,{\bar z}_1)}\qty[\psi\qty(z_2,{\bar z}_2)]\exp\qty(-S)}\mathcal O\\
    0&=\int\mathcal D\psi\qty{\fdv{}{\psi\qty(z_1,\bar z_1)}\qty[-S]\psi\qty(z_2,{\bar z}_2)+\delta^{\qty(2)}\qty(z_2-z_1)}\exp\qty(-S)\mathcal O
\end{align*}

As we derived before, $\fdv{}{\psi\qty(z_1,\bar z_1)}\qty[-S]=-\frac{1}{2\pi}\partial_{{\bar z}_1}\psi\qty(z_1,{\bar z}_1)$,

\begin{align*}
    0&=\int\mathcal D\psi\qty{-\frac{1}{2\pi}\partial_{{\bar z}_1}\psi\qty(z_1,{\bar z}_1)\psi\qty(z_2,{\bar z}_2)+\delta^{\qty(2)}\qty(z_2-z_1)}\exp\qty(-S)\mathcal O
\end{align*}

Which is, translating to the operator formalism,

\begin{align*}
    \partial_{{\bar z}_1}\expval{\psi\qty(z_1)\psi\qty(z_2)\mathcal O}&=\expval{2\pi\delta^{\qty(2)}\qty(z_2-z_1)\mathcal O}\\
    \partial_{{\bar z}_1}\psi\qty(z_1)\psi\qty(z_2)&=2\pi\delta^{\qty(2)}\qty(z_2-z_1),\ \ \ \abs{z_1}\geq\abs{z_2}
\end{align*}

So that we can interpret this as a operator equality. The condition $\abs{z_1}\geq \abs{z_2}$ is due to the implicit radial ordering in the expectation 
value, and also notice that in the path integral formulation $\psi=\psi\qty(z_1,{\bar z}_1)$, but in the operator formalism $\psi=\psi\qty(z_1)$, as every operator is 
always `\textit{on-shell}', while the path integral integrands aren't. Now we integrate this two point function, pick any compact closed region $R$ in the complex plane which 
contains $z_2$ and not any other of the points of the insertions $\mathcal O$, with also the boundary being a continuous curve, for all $m\in\mathbb N$ the following is true,

\begin{align*}
    \partial_{{\bar z}_1}\expval{\psi\qty(z_1)\psi\qty(z_2)\mathcal O}&=2\pi\delta^{\qty(2)}\qty(z_2-z_1)\expval{\mathcal O}\\
    \qty(z_1-z_2)^m\partial_{{\bar z}_1}\expval{\psi\qty(z_1)\psi\qty(z_2)\mathcal O}&=2\pi\qty(z_1-z_2)^m\delta^{\qty(2)}\qty(z_2-z_1)\expval{\mathcal O},\ \ \ m\in\mathbb N\\
    \int\limits_R\dd[2]{z_1}\qty(z_1-z_2)^m\partial_{{\bar z}_1}\expval{\psi\qty(z_1)\psi\qty(z_2)\mathcal O}&=2\pi\int\limits_R\dd[2]{z_1}\qty(z_1-z_2)^m\delta^{\qty(2)}\qty(z_2-z_1)\expval{\mathcal O},\ \ \ m\in\mathbb N\\
    \int\limits_R\dd[2]{z_1}\partial_{{\bar z}_1}\qty{\qty(z_1-z_2)^m\expval{\psi\qty(z_1)\psi\qty(z_2)\mathcal O}}&=2\pi\qty(z_1-z_2)^m\eval_{z_1=z_2}\expval{\mathcal O},\ \ \ m\in\mathbb N
\end{align*}

In the last line we used the fact `\textit{trivial}' fact that $\partial_{{\bar z}_1}\qty(z_1-z_2)^m=0$, which is valid for $m\in\mathbb N$, but isn't for 
negative non-integer values.

\begin{align*}
    \int\limits_R\dd[2]{z_1}\partial_{{\bar z}_1}\qty{\qty(z_1-z_2)^m\expval{\psi\qty(z_1)\psi\qty(z_2)\mathcal O}}&=2\pi\qty(z_1-z_2)^m\eval_{z_1=z_2}\expval{\mathcal O},\ \ \ m\in\mathbb N\\
    -\im\int\limits_{\partial R}\dd{z_1}\qty(z_1-z_2)^m\expval{\psi\qty(z_1)\psi\qty(z_2)\mathcal O}&=2\pi\delta_{m,0}\expval{\mathcal O},\ \ \ m\in\mathbb N
\end{align*}

This is simply the complex version of the divergence theorem,

\begin{align*}
    -\im\int\limits_{\partial R}\dd{z_1}\qty(z_1-z_2)^m\expval{\psi\qty(z_1)\psi\qty(z_2)\mathcal O}&=2\pi\delta_{m,0}\expval{\mathcal O},\ \ \ m\in\mathbb N\\
    \int\limits_{\partial R}\frac{\dd{z_1}}{2\pi\im}\qty(z_1-z_2)^m\expval{\psi\qty(z_1)\psi\qty(z_2)\mathcal O}&=\delta_{m,0}\expval{\mathcal O},\ \ \ m\in\mathbb N
\end{align*}

Well, the left-hand side of this last equation picks up the pole of $m$-th order of the expression $\expval{\psi\qty(z_1)\psi\qty(z_2)\mathcal O}$ under $z_1\rightarrow z_2$, but, 
the right-hand side is only non-zero for $m=0$, this is telling us that $\expval{\psi\qty(z_1)\psi\qty(z_2)\mathcal O}$ doesn't have any pole besides the first order one with residue $\expval{\mathcal O}$, 
so we can write,

\begin{align*}
    \expval{\psi\qty(z_1)\psi\qty(z_2)\mathcal O}=\frac{1}{z_1-z_2}\expval{\mathcal O}+\textnormal{regular}
\end{align*}

Or, in the operator formalism,

\begin{align*}
    \psi\qty(z_1)\psi\qty(z_2)&=\frac{1}{z_1-z_2}+\textnormal{regular},\ \ \ \abs{z_1}\geq \abs{z_2}\\
    \psi\qty(z_1)\psi\qty(z_2)&=\frac{1}{z_1-z_2}+\cnord{\psi\qty(z_1)\psi\qty(z_2)},\ \ \ \abs{z_1}\geq \abs{z_2}
\end{align*}

Which is the OPE.

\probitem{}

The energy momentum tensor has two components, a meromorphic one $T\qty(z)$, and an anti-meromorphic one $\bar T\qty(\bar z)$, 
as we have at our disposal only a weight $\qty(\frac12,0)$ field, it's impossible to construct a anti-meromorphic energy momentum tensor, 
but, for the meromorphic one, let us see what kind of combinations have the right weights. First, we remember that our energy momentum tensor 
must be normal ordered, so, all the possible normal ordered weight $\qty(2,0)$ combinations of $\psi$ and derivatives are,

\begin{align*}
    \cnord{\psi\partial\psi}\qty(z),\ \cnord{(\partial\psi)\psi}\qty(z), \ \cnord{\partial\qty(\psi\psi)}\qty(z),\ \cnord{\psi\psi\psi\psi}\qty(z)
\end{align*} 

The third and fourth are zero due to the fermionic statistic, and the first and second are linear dependent also because the 
fermionic statistic. Hence, up to a unknown constant $\alpha$, the meromorphic energy momentum tensor is,

\begin{align*}
    T\qty(z)=\alpha\cnord{\psi\partial\psi}\qty(z)
\end{align*}

Let's use this expression to compute the following,

\begin{align*}
    \cnord{T\qty(z_1) T\qty(z_2)}&=T\qty(z_1)T\qty(z_2)+\alpha^2\textnormal{ contractions}\qty{\cnord{\qty(\psi\partial\psi)\qty(z_1)\qty(\psi\partial\psi)\qty(z_2)}},\ \ \ \abs{z_1}\geq \abs{z_2}
\end{align*}

Where the contraction part means exchanging pairs of $\psi$ with different $z$ by $-\qty(z_1-z_2)^{-1}$, as cause of the fermion 
statistic, we have to pay attention to the signs, also, we'll omit for now the $\abs{z_1}\geq\abs{z_2}$, but it will still be imposed, 

\begin{align*}
    \cnord{T\qty(z_1) T\qty(z_2)}&={ T\qty(z_1)T\qty(z_2)}+\alpha^2\cnord{\wick[offset=1.2em]{(\c{\psi}\partial\psi)\qty(z_1)(\c{\psi}\partial\psi)\qty(z_2)}}+\alpha^2\cnord{\wick[offset=1.2em]{(\c{\psi}\partial\psi)\qty(z_1)({\psi}\partial\c{\psi})\qty(z_2)}}\\
    &\quad\quad\quad+\alpha^2\cnord{\wick[offset=1.2em]{({\psi}\partial\c{\psi})\qty(z_1)(\c{\psi}\partial{\psi})\qty(z_2)}}+\alpha^2\cnord{\wick[offset=1.2em]{({\psi}\partial\c{\psi})\qty(z_1)({\psi}\partial\c{\psi})\qty(z_2)}}\\
    &\quad\quad\quad+\alpha^2\cnord{\wick[offset=1.2em]{(\c1{\psi}\partial\c2{\psi})\qty(z_1)(\c1{\psi}\partial\c2{\psi})\qty(z_2)}}+\alpha^2\cnord{\wick[offset=1.2em]{(\c1{\psi}\partial\c2{\psi})\qty(z_1)(\c2{\psi}\partial\c1{\psi})\qty(z_2)}}\\
    \cnord{T\qty(z_1) T\qty(z_2)}&={ T\qty(z_1)T\qty(z_2)}+\frac{1}{z_1-z_2}\alpha^2\cnord{\partial\psi\qty(z_1)\partial\psi\qty(z_2)}-\partial_{z_2}\qty(\frac{1}{z_1-z_2})\alpha^2\cnord{\partial\psi\qty(z_1)\psi\qty(z_2)}\\
    &\quad\quad\quad-\alpha^2\partial_{z_1}\qty(\frac{1}{z_1-z_2})\cnord{{\psi}\qty(z_1)\partial{\psi}\qty(z_2)}+\alpha^2\partial_{z_1}\partial_{z_2}\qty(\frac{1}{z_1-z_2})\cnord{{\psi}\qty(z_1){\psi}\qty(z_2)}\\
    &\quad\quad\quad+\alpha^2\frac{1}{z_1-z_2}\partial_{z_1}\partial_{z_2}\qty(\frac{1}{z_1-z_2})-\alpha^2\partial_{z_2}\qty(\frac{1}{z_1-z_2})\partial_{z_1}\qty(\frac{1}{z_1-z_2})\\
    \cnord{T\qty(z_1) T\qty(z_2)}&={ T\qty(z_1)T\qty(z_2)}+\frac{\alpha^2}{z_1-z_2}\cnord{\partial\psi\qty(z_1)\partial\psi\qty(z_2)}-\frac{\alpha^2}{\qty(z_1-z_2)^2}\cnord{\partial\psi\qty(z_1)\psi\qty(z_2)}\\
    &\quad\quad\quad+\frac{\alpha^2}{\qty(z_1-z_2)^2}\cnord{{\psi}\qty(z_1)\partial{\psi}\qty(z_2)}+\partial_{z_1}\qty(\frac{\alpha^2}{\qty(z_1-z_2)^2})\cnord{{\psi}\qty(z_1){\psi}\qty(z_2)}\\
    &\quad\quad\quad+\frac{\alpha^2}{z_1-z_2}\partial_{z_1}\qty(\frac{1}{\qty(z_1-z_2)^2})-\frac{\alpha^2}{\qty(z_1-z_2)^2}\partial_{z_1}\qty(\frac{1}{z_1-z_2})\\
    \cnord{T\qty(z_1) T\qty(z_2)}&={ T\qty(z_1)T\qty(z_2)}+\frac{\alpha^2}{z_1-z_2}\cnord{\partial\psi\qty(z_1)\partial\psi\qty(z_2)}-\frac{\alpha^2}{\qty(z_1-z_2)^2}\cnord{\partial\psi\qty(z_1)\psi\qty(z_2)}\\
    &\quad\quad\quad+\frac{\alpha^2}{\qty(z_1-z_2)^2}\cnord{{\psi}\qty(z_1)\partial{\psi}\qty(z_2)}-2\frac{\alpha^2}{\qty(z_1-z_2)^3}\cnord{{\psi}\qty(z_1){\psi}\qty(z_2)}\\
    &\quad\quad\quad-2\frac{\alpha^2}{z_1-z_2}\frac{1}{\qty(z_1-z_2)^3}+\frac{\alpha^2}{\qty(z_1-z_2)^2}\frac{1}{\qty(z_1-z_2)^2}\\
    \cnord{T\qty(z_1) T\qty(z_2)}&={ T\qty(z_1)T\qty(z_2)}+\frac{\alpha^2}{z_1-z_2}\cnord{\partial\psi\qty(z_1)\partial\psi\qty(z_2)}-\frac{\alpha^2}{\qty(z_1-z_2)^2}\cnord{\partial\psi\qty(z_1)\psi\qty(z_2)}\\
    &\quad\quad\quad+\frac{\alpha^2}{\qty(z_1-z_2)^2}\cnord{{\psi}\qty(z_1)\partial{\psi}\qty(z_2)}-2\frac{\alpha^2}{\qty(z_1-z_2)^3}\cnord{{\psi}\qty(z_1){\psi}\qty(z_2)}\\
    &\quad\quad\quad-\frac{\alpha^2}{\qty(z_1-z_2)^4}
\end{align*}

To proceed further we have to Taylor expand both $\psi\qty(z_1),\partial\psi\qty(z_1)$, of course this will generate regular terms in our expansion, as for example, the second term in the right-hand side,

\begin{align*}
    \frac{1}{z_1-z_2}\cnord{\partial\psi\qty(z_1)\partial\psi\qty(z_2)}&=\frac{1}{z_1-z_2}\cnord{\partial\psi\qty(z_2)\partial\psi\qty(z_2)}+\sum\limits_{n=1}^\infty\frac{1}{z_1-z_2}\frac{1}{n!}\qty(z_1-z_2)^n\cnord{\partial^n\psi\qty(z_2)\partial\psi\qty(z_2)}\\
    \frac{1}{z_1-z_2}\cnord{\partial\psi\qty(z_1)\partial\psi\qty(z_2)}&=\frac{1}{z_1-z_2}\cnord{\partial\psi\partial\psi}\qty(z_2)+\sum\limits_{n=1}^\infty\frac{\qty(z_1-z_2)^{n-1}}{n!}\cnord{\partial^n\psi\partial\psi}\qty(z_2)
\end{align*}

It's clear that the sum in the right-hand side is of only regular terms, so that,

\begin{align*}
    \frac{1}{z_1-z_2}\cnord{\partial\psi\qty(z_1)\partial\psi\qty(z_2)}&=\frac{1}{z_1-z_2}\cnord{\partial\psi\partial\psi}\qty(z_2)+\textnormal{regular}
\end{align*}

But $\cnord{\partial\psi\partial\psi}$ is zero by statistics, hence, 

\begin{align*}
    \frac{1}{z_1-z_2}\cnord{\partial\psi\qty(z_1)\partial\psi\qty(z_2)}&=\textnormal{regular}
\end{align*}

We do this procedure for all the terms in the expansion,

\begin{align*}
    \cnord{T\qty(z_1) T\qty(z_2)}&={ T\qty(z_1)T\qty(z_2)}-\frac{\alpha^2}{\qty(z_1-z_2)^2}\cnord{\partial\psi\psi}\qty(z_2)-\frac{\alpha^2}{\qty(z_1-z_2)}\cnord{\partial^2\psi\psi}\qty(z_2)\\
    &\quad\quad\quad+\frac{\alpha^2}{\qty(z_1-z_2)^2}\cnord{{\psi}\partial{\psi}}\qty(z_2)+\frac{\alpha^2}{\qty(z_1-z_2)}\cnord{\partial{\psi}\partial{\psi}}\qty(z_2)\\
    &\quad\quad\quad-2\frac{\alpha^2}{\qty(z_1-z_2)^3}\cnord{{\psi}{\psi}}\qty(z_2)-2\frac{\alpha^2}{\qty(z_1-z_2)^2}\cnord{\partial{\psi}{\psi}}\qty(z_2)-\frac{\alpha^2}{\qty(z_1-z_2)}\cnord{\partial^2{\psi}{\psi}}\qty(z_2)\\
    &\quad\quad\quad-\frac{\alpha^2}{\qty(z_1-z_2)^4}+\textnormal{regular}
\end{align*}

Removing the terms which are zero by statistics and grouping the others,

\begin{align*}
    \cnord{T\qty(z_1) T\qty(z_2)}&={ T\qty(z_1)T\qty(z_2)}+\frac{4\alpha^2}{\qty(z_1-z_2)^2}\cnord{\psi\partial\psi}\qty(z_2)+\frac{2\alpha^2}{\qty(z_1-z_2)}\cnord{\psi\partial^2\psi}\qty(z_2)\\
    &\quad\quad\quad-\frac{\alpha^2}{\qty(z_1-z_2)^4}+\textnormal{regular}
\end{align*}

As $\cnord{\partial\psi\partial\psi}\equiv0$ we can add it as we please,

\begin{align*}
    \cnord{T\qty(z_1) T\qty(z_2)}&={ T\qty(z_1)T\qty(z_2)}+\frac{4\alpha^2}{\qty(z_1-z_2)^2}\cnord{\psi\partial\psi}\qty(z_2)+\frac{2\alpha^2}{\qty(z_1-z_2)}\cnord{\psi\partial^2\psi+\partial\psi\partial\psi}\qty(z_2)\\
    &\quad\quad\quad-\frac{\alpha^2}{\qty(z_1-z_2)^4}+\textnormal{regular}\\
    \cnord{T\qty(z_1) T\qty(z_2)}&={ T\qty(z_1)T\qty(z_2)}+\frac{4\alpha^2}{\qty(z_1-z_2)^2}\cnord{\psi\partial\psi}\qty(z_2)+\frac{2\alpha^2}{\qty(z_1-z_2)}\partial\cnord{\psi\partial\psi}\qty(z_2)\\
    &\quad\quad\quad-\frac{\alpha^2}{\qty(z_1-z_2)^4}+\textnormal{regular}\\
    \cnord{T\qty(z_1) T\qty(z_2)}&={ T\qty(z_1)T\qty(z_2)}+\frac{4\alpha}{\qty(z_1-z_2)^2}T\qty(z_2)+\frac{2\alpha}{\qty(z_1-z_2)}\partial T\qty(z_2)\\
    &\quad\quad\quad-\frac{\alpha^2}{\qty(z_1-z_2)^4}+\textnormal{regular}
\end{align*}

Well, $\cnord{T\qty(z_1)T\qty(z_2)}$ itself is regular, then,

\begin{align*}
    { T\qty(z_1)T\qty(z_2)}&=-\frac{4\alpha}{\qty(z_1-z_2)^2}T\qty(z_2)-\frac{2\alpha}{\qty(z_1-z_2)}\partial T\qty(z_2)+\frac{\alpha^2}{\qty(z_1-z_2)^4}+\textnormal{regular}
\end{align*}

But we know the general form of the $TT$ OPE,

\begin{align*}
    { T\qty(z_1)T\qty(z_2)}&=\frac{2}{\qty(z_1-z_2)^2}T\qty(z_2)+\frac{1}{\qty(z_1-z_2)}\partial T\qty(z_2)+\frac{c}{2\qty(z_1-z_2)^4}+\textnormal{regular}
\end{align*}

From where is easy to read $\alpha=-\frac12$, and also, $c=2\alpha^2=2\frac14=\frac12$, that is, the meromorphic component of the 
energy momentum tensor, and the central charge are,

\begin{align*}
    T\qty(z)&=-\frac12\cnord{\psi\partial\psi}\qty(z),\ \ \ c=\frac12
\end{align*}

\probitem{}
\probitem{}
\probitem{}
\probitem{}