\section{Formulation of SRS}
\label{sec:form}

\subsection{Intuitive Description of SRS}

As mentioned in the last section, for the action\footnote{We're going to forget about the 
world-sheet gravitino.} \eqref{action:sst1} to be on-shell invariant under \eqref{susy:pol1}, 
it's needed to add an auxiliary field in it. We'll try to understand why, and at the same time introduce the 
superspace formalism. We know that the supersymmetry algebra contains a commutation relation such,
$$\comm{Q}{Q}\propto P$$
so that they have to be interpreted as a space-time --- in this case world-sheet --- symmetry (redundancy), 
instead of a internal symmetry. This is interesting, because we can understand the world-sheet as being 
the quotient group, $$\mathbb R^2\cong\mathbb C\cong ISO\qty(1,1)/SO(1,1)$$ what implies that we 
can also understand our space-time (world-sheet) with SUSY --- Super-Space ---, as being the quotient group of the 
Super Poincaré group with respect to the Lorentz group, 
$$\textnormal{Super-Space}\cong\mathbb C^{1|1}\cong ISO\qty(1,1|1)/SO(1,1)$$

Why does this is of relevance? Due to we being able to write a generic element of the group $ISO(1,1|1)$ as,
$$ISO(1,1|1)\ni g(\sigma,\theta,\omega)=\exp\qty(-\im \sigma_aP^a-\im \theta_A Q^A+\frac\im2\omega_{ab}J^{ab})$$
if we factor out the Lorentz group, we obtain an expression for the elements of the Super-Space, they're 
parametrized by two bosonic coordinates $\sigma^a$, and two fermionic coordinates $\theta^A$ --- 
the fermionic nature is guarantee by the fermionic nature of the SUSY generators $Q^A$ ---. As 
we're ultimately interested in the complex structure, we switch to $z,\bar z,\theta,\bar\theta$ by 
the usual substitutions, $z=\sigma^1-\sigma^0$ and similarly for $\theta$. These coordinates we introduced 
are useful because they allow for a differential representation of the SUSY algebra, which is analogous 
to the differential representation of the translations which we're accustomed $P\sim L_{-1}\sim \partial_z$. 
Inspection shows that the right choice is $Q_\theta=\partial_\theta-\theta\partial_z$, with the analogous 
anti-holomorphic one, $$\comm{Q_\theta}{Q_\theta}=2\partial_\theta\partial_\theta-2\theta\partial_z\partial_\theta-2\partial_\theta\qty(\theta\partial_z)+2\theta\partial_z\qty(\theta\partial_z)=-2\partial_z$$
what neatly satisfy the correct algebra. The good thing about this kind of differential representation is that 
possesses a natural action on functions of the Super-Space --- which we'll shown in a bit ---, instead of the mysterious action in \eqref{susy:pol1}. 
Together with this differential representation of the SUSY generator, it's useful to introduce a 
\textit{covariant derivative} $D_\theta$, in the sense that it preserves the supersymmetry transformation 
of the object it's acting on, $$D_\theta=\partial_\theta+\theta\partial_z,\ \ \ \comm{D_\theta}{Q_\theta}=0,\ \ \ \comm{D_\theta}{D_\theta}=2\partial_z$$
while we'll only be able to show that this is the right choice in the next subsection, there are a few 
motifs behind this definition. Remember, our main goal here is to obtain a geometric visualization of this 
supersymmetry, in other words, a geometric visualization of the Super Conformal group. As we know, a conformal 
transformation can be defined as being a coordinate change such that $\partial_z$ is changed to a 
multiple of itself, as we already argued here, in the Super Conformal group, we have not only $z$, 
but also $\theta$, so we need a differential operator such: (i) It commutes with the SUSY generator. 
(ii) A Super Conformal transformation can be defined as a coordinate change $z,\theta\rightarrow z'(z,\bar z,\theta,\bar\theta),\theta'(z,\bar z,\theta,\bar\theta)$ 
that preserves this differential operator. Notice that $\partial_z$ is consistent with condition (i), but, 
if we try to impose condition (ii) we gain only the usual bosonic conformal transformations. Our claim is, 
the most general differential operator that satisfy both conditions is a multiple of $D_\theta$. We'll 
not prove here, but under such a super conformal transformation this differential operator transforms as 
$D_\theta=\qty(D_\theta\theta')D_{\theta'}$. 

The analogy proposes us to define Super Fields. 
A Super Field is a function of the Super-Space $\mathbb A\qty(z,\bar z,\theta,\bar\theta)$, it's said to 
have weights $(h,\tilde h)$ if it changes as, $$\qty(D_\theta\theta')^{2h}\qty(D_{\bar\theta}{\bar\theta}')^{2\tilde h}\mathbb A'\qty(z',\bar z',\theta',\bar\theta')=\mathbb A\qty(z,\bar z,\theta,\bar\theta)$$
a super field turns out to be a useful construction due to the natural action of a SUSY transformation,
$$\comm{\mathbb A\qty(z,\theta)}{Q}=-\im Q_\theta\mathbb A\qty(z,\theta)$$
and the natural transformation of the measure,
$$\dd[2]{z'}\dd[2]{\theta'}=\dd[2]{z}\dd[2]{\theta}D_\theta\theta'D_{\bar\theta}{\bar\theta}'$$
which allows for a easy construction of an action, as the measure transforms as a weight $(-\frac12,-\frac12)$ 
super field, we just need to integrate a $(\frac12,\frac12)$ super field, and as each covariant 
derivative transforms as $(\frac12,0)$, a natural candidate is the derivative of a $(0,0)$ super field $\mathbb X^\mu$. 
Due to the fermionic nature of the $\theta$, this can be expanded as, using a little of foresight,
$$\mathbb X^\mu\qty(z,\bar z,\theta,\bar\theta)=X^\mu\qty(z,\bar z)+\im\theta\psi^\mu\qty(z)+\im\bar\theta{\tilde\psi}^\mu\qty(\bar z)+\bar\theta\theta F^\mu\qty(z,\bar z)$$
here we see already our familiar fields $X^\mu,\psi^\mu$, and the presence of an additional field $F^\mu$, 
which transforms as $(\frac12,\frac12)$, and also non trivially under the SUSY, as hinted by $Q_\theta\mathbb X^\mu$, 
this field is necessary to ensure the off-shell SUSY invariance, and it modifies the \eqref{susy:pol1}. 
As we mentioned before, a super conformal invariant action can be build as,
\begin{align}
    S&=\frac{1}{4\pi}\int\limits_\Sigma\dd[2]{z}\dd[2]{\theta}D_{\bar\theta}\mathbb X^\mu D_\theta\mathbb X_\mu\label{action:sst2}
\end{align}
where the fermionic integration will of course only extract the term proportional to $\bar\theta\theta$, this 
computation is straightforward, giving,
\begin{align*}
    S&=\frac{1}{4\pi}\int\limits_\Sigma\dd[2]{z}\qty(\partial_zX^\mu\partial_{\bar z}X^\mu+\psi^\mu\partial_{\bar z}\psi_\mu+{\tilde\psi}^\mu\partial_z{\tilde\psi}_\mu+F^\mu F_\mu)
\end{align*}
exactly our starting action! With of course the auxiliary field $F^\mu$, which has the trivial equation of motion $F^\mu=0$. 
This implies that the action \eqref{action:sst1} is just \eqref{action:sst2} with the auxiliary field integrated out.

\subsection{Formal Definition of SRS}

Now we go on to formalize the description made before, a Super Riemann Surface is a 
special kind of a complex supermanifold, we're particularly interested in SRSs of 
dimension $1|1$, let's start step by step. 
\begin{definition}
    A \textbf{complex supermanifold} $\Sigma$ of dimension $1|1$ is a space locally isomorphic 
    to $\mathbb C^{1|1}$, that is, it's locally covered by coordinate charts $z|\theta:U\subset\Sigma\rightarrow\mathbb C^{1|1}$ 
    such that $z$ is a complex even coordinate, and $\theta$ is a complex odd coordinate.
\end{definition}
The definition of a SRS is build on top of this one,
\begin{definition}
    A \textbf{Super Riemann Surface} $\Sigma$ is a complex supermanifold of dimension $1|1$ that 
    possesses a completely non-integrable dimension $0|1$ holomorphic subbundle $\mathcal D\subset T\Sigma$.
\end{definition}
This definition might hide what it's trying to convey through other definitions, so, we're going to break it 
in pieces. $T\Sigma$ is the tangent bundle/space of the supermanifold $\Sigma$, which locally is spanned by a coordinate 
basis $\partial_\theta,\partial_z$, this is exactly the same notion that we have 
in real manifolds. Now, by \textit{dimension }$0|1$\textit{ holomorphic subbundle }$\mathcal D$ we 
mean we have a closed subspace of the tangent vector field space $T\Sigma$ in which every element is 
odd. Lastly, to say $\mathcal D$ is \textit{completely non-integrable} is to say\footnote{Here $\comm{\cdot}{\cdot}$ should 
be interpreted as a graded vector field Lie Bracket.},
\begin{align*}
    \forall D\in\mathcal D\,:\, D\textnormal{ non zero }\Rightarrow D^2\coloneq\frac12\comm{D}{D}\notin\mathcal D
\end{align*}
By $D$ being non-zero it means that when it's written in a given basis $z,\theta:U\subset\Sigma\rightarrow \mathbb C^{1|1}$,
\begin{align*}
    D\textnormal{ is non zero }\Leftrightarrow D\eval_U=a\qty(z,\theta)\partial_\theta+b\qty(z,\theta)\partial_z,\ \ \ a\qty(z,\theta)\neq 0
\end{align*}