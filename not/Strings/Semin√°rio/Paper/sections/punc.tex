\section{Punctures in SRS}
\label{sec:punc}

We will dwell with insertions of vertex operators in this last section, mainly because 
they're the arguments of a correlation function related to scattering, and also due to 
there being a slight sparkle of insight about why the R sector super Virasoro algebra 
don't preserve the structure $\mathcal D$ of the SRS. First, let us recall what we 
know of vertex operators insertions in BST, naively, given a vertex operator $V\qty(z)$ 
we \textit{insert} it at a point $z=z_0$, this is also called a puncture 
because is as if the chosen point $z=z_0$ was removed from the manifold --- weather we 
should just \textit{insert} $V$ at a chosen point $z=z_0$, or we integrate $V$ over 
the whole manifold is a matter of moduli space, we will not linger much about the 
moduli space of SRS, but still is a needed issue to deal with perturbative SST ---, but 
remember, the vertex operator cannot be anything, it has to be physical, that is, it has to 
satisfy the physical condition \[Q_{\textnormal{BRST}}V\qty(z_0)=0\] this imposes 
various constrains --- in particular with respect to the ghost number ---, the 
one we are particularly interested in is, \[L_n V\qty(z_0)=0,\ \ \ n\geq0\numberthis\label{brstconstraintbst}\] which 
comes from $Q_{\textnormal{BRST}}\sim\sum\limits_n c_{-n}L_n+\cdots$, despite the constrains imposed in the form of the 
vertex operator, there is a kind of overlooked constraint imposed into the insertion point $z_0$. The constraint is: 
the point $z_0$ has to be preserved by the BRST transformations, in particular, by \cref{brstconstraintbst}, the point 
$z_0$ has to be preserved by the actions of $L_n$. For the BST this is a trivial constraint, as the differential 
form of $L_n$ expanded at $z=z_0$ is $L_n=-\qty(z-z_0)^{n+1}\partial_z$, which trivially preserve the point $z=z_0$ 
for $n\geq0$.

If we would try to naively extend this definition to SRS, we get that we should insert a 
vertex operator $V\qty(z|\theta)$ at a chosen point $z_0|\theta_0$, we shouldn't forget that the vertex operator is 
subjected to the physicality conditions, \[Q_{\textnormal{BRST}} V\qty(z_0|\theta_0)=0\numberthis\label{BRSTcondition}\] here already we have to 
make some assumptions. To open the BRST operator in terms of the modes we have to say weather we're in the 
NS sector, or in the R sector. Let's start with our more familiar NS sector:

\subsection{NS sector}

The physicality condition, \cref{BRSTcondition}, open in the NS sector gives the following constraint, neglecting the ghost ones,
\begin{align*}
    L_n V\qty(z_0|\theta_0)&=0,\ \ \ n\geq 0\\
    G_r V\qty(z_0|\theta_0)&=0,\ \ \ r\geq \frac12
\end{align*}
again, those non-trivially restricts the form of the vertex operator, but, also, 
they impose conditions on the insertion point, as was shown in the BST case. To 
better understand what those conditions on the insertion point are let's go to the 
differential form of those operators, \cref{generators:ns2}, which we rewrite here not 
expanded around $z=0,\theta=0$, but around $z_0,\theta_0$,
\begin{align*}
    L_n&=-\qty(z-z_0)^{n+1}\partial_z-\frac{\qty(n+1)}{2}\qty(z-z_0)^n\qty(\theta-\theta_0)\partial_\theta\\
    G_r&=\qty(z-z_0)^{r+\frac12}\qty(\partial_\theta-\qty(\theta-\theta_0)\partial_z)
\end{align*}
It is clear that both these operators, for $n\geq0,r\geq\frac12$, when evaluated at the point $z_0,\theta_0$ are 
identically zero, hence, they preserve the insertion point. In other words, for the NS sector it actually 
makes sense to \textit{insert} a vertex operator at a point of the SRS. Another point about the NS sector 
that is elucidative is the boundary conditions phrased in the superfield formalism. The usual 
way they are imposed is in the complex cylinder, by $\psi^\mu\qty(w+2\pi)=-\psi^\mu\qty(w)$, but notice, 
from \cref{xdecomposition} there is a nice description of the NS boundary condition as, 
\[\mathbb X^\mu\qty(z\e^{2\pi\im},\bar z\e^{-2\pi\im},-\theta,-\bar\theta)=\mathbb X^\mu\qty(z,\bar z,\theta,\bar\theta)\] 
This is kind of suggestive for us to go on with our identification $z\sim z\e^{2\pi\im}$, and also identify $\theta\sim-\theta$. 
Actually, this seems to be the case, as for the case of zero genus the SRS is $\mathbb{CP}^{1|1}$.

\subsection{R sector}

We know by our previous experience that something will go wrong with R fermions, let's start by supposing a 
vertex operator is inserted at a fixed point $z_0|\theta_0$ and analyzing the physicality 
condition, \cref{BRSTcondition},
\begin{align*}
    L_n V\qty(z_0|\theta_0)&=0,\ \ \ n\geq 0\\
    G_r V\qty(z_0|\theta_0)&=0,\ \ \ r\geq 0
\end{align*}
as we saw before, the super Virasoro generators in the R sector are different from the NS sector. Let's rewrite them here, \cref{rgenerators}, but expanded around 
the point $z_0|\theta_0$,
\begin{align*}
    L_n&=-\qty(z-z_0)^{n+1}\partial_z-\frac{n}{2}\qty(z-z_0)^n\qty(\theta-\theta_0)\partial_\theta\\
    G_r&=\qty(z-z_0)^{r}\qty(\partial_\theta-\qty(\theta-\theta_0)\qty(z-z_0)\partial_z)
\end{align*} 
for $L_n,G_r n\geq0,r\geq 1$ acting on the point $z_0|\theta_0$ there is no problem, they're identically zero, hence they preserve 
the insertion point. The only problematically generator is $G_0$, which when evaluated at the insertion point gives,
\[G_0\eval_{z_0|\theta_0}=\partial_\theta\]
This is not zero, neither preserves the insertion point, it generate a translation in the odd variable, 
$z_0|\theta_0\rightarrow z_0|\theta_0+\eta$. Hence, it is not possible for an insertion of a R fermion to be 
associated with a given marked point $z_0|\theta_0$ on the SRS. In fact, by this argument we get that an insertion 
of a R sector vertex operator has to be associated with a divisor, that is, a whole $0|1$ dimensional submanifold 
defined by the choice of a bosonic coordinate $z=z_0$. That is equivalent, in some sense, to say that a puncture of 
a R fermion in the SRS is associated with a singularity of the manifold at that point. This is not the correct statement. 
We can use our discoveries about the particularities of the R fermions to give a more precise meaning of what we mean 
by a singularity in the manifold. We've seen before that the generators of the super Virasoro algebra in the R sector, 
\cref{rgenerators}, cannot possibly preserve the structure the structure $\mathcal D$, \cref{lemma:basissuperconformal,lemma:nsgenerators}, 
but also, during the proof of \cref{lemma:formd}, in particular at \cref{temp:eq1}, we saw that the only way to have 
$\comm{\mathcal D}{\mathcal D}\notin\mathcal D$ is to have $\comm{\mathcal D}{\mathcal D}=0$ at some point. This also can be interpreted, 
in some sense, as being a singularity of the complex structure $\mathcal D$. All the hints points that the singularity 
happening in the degeneracy of the $0|1$ dimensional submanifold when a vertex operator is inserted, is the same 
happening when $\comm{\mathcal D}{\mathcal D}=0$. We will now try to reconcile these two aspects.

By the proof of \cref{lemma:formd}, in particular with \cref{temp:eq1}, we get that if at some point $z=z_0$ we have 
$\comm{D_{U_{z_0}}}{D_{U_{z_0}}}=0$, then $b_1\propto z-z_0$, and also exists a coordinate basis $z|\theta$ such that,
\[D^\ast_\theta\eqcolon D_{U_{z_0}}=\partial_\theta+\theta\qty(z-z_0)\partial_z\]
Notice, if we sit at an open set located away from $z=z_0$, it's always possible to make the coordinate change $\hat z=\ln\qty(z-z_0)$, 
this coordinate change induces the following,
\begin{align*}
    D^\ast_\theta&=\partial_\theta+\theta\qty(z-z_0)\partial_z\\
    D^\ast_\theta&=\partial_\theta+\theta\qty(z-z_0)\partial_z\hat z\partial_{\hat z}=\partial_\theta+\theta\partial_{\hat z}=D_\theta
\end{align*}
Hence, away from the degeneracy point $z=z_0$, the \cref{lemma:basissuperconformal} still holds for the complex structure 
$D^\ast_\theta=\partial_\theta+\theta\qty(z-z_0)\partial_z$, that is, the superconformal symmetries induced by $D^\ast_\theta$ 
away from $z=z_0$ are exactly the ones given by the NS sector super Virasoro algebra. The question is, what happens at $z=z_0$? 
\begin{lemma}
    Given a degenerated conformal structure $D^\ast_\theta=\partial_\theta+\theta\qty(z-z_0)\partial_z$, the set of all 
    vector fields $W$ that generate superconformal transformations $\comm{W}{D^\ast_\theta}\in\mathcal D$, 
    can be decomposed in a basis of even and odd vector fields such,
\end{lemma}
\begin{proof}
    We'll proceed similar to \cref{lemma:basissuperconformal}, $W=a\partial_\theta+b\partial_z$,
    \begin{align*}
    \comm{W}{D^\ast_\theta}&=\comm{a\partial_\theta+b\partial_z}{D^\ast_\theta}=a\comm{\partial_\theta}{D^\ast_\theta}\mp D^\ast_\theta a\partial_\theta+b\comm{\partial_z}{D^\ast_\theta}\mp D^\ast_\theta b\partial_z\\
    \comm{W}{D^\ast_\theta}&=a\qty(z-z_0)\partial_z\mp D^\ast_\theta a\partial_\theta+b\theta\partial_z\mp D^\ast_\theta b\partial_z\\
    \comm{W}{D^\ast_\theta}&=\mp D^\ast_\theta a\partial_\theta\mp\qty(D^\ast_\theta b\mp b\theta\mp a\qty(z-z_0))\partial_z
\end{align*}
    Here the condition $\comm{W}{D^\ast_\theta}\propto D^\ast_\theta$ is,
    \begin{align*}
        D^\ast_\theta b\mp b\theta\mp a \qty(z-z_0)&=D^\ast_\theta a\theta\qty(z-z_0)
    \end{align*}
\end{proof}