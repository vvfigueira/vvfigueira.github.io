\section{Introduction/Motivation}
\label{sec:intro}

The Bosonic String Theory (BST) is known to achieve several desirable properties which up to present date 
haven't been done in usual Quantum Field Theory, the most prominent one is it being a perturbatively 
renormalizable theory which contains in its spectrum a massless spin-2 particle, this 
perturbative computation of amplitudes in BST is almost only possible 
due to the heavy simplifications the 
anomaly free gauge group Diff$\qty(M)\times$Weyl allows\cite{polchinski:vol1}. This means, as in the path integral we're 
integrating over metrics, the gauge redundancies permits us to forget about the metrics and to integrate 
over only the different kinds of topologies of two dimensional manifolds, so that in a generic string 
scattering situation, what would be a non-compact generic two dimensional manifold turns into a 
compact two dimensional manifold --- a choice over the equivalence class created by the gauge group: 
sphere, torus, ... ---, and what was the asymptotic states --- the \textit{non-compact part} of the original 
manifold --- turns into \textit{punctures} in the new compact two dimensional manifold. The advantages is, 
this process is nicely described by complex coordinates in the two dimensional (real) manifold, where the 
gauge transformations amounts to holomorphic change of complex coordinates, and the study of such objects, 
complex coordinates in two dimensional (real) manifolds, or better, one dimensional complex manifolds, 
has already lots of years of development in mathematics which we can borrow, these are called Riemann 
Surfaces\footnote{There is actually a distinction of a Riemann Surface and a two dimensional (real) manifold, every 
Riemann Surface is a two dimensional (real) manifold, but the converse is not true.} (RS).

Despite being a astonishing success in some points, BST still fails, at least perturbatively, to give 
any room to accommodate the particle zoo present at our world, principally, there are no means of 
introducing fermions in the target space theory, this, among other reasons, is the motif of pursuing other 
types of theories. A natural guess to overcome the fermion problem is to introduce world-sheet fermions $\psi^\mu$\cite{polchinski:vol2,witten:vol1}\footnote{We'll ignore multiplicative 
factors and set $\alpha'=2$ which can be restored by dimensional analysis.}, 
\begin{align}
    S\sim\int\limits_M\dd[2]{z}\qty(\partial X^\mu\bar\partial X_\mu+\psi^\mu\bar\partial\psi_{\mu}+{\tilde\psi}^\mu\partial{\tilde\psi}_{\mu}+\textnormal{ghosts})\label{action:sst1}
\end{align}
which under quantization gives an analogous problem with the one present in BST\footnote{We're using the graded commutator notation.},
\begin{align*}
    \comm{X^\mu\qty(\tau,\sigma)}{\dot X^\nu\qty(\tau,\sigma')}&=\im\pi\eta^{\mu\nu}\delta\qty(\sigma-\sigma')\\
    \comm{\psi^\mu\qty(\tau,\sigma)}{\psi^\nu\qty(\tau,\sigma')}&=\comm{{\tilde\psi}^\mu\qty(\tau,\sigma)}{{\tilde\psi}^\nu\qty(\tau,\sigma')}=\pi\eta^{\mu\nu}\delta\qty(\sigma-\sigma')
\end{align*}
that is, time-like fields $X^0,\psi^0,{\tilde\psi}^0$ have wrong sign commutator, which implies they will create ghost states in 
the theory, the resolution in BST is to use the gauge group --- a.k.a. the Virasoro constrains ---, to remove 
these non-physical states, but here, the best we could do is to use again the Virasoro constrains to get rid of the 
bosonic wrong sign states, and we would still had the fermionic wrong sign states. Here the only possible resolution 
is to find an other gauge redundancy of this theory, such that we can use it to eliminate the non-physical states. 
Luckily, this new action provides a possible candidate of gauge redundancy, as it has a $\mathcal N=1$ global supersymmetry (SUSY),\begin{subequations}
\begin{align}
    \delta_\epsilon X^\mu&=-\epsilon\psi^\mu-\epsilon^\ast{\tilde\psi^\mu}\\
    \delta_\epsilon\psi^\mu&=\epsilon\partial X^\mu,\ \ \ \delta_\epsilon{\tilde\psi}^\mu=\epsilon^\ast\bar\partial X^\mu
\end{align}\label{susy:pol1}\end{subequations}
Sadly enough, this supersymmetry algebra only closes on-shell and is global instead of local, despite this, 
one by one these issues can be unveiled. The uplift from a global symmetry to a local redundancy can be 
done by means of introducing a new field in the action, the world-sheet gravitino, and the promotion of the 
algebra closing off-shell can also be addressed by the inclusion of an auxiliary field in the action. Both 
these constructions are essential to Superstring Theory (SST), and the resulting theory enjoys a superconformal gauge 
group, which is given by our familiar algebra\footnote{With the inclusion of ghosts.},
\begin{align*}
    T\qty(z)T\qty(w)&\sim\frac{2T\qty(w)}{\qty(z-w)^2}+\frac{\partial T\qty(w)}{z-w},\ \ \ T\qty(z)\sim\partial X\partial X+\cdots\\
    T\qty(z)G\qty(w)&\sim\frac32\frac{G\qty(w)}{\qty(z-w)^2}+\frac{\partial G\qty(w)}{z-w},\ \ \ G\qty(z)\sim\psi\partial X+\cdots\\
    G\qty(z)G\qty(w)&\sim\frac{2T\qty(w)}{\qty(z-w)^2}
\end{align*}
The downside here is: in going from a conformal theory --- which we could benefit from developments in RS ---, 
to a superconformal theory, there seems to be a loss of geometrical visualization --- as due to $G\qty(z)$ 
being fermionic is not clear how it's action on the coordinates $z$ should be interpreted --- that could affect our, before 
mentioned, \textit{ease} of computing scattering amplitudes. To maintain the geometric interpretation and 
the off-shell supersymmetry is the role of the Super Riemann Surfaces (SRS).