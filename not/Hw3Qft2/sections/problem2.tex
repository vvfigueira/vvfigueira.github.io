\section{Beta Functions of Non-Abelian Gauge-Scalar Theory}

\subsection{Construction of the Theory}

We had a lot to say about Gauge theory coupled with fermions, to our luck, many of the things 
said there can be reused here, because, the Gauge part of the theory stays the same, there are 
no changes in it, the changes happen in the matter content, and in the covariant derivatives. 
All the motivation is the same, we start with a given group `$G$', which associated algebra is 
a direct sum over commuting compact simple and `$\mathfrak{u}\qty(1)$' Lie Algebras, this group 
is supposedly a global symmetry of the Lagrangian, that means, our complex scalar field 
`$\boldsymbol \Phi\qty(x)$' transform under some representation of the group, so, the statement 
of the Lagrangian being invariant through the group action is summarized in,

\begin{align}
    \boldsymbol\Phi\qty(x)\rightarrow \exp\qty(-\im g\alpha_a\vb T^a_{\textnormal{s}})\boldsymbol\Phi\qty(x)\nonumber
\end{align}

Leaving the Lagrangian invariant. Here we assume the complex scalar field transforms under a generic 
representation of the group. We would like to promote such a global transformation into a 
local one, 

\begin{align}
    \boldsymbol\Phi\qty(x)\rightarrow \exp\qty(-\im g\alpha_a\qty(x)\vb T^a_{\textnormal{s}})\boldsymbol\Phi\qty(x)\nonumber
\end{align}

Which, as discussed before, will require a notion of covariant derivative, that require Gauge fields,

\begin{align}
    \vb D_{\mu }&=\mathbbm 1\partial_\mu-\im g \vb A_\mu\nonumber
\end{align}

And clearly our kinetic term will be,

\begin{align}
    \mathcal L \supset -\qty[\vb D_\mu\boldsymbol\Phi]^\dagger\vb D^\mu\boldsymbol\Phi\nonumber
\end{align}

To get our full Lagrangian we need to include also all terms compatible with the required symmetries, 
thus, there are still missing terms apart from the kinetic terms of both the Gauge and the complex scalar 
fields, those missing terms are the beloved quadratic and quartic couplings of the scalar. Thus,
our full Lagrangian is,

\begin{align}
    \mathcal L&=-\frac12\Tr\qty[\vb F_{\mu\nu}\vb F^{\mu\nu}]-\qty[\vb D_\mu\boldsymbol\Phi]^\dagger
    \vb D^\mu\boldsymbol\Phi-m^2\boldsymbol\Phi^\dagger\boldsymbol\Phi-\frac{\lambda}{4}\qty[\boldsymbol\Phi^\dagger\boldsymbol\Phi]^2\nonumber
\end{align}

We now should under go the procedure of gauge fixing to properly quantize the theory, but, luckily, 
we have already done this, and more than that, we have already seen that this procedure does not depend 
upon the matter content of the theory. Thus, in the Gauge field part of the theory we must add the 
Gauge fixing terms and also the ghosts terms. Rewriting here these terms to remind,

\begin{align}
    {\mathcal L}_{\textnormal{YM}}+{\mathcal L}_{\textnormal{gf}}+{\mathcal L}_{\textnormal{gh}}&=\frac12A^{a\mu}\delta_{ab}\qty(g_{\mu\nu}\partial^2 -\partial_\mu\partial_\nu )A^{b\nu}+\frac{1}{2\xi} A_{a\mu}\partial^\mu\partial^\nu \tensor{A}{^a_\nu}\nonumber\\
    &\quad\quad\quad-g\tensor{f}{^a^b_c}A_{a\mu}A_{b\nu}\partial^\mu A^{c\nu}-\frac{g^2}{4}\tensor{f}{^a^b_e}\tensor{f}{_c_d^e}A_{a\mu}A_{b\nu}A^{c\mu}A^{d\nu}\nonumber\\
    &\quad\quad\quad -\partial^\mu\bar c^a\partial^\mu c_a+g A_{c\mu}\tensor{f}{^a^b^c}\partial^\mu \bar c_ac_b\nonumber
\end{align}

So that our full theory generating functional is given by this part and the scalar one,

\begin{align}
    Z&=\int\Dd{A}\Dd{\Phi^\dagger}\Dd{\Phi}\exp\qty{\im S_{\textnormal{YM}}+\im S_{\textnormal{gf}}+\im S_{\textnormal{gh}}+\im S_{\textnormal{s}}}\nonumber
\end{align}

\subsection{Renormalization and Feynman Rules}

We have already open up all terms in the Gauge+Ghost Lagrangian in the anterior problem,
what remains to do now is open the scalar terms in this Lagrangian. We start by the kinetic 
term,

\begin{align}
    -\qty[\vb D_\mu\boldsymbol\Phi]^\dagger\vb D^\mu\boldsymbol\Phi&=
    -\qty[\partial_\mu\mathbbm 1\boldsymbol\Phi-\im g\vb A_\mu\boldsymbol\Phi]^\dagger
    \qty[\partial^\mu\mathbbm 1\boldsymbol\Phi-\im g\vb A^\mu\boldsymbol\Phi]\nonumber\\
    &=-\delta_{ij}\qty[\partial_\mu\Phi^i-\im gA_{a\mu}\tensor{\qty[\vb T^a_{\textnormal s}]}{^i_l}\Phi^l]^\dagger
    \qty[\partial^\mu\Phi^j-\im g A_b^{\mu}\tensor{\qty[\vb T^b_{\textnormal s}]}{^j_k}\Phi^k]\nonumber\\
    &=-\delta_{ij}\qty[\partial_\mu\Phi^{i\dagger}+\im gA_{a\mu}\tensor{\qty[\vb T^{a\ast}_{\textnormal s}]}{^i_l}\Phi^{l\dagger}]
    \qty[\partial^\mu\Phi^j-\im g A_b^{\mu}\tensor{\qty[\vb T^b_{\textnormal s}]}{^j_k}\Phi^k]\nonumber\\
    &=-\delta_{ij}\qty[\partial_\mu\Phi^{i\dagger}+\im gA_{a\mu}\tensor{\qty[\vb T^{a}_{\textnormal s}]}{_l^i}\Phi^{l\dagger}]
    \qty[\partial^\mu\Phi^j-\im g A_b^{\mu}\tensor{\qty[\vb T^b_{\textnormal s}]}{^j_k}\Phi^k]\nonumber\\
    &=-\delta_{ij}\partial_\mu\Phi^{i\dagger}\partial^\mu\Phi^j
    +\im g \delta_{ij}\partial_\mu\Phi^{i\dagger}A_b^{\mu}\tensor{\qty[\vb T^b_{\textnormal s}]}{^j_k}\Phi^k
    -\im g\delta_{ij}A_{a\mu}\tensor{\qty[\vb T^{a}_{\textnormal s}]}{_l^i}\Phi^{l\dagger}\partial^\mu\Phi^j\nonumber\\
    &\quad\quad\quad-\delta_{ij}g^2A_{a\mu}\tensor{\qty[\vb T^{a}_{\textnormal s}]}{_l^i}\Phi^{l\dagger}A_b^{\mu}\tensor{\qty[\vb T^b_{\textnormal s}]}{^j_k}\Phi^k
    \nonumber\\
    &=-\delta_{ij}\partial_\mu\Phi^{i\dagger}\partial^\mu\Phi^j\nonumber\\
    &\quad\quad\quad
    -\im gA_{a\mu}\tensor{\qty[\vb T^{a}_{\textnormal s}]}{_i_j}\Phi^{i\dagger}\partial^\mu\Phi^j
    +\im gA_{a\mu}\tensor{\qty[\vb T^a_{\textnormal s}]}{_i_j}\partial^\mu\Phi^{i\dagger}\Phi^j\nonumber\\
    &\quad\quad\quad-\delta_{ij}g^2A_{a\mu}A_b^\mu\tensor{\qty[\vb T^{a}_{\textnormal s}]}{_l^i}\tensor{\qty[\vb T^b_{\textnormal s}]}{^j_k}\Phi^{l\dagger}\Phi^k
    \nonumber\\
    &=-\delta_{ij}\partial_\mu\Phi^{i\dagger}\partial^\mu\Phi^j\nonumber\\
    &\quad\quad\quad
    -\im gA_{a\mu}\tensor{\qty[\vb T^{a}_{\textnormal s}]}{_i_j}\qty(\Phi^{i\dagger}\partial^\mu\Phi^j-\partial^\mu\Phi^{i\dagger}\Phi^j)\nonumber\\
    &\quad\quad\quad-g^2A_{a\mu}A_b^\mu
    \tensor{\qty[\vb T^{a}_{\textnormal s}\vb T^b_{\textnormal s}]}{_i_j}
    \Phi^{i\dagger}\Phi^j\nonumber
\end{align}

The quadratic term is trivial,

\begin{align}
    -m^2\boldsymbol\Phi^\dagger\boldsymbol\Phi&=-m^2\delta_{ij}\Phi^{i\dagger}\Phi^j\nonumber
\end{align}

And finally the quartic one,

\begin{align}
    -\frac{\lambda}{4}\qty[\boldsymbol\Phi^\dagger\boldsymbol\Phi]^2&=-\frac\lambda4\qty[\delta_{ij}\Phi^{i\dagger}\Phi^{j}]^2\nonumber\\
    &=-\frac\lambda4\delta_{ij}\delta_{kl}\Phi^{i\dagger}\Phi^{j}\Phi^{k\dagger}\Phi^{l}\nonumber
\end{align}

Putting all together we have the full Lagrangian as,

\begin{align}
    \mathcal L&=\frac12A^{a\mu}\delta_{ab}\qty(g_{\mu\nu}\partial^2 -\partial_\mu\partial_\nu )A^{b\nu}+\frac{1}{2\xi} A_{a\mu}\partial^\mu\partial^\nu \tensor{A}{^a_\nu}\nonumber\\
    &\quad\quad\quad-g\tensor{f}{^a^b_c}A_{a\mu}A_{b\nu}\partial^\mu A^{c\nu}-\frac{g^2}{4}\tensor{f}{^a^b_e}\tensor{f}{_c_d^e}A_{a\mu}A_{b\nu}A^{c\mu}A^{d\nu}\nonumber\\
    &\quad\quad\quad -\partial_\mu\bar c^a\partial^\mu c_a+g A_{c\mu}\tensor{f}{^a^b^c}\partial^\mu \bar c_ac_b\nonumber\\
    &\quad\quad\quad-\delta_{ij}\partial_\mu\Phi^{i\dagger}\partial^\mu\Phi^j-m^2\delta_{ij}\Phi^{i\dagger}\Phi^j
    -\frac\lambda4\delta_{ij}\delta_{kl}\Phi^{i\dagger}\Phi^{j}\Phi^{k\dagger}\Phi^{l}\nonumber\\
    &\quad\quad\quad
    -\im gA_{a\mu}\tensor{\qty[\vb T^{a}_{\textnormal s}]}{_i_j}\qty(\Phi^{i\dagger}\partial^\mu\Phi^j-\partial^\mu\Phi^{i\dagger}\Phi^j)
    -g^2A_{a\mu}A_b^\mu
    \tensor{\qty[\vb T^{a}_{\textnormal s}\vb T^b_{\textnormal s}]}{_i_j}
    \Phi^{i\dagger}\Phi^j\nonumber
\end{align}

This should really be the bare fields and parameters,

\begin{align}
    \mathcal L&=\frac12{A_0}^{a\mu}\delta_{ab}\qty(g_{\mu\nu}\partial^2 -\partial_\mu\partial_\nu ){A_0}^{b\nu}+\frac{1}{2\xi} {A_0}_{a\mu}\partial^\mu\partial^\nu \tensor{{A_0}}{^a_\nu}\nonumber\\
    &\quad\quad\quad-g\tensor{f}{^a^b_c}{A_0}_{a\mu}{A_0}_{b\nu}\partial^\mu {A_0}^{c\nu}-\frac{g^2}{4}\tensor{f}{^a^b_e}\tensor{f}{_c_d^e}{A_0}_{a\mu}{A_0}_{b\nu}{A_0}^{c\mu}{A_0}^{d\nu}\nonumber\\
    &\quad\quad\quad -\partial_\mu\bar {c_0}^a\partial^\mu {c_0}_a+g {A_0}_{c\mu}\tensor{f}{^a^b^c}\partial^\mu \bar {c_0}_a{c_0}_b\nonumber\\
    &\quad\quad\quad-\delta_{ij}\partial_\mu{\Phi_0}^{i\dagger}\partial^\mu{\Phi_0}^j-m^2\delta_{ij}{\Phi_0}^{i\dagger}{\Phi_0}^j
    -\frac\lambda4\delta_{ij}\delta_{kl}{\Phi_0}^{i\dagger}{\Phi_0}^{j}{\Phi_0}^{k\dagger}{\Phi_0}^{l}\nonumber\\
    &\quad\quad\quad
    -\im g{A_0}_{a\mu}\tensor{\qty[\vb T^{a}_{\textnormal s}]}{_i_j}\qty({\Phi_0}^{i\dagger}\partial^\mu{\Phi_0}^j-\partial^\mu{\Phi_0}^{i\dagger}{\Phi_0}^j)
    -g^2{A_0}_{a\mu}{A_0}_b^\mu
    \tensor{\qty[\vb T^{a}_{\textnormal s}\vb T^b_{\textnormal s}]}{_i_j}
    {\Phi_0}^{i\dagger}{\Phi_0}^j\nonumber
\end{align}

So now we pass to the renormalized fields and parameters,

\begin{align}
    \mathcal L&=\frac12Z_AA^{a\mu}\delta_{ab}\qty(g_{\mu\nu}\partial^2 -\partial_\mu\partial_\nu )A^{b\nu}+\frac{1}{2\xi} A_{a\mu}\partial^\mu\partial^\nu \tensor{A}{^a_\nu}\nonumber\\
    &\quad\quad\quad-gZ_{3g}\tensor{f}{^a^b_c}A_{a\mu}A_{b\nu}\partial^\mu A^{c\nu}-\frac{g^2}{4}Z_{4g}\tensor{f}{^a^b_e}\tensor{f}{_c_d^e}A_{a\mu}A_{b\nu}A^{c\mu}A^{d\nu}\nonumber\\
    &\quad\quad\quad -Z_{c}\partial_\mu\bar c^a\partial^\mu c_a+g Z_{gc}A_{c\mu}\tensor{f}{^a^b^c}\partial^\mu \bar c_ac_b\nonumber\\
    &\quad\quad\quad-\delta_{ij}Z_\Phi\partial_\mu\Phi^{i\dagger}\partial^\mu\Phi^j-m^2Z_m\delta_{ij}\Phi^{i\dagger}\Phi^j
    -\frac\lambda4Z_\lambda\delta_{ij}\delta_{kl}\Phi^{i\dagger}\Phi^{j}\Phi^{k\dagger}\Phi^{l}\nonumber\\
    &\quad\quad\quad
    -\im gZ_{\Phi 1g}A_{a\mu}\tensor{\qty[\vb T^{a}_{\textnormal s}]}{_i_j}\qty(\Phi^{i\dagger}\partial^\mu\Phi^j-\partial^\mu\Phi^{i\dagger}\Phi^j)
    -g^2Z_{\Phi 2g}A_{a\mu}A_b^\mu
    \tensor{\qty[\vb T^{a}_{\textnormal s}\vb T^b_{\textnormal s}]}{_i_j}
    \Phi^{i\dagger}\Phi^j\nonumber
\end{align}

Where we used that `$Z_\xi=1$', that is, the Gauge fixing term do not get corrections, a result we showed 
before to 1-loop, but it's actually true to any loop. Just rearranging the terms and already setting `$\xi=1$',

\begin{align}
    \mathcal L&=\frac12\delta_{ab}g_{\mu\nu}A^{a\mu}\partial^2A^{b\nu}+\delta_{ij}\Phi^{i\dagger}\qty(\partial^2-m^2)\Phi^j-\partial_\mu\bar c^a\partial^\mu c_a\nonumber\\
    &\quad\quad\quad-gZ_{3g}\tensor{f}{^a^b_c}A_{a\mu}A_{b\nu}\partial^\mu A^{c\nu}-\frac{g^2}{4}Z_{4g}\tensor{f}{^a^b_e}\tensor{f}{_c_d^e}A_{a\mu}A_{b\nu}A^{c\mu}A^{d\nu}\nonumber\\
    &\quad\quad\quad +g Z_{gc}A_{c\mu}\tensor{f}{^a^b^c}\partial^\mu \bar c_ac_b\nonumber\\
    &\quad\quad\quad-\frac\lambda4Z_\lambda\delta_{ij}\delta_{kl}\Phi^{i\dagger}\Phi^{j}\Phi^{k\dagger}\Phi^{l}\nonumber\\
    &\quad\quad\quad
    -\im gZ_{\Phi 1g}A_{a\mu}\tensor{\qty[\vb T^{a}_{\textnormal s}]}{_i_j}\qty(\Phi^{i\dagger}\partial^\mu\Phi^j-\partial^\mu\Phi^{i\dagger}\Phi^j)
    -g^2Z_{\Phi 2g}A_{a\mu}A_b^\mu
    \tensor{\qty[\vb T^{a}_{\textnormal s}\vb T^b_{\textnormal s}]}{_i_j}
    \Phi^{i\dagger}\Phi^j\nonumber\\
    &\quad\quad\quad+\frac12\qty(Z_A-1)A^{a\mu}\delta_{ab}g_{\mu\nu}\partial^2A^{b\nu}+\delta_{ij}\qty(Z_\Phi-1)\Phi^{i\dagger}\partial^2\Phi^j-m^2\qty(Z_m-1)\delta_{ij}\Phi^{i\dagger}\Phi^j\nonumber\\
    &\quad\quad\quad-\qty(Z_c-1)\partial_\mu \bar c^a\delta_{ab}\partial^\mu c^b\nonumber
\end{align}

This isn't yet the final version, because, as we are going to use dimensional regularization, 
all couplings dimensions need to be reset with a choice of particular mass scale `$\Lambda$', 
in `$D$'-dimensional space-time we have,

\begin{align}
    \qty[A]=\frac{D-2}{2},\ \ \ \qty[\Phi]=\frac{D-2}{2},\ \ \ \qty[c]=\frac{D-2}{2}\nonumber
\end{align}

Setting `$4-D=2\epsilon$',

\begin{align}
    \qty[gAA\partial A]&=\frac32\qty(D-2)+1+\qty[g]=D\nonumber\\
    \qty[g]&=-\frac12 \qty(D-4)=\epsilon
\end{align}

\begin{align}
    \qty[g^2AAAA]&=\frac42\qty(D-2)+2\qty[g]=D\nonumber\\
    \qty[g]&=\frac12\qty(-D+4)=\epsilon
\end{align}

\begin{align}
    \qty[gA\partial\bar cc]&=\frac32\qty(D-2)+1+\qty[g]=D\nonumber\\
    \qty[g]&=-\frac12\qty(D-4)=\epsilon
\end{align}

\begin{align}
    \qty[gA\Phi^\dagger\partial\Phi]&=\frac12\qty(D-2)+\qty(D-2)+1+\qty[g]=D\nonumber\\
    \qty[g]&=-\frac12\qty(D-4)=\epsilon
\end{align}

\begin{align}
    \qty[g^2AA\Phi^\dagger\Phi]&=\qty(D-2)+\qty(D-2)+2\qty[g]=D\nonumber\\
    \qty[g]&=\frac12\qty(4-D)=\epsilon
\end{align}

\begin{align}
    \qty[\lambda\Phi^\dagger\Phi\Phi^\dagger\Phi]&=2\qty(D-2)+\qty[\lambda]=D\nonumber\\
    \qty[\lambda]&=4-D=2\epsilon
\end{align}

Applying the changes we get,

\begin{align}
    \mathcal L&=\frac12\delta_{ab}g_{\mu\nu}A^{a\mu}\partial^2A^{b\nu}+\delta_{ij}\Phi^{i\dagger}\qty(\partial^2-m^2)\Phi^j-\partial_\mu\bar c^a\partial^\mu c_a\nonumber\\
    &\quad\quad\quad-gZ_{3g}\Lambda^\epsilon\tensor{f}{^a^b_c}A_{a\mu}A_{b\nu}\partial^\mu A^{c\nu}-\frac{g^2}{4}Z_{4g}\Lambda^{2\epsilon}\tensor{f}{^a^b_e}\tensor{f}{_c_d^e}A_{a\mu}A_{b\nu}A^{c\mu}A^{d\nu}\nonumber\\
    &\quad\quad\quad +g Z_{gc}\Lambda^\epsilon A_{c\mu}\tensor{f}{^a^b^c}\partial^\mu \bar c_ac_b\nonumber\\
    &\quad\quad\quad-\frac\lambda4Z_\lambda\Lambda^{2\epsilon}\delta_{ij}\delta_{kl}\Phi^{i\dagger}\Phi^{j}\Phi^{k\dagger}\Phi^{l}\nonumber\\
    &\quad\quad\quad
    -\im gZ_{\Phi 1g}\Lambda^{\epsilon}A_{a\mu}\tensor{\qty[\vb T^{a}_{\textnormal s}]}{_i_j}\qty(\Phi^{i\dagger}\partial^\mu\Phi^j-\partial^\mu\Phi^{i\dagger}\Phi^j)
    -g^2Z_{\Phi 2g}\Lambda^{2\epsilon}A_{a\mu}A_b^\mu
    \tensor{\qty[\vb T^{a}_{\textnormal s}\vb T^b_{\textnormal s}]}{_i_j}
    \Phi^{i\dagger}\Phi^j\nonumber\\
    &\quad\quad\quad+\frac12\qty(Z_A-1)A^{a\mu}\delta_{ab}g_{\mu\nu}\partial^2A^{b\nu}+\delta_{ij}\qty(Z_\Phi-1)\Phi^{i\dagger}\partial^2\Phi^j-m^2\qty(Z_m-1)\delta_{ij}\Phi^{i\dagger}\Phi^j\nonumber\\
    &\quad\quad\quad-\qty(Z_c-1)\partial_\mu \bar c^a\delta_{ab}\partial^\mu c^b\nonumber
\end{align}

From where we can clearly read the propagators Feynman rules,

\begin{itemize}
    \item Scalar Propagator\ \ \feynmandiagram [baseline=(a.base),horizontal=a to b] {
    a [particle=\(i\)] -- [charged scalar] b [particle=\(j\)],
    }; $= \frac1\im\frac{1}{p^2+m^2}\delta_{ij}$
    \item Gauge Propagator\ \ \feynmandiagram [baseline=(a.base),horizontal=a to b] {
    a [particle=\(\mqty{a\\\mu}\)] -- [photon] b [particle=\(\mqty{b\\\nu}\)],
    }; $= \frac1\im\frac{\delta_{ab}}{p^2}g_{\mu\nu}$
    \item Ghost Propagator\ \ \feynmandiagram [baseline=(a.base),horizontal=a to b] {
    a [particle=\(a\)] -- [ghost, with arrow=0.5] b [particle=\(b\)],
    }; $= \frac1\im\frac{1}{p^2}\delta_{ab}$
    \item Scalar Counter-Term\ \ \feynmandiagram [baseline = (c.base),horizontal=a to b,layered layout] {
    a [particle=\(i\)] -- [charged scalar] b [crossed dot] -- [charged scalar] c [particle=\(j\)],
    }; $=-\im\qty(Z_\Phi-1)\delta_{ij}p^2-\im\qty(Z_{m}-1)\delta_{ij}m$
    \item Gauge Counter-Term\ \ \feynmandiagram [baseline = (c.base),horizontal=a to b,layered layout] {
    a [particle=\(\mqty{a\\\mu}\)]-- [photon] b [crossed dot]-- [photon] c[particle=\(\mqty{b\\\nu}\)],
    }; $=-\im\qty(Z_A-1)\delta_{ab}\qty(g_{\mu\nu}p^2-p_\mu p_\nu)$
    \item Ghost Counter-Term\ \ \feynmandiagram [baseline = (c.base),horizontal=a to b,layered layout] {
    a [particle=\(a\)]-- [ghost, with arrow=0.5] b [crossed dot]-- [ghost, with arrow=0.5] c[particle=\(b\)],
    }; $=-\im\qty(Z_c-1)\delta_{ab}p^2$
\end{itemize}

For the interactions Feynman rules we already got all the Gauge-Gauge and Gauge-Ghost 
rules from the last problem,

\begin{itemize}
    \item Ghost-Gauge Interaction\ \ \feynmandiagram [small,baseline = (b.base),horizontal=a to b] {
    a [particle=\(\mqty{a\\\mu}\)]-- [photon] b ,
    b -- [ghost, with arrow=0.5,momentum=\(p\)] c[particle=\(b\)],
    b -- [ghost, with reversed arrow=0.5] d[particle=\(c\)],
    }; $=gZ_{gc}\Lambda^\epsilon f^{abc}p_\mu$
    \item Gauge Cubic Interaction\ \ \feynmandiagram [small,baseline = (b.base),horizontal=a to b] {
    a [particle=\(\mqty{a\\\mu}\)]-- [photon,reversed momentum'=\(p\)] b ,
    b -- [photon,momentum=\(k\)] c[particle=\(\mqty{c\\\alpha}\)],
    b -- [photon,momentum =\(q\)] d[particle=\(\mqty{b\\\nu}\)],
    }; \begin{align}=gZ_{3g}\Lambda^\epsilon f^{abc}\qty(\qty(q-k)_\mu g_{\nu\alpha}+\qty(k-p)_\nu g_{\alpha\mu}+\qty(p-q)_\alpha g_{\mu\nu})\end{align}
    \item Gauge Quartic Interaction\ \ \feynmandiagram [small,baseline = (b.base), layered layout] {
    a [particle=\(\mqty{a\\\mu}\)]-- [photon] b ,
    e [particle=\(\mqty{d\\\beta}\)]--[photon] b,
    b -- [photon] c[particle=\(\mqty{b\\\nu}\)],
    b -- [photon] d[particle=\(\mqty{c\\\alpha}\)],
    }; \begin{align}=-\im g^2Z_{4g}\Lambda^{2\epsilon}&\left[\tensor{f}{^a^b_e}\tensor{f}{^c^d^e}\qty(g_{\mu\alpha}g_{\nu\beta}-g_{\mu\beta}g_{\nu\alpha})\right.\nonumber\\
    &\quad\quad+\tensor{f}{^a^c_e}\tensor{f}{^d^b^e}\qty(g_{\mu\beta}g_{\alpha\nu}-g_{\mu\nu}g_{\beta\alpha})\nonumber\\
    &\quad\quad\left.+\tensor{f}{^a^d_e}\tensor{f}{^b^c^e}\qty(g_{\mu\nu}g_{\alpha\beta}-g_{\mu\alpha}g_{\beta\nu})\right]\end{align}
\end{itemize}


What remains to be done is the Gauge-Scalar and Scalar-Scalar Feynman rules, 
we start with the Scalar-Scalar one, which requires a simple symmetrization,

\begin{itemize}
    \item Scalar Quartic Interaction\ \ \feynmandiagram [small,baseline = (b.base), layered layout] {
    a [particle=\(j\)] -- [charged scalar] b,
    e [particle=\(i\)]-- [anti charged scalar] b,
    b -- [anti charged scalar] c[particle=\(l\)],
    b -- [charged scalar] d[particle=\(k\)],
    }; $=-\im \frac{\lambda}{2} Z_{\lambda}\Lambda^{2\epsilon}\qty[\delta_{ij}\delta_{kl}+\delta_{il}\delta_{jk}]$
\end{itemize}


Now to the Gauge-Scalar Cubic interaction, the derivative coupling gives a factor of `$\im p^\mu$' for `$\Phi$', 
and a factor of `$-\im p^\mu$' for the `$\Phi^\dagger$', so we get,

\begin{itemize}
    \item Ghost-Scalar Cubic Interaction\ \ \feynmandiagram [small,baseline = (b.base),horizontal=a to b] {
    a [particle=\(j\)]-- [charged scalar, momentum'=\(p\)] b ,
    b -- [charged scalar, momentum=\(p'\)] c[particle=\(i\)],
    b -- [photon] d[particle=\(\mqty{a\\\mu}\)],
    }; $=\im g Z_{\Phi1g}\Lambda^{\epsilon}\qty[\vb T^a_{\textnormal{s}}]_{ij}\qty(p+p')^\mu$
\end{itemize}

The Gauge-Scalar Quartic interaction also needs a symmetrization, and gives,

\begin{itemize}
    \item Scalar Quartic Interaction\ \ \feynmandiagram [small,baseline = (b.base), layered layout] {
    a [particle=\(j\)] -- [charged scalar] b,
    e [particle=\(i\)]-- [anti charged scalar] b,
    b -- [photon] c[particle=\(\mqty{a\\\mu}\)],
    b -- [photon] d[particle=\(\mqty{b\\\nu}\)],
    }; $=-\im g^2Z_{\Phi2g}\Lambda^{2\epsilon}g_{\mu\nu}\qty(\qty[\vb T^a_{\textnormal{s}}\vb T^b_{\textnormal{s}}]_{ij}+\qty[\vb T^a_{\textnormal{s}}\vb T^b_{\textnormal{s}}]_{ij})$
\end{itemize}

This finishes all Feynman rules we'll need.

\subsection{Gauge Boson Self Energy}

Well, we already discussed and computed a lot of things when we computed the Gauge Self Energy in problem 1, 
so we're going to reutilize most of it here, mostly because the self energy differs only in two diagrams,

\begin{align}
    \im\Pi^{ab}_{\mu\nu}\qty(p^2)&=\feynmandiagram [baseline = (b.base),horizontal=a to b,layered layout] {
    a -- [photon] b -- [ loop,photon,min distance =2cm] b--[photon] c,
    }; + \feynmandiagram [baseline = (b.base),horizontal=a to b,layered layout] {
    a -- [photon] b -- [ half left,photon] c --[half left,photon] b,
    c -- [photon] d,
    };+ \feynmandiagram [baseline = (b.base),horizontal=a to b,layered layout] {
    a -- [photon] b -- [ half left,charged scalar] c --[half left,charged scalar] b,
    c -- [photon] d,
    };\nonumber\\
    &\quad+ \feynmandiagram [baseline = (b.base),horizontal=a to b,layered layout] {
    a -- [photon] b -- [ half left,ghost,with arrow=0.5] c --[half left,ghost, with arrow=0.5] b,
    c -- [photon] d,
    };+\feynmandiagram [baseline = (b.base),horizontal=a to b,layered layout] {
    a -- [photon] b [crossed dot] -- [photon] c ,
    };+\feynmandiagram [baseline = (b.base),horizontal=a to b,layered layout] {
    a -- [photon] b -- [ loop,charged scalar,min distance =2cm] b--[photon] c,
    };+\cdots\nonumber
\end{align}

The differences are the Fermion loop which was replaced by the Scalar loop and the new Scalar bubble. We start 
by the Scalar loop,

\begin{align}
    \im {\Pi^{\qty(3)}}^{ab}_{\mu\nu}&=\im gZ_{\Phi1g}\Lambda^\epsilon\qty[\vb T^a_{\textnormal{s}}]_{ij}
    \int\frac{\dd[D]{k}}{\qty(2\pi)^D}\qty(2k-p)_\mu\frac1\im\frac{\delta^{ik}}{k^2+m^2}
    \qty(2k-p)_\nu\frac1\im\frac{\delta^{jl}}{\qty(k-p)^2+m^2}\times\nonumber\\
    &\quad\quad\quad\times\im gZ_{\Phi1g}\Lambda^\epsilon\qty[\vb T^b_{\textnormal{s}}]_{lk}\nonumber\\
    &=g^2Z^2_{\Phi1g}\Lambda^{2\epsilon}\Tr\qty[\vb T^b_{\textnormal{s}}\vb T^a_{\textnormal{s}}]
    \int\frac{\dd[D]{k}}{\qty(2\pi)^D}\frac{\qty(2k-p)_\mu\qty(2k-p)_\nu}{\qty(k^2+m^2)\qty(\qty(k-p)^2+m^2)}\nonumber\\
    &=g^2Z^2_{\Phi1g}\Lambda^{2\epsilon}\delta^{ab}T\qty(\textnormal{S})
    \int\frac{\dd[D]{k}}{\qty(2\pi)^D}\frac{\qty(2k-p)_\mu\qty(2k-p)_\nu}{\qty(k^2+m^2)\qty(\qty(k-p)^2+m^2)}\nonumber
\end{align}

Going for the Feynman parametrization,

\begin{align}
    \im {\Pi^{\qty(3)}}^{ab}_{\mu\nu}&=g^2Z^2_{\Phi1g}\Lambda^{2\epsilon}\delta^{ab}T\qty(\textnormal{S})
    \int\limits_0^1\dd{x}\int\frac{\dd[D]{k}}{\qty(2\pi)^D}\frac{\qty(2k-p)_\mu\qty(2k-p)_\nu}{\qty[x\qty(k^2+m^2)+\qty(1-x)\qty(\qty(k-p)^2+m^2)]^2}\nonumber\\
    &=g^2Z^2_{\Phi1g}\Lambda^{2\epsilon}\delta^{ab}T\qty(\textnormal{S})
    \int\limits_0^1\dd{x}\int\frac{\dd[D]{k}}{\qty(2\pi)^D}\frac{\qty(2k-p)_\mu\qty(2k-p)_\nu}{\qty[\qty(k-p\qty(1-x))^2+p^2x\qty(1-x)+m^2]^2}\nonumber
\end{align}

Switching integration variables to `$q=k-p\qty(1-x)$', and neglecting linear terms in `$q$' in the numerator, which when 
integrated get us zeros,

\begin{align}
    \im {\Pi^{\qty(3)}}^{ab}_{\mu\nu}&=g^2Z^2_{\Phi1g}\Lambda^{2\epsilon}\delta^{ab}T\qty(\textnormal{S})
    \int\limits_0^1\dd{x}\int\frac{\dd[D]{q}}{\qty(2\pi)^D}
    \frac{4q_\mu q_\nu+\qty(1-2x)^2p_\mu p_\nu}{\qty[q^2+p^2x\qty(1-x)+m^2]^2}\nonumber
\end{align}

Now we use our bag of tricks to simplify this expression, starting by the replacement, 
`$q_\mu q_\nu\rightarrow \frac1Dg_{\mu\nu}q^2$', and the other already mentioned property, 
`$q^2\rightarrow \frac{D}{2-D}\qty[p^2x\qty(q-x)+m^2]$' which again we reforce is deduced in the appendix,

\begin{align}
    \im {\Pi^{\qty(3)}}^{ab}_{\mu\nu}&=g^2Z^2_{\Phi1g}\Lambda^{2\epsilon}\delta^{ab}T\qty(\textnormal{S})
    \int\limits_0^1\dd{x}\qty[\frac{4g_{\mu\nu}}{2-D}\qty(p^2x\qty(1-x)+m^2)+\qty(1-2x)^2p_\mu p_\nu]\times\nonumber\\
    &\quad\quad\quad\times\int\frac{\dd[D]{q}}{\qty(2\pi)^D}
    \frac{1}{\qty[q^2+p^2x\qty(1-x)+m^2]^2}\nonumber
\end{align}

The remaining integral is very familiar to us,

\begin{align}
    \int\frac{\dd[D]{q}}{\qty(2\pi)^D}
    \frac{1}{\qty[q^2+p^2x\qty(1-x)+m^2]^2}&=\frac{\im}{\qty(4\pi)^{\frac D2}}
    \Gamma\qty(2-\frac D2)\qty(p^2x\qty(1-x)+m^2)^{\frac D2-2}\nonumber
\end{align}

So,

\begin{align}
    \im {\Pi^{\qty(3)}}^{ab}_{\mu\nu}&=g^2Z^2_{\Phi1g}\Lambda^{2\epsilon}\delta^{ab}T\qty(\textnormal{S})
    \int\limits_0^1\dd{x}\qty[\frac{4g_{\mu\nu}}{2-D}\qty(p^2x\qty(1-x)+m^2)+\qty(1-2x)^2p_\mu p_\nu]\times\nonumber\\
    &\quad\quad\quad\times\frac{\im}{\qty(4\pi)^{\frac D2}}
    \Gamma\qty(2-\frac D2)\qty(p^2x\qty(1-x)+m^2)^{\frac D2-2}\nonumber
\end{align}

Now going back to `$D=4$' except in the Gamma function,

\begin{align}
    \im {\Pi^{\qty(3)}}^{ab}_{\mu\nu}&=\frac{\im g^2}{16\pi^2}Z^2_{\Phi1g}\delta^{ab}T\qty(\textnormal{S})
    \Gamma\qty(\epsilon)\int\limits_0^1\dd{x}\qty[-2g_{\mu\nu}\qty(p^2x\qty(1-x)+m^2)+\qty(1-2x)^2p_\mu p_\nu]\nonumber
\end{align}

What remains to be done is the `$x$' integral, which is trivial,

\begin{align}
    \im {\Pi^{\qty(3)}}^{ab}_{\mu\nu}&=\frac{\im g^2}{16\pi^2}Z^2_{\Phi1g}\delta^{ab}T\qty(\textnormal{S})
    \Gamma\qty(\epsilon)\qty[-2g_{\mu\nu}\qty(\frac16p^2+m^2)+\frac13p_\mu p_\nu]\nonumber\\
    &=\frac{\im g^2}{16\pi^2}Z^2_{\Phi1g}\delta^{ab}T\qty(\textnormal{S})
    \Gamma\qty(\epsilon)\qty[-\frac13\qty(g_{\mu\nu}p^2-p_\mu p_\nu)-2g_{\mu\nu}m^2]\nonumber
\end{align}

Now, the Scalar bubble,

\begin{align}
    \im {\Pi^{\qty(6)}}^{ab}_{\mu\nu}&=-\im g^2Z_{\Phi2g}\Lambda^{2\epsilon}g_{\mu\nu}
    \qty(\qty[\vb T^a_{\textnormal{s}}\vb T^b_{\textnormal{s}}]_{ij}
        +\qty[\vb T^a_{\textnormal{s}}\vb T^b_{\textnormal{s}}]_{ij})\times\nonumber\\
    &\quad\quad\quad\times\int\frac{\dd[D]{k}}{\qty(2\pi)^D}\frac1\im\frac{\delta^{ij}}{k^2+m^2}\nonumber\\
    &=-2g^2Z_{\Phi2g}\Lambda^{2\epsilon}g_{\mu\nu}
    \Tr\qty[\vb T^a_{\textnormal{s}}\vb T^b_{\textnormal{s}}]\int\frac{\dd[D]{k}}{\qty(2\pi)^D}\frac{1}{k^2+m^2}\nonumber
\end{align}

Substituting the Trace factor, and also integrating the momentum 
integral, which is carefully done in the appendix,

\begin{align}
    \im {\Pi^{\qty(6)}}^{ab}_{\mu\nu}&=-2g^2Z_{\Phi2g}\Lambda^{2\epsilon}g_{\mu\nu}
    \delta^{ab}T\qty(\textnormal S)\frac{\im}{\qty(4\pi)^{\frac D2}}\Gamma\qty(1-\frac D2)\qty(m^2)^{\frac D2-1}\nonumber\\
    &=-2\frac{\im g^2}{16\pi^2}g^2Z_{\Phi2g}g_{\mu\nu}
    \delta^{ab}T\qty(\textnormal S)\Gamma\qty(\epsilon-1)m^2\nonumber
\end{align}

Now using the Laurent expansion of Gamma and keeping only the divergent 
terms,

\begin{align}
    \im {\Pi^{\qty(6)}}^{ab}_{\mu\nu}&=m^2\frac{\im g^2}{8\pi^2\epsilon}g^2Z_{\Phi2g}g_{\mu\nu}
    \delta^{ab}T\qty(\textnormal S)\nonumber
\end{align}

So, summing all the contributions for the self-energy, including the ones 
derived in the last problem,

\begin{align}
    \im \Pi^{ab}_{\mu\nu}&=\im {\Pi^{\qty(1)}}^{ab}_{\mu\nu}+\im {\Pi^{\qty(2)}}^{ab}_{\mu\nu}+\im {\Pi^{\qty(3)}}^{ab}_{\mu\nu}+\im {\Pi^{\qty(4)}}^{ab}_{\mu\nu}+\im {\Pi^{\qty(5)}}^{ab}_{\mu\nu}+\im {\Pi^{\qty(6)}}^{ab}_{\mu\nu}\nonumber\\
    &=\frac{\im\delta^{ab}g^2}{32\pi^2\epsilon}Z_{3g}^2T\qty(\textnormal{A})\qty[\frac{19}{6}g_{\mu\nu}p^2-\frac{11}{3}p_\mu p_\nu]+\frac{\im g^2}{16\pi^2\epsilon}Z^2_{\Phi1g}\delta^{ab}T\qty(\textnormal{S})
    \qty[-\frac13\qty(g_{\mu\nu}p^2-p_\mu p_\nu)-2g_{\mu\nu}m^2]\nonumber\\
    &\quad\quad\quad+\frac{\im\delta^{ab}g^2}{32\pi^2\epsilon}Z_{gc}^2T\qty(\textnormal{A})\qty(\frac{1}{6}g_{\mu\nu}p^2+\frac13p_\mu p_\nu)-\im\delta^{ab}\qty(Z_A-1)\qty(g_{\mu\nu}p^2-p_\mu p_\nu)\nonumber\\
    &\quad\quad\quad+m^2\frac{\im g^2}{8\pi^2\epsilon}g^2Z_{\Phi2g}g_{\mu\nu}
    \delta^{ab}T\qty(\textnormal S)\nonumber
\end{align}

Keeping only the leading terms in `$Z$', and also grouping the similar terms,

\begin{align}
    \im \Pi^{ab}_{\mu\nu}&=\frac{\im\delta^{ab}g^2}{32\pi^2\epsilon}T\qty(\textnormal{A})\qty[\frac{19}{6}g_{\mu\nu}p^2-\frac{11}{3}p_\mu p_\nu]+\frac{\im g^2}{16\pi^2\epsilon}\delta^{ab}T\qty(\textnormal{S})
    \qty[-\frac13\qty(g_{\mu\nu}p^2-p_\mu p_\nu)]\nonumber\\
    &\quad\quad\quad+\frac{\im\delta^{ab}g^2}{32\pi^2\epsilon}T\qty(\textnormal{A})\qty(\frac{1}{6}g_{\mu\nu}p^2+\frac13p_\mu p_\nu)-\im\delta^{ab}\qty(Z_A-1)\qty(g_{\mu\nu}p^2-p_\mu p_\nu)\nonumber\\
    &=\frac{\im\delta^{ab}g^2}{32\pi^2\epsilon}T\qty(\textnormal{A})\qty[\frac{20}{6}g_{\mu\nu}p^2-\frac{10}{3}p_\mu p_\nu]-\frac{\im g^2}{16\pi^2\epsilon}\delta^{ab}\frac13T\qty(\textnormal{S})
    \qty[g_{\mu\nu}p^2-p_\mu p_\nu]\nonumber\\
    &\quad\quad\quad-\im\delta^{ab}\qty(Z_A-1)\qty(g_{\mu\nu}p^2-p_\mu p_\nu)\nonumber\\
    &=\frac{\im\delta^{ab}g^2}{16\pi^2\epsilon}\frac{5}{3}T\qty(\textnormal{A})\qty[g_{\mu\nu}p^2-p_\mu p_\nu]-\frac{\im g^2}{16\pi^2\epsilon}\delta^{ab}\frac13T\qty(\textnormal{S})
    \qty[g_{\mu\nu}p^2-p_\mu p_\nu]\nonumber\\
    &\quad\quad\quad-\im\delta^{ab}\qty(Z_A-1)\qty(g_{\mu\nu}p^2-p_\mu p_\nu)\nonumber
\end{align}

As this is only the divergent piece, we must set it to zero,

\begin{align}
    \im\delta^{ab}\qty(Z_A-1)\qty(g_{\mu\nu}p^2-p_\mu p_\nu)&=\frac{\im\delta^{ab}g^2}{16\pi^2\epsilon}\frac13\qty(5T\qty(\textnormal{A})-T\qty(\textnormal{A}))\qty[g_{\mu\nu}p^2-p_\mu p_\nu]\nonumber\\
    Z_A&=1+\frac{g^2}{16\pi^2\epsilon}\frac13\qty(5T\qty(\textnormal{A})-T\qty(\textnormal{A}))\nonumber
\end{align}

Getting us then the Wave-function renormalization factor.

\subsection{Further Counter-Terms}

Now, what we should do for continuing the computation is: Computing 
the Wave-Function renormalization factor for the scalar, and then 
computing the renormalization factor of the Gauge-Scalar interaction. But, 
we have something to our use here, there is a combination of the `$Z$' 
factors which is in fact independent of the matter content. For this we 
have to look at the relation of the bare to renormalized parameters,

\begin{align}
    g_0^2&=\frac{Z_{3g}^2}{Z_A^3}g^2\Lambda^{2\epsilon}
    =\frac{Z_{4g}}{Z_A^4}g^2\Lambda^{2\epsilon}
    =\frac{Z_{\Phi1g}^2}{Z_AZ_{\Phi}^2}g^2\Lambda^{2\epsilon}
    =\frac{Z_{\Phi2g}}{Z_AZ_{\Phi}}g^2\Lambda^{2\epsilon}\nonumber
\end{align}

What is related to the Slavnov-Taylor relations. Instead of computing 
`$Z_{\Phi1g}$' and `$Z_{\Phi}$' explicitly, we're just using these relations 
to get that as,

\begin{align}
    g_0^2&=\frac{Z_{\Phi1g}^2}{Z_AZ_\Phi^2}g^2\Lambda^{2\epsilon}\nonumber
\end{align}

is a bare parameter, it's independent of the matter content, and thus we may 
equal it to the same combinations of counter-terms derived with fermions in 
the first problem. That is,

\begin{align}
    g_0^2&=\frac{Z_{\Phi1g}^2}{Z_AZ_\Phi^2}g^2\Lambda^{2\epsilon}=g^2\frac{Z_{g\Psi}^2}{Z_AZ_\Psi^2}\Lambda^{2\epsilon}\nonumber
\end{align}

And now we just use the already computed factors of `$Z_{g\Psi}$' and `$Z_\Psi$'.

\subsection{Computation of the Beta Function}

With the remarks made in the last sub-section, we summarize the relevant results here,

\begin{align}
    \begin{cases}
        Z_A&=1+\frac{g^2}{16\pi^2\epsilon}\qty[\frac{5}{3}T\qty(\textnormal{A})-\frac13T\qty(\textnormal{S})]\\
        Z_\Psi&=1-\frac{g^2}{16\pi^2\epsilon}C\qty(\textnormal{F})\\
        Z_{g\Psi}&=1-\frac{g^2}{16\pi^2\epsilon}\qty[C\qty(\textnormal F)+T\qty(\textnormal{A})]
    \end{cases}\nonumber
\end{align}

This may seem odd, due to the explicit dependence of both `$Z_\Psi$' and `$Z_{g\Psi}$' 
in the fermion representation factor `$C\qty(\textnormal{F})$' which is clearly 
not present in our theory, but, as we have argued, the combination `$Z_{g\Psi}^2Z_{\Psi}^{-2}$' 
is matter content independent, and hence, shouldn't depend on `$C\qty(\textnormal{F})$'. Let's 
show this,

\begin{align}
    Z_{g\Psi}^{2}Z_\Psi^{-2}&=\qty[1-\frac{g^2}{16\pi^2\epsilon}\qty[C\qty(\textnormal F)+T\qty(\textnormal{A})]]^2
    \qty[1-\frac{g^2}{16\pi^2\epsilon}C\qty(\textnormal{F})]^{-2}\nonumber\\
    &=\qty[1-\frac{g^2}{8\pi^2\epsilon}\qty[C\qty(\textnormal F)+T\qty(\textnormal{A})]]\qty[1+\frac{g^2}{8\pi^2\epsilon}C\qty(\textnormal{F})]\nonumber
\end{align}

Where we proceed as usual keeping only the leading singularities,

\begin{align}
    Z_{g\Psi}^{2}Z_\Psi^{-2}&=1-\frac{g^2}{8\pi^2\epsilon}\qty[C\qty(\textnormal F)+T\qty(\textnormal{A})]+\frac{g^2}{8\pi^2\epsilon}C\qty(\textnormal{F})\nonumber\\
    Z_{g\Psi}^{2}Z_\Psi^{-2}&=1-\frac{g^2}{8\pi^2\epsilon}T\qty(\textnormal{A})\nonumber
\end{align}

As was expected, it doesn't depend on any form on the fermions, and only 
on the Gauge part of the theory. We can without guilt state that,

\begin{align}
    g_0^2&=\frac{Z_{\Phi1g}^2}{Z_AZ_\Phi^2}g^2\Lambda^{2\epsilon}=g^2\frac{Z_{g\Psi}^2}{Z_AZ_\Psi^2}g^2\Lambda^{2\epsilon}\nonumber\\
    g_0^2&=\qty[1+\frac{g^2}{16\pi^2\epsilon}\qty[\frac{5}{3}T\qty(\textnormal{A})-\frac13T\qty(\textnormal{S})]]^{-1}\qty[1-\frac{g^2}{8\pi^2\epsilon}T\qty(\textnormal{A})]g^2\Lambda^{2\epsilon}\nonumber\\
    g_0^2&=\qty[1-\frac{g^2}{16\pi^2\epsilon}\qty[\frac{5}{3}T\qty(\textnormal{A})-\frac13T\qty(\textnormal{S})]]\qty[1-\frac{g^2}{8\pi^2\epsilon}T\qty(\textnormal{A})]g^2\Lambda^{2\epsilon}\nonumber\\
    g_0^2&=g^2\Lambda^{2\epsilon}\qty[1-\frac{g^2}{16\pi^2\epsilon}\qty[\frac{5}{3}T\qty(\textnormal{A})-\frac13T\qty(\textnormal{S})]-\frac{g^2}{8\pi^2\epsilon}T\qty(\textnormal{A})]\nonumber\\
    g_0^2&=g^2\Lambda^{2\epsilon}\qty[1-\frac{g^2}{16\pi^2\epsilon}\qty[\frac{11}{3}T\qty(\textnormal{A})-\frac13T\qty(\textnormal{S})]]\nonumber
\end{align}

So now we just follow the usual procedure, calling `$g^2=\alpha$' just for 
convenience,

\begin{align}
    \Lambda\dv{\alpha_0}{\Lambda}=0&=\qty(\beta+2\epsilon \alpha)\Lambda^{2\epsilon}\qty[1-\frac{\alpha}{16\pi^2\epsilon}\qty[\frac{11}{3}T\qty(\textnormal{A})-\frac13T\qty(\textnormal{S})]]-\alpha\Lambda^{2\epsilon}\frac{\beta}{16\pi^2\epsilon}\qty[\frac{11}{3}T\qty(\textnormal{A})-\frac13T\qty(\textnormal{S})]\nonumber\\
    0&=\qty(\beta+2\epsilon \alpha)\qty[1-\frac{\alpha}{16\pi^2\epsilon}\qty[\frac{11}{3}T\qty(\textnormal{A})-\frac13T\qty(\textnormal{S})]]-\alpha\frac{\beta}{16\pi^2\epsilon}\qty[\frac{11}{3}T\qty(\textnormal{A})-\frac13T\qty(\textnormal{S})]\nonumber\\
    0&=\qty(\beta+\epsilon \alpha)\qty[1-\frac{\alpha}{8\pi^2\epsilon}\qty[\frac{11}{3}T\qty(\textnormal{A})-\frac13T\qty(\textnormal{S})]]+\epsilon\alpha\nonumber\\
    -\epsilon\alpha&=\qty(\beta+\epsilon \alpha)\qty[1-\frac{\alpha}{8\pi^2\epsilon}\qty[\frac{11}{3}T\qty(\textnormal{A})-\frac13T\qty(\textnormal{S})]]\nonumber\\
    \beta+\epsilon \alpha&=-\epsilon\alpha\qty[1+\frac{\alpha}{8\pi^2\epsilon}\qty[\frac{11}{3}T\qty(\textnormal{A})-\frac13T\qty(\textnormal{S})]]\nonumber\\
    \beta&=-2\epsilon\alpha -\frac{\alpha^2}{8\pi^2}\qty[\frac{11}{3}T\qty(\textnormal{A})-\frac13T\qty(\textnormal{S})]\nonumber
\end{align}

Now, changing for `$g$',

\begin{align}
    2g\beta&=-2\epsilon g^2 -\frac{g^4}{8\pi^2}\qty[\frac{11}{3}T\qty(\textnormal{A})-\frac13T\qty(\textnormal{S})]\nonumber\\
    \beta&=-\epsilon g -\frac{g^3}{16\pi^2}\qty[\frac{11}{3}T\qty(\textnormal{A})-\frac13T\qty(\textnormal{S})]\nonumber
\end{align}

And passing now to the `$\epsilon\rightarrow 0$' limit,

\begin{align}
    \beta&= -\frac{g^3}{16\pi^2}\qty[\frac{11}{3}T\qty(\textnormal{A})-\frac13T\qty(\textnormal{S})]\nonumber
\end{align}

As we desired.