\section{Not the Standard Model}

The Lagrangian of our theory is very similar to the one already wrote down in the last problem, 
that is,

\begin{align}
    \mathcal L&=-\qty[\vb D_\mu \boldsymbol\Phi]^\dagger\qty[\vb D^\mu\boldsymbol\Phi]
    -\frac12\Tr\qty[\vb F_{\mu\nu}\vb F^{\mu\nu}]-V\qty(\boldsymbol\Phi^\dagger\boldsymbol\Phi)\nonumber
\end{align}

The kinetic term of the scalar was already open up in the last problem, being,

\begin{align}
    -\qty[\vb D_\mu \boldsymbol\Phi]^\dagger\qty[\vb D^\mu\boldsymbol\Phi]&=-\delta_{ij}\partial_\mu\Phi^{i\dagger}\partial^\mu\Phi^j\nonumber\\
    &\quad\quad\quad
    -\im gA_{a\mu}\tensor{\qty[\vb T^{a}_{\textnormal s}]}{_i_j}\qty(\Phi^{i\dagger}\partial^\mu\Phi^j-\partial^\mu\Phi^{i\dagger}\Phi^j)\nonumber\\
    &\quad\quad\quad-g^2A_{a\mu}A_b^\mu
    \tensor{\qty[\vb T^{a}_{\textnormal s}\vb T^b_{\textnormal s}]}{_i_j}
    \Phi^{i\dagger}\Phi^j\nonumber
\end{align}

Setting to the scalar to transform under the adjoint representation,

\begin{align}
    \tensor{\qty[\vb T^{a}_{\textnormal s}]}{_b_c}&=-\im\tensor{f}{^a_b_c}\nonumber
\end{align}

We have the kinetic term of the scalar as being,

\begin{align}
    -\qty[\vb D_\mu \boldsymbol\Phi]^\dagger\qty[\vb D^\mu\boldsymbol\Phi]&=-\delta_{ab}\partial_\mu\Phi^{a\dagger}\partial^\mu\Phi^b\nonumber\\
    &\quad\quad\quad
    -gA_{a\mu}\tensor{f}{^a_b_c}\qty(\Phi^{b\dagger}\partial^\mu\Phi^c-\partial^\mu\Phi^{b\dagger}\Phi^c)\nonumber\\
    &\quad\quad\quad+g^2A_{a\mu}A_b^\mu
    \tensor{f}{^a_c_e}\tensor{f}{^b^e_d}
    \Phi^{c\dagger}\Phi^d\nonumber
\end{align}

Now we invoke that the algebra is the `$\mathfrak{su}\qty(2)$' algebra, so the structure constants are just the Levi-Civita 
symbol,

\begin{align}
    -\qty[\vb D_\mu \boldsymbol\Phi]^\dagger\qty[\vb D^\mu\boldsymbol\Phi]&=-\delta_{ab}\partial_\mu\Phi^{a\dagger}\partial^\mu\Phi^b\nonumber\\
    &\quad\quad\quad
    -gA_{a\mu}\tensor{\epsilon}{^a_b_c}\qty(\Phi^{b\dagger}\partial^\mu\Phi^c-\partial^\mu\Phi^{b\dagger}\Phi^c)\nonumber\\
    &\quad\quad\quad+g^2A_{a\mu}A_b^\mu
    \tensor{\epsilon}{_e^a_c}\tensor{\epsilon}{^e_d^b}
    \Phi^{c\dagger}\Phi^d\nonumber
\end{align}

The Levi-Civita contraction is well known, being just,

\begin{align}
    \tensor{\epsilon}{_e^a_c}\tensor{\epsilon}{^e_d^b}&=\tensor{\delta}{^a_d}\tensor{\delta}{_c^b}-\tensor{\delta}{^a^b}\tensor{\delta}{_c_d}\nonumber
\end{align}

Thus,

\begin{align}
    -\qty[\vb D_\mu \boldsymbol\Phi]^\dagger\qty[\vb D^\mu\boldsymbol\Phi]&=-\delta_{ab}\partial_\mu\Phi^{a\dagger}\partial^\mu\Phi^b\nonumber\\
    &\quad\quad\quad
    -gA_{a\mu}\tensor{\epsilon}{^a_b_c}\qty(\Phi^{b\dagger}\partial^\mu\Phi^c-\partial^\mu\Phi^{b\dagger}\Phi^c)\nonumber\\
    &\quad\quad\quad+g^2A_{a\mu}A_b^\mu
    \qty[\tensor{\delta}{^a_d}\tensor{\delta}{_c^b}-\tensor{\delta}{^a^b}\tensor{\delta}{_c_d}]
    \Phi^{c\dagger}\Phi^d\nonumber\\
    -\qty[\vb D_\mu \boldsymbol\Phi]^\dagger\qty[\vb D^\mu\boldsymbol\Phi]&=-\delta_{ab}\partial_\mu\Phi^{a\dagger}\partial^\mu\Phi^b\nonumber\\
    &\quad\quad\quad
    -gA_{a\mu}\tensor{\epsilon}{^a_b_c}\qty(\Phi^{b\dagger}\partial^\mu\Phi^c-\partial^\mu\Phi^{b\dagger}\Phi^c)\nonumber\\
    &\quad\quad\quad+g^2A_{a\mu}A_b^\mu
    \qty[\Phi^{b\dagger}\Phi^a-\tensor{\delta}{^a^b}\boldsymbol\Phi^{\dagger}\boldsymbol\Phi]\nonumber
\end{align}

So now we shift the fields by a vector, `$\Phi^a\rightarrow \Phi^a+v^a$',

\begin{align}
    -\qty[\vb D_\mu \boldsymbol\Phi]^\dagger\qty[\vb D^\mu\boldsymbol\Phi]&=-\delta_{ab}\partial_\mu\Phi^{a\dagger}\partial^\mu\Phi^b\nonumber\\
    &\quad\quad\quad
    -gA_{a\mu}\tensor{\epsilon}{^a_b_c}\qty(\qty(\Phi^{b\dagger}+v^{b\dagger})\partial^\mu\Phi^c-\partial^\mu\Phi^{b\dagger}\qty(\Phi^c+v^c))\nonumber\\
    &\quad\quad\quad+g^2A_{a\mu}A_b^\mu
    \qty[\qty(\Phi^{b\dagger}+v^{b\dagger})\qty(\Phi^a+v^a)
    -\tensor{\delta}{^a^b}\qty(\boldsymbol\Phi^{\dagger}\boldsymbol\Phi+\boldsymbol v^\dagger\boldsymbol \Phi
    +\boldsymbol\Phi^\dagger\boldsymbol v+\boldsymbol v^\dagger\boldsymbol v)]\nonumber
\end{align}

Keeping only the terms that depend only upon `$\boldsymbol A_\mu$' and `$\boldsymbol v$',

\begin{align}
    -\qty[\vb D_\mu \boldsymbol\Phi]^\dagger\qty[\vb D^\mu\boldsymbol\Phi]&\supset -\frac12 \qty(-2)g^2A_{a\mu}A_b^\mu
    \qty[v^{b\dagger}v^a
    -\tensor{\delta}{^a^b}\boldsymbol v^\dagger\boldsymbol v]\nonumber
\end{align}

That is, the mass matrix, generated by a vacuum of `$\expval{\boldsymbol\Phi}=\boldsymbol v$', in the Gauge-Bosons, is

\begin{align}
    M^2_{ab}&=-2g^2\qty[v_b^{\dagger}v_a
    -\tensor{\delta}{_a_b}\boldsymbol v^\dagger\boldsymbol v]\nonumber
\end{align}

Of course, in this form, this matrix isn't diagonal, hence is not clear what are the eigenvalues, that is, the masses. 
To this matter, let's compute them,

\begin{align}
    M^2_{ab}w^b&=M^2w_a\nonumber\\
    -2g^2\qty[v_b^{\dagger}v_a
    -\tensor{\delta}{_a_b}\boldsymbol v^\dagger\boldsymbol v]w^b&=M^2w_a\nonumber
\end{align}

Here, and all along this problem, we're assuming `$\boldsymbol v^\dagger \boldsymbol v\neq 0$', otherwise the vacuum is 
trivial, that is, `$\expval{\boldsymbol\Phi}=0$', and none of the problem makes sense. But, what is not certain it's true 
is, `$\boldsymbol v^\dagger \boldsymbol w=0$'. Let us divide in cases,

\begin{itemize}
    \item $\boldsymbol v^\dagger \boldsymbol w=0$
\end{itemize}

\begin{align}
    M^2_{ab}w^b&=M^2w_a\nonumber\\
    2g^2\tensor{\delta}{_a_b}\boldsymbol v^\dagger\boldsymbol v w^b&=M^2w_a\nonumber\\
    2g^2\boldsymbol v^\dagger\boldsymbol v w_a&=M^2w_a\nonumber\\
    2g^2\boldsymbol v^\dagger\boldsymbol v&=M^2\nonumber
\end{align}

The other possible case is,

\begin{itemize}
    \item $\boldsymbol v^\dagger \boldsymbol w\neq0$
\end{itemize}

\begin{align}
    M^2_{ab}w^b&=M^2w_a\nonumber\\
    -2g^2\qty[v_b^{\dagger}v_a
    -\tensor{\delta}{_a_b}\boldsymbol v^\dagger\boldsymbol v]w^b&=M^2w_a\nonumber\\
    -2g^2v^{a\dagger}\qty[v_b^{\dagger}v_a
    -\tensor{\delta}{_a_b}\boldsymbol v^\dagger\boldsymbol v]w^b&=M^2v^{a\dagger}w_a\nonumber\\
    -2g^2\qty[\boldsymbol v^\dagger \boldsymbol w\boldsymbol v^\dagger\boldsymbol v
    -\boldsymbol v^\dagger \boldsymbol w\boldsymbol v^\dagger\boldsymbol v]&=M^2\boldsymbol v^\dagger \boldsymbol w\nonumber\\
    0&=M^2\nonumber
\end{align}

Let us summarize our results,

\begin{align}
    \begin{cases}
        M^2=0&\Leftrightarrow \boldsymbol v^\dagger \boldsymbol w\neq0\\
        M^2=2g^2\boldsymbol v^\dagger\boldsymbol v&\Leftrightarrow\boldsymbol v^\dagger \boldsymbol w=0
    \end{cases}\nonumber
\end{align}

Ok, but this still left us with the question, how many eigenvectors of each eigenvalue we have? To answer this question 
we have to notice a very important fact, the Gauge-Boson mass matrix is Hermitian!

\begin{align}
    M^2_{ab}&=-2g^2\qty[v_b^{\dagger}v_a
    -\tensor{\delta}{_a_b}\boldsymbol v^\dagger\boldsymbol v]\nonumber\\
    M^2_{ba}&=-2g^2\qty[v_a^{\dagger}v_b
    -\tensor{\delta}{_b_a}\boldsymbol v^\dagger\boldsymbol v]\nonumber\\
    {M^2_{ba}}^\ast&=-2g^2\qty[v_av_b^{\dagger}
    -\tensor{\delta}{_b_a}\boldsymbol v^\dagger\boldsymbol v]={M^2}^\dagger_{ab}\nonumber\\
    M^2_{ab}&={M^2}^\dagger_{ab}\nonumber
\end{align}

But this is not enough, as we had already said, `$\boldsymbol v^\dagger \boldsymbol v\neq 0$', otherwise, `$M^2_{ab}$' 
would be a null matrix, which is of course Hermitian, but, to which doesn't hold the property we're going to state now, 
\textit{the eigenvectors of a Hermitian matrix belonging to different eigenvalues are orthogonal to each other, and, the 
eigenvectors constitute a basis of the space}. The last statement is equivalent to say that all eigenvectors are linearly 
independent, and of course, by definition of eigenvector, they must be non-null. Our space is $3$-dimensional, hence, there 
are only $2$ linearly independent vectors satisfying `$\boldsymbol v^\dagger\boldsymbol w=0$' for a given `$\boldsymbol v$',
for `$\boldsymbol v^\dagger\boldsymbol w\neq 0$' there is just $1$, and as this set needs to be linear independent and 
complete we got the right answer,

\begin{align}
    \begin{cases}
        M^2=0&\Leftrightarrow \boldsymbol v^\dagger \boldsymbol w\neq0,\ \ \ \textnormal{$1$ degree of freedom}\\
        M^2=2g^2\boldsymbol v^\dagger\boldsymbol v&\Leftrightarrow\boldsymbol v^\dagger \boldsymbol w=0,\ \ \ \textnormal{$2$ degree of freedom}
    \end{cases}\nonumber
\end{align}

In other words, we broke two generators, and just one of them was left unbroken, so, one of the Gauge-Bosons is left 
still massless, while the other two becomes massive with mass `$M^2=2g^2\boldsymbol v^\dagger\boldsymbol v$'. As shown, 
this result is independent of what `$\boldsymbol v$' is, in particular the direction of it. Of course the modulus is 
directly related to the mass, but, the direction of it is not relevant, as we can always make a rotation of the basis of 
the group generators to brake/unbroken a different one. To the matter of what is the group breaking pattern, we started 
with the group `$SU\qty(2)$' which has three generators, as we broke two of them, we are left with just one generator. 
There is just one such group, continuous, over `$\mathbb C$' and with just one generator, the `$U\qty(1)$' group. Hence 
the group breaking pattern is,

\begin{align}
    SU\qty(2)\rightarrow U\qty(1)\nonumber
\end{align}