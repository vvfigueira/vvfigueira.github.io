\section{Introduction}

We will work most with the scalar proxy given by the lagrangian,
\[\mathcal L =-\frac12\partial_\mu \phi\partial^\mu \phi-\frac{1}{2M ^2}\Box \phi\Box\phi-\frac{\kappa}{2}\Box \phi \phi^2\numberthis\]
The idea here is reintegrate the higher derivative term, in order to obtain a lower derivative term, but in terms of additional 
fields. This is easily done by,
\[\mathcal L =-\frac12\partial_\mu \phi\partial^\mu \phi+\Box \phi\eta+\frac{M^2}{2}\eta^2-\frac{\kappa}{2}\Box \phi \phi^2\numberthis\]
The new lagrangian has mixed propagator terms, to diagonalize it is also easy, we just open in terms of $\phi=h-\eta$,
\begin{align}
    \mathcal L&=-\frac12\partial_\mu h\partial^\mu h +\partial_\mu h\partial^\mu \eta-\frac12\partial_\mu\eta\partial^\mu\eta-\frac\kappa2\Box\qty(h-\eta)\qty(h-\eta)^2+\eta\Box\qty(h-\eta)+\frac{M^2}{2}\eta^2\\
    \mathcal L&=-\frac12\partial_\mu h\partial^\mu h +\partial_\mu h\partial^\mu \eta-\frac12\partial_\mu\eta\partial^\mu\eta-\frac\kappa2\Box\qty(h-\eta)\qty(h-\eta)^2-\partial_\mu\eta\partial^\mu\qty(h-\eta)+\frac{M^2}{2}\eta^2\\
    \mathcal L&=-\frac12\partial_\mu h\partial^\mu h +\frac12\partial_\mu\eta\partial^\mu\eta+\frac{M^2}{2}\eta^2-\frac\kappa2\Box\qty(h-\eta)\qty(h-\eta)^2
\end{align}
The Feynman rules are easily red as,
\begin{itemize}
    \item \ \ \feynmandiagram [horizontal=a to b] {
		a [particle=\(h\)] -- [scalar] b [particle=\(h\)],
		}; $=\frac1\im\frac{1}{p^2}$
    \item \ \ \feynmandiagram [horizontal=a to b] {
		a [particle=\(\eta\)] -- [scalar] b [particle=\(\eta\)],
		}; $=-\frac1\im\frac{1}{p^2+M^2}$
	\item \ \ \feynmandiagram [small,baseline = (b.base),horizontal=a to b] {
		a [particle=\(h_1\)] -- [scalar] b  ,
		b -- [scalar] c [particle=\(h_2\)],
		b -- [scalar] d [particle=\(h_3\)],
		}; $=\im\kappa\qty(p_1^2+p_2^2+p_3^2)$
	\item \ \ \feynmandiagram [small,baseline = (b.base),horizontal=a to b] {
		a [particle=\(h_1\)] -- [scalar] b  ,
		b -- [scalar] c [particle=\(h_2\)],
		b -- [scalar] d [particle=\(\eta_3\)],
		}; $=-\im\kappa\qty(p_1^2+p_2^2+p_3^2)$
	\item \ \ \feynmandiagram [small,baseline = (b.base),horizontal=a to b] {
		a [particle=\(h_1\)] -- [scalar] b  ,
		b -- [scalar] c [particle=\(\eta_2\)],
		b -- [scalar] d [particle=\(\eta_3\)],
		}; $=\im\kappa\qty(p_1^2+p_2^2+p_3^2)$
	\item \ \ \feynmandiagram [small,baseline = (b.base),horizontal=a to b] {
		a [particle=\(\eta_1\)] -- [scalar] b  ,
		b -- [scalar] c [particle=\(\eta_2\)],
		b -- [scalar] d [particle=\(\eta_3\)],
		}; $=-\im\kappa\qty(p_1^2+p_2^2+p_3^2)$
\end{itemize}

Which can also be seen directly from the Feynman rules of the $\phi$ field,

\begin{itemize}
    \item \ \ \feynmandiagram [horizontal=a to b] {
		a [particle=\(\phi\)] -- [scalar] b [particle=\(\phi\)],
		}; $=\frac1\im\frac{1}{p^2+\frac{p^4}{M^2}}$
	\item \ \ \feynmandiagram [small,baseline = (b.base),horizontal=a to b] {
		a [particle=\(\phi_1\)] -- [scalar] b  ,
		b -- [scalar] c [particle=\(\phi_2\)],
		b -- [scalar] d [particle=\(\phi_3\)],
		}; $=\im\kappa\qty(p_1^2+p_2^2+p_3^2)=\im\kappa\qty(p_1+p_2+p_3)^2-2\im\kappa\qty(p_1\cdot p_2+p_2\cdot p_3+p_3\cdot p_1)=-\im\kappa\qty(\langle12 \rangle[12]+\langle23 \rangle[23]+\langle31 \rangle[31])$
\end{itemize}

So that the four point amplitude can be computed by,

\begin{align}
    \feynmandiagram [small,baseline = (b.base),horizontal=b to d] {
		a [particle=\(\phi_1\)] -- [scalar] b  ,
		b -- [scalar] c [particle=\(\phi_2\)],
		b -- [scalar,momentum=\(P\)] d,
        d -- [scalar] e [particle=\(\phi_4\)],
        d -- [scalar] f [particle=\(\phi_3\)]
    }; &=\frac{1}{\im}\qty(-\im\kappa)^2\frac{1}{P^2+\frac{P^4}{M^2}}\qty(\langle12 \rangle[12]+\langle2P \rangle[2P]+\langle P1 \rangle[P1])\qty(\langle34 \rangle[34]-\langle4P \rangle[4P]-\langle P3 \rangle[P3])\\
    &=\frac{1}{\im}\qty(-\im\kappa)^2\frac{1}{P^2+\frac{P^4}{M^2}}\qty(\langle12 \rangle[12]-\langle P2 \rangle[2P]-\langle P1 \rangle[1P])\qty(\langle34 \rangle[34]+\langle P4 \rangle[4P]+\langle P3 \rangle[3P])\\
    &=\frac{1}{\im}\qty(-\im\kappa)^2\frac{1}{P^2+\frac{P^4}{M^2}}\qty(\langle12 \rangle[12]+\bra{P}1+2|P])\qty(\langle34 \rangle[34]-\bra{ P} 3+4|P])\\
    &=\frac{1}{\im}\qty(-\im\kappa)^2\frac{1}{P^2+\frac{P^4}{M^2}}\qty(\langle12 \rangle[12]-\bra{P}P|P])\qty(\langle34 \rangle[34]-\bra{ P} P|P])\\
    &=\frac{1}{\im}\qty(-\im\kappa)^2\frac{1}{P^2+\frac{P^4}{M^2}}\qty(\langle12 \rangle[12]-2P^2)\qty(\langle34 \rangle[34]-2P^2)\\
    &=-\im\frac{\qty(\kappa M)^2}{s\qty(M^2-s)}\qty(\langle12 \rangle[12]+2s)\qty(\langle34 \rangle[34]+2s)
\end{align}

It's trivial to read the $t$ and $u$ channels from this expression,

\begin{align}
    \feynmandiagram [small,baseline = (b.base),vertical=b to d] {
		a [particle=\(\phi_2\)] -- [scalar] b  ,
		b -- [scalar] c [particle=\(\phi_3\)],
		b -- [scalar,momentum=\(P\)] d,
        d -- [scalar] e [particle=\(\phi_1\)],
        d -- [scalar] f [particle=\(\phi_4\)]
    }; &=-\im\frac{\qty(\kappa M)^2}{t\qty(M^2-t)}\qty(\langle23 \rangle[23]+2t)\qty(\langle41 \rangle[41]+2t)\\
	\feynmandiagram [small,baseline = (b.base),vertical=b to d] {
		a [particle=\(\phi_2\)] -- [scalar] b  ,
		b -- [scalar] c [particle=\(\phi_4\)],
		b -- [scalar,momentum=\(P\)] d,
        d -- [scalar] e [particle=\(\phi_1\)],
        d -- [scalar] f [particle=\(\phi_3\)]
    }; &=-\im\frac{\qty(\kappa M)^2}{u\qty(M^2-u)}\qty(\langle24 \rangle[24]+2u)\qty(\langle31 \rangle[31]+2u)
\end{align}

So that the full $4$--point amplitude is,



\begin{align}
	\vcenter{\hbox{\usebox{\phifourpoint}}} &=-\im\frac{\qty(\kappa M)^2}{stu\qty(M^2-s)\qty(M^2-t)\qty(M^2-u)}\qty[\qty(\langle12 \rangle[12]+2s)\qty(\langle34 \rangle[34]+2s)tu\qty(M^2-t)\qty(M^2-u)+\qty(\langle23 \rangle[23]+2t)\qty(\langle41 \rangle[41]+2t)su\qty(M^2-s)\qty(M^2-u)+\qty(\langle24 \rangle[24]+2u)\qty(\langle31 \rangle[31]+2u)st\qty(M^2-s)\qty(M^2-t)]
\end{align}

Let us specialize when $1,2$ are massless and $3,4$ are massive, then,

\begin{align}
	\vcenter{\hbox{\usebox{\phifourpoint}}} &=-\im\frac{\qty(\kappa M)^2}{stu\qty(M^2-s)\qty(M^2-t)\qty(M^2-u)}\qty[s\qty(-s+2M^2+2s)tu\qty(M^2-t)\qty(M^2-u)+\qty(-t+M^2+2t)\qty(-t+M^2+2t)su\qty(M^2-s)\qty(M^2-u)+\qty(-u+M^2+2u)\qty(-u+M^2+2u)st\qty(M^2-s)\qty(M^2-t)]\\
	\vcenter{\hbox{\usebox{\phifourpoint}}} &=-\im\frac{\qty(\kappa M)^2}{stu\qty(M^2-s)\qty(M^2-t)\qty(M^2-u)}\qty[stu\qty(2M^2+s)\qty(M^2-t)\qty(M^2-u)+su\qty(M^2+t)\qty(M^2+t)\qty(M^2-s)\qty(M^2-u)+st\qty(M^2+u)\qty(M^2+u)\qty(M^2-s)\qty(M^2-t)]\\
	\vcenter{\hbox{\usebox{\phifourpoint}}} &=-\im\frac{\qty(\kappa M)^2}{tu\qty(M^2-s)\qty(M^2-t)\qty(M^2-u)}\qty[tu\qty(2M^2+s)\qty(M^2-t)\qty(M^2-u)+u\qty(M^2+t)^2\qty(M^2-s)\qty(M^2-u)+t\qty(M^2+u)^2\qty(M^2-s)\qty(M^2-t)]
\end{align}