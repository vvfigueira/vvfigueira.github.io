\section{Cut solutions}

Of course in each amplitude we have different cut solutions. Now let us solve them,

\subsection{all massless}

The cut condition is,
\[k_1^2=k_2^2=(3-k_1)^2=(3-k_1-k_2)^2=(3+4-k_1-k_2)^2=0\]
The first and third condition enforces $k_1=-| k_1]\langle 3|$. But the fourth and fifth conditions 
enforces $3-k_1-k_2=n$, with $n\cdot 4=0\ \&\ n^2=0$. Lastly, the second condition imposes $(3-k_1-n)^2=-23\cdot n+2k_1\cdot n=0$, that is, \[[3n]\langle n3\rangle=[k_1n]\langle n3\rangle\] which has two solutions, $|n]=|k_1]-|3]\ \&\ |n\rangle=z|4\rangle$ or $|n\rangle=|3\rangle\ \&\ |n]=z|4]$. 
When working with scalar particles it's better to choose the first solution, as this avoids singularities in denominators such as $(k_1\cdot k_2)^{-1}$. Hence, the solution we're going to choose is,
\begin{align*}
    \begin{cases}
        k_1&=-|k_1]\langle 3|\\
        k_2&=-|3]\langle 3|+|k_1]\langle 3|+z\qty(|k_1]-|3])\langle 4|
    \end{cases}
\end{align*}

\subsection{massive legs first topology}

Our approach to massive legs is to shift the solution with massless, in order to obtain a well behaved solution in the 
$m^2\rightarrow0$ limit. For this topology the cut constrains are,
\[l_1^2=l_2^2=(3-l_1)^2=-m^2\ \&\ (3-l_1-l_2)^2=(3+4-l_1-l_2)^2=0\]
The ideia here is to define, $l_i=k_i+\alpha_i q_i$ (no sum), with $q_i^2=0$ and $\alpha_i=-m^2\qty(2k_i\cdot q_i)^{-1}$ , 
then, $q_i,k_i$ are not allowed to have any dependence on $m^2$. The first and second constrains are already satisfied. The third one gives,
\[-23\cdot l_1=0\rightarrow 3\cdot(k_1+\alpha_1 q_1)=0\rightarrow 3\cdot q_1=0\]
As $|q_1\rangle = |3\rangle$ is forbidden, $|q_1]=|3]$. The fourth and fifth constrains imposes,
\begin{align*}
    \begin{cases}
        -n\cdot\qty(\alpha_1 q_1+\alpha_2 q_2)+\alpha_1\alpha_2 q_1\cdot q_2&=0\\
        4\cdot \qty(\alpha_1 q_1+\alpha_2 q_2)&=0
    \end{cases}
\end{align*}
This imposes actually $q_1\cdot q_2=0$, for this to be true we have to options, either $|q_2]=|3]$, or $|q_2\rangle=|q_1\rangle$. If we choose the first, we can shift $k_1$ by $3$ such to 
make $|q_1\rangle=|4\rangle$, this imposes further $|q_2\rangle=|4\rangle$. Hence, a possible solution is,
\begin{align*}
    q_1=q_2=-|3]\langle 4|
\end{align*}

\subsection{massive legs second topology}

The constrains now are slightly different,
\[l_1^2=(3-l_1)^2=0\ \&\ l^2_2=(3-l_1-l_2)^2=(3+4-l_1-l_2)^2=-m^2\]
Which has as solution $q_2=-|4]\langle3|$

\subsection{massive legs third topology}

Now the constrain is difficult to solve,
\[l_2^2=0\ \&\ l^2_1=(3-l_1)^2=(3-l_1-l_2)^2=(3+4-l_1-l_2)^2=-m^2\]
The first, second and third constrains give, $l_2=-|l_2]\langle l_2|,\ l_1=-|l_1]\langle 3|-\alpha|3]\langle l_1|$, which is get by shifting $|l_1]$. 
Now, the fourth constrain gives,
\[l_2\cdot \qty(3-l_1)=0\]
Expanding $|l_2]=x|l_1]+y|3]$,
\begin{align*}
    x[l_13]\langle 3l_2\rangle-y[3l_1]\langle 3l_2\rangle - x\alpha[l_13]\langle l_1l_2\rangle&=0\\
    (x+y)\langle 3l_2\rangle &= x\alpha\langle l_1l_2\rangle
\end{align*}
The fifth constrain gives,
\begin{align*}
    4\cdot \qty(3-l_1-l_2)&=0\\
    [4|(-|3]\langle3|+|l_1]\langle 3|+\alpha|3]\langle l_1|+x|l_1]\langle l_2|+y|3]\langle l_2|)|4\rangle&=0\\
    -[43]\langle34\rangle+[4l_1]\langle 34\rangle+\alpha[43]\langle l_14\rangle+x[4l_1]\langle l_24\rangle+y[43]\langle l_24\rangle&=0\\
    [4l_1](\langle 34\rangle+x\langle l_24\rangle)+[43](y\langle l_24\rangle-\langle34\rangle+\alpha\langle l_14\rangle)&=0
\end{align*}
This fixes, $x\langle l_24\rangle=-\langle34\rangle=-y\langle l_24\rangle-\alpha\langle l_14\rangle$, hence, $|l_2\rangle = -\frac1x|3\rangle+\mu|4\rangle,\ \&\ |l_1\rangle = \frac1\alpha\qty(1+\frac yx)|3\rangle+\nu|4\rangle$ 
Plugging this back on the other constrain gives,
\begin{align*}
    (x+y)\mu\langle34\rangle&=x\alpha\qty(\frac\mu\alpha\qty(1+\frac yx)\langle34\rangle-\frac \nu x\langle43\rangle)\\
    (x+y)\mu\langle34\rangle&=\mu\qty(x+y)\langle34\rangle-\alpha\nu \langle43\rangle
\end{align*}
This forces $\nu=0$, which is not a possible solution. The only way to get around this is to impose $|l_1\rangle=|l_2\rangle$. This imposes $x+y=0$ and 
$|l_2\rangle=z|4\rangle$. But this itself is only a solution if we shift $|l_1]\rightarrow|l_1]+|3]$. That is,
\begin{align*}
    \begin{cases}
        l_1&=-|3]\langle3|-|l_1]\langle3|-\frac{m^2}{[l_13]\langle 43\rangle}|3]\langle 4|\\
        l_2&=-z(|3]-|l_1])\langle4|
    \end{cases}
\end{align*}