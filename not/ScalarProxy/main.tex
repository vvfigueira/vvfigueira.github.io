\documentclass[twoside]{amsart}

\usepackage[english]{babel}
\usepackage{csquotes}
\usepackage[style=numeric-comp, backend=biber]{biblatex}
\usepackage{amsmath}
\usepackage{amssymb}
\usepackage{bbm}
\usepackage{graphics}
\usepackage{mathtools}
\usepackage[hidelinks]{hyperref}
\usepackage{physics}
\usepackage{enumitem}
\usepackage{slashed}
\usepackage{tensor}
\usepackage[lmargin=0.5cm,rmargin=0.5cm, tmargin =1cm,bmargin =1cm]{geometry}
\usepackage{tensor}
\usepackage[english]{cleveref}
\usepackage{stackengine}
\usepackage[compat=1.1.0]{tikz-feynman}
\usepackage{graphicx}
\usepackage{setspace}

% Custom command to put a wedge over the comma
\newcommand{\wedgecomma}{\stackon[1pt]{,}{\smash{\scriptsize$\wedge$}}}
\newcommand{\wedgecomm}[2]{\qty[ #1\ \wedgecomma\ #2 ]}

\AtBeginDocument{\renewcommand*{\hbar}{{\mkern-1mu\mathchar'26\mkern-8mu\textnormal{h}}}}
\AtBeginDocument{\newcommand{\e}{\textnormal{e}}}
\AtBeginDocument{\newcommand{\im}{\textnormal{i}}}
\AtBeginDocument{\newcommand{\luz}{\textnormal{c}}}
\AtBeginDocument{\newcommand{\grav}{\textnormal{G}}}
\AtBeginDocument{\newcommand{\kb}{{\textnormal{k}_{\textnormal{B}}}}}
\newcommand{\Dd}[1]{\mathcal D #1}
\newcommand{\Det}[1]{\textup{Det} #1}
\newcommand{\sgn}[1]{\mbox{sgn}\qty(#1)}
\newcommand{\cqd}{\hfill$\blacksquare$}
\newcommand{\dbar}{\mbox{\dj}}

\numberwithin{equation}{section}

\newtheorem{teo}{Teorema}[section]
\newtheorem{defi}{Definição}[section]
\newtheorem{lem}{Lema}[section]
\newtheorem{hip}{Hipótese}[subsection]

\pagestyle{plain}

%\AddToHook{cmd/section/before}{\clearpage}

\addbibresource{refs.bib}

\newtheorem{teorema}{Teorema}[section]
\newtheorem{definicao}{Definição}[section]
\newtheorem{lema}{Lema}[section]
\newtheorem{hipotese}{Hipótese}[section]
\newtheorem{postulado}{Postulado}[section]

\newcommand{\thistheoremname}{}
\newtheorem*{genericthm}{\thistheoremname}
\newenvironment{namedthm}[1]
  {\renewcommand{\thistheoremname}{#1}
   \begin{genericthm}}
  {\end{genericthm}}


  \newcommand\numberthis{\addtocounter{equation}{1}\tag{\theequation}}
  
\title{
Scalar Proxy
}
\author{
  Vicente V. Figueira
       }
% \date{\today}

\allowdisplaybreaks

\newsavebox{\phifourpoint}
\savebox{\phifourpoint}{%
  \begin{tikzpicture}% [scale=1] ajusta se quiser
    \begin{feynman}
      \vertex (A) at ( 1.2,  1.2) {$\phi_3$};
      \vertex (B) at (-1.2, -1.2) {$\phi_1$};
      \vertex (C) at (-1.2,  1.2) {$\phi_2$};
      \vertex (D) at ( 1.2, -1.2) {$\phi_4$};
      \vertex[blob] (V) at (0,0) {};
      \diagram*{
        (A) -- [scalar] (V),
        (B) -- [scalar] (V),
        (C) -- [scalar]  (V),
        (D) -- [scalar]  (V),
      };
    \end{feynman}
  \end{tikzpicture}%
}

\begin{document}

\maketitle

%\tableofcontents

%%%%%%%%%%%%%%%%%%%%%%%%%%%%%%%%%%%%%%%%%%%%%%%%%%%%%%%%%%%%%

\nocite{*}

\section{Introduction/Motivation}
\label{sec:intro}

The Bosonic String Theory (BST) is known to achieve several desirable properties which up to present date 
haven't been done in usual Quantum Field Theory, the most prominent one is it being a perturbatively 
renormalizable theory which contains in its spectrum a massless spin-2 particle, this 
perturbative computation of amplitudes in BST is almost only possible 
due to the heavy simplifications the 
anomaly free gauge group Diff$\qty(M)\times$Weyl allows\cite{polchinski:vol1}. This means, as in the path integral we're 
integrating over metrics, the gauge redundancies permits us to forget about the metrics and to integrate 
over only the different kinds of topologies of two dimensional manifolds, so that in a generic string 
scattering situation, what would be a non-compact generic two dimensional manifold turns into a 
compact two dimensional manifold --- a choice over the equivalence class created by the gauge group: 
sphere, torus, ... ---, and what was the asymptotic states --- the \textit{non-compact part} of the original 
manifold --- turns into \textit{punctures} in the new compact two dimensional manifold. The advantages is, 
this process is nicely described by complex coordinates in the two dimensional (real) manifold, where the 
gauge transformations amounts to holomorphic change of complex coordinates, and the study of such objects, 
complex coordinates in two dimensional (real) manifolds, or better, one dimensional complex manifolds, 
has already lots of years of development in mathematics which we can borrow, these are called Riemann 
Surfaces\footnote{There is actually a distinction of a Riemann Surface and a two dimensional (real) manifold, every 
Riemann Surface is a two dimensional (real) manifold, but the converse is not true.} (RS).

Despite being a astonishing success in some points, BST still fails, at least perturbatively, to give 
any room to accommodate the particle zoo present at our world, principally, there are no means of 
introducing fermions in the target space theory, this, among other reasons, is the motif of pursuing other 
types of theories. A natural guess to overcome the fermion problem is to introduce world-sheet fermions $\psi^\mu$\cite{polchinski:vol2,witten:vol1}\footnote{We'll ignore multiplicative 
factors and set $\alpha'=2$ which can be restored by dimensional analysis.}, 
\begin{align}
    S\sim\int\limits_M\dd[2]{z}\qty(\partial X^\mu\bar\partial X_\mu+\psi^\mu\bar\partial\psi_{\mu}+{\tilde\psi}^\mu\partial{\tilde\psi}_{\mu}+\textnormal{ghosts})\label{action:sst1}
\end{align}
which under quantization gives an analogous problem with the one present in BST\footnote{We're using the graded commutator notation.},
\begin{align*}
    \comm{X^\mu\qty(\tau,\sigma)}{\dot X^\nu\qty(\tau,\sigma')}&=\im\pi\eta^{\mu\nu}\delta\qty(\sigma-\sigma')\\
    \comm{\psi^\mu\qty(\tau,\sigma)}{\psi^\nu\qty(\tau,\sigma')}&=\comm{{\tilde\psi}^\mu\qty(\tau,\sigma)}{{\tilde\psi}^\nu\qty(\tau,\sigma')}=\pi\eta^{\mu\nu}\delta\qty(\sigma-\sigma')
\end{align*}
that is, time-like fields $X^0,\psi^0,{\tilde\psi}^0$ have wrong sign commutator, which implies they will create ghost states in 
the theory, the resolution in BST is to use the gauge group --- a.k.a. the Virasoro constrains ---, to remove 
these non-physical states, but here, the best we could do is to use again the Virasoro constrains to get rid of the 
bosonic wrong sign states, and we would still had the fermionic wrong sign states. Here the only possible resolution 
is to find an other gauge redundancy of this theory, such that we can use it to eliminate the non-physical states. 
Luckily, this new action provides a possible candidate of gauge redundancy, as it has a $\mathcal N=1$ global supersymmetry (SUSY),\begin{subequations}
\begin{align}
    \delta_\epsilon X^\mu&=-\epsilon\psi^\mu-\epsilon^\ast{\tilde\psi^\mu}\\
    \delta_\epsilon\psi^\mu&=\epsilon\partial X^\mu,\ \ \ \delta_\epsilon{\tilde\psi}^\mu=\epsilon^\ast\bar\partial X^\mu
\end{align}\label{susy:pol1}\end{subequations}
Sadly enough, this supersymmetry algebra only closes on-shell and is global instead of local, despite this, 
one by one these issues can be unveiled. The uplift from a global symmetry to a local redundancy can be 
done by means of introducing a new field in the action, the world-sheet gravitino, and the promotion of the 
algebra closing off-shell can also be addressed by the inclusion of an auxiliary field in the action. Both 
these constructions are essential to Superstring Theory (SST), and the resulting theory enjoys a superconformal gauge 
group, which is given by our familiar super Virasoro algebra\footnote{With the inclusion of ghosts.},
\begin{align*}
    \begin{cases}
        T\qty(z)&\sim\partial X\partial X+\cdots\\
        G\qty(z)&\sim\psi\partial X+\cdots
    \end{cases}\Rightarrow
    \begin{cases}
    T\qty(z)T\qty(w)&\sim\frac{2T\qty(w)}{\qty(z-w)^2}+\frac{\partial T\qty(w)}{z-w}\\
    T\qty(z)G\qty(w)&\sim\frac32\frac{G\qty(w)}{\qty(z-w)^2}+\frac{\partial G\qty(w)}{z-w}\\
    G\qty(z)G\qty(w)&\sim\frac{2T\qty(w)}{\qty(z-w)^2}\end{cases}\Rightarrow\begin{cases}
        \comm{L_m}{L_n}&=\qty(m-n)L_{m+n}\\
        \comm{G_r}{G_s}&=2L_{r+s}\\
        \comm{L_m}{G_r}&=\qty(\frac m2-r)G_{m+r}
    \end{cases}
\end{align*}
The downside here is: in going from a conformal theory --- which we could benefit from developments in RS ---, 
to a superconformal theory, there seems to be a loss of geometrical visualization --- as due to $G\qty(z)$ 
being fermionic is not clear how it's action on the coordinates $z$ should be interpreted --- that could affect our, before 
mentioned, \textit{ease} of computing scattering amplitudes. To maintain the geometric interpretation and 
the off-shell supersymmetry is the role of the Super Riemann Surfaces (SRS).

\section{Conformal Toy Model}

Consider the following Lagrangian, 
\[\mathcal L=-\frac12\Box \phi\Box\phi-\frac{g}{2}\phi^2\Box\phi-\frac{g^2}{8}\phi^4+m^2\qty(-\frac12\partial_\mu\phi\partial^\mu\phi+\frac{g}{3!}\phi^3)\]
Notice the form of the Lagrangian,
\[\mathcal L=-\frac12\qty(\Box\phi+\frac{g}{2}\phi^2)^2+m^2\qty(-\frac12\partial_\mu\phi\partial^\mu\phi+\frac{g}{3!}\phi^3)\]
It possesses the Feynman rules,
\begin{itemize}
    \item \ \ \feynmandiagram [horizontal=a to b] {
		a [particle=\(\phi\)] -- [scalar] b [particle=\(\phi\)],
		}; $=\frac1\im\frac{1}{m^2p^2+p^4}$
	\item \ \ \feynmandiagram [small,baseline = (b.base),horizontal=a to b] {
		a [particle=\(\phi_1\)] -- [scalar] b  ,
		b -- [scalar] c [particle=\(\phi_2\)],
		b -- [scalar] d [particle=\(\phi_3\)],
		}; $=\im g\qty(p_1^2+p_2^2+p_3^2+m^2)$
	\item \ \ \feynmandiagram [small,baseline = (b.base)] {
		a [particle=\(\phi_1\)] -- [scalar] b  ,
		e [particle=\(\phi_1\)] -- [scalar] b  ,
		b -- [scalar] c [particle=\(\phi_2\)],
		b -- [scalar] d [particle=\(\phi_3\)],
		}; $=-3\im g^2$
\end{itemize}

Let's compute the self energy,
\begin{align}
    \im\Pi\qty(p^2)&=\feynmandiagram [baseline = (b.base),horizontal=a to b,layered layout] {
     a -- [scalar] b -- [ loop,scalar,min distance =2cm] b--[scalar] c,
     }; + \feynmandiagram [baseline = (b.base),horizontal=a to b,layered layout] {
     a -- [scalar] b -- [ half left,scalar] c --[half left,scalar] b,
     c -- [scalar] d,
     };+\cdots
\end{align}

\begin{align}
    \im{\Pi^{\qty(1)}}&=-\frac32\im g^2\int\frac{\dd[D]{\ell}}{\qty(2\pi)^D}\frac{1}{\im}\frac{1}{\ell^2}\frac{1}{\ell^2+m^2}\\
    \im{\Pi^{\qty(1)}}&=-\frac32 g^2\int\frac{\dd[D]{\ell}}{\qty(2\pi)^D}\frac{1}{\ell^2}\frac{1}{\ell^2+m^2}\\
    \im{\Pi^{\qty(1)}}&=-\frac32 g^2\frac{\im}{\qty(4\pi)^{\frac D2}\Gamma\qty(\frac D2)}\qty(m^2)^{\frac D2-2}\frac{\Gamma\qty(2-\frac D2)\Gamma\qty(\frac D2-1)}{\Gamma\qty(1)}\\
    \im{\Pi^{\qty(1)}}&=-\frac32\im g^2\frac{\qty(m^2)^{-\epsilon}\Gamma\qty(\epsilon)\Gamma\qty(1-\epsilon)}{\qty(4\pi)^{2-\epsilon}\Gamma\qty(2-\epsilon)}
\end{align}

\begin{align}
    \im{\Pi^{\qty(2)}}&=\frac12\qty(\im g)^2\int\frac{\dd[D]{\ell}}{\qty(2\pi)^D}\frac{1}{\im^2}\frac{1}{\ell^2\qty(\ell+p)^2}\frac{\qty(m^2+\ell^2+p^2+\qty(\ell+p)^2)^2}{\ell^2+m^2}\frac{1}{\qty(\ell+p)^2+m^2}
\end{align}
For the mass renormalization we can take $p=0$,
\begin{align}
    \im{\Pi^{\qty(2)}}&=\frac12g^2\int\frac{\dd[D]{\ell}}{\qty(2\pi)^D}\frac{\qty(m^2+2\ell^2)^2}{\ell^4\qty(\ell^2+m^2)^2}
\end{align}
Let's compute the four point amplitude for this theory,
\begin{align}
    \feynmandiagram [small,baseline = (b.base),horizontal=b to d] {
		a [particle=\(\phi_1\)] -- [scalar] b  ,
		b -- [scalar] c [particle=\(\phi_2\)],
		b -- [scalar,momentum=\(P\)] d,
        d -- [scalar] e [particle=\(\phi_4\)],
        d -- [scalar] f [particle=\(\phi_3\)]
    }; &=\qty(\im g)^2\frac{\qty(p_1^2+p_2^2+\qty(p_1+p_2)^2+m^2)\qty(p_3^2+p_4^2+\qty(p_3+p_4)^2+m^2)}{\im\qty(p_1+p_2)^2\qty(\qty(p_1+p_2)^2+m^2)}
\end{align}
First let's consider all legs massless,
\begin{align}
    \feynmandiagram [small,baseline = (b.base),horizontal=b to d] {
		a [particle=\(\phi_1\)] -- [scalar] b  ,
		b -- [scalar] c [particle=\(\phi_2\)],
		b -- [scalar,momentum=\(P\)] d,
        d -- [scalar] e [particle=\(\phi_4\)],
        d -- [scalar] f [particle=\(\phi_3\)]
    }; &=\im g^2\frac{\qty(-s+m^2)\qty(-s+m^2)}{\qty(-s)\qty(-s+m^2)}=-\im g^2\frac{\qty(-s+m^2)}{s}
\end{align}
So,
\begin{align}
	\vcenter{\hbox{\usebox{\phifourpoint}}} &=-\im g^2\frac{\qty(-s+m^2)}{s}-\im g^2\frac{\qty(-t+m^2)}{t}-\im g^2\frac{\qty(-u+m^2)}{u}-3\im g^2\\
	\vcenter{\hbox{\usebox{\phifourpoint}}} &=-\im g^2\frac{\qty(-s+m^2)}{s}-\im g^2\frac{\qty(-t+m^2)}{t}-\im g^2\frac{\qty(-u+m^2)}{u}-\im g^2\frac{s}{s}-\im g^2\frac{t}{t}-\im g^2\frac{u}{u}\\
	\vcenter{\hbox{\usebox{\phifourpoint}}} &=-\im g^2m^2\qty(\frac1s+\frac1t+\frac1u)=\im g^2 m^2\qty(\frac{1}{\langle12\rangle[12]}+\frac{1}{\langle14\rangle[14]}+\frac{1}{\langle13\rangle[13]})
\end{align}
Para uma perna massiva, $\phi_4$,
\begin{align}
    \feynmandiagram [small,baseline = (b.base),horizontal=b to d] {
		a [particle=\(\phi_1\)] -- [scalar] b  ,
		b -- [scalar] c [particle=\(\phi_2\)],
		b -- [scalar,momentum=\(P\)] d,
        d -- [scalar] e [particle=\(\phi_4\)],
        d -- [scalar] f [particle=\(\phi_3\)]
    }; &=\qty(\im g)^2\frac{\qty(-s+m^2)\qty(-s)}{\im\qty(-s)\qty(-s+m^2)}=\im g^2
\end{align}
So,
\begin{align}
	\vcenter{\hbox{\usebox{\phifourpoint}}} &=\im g^2+\im g^2+\im g^2-3\im g^2=0
\end{align}
Para duas pernas massivas, $\phi_{3,4}$,
\begin{align}
    \feynmandiagram [small,baseline = (b.base),horizontal=b to d] {
		a [particle=\(\phi_1\)] -- [scalar] b  ,
		b -- [scalar] c [particle=\(\phi_2\)],
		b -- [scalar,momentum=\(P\)] d,
        d -- [scalar] e [particle=\(\phi_4\)],
        d -- [scalar] f [particle=\(\phi_3\)]
    }; &=\qty(\im g)^2\frac{\qty(-s+m^2)\qty(-s-m^2)}{\im\qty(-s)\qty(-s+m^2)}=\im g^2\frac{s+m^2}{s}\\
    \feynmandiagram [small,baseline = (b.base),vertical=b to d] {
		a [particle=\(\phi_2\)] -- [scalar] b  ,
		b -- [scalar] c [particle=\(\phi_3\)],
		b -- [scalar,momentum=\(P\)] d,
        d -- [scalar] e [particle=\(\phi_1\)],
        d -- [scalar] f [particle=\(\phi_4\)]
    }; &=\qty(\im g)^2\frac{\qty(-t)\qty(-t)}{\im\qty(-t)\qty(-t+m^2)}=-\im g^2\frac{t}{-t+m^2}\\
    \feynmandiagram [small,baseline = (b.base),vertical=b to d] {
		a [particle=\(\phi_2\)] -- [scalar] b  ,
		b -- [scalar] c [particle=\(\phi_4\)],
		b -- [scalar,momentum=\(P\)] d,
        d -- [scalar] e [particle=\(\phi_1\)],
        d -- [scalar] f [particle=\(\phi_3\)]
    }; &=-\im g^2\frac{u}{-u+m^2}
\end{align}
So,
\begin{align}
	\vcenter{\hbox{\usebox{\phifourpoint}}} &=\im g^2\frac{s+m^2}{s}-\im g^2\frac{t}{-t+m^2}-\im g^2\frac{u}{-u+m^2}-3\im g^2\\
	\vcenter{\hbox{\usebox{\phifourpoint}}} &=\im g^2\frac{s+m^2}{s}-\im g^2\frac{t}{-t+m^2}-\im g^2\frac{u}{-u+m^2}-\im g^2 \frac{s}{s}-\im g^2\frac{-t+m^2}{-t+m^2}-\im g^2\frac{-u+m^2}{-u+m^2}\\
	\vcenter{\hbox{\usebox{\phifourpoint}}} &=-\im g^2 m^2\qty(-\frac{1}{s}+\frac{1}{-t+m^2}+\frac{1}{-u+m^2})=-\im g^2 m^2\qty(\frac{1}{\langle12\rangle[12]}+\frac{1}{\langle1\vb4\rangle[1\vb 4]}+\frac{1}{\langle1\vb3\rangle[1\vb3]})
\end{align}
Para uma perna sem massa $\phi_1$,
\begin{align}
    \feynmandiagram [small,baseline = (b.base),horizontal=b to d] {
		a [particle=\(\phi_1\)] -- [scalar] b  ,
		b -- [scalar] c [particle=\(\phi_2\)],
		b -- [scalar,momentum=\(P\)] d,
        d -- [scalar] e [particle=\(\phi_4\)],
        d -- [scalar] f [particle=\(\phi_3\)]
    }; &=\qty(\im g)^2\frac{\qty(-s)\qty(-s-m^2)}{\im\qty(-s)\qty(-s+m^2)}=-\im g^2\frac{s+m^2}{-s+m^2}\\
    \feynmandiagram [small,baseline = (b.base),vertical=b to d] {
		a [particle=\(\phi_2\)] -- [scalar] b  ,
		b -- [scalar] c [particle=\(\phi_3\)],
		b -- [scalar,momentum=\(P\)] d,
        d -- [scalar] e [particle=\(\phi_1\)],
        d -- [scalar] f [particle=\(\phi_4\)]
    }; &=\qty(\im g)^2\frac{\qty(-t)\qty(-t-m^2)}{\im\qty(-t)\qty(-t+m^2)}=-\im g^2\frac{t+m^2}{-t+m^2}\\
    \feynmandiagram [small,baseline = (b.base),vertical=b to d] {
		a [particle=\(\phi_2\)] -- [scalar] b  ,
		b -- [scalar] c [particle=\(\phi_4\)],
		b -- [scalar,momentum=\(P\)] d,
        d -- [scalar] e [particle=\(\phi_1\)],
        d -- [scalar] f [particle=\(\phi_3\)]
    }; &=\qty(\im g)^2\frac{\qty(-u)\qty(-u-m^2)}{\im\qty(-u)\qty(-u+m^2)}=-\im g^2\frac{u+m^2}{-u+m^2}\\
\end{align}
So,
\begin{align}
	\vcenter{\hbox{\usebox{\phifourpoint}}} &=-\im g^2\frac{s+m^2}{-s+m^2}-\im g^2\frac{t+m^2}{-t+m^2}-\im g^2\frac{u+m^2}{-u+m^2}-3\im g^2\\
	\vcenter{\hbox{\usebox{\phifourpoint}}} &=-\im g^2\frac{s+m^2}{-s+m^2}-\im g^2\frac{t+m^2}{-t+m^2}-\im g^2\frac{u+m^2}{-u+m^2}-\im g^2\frac{-s+m^2}{-s+m^2}-\im g^2\frac{-t+m^2}{-t+m^2}-\im g^2\frac{-u+m^2}{-u+m^2}\\
	\vcenter{\hbox{\usebox{\phifourpoint}}} &=-\im g^2m^2\qty(\frac{1}{-s+m^2}+\frac{1}{-t+m^2}+\frac{1}{-u+m^2})=-\im g^2 m^2\qty(\frac{1}{\langle1\vb 2\rangle[1\vb2]}+\frac{1}{\langle1\vb4\rangle[1\vb 4]}+\frac{1}{\langle1\vb3\rangle[1\vb3]})
\end{align}

Cut comparison, only massless legs

\begin{align}
    \feynmandiagram [baseline = (b.base),horizontal=b to h, layered layout,small] {
        a -- [scalar] b [blob],
        f -- [scalar] b ,
        b -- [ scalar, momentum =\(l+3+4\)] g [blob], 
        g --[scalar,momentum =\(l+4\)] h [blob], 
        h --[scalar, momentum=\(l\)] b,
        g -- [scalar] d,
        h -- [scalar] e,
    };&=\frac{\im g^2m^2\qty(\im g m^2)^2}{\qty(\im m^2)^3}\qty(\frac{1}{\langle12\rangle[12]}-\frac{1}{\langle1l\rangle[1l]}-\frac{1}{\langle2l\rangle[2l]})
\end{align}

to solve for the cuts, $l^2=\qty(l+3+4)^2=\qty(l+4)^2=0$,
\begin{align}
    l^2&=0\Rightarrow l=-|l\rangle [l|\\
    0=\qty(l+4)^2&=\langle l4\rangle[l4]=0\Rightarrow |l]=|4]\\
    0=\qty(l+3+4)^2&=\langle lP_{34}\rangle[lP_{34}]+(3+4)^2=\langle l| 3+4|l]+\langle34\rangle[34]=\langle l| 3+4|4]+\langle34\rangle[34]\\
    \langle43\rangle[34]&=-\langle l3\rangle[34]\Rightarrow |l\rangle = -|4\rangle+z|3\rangle\\
    l&=-\qty(-|4\rangle+z|3\rangle)[4|
\end{align}
The cuts are solved by this. Hence,
\begin{align*}
    &=g^4\qty(\frac{1}{\langle12\rangle[12]}-\frac{1}{\langle1l\rangle[1l]}-\frac{1}{\langle2l\rangle[2l]})\\
    &=g^4\qty(\frac{1}{\langle12\rangle[12]}-\frac{1}{\qty(-\langle14\rangle+z\langle13\rangle)[14]}-\frac{1}{\qty(-\langle24\rangle+z\langle23\rangle)[24]})
\end{align*}
Now for internal massive lines,
\begin{align}
    \feynmandiagram [baseline = (b.base),horizontal=b to h, layered layout,small] {
        a -- [scalar] b [blob],
        f -- [scalar] b ,
        b -- [ scalar, momentum =\(l+3+4\)] g [blob], 
        g --[scalar,momentum =\(l+4\)] h [blob], 
        h --[scalar, momentum=\(l\)] b,
        g -- [scalar] d,
        h -- [scalar] e,
    };&=\frac{-\im g^2m^2\qty(-\im g m^2)^2}{\qty(-\im m^2)^3}\qty(\frac{1}{\langle12\rangle[12]}-\frac{1}{\langle1l\rangle[1l]}-\frac{1}{\langle2l\rangle[2l]})
\end{align}
With the cuts being, $l^2=\qty(l+3+4)^2=\qty(l+4)^2=-m^2$,
\begin{align}
    0=(l+4)^2-l^2&=2l\cdot p_4\\
    0=(l+4+3)^2-l^2&=2l\cdot \qty(4+3)+(4+3)^2=2l\cdot p_3+(4+3)^2
\end{align}
As ansatz, $l=|4\rangle [4|+\alpha|4\rangle [3|+\beta|3\rangle[4|$ satisfy both conditions above. The remaining condition is,
\begin{align}
    l^2&=-m^2\\
    -\alpha\beta[43]\langle43\rangle&=-m^2\Rightarrow \alpha=\frac{m^2}{\beta\langle34\rangle[34]}
\end{align}
Setting now $-\beta=z$,
\begin{align}
    l=|4\rangle [4|-\frac{m^2}{z\langle34\rangle[34]}|4\rangle [3|-z|3\rangle[4|
\end{align}
The value of the diagram is,
\begin{align*}
    &=g^4\qty(\frac{1}{\langle12\rangle[12]}+\frac{1}{\langle1l\rangle[l1]}+\frac{1}{\langle2l\rangle[l2]})\\
    &=g^4\qty(\frac{1}{\langle12\rangle[12]}+\frac{1}{-\langle1|l|1]}+\frac{1}{-\langle2|l|2]})\\
    &=g^4\qty(\frac{1}{\langle12\rangle[12]}-\frac{1}{\langle14\rangle[41]-\frac{m^2}{z\langle34\rangle[34]}\langle14\rangle[31]-z\langle13\rangle[41]}-\frac{1}{\langle2|l|2]})
\end{align*}
The explicit cut loop amplitude is,
\begin{align*}
    \feynmandiagram [baseline = (b.base),horizontal=b to g,small] {
        a -- [scalar] b ,
        f -- [scalar] c ,
        b -- [ scalar, momentum =\(l+3+4\)] g , 
        g --[scalar,momentum =\(l+4\)] h , 
        h --[scalar, momentum=\(l\)] c,
        c --[scalar, momentum=\(l-1\)] b,
        g -- [scalar] e,
        h -- [scalar] d,
    };
\end{align*}
Triple cut has no improvement, what about a double cut,

\begin{align}
    \vcenter{\hbox{\usebox{\phifourloop}}}&=\frac{\qty(\im g^2m^2)^2}{\qty(\im m^2)^2}\qty(\frac{1}{\langle12\rangle[12]}-\frac{1}{\langle1\ell\rangle[1\ell]}-\frac{1}{\langle2\ell\rangle[2\ell]})\qty(\frac{1}{\langle34\rangle[34]}+\frac{1}{\langle3\ell\rangle[3\ell]}+\frac{1}{\langle4\ell\rangle[4\ell]})
\end{align}

Five point amplitude,

\begin{align*}
    \feynmandiagram [small] {
        i1 [particle = \(3\)]-- [scalar] v1,
        i2 [particle=\(4\)]-- [scalar] v1,
        v1 -- [scalar] v2,
        v2 -- [scalar] i5 [particle=\(5\)],
        v2 -- [scalar] v3,
        v3 -- [scalar] i3 [particle=\(1\)],
        v3 -- [scalar] i4 [particle=\(2\)]
    }; &=  \frac{\qty(\im g)^3\qty(m^2+p_1^2+p_2^2+\qty(p_1+p_2)^2)\qty(m^2+p_5^2+\qty(p_3+p_4)^2+\qty(p_1+p_2)^2)\qty(m^2+p_3^2+p_4^2+\qty(p_3+p_4)^2)}{i^2\qty(p_1+p_2)^2\qty(\qty(p_1+p_2)^2+m^2)\qty(p_3+p_4)^2\qty(\qty(p_3+p_4)^2+m^2)}
\end{align*}

Let's consider the special case of all massless,

\begin{align*}
    \feynmandiagram [small] {
        i1 [particle = \(3\)]-- [scalar] v1,
        i2 [particle=\(4\)]-- [scalar] v1,
        v1 -- [scalar] v2,
        v2 -- [scalar] i5 [particle=\(5\)],
        v2 -- [scalar] v3,
        v3 -- [scalar] i3 [particle=\(1\)],
        v3 -- [scalar] i4 [particle=\(2\)]
    }; &=\frac{ \im g^3}{\qty(p_1+p_2)^2} \frac{\qty(m^2+\qty(p_3+p_4)^2+\qty(p_1+p_2)^2)\qty(m^2+\qty(p_3+p_4)^2)}{\qty(p_3+p_4)^2\qty(\qty(p_3+p_4)^2+m^2)}=\frac{ \im g^3}{\qty(p_1+p_2)^2} \frac{\qty(m^2+\qty(p_3+p_4)^2+\qty(p_1+p_2)^2)}{\qty(p_3+p_4)^2}
\end{align*}

Combining this graph with,

\begin{align*}
    \feynmandiagram [small] {
        i1 [particle = \(4\)]-- [scalar] v1,
        i2 [particle=\(5\)]-- [scalar] v1,
        v1 -- [scalar] i5 [particle=\(3\)],
        v1 -- [scalar] v2,
        v2 -- [scalar] i3 [particle=\(2\)],
        v2 -- [scalar] i4 [particle=\(1\)]
    }; &=\frac{-\im 3g^2\im g\qty(m^2+\qty(p_1+p_2)^2)}{\im\qty(p_1+p_2)^2\qty(\qty(p_1+p_2)^2+m^2)}=\frac{-3\im g^3}{\qty(p_1+p_2)^2}
\end{align*}

We get,

\begin{align*}
    \feynmandiagram [small] {
        i1 [particle = \(4\)]-- [scalar] v1 [blob],
        i2 [particle=\(5\)]-- [scalar] v1,
        v1 -- [scalar] i5 [particle=\(3\)],
        v1 -- [scalar] v2,
        v2 -- [scalar] i3 [particle=\(2\)],
        v2 -- [scalar] i4 [particle=\(1\)]
    }; &=\frac{ \im g^3}{\qty(p_1+p_2)^2}\qty[ \frac{\qty(m^2+\qty(p_3+p_4)^2+\qty(p_1+p_2)^2)}{\qty(p_3+p_4)^2}-\frac{\qty(p_3+p_4)^2}{\qty(p_3+p_4)^2}+\frac{\qty(m^2+\qty(p_3+p_5)^2+\qty(p_1+p_2)^2)}{\qty(p_3+p_5)^2}-\frac{\qty(p_3+p_5)^2}{\qty(p_3+p_5)^2}+\frac{\qty(m^2+\qty(p_5+p_4)^2+\qty(p_1+p_2)^2)}{\qty(p_5+p_4)^2}-\frac{\qty(p_5+p_4)^2}{\qty(p_5+p_4)^2}]\\
        &=\frac{ \im g^3\qty(m^2+\qty(p_1+p_2)^2)}{\qty(p_1+p_2)^2}\qty[ \frac{1}{\qty(p_3+p_4)^2}+\frac{1}{\qty(p_3+p_5)^2}+\frac{1}{\qty(p_5+p_4)^2}]
\end{align*}

Now we have to sum the contributions of $1$ being in the middle,

\begin{align*}
    \feynmandiagram [small] {
        i1 [particle = \(3\)]-- [scalar] v1,
        i2 [particle=\(4\)]-- [scalar] v1,
        v1 -- [scalar] v2,
        v2 -- [scalar] i5 [particle=\(1\)],
        v2 -- [scalar] v3,
        v3 -- [scalar] i3 [particle=\(5\)],
        v3 -- [scalar] i4 [particle=\(2\)]
    }; + \feynmandiagram [small] {
        i1 [particle = \(4\)]-- [scalar] v1,
        i2 [particle=\(1\)]-- [scalar] v1,
        v1 -- [scalar] i5 [particle=\(3\)],
        v1 -- [scalar] v2,
        v2 -- [scalar] i3 [particle=\(2\)],
        v2 -- [scalar] i4 [particle=\(5\)]
    };+\textnormal{ perm}
\end{align*}

Which will be,

\begin{align*}
    \frac{\im g^3}{\qty(p_2+p_3)^2}\frac{m^2+\qty(p_2+p_3)^2+\qty(p_4+p_5)^2}{\qty(p_4+p_5)^2}-\frac{3\im g^3}{\qty(p_2+p_3)^2}-\frac{3\im g^3}{\qty(p_4+p_5)^2}+\qty(3\leftrightarrow4,5)
\end{align*}

Summing all the contributions we have,

\begin{align*}
    =&\frac{ \im g^3m^2}{\qty(p_1+p_2)^2}\qty[ \frac{1}{\qty(p_3+p_4)^2}+\frac{1}{\qty(p_3+p_5)^2}+\frac{1}{\qty(p_5+p_4)^2}]+\qty(2\leftrightarrow3,4,5)\\
    &\quad\quad\quad+ \im g^3\qty[ \frac{1}{\qty(p_3+p_4)^2}+\frac{1}{\qty(p_3+p_5)^2}+\frac{1}{\qty(p_5+p_4)^2}+\frac{1}{\qty(p_2+p_4)^2}+\frac{1}{\qty(p_2+p_5)^2}+\frac{1}{\qty(p_5+p_4)^2}+\frac{1}{\qty(p_3+p_2)^2}+\frac{1}{\qty(p_3+p_5)^2}+\frac{1}{\qty(p_5+p_2)^2}+\frac{1}{\qty(p_3+p_4)^2}+\frac{1}{\qty(p_3+p_2)^2}+\frac{1}{\qty(p_2+p_4)^2}]\\
    &\quad\quad\quad+ \frac{\im g^3m^2}{\qty(p_2+p_3)^2\qty(p_4+p_5)^2}-\frac{2\im g^3}{\qty(p_2+p_3)^2}-\frac{2\im g^3}{\qty(p_4+p_5)^2}+\qty(3\leftrightarrow4,5)\\
    =&\frac{ \im g^3m^2}{\qty(p_1+p_2)^2}\qty[ \frac{1}{\qty(p_3+p_4)^2}+\frac{1}{\qty(p_3+p_5)^2}+\frac{1}{\qty(p_5+p_4)^2}]+\qty(2\leftrightarrow3,4,5)\\
    &\quad\quad\quad+ \frac{\im g^3m^2}{\qty(p_2+p_3)^2\qty(p_4+p_5)^2}+ \frac{\im g^3m^2}{\qty(p_2+p_4)^2\qty(p_3+p_5)^2}+ \frac{\im g^3m^2}{\qty(p_2+p_5)^2\qty(p_4+p_3)^2}
\end{align*}

By residue, any amplitude with just one massive external on-shell leg is zero. For two massive external on-shell legs, let's take as massive $1,2$,

\begin{align*}
    \feynmandiagram [small] {
        i1 [particle = \(3\)]-- [scalar] v1,
        i2 [particle=\(4\)]-- [scalar] v1,
        v1 -- [scalar] v2,
        v2 -- [scalar] i5 [particle=\(5\)],
        v2 -- [scalar] v3,
        v3 -- [scalar] i3 [particle=\(1\)],
        v3 -- [scalar] i4 [particle=\(2\)]
    }; 
    &= \im g^3\frac{\qty(\qty(p_1+p_2)^2-m^2)}{\qty(p_1+p_2)^2\qty(m^2+\qty(p_1+p_2)^2)}\frac{\qty(m^2+\qty(p_1+p_2)^2+\qty(p_3+p_4)^2)}{\qty(p_3+p_4)^2}
\end{align*}

Combining this graph with,

\begin{align*}
    \feynmandiagram [small] {
        i1 [particle = \(4\)]-- [scalar] v1,
        i2 [particle=\(5\)]-- [scalar] v1,
        v1 -- [scalar] i5 [particle=\(3\)],
        v1 -- [scalar] v2,
        v2 -- [scalar] i3 [particle=\(2\)],
        v2 -- [scalar] i4 [particle=\(1\)]
    }; 
    &=-3\im g^3\frac{\qty(p_1+p_2)^2-m^2}{\qty(p_1+p_2)^2\qty(m^2+\qty(p_1+p_2)^2)}
\end{align*}

so,

\begin{align*}
    \feynmandiagram [small] {
        i1 [particle = \(4\)]-- [scalar] v1 [blob],
        i2 [particle=\(5\)]-- [scalar] v1,
        v1 -- [scalar] i5 [particle=\(3\)],
        v1 -- [scalar] v2,
        v2 -- [scalar] i3 [particle=\(2\)],
        v2 -- [scalar] i4 [particle=\(1\)]
    };
    &=\im g^3\frac{\qty(\qty(p_1+p_2)^2-m^2)}{\qty(p_1+p_2)^2\qty(m^2+\qty(p_1+p_2)^2)}\qty[\frac{\qty(m^2+\qty(p_1+p_2)^2+\qty(p_3+p_4)^2)}{\qty(p_3+p_4)^2}-\frac{\qty(p_3+p_4)^2}{\qty(p_3+p_4)^2}+\qty(5\leftrightarrow3,4)]\\
    &=\im g^3\frac{\qty(\qty(p_1+p_2)^2-m^2)}{\qty(p_1+p_2)^2}\qty[\frac{1}{\qty(p_3+p_4)^2}+\frac{1}{\qty(p_5+p_4)^2}+\frac{1}{\qty(p_3+p_5)^2}]
\end{align*}

The other contributions are,

\begin{align*}
    \feynmandiagram [small] {
        i1 [particle = \(4\)]-- [scalar] v1 [blob],
        i2 [particle=\(5\)]-- [scalar] v1,
        v1 -- [scalar] i5 [particle=\(2\)],
        v1 -- [scalar] v2,
        v2 -- [scalar] i3 [particle=\(3\)],
        v2 -- [scalar] i4 [particle=\(1\)]
    };
    &=\frac{\im g^3\qty(p_1+p_3)^2}{\qty(m^2+\qty(p_1+p_3)^2)}\qty[\frac{1}{m^2+\qty(p_2+p_4)^2}+\frac{1}{m^2+\qty(p_2+p_5)^2}+\frac{1}{(p_4+p_5)^2}]
\end{align*}

Now we have to sum the contributions of $1$ being in the middle,

\begin{align*}
    \feynmandiagram [small] {
        i1 [particle = \(3\)]-- [scalar] v1,
        i2 [particle=\(4\)]-- [scalar] v1,
        v1 -- [scalar] v2,
        v2 -- [scalar] i5 [particle=\(1\)],
        v2 -- [scalar] v3,
        v3 -- [scalar] i3 [particle=\(5\)],
        v3 -- [scalar] i4 [particle=\(2\)]
    }; + \feynmandiagram [small] {
        i1 [particle = \(4\)]-- [scalar] v1,
        i2 [particle=\(1\)]-- [scalar] v1,
        v1 -- [scalar] i5 [particle=\(3\)],
        v1 -- [scalar] v2,
        v2 -- [scalar] i3 [particle=\(2\)],
        v2 -- [scalar] i4 [particle=\(5\)]
    };+\textnormal{ perm}
\end{align*}

which are,

\begin{align*}
    &=\im g^3\frac{\qty(p_2+p_3)^2+\qty(p_4+p_5)^2}{\qty(m^2+\qty(p_2+p_3)^2)\qty(p_4+p_5)^2}-\frac{3\im g^3}{m^2+\qty(p_2+p_3)^2}-\frac{3\im g^3}{\qty(p_4+p_5)^2}+\qty(3\leftrightarrow4,5)
\end{align*}

So, summing all the contributions,

\begin{align*}
    &=\im g^3\frac{\qty(\qty(p_1+p_2)^2-m^2)}{\qty(p_1+p_2)^2}\qty[\frac{1}{\qty(p_3+p_4)^2}+\frac{1}{\qty(p_5+p_4)^2}+\frac{1}{\qty(p_3+p_5)^2}]\\
    &\quad\quad\quad+\frac{\im g^3\qty(p_1+p_3)^2}{\qty(m^2+\qty(p_1+p_3)^2)}\qty[\frac{1}{m^2+\qty(p_2+p_4)^2}+\frac{1}{m^2+\qty(p_2+p_5)^2}+\frac{1}{(p_4+p_5)^2}]+\qty(3\leftrightarrow4,5)\\
    &\quad\quad\quad+\im g^3\frac{\qty(p_2+p_3)^2+\qty(p_4+p_5)^2}{\qty(m^2+\qty(p_2+p_3)^2)\qty(p_4+p_5)^2}-\frac{3\im g^3}{m^2+\qty(p_2+p_3)^2}-\frac{3\im g^3}{\qty(p_4+p_5)^2}+\qty(3\leftrightarrow4,5)\\
    &=-\im g^3\frac{m^2}{\qty(p_1+p_2)^2}\qty[\frac{1}{\qty(p_3+p_4)^2}+\frac{1}{\qty(p_5+p_4)^2}+\frac{1}{\qty(p_3+p_5)^2}]\\
    &\quad\quad\quad+\im g^3\qty[\frac{1}{\qty(p_3+p_4)^2}+\frac{1}{\qty(p_5+p_4)^2}+\frac{1}{\qty(p_3+p_5)^2}]\\
    &\quad\quad\quad+\im g^3\qty[\frac{1}{2p_2\cdot p_4}+\frac{1}{2p_2\cdot p_5}+\frac{1}{(p_4+p_5)^2}]+\qty(3\leftrightarrow4,5)\\
    &\quad\quad\quad-\im g^3\frac{m^2}{\qty(2p_1\cdot p_3)}\qty[\frac{1}{2p_2\cdot p_4}+\frac{1}{2p_2\cdot p_5}+\frac{1}{(p_4+p_5)^2}]+\qty(3\leftrightarrow4,5)\\
    &\quad\quad\quad+\im g^3\frac{-m^2+2p_2\cdot p_3+\qty(p_4+p_5)^2}{\qty(2p_2\cdot p_3)\qty(p_4+p_5)^2}-\frac{3\im g^3}{2p_2\cdot p_3}-\frac{3\im g^3}{\qty(p_4+p_5)^2}+\qty(3\leftrightarrow4,5)\\
    &=-\frac{\im g^3m^2}{\qty(p_1+p_2)^2}\qty[\frac{1}{\qty(p_3+p_4)^2}+\frac{1}{\qty(p_5+p_4)^2}+\frac{1}{\qty(p_3+p_5)^2}]\\
    &\quad\quad\quad+\im g^3\qty[\frac{1}{\qty(p_3+p_4)^2}+\frac{1}{\qty(p_5+p_4)^2}+\frac{1}{\qty(p_3+p_5)^2}]\\
    &\quad\quad\quad+\im g^3\qty[\frac{1}{2p_2\cdot p_4}+\frac{1}{2p_2\cdot p_5}+\frac{1}{(p_4+p_5)^2}+\frac{1}{2p_2\cdot p_3}+\frac{1}{2p_2\cdot p_5}+\frac{1}{(p_3+p_5)^2}+\frac{1}{2p_2\cdot p_4}+\frac{1}{2p_2\cdot p_3}+\frac{1}{(p_4+p_3)^2}]\\
    &\quad\quad\quad-\frac{\im g^3m^2}{\qty(2p_1\cdot p_3)}\qty[\frac{1}{2p_2\cdot p_4}+\frac{1}{2p_2\cdot p_5}+\frac{1}{(p_4+p_5)^2}]+\qty(3\leftrightarrow4,5)\\
    &\quad\quad\quad-\frac{\im g^3m^2}{\qty(2p_2\cdot p_3)\qty(p_4+p_5)^2}+\qty(3\leftrightarrow4,5)\\
    &\quad\quad\quad-\frac{2\im g^3}{2p_2\cdot p_3}-\frac{2\im g^3}{\qty(p_4+p_5)^2}-\frac{2\im g^3}{2p_2\cdot p_4}-\frac{2\im g^3}{\qty(p_3+p_5)^2}-\frac{2\im g^3}{2p_2\cdot p_5}-\frac{2\im g^3}{\qty(p_4+p_3)^2}\\
    &=-\frac{\im g^3m^2}{\qty(p_1+p_2)^2}\qty[\frac{1}{\qty(p_3+p_4)^2}+\frac{1}{\qty(p_5+p_4)^2}+\frac{1}{\qty(p_3+p_5)^2}]\\
    &\quad\quad\quad-\frac{\im g^3m^2}{\qty(2p_1\cdot p_3)}\qty[\frac{1}{2p_2\cdot p_4}+\frac{1}{2p_2\cdot p_5}+\frac{1}{(p_4+p_5)^2}]+\qty(3\leftrightarrow4,5)\\
    &\quad\quad\quad-\frac{\im g^3m^2}{\qty(2p_2\cdot p_3)\qty(p_4+p_5)^2}+\qty(3\leftrightarrow4,5)
\end{align*}

Is almost the same of the all massless, but with a different denominator in the $1,2$ channel. Now with three massive legs, being $3,4,5$,

\begin{align*}
    \feynmandiagram [small] {
        i1 [particle = \(3\)]-- [scalar] v1,
        i2 [particle=\(4\)]-- [scalar] v1,
        v1 -- [scalar] v2,
        v2 -- [scalar] i5 [particle=\(5\)],
        v2 -- [scalar] v3,
        v3 -- [scalar] i3 [particle=\(1\)],
        v3 -- [scalar] i4 [particle=\(2\)]
    }; 
    &= \im g^3\frac{\qty(\qty(p_1+p_2)^2+\qty(p_3+p_4)^2)\qty(-m^2+\qty(p_3+p_4)^2)}{\qty(p_1+p_2)^2\qty(p_3+p_4)^2\qty(m^2+\qty(p_3+p_4)^2)}
\end{align*}

With,

\begin{align*}
    \feynmandiagram [small] {
        i1 [particle = \(4\)]-- [scalar] v1,
        i2 [particle=\(5\)]-- [scalar] v1,
        v1 -- [scalar] i5 [particle=\(3\)],
        v1 -- [scalar] v2,
        v2 -- [scalar] i3 [particle=\(2\)],
        v2 -- [scalar] i4 [particle=\(1\)]
    }; &=\frac{-\im 3g^2\im g\qty(m^2+\qty(p_1+p_2)^2)}{\im\qty(p_1+p_2)^2\qty(\qty(p_1+p_2)^2+m^2)}=\frac{-3\im g^3}{\qty(p_1+p_2)^2}
\end{align*}

So,

\begin{align*}
    \feynmandiagram [small] {
        i1 [particle = \(4\)]-- [scalar] v1 [blob],
        i2 [particle=\(5\)]-- [scalar] v1,
        v1 -- [scalar] i5 [particle=\(3\)],
        v1 -- [scalar] v2,
        v2 -- [scalar] i3 [particle=\(2\)],
        v2 -- [scalar] i4 [particle=\(1\)]
    };
    &=\im g^3\frac{\qty(\qty(p_1+p_2)^2+\qty(p_3+p_4)^2)\qty(-m^2+\qty(p_3+p_4)^2)}{\qty(p_1+p_2)^2\qty(p_3+p_4)^2\qty(m^2+\qty(p_3+p_4)^2)}-\frac{3\im g^3}{\qty(p_1+p_2)^2}\\
    &=\frac{\im g^3}{\qty(p_1+p_2)^2}\qty[\frac{\qty(\qty(p_1+p_2)^2+\qty(p_3+p_4)^2)\qty(-m^2+\qty(p_3+p_4)^2)}{\qty(p_3+p_4)^2\qty(m^2+\qty(p_3+p_4)^2)}-\frac{\qty(p_3+p_4)^2\qty(m^2+\qty(p_3+p_4)^2)}{\qty(p_3+p_4)^2\qty(m^2+\qty(p_3+p_4)^2)}]+\qty(5\leftrightarrow3,4)\\
    &=\frac{\im g^3}{\qty(p_1+p_2)^2}\qty[\frac{\qty(p_1+p_2)^2\qty(-m^2+\qty(p_3+p_4)^2)-2m^2\qty(p_3+p_4)^2}{\qty(p_3+p_4)^2\qty(m^2+\qty(p_3+p_4)^2)}]+\qty(5\leftrightarrow3,4)
\end{align*}

%%%%%%%%%%%%%%%%%%%%%%%%%%%%%%%%%%%%%%%%%%%%%%%%%%%%%%%%%%%%%

\newpage

\printbibliography

\end{document}