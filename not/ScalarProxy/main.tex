\documentclass[twoside]{amsart}

\usepackage[english]{babel}
\usepackage{csquotes}
\usepackage[style=numeric-comp, backend=biber]{biblatex}
\usepackage{amsmath}
\usepackage{amssymb}
\usepackage{bbm}
\usepackage{graphics}
\usepackage{mathtools}
\usepackage[hidelinks]{hyperref}
\usepackage{physics}
\usepackage{enumitem}
\usepackage{slashed}
\usepackage{tensor}
\usepackage[lmargin=0.5cm,rmargin=0.5cm, tmargin =1cm,bmargin =1cm]{geometry}
\usepackage{tensor}
\usepackage[english]{cleveref}
\usepackage{stackengine}
\usepackage[compat=1.1.0]{tikz-feynman}
\usepackage{graphicx}
\usepackage{setspace}

% Custom command to put a wedge over the comma
\newcommand{\wedgecomma}{\stackon[1pt]{,}{\smash{\scriptsize$\wedge$}}}
\newcommand{\wedgecomm}[2]{\qty[ #1\ \wedgecomma\ #2 ]}

\AtBeginDocument{\renewcommand*{\hbar}{{\mkern-1mu\mathchar'26\mkern-8mu\textnormal{h}}}}
\AtBeginDocument{\newcommand{\e}{\textnormal{e}}}
\AtBeginDocument{\newcommand{\im}{\textnormal{i}}}
\AtBeginDocument{\newcommand{\luz}{\textnormal{c}}}
\AtBeginDocument{\newcommand{\grav}{\textnormal{G}}}
\AtBeginDocument{\newcommand{\kb}{{\textnormal{k}_{\textnormal{B}}}}}
\newcommand{\Dd}[1]{\mathcal D #1}
\newcommand{\Det}[1]{\textup{Det} #1}
\newcommand{\sgn}[1]{\mbox{sgn}\qty(#1)}
\newcommand{\cqd}{\hfill$\blacksquare$}
\newcommand{\dbar}{\mbox{\dj}}

\numberwithin{equation}{section}

\newtheorem{teo}{Teorema}[section]
\newtheorem{defi}{Definição}[section]
\newtheorem{lem}{Lema}[section]
\newtheorem{hip}{Hipótese}[subsection]

\pagestyle{plain}

%\AddToHook{cmd/section/before}{\clearpage}

\addbibresource{refs.bib}

\newtheorem{teorema}{Teorema}[section]
\newtheorem{definicao}{Definição}[section]
\newtheorem{lema}{Lema}[section]
\newtheorem{hipotese}{Hipótese}[section]
\newtheorem{postulado}{Postulado}[section]

\newcommand{\thistheoremname}{}
\newtheorem*{genericthm}{\thistheoremname}
\newenvironment{namedthm}[1]
  {\renewcommand{\thistheoremname}{#1}
   \begin{genericthm}}
  {\end{genericthm}}


  \newcommand\numberthis{\addtocounter{equation}{1}\tag{\theequation}}
  
\title{
Scalar Proxy
}
\author{
  Vicente V. Figueira
       }
% \date{\today}

\allowdisplaybreaks

\newsavebox{\phifourpoint}
\savebox{\phifourpoint}{%
  \begin{tikzpicture}% [scale=1] ajusta se quiser
    \begin{feynman}
      \vertex (A) at ( 1.2,  1.2) {$\phi_3$};
      \vertex (B) at (-1.2, -1.2) {$\phi_1$};
      \vertex (C) at (-1.2,  1.2) {$\phi_2$};
      \vertex (D) at ( 1.2, -1.2) {$\phi_4$};
      \vertex[blob] (V) at (0,0) {};
      \diagram*{
        (A) -- [scalar] (V),
        (B) -- [scalar] (V),
        (C) -- [scalar]  (V),
        (D) -- [scalar]  (V),
      };
    \end{feynman}
  \end{tikzpicture}%
}

\begin{document}

\maketitle

%\tableofcontents

%%%%%%%%%%%%%%%%%%%%%%%%%%%%%%%%%%%%%%%%%%%%%%%%%%%%%%%%%%%%%

\nocite{*}

\section{Introdução}

O Espaço-Tempo (A)dS é definido como, \[\mp\qty(x^{-1})^2-\qty(x^0)^2+\qty(x^1)^2+\qty(x^2)^2+\qty(x^3)^2=\mp L^2\] 
Onde o sinal de cima é para AdS, e o sinal de baixo para dS. A álgebra de isometria no embedding 5 dimensional é: 
\begin{align*}
    \comm{J^{IK}}{J^{LM}}&=-4\im\eta^{[I|[L}J^{M]|K]}\\
    \comm{J^{IK}}{J^{LM}}&=-\im\eta^{IL}J^{MK}+\im\eta^{IM}J^{LK}+\im\eta^{KL}J^{MI}-\im\eta^{KM}J^{LI}
\end{align*}
Da qual podemos interpretar como geradores de translação $J^{-1\alpha}$,
\begin{align*}
    \comm{J^{-1\alpha}}{J^{-1\beta}}&=-\im\eta^{-1-1}J^{\beta\alpha}+\im\eta^{-1\beta}J^{-1\alpha}+\im\eta^{\alpha-1}J^{\beta-1}-\im\eta^{\alpha\beta}J^{-1-1}\\
    \comm{J^{-1\alpha}}{J^{-1\beta}}&=\mp\im J^{\alpha\beta}
\end{align*}
As outras relações de comutação são,
\begin{align*}
    \comm{J^{\alpha\beta}}{J^{-1\mu}}&=-\im\eta^{\alpha-1}J^{\mu\beta}+\im\eta^{\alpha\mu}J^{-1\beta}+\im\eta^{\beta-1}J^{\mu\alpha}-\im\eta^{\beta\mu}J^{-1\alpha}\\
    \comm{J^{\alpha\beta}}{J^{-1\mu}}&=\im\eta^{\alpha\mu}J^{-1\beta}-\im\eta^{\beta\mu}J^{-1\alpha}\\
    \comm{J^{\alpha\beta}}{J^{-1\mu}}&=-2\im J^{-1[\alpha}\eta^{\beta]\mu}
\end{align*}
E,
\begin{align*}
    \comm{J^{\alpha\beta}}{J^{\mu\nu}}&=-4\im\eta^{[\alpha|[\mu}J^{\nu]|\beta]}
\end{align*}
Como os geradores de translação necessitam ter dimensão, $P^\alpha=\frac1LJ^{-1\alpha}$. A álgebra completa é,
\[\comm{J^{\alpha\beta}}{J^{\mu\nu}}=-4\im\eta^{[\alpha|[\mu}J^{\nu]|\beta]},\ \ \ \comm{J^{\alpha\beta}}{P^\mu}=-2\im P^{[\alpha}\eta^{\beta]\mu},\ \ \ \comm{P^{\alpha}}{P^{\beta}}=\mp\frac{\im}{L^2} J^{\alpha\beta}\]
O bilinear mais geral para essa álgebra é,
\[\expval{J_{\alpha\beta},J_{\mu\nu}}=\pm2\lambda\eta_{\alpha[\mu}\eta_{\nu]\beta},\ \ \ \expval{J_{\alpha\beta},P_\mu}=0,\ \ \ \expval{P_\alpha,P_\mu}=\frac{\lambda}{L^2}\eta_{\alpha\mu}\]
A ação de Einstein-Hilbert com termo de constante cosmológica é,
\begin{align*}
    S_{\textnormal{EH}}&=\frac{1}{2\kappa}\int\limits_M\star\vb{R}_{\alpha\beta}\wedge\vb e^\alpha\wedge\vb e^\beta-\frac{\Lambda}{\kappa4!}\int\limits_M\epsilon_{\alpha\beta\mu\nu}\vb e^\alpha\wedge\vb e^\beta\wedge\vb e^\mu\wedge\vb e^\nu\\
    S_{\textnormal{EH}}&=\frac{1}{2\kappa}\eta_{\mu\alpha}\eta_{\beta\nu}\int\limits_M\star\vb{R}^{\mu\nu}\wedge\vb e^\alpha\wedge\vb e^\beta\pm\frac{3\cdot 2}{\kappa L^24!}\int\limits_M\vb e^\alpha\wedge\vb e^\beta\wedge\star\qty(\vb e_\alpha\wedge\vb e_\beta)\\
    S_{\textnormal{EH}}&=\frac{\pm}{4\lambda\kappa}\pm2\lambda\eta_{\mu[\alpha}\eta_{\beta]\nu}\int\limits_M\star\vb{R}^{\mu\nu}\wedge\vb e^\alpha\wedge\vb e^\beta\pm\frac{3\cdot 2}{\lambda\kappa L^24!}\lambda\eta_{\mu[\alpha}\eta_{\beta]\nu}\int\limits_M\vb e^\mu\wedge\vb e^\nu\wedge\star\qty(\vb e^\alpha\wedge\vb e^\beta)\\
    S_{\textnormal{EH}}&=\frac{\pm}{4\lambda\kappa}\expval{J_{\mu\nu},J_{\alpha\beta}}\int\limits_M\star\vb{R}^{\mu\nu}\wedge\vb e^\alpha\wedge\vb e^\beta+\frac{3}{\lambda\kappa L^24!}\expval{J_{\mu\nu},J_{\alpha\beta}}\int\limits_M\vb e^\mu\wedge\vb e^\nu\wedge\star\qty(\vb e^\alpha\wedge\vb e^\beta)\\
    S_{\textnormal{EH}}&=\frac{\pm}{4\lambda\kappa}\pm\im L ^2\expval{J_{\mu\nu},\comm{P_\alpha}{P_\beta}}\int\limits_M\star\vb{R}^{\mu\nu}\wedge\vb e^\alpha\wedge\vb e^\beta+\frac{3\qty(\pm)^2\im^2L^4}{\lambda\kappa L^24!}\expval{\comm{P_\mu}{P_\nu},\comm{P_\alpha}{P_\beta}}\int\limits_M\vb e^\mu\wedge\vb e^\nu\wedge\star\qty(\vb e^\alpha\wedge\vb e^\beta)\\
    S_{\textnormal{EH}}&=-\frac{L^2}{4\lambda\kappa}\expval{\im J_{\mu\nu},\comm{\im P_\alpha}{\im P_\beta}}\int\limits_M\star\vb{R}^{\mu\nu}\wedge\vb e^\alpha\wedge\vb e^\beta-\frac{L^2}{8\lambda\kappa}\expval{\comm{\im P_\mu}{\im P_\nu},\comm{\im P_\alpha}{\im P_\beta}}\int\limits_M\vb e^\mu\wedge\vb e^\nu\wedge\star\qty(\vb e^\alpha\wedge\vb e^\beta)\\
    S_{\textnormal{EH}}&=-\frac{L^2}{2\lambda\kappa}\int\limits_M\expval{\star\vb{R}\ \wedgecomma\ \wedgecomm{\vb e}{\vb e}}-\frac{L^2}{8\lambda\kappa}\int\limits_M\expval{\wedgecomm{\vb e}{\vb e}\ \wedgecomma \star\wedgecomm{\vb e}{\vb e}}
\end{align*}
Com $\vb R=\frac12\im J_{\mu\nu}\vb R^{\mu\nu}$ e $\vb e=\im P_\mu\vb e^\mu$. Note,
\begin{align*}
    \vb R&=\frac12\im\vb R_{\alpha\beta}J^{\alpha\beta}=\frac12\vb d\boldsymbol\omega_{\alpha\beta}\im J^{\alpha\beta}+\frac12\tensor{{\boldsymbol\omega}}{_\alpha^\rho}\wedge\tensor{{\boldsymbol\omega}}{_\rho_\beta}\im J^{\alpha\beta}=\vb d\boldsymbol\omega+\frac12\tensor{{\boldsymbol\omega}}{_\mu_\nu}\wedge\tensor{{\boldsymbol\omega}}{_\rho_\sigma}\eta^{\rho\nu}\im J^{\mu\sigma}=\vb d\boldsymbol\omega+\frac18\tensor{{\boldsymbol\omega}}{_\mu_\nu}\wedge\tensor{{\boldsymbol\omega}}{_\rho_\sigma}4\im \eta^{[\rho|[\nu}J^{\mu]|\sigma]}\\
    \vb R&=\vb d\boldsymbol\omega-\frac18\tensor{{\boldsymbol\omega}}{_\mu_\nu}\wedge\tensor{{\boldsymbol\omega}}{_\rho_\sigma}\comm{J^{\rho\sigma}}{J^{\nu\mu}}=\vb d\boldsymbol\omega+\frac12\tensor{{\boldsymbol\omega}}{_\mu_\nu}\wedge\tensor{{\boldsymbol\omega}}{_\rho_\sigma}\comm{\frac\im 2J^{\mu\nu}}{\frac\im2J^{\rho\sigma}}=\vb d\boldsymbol\omega+\frac12\wedgecomm{\boldsymbol\omega}{\boldsymbol\omega}
\end{align*}
Seja então,
\begin{align*}
    \vb F&=\vb d\qty(\boldsymbol \omega+\vb e)+\frac12\wedgecomm{\boldsymbol \omega+\vb e}{\boldsymbol \omega+\vb e}\\
    \vb F&=\vb d\boldsymbol \omega+\frac12\wedgecomm{\boldsymbol \omega}{\boldsymbol \omega}+\frac12\wedgecomm{\vb e}{\vb e}+\vb d\vb e+\wedgecomm{\boldsymbol \omega}{\vb e}\\
    \vb F&=\vb R+\frac12\wedgecomm{\vb e}{\vb e}+\vb T
\end{align*}
Logo, 
\begin{align*}
    \tilde S&=\int\limits_{M}\expval{\vb F\ \wedgecomma \star \vb F}\\
    \tilde S&=\int\limits_{M}\expval{\vb R\ \wedgecomma \star \vb R}+\frac12\int\limits_{M}\expval{\vb R\ \wedgecomma \star \wedgecomm{\vb e}{\vb e}}+\frac12\int\limits_{M}\expval{ \wedgecomm{\vb e}{\vb e}\ \wedgecomma \star\vb R}+\int\limits_{M}\expval{\vb T\ \wedgecomma \star \vb T}+\frac14\int\limits_{M}\expval{\wedgecomm{\vb e}{\vb e}\ \wedgecomma \star \wedgecomm{\vb e}{\vb e}}\\
    \tilde S&=\int\limits_{M}\expval{\vb R\ \wedgecomma \star \vb R}+\int\limits_{M}\expval{\vb R\ \wedgecomma \star \wedgecomm{\vb e}{\vb e}}+\frac14\int\limits_{M}\expval{\wedgecomm{\vb e}{\vb e}\ \wedgecomma \star \wedgecomm{\vb e}{\vb e}}+\int\limits_{M}\expval{\vb T\ \wedgecomma \star \vb T}
\end{align*}
Isto é, a ação de Yang-Mills para o grupo de isometria de (A)dS é a ação de Einstein-Hilbert com os termos adicionais do tensor de Riemann quadrado 
e o tensor de torção quadrado. A equação de movimento para $\boldsymbol \omega$ é,
\begin{align*}
    0=\delta_{\boldsymbol\omega}\tilde S&=2\int\limits_{M}\expval{\delta_{\boldsymbol \omega}\vb R\ \wedgecomma \star \vb R}+\int\limits_{M}\expval{\delta_{\boldsymbol \omega}\vb R\ \wedgecomma \star \wedgecomm{\vb e}{\vb e}}+2\int\limits_{M}\expval{\delta_{\boldsymbol \omega}\vb T\ \wedgecomma \star \vb T}\\
    0&=2\int\limits_{M}\expval{\vb d\delta\boldsymbol \omega+\wedgecomm{\delta\boldsymbol\omega}{\boldsymbol\omega}\ \wedgecomma \star\qty( \vb R+\frac12\wedgecomm{\vb e}{\vb e})}+2\int\limits_{M}\expval{\wedgecomm{\delta\boldsymbol\omega}{\vb e}\ \wedgecomma \star \vb T}\\
    0&=\int\limits_{M}\expval{\delta\boldsymbol \omega\ \wedgecomma\ \vb d\star\qty( \vb R+\frac12\wedgecomm{\vb e}{\vb e})+\wedgecomm{\boldsymbol\omega}{\star\qty( \vb R+\frac12\wedgecomm{\vb e}{\vb e})}+\wedgecomm{\vb e}{\star\vb T}}\\
    0&=\vb d_\nabla\star \vb R+\frac12\vb d_\nabla\star\wedgecomm{\vb e}{\vb e}+\wedgecomm{\vb e}{\star\vb d_\nabla \vb e}
\end{align*}

\section{Conformal Toy Model}

Consider the following Lagrangian, 
\[\mathcal L=-\frac12\Box \phi\Box\phi-\frac{g}{2}\phi^2\Box\phi-\frac{g^2}{8}\phi^4+m^2\qty(-\frac12\partial_\mu\phi\partial^\mu\phi+\frac{g}{3!}\phi^3)\]
Notice the form of the Lagrangian,
\[\mathcal L=-\frac12\qty(\Box\phi+\frac{g}{2}\phi^2)^2+m^2\qty(-\frac12\partial_\mu\phi\partial^\mu\phi+\frac{g}{3!}\phi^3)\]
It possesses the Feynman rules,
\begin{itemize}
    \item \ \ \feynmandiagram [horizontal=a to b] {
		a [particle=\(\phi\)] -- [scalar] b [particle=\(\phi\)],
		}; $=\frac1\im\frac{1}{m^2p^2+p^4}$
	\item \ \ \feynmandiagram [small,baseline = (b.base),horizontal=a to b] {
		a [particle=\(\phi_1\)] -- [scalar] b  ,
		b -- [scalar] c [particle=\(\phi_2\)],
		b -- [scalar] d [particle=\(\phi_3\)],
		}; $=\im g\qty(p_1^2+p_2^2+p_3^2+m^2)$
	\item \ \ \feynmandiagram [small,baseline = (b.base)] {
		a [particle=\(\phi_1\)] -- [scalar] b  ,
		e [particle=\(\phi_1\)] -- [scalar] b  ,
		b -- [scalar] c [particle=\(\phi_2\)],
		b -- [scalar] d [particle=\(\phi_3\)],
		}; $=-3\im g^2$
\end{itemize}

Let's compute the self energy,
\begin{align}
    \im\Pi\qty(p^2)&=\feynmandiagram [baseline = (b.base),horizontal=a to b,layered layout] {
     a -- [scalar] b -- [ loop,scalar,min distance =2cm] b--[scalar] c,
     }; + \feynmandiagram [baseline = (b.base),horizontal=a to b,layered layout] {
     a -- [scalar] b -- [ half left,scalar] c --[half left,scalar] b,
     c -- [scalar] d,
     };+\cdots
\end{align}

\begin{align}
    \im{\Pi^{\qty(1)}}&=-\frac32\im g^2\int\frac{\dd[D]{\ell}}{\qty(2\pi)^D}\frac{1}{\im}\frac{1}{\ell^2}\frac{1}{\ell^2+m^2}\\
    \im{\Pi^{\qty(1)}}&=-\frac32 g^2\int\frac{\dd[D]{\ell}}{\qty(2\pi)^D}\frac{1}{\ell^2}\frac{1}{\ell^2+m^2}\\
    \im{\Pi^{\qty(1)}}&=-\frac32 g^2\frac{\im}{\qty(4\pi)^{\frac D2}\Gamma\qty(\frac D2)}\qty(m^2)^{\frac D2-2}\frac{\Gamma\qty(2-\frac D2)\Gamma\qty(\frac D2-1)}{\Gamma\qty(1)}\\
    \im{\Pi^{\qty(1)}}&=-\frac32\im g^2\frac{\qty(m^2)^{-\epsilon}\Gamma\qty(\epsilon)\Gamma\qty(1-\epsilon)}{\qty(4\pi)^{2-\epsilon}\Gamma\qty(2-\epsilon)}
\end{align}

\begin{align}
    \im{\Pi^{\qty(2)}}&=\frac12\qty(\im g)^2\int\frac{\dd[D]{\ell}}{\qty(2\pi)^D}\frac{1}{\im^2}\frac{1}{\ell^2\qty(\ell+p)^2}\frac{\qty(m^2+\ell^2+p^2+\qty(\ell+p)^2)^2}{\ell^2+m^2}\frac{1}{\qty(\ell+p)^2+m^2}
\end{align}
For the mass renormalization we can take $p=0$,
\begin{align}
    \im{\Pi^{\qty(2)}}&=\frac12g^2\int\frac{\dd[D]{\ell}}{\qty(2\pi)^D}\frac{\qty(m^2+2\ell^2)^2}{\ell^4\qty(\ell^2+m^2)^2}
\end{align}
Let's compute the four point amplitude for this theory,
\begin{align}
    \feynmandiagram [small,baseline = (b.base),horizontal=b to d] {
		a [particle=\(\phi_1\)] -- [scalar] b  ,
		b -- [scalar] c [particle=\(\phi_2\)],
		b -- [scalar,momentum=\(P\)] d,
        d -- [scalar] e [particle=\(\phi_4\)],
        d -- [scalar] f [particle=\(\phi_3\)]
    }; &=\qty(\im g)^2\frac{\qty(p_1^2+p_2^2+\qty(p_1+p_2)^2+m^2)\qty(p_3^2+p_4^2+\qty(p_3+p_4)^2+m^2)}{\im\qty(p_1+p_2)^2\qty(\qty(p_1+p_2)^2+m^2)}
\end{align}
First let's consider all legs massless,
\begin{align}
    \feynmandiagram [small,baseline = (b.base),horizontal=b to d] {
		a [particle=\(\phi_1\)] -- [scalar] b  ,
		b -- [scalar] c [particle=\(\phi_2\)],
		b -- [scalar,momentum=\(P\)] d,
        d -- [scalar] e [particle=\(\phi_4\)],
        d -- [scalar] f [particle=\(\phi_3\)]
    }; &=\im g^2\frac{\qty(-s+m^2)\qty(-s+m^2)}{\qty(-s)\qty(-s+m^2)}=-\im g^2\frac{\qty(-s+m^2)}{s}
\end{align}
So,
\begin{align}
	\vcenter{\hbox{\usebox{\phifourpoint}}} &=-\im g^2\frac{\qty(-s+m^2)}{s}-\im g^2\frac{\qty(-t+m^2)}{t}-\im g^2\frac{\qty(-u+m^2)}{u}-3\im g^2\\
	\vcenter{\hbox{\usebox{\phifourpoint}}} &=-\im g^2\frac{\qty(-s+m^2)}{s}-\im g^2\frac{\qty(-t+m^2)}{t}-\im g^2\frac{\qty(-u+m^2)}{u}-\im g^2\frac{s}{s}-\im g^2\frac{t}{t}-\im g^2\frac{u}{u}\\
	\vcenter{\hbox{\usebox{\phifourpoint}}} &=-\im g^2m^2\qty(\frac1s+\frac1t+\frac1u)=\im g^2 m^2\qty(\frac{1}{\langle12\rangle[12]}+\frac{1}{\langle14\rangle[14]}+\frac{1}{\langle13\rangle[13]})
\end{align}
Para uma perna massiva, $\phi_4$,
\begin{align}
    \feynmandiagram [small,baseline = (b.base),horizontal=b to d] {
		a [particle=\(\phi_1\)] -- [scalar] b  ,
		b -- [scalar] c [particle=\(\phi_2\)],
		b -- [scalar,momentum=\(P\)] d,
        d -- [scalar] e [particle=\(\phi_4\)],
        d -- [scalar] f [particle=\(\phi_3\)]
    }; &=\qty(\im g)^2\frac{\qty(-s+m^2)\qty(-s)}{\im\qty(-s)\qty(-s+m^2)}=\im g^2
\end{align}
So,
\begin{align}
	\vcenter{\hbox{\usebox{\phifourpoint}}} &=\im g^2+\im g^2+\im g^2-3\im g^2=0
\end{align}
Para duas pernas massivas, $\phi_{3,4}$,
\begin{align}
    \feynmandiagram [small,baseline = (b.base),horizontal=b to d] {
		a [particle=\(\phi_1\)] -- [scalar] b  ,
		b -- [scalar] c [particle=\(\phi_2\)],
		b -- [scalar,momentum=\(P\)] d,
        d -- [scalar] e [particle=\(\phi_4\)],
        d -- [scalar] f [particle=\(\phi_3\)]
    }; &=\qty(\im g)^2\frac{\qty(-s+m^2)\qty(-s-m^2)}{\im\qty(-s)\qty(-s+m^2)}=\im g^2\frac{s+m^2}{s}\\
    \feynmandiagram [small,baseline = (b.base),vertical=b to d] {
		a [particle=\(\phi_2\)] -- [scalar] b  ,
		b -- [scalar] c [particle=\(\phi_3\)],
		b -- [scalar,momentum=\(P\)] d,
        d -- [scalar] e [particle=\(\phi_1\)],
        d -- [scalar] f [particle=\(\phi_4\)]
    }; &=\qty(\im g)^2\frac{\qty(-t)\qty(-t)}{\im\qty(-t)\qty(-t+m^2)}=-\im g^2\frac{t}{-t+m^2}\\
    \feynmandiagram [small,baseline = (b.base),vertical=b to d] {
		a [particle=\(\phi_2\)] -- [scalar] b  ,
		b -- [scalar] c [particle=\(\phi_4\)],
		b -- [scalar,momentum=\(P\)] d,
        d -- [scalar] e [particle=\(\phi_1\)],
        d -- [scalar] f [particle=\(\phi_3\)]
    }; &=-\im g^2\frac{u}{-u+m^2}
\end{align}
So,
\begin{align}
	\vcenter{\hbox{\usebox{\phifourpoint}}} &=\im g^2\frac{s+m^2}{s}-\im g^2\frac{t}{-t+m^2}-\im g^2\frac{u}{-u+m^2}-3\im g^2\\
	\vcenter{\hbox{\usebox{\phifourpoint}}} &=\im g^2\frac{s+m^2}{s}-\im g^2\frac{t}{-t+m^2}-\im g^2\frac{u}{-u+m^2}-\im g^2 \frac{s}{s}-\im g^2\frac{-t+m^2}{-t+m^2}-\im g^2\frac{-u+m^2}{-u+m^2}\\
	\vcenter{\hbox{\usebox{\phifourpoint}}} &=-\im g^2 m^2\qty(-\frac{1}{s}+\frac{1}{-t+m^2}+\frac{1}{-u+m^2})=-\im g^2 m^2\qty(\frac{1}{\langle12\rangle[12]}+\frac{1}{\langle1\vb4\rangle[1\vb 4]}+\frac{1}{\langle1\vb3\rangle[1\vb3]})
\end{align}
Para uma perna sem massa $\phi_1$,
\begin{align}
    \feynmandiagram [small,baseline = (b.base),horizontal=b to d] {
		a [particle=\(\phi_1\)] -- [scalar] b  ,
		b -- [scalar] c [particle=\(\phi_2\)],
		b -- [scalar,momentum=\(P\)] d,
        d -- [scalar] e [particle=\(\phi_4\)],
        d -- [scalar] f [particle=\(\phi_3\)]
    }; &=\qty(\im g)^2\frac{\qty(-s)\qty(-s-m^2)}{\im\qty(-s)\qty(-s+m^2)}=-\im g^2\frac{s+m^2}{-s+m^2}\\
    \feynmandiagram [small,baseline = (b.base),vertical=b to d] {
		a [particle=\(\phi_2\)] -- [scalar] b  ,
		b -- [scalar] c [particle=\(\phi_3\)],
		b -- [scalar,momentum=\(P\)] d,
        d -- [scalar] e [particle=\(\phi_1\)],
        d -- [scalar] f [particle=\(\phi_4\)]
    }; &=\qty(\im g)^2\frac{\qty(-t)\qty(-t-m^2)}{\im\qty(-t)\qty(-t+m^2)}=-\im g^2\frac{t+m^2}{-t+m^2}\\
    \feynmandiagram [small,baseline = (b.base),vertical=b to d] {
		a [particle=\(\phi_2\)] -- [scalar] b  ,
		b -- [scalar] c [particle=\(\phi_4\)],
		b -- [scalar,momentum=\(P\)] d,
        d -- [scalar] e [particle=\(\phi_1\)],
        d -- [scalar] f [particle=\(\phi_3\)]
    }; &=\qty(\im g)^2\frac{\qty(-u)\qty(-u-m^2)}{\im\qty(-u)\qty(-u+m^2)}=-\im g^2\frac{u+m^2}{-u+m^2}\\
\end{align}
So,
\begin{align}
	\vcenter{\hbox{\usebox{\phifourpoint}}} &=-\im g^2\frac{s+m^2}{-s+m^2}-\im g^2\frac{t+m^2}{-t+m^2}-\im g^2\frac{u+m^2}{-u+m^2}-3\im g^2\\
	\vcenter{\hbox{\usebox{\phifourpoint}}} &=-\im g^2\frac{s+m^2}{-s+m^2}-\im g^2\frac{t+m^2}{-t+m^2}-\im g^2\frac{u+m^2}{-u+m^2}-\im g^2\frac{-s+m^2}{-s+m^2}-\im g^2\frac{-t+m^2}{-t+m^2}-\im g^2\frac{-u+m^2}{-u+m^2}\\
	\vcenter{\hbox{\usebox{\phifourpoint}}} &=-\im g^2m^2\qty(\frac{1}{-s+m^2}+\frac{1}{-t+m^2}+\frac{1}{-u+m^2})=-\im g^2 m^2\qty(\frac{1}{\langle1\vb 2\rangle[1\vb2]}+\frac{1}{\langle1\vb4\rangle[1\vb 4]}+\frac{1}{\langle1\vb3\rangle[1\vb3]})
\end{align}

Cut comparison, only massless legs

\begin{align}
    \feynmandiagram [baseline = (b.base),horizontal=b to h, layered layout,small] {
        a -- [scalar] b [blob],
        f -- [scalar] b ,
        b -- [ scalar, momentum =\(l+3+4\)] g [blob], 
        g --[scalar,momentum =\(l+4\)] h [blob], 
        h --[scalar, momentum=\(l\)] b,
        g -- [scalar] d,
        h -- [scalar] e,
    };&=\frac{\im g^2m^2\qty(\im g m^2)^2}{\qty(\im m^2)^3}\qty(\frac{1}{\langle12\rangle[12]}-\frac{1}{\langle1l\rangle[1l]}-\frac{1}{\langle2l\rangle[2l]})
\end{align}

to solve for the cuts, $l^2=\qty(l+3+4)^2=\qty(l+4)^2=0$,
\begin{align}
    l^2&=0\Rightarrow l=-|l\rangle [l|\\
    0=\qty(l+4)^2&=\langle l4\rangle[l4]=0\Rightarrow |l]=|4]\\
    0=\qty(l+3+4)^2&=\langle lP_{34}\rangle[lP_{34}]+(3+4)^2=\langle l| 3+4|l]+\langle34\rangle[34]=\langle l| 3+4|4]+\langle34\rangle[34]\\
    \langle43\rangle[34]&=-\langle l3\rangle[34]\Rightarrow |l\rangle = -|4\rangle+z|3\rangle\\
    l&=-\qty(-|4\rangle+z|3\rangle)[4|
\end{align}
The cuts are solved by this. Hence,
\begin{align*}
    &=g^4\qty(\frac{1}{\langle12\rangle[12]}-\frac{1}{\langle1l\rangle[1l]}-\frac{1}{\langle2l\rangle[2l]})\\
    &=g^4\qty(\frac{1}{\langle12\rangle[12]}-\frac{1}{\qty(-\langle14\rangle+z\langle13\rangle)[14]}-\frac{1}{\qty(-\langle24\rangle+z\langle23\rangle)[24]})
\end{align*}
Now for internal massive lines,
\begin{align}
    \feynmandiagram [baseline = (b.base),horizontal=b to h, layered layout,small] {
        a -- [scalar] b [blob],
        f -- [scalar] b ,
        b -- [ scalar, momentum =\(l+3+4\)] g [blob], 
        g --[scalar,momentum =\(l+4\)] h [blob], 
        h --[scalar, momentum=\(l\)] b,
        g -- [scalar] d,
        h -- [scalar] e,
    };&=\frac{-\im g^2m^2\qty(-\im g m^2)^2}{\qty(-\im m^2)^3}\qty(\frac{1}{\langle12\rangle[12]}-\frac{1}{\langle1l\rangle[1l]}-\frac{1}{\langle2l\rangle[2l]})
\end{align}
With the cuts being, $l^2=\qty(l+3+4)^2=\qty(l+4)^2=-m^2$,
\begin{align}
    0=(l+4)^2-l^2&=2l\cdot p_4\\
    0=(l+4+3)^2-l^2&=2l\cdot \qty(4+3)+(4+3)^2=2l\cdot p_3+(4+3)^2
\end{align}
As ansatz, $l=|4\rangle [4|+\alpha|4\rangle [3|+\beta|3\rangle[4|$ satisfy both conditions above. The remaining condition is,
\begin{align}
    l^2&=-m^2\\
    -\alpha\beta[43]\langle43\rangle&=-m^2\Rightarrow \alpha=\frac{m^2}{\beta\langle34\rangle[34]}
\end{align}
Setting now $-\beta=z$,
\begin{align}
    l=|4\rangle [4|-\frac{m^2}{z\langle34\rangle[34]}|4\rangle [3|-z|3\rangle[4|
\end{align}
The value of the diagram is,
\begin{align*}
    &=g^4\qty(\frac{1}{\langle12\rangle[12]}+\frac{1}{\langle1l\rangle[l1]}+\frac{1}{\langle2l\rangle[l2]})\\
    &=g^4\qty(\frac{1}{\langle12\rangle[12]}+\frac{1}{-\langle1|l|1]}+\frac{1}{-\langle2|l|2]})\\
    &=g^4\qty(\frac{1}{\langle12\rangle[12]}-\frac{1}{\langle14\rangle[41]-\frac{m^2}{z\langle34\rangle[34]}\langle14\rangle[31]-z\langle13\rangle[41]}-\frac{1}{\langle2|l|2]})
\end{align*}
The explicit cut loop amplitude is,
\begin{align*}
    \feynmandiagram [baseline = (b.base),horizontal=b to g,small] {
        a -- [scalar] b ,
        f -- [scalar] c ,
        b -- [ scalar, momentum =\(l+3+4\)] g , 
        g --[scalar,momentum =\(l+4\)] h , 
        h --[scalar, momentum=\(l\)] c,
        c --[scalar, momentum=\(l-1\)] b,
        g -- [scalar] e,
        h -- [scalar] d,
    };
\end{align*}
Triple cut has no improvement, what about a double cut,

\begin{align}
    \vcenter{\hbox{\usebox{\phifourloop}}}&=\frac{\qty(\im g^2m^2)^2}{\qty(\im m^2)^2}\qty(\frac{1}{\langle12\rangle[12]}-\frac{1}{\langle1\ell\rangle[1\ell]}-\frac{1}{\langle2\ell\rangle[2\ell]})\qty(\frac{1}{\langle34\rangle[34]}+\frac{1}{\langle3\ell\rangle[3\ell]}+\frac{1}{\langle4\ell\rangle[4\ell]})
\end{align}

Five point amplitude,

\begin{align*}
    \feynmandiagram [small] {
        i1 [particle = \(3\)]-- [scalar] v1,
        i2 [particle=\(4\)]-- [scalar] v1,
        v1 -- [scalar] v2,
        v2 -- [scalar] i5 [particle=\(5\)],
        v2 -- [scalar] v3,
        v3 -- [scalar] i3 [particle=\(1\)],
        v3 -- [scalar] i4 [particle=\(2\)]
    }; &=  \frac{\qty(\im g)^3\qty(m^2+p_1^2+p_2^2+\qty(p_1+p_2)^2)\qty(m^2+p_5^2+\qty(p_3+p_4)^2+\qty(p_1+p_2)^2)\qty(m^2+p_3^2+p_4^2+\qty(p_3+p_4)^2)}{i^2\qty(p_1+p_2)^2\qty(\qty(p_1+p_2)^2+m^2)\qty(p_3+p_4)^2\qty(\qty(p_3+p_4)^2+m^2)}
\end{align*}

Let's consider the special case of all massless,

\begin{align*}
    \feynmandiagram [small] {
        i1 [particle = \(3\)]-- [scalar] v1,
        i2 [particle=\(4\)]-- [scalar] v1,
        v1 -- [scalar] v2,
        v2 -- [scalar] i5 [particle=\(5\)],
        v2 -- [scalar] v3,
        v3 -- [scalar] i3 [particle=\(1\)],
        v3 -- [scalar] i4 [particle=\(2\)]
    }; &=\frac{ \im g^3}{\qty(p_1+p_2)^2} \frac{\qty(m^2+\qty(p_3+p_4)^2+\qty(p_1+p_2)^2)\qty(m^2+\qty(p_3+p_4)^2)}{\qty(p_3+p_4)^2\qty(\qty(p_3+p_4)^2+m^2)}=\frac{ \im g^3}{\qty(p_1+p_2)^2} \frac{\qty(m^2+\qty(p_3+p_4)^2+\qty(p_1+p_2)^2)}{\qty(p_3+p_4)^2}
\end{align*}

Combining this graph with,

\begin{align*}
    \feynmandiagram [small] {
        i1 [particle = \(4\)]-- [scalar] v1,
        i2 [particle=\(5\)]-- [scalar] v1,
        v1 -- [scalar] i5 [particle=\(3\)],
        v1 -- [scalar] v2,
        v2 -- [scalar] i3 [particle=\(2\)],
        v2 -- [scalar] i4 [particle=\(1\)]
    }; &=\frac{-\im 3g^2\im g\qty(m^2+\qty(p_1+p_2)^2)}{\im\qty(p_1+p_2)^2\qty(\qty(p_1+p_2)^2+m^2)}=\frac{-3\im g^3}{\qty(p_1+p_2)^2}
\end{align*}

We get,

\begin{align*}
    \feynmandiagram [small] {
        i1 [particle = \(4\)]-- [scalar] v1 [blob],
        i2 [particle=\(5\)]-- [scalar] v1,
        v1 -- [scalar] i5 [particle=\(3\)],
        v1 -- [scalar] v2,
        v2 -- [scalar] i3 [particle=\(2\)],
        v2 -- [scalar] i4 [particle=\(1\)]
    }; &=\frac{ \im g^3}{\qty(p_1+p_2)^2}\qty[ \frac{\qty(m^2+\qty(p_3+p_4)^2+\qty(p_1+p_2)^2)}{\qty(p_3+p_4)^2}-\frac{\qty(p_3+p_4)^2}{\qty(p_3+p_4)^2}+\frac{\qty(m^2+\qty(p_3+p_5)^2+\qty(p_1+p_2)^2)}{\qty(p_3+p_5)^2}-\frac{\qty(p_3+p_5)^2}{\qty(p_3+p_5)^2}+\frac{\qty(m^2+\qty(p_5+p_4)^2+\qty(p_1+p_2)^2)}{\qty(p_5+p_4)^2}-\frac{\qty(p_5+p_4)^2}{\qty(p_5+p_4)^2}]\\
        &=\frac{ \im g^3\qty(m^2+\qty(p_1+p_2)^2)}{\qty(p_1+p_2)^2}\qty[ \frac{1}{\qty(p_3+p_4)^2}+\frac{1}{\qty(p_3+p_5)^2}+\frac{1}{\qty(p_5+p_4)^2}]
\end{align*}

Now we have to sum the contributions of $1$ being in the middle,

\begin{align*}
    \feynmandiagram [small] {
        i1 [particle = \(3\)]-- [scalar] v1,
        i2 [particle=\(4\)]-- [scalar] v1,
        v1 -- [scalar] v2,
        v2 -- [scalar] i5 [particle=\(1\)],
        v2 -- [scalar] v3,
        v3 -- [scalar] i3 [particle=\(5\)],
        v3 -- [scalar] i4 [particle=\(2\)]
    }; + \feynmandiagram [small] {
        i1 [particle = \(4\)]-- [scalar] v1,
        i2 [particle=\(1\)]-- [scalar] v1,
        v1 -- [scalar] i5 [particle=\(3\)],
        v1 -- [scalar] v2,
        v2 -- [scalar] i3 [particle=\(2\)],
        v2 -- [scalar] i4 [particle=\(5\)]
    };+\textnormal{ perm}
\end{align*}

Which will be,

\begin{align*}
    \frac{\im g^3}{\qty(p_2+p_3)^2}\frac{m^2+\qty(p_2+p_3)^2+\qty(p_4+p_5)^2}{\qty(p_4+p_5)^2}-\frac{3\im g^3}{\qty(p_2+p_3)^2}-\frac{3\im g^3}{\qty(p_4+p_5)^2}+\qty(3\leftrightarrow4,5)
\end{align*}

Summing all the contributions we have,

\begin{align*}
    =&\frac{ \im g^3m^2}{\qty(p_1+p_2)^2}\qty[ \frac{1}{\qty(p_3+p_4)^2}+\frac{1}{\qty(p_3+p_5)^2}+\frac{1}{\qty(p_5+p_4)^2}]+\qty(2\leftrightarrow3,4,5)\\
    &\quad\quad\quad+ \im g^3\qty[ \frac{1}{\qty(p_3+p_4)^2}+\frac{1}{\qty(p_3+p_5)^2}+\frac{1}{\qty(p_5+p_4)^2}+\frac{1}{\qty(p_2+p_4)^2}+\frac{1}{\qty(p_2+p_5)^2}+\frac{1}{\qty(p_5+p_4)^2}+\frac{1}{\qty(p_3+p_2)^2}+\frac{1}{\qty(p_3+p_5)^2}+\frac{1}{\qty(p_5+p_2)^2}+\frac{1}{\qty(p_3+p_4)^2}+\frac{1}{\qty(p_3+p_2)^2}+\frac{1}{\qty(p_2+p_4)^2}]\\
    &\quad\quad\quad+ \frac{\im g^3m^2}{\qty(p_2+p_3)^2\qty(p_4+p_5)^2}-\frac{2\im g^3}{\qty(p_2+p_3)^2}-\frac{2\im g^3}{\qty(p_4+p_5)^2}+\qty(3\leftrightarrow4,5)\\
    =&\frac{ \im g^3m^2}{\qty(p_1+p_2)^2}\qty[ \frac{1}{\qty(p_3+p_4)^2}+\frac{1}{\qty(p_3+p_5)^2}+\frac{1}{\qty(p_5+p_4)^2}]+\qty(2\leftrightarrow3,4,5)\\
    &\quad\quad\quad+ \frac{\im g^3m^2}{\qty(p_2+p_3)^2\qty(p_4+p_5)^2}+ \frac{\im g^3m^2}{\qty(p_2+p_4)^2\qty(p_3+p_5)^2}+ \frac{\im g^3m^2}{\qty(p_2+p_5)^2\qty(p_4+p_3)^2}
\end{align*}

By residue, any amplitude with just one massive external on-shell leg is zero. For two massive external on-shell legs, let's take as massive $1,2$,

\begin{align*}
    \feynmandiagram [small] {
        i1 [particle = \(3\)]-- [scalar] v1,
        i2 [particle=\(4\)]-- [scalar] v1,
        v1 -- [scalar] v2,
        v2 -- [scalar] i5 [particle=\(5\)],
        v2 -- [scalar] v3,
        v3 -- [scalar] i3 [particle=\(1\)],
        v3 -- [scalar] i4 [particle=\(2\)]
    }; 
    &= \im g^3\frac{\qty(\qty(p_1+p_2)^2-m^2)}{\qty(p_1+p_2)^2\qty(m^2+\qty(p_1+p_2)^2)}\frac{\qty(m^2+\qty(p_1+p_2)^2+\qty(p_3+p_4)^2)}{\qty(p_3+p_4)^2}
\end{align*}

Combining this graph with,

\begin{align*}
    \feynmandiagram [small] {
        i1 [particle = \(4\)]-- [scalar] v1,
        i2 [particle=\(5\)]-- [scalar] v1,
        v1 -- [scalar] i5 [particle=\(3\)],
        v1 -- [scalar] v2,
        v2 -- [scalar] i3 [particle=\(2\)],
        v2 -- [scalar] i4 [particle=\(1\)]
    }; 
    &=-3\im g^3\frac{\qty(p_1+p_2)^2-m^2}{\qty(p_1+p_2)^2\qty(m^2+\qty(p_1+p_2)^2)}
\end{align*}

so,

\begin{align*}
    \feynmandiagram [small] {
        i1 [particle = \(4\)]-- [scalar] v1 [blob],
        i2 [particle=\(5\)]-- [scalar] v1,
        v1 -- [scalar] i5 [particle=\(3\)],
        v1 -- [scalar] v2,
        v2 -- [scalar] i3 [particle=\(2\)],
        v2 -- [scalar] i4 [particle=\(1\)]
    };
    &=\im g^3\frac{\qty(\qty(p_1+p_2)^2-m^2)}{\qty(p_1+p_2)^2\qty(m^2+\qty(p_1+p_2)^2)}\qty[\frac{\qty(m^2+\qty(p_1+p_2)^2+\qty(p_3+p_4)^2)}{\qty(p_3+p_4)^2}-\frac{\qty(p_3+p_4)^2}{\qty(p_3+p_4)^2}+\qty(5\leftrightarrow3,4)]\\
    &=\im g^3\frac{\qty(\qty(p_1+p_2)^2-m^2)}{\qty(p_1+p_2)^2}\qty[\frac{1}{\qty(p_3+p_4)^2}+\frac{1}{\qty(p_5+p_4)^2}+\frac{1}{\qty(p_3+p_5)^2}]
\end{align*}

The other contributions are,

\begin{align*}
    \feynmandiagram [small] {
        i1 [particle = \(4\)]-- [scalar] v1 [blob],
        i2 [particle=\(5\)]-- [scalar] v1,
        v1 -- [scalar] i5 [particle=\(2\)],
        v1 -- [scalar] v2,
        v2 -- [scalar] i3 [particle=\(3\)],
        v2 -- [scalar] i4 [particle=\(1\)]
    };
    &=\frac{\im g^3\qty(p_1+p_3)^2}{\qty(m^2+\qty(p_1+p_3)^2)}\qty[\frac{1}{m^2+\qty(p_2+p_4)^2}+\frac{1}{m^2+\qty(p_2+p_5)^2}+\frac{1}{(p_4+p_5)^2}]
\end{align*}

Now we have to sum the contributions of $1$ being in the middle,

\begin{align*}
    \feynmandiagram [small] {
        i1 [particle = \(3\)]-- [scalar] v1,
        i2 [particle=\(4\)]-- [scalar] v1,
        v1 -- [scalar] v2,
        v2 -- [scalar] i5 [particle=\(1\)],
        v2 -- [scalar] v3,
        v3 -- [scalar] i3 [particle=\(5\)],
        v3 -- [scalar] i4 [particle=\(2\)]
    }; + \feynmandiagram [small] {
        i1 [particle = \(4\)]-- [scalar] v1,
        i2 [particle=\(1\)]-- [scalar] v1,
        v1 -- [scalar] i5 [particle=\(3\)],
        v1 -- [scalar] v2,
        v2 -- [scalar] i3 [particle=\(2\)],
        v2 -- [scalar] i4 [particle=\(5\)]
    };+\textnormal{ perm}
\end{align*}

which are,

\begin{align*}
    &=\im g^3\frac{\qty(p_2+p_3)^2+\qty(p_4+p_5)^2}{\qty(m^2+\qty(p_2+p_3)^2)\qty(p_4+p_5)^2}-\frac{3\im g^3}{m^2+\qty(p_2+p_3)^2}-\frac{3\im g^3}{\qty(p_4+p_5)^2}+\qty(3\leftrightarrow4,5)
\end{align*}

So, summing all the contributions,

\begin{align*}
    &=\im g^3\frac{\qty(\qty(p_1+p_2)^2-m^2)}{\qty(p_1+p_2)^2}\qty[\frac{1}{\qty(p_3+p_4)^2}+\frac{1}{\qty(p_5+p_4)^2}+\frac{1}{\qty(p_3+p_5)^2}]\\
    &\quad\quad\quad+\frac{\im g^3\qty(p_1+p_3)^2}{\qty(m^2+\qty(p_1+p_3)^2)}\qty[\frac{1}{m^2+\qty(p_2+p_4)^2}+\frac{1}{m^2+\qty(p_2+p_5)^2}+\frac{1}{(p_4+p_5)^2}]+\qty(3\leftrightarrow4,5)\\
    &\quad\quad\quad+\im g^3\frac{\qty(p_2+p_3)^2+\qty(p_4+p_5)^2}{\qty(m^2+\qty(p_2+p_3)^2)\qty(p_4+p_5)^2}-\frac{3\im g^3}{m^2+\qty(p_2+p_3)^2}-\frac{3\im g^3}{\qty(p_4+p_5)^2}+\qty(3\leftrightarrow4,5)\\
    &=-\im g^3\frac{m^2}{\qty(p_1+p_2)^2}\qty[\frac{1}{\qty(p_3+p_4)^2}+\frac{1}{\qty(p_5+p_4)^2}+\frac{1}{\qty(p_3+p_5)^2}]\\
    &\quad\quad\quad+\im g^3\qty[\frac{1}{\qty(p_3+p_4)^2}+\frac{1}{\qty(p_5+p_4)^2}+\frac{1}{\qty(p_3+p_5)^2}]\\
    &\quad\quad\quad+\im g^3\qty[\frac{1}{2p_2\cdot p_4}+\frac{1}{2p_2\cdot p_5}+\frac{1}{(p_4+p_5)^2}]+\qty(3\leftrightarrow4,5)\\
    &\quad\quad\quad-\im g^3\frac{m^2}{\qty(2p_1\cdot p_3)}\qty[\frac{1}{2p_2\cdot p_4}+\frac{1}{2p_2\cdot p_5}+\frac{1}{(p_4+p_5)^2}]+\qty(3\leftrightarrow4,5)\\
    &\quad\quad\quad+\im g^3\frac{-m^2+2p_2\cdot p_3+\qty(p_4+p_5)^2}{\qty(2p_2\cdot p_3)\qty(p_4+p_5)^2}-\frac{3\im g^3}{2p_2\cdot p_3}-\frac{3\im g^3}{\qty(p_4+p_5)^2}+\qty(3\leftrightarrow4,5)\\
    &=-\frac{\im g^3m^2}{\qty(p_1+p_2)^2}\qty[\frac{1}{\qty(p_3+p_4)^2}+\frac{1}{\qty(p_5+p_4)^2}+\frac{1}{\qty(p_3+p_5)^2}]\\
    &\quad\quad\quad+\im g^3\qty[\frac{1}{\qty(p_3+p_4)^2}+\frac{1}{\qty(p_5+p_4)^2}+\frac{1}{\qty(p_3+p_5)^2}]\\
    &\quad\quad\quad+\im g^3\qty[\frac{1}{2p_2\cdot p_4}+\frac{1}{2p_2\cdot p_5}+\frac{1}{(p_4+p_5)^2}+\frac{1}{2p_2\cdot p_3}+\frac{1}{2p_2\cdot p_5}+\frac{1}{(p_3+p_5)^2}+\frac{1}{2p_2\cdot p_4}+\frac{1}{2p_2\cdot p_3}+\frac{1}{(p_4+p_3)^2}]\\
    &\quad\quad\quad-\frac{\im g^3m^2}{\qty(2p_1\cdot p_3)}\qty[\frac{1}{2p_2\cdot p_4}+\frac{1}{2p_2\cdot p_5}+\frac{1}{(p_4+p_5)^2}]+\qty(3\leftrightarrow4,5)\\
    &\quad\quad\quad-\frac{\im g^3m^2}{\qty(2p_2\cdot p_3)\qty(p_4+p_5)^2}+\qty(3\leftrightarrow4,5)\\
    &\quad\quad\quad-\frac{2\im g^3}{2p_2\cdot p_3}-\frac{2\im g^3}{\qty(p_4+p_5)^2}-\frac{2\im g^3}{2p_2\cdot p_4}-\frac{2\im g^3}{\qty(p_3+p_5)^2}-\frac{2\im g^3}{2p_2\cdot p_5}-\frac{2\im g^3}{\qty(p_4+p_3)^2}\\
    &=-\frac{\im g^3m^2}{\qty(p_1+p_2)^2}\qty[\frac{1}{\qty(p_3+p_4)^2}+\frac{1}{\qty(p_5+p_4)^2}+\frac{1}{\qty(p_3+p_5)^2}]\\
    &\quad\quad\quad-\frac{\im g^3m^2}{\qty(2p_1\cdot p_3)}\qty[\frac{1}{2p_2\cdot p_4}+\frac{1}{2p_2\cdot p_5}+\frac{1}{(p_4+p_5)^2}]+\qty(3\leftrightarrow4,5)\\
    &\quad\quad\quad-\frac{\im g^3m^2}{\qty(2p_2\cdot p_3)\qty(p_4+p_5)^2}+\qty(3\leftrightarrow4,5)
\end{align*}

Is almost the same of the all massless, but with a different denominator in the $1,2$ channel. Now with three massive legs, being $3,4,5$,

\begin{align*}
    \feynmandiagram [small] {
        i1 [particle = \(3\)]-- [scalar] v1,
        i2 [particle=\(4\)]-- [scalar] v1,
        v1 -- [scalar] v2,
        v2 -- [scalar] i5 [particle=\(5\)],
        v2 -- [scalar] v3,
        v3 -- [scalar] i3 [particle=\(1\)],
        v3 -- [scalar] i4 [particle=\(2\)]
    }; 
    &= \im g^3\frac{\qty(\qty(p_1+p_2)^2+\qty(p_3+p_4)^2)\qty(-m^2+\qty(p_3+p_4)^2)}{\qty(p_1+p_2)^2\qty(p_3+p_4)^2\qty(m^2+\qty(p_3+p_4)^2)}
\end{align*}

With,

\begin{align*}
    \feynmandiagram [small] {
        i1 [particle = \(4\)]-- [scalar] v1,
        i2 [particle=\(5\)]-- [scalar] v1,
        v1 -- [scalar] i5 [particle=\(3\)],
        v1 -- [scalar] v2,
        v2 -- [scalar] i3 [particle=\(2\)],
        v2 -- [scalar] i4 [particle=\(1\)]
    }; &=\frac{-\im 3g^2\im g\qty(m^2+\qty(p_1+p_2)^2)}{\im\qty(p_1+p_2)^2\qty(\qty(p_1+p_2)^2+m^2)}=\frac{-3\im g^3}{\qty(p_1+p_2)^2}
\end{align*}

So,

\begin{align*}
    \feynmandiagram [small] {
        i1 [particle = \(4\)]-- [scalar] v1 [blob],
        i2 [particle=\(5\)]-- [scalar] v1,
        v1 -- [scalar] i5 [particle=\(3\)],
        v1 -- [scalar] v2,
        v2 -- [scalar] i3 [particle=\(2\)],
        v2 -- [scalar] i4 [particle=\(1\)]
    };
    &=\im g^3\frac{\qty(\qty(p_1+p_2)^2+\qty(p_3+p_4)^2)\qty(-m^2+\qty(p_3+p_4)^2)}{\qty(p_1+p_2)^2\qty(p_3+p_4)^2\qty(m^2+\qty(p_3+p_4)^2)}-\frac{3\im g^3}{\qty(p_1+p_2)^2}\\
    &=\frac{\im g^3}{\qty(p_1+p_2)^2}\qty[\frac{\qty(\qty(p_1+p_2)^2+\qty(p_3+p_4)^2)\qty(-m^2+\qty(p_3+p_4)^2)}{\qty(p_3+p_4)^2\qty(m^2+\qty(p_3+p_4)^2)}-\frac{\qty(p_3+p_4)^2\qty(m^2+\qty(p_3+p_4)^2)}{\qty(p_3+p_4)^2\qty(m^2+\qty(p_3+p_4)^2)}]+\qty(5\leftrightarrow3,4)\\
    &=\frac{\im g^3}{\qty(p_1+p_2)^2}\qty[\frac{\qty(p_1+p_2)^2\qty(-m^2+\qty(p_3+p_4)^2)-2m^2\qty(p_3+p_4)^2}{\qty(p_3+p_4)^2\qty(m^2+\qty(p_3+p_4)^2)}]+\qty(5\leftrightarrow3,4)
\end{align*}

%%%%%%%%%%%%%%%%%%%%%%%%%%%%%%%%%%%%%%%%%%%%%%%%%%%%%%%%%%%%%

\newpage

\printbibliography

\end{document}