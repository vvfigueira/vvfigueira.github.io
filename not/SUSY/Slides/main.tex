\documentclass{beamer}

\AtBeginDocument{\newcommand{\im}{\textnormal{i}}}

% Tema da apresentação
\usetheme{Madrid} % Outros temas: AnnArbor, Copenhagen, Dresden, Warsaw, etc.

\newcommand{\lastframetitle}{}

% Atualiza o título do último slide em cada frame
\addtobeamertemplate{frametitle}{}{\xdef\lastframetitle{\insertframetitle}}

\AtBeginSection[]{
    \addtocounter{framenumber}{-1}
  \begin{frame}{Sumário}
    \tableofcontents[currentsection] % Destaca a seção atual
  \end{frame}
    
}

\useoutertheme{split}
\setbeamertemplate{headline}{%
  \leavevmode%
  \hbox{%
    \begin{beamercolorbox}[wd=0.5\paperwidth,ht=2.5ex,dp=1ex,left]{section in head/foot}%
      \hspace{1em}\insertsectionhead
    \end{beamercolorbox}%
    \begin{beamercolorbox}[wd=0.5\paperwidth,ht=2.5ex,dp=1ex,left]{subsection in head/foot}%
      \hspace{1em}\ifx\insertframetitle\empty
      \lastframetitle % Mostra o último título válido
    \else
    \fi
    \end{beamercolorbox}%
  }
  \vskip0pt%
}

% Pacotes adicionais
\usepackage[utf8]{inputenc} % Codificação UTF-8
\usepackage[brazil]{babel} % Língua portuguesa do Brasil
\usepackage{graphicx} % Para inserir imagens
\usepackage{amsmath,amssymb} % Para matemática avançada
\usepackage{physics}
\usepackage{tensor}
\usepackage{bbm}
\usepackage{mathtools}

\usefonttheme{serif}

% Informações da apresentação
\title{SuSy}
\author{Vicente V. Figueira}
\institute{IF-USP}
\date{\today}

\begin{document}



% Slide de título
\begin{frame}
    \titlepage
\end{frame}

% Slide de sumário
\begin{frame}{Sumário}
    \tableofcontents
\end{frame}

% Seção: Introdução
\section{Motivação}
\begin{frame}{Perspectiva Histórica}
    \begin{itemize}
        \item 1964 --- $SU\qty(3)$ proposto por Gell-Mann e Ne'eman
        \item Simetria $SU\qty(6)$ aproximada no modelo não relativístico de quarks
        \item Tentativas de extender $SU\qty(3)$ para sabores e spin
    \end{itemize}
    Todas as tentativas de obter análogo à $SU\qty(6)$ para $SU\qty(3)$ falham. Por quê?
    \begin{itemize}
        \item Interesse crescente sobre propriedades da matrix $S$
        \item 1967 --- Coleman e Mandula catalogam todas as simetrias da matrix $S$
    \end{itemize}
\end{frame}

\begin{frame}
    Hipóteses utilizadas por Coleman e Mandula no seu Teorema:
    \begin{enumerate}
        \item Mecânica Quântica + Simetria de Poincaré
        \item Geradores levam 1PS $\rightarrow$ 1PS
        \item Ação sobre MPS como soma direta de 1PS
        \item Para uma dada escala de energia, o número de partículas com massa menor que esta é finito
        \item Reações 2 $\rightarrow$ 2 acontecem para quase todas energias
        \item Amplitudes de 2 $\rightarrow$ 2 são analíticas para quase todos ângulos e energias
    \end{enumerate}
    Qual a Álgebra de Lie mais geral dos geradores de simetria que satisfazem as hipóteses?
\end{frame}

\begin{frame}
    O resultado do Teorema é: A Álgebra de Lie mais geral dos geradores de simetria é,
    \[\mathfrak {iso}^+\qty(3,1)\oplus \mathfrak g\]
    No qual $\mathfrak g$ é uma soma direta de Álgebras de Lie compactas e semi-simples,
    \begin{align*}
            \comm{M^{\alpha\beta}}{M^{\mu\nu}}&=\im\qty(g^{\alpha\mu}M^{\beta\nu}-g^{\beta\mu}M^{\alpha\nu}+g^{\beta\nu}M^{\alpha\mu}-g^{\alpha\nu}M^{\beta\mu})\\
            \comm{P^\alpha}{M^{\mu\nu}}&=\im\qty(g^{\alpha\nu}P^\mu-g^{\alpha\mu}P^{\nu})\\
            \comm{P^\alpha}{P^\mu}&=\comm{P^\alpha}{Q^A}=\comm{M^{\alpha\beta}}{Q^A}=0\\
            \comm{Q^A}{Q^B}&=\im \tensor{f}{^A^B_C}Q^C
    \end{align*}
    Simetrias internas, $Q^A$, não podem transformar por uma representação não trivial de $\mathcal L_+^\uparrow$.
\end{frame}

\begin{frame}
    \begin{itemize}
        \item 1971 --- Gervais e Sakita descobrem uma simetria entre Bósons e Férmions na Teoria de Cordas
        \item 1974 --- Wess e Zumino constroem vários modelos com esta simetria em 4 dimensões,
        \begin{align*}
            \mathcal L&=-\partial_\mu \phi^\dagger\partial^\mu\phi+\im\psi^\dagger{\bar\sigma}^\mu\partial_\mu\psi\\
            \delta\phi&=\sqrt{2}\epsilon\psi\\
            \delta\psi&=-\im\sqrt 2\sigma^\mu\epsilon^\dagger\partial_\mu\phi
        \end{align*}
    \end{itemize}
    Como conciliar isto com o Teorema de Coleman-Mandula?
\end{frame}

\begin{frame}
    Isto de fato é uma brecha no Teorema, devido a teoria conter férmions,
    \begin{align*}
        \comm{\phi\qty(t,\vb x)}{\partial_0\phi^\dagger\qty(t,\vb y)}&=\im\delta^{\qty(3)}\qty(\vb x-\vb y)\\
        \acomm{\psi_a\qty(t,\vb x)}{\psi^\dagger_{\dot b}\qty(t,\vb y)}&=\sigma^0_{a\dot b}\delta^{\qty(3)}\qty(\vb x-\vb y)
    \end{align*}
    Além de comutadores, é necessário o uso de anti-comutadores. O erro então está nas hipóteses do teorema, uma 
    vez que Álgebras de Lie não possuem estrutura adequada para estes, é necessário o uso de \textbf{Álgebras de Lie Graduadas}.
    \begin{itemize}
        \item 1975 --- Haag, Lopuszanski e Sohnius estendem o Teorema de Coleman-Mandula para Álgebras de Lie Graduadas 
    \end{itemize}
\end{frame}

\begin{frame}
    Resultado do Teorema: Os únicos tipos de geradores de simetrias, não internas, que podem extender a Álgebra de Poincaré são os que pertencem 
    as representações $\qty(\frac12,0)$ e $\qty(0,\frac12)$ de $\mathcal L^\uparrow_+$, e possuem estatística fermiônica.
\end{frame}

\section{Álgebra de Super-Poincaré}
\begin{frame}{Álgebras de Lie Graduadas}
    \begin{itemize}
        \item Espaço Vetorial sobre $\mathbb R$
        \item Cada elemento $T^A\in\mathfrak g$ possui um peso $\eta\qty(T^A)=\eta_A=0,1$
        \item Produto de elementos tem peso,
            \[\eta\qty(T^A\cdots T^Z)=\sum\eta_i\ \qty(\textnormal{mod }2)\]
        \item Operação bilinear,
            \[\comm{T^A}{T^B}=T^AT^B-\qty(-)^{\eta_A\eta_B}T^BT^A=\im\tensor{f}{^A^B_C}T^C\]
            \item Operadores fermiônicos recebem peso 1 e operadores bosônicos recebem peso 0.
    \end{itemize}
\end{frame}

\begin{frame}{Super-Poincaré}
    O resultado do Teorema de Haag-Lopuszanski-Sohnius é de que a Álgebra ($\mathbb Z_2$-Graduada) mais geral é a de Poincaré,
    \begin{align*}
        \comm{M^{\alpha\beta}}{M^{\mu\nu}}&=\im\qty(g^{\alpha\mu}M^{\beta\nu}-g^{\beta\mu}M^{\alpha\nu}+g^{\beta\nu}M^{\alpha\mu}-g^{\alpha\nu}M^{\beta\mu})\\
        \comm{P^\alpha}{M^{\mu\nu}}&=\im\qty(g^{\alpha\nu}P^\mu-g^{\alpha\mu}P^{\nu})\\
        \comm{P^\alpha}{P^\mu}&=0
    \end{align*}
    Estendida por uma quantidade arbitrária, $\mathcal N$, de geradores das representações $\qty(\frac12,0)$ e 
    $\qty(0,\frac12)$,
    \[Q^A_a,\ Q^B_{\dot b};\ \ \ A,B=1,\cdots, \mathcal N\]
\end{frame}

\begin{frame}
    Covariância por Lorentz fixa,
    \begin{align*}
        \comm{Q^A_a}{M^{\mu\nu}}&=\tensor{{\sigma^{\mu\nu}}}{_a^b}Q^A_b\\
        \comm{{Q^{\dagger A\dot a}}}{M^{\mu\nu}}&=\tensor{{\bar\sigma}}{^\mu^\nu^{\dot a}_{\dot b}}{Q^{\dagger A\dot b}}\\
    \end{align*}
    Utilizando também identidades de Jacobi, o Teorema diz que sempre podemos diagonalizar a Álgebra como,
    \begin{align*}
        \comm{Q^A_a}{{Q^{\dagger B}_{\dot b}}}&=-2\delta^{AB}\tensor{\sigma}{_\mu_{a}_{\dot b}}P^\mu\\
        \comm{Q^A_a}{Q^B_b}&=Z^{AB}\epsilon_{ab}\\
        \comm{Q_a}{P^\mu}&=0
    \end{align*}
    $Z^{AB}$ são cargas centrais da Álgebra.
\end{frame}

\begin{frame}{Consequências}
    \begin{itemize}
        \item A energia é sempre positiva
    \end{itemize}
    \begin{align*}
        {\bar\sigma}^{0\dot b a}\comm{Q_a^A}{Q_{\dot b}^{\dagger A}}&=-2{\bar\sigma}^{0\dot b a}\sigma_{\mu a \dot b}P^\mu\\
        \comm{Q^A_1}{Q^{\dagger A}_{\dot 1}}+\comm{Q^A_2}{Q^{\dagger A}_{\dot 2}}&=4P^0
    \end{align*}
    \begin{align*}
        P^0&=\frac14\qty(Q^A_1Q^{\dagger A}_{\dot 1}+Q^A_2Q^{\dagger A}_{\dot 2}+Q^{\dagger A}_{\dot 1}Q^A_1+Q^{\dagger A}_{\dot 2}Q^A_2)\\
        \qty(\Psi,P^0\Psi)&=\frac14\qty(\Psi,Q^A_1Q^{\dagger A}_{\dot 1}\Psi)+\cdots\\
        \qty(\Psi,P^0\Psi)&=\frac14\norm{Q^{\dagger A}_{\dot 1}\Psi}^2+\cdots\geq 0
    \end{align*}
\end{frame}

\begin{frame}
    \begin{itemize}
        \item Super-Simetria relaciona partículas de mesma massa,
    \end{itemize}
    \begin{align*}
        \comm{Q_a^A}{P^\mu}=0\rightarrow\comm{Q_a^A}{P^\mu P_\mu}=0
    \end{align*}\pause
    \begin{itemize}
        \item Mas de spin diferente,
    \end{itemize}
    \begin{align*}
        \comm{Q_1^A}{M^{12}}=\comm{Q_1^A}{J^3}=\tensor{\sigma}{^1^2_1^b}Q_b^A&=\frac12Q_1^A,\ \ \ 
        \comm{Q_2^A}{J^3}=-\frac12Q_2^A\\
        \comm{Q_{\dot 1}^{\dagger A}}{J^3}&=-\frac12Q_{\dot 1}^{\dagger A},\ \ \ \comm{Q_{\dot 2}^{\dagger A}}{J^3}=\frac12Q_{\dot 2}^{\dagger A}
    \end{align*}
    \begin{itemize}
        \item $Q^A_1$ e $Q^{\dagger A}_{\dot 2}$ diminuem o spin em $\frac12$, enquanto $Q^A_2$ e $Q^{\dagger A}_{\dot 1}$ 
        aumentam o spin em $\frac12$.
    \end{itemize}
\end{frame}

\begin{frame}{Representações Massivas N=1}
    No referencial $P^\mu=\mqty(m&\vb0)$,
    \begin{align*}
        \comm{Q_a}{Q_{\dot b}}&=-2\sigma_{\mu a\dot b}P^\mu=2m\delta_{a\dot b}\\
        \comm{Q_a}{Q_{\dot b}}&=0
    \end{align*}
    Mesma álgebra de dois osciladores harmônicos fermiônicos desacoplados! Dado um vácuo Cliffordiano $\ket\Omega$ 
    --- é aniquilado por $Q_a$ --- de 
    spin $j$,
    \begin{align*}
        \ket\Omega\rightarrow Q_{\dot 2}^{\dagger}\ket\Omega&,Q_{\dot 1}^\dagger\ket\Omega\rightarrow Q_{\dot 1}^\dagger Q_{\dot 2}^\dagger\ket\Omega\\
        j\rightarrow\qty(j-\frac12)&\oplus\qty(j+\frac12)\rightarrow j
    \end{align*}
\end{frame}

\begin{frame}
    \begin{itemize}
        \item Multipleto Chiral Massivo $\qty(j=0)$,
    \end{itemize}
    \begin{align*}
        &\begin{rcases*}
            \ket\Omega\\
            Q_{\dot 1}^\dagger Q_{\dot 2}^\dagger\ket\Omega
        \end{rcases*}\textnormal{Spin }0\rightarrow \phi,\phi^\dagger
        &\begin{rcases*}
            Q_{\dot 1}^\dagger \ket\Omega\\
            Q_{\dot 2}^\dagger\ket\Omega
        \end{rcases*}\textnormal{Spin }\frac12\rightarrow \chi_a,\chi^\dagger_{\dot b}
    \end{align*}
    \begin{itemize}
        \item Multipleto Vetorial Massivo $\qty(j=\frac12)$,
    \end{itemize}
    \begin{align*}
        &\begin{rcases*}
            \ket\Omega\\
            Q_{\dot 1}^\dagger Q_{\dot 2}^\dagger\ket\Omega
        \end{rcases*}\textnormal{Spin }\frac12\rightarrow \chi_a,\chi^\dagger_{\dot b},\xi_a,\xi^\dagger_{\dot b}
        &\begin{rcases*}
            Q_{\dot 1}^\dagger \ket\Omega\\
            Q_{\dot 2}^\dagger\ket\Omega
        \end{rcases*}\textnormal{Spin }1,0\rightarrow A^\mu,\phi
    \end{align*}
\end{frame}

\begin{frame}{Representações não Massivas N=1}
    No referencial $P^\mu=\mqty(E&0&0&E)$,
    \begin{align*}
        \comm{Q_a}{Q_{\dot b}}&=-2\sigma_{\mu a\dot b}P^\mu=4E\delta_{a\dot b}\delta_{a2}\\
        \comm{Q_a}{Q_{\dot b}}&=0
    \end{align*}
    Álgebra de um oscilador harmônico fermiônico. Dado um vácuo Cliffordiano $\ket\Omega$ de 
    helicidade $h$,
    \begin{align*}
        \ket\Omega&\rightarrow Q_{\dot 2}^{\dagger}\ket\Omega&\\
        j&\rightarrow j-\frac12
    \end{align*}
\end{frame}

\begin{frame}
    \begin{itemize}
        \item Multipleto Chiral $\qty(h=\frac12)$,
    \end{itemize}
    \begin{align*}
        &\begin{rcases*}
            \ket\Omega
        \end{rcases*}\textnormal{Spin }\frac12\rightarrow \chi^\dagger_{\dot a}
        &\begin{rcases*}
            Q_{\dot 2}^\dagger\ket\Omega
        \end{rcases*}\textnormal{Spin }0\rightarrow \phi
    \end{align*}\pause
    Não é possível construir uma teoria Unitária e Lorentz covariante. É 
    necessário impor $\mathsf{CPT}$
    \begin{align*}
        \mathsf{CPT}\ket{h}\rightarrow\ket{-h}
    \end{align*}
\end{frame}

\begin{frame}
    \begin{itemize}
        \item Multipleto Chiral $\qty(h=\frac12)$,
    \end{itemize}
    \begin{align*}
        &\begin{rcases*}
            \ket\Omega\\
            \mathsf{CPT}\ket\Omega
        \end{rcases*}\textnormal{Spin }\frac12\rightarrow\chi_a, \chi^\dagger_{\dot a}
        &\begin{rcases*}
            Q_{\dot 2}^\dagger\ket\Omega\\
            \mathsf{CPT}Q_{\dot 2}^\dagger\ket\Omega
        \end{rcases*}\textnormal{Spin }0\rightarrow \phi,\phi^\dagger
    \end{align*}
    \begin{itemize}
        \item Multipleto Vetorial $\qty(h=1)$,
    \end{itemize}
    \begin{align*}
        &\begin{rcases*}
            \ket\Omega\\
            \mathsf{CPT}\ket\Omega
        \end{rcases*}\textnormal{Spin }1\rightarrow A^\mu
        &\begin{rcases*}
            Q_{\dot 2}^\dagger\ket\Omega\\
            \mathsf{CPT}Q_{\dot 2}^\dagger\ket\Omega
        \end{rcases*}\textnormal{Spin }\frac12\rightarrow \chi_a,\chi^\dagger_{\dot a}
    \end{align*}
    \begin{itemize}
        \item Multipleto Gravitacional $\qty(h=2)$,
    \end{itemize}
    \begin{align*}
        &\begin{rcases*}
            \ket\Omega\\
            \mathsf{CPT}\ket\Omega
        \end{rcases*}\textnormal{Spin }2\rightarrow A^\mu
        &\begin{rcases*}
            Q_{\dot 2}^\dagger\ket\Omega\\
            \mathsf{CPT}Q_{\dot 2}^\dagger\ket\Omega
        \end{rcases*}\textnormal{Spin }\frac32\rightarrow \chi^\mu_a,\chi^{\dagger\mu}_{\dot a}
    \end{align*}
\end{frame}

\begin{frame}
    \begin{itemize}
        \item Todos os multipletos possuem valores iguais de d.o.f de bósons e férmions. Coincidência? \pause Não.
    \end{itemize}
    \begin{align*}
        \mathbb Q=Q_1+Q^\dagger_{\dot 2}\rightarrow \comm{\mathbb Q}{\mathbb Q^\dagger}=4H\\
        \textnormal{se }H\neq0\Rightarrow\ket \Omega, \mathbb Q^\dagger\ket \Omega
    \end{align*}
    Se $H=0$, não há nada que nos previna de ter números arbitrários de estados bosônicos e fermiônicos,
    \begin{itemize}
        \item Índice de Witten, 
    \end{itemize}
    \[\Tr\qty[\qty(-1)^F]=n_{B,E=0}-n_{F,E=0}\]
    \begin{itemize}
        \item SSUSYB é possível $\Leftrightarrow\Tr\qty[\qty(-1)^F]=0$
    \end{itemize}
\end{frame}

\section{Super-Campos}

\begin{frame}{Super-Espaço}
    \begin{itemize}
        \item O espaço de Minkowski pode ser definido como,
    \end{itemize}
    \begin{align*}
        \mathbb R^{3,1}&=ISO^+\qty(3,1)/SO^+\qty(3,1)\\
        g\qty(\omega,a)&=\exp\qty(-\frac\im2\omega_{\mu\nu}M^{\mu\nu}-\im a_\mu P^\mu)
    \end{align*}
    \begin{itemize}\pause
        \item Definimos o Super-Espaço como,
    \end{itemize}    
    \begin{align*}
        \textnormal{Super-Espaço}&=\textnormal{Super-Poincaré}/SO^+\qty(3,1)\\
        g\qty(\omega,a,\theta,\theta^\dagger)&=\exp\qty(-\frac\im2\omega_{\mu\nu}M^{\mu\nu}-\im a_\mu P^\mu-\im\theta Q-\im\theta^\dagger Q^\dagger)
    \end{align*}
\end{frame}

\begin{frame}
    \begin{itemize}
        \item Um Super-Campo é uma função de $x,\theta,\theta^\dagger$ nos complexos,
        \[\Phi\qty(x,\theta,\theta^\dagger)\]
    \end{itemize}\pause
    \begin{itemize}
        \item $P^\mu$ gera translações em $x^\mu$,
    \end{itemize}
    \[\comm{\Phi\qty(x,\theta,\theta^\dagger)}{P^\mu}=-\im \partial^\mu\Phi\qty(x,\theta,\theta^\dagger)\]\pause
    \begin{itemize}
        \item $Q_a$ gera translações em $\theta_a$?
    \end{itemize}
    \[\comm{\Phi\qty(x,\theta,\theta^\dagger)}{Q_a}=-\im\mathcal Q_a\Phi\qty(x,\theta,\theta^\dagger);\ \ \ \mathcal Q_a\stackrel{?}{=}\pdv{}{\theta^a}\]
\end{frame}

\begin{frame}
    A representação da Álgebra no Super-Espaço é feita por,
    \begin{align*}
        {\mathcal {Q}}_a&=\partial_a+\im\sigma^\mu_{a\dot b}\theta^{\dagger\dot b}\partial_\mu\\
        {\mathcal {Q}}^\dagger_{\dot a}&=-\partial^\dagger_{\dot a}-\im\theta^b\sigma^\mu_{b\dot a}\partial_\mu\\
        \comm{\mathcal Q_a}{\mathcal Q_b}&=0\\
        \comm{\mathcal Q_a}{\mathcal Q^\dagger_{\dot b}}&=-2\im\sigma^\mu_{a\dot b}\partial_\mu
    \end{align*}
    Logo, a ação de uma transformação, infinitesimal, Super-Simétrica sobre um Super-Campo é,
    \[\Phi\rightarrow\Phi+\epsilon\mathcal Q\Phi+\epsilon^\dagger\mathcal Q^\dagger\Phi\]
\end{frame}

\begin{frame}
    Para obter uma Ação Super-Simétrica fazemos um análogo, dado uma combinação real de Super-Campos, $K=K^\dagger$, 
    integramos por todo o Super-Espaço,
    \begin{align*}
        S=\int\dd[4]{x}\dd[2]{\theta}\dd[2]{\theta^\dagger}K\qty(x,\theta,\theta^\dagger)
    \end{align*}
    Esta combinação é manifestamente hermitiana, Lorentz invariante e,
    \[\delta S=\int\dd[4]{x}\dd[2]{\theta}\dd[2]{\theta^\dagger}\delta K=\int\dd[4]{x}\dd[2]{\theta}\dd[2]{\theta^\dagger}\qty(\epsilon\mathcal Q K+\epsilon^\dagger\mathcal Q^\dagger K)=0\]
    Invariante Super-Simetricamente!
\end{frame}

\begin{frame}{Super-Campo Geral}
    Devido à natureza Grassmanniana de $\theta,\theta^\dagger$, o Super-Campo mais geral é,
    \begin{align*}
        \Phi\qty(x,\theta,\theta^\dagger)&=\phi\qty(x)+\theta\psi\qty(x)+\theta^\dagger\chi^\dagger\qty(x)
        +\theta\theta M\qty(x)+\theta^\dagger\theta^\dagger N\qty(x)\\ &\quad\quad+\theta\sigma^\mu\theta^\dagger v_\mu\qty(x)
        +\theta\theta\theta^\dagger\lambda^\dagger\qty(x)+\theta^\dagger\theta^\dagger\theta\xi\qty(x)+\theta\theta\theta^\dagger\theta^\dagger D\qty(x)
    \end{align*}
    Já apresenta todas as representações de nosso interesse! Porém até demais, como restringir?
\end{frame}

\begin{frame}{Super-Campo Chiral}
    Utilizando de nossa virtude da visão além do alcance, introduzimos,
    \begin{align*}
        \mathcal D_a&=\partial_a-\im\sigma^\mu_{a\dot b}\theta^{\dagger\dot b}\partial_\mu,\ \ \ \mathcal D^\dagger_{\dot a}
        =-\partial_{\dot a}+\im\theta^{b}\sigma^\mu_{b\dot a}\partial_\mu\\
        \comm{\mathcal D_a}{\mathcal D_b}&=0,\ \ \ 
        \comm{\mathcal D_a}{\mathcal D_{\dot b}}=2\im\sigma^\mu_{a\dot b}\partial_\mu\\
        \comm{\mathcal D_a}{\mathcal Q_b}&=\comm{\mathcal D_a}{\mathcal Q^\dagger_{\dot b}}=0
    \end{align*}
    Podemos então impor,
    \begin{itemize}
        \item Chiral $\mathcal D^\dagger_{\dot a}\Phi=0$
        \item Anti-Chiral $\mathcal D_a\Phi=0$
    \end{itemize}
\end{frame}

\begin{frame}
    \begin{itemize}
        \item $\mathcal D^\dagger_{\dot a}\Phi=0$
    \end{itemize}
    Pode ser facilmente resolvida nas variáveis,
    \begin{align*}
        y^\mu&=x^\mu-\im\theta^b\sigma^\mu_{b\dot c}\theta^{\dagger \dot c}\\
        \mathcal D^\dagger_{\dot a}\theta_b&=0,\ \ \ \mathcal D^\dagger_{\dot a}y^\mu=0
    \end{align*}
    Portanto,
    \begin{align*}
        \Phi\qty(x,\theta,\theta^\dagger)&=\Phi\qty(y,\theta)=\phi\qty(y)+\sqrt2\theta\psi\qty(y)+\theta\theta F\qty(y)\\
        &=\phi\qty(x)+\sqrt2\theta\psi\qty(x)+\theta\theta F\qty(x)-\im\theta\sigma^\mu\theta^\dagger\partial_\mu\phi\qty(x)\\
        &\quad\quad-\frac{\im}{\sqrt2}\theta\theta\theta^\dagger{\bar\sigma}^\mu\partial_\mu\psi\qty(x)+\frac14\theta\theta\theta^\dagger\theta^\dagger\partial^2\phi\qty(x)
    \end{align*}
\end{frame}

\begin{frame}
    \begin{itemize}
        \item $\mathcal D_{ a}\Phi^\dagger=0$
    \end{itemize}
    De fato, é apenas o conjugado de um chiral, pois,
    \begin{align*}
        y^{\dagger\mu}&=x^\mu+\im\theta^b\sigma^\mu_{b\dot c}\theta^{\dagger \dot c}\\
        \mathcal D_{ a}\theta^\dagger_b&=0,\ \ \ \mathcal D_{ a}y^{\dagger\mu}=0
    \end{align*}
    Portanto,
    \begin{align*}
        \Phi^\dagger\qty(x,\theta,\theta^\dagger)&=\phi^\dagger\qty(x)+\sqrt2\theta^\dagger\psi^\dagger\qty(x)+\theta^\dagger\theta^\dagger F^\dagger\qty(x)+\im\theta\sigma^\mu\theta^\dagger\partial_\mu\phi^\dagger\qty(x)\\
        &\quad\quad+\frac{\im}{\sqrt2}\theta^\dagger\theta^\dagger\partial_\mu\psi^\dagger\qty(x){\bar\sigma}^\mu\theta+\frac14\theta^\dagger\theta^\dagger\theta\theta\partial^2\phi^\dagger\qty(x)
    \end{align*}
    Que são muito próximos ao Multipleto Chiral! 
\end{frame}

\begin{frame}{Modelo de Wess-Zumino}
    A combinação $\Phi^\dagger\Phi$ é real. Logo, podemos definir uma teoria como,
    \[S=\int\dd[4]{x}\dd[2]{\theta}\dd[2]{\theta^\dagger}\Phi^\dagger\Phi\]
    A integral Grassmanniana seleciona o termo com coeficiente $\theta\theta\theta^\dagger\theta^\dagger$, um 
    rápido cálculo retorna,
    \begin{align*}
        \Phi^\dagger\Phi&\supset\theta\theta\theta^\dagger\theta^\dagger\qty[F^\dagger F-\partial_\mu\phi^\dagger\partial^\mu\phi+\im\psi^\dagger{\bar\sigma}^\mu\partial_\mu\psi]\\
        S&=\int\dd[4]{x}\qty[F^\dagger F-\partial_\mu\phi^\dagger\partial^\mu\phi+\im\psi^\dagger{\bar\sigma}^\mu\partial_\mu\psi]
    \end{align*}
    E interações?
\end{frame}

\begin{frame}
    Interações do tipo Yukawa não são geradas por termos como $f\qty(\Phi^\dagger\Phi)$. Vimos que,
    \[\Phi\qty(x,\theta,\theta^\dagger)=\Phi\qty(y,\theta)\]
    Podemos então introduzir um termo na ação como,
    \[S=\int\dd[4]{y}\dd[2]{\theta}\Phi\qty(y,\theta)\]
    Este termo é manifestamente Poincaré invariante, basta confirmar que 
    seja Super-Simetricamente invariante,
    \begin{align*}
        \delta S&=\int\dd[4]{y}\dd[2]{\theta}\qty(\epsilon\partial\Phi+\im\epsilon\sigma^\mu\theta^\dagger\partial_\mu\Phi
        -\epsilon^\dagger\partial^\dagger\Phi-\im\theta\sigma^\mu\epsilon^\dagger\partial_\mu\Phi)\\
        \delta S&=0
    \end{align*}
\end{frame}

\begin{frame}
    O mesmo continua valendo para qualquer função holomorfa, só precisamos garantir que seja real,
    \begin{align*}
        S&=\int\dd[4]{x}\dd[2]{\theta}W\qty(\Phi)+\int\dd[4]{x}\dd[2]{\theta^\dagger}W^\dagger\qty(\Phi^\dagger)\\
        W\qty(\Phi)&=W\qty(\phi+\sqrt2\theta\psi+\theta^2F)\\
        W\qty(\Phi)&=W\qty(\phi)+\sqrt2\pdv{W}{\phi}\theta\psi+\theta^2\qty(\pdv{W}{\phi}F-\frac12\pdv[2]{W}{\phi}\psi\psi)
    \end{align*}
    A integral Grassmanniana seleciona o termo com $\theta\theta$ ou $\theta^{\dagger}\theta^\dagger$. Por questões de 
    renormalizabilidade, a função mais geral que podemos tomar é,
    \[W\qty(\Phi)=\frac m2\Phi^2+\frac\lambda6\Phi^3\]
\end{frame}

\begin{frame}
    \begin{align*}
        S&=\int\dd[4]{x}\qty[\int\dd[2]{\theta}\dd[2]{\theta^\dagger}\Phi^\dagger\Phi+
        \int\dd[2]{\theta}W\qty(\Phi)+\int\dd[2]{\theta^\dagger}W^\dagger\qty(\Phi^\dagger)]\\
        S&=\int\dd[4]{x}\left[-\partial_\mu\phi^\dagger\partial^\mu\phi+\im\psi^\dagger{\bar\sigma}^\mu\partial_\mu\psi+F^\dagger F\right.\\
        &\quad\quad\quad\left.+\qty(F\pdv{W}{\phi}-\frac12\pdv[2]{W}{\phi}\psi\psi+\textnormal{h.c.})\right]\\
        S&=\int\dd[4]{x}\left[-\partial_\mu\phi^\dagger\partial^\mu\phi+\im\psi^\dagger{\bar\sigma}^\mu\partial_\mu\psi-\norm{\pdv{W}{\phi}}^2-\frac12\qty(\pdv[2]{W}{\phi}\psi\psi+\textnormal{h.c.})\right]
    \end{align*}
\end{frame}

\begin{frame}{Super-Campo Vetorial}
    Outra condição que poderíamos impor é,
    \[V=V^\dagger\]
    Que resulta em,
    \begin{align*}
        V&=C\qty(x)+\theta\chi\qty(x)+\theta^\dagger\chi\qty(x)+\theta\theta M\qty(x)+\theta^\dagger\theta^\dagger M^\dagger\qty(x)\\
        &\quad\quad+\theta\sigma^\mu\theta^\dagger v_\mu\qty(x)+\theta\theta\theta^\dagger\lambda^\dagger\qty(x)+\theta^\dagger\theta^\dagger\theta\lambda\qty(x)+\theta\theta\theta^\dagger\theta^\dagger D\qty(x)
    \end{align*}
    Como realizar uma transformação de Gauge? \pause Se $\Xi$ é Chiral,\[\im\qty(\Xi^\dagger-\Xi)\] É real/vetorial.
\end{frame}

\begin{frame}
    Seja,
    \begin{align*}
        \Xi&=B\qty(x)+\theta\xi\qty(x)+\theta\theta G\qty(x)-\im\theta\sigma^\mu\theta^\dagger\partial_\mu B\qty(x)\\
        &\quad\quad-\frac\im2\theta\theta\theta^\dagger{\bar\sigma}^\mu\partial_\mu\xi\qty(x)+\frac14\theta\theta\theta^\dagger\theta^\dagger\partial^2 B\qty(x)
    \end{align*}
    Um rápido cálculo retorna,
    \begin{align*}
        \im\qty(\Xi^\dagger-\Xi)&\supset-2\theta\sigma^\mu\theta^\dagger\partial_\mu\textnormal{Re}B\qty(x)
    \end{align*}
    E Principalmente,
    \[V\rightarrow V+\im\qty(\Xi^\dagger-\Xi)\Rightarrow v_\mu\rightarrow v_\mu -2\partial_\mu\textnormal{Re}B\]
    Definimos então como transformação de Gauge.
\end{frame}

\begin{frame}
    Por uma escolha conveniente de $B\qty(x),\xi\qty(x)$ e $G\qty(x)$ é 
    sempre possível escolher o \textbf{Gauge de Wess-Zumino},
    \begin{align*}
        V&=\theta\sigma^\mu\theta^\dagger v_\mu\qty(x)+\theta\theta\theta^\dagger\lambda^\dagger\qty(x)
        +\theta^\dagger\theta^\dagger\theta\lambda\qty(x)+\frac12\theta\theta\theta^\dagger\theta^\dagger D\qty(x)
    \end{align*}
    \begin{itemize}
        \item Como escrever um termo cinético?
    \end{itemize}

    \pause\[\mathcal D^\dagger_{\dot a}\mathcal D_b V\rightarrow\mathcal D^\dagger_{\dot a}\mathcal D_b\qty(V+\im\qty(\Xi^\dagger-\Xi))\stackrel{?}{=}\mathcal D^\dagger_{\dot a}\mathcal D_bV\]

    \pause\[\mathcal D^\dagger_{\dot a}\mathcal D^{\dagger\dot a}\mathcal D_b V\rightarrow\mathcal D^\dagger_{\dot a}\mathcal D^{\dagger\dot a}\mathcal D_b\qty(V+\im\qty(\Xi^\dagger-\Xi))\stackrel{!}{=}\mathcal D^\dagger_{\dot a}\mathcal D^{\dagger\dot a}\mathcal D_bV\]
\end{frame}

\begin{frame}
    Definimos o Super-Campo,
    \[W_a=\frac14\mathcal D^\dagger_{\dot a}\mathcal D^{\dagger\dot a}\mathcal D_b V\]
    Que é de fato Chiral!
    \[\mathcal D^\dagger_{\dot b}W_a=0\]
    Podemos utilizar da mesma expansão em $y,\theta$ feita anteriormente para Campos Chirais,
    \[W_a=\lambda_a\qty(y)+\theta_a D\qty(y)-\tensor{{\sigma^{\mu\nu}}}{_a^b}\theta_bF_{\mu\nu}\qty(y)+\im\theta\theta\sigma^\mu_{a\dot b}\partial_\mu\lambda^{\dagger \dot b}\qty(y)\]
\end{frame}

\begin{frame}
    Finalmente podemos obter uma quantidade Lorentz invariante Chiral,
    \[W^aW_a\supset\theta\theta\qty[2\im\lambda\sigma^\mu\partial_\mu\lambda^\dagger-\frac12F^{\mu\nu}F_{\mu\nu}-\frac\im2{\star} F^{\mu\nu}F_{\mu\nu}+D^2]\]
    Utilizando do mesmo método para obtenção de interações para Super-Campos chirais,
    \begin{align*}
        S&=\int\dd[4]{x}\qty[\int\dd[2]{\theta}\frac14W^aW_a+\int\dd[2]{\theta^\dagger}\frac14W^\dagger_{\dot a}W^{\dagger\dot a}]\\
        S&=\int\dd[4]{x}\qty[-\frac14 F_{\mu\nu}F^{\mu\nu}+\im\lambda^\dagger{\bar\sigma}^\mu\partial_\mu\lambda+\frac12D^2]
    \end{align*}
\end{frame}

\begin{frame}{Calibrando uma Teoria Chiral}
    Sabemos que devemos alterar o termo cinético canônico,\[\Phi^\dagger\Phi\]
    Para ser invariante por uma transformação de calibre,
    \begin{align*}
        \Phi&\rightarrow\exp\qty(-2\im g\Xi)\Phi\\
        \Phi^\dagger&\rightarrow\Phi^\dagger\exp\qty(2\im g \Xi^\dagger)\\
        V&\rightarrow V+\im\qty(\Xi^\dagger-\Xi)
    \end{align*}
\end{frame}

\begin{frame}
    Como a transformação em $V$ é linear, e $\Xi$ está dentro da exponencial,
    \[\Phi^\dagger\exp\qty(-2gV)\Phi\]
    É uma opção para termo cinético invariante por calibre. No gauge de Wess-Zumino,
    \begin{align*}
        V&=\theta\sigma^\mu\theta^\dagger v_\mu\qty(x)+\theta\theta\theta^\dagger\lambda^\dagger\qty(x)
        +\theta^\dagger\theta^\dagger\theta\lambda\qty(x)+\frac12\theta\theta\theta^\dagger\theta^\dagger D\qty(x)\\
        V^2&=-\frac12\theta\theta\theta^\dagger\theta^\dagger v^\mu v_\mu\\
        V^3&=0
    \end{align*}
    Logo,
    \begin{align*}
        \exp\qty(-2gV)&=1-2g\theta\sigma^\mu\theta^\dagger v_\mu-2g\theta\theta\theta^\dagger\lambda^\dagger-2g\theta^\dagger\theta^\dagger\theta\lambda\\
        &\quad\quad-\theta\theta\theta^\dagger\theta^\dagger\qty(gD+g^2v^\mu v_\mu)
    \end{align*}
\end{frame}

\begin{frame}
    Como a integral Grassmanniana seleciona apenas os termos com $\theta\theta\theta^\dagger\theta^\dagger$. 
    Com um pouco de algebra obtemos,
    \begin{align*}
        \Phi^\dagger\exp\qty(-2gV)\Phi\supset\theta\theta\theta^\dagger\theta^\dagger&\left[-D_\mu\phi^\dagger D^\mu\phi+\im\psi^\dagger{\bar\sigma}^\mu D_\mu\psi+F^\dagger F\right.\\
        &\left.+\sqrt2g\psi^\dagger\lambda^\dagger\phi+\sqrt2g\phi^\dagger\lambda\psi-g\phi^\dagger\phi D\right]
    \end{align*}
    Aparecimento natural da derivada covariante,
    \[D_\mu=\partial_\mu-\im gv_\mu\]
\end{frame}

\section{Possíveis continuações}
\begin{frame}
    \begin{itemize}
        \item Teoremas de não-renormalização
        \item Teorias com $\mathcal N>1$
        \item Super-Conforme
    \end{itemize}
\end{frame}

\begin{frame}\addtocounter{framenumber}{-1}
    \centering{Muito Obrigado!}
\end{frame}

\addtocounter{framenumber}{2000}

\end{document}