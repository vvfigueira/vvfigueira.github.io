\section{Álgebra de Super-Poincaré}

Conforme mencionado, o Teorema provado por Haag-Łopuszański-Sohnius nos garante que a Álgebra de Poincaré \ref{poincare}, somente 
pode ser estendida de maneira não trivial se os geradores adicionais forem fermiônicos com índices de mão esquerda e direita. A 
princípio podemos ter uma quantidade arbitrária $\mathcal N$ linearmente independente\footnote{Esta hipótese está aqui por 
motivos profiláticos, se por acaso uma quantidade destes geradores não fosse linearmente independente poderíamos eliminar-los para 
obter uma coleção menor com independência linear. Também é necessária a hipótese que estes não aniquilem ao menos um estado físico 
do espaço de Hilbert, do contrário são apenas operadores nulos.} destes,
\begin{align*}
    Q_a^A, Q_{\dot a}^{\dagger A};\ \ \ A,B=1,\cdots,\mathcal N;\ \ \ a=1,2;\ \ \ \dot a =\dot 1,\dot 2
\end{align*}

E com a seguinte graduação,
\begin{align*}
    \eta\qty(Q_a^A)=\eta\qty(Q_{\dot a}^{\dagger A})=1,\ \ \ \eta\qty(P^\mu)=\eta\qty(M^{\mu\nu})=0
\end{align*}

O restante do conteúdo do Teorema fixa unicamente qual são as relações de comutação. Não iremos aqui derivar na íntegra todas 
as relações, mas comentar brevemente como covariância por Lorentz, unido das Identidades de Jacobi, são suficientes para se fixar 
todas as relações de comutação. A mais trivial é a das Super-Cargas com os geradores de 
momento angular e boost, pois como as Super-Cargas carregam índices spinoriais, suas leis de transformação por Lorentz são 
fixas por sua representação do grupo. Assim,
\begin{align*}
    \comm{Q_a^A}{M^{\mu\nu}}=\tensor{\sigma}{^\mu^\nu_a^b}Q^A_b,\ \ \ \comm{Q^{\dagger A\dot a}}{M^{\mu\nu}}=\tensor{{\bar\sigma}}{^\mu^\nu^{\dot a}_{\dot b}}Q^{\dagger A\dot b}\numberthis\label{qlorentz}
\end{align*}

Outra maneira de entender por que não é possível alterar estas relações é devido ao lado direito do comutador possuir graduação 
fermiônica, logo, deve apenas conter uma quantidade ímpar de $Q_a^A,Q_{\dot a}^{\dagger A}$, mas, como estes pertencem a álgebra, 
é obrigatório do lado direto do comutador ser linear nestes, assim, o único outro termo que poderia ser incluso no lado direito da 
primeira relação de comutação\footnote{Para a segunda relação de comutação basta tomar o adjunto da primeira. $Q_{\dot a}^{\dagger A}=\qty(Q_a^A)^{\dagger}$.} 
seria, \[\tensor{C}{^\mu^\nu_a^{\dot b}}Q^{\dagger A}_{\dot b}\]

Isto iria requerer a existência de uma quantidade invariante $\tensor{C}{^\mu^\nu_a^{\dot b}}$, mas pela decomposição de representações,
\[\qty(\frac12,\frac12)\otimes\qty(\frac12,\frac12)\otimes\qty(\frac12,0)\otimes\qty(0,\frac12)=\qty(\frac12,\frac12)\oplus\qty(\frac32,\frac12)\oplus\qty(\frac12,\frac32)\oplus\qty(\frac32,\frac32)\]
podemos observar que não há nenhuma decomposição na representação escalar, $\qty(0,0)$, portanto este objeto invariante não existe e as relações de comutação são 
fixadas como mencionado. A próxima relação de comutação é de $Q_a^A$ com $P^\mu$, a única combinação linear nos geradores, Lorentz 
covariante e com graduação correta é, \[\comm{Q_a^A}{P^\mu}=\sigma^\mu_{a\dot b}\tensor{Z}{^A_B}Q^{\dagger B\dot b}\] 

Para determinar $\tensor{Z}{^A_B}$ usamos das Identidades de Jacobi,
\begin{align*}
    \comm{\comm{Q_a^A}{P^\mu}}{P^\nu}+\comm{\comm{P^\nu}{Q_a^A}}{P^\mu}+\comm{\comm{P^\mu}{P^\nu}}{Q_a^A}&=0\\
    \sigma^\mu_{a\dot b}\tensor{Z}{^A_B}\comm{Q^{\dagger B\dot b}}{P^\nu}-\sigma^\nu_{a\dot b}\tensor{Z}{^A_B}\comm{Q^{\dagger B\dot b}}{P^\mu}&=0\\
    \sigma^\mu_{a\dot b}{\bar\sigma}^{\nu\dot b c}\tensor{Z}{^A_B}\tensor{{Z^\ast}}{^B_C}Q^C_c-\sigma^\nu_{a\dot b}{\bar\sigma}^{\mu\dot b c}\tensor{Z}{^A_B}\tensor{{Z^\ast}}{^B_C}Q^C_c&=0\\
    \tensor{Z}{^A_B}\tensor{{Z^\ast}}{^B_C}\sigma^{\mu\nu}Q^C&=0
\end{align*}

O único modo de se garantir que isto é verdade para todos $\mu,\nu, Q^C$ é tomando $\tensor{Z}{^A_B}=0$,\[\comm{Q_a^A}{P^\mu}=0\numberthis\label{qp}\]

A próxima relação de comutação que analisaremos é de $Q_a^A$ e $Q_b^B$, a forma mais geral do comutador é, 
\[\comm{Q_a^A}{Q_b^B}=Z^{AB}\epsilon_{ab}+\tensor{\sigma}{^\mu^\nu_a_b}M_{\mu\nu}X^{AB}\]

Aplicando a Identidade de Jacobi,
\begin{align*}
    \comm{\comm{Q_a^A}{Q_b^B}}{P^\mu}+\comm{\comm{P^\mu}{Q_a^A}}{Q_b^B}-\comm{\comm{Q_b^B}{P^\mu}}{Q_a^A}&=0\\
    \epsilon_{ab}\comm{Z^{AB}}{P^\mu}+\tensor{\sigma}{^\alpha^\beta_a_b}X^{AB}\comm{M_{\alpha\beta}}{P^\mu}&=0
\end{align*}

Aqui necessariamente $Z^{AB}$ deve ser proporcional a identidade, pois tem graduação zero e nenhum índice de Lorentz, não é possível construir tal objeto utilizando-se apenas 
de combinação lineares dos geradores, assim nossa condição é de que o segundo comutador deve ser zero! Condição que só pode ser satisfeita se $X^{AB}=0$, uma vez que $\comm{M^{\alpha\beta}}{P^\mu}\neq 0$. 
Assim\footnote{Conforme mencionado, $Z^{AB}$ são proporcionais a identidade, isto é, são cargas centrais.}, \[\comm{Q_a^A}{Q_b^B}=Z^{AB}\epsilon_{ab}\numberthis\label{qq}\]

A última relação que nos resta analisar é de $Q_a^A$ com $Q_{\dot b}^{\dagger B}$, a combinação mais geral é\footnote{O fator de $2$ é meramente convencional.},
\[\comm{Q_a^A}{Q_{\dot b}^{\dagger B}}=-2X^{AB}\sigma_{\mu a\dot b}P^\mu\]

Claramente aqui $X^{AB}$ é manifestamente hermitiano, iremos mostrar que é também positivo-definido. Para isto temos que invocar 
condições sobre os geradores das Super-Simetrias, que são linearmente independentes, e que exista ao menos um estado não aniquilado por estes, assim, considere,
\[\mathbb Q=y_Aw^aQ_a^A\]
Certamente então, para algum estado $\Psi$ que não é aniquilado por $\mathbb Q$,
\begin{align*}
    0<\qty(\Psi,\comm{\mathbb Q}{\mathbb Q^\dagger}\Psi)=-w^ay_Ay_B^\ast w^{\ast\dot b}2X^{AB}\sigma_{\mu a\dot b}\qty(\Psi,P^\mu\Psi)
\end{align*}

Isto é, a quantidade $y_AX^{AB}y^\ast_B$ é não nula para todos $y_A$, ou seja, ou $X^{AB}$ é positivo-definido, ou é negativo-definido. Para decidir entre estas duas 
possibilidades, tomemos o caso de um estado físico com $P^\mu\Psi=-mg^{\mu 0}\Psi$,
\begin{align*}
    0&<2y_AX^{AB}y_B^\ast w^aw^{\ast\dot b}\sigma_{\mu a\dot b}mg^{\mu 0}\qty(\Psi,\Psi)\\
    0&<2y_AX^{AB}y_B^\ast w^aw^{\ast\dot b}{\sigma}^0_{ a\dot b}\\
    0&<2y_AX^{AB}y_B^\ast\norm{\vb w}^2
\end{align*}

Isto é, se desejamos ter estados de energia positiva é necessário que $X^{AB}$ seja positivo-definido\footnote{Do ponto de vista da Álgebra 
não há nada que impeça de ter $X^{AB}$ negativo-definido, isto faria com que o operador momento \textit{físico} fosse na verdade $-P^\mu$.}. 
Sabendo disto e da hipótese que todos os $Q^A_a$ são linearmente independentes, a matriz $X^{AB}$ será não degenerada, assim, podemos, 
fazendo uma mudança de variáveis dos $Q^A_a$, diagonalizá-la de forma a obter,
\[\comm{Q^A_a}{Q_{\dot b}^{\dagger B}}=-2\delta^{AB}\sigma_{\mu a\dot b}P^\mu\numberthis\label{qqdagger}\]
As relações de comutação obtidas, \cref{poincare,qq,qp,qlorentz,qqdagger}, constituem a \textbf{Álgebra de Super-Poincaré}, e são o principal 
resultado do Teorema de Haag-Łopuszański-Sohnius.