\documentclass[twoside]{amsart}

\usepackage[brazilian]{babel}
\usepackage{csquotes}
\usepackage[style=numeric-comp, backend=bibtex]{biblatex}
\usepackage{amsmath}
\usepackage{amssymb}
\usepackage{bbm}
\usepackage{graphics}
\usepackage{mathtools}
\usepackage[hidelinks]{hyperref}
\usepackage{physics}
\usepackage{enumitem}
\usepackage{slashed}
\usepackage{tensor}
\usepackage[lmargin=0.5cm,rmargin=0.5cm, tmargin =1cm,bmargin =1cm]{geometry}
\usepackage{tensor}
\usepackage[brazilian]{cleveref}

\AtBeginDocument{\renewcommand*{\hbar}{{\mkern-1mu\mathchar'26\mkern-8mu\textnormal{h}}}}
\AtBeginDocument{\newcommand{\e}{\textnormal{e}}}
\AtBeginDocument{\newcommand{\im}{\textnormal{i}}}
\AtBeginDocument{\newcommand{\luz}{\textnormal{c}}}
\AtBeginDocument{\newcommand{\grav}{\textnormal{G}}}
\AtBeginDocument{\newcommand{\kb}{{\textnormal{k}_{\textnormal{B}}}}}
\newcommand{\Dd}[1]{\mathcal D #1}
\newcommand{\Det}[1]{\textup{Det} #1}
\newcommand{\sgn}[1]{\mbox{sgn}\qty(#1)}
\newcommand{\cqd}{\hfill$\blacksquare$}
\newcommand{\dbar}{\mbox{\dj}}

\numberwithin{equation}{section}

\newtheorem{teo}{Teorema}[section]
\newtheorem{defi}{Definição}[section]
\newtheorem{lem}{Lema}[section]
\newtheorem{hip}{Hipótese}[subsection]

\pagestyle{plain}

%\AddToHook{cmd/section/before}{\clearpage}

\addbibresource{refs.bib}

\newtheorem{teorema}{Teorema}[section]
\newtheorem{definicao}{Definição}[section]
\newtheorem{lema}{Lema}[section]
\newtheorem{hipotese}{Hipótese}[section]
\newtheorem{postulado}{Postulado}[section]

\newcommand{\thistheoremname}{}
\newtheorem*{genericthm}{\thistheoremname}
\newenvironment{namedthm}[1]
  {\renewcommand{\thistheoremname}{#1}
   \begin{genericthm}}
  {\end{genericthm}}


  \newcommand\numberthis{\addtocounter{equation}{1}\tag{\theequation}}
  
\title{
Teorias de Campos Super-Simétricas
}
\author{
  Vicente V. Figueira
       }
\date{\today}

\begin{document}

\maketitle

%\tableofcontents

%%%%%%%%%%%%%%%%%%%%%%%%%%%%%%%%%%%%%%%%%%%%%%%%%%%%%%%%%%%%%

\nocite{*}

\section{Introduction/Motivation}
\label{sec:intro}

The Bosonic String Theory (BST) is known to achieve several desirable properties which up to present date 
haven't been done in usual Quantum Field Theory, the most prominent one is it being a perturbatively 
renormalizable theory which contains in its spectrum a massless spin-2 particle, this 
perturbative computation of amplitudes in BST is almost only possible 
due to the heavy simplifications the 
anomaly free gauge group Diff$\qty(M)\times$Weyl allows\cite{polchinski:vol1}. This means, as in the path integral we're 
integrating over metrics, the gauge redundancies permits us to forget about the metrics and to integrate 
over only the different kinds of topologies of two dimensional manifolds, so that in a generic string 
scattering situation, what would be a non-compact generic two dimensional manifold turns into a 
compact two dimensional manifold --- a choice over the equivalence class created by the gauge group: 
sphere, torus, ... ---, and what was the asymptotic states --- the \textit{non-compact part} of the original 
manifold --- turns into \textit{punctures} in the new compact two dimensional manifold. The advantages is, 
this process is nicely described by complex coordinates in the two dimensional (real) manifold, where the 
gauge transformations amounts to holomorphic change of complex coordinates, and the study of such objects, 
complex coordinates in two dimensional (real) manifolds, or better, one dimensional complex manifolds, 
has already lots of years of development in mathematics which we can borrow, these are called Riemann 
Surfaces\footnote{There is actually a distinction of a Riemann Surface and a two dimensional (real) manifold, every 
Riemann Surface is a two dimensional (real) manifold, but the converse is not true.} (RS).

Despite being a astonishing success in some points, BST still fails, at least perturbatively, to give 
any room to accommodate the particle zoo present at our world, principally, there are no means of 
introducing fermions in the target space theory, this, among other reasons, is the motif of pursuing other 
types of theories. A natural guess to overcome the fermion problem is to introduce world-sheet fermions $\psi^\mu$\cite{polchinski:vol2,witten:vol1}\footnote{We'll ignore multiplicative 
factors and set $\alpha'=2$ which can be restored by dimensional analysis.}, 
\begin{align}
    S\sim\int\limits_M\dd[2]{z}\qty(\partial X^\mu\bar\partial X_\mu+\psi^\mu\bar\partial\psi_{\mu}+{\tilde\psi}^\mu\partial{\tilde\psi}_{\mu}+\textnormal{ghosts})\label{action:sst1}
\end{align}
which under quantization gives an analogous problem with the one present in BST\footnote{We're using the graded commutator notation.},
\begin{align*}
    \comm{X^\mu\qty(\tau,\sigma)}{\dot X^\nu\qty(\tau,\sigma')}&=\im\pi\eta^{\mu\nu}\delta\qty(\sigma-\sigma')\\
    \comm{\psi^\mu\qty(\tau,\sigma)}{\psi^\nu\qty(\tau,\sigma')}&=\comm{{\tilde\psi}^\mu\qty(\tau,\sigma)}{{\tilde\psi}^\nu\qty(\tau,\sigma')}=\pi\eta^{\mu\nu}\delta\qty(\sigma-\sigma')
\end{align*}
that is, time-like fields $X^0,\psi^0,{\tilde\psi}^0$ have wrong sign commutator, which implies they will create ghost states in 
the theory, the resolution in BST is to use the gauge group --- a.k.a. the Virasoro constrains ---, to remove 
these non-physical states, but here, the best we could do is to use again the Virasoro constrains to get rid of the 
bosonic wrong sign states, and we would still had the fermionic wrong sign states. Here the only possible resolution 
is to find an other gauge redundancy of this theory, such that we can use it to eliminate the non-physical states. 
Luckily, this new action provides a possible candidate of gauge redundancy, as it has a $\mathcal N=1$ global supersymmetry (SUSY),\begin{subequations}
\begin{align}
    \delta_\epsilon X^\mu&=-\epsilon\psi^\mu-\epsilon^\ast{\tilde\psi^\mu}\\
    \delta_\epsilon\psi^\mu&=\epsilon\partial X^\mu,\ \ \ \delta_\epsilon{\tilde\psi}^\mu=\epsilon^\ast\bar\partial X^\mu
\end{align}\label{susy:pol1}\end{subequations}
Sadly enough, this supersymmetry algebra only closes on-shell and is global instead of local, despite this, 
one by one these issues can be unveiled. The uplift from a global symmetry to a local redundancy can be 
done by means of introducing a new field in the action, the world-sheet gravitino, and the promotion of the 
algebra closing off-shell can also be addressed by the inclusion of an auxiliary field in the action. Both 
these constructions are essential to Superstring Theory (SST), and the resulting theory enjoys a superconformal gauge 
group, which is given by our familiar super Virasoro algebra\footnote{With the inclusion of ghosts.},
\begin{align*}
    \begin{cases}
        T\qty(z)&\sim\partial X\partial X+\cdots\\
        G\qty(z)&\sim\psi\partial X+\cdots
    \end{cases}\Rightarrow
    \begin{cases}
    T\qty(z)T\qty(w)&\sim\frac{2T\qty(w)}{\qty(z-w)^2}+\frac{\partial T\qty(w)}{z-w}\\
    T\qty(z)G\qty(w)&\sim\frac32\frac{G\qty(w)}{\qty(z-w)^2}+\frac{\partial G\qty(w)}{z-w}\\
    G\qty(z)G\qty(w)&\sim\frac{2T\qty(w)}{\qty(z-w)^2}\end{cases}\Rightarrow\begin{cases}
        \comm{L_m}{L_n}&=\qty(m-n)L_{m+n}\\
        \comm{G_r}{G_s}&=2L_{r+s}\\
        \comm{L_m}{G_r}&=\qty(\frac m2-r)G_{m+r}
    \end{cases}
\end{align*}
The downside here is: in going from a conformal theory --- which we could benefit from developments in RS ---, 
to a superconformal theory, there seems to be a loss of geometrical visualization --- as due to $G\qty(z)$ 
being fermionic is not clear how it's action on the coordinates $z$ should be interpreted --- that could affect our, before 
mentioned, \textit{ease} of computing scattering amplitudes. To maintain the geometric interpretation and 
the off-shell supersymmetry is the role of the Super Riemann Surfaces (SRS).

\section{Álgebra de Super-Poincaré}

Conforme mencionado, o Teorema provado por Haag-Łopuszański-Sohnius nos garante que a Álgebra de Poincaré \ref{poincare}, somente 
pode ser estendida de maneira não trivial se os geradores adicionais forem fermiônicos com índices de mão esquerda e direita. A 
princípio podemos ter uma quantidade arbitrária $\mathcal N$ linearmente independente\footnote{Esta hipótese está aqui por 
motivos profiláticos, se por acaso uma quantidade destes geradores não fosse linearmente independente poderíamos eliminar-los para 
obter uma coleção menor com independência linear. Também é necessária a hipótese que estes não aniquilem ao menos um estado físico 
do espaço de Hilbert, do contrário são apenas operadores nulos.} destes,
\begin{align*}
    Q_a^A, Q_{\dot a}^{\dagger A};\ \ \ A,B=1,\cdots,\mathcal N;\ \ \ a=1,2;\ \ \ \dot a =\dot 1,\dot 2
\end{align*}

E com a seguinte graduação,
\begin{align*}
    \eta\qty(Q_a^A)=\eta\qty(Q_{\dot a}^{\dagger A})=1,\ \ \ \eta\qty(P^\mu)=\eta\qty(M^{\mu\nu})=0
\end{align*}

O restante do conteúdo do Teorema fixa unicamente qual são as relações de comutação. Não iremos aqui derivar na íntegra todas 
as relações, mas comentar brevemente como covariância por Lorentz, unido das Identidades de Jacobi, são suficientes para se fixar 
todas as relações de comutação. A mais trivial é a das Super-Cargas com os geradores de 
momento angular e boost, pois como as Super-Cargas carregam índices spinoriais, suas leis de transformação por Lorentz são 
fixas por sua representação do grupo. Assim,
\begin{align*}
    \comm{Q_a^A}{M^{\mu\nu}}=\tensor{\sigma}{^\mu^\nu_a^b}Q^A_b,\ \ \ \comm{Q^{\dagger A\dot a}}{M^{\mu\nu}}=\tensor{{\bar\sigma}}{^\mu^\nu^{\dot a}_{\dot b}}Q^{\dagger A\dot b}\numberthis\label{qlorentz}
\end{align*}

Outra maneira de entender por que não é possível alterar estas relações é devido ao lado direito do comutador possuir graduação 
fermiônica, logo, deve apenas conter uma quantidade ímpar de $Q_a^A,Q_{\dot a}^{\dagger A}$, mas, como estes pertencem a álgebra, 
é obrigatório do lado direto do comutador ser linear nestes, assim, o único outro termo que poderia ser incluso no lado direito da 
primeira relação de comutação\footnote{Para a segunda relação de comutação basta tomar o adjunto da primeira. $Q_{\dot a}^{\dagger A}=\qty(Q_a^A)^{\dagger}$.} 
seria, \[\tensor{C}{^\mu^\nu_a^{\dot b}}Q^{\dagger A}_{\dot b}\]

Isto iria requerer a existência de uma quantidade invariante $\tensor{C}{^\mu^\nu_a^{\dot b}}$, mas pela decomposição de representações,
\[\qty(\frac12,\frac12)\otimes\qty(\frac12,\frac12)\otimes\qty(\frac12,0)\otimes\qty(0,\frac12)=\qty(\frac12,\frac12)\oplus\qty(\frac32,\frac12)\oplus\qty(\frac12,\frac32)\oplus\qty(\frac32,\frac32)\]
podemos observar que não há nenhuma decomposição na representação escalar, $\qty(0,0)$, portanto este objeto invariante não existe e as relações de comutação são 
fixadas como mencionado. A próxima relação de comutação é de $Q_a^A$ com $P^\mu$, a única combinação linear nos geradores, Lorentz 
covariante e com graduação correta é, \[\comm{Q_a^A}{P^\mu}=\sigma^\mu_{a\dot b}\tensor{Z}{^A_B}Q^{\dagger B\dot b}\] 

Para determinar $\tensor{Z}{^A_B}$ usamos das Identidades de Jacobi,
\begin{align*}
    \comm{\comm{Q_a^A}{P^\mu}}{P^\nu}+\comm{\comm{P^\nu}{Q_a^A}}{P^\mu}+\comm{\comm{P^\mu}{P^\nu}}{Q_a^A}&=0\\
    \sigma^\mu_{a\dot b}\tensor{Z}{^A_B}\comm{Q^{\dagger B\dot b}}{P^\nu}-\sigma^\nu_{a\dot b}\tensor{Z}{^A_B}\comm{Q^{\dagger B\dot b}}{P^\mu}&=0\\
    \sigma^\mu_{a\dot b}{\bar\sigma}^{\nu\dot b c}\tensor{Z}{^A_B}\tensor{{Z^\ast}}{^B_C}Q^C_c-\sigma^\nu_{a\dot b}{\bar\sigma}^{\mu\dot b c}\tensor{Z}{^A_B}\tensor{{Z^\ast}}{^B_C}Q^C_c&=0\\
    \tensor{Z}{^A_B}\tensor{{Z^\ast}}{^B_C}\sigma^{\mu\nu}Q^C&=0
\end{align*}

O único modo de se garantir que isto é verdade para todos $\mu,\nu, Q^C$ é tomando $\tensor{Z}{^A_B}=0$,\[\comm{Q_a^A}{P^\mu}=0\numberthis\label{qp}\]

A próxima relação de comutação que analisaremos é de $Q_a^A$ e $Q_b^B$, a forma mais geral do comutador é, 
\[\comm{Q_a^A}{Q_b^B}=Z^{AB}\epsilon_{ab}+\tensor{\sigma}{^\mu^\nu_a_b}M_{\mu\nu}X^{AB}\]

Aplicando a Identidade de Jacobi,
\begin{align*}
    \comm{\comm{Q_a^A}{Q_b^B}}{P^\mu}+\comm{\comm{P^\mu}{Q_a^A}}{Q_b^B}-\comm{\comm{Q_b^B}{P^\mu}}{Q_a^A}&=0\\
    \epsilon_{ab}\comm{Z^{AB}}{P^\mu}+\tensor{\sigma}{^\alpha^\beta_a_b}X^{AB}\comm{M_{\alpha\beta}}{P^\mu}&=0
\end{align*}

Aqui necessariamente $Z^{AB}$ deve ser proporcional a identidade, pois tem graduação zero e nenhum índice de Lorentz, não é possível construir tal objeto utilizando-se apenas 
de combinação lineares dos geradores, assim nossa condição é de que o segundo comutador deve ser zero! Condição que só pode ser satisfeita se $X^{AB}=0$, uma vez que $\comm{M^{\alpha\beta}}{P^\mu}\neq 0$. 
Assim\footnote{Conforme mencionado, $Z^{AB}$ são proporcionais a identidade, isto é, são cargas centrais.}, \[\comm{Q_a^A}{Q_b^B}=Z^{AB}\epsilon_{ab}\numberthis\label{qq}\]

A última relação que nos resta analisar é de $Q_a^A$ com $Q_{\dot b}^{\dagger B}$, a combinação mais geral é\footnote{O fator de $2$ é meramente convencional.},
\[\comm{Q_a^A}{Q_{\dot b}^{\dagger B}}=-2X^{AB}\sigma_{\mu a\dot b}P^\mu\]

Claramente aqui $X^{AB}$ é manifestamente hermitiano, iremos mostrar que é também positivo-definido. Para isto temos que invocar 
condições sobre os geradores das Super-Simetrias, que são linearmente independentes, e que exista ao menos um estado não aniquilado por estes, assim, considere,
\[\mathbb Q=y_Aw^aQ_a^A\]
Certamente então, para algum estado $\Psi$ que não é aniquilado por $\mathbb Q$,
\begin{align*}
    0<\qty(\Psi,\comm{\mathbb Q}{\mathbb Q^\dagger}\Psi)=-w^ay_Ay_B^\ast w^{\ast\dot b}2X^{AB}\sigma_{\mu a\dot b}\qty(\Psi,P^\mu\Psi)
\end{align*}

Isto é, a quantidade $y_AX^{AB}y^\ast_B$ é não nula para todos $y_A$, ou seja, ou $X^{AB}$ é positivo-definido, ou é negativo-definido. Para decidir entre estas duas 
possibilidades, tomemos o caso de um estado físico com $P^\mu\Psi=-mg^{\mu 0}\Psi$,
\begin{align*}
    0&<2y_AX^{AB}y_B^\ast w^aw^{\ast\dot b}\sigma_{\mu a\dot b}mg^{\mu 0}\qty(\Psi,\Psi)\\
    0&<2y_AX^{AB}y_B^\ast w^aw^{\ast\dot b}{\sigma}^0_{ a\dot b}\\
    0&<2y_AX^{AB}y_B^\ast\norm{\vb w}^2
\end{align*}

Isto é, se desejamos ter estados de energia positiva é necessário que $X^{AB}$ seja positivo-definido\footnote{Do ponto de vista da Álgebra 
não há nada que impeça de ter $X^{AB}$ negativo-definido, isto faria com que o operador momento \textit{físico} fosse na verdade $-P^\mu$.}. 
Sabendo disto e da hipótese que todos os $Q^A_a$ são linearmente independentes, a matriz $X^{AB}$ será não degenerada, assim, podemos, 
fazendo uma mudança de variáveis dos $Q^A_a$, diagonalizá-la de forma a obter,
\[\comm{Q^A_a}{Q_{\dot b}^{\dagger B}}=-2\delta^{AB}\sigma_{\mu a\dot b}P^\mu\numberthis\label{qqdagger}\]
As relações de comutação obtidas, \cref{poincare,qq,qp,qlorentz,qqdagger}, constituem a \textbf{Álgebra de Super-Poincaré}, e são o principal 
resultado do Teorema de Haag-Łopuszański-Sohnius.

\section{Consequências de Super-Poincaré}

Vamos discutir aqui alguns aspectos não triviais de teorias que satisfazem a Álgebra de Super-Poincaré.
\subsection{Positividade da Energia}

Na verdade isto não é uma consequência, e sim uma hipótese na derivação da relação de comutação 
da \cref{qqdagger}, porém, mostraremos aqui de uma maneira mais clara como podemos obter este fato\footnote{Aqui os índices $^A$ não estão sendo somados.}.
\begin{align*}
    \comm{Q^A_a}{Q^{\dagger A}_{\dot b}}&=-2\delta^{AA}\sigma_{\mu a\dot b}P^\mu,\ \textnormal{contração com }{\bar\sigma}^{\dot b a}_0\\
    {\bar\sigma}^{0\dot b a}\comm{Q^A_a}{Q^{\dagger A}_{\dot b}}&=-2{\bar\sigma}^{0\dot b a}\sigma_{\mu a\dot b}P^\mu,\ \Tr\qty[{\bar\sigma}^\nu\sigma_\mu]=-2\tensor{g}{^\nu_\mu}\\
    {\bar\sigma}^{0\dot b a}\comm{Q^A_a}{Q^{\dagger A}_{\dot b}}&=4P^0,\ {\bar\sigma}^{0\dot b a}=\delta^{a1}\delta^{\dot b\dot 1}+\delta^{a2}\delta^{\dot b\dot 2}\\
    P^0&=\frac14\qty(Q_1^AQ^{\dagger A}_{\dot 1}+Q^{\dagger A}_{\dot 1}Q_1+Q_2Q^{\dagger A}_{\dot 2}+Q^{\dagger A}_{\dot 2}Q_2)
\end{align*}
Tomemos agora um estado arbitrário $\Psi$ do espaço de Hilbert,
\begin{align*}
    \qty(\Psi,P^0\Psi)&=\frac14\qty[\qty(\Psi,Q_1^AQ^{\dagger A}_{\dot 1}\Psi)+\qty(\Psi,Q^{\dagger A}_{\dot 1}Q_1^A\Psi)+\qty(\Psi,Q_2^AQ^{\dagger A}_{\dot 2}\Psi)+\qty(\Psi,Q^{\dagger A}_{\dot 2}Q_2^A\Psi)]\\
    \qty(\Psi,P^0\Psi)&=\frac14\qty[\norm{Q^{\dagger A}_{\dot 1}\Psi}^2+\norm{Q_1^A\Psi}^2+\norm{Q^{\dagger A}_{\dot 2}\Psi}^2+\norm{Q_2^A\Psi}^2]\geq 0
\end{align*}
Isto por sí já é uma propriedade desejável de qualquer teoria física, e certamente impõe restrições fortes 
sobre o tipo de espectro de partículas. Por exemplo, uma teoria livre com apenas férmions possui energia do 
vácuo sendo menos infinito\footnote{Estamos aqui nos referindo ao Hamiltoniano sem o ordenamento normal, vale 
lembrar que a prescrição de ordenamento normal é arbitrária.}. 
\subsection{Índice de Witten}
A positividade da energia deduzida no item anterior nos leva a crer que há de fato alguma relação entre o 
número de estados bosônicos e estados fermiônicos em uma teoria que respeita a Álgebra de Super-Poincaré. 
Um modo de verificar isto é utilizando-se do \textbf{Operador de Número Fermiônico}, $\mathsf F$, que retorna 
0 quando atua num estado com estatística bosônica, e retorna 1 quando atua em um estado com estatística 
fermiônica,\[\mathsf F\Psi_{\textnormal{Boson}}=0,\ \ \ \mathsf F\Psi_{\textnormal{Fermion}}=\Psi_{\textnormal{Fermion}}\]
Com o qual definimos o \textbf{Operador de Estatística}, $\qty(-1)^{\mathsf F}$,
\[\qty(-1)^{\mathsf F}\Psi_{\textnormal{Boson}}=\Psi_{\textnormal{Boson}},\ \ \ \qty(-1)^{\mathsf F}\Psi_{\textnormal{Fermion}}=-\Psi_{\textnormal{Fermion}}\]
Como queremos saber a relação entre o número de estados bosônicos e fermiônicos vamos calcular, \[\Tr\qty[\qty(-1)^{\mathsf F}]\] 
Para realizar esse cálculo vamos escolher uma base 
do espaço de Hilbert que diagonaliza o operador de momentum, $\Psi_{q}^n$, no qual $P^\mu\Psi_{q}^n=q^\mu\Psi_{q}^n$, 
e $n$ simboliza quaisquer outros números quânticos da base. Assim podemos escrever\footnote{Aqui 
o símbolo $\ \ \mathclap{\displaystyle\int\limits_n}\mathclap{\textstyle\sum}\ \ $ é meramente uma soma formal sobre 
quaisquer que sejam os índices $n$, e qualquer que seja seu espectro, discreto ou contínuo.}$^{,}$\footnote{Estamos usando de que o espaço de Hilbert 
físico na base de momentum respeita as condições $P^0\geq 0$, derivada anteriormente, e também a condição $P_\mu P^\mu\leq 0$, positividade da massa.},
\begin{align*}
    \Tr\qty[\qty(-1)^{\mathsf F}P^0]=\ \ \ \ \ \ \mathclap{\displaystyle\int\limits_{q^2\leq 0,\ q^0\geq 0}}\mathclap{\textstyle\sum}\ \ \ \ \ \ \ \ \mathclap{\displaystyle\int\limits_n}\mathclap{\textstyle\sum}\ \ \qty(\Psi^n_q,\qty(-1)^{\mathsf F} P^0\Psi^n_q)&=\Tr\qty[\qty(-1)^{\mathsf F}\frac14{\bar\sigma}^{0 \dot b a}\comm{Q^A_a}{Q^{\dagger A}_{\dot b}}]\\
    \ \ \ \ \ \ \mathclap{\displaystyle\int\limits_{q^2\leq 0,\ q^0\geq 0}}\mathclap{\textstyle\sum}\ \ \ \ \ \ \ \ \mathclap{\displaystyle\int\limits_n}\mathclap{\textstyle\sum}\ \ \qty(\Psi^n_q,\qty(-1)^{\mathsf F} q^0\Psi^n_q)&=\frac14\Tr\qty[\qty(-1)^{\mathsf F}\qty(Q_1^AQ^{\dagger A}_{\dot 1}+Q^{\dagger A}_{\dot 1}Q_1+Q_2Q^{\dagger A}_{\dot 2}+Q^{\dagger A}_{\dot 2}Q_2)]\\
    \ \ \ \ \ \ \mathclap{\displaystyle\int\limits_{q^2\leq 0,\ q^0\geq 0}}\mathclap{\textstyle\sum}\ \ \ \ \ \ q^0\ \ \mathclap{\displaystyle\int\limits_n}\mathclap{\textstyle\sum}\ \ \qty(\Psi^n_q,\qty(-1)^{\mathsf F} \Psi^n_q)&=\frac14\Tr\qty[-Q_1^A\qty(-1)^{\mathsf F}Q^{\dagger A}_{\dot 1}+\qty(-1)^{\mathsf F}Q^{\dagger A}_{\dot 1}Q_1-Q_2\qty(-1)^{\mathsf F}Q^{\dagger A}_{\dot 2}+\qty(-1)^{\mathsf F}Q^{\dagger A}_{\dot 2}Q_2)]
\end{align*}
Aqui nos usamos o fato de que, $\comm{\qty(-1)^F}{Q^A_a}=\comm{\qty(-1)^F}{Q^{\dagger}_{\dot b}}=0$, isso é verdade 
necessariamente por causa que as super-cargas mudam a graduação de um estado quando atuam neste. Utilizando agora da 
propriedade cíclica do traço,
\begin{align*}
    \Tr\qty[\qty(-1)^{\mathsf F}P^0]=\ \ \ \ \ \ \mathclap{\displaystyle\int\limits_{q^2\leq 0,\ q^0\geq 0}}\mathclap{\textstyle\sum}\ \ \ \ \ \ q^0\ \ \mathclap{\displaystyle\int\limits_n}\mathclap{\textstyle\sum}\ \ \qty(\Psi^n_q,\qty(-1)^{\mathsf F} \Psi^n_q)&=\frac14\Tr\qty[-\qty(-1)^{\mathsf F}Q^{\dagger A}_{\dot 1}Q_1^A+\qty(-1)^{\mathsf F}Q^{\dagger A}_{\dot 1}Q_1-\qty(-1)^{\mathsf F}Q^{\dagger A}_{\dot 2}Q_2+\qty(-1)^{\mathsf F}Q^{\dagger A}_{\dot 2}Q_2)]\\
    \ \ \ \ \ \ \mathclap{\displaystyle\int\limits_{q^2\leq 0,\ q^0\geq 0}}\mathclap{\textstyle\sum}\ \ \ \ \ \ q^0\ \ \mathclap{\displaystyle\int\limits_n}\mathclap{\textstyle\sum}\ \ \qty(\Psi^n_q,\qty(-1)^{\mathsf F} \Psi^n_q)&=0
\end{align*}
Bem, para o caso de $q^0=0\rightarrow q^\mu=0$ --- que nada mais nada menos é o estado de vácuo --- a contribuição para a igualdade é trivial, logo a restrição que obtemos é,
\begin{align*}
    \ \ \ \ \ \ \mathclap{\displaystyle\int\limits_{q^2\leq 0,\ q^0> 0}}\mathclap{\textstyle\sum}\ \ \ \ \ \ q^0\ \ \mathclap{\displaystyle\int\limits_n}\mathclap{\textstyle\sum}\ \ \qty(\Psi^n_q,\qty(-1)^{\mathsf F} \Psi^n_q)&=0
\end{align*}
Como necessariamente $q^0>0$, o único modo da igualdade ser satisfeita é se,
\begin{align*}
    \ \ \mathclap{\displaystyle\int\limits_n}\mathclap{\textstyle\sum}\ \ \qty(\Psi^n_q,\qty(-1)^{\mathsf F} \Psi^n_q)&=0,\ \ \ \forall q,\ q^2\leq 0,\ q^0>0
\end{align*}
Este é um resultado impressionante, a interpretação é de que para um dado momentum não nulo existem o mesmo número de 
graus de liberdade bosônico e fermiônico, ou seja, teorias super-simétricas possuem o mesmo número de partículas bosônicas 
e fermiônicas. Somos fortemente tentados a concluir que, $\Tr\qty[\qty(-1)^{\mathsf F}]=0$, porém isso não é necessariamente 
verdade. Note que\footnote{Ao colocar um índice $n$ no estado de vácuo $\Psi_{\textnormal{vac}}^n$ estamos levando em conta possíveis degenerescências do vácuo.},
\begin{align*}
    \Tr\qty[\qty(-1)^{\mathsf F}]&=\ \ \ \ \ \ \mathclap{\displaystyle\int\limits_{q^2\leq 0,\ q^0\geq 0}}\mathclap{\textstyle\sum}\ \ \ \ \ \ \ \ \mathclap{\displaystyle\int\limits_n}\mathclap{\textstyle\sum}\ \ \qty(\Psi^n_q,\qty(-1)^{\mathsf F} \Psi^n_q)\\
    \Tr\qty[\qty(-1)^{\mathsf F}]&=\ \ \mathclap{\displaystyle\int\limits_n}\mathclap{\textstyle\sum}\ \ \qty(\Psi^n_{\textnormal{vac}},\qty(-1)^{\mathsf F}\Psi^n_{\textnormal{vac}})+\ \ \ \ \ \ \mathclap{\displaystyle\int\limits_{q^2\leq 0,\ q^0> 0}}\mathclap{\textstyle\sum}\ \ \ \ \ \ \ \ \mathclap{\displaystyle\int\limits_n}\mathclap{\textstyle\sum}\ \ \qty(\Psi^n_q,\qty(-1)^{\mathsf F} \Psi^n_q)\\
    \Tr\qty[\qty(-1)^{\mathsf F}]&=\ \ \mathclap{\displaystyle\int\limits_n}\mathclap{\textstyle\sum}\ \ \qty(\Psi^n_{\textnormal{vac}},\qty(-1)^{\mathsf F}\Psi^n_{\textnormal{vac}})=\#\qty(\textnormal{vácuos bosônicos})-\#\qty(\textnormal{vácuos fermiônicos})\numberthis\label{indice}
\end{align*}
A quantidade definida na \cref{indice} é chamada de \textbf{Índice de Witten}\cite{witten1982}, e ela nos diz que a única possibilidade de não ter um pareamento de um 
para um de estados bosônicos e fermiônicos é no vácuo. Além disso, vale mencionar que o Índice de Witten é um \textit{invariante topológico} 
da teoria analisada, isto quer dizer, ele é invariante por mudança de parâmetros na lagrangiana. Uma das principais utilidades é na \textit{Quebra 
espontânea de Super-Simetria}, quando o vácuo da teoria não é super-simétrico. Por exemplo, uma teoria com $\Tr\qty[\qty(-1)^{\mathsf F}]>0$, possui ao menos um vácuo bosônico que 
não possui parceiro fermiônico, assim,\[Q^A_a\Psi_{\textnormal{vac}}^n=Q^{\dagger B}_{\dot b}\Psi_{\textnormal{vac}}^n=0\] necessariamente, pois não há um vácuo fermiônico disponível para estar no lado 
direito da igualdade\footnote{Note que a situação $\Tr\qty[\qty(-1)^{\mathsf F}]>0$ possibilita a existência de inúmeros vácuos, porém, assim como é feito na 
análise de quebra espontânea de simetrias globais, alterações dos parâmetros da lagrangiana promovem certos vácuos --- ou combinações lineares deles --- a estados de partículas sem massa, 
no nosso caso, para $\Tr\qty[\qty(-1)^{\mathsf F}]>0$, esta promoção de \textit{vácuos falsos} para estados de partículas é feita sempre em pares, uma bosônica e outra fermiônica, até 
restarem apenas \textit{vácuos verdadeiros} bosônicos, nesta descrição de não quebra espontânea de super-simetria está claro o motivo de $Q^A_a\Psi^n_{\textnormal{vac}}=Q^{\dagger B}_{\dot b}\Psi_{\textnormal{vac}}^n=0$. Uma 
vez que todos os \textit{vácuos falsos} foram promovidos para estados de partículas ao se variar os parâmetros da lagrangiana, não resta nenhum \textit{vácuo verdadeiro} fermiônico para entrar no lado direito da igualdade.}--- uma vez que não há um número igual de vácuos bosônicos e fermiônicos ---, e assim o vácuo é super-simétrico. O caso de 
maior interesse físico\footnote{Até agora não há nenhuma evidência experimental da existência dos parceiros super-simétricos dos integrantes do Modelo Padrão, logo, 
ou a super-simetria é quebrada espontâneamente, ou é quebrada explicitamente. Nem por isso a super-simetria deixa de ser um assunto interessante a ser estudado, teorias super-simétricas 
fornecem uma grande gama de \textit{toy models} que extraem ao máximo os limites do que é possível uma Teoria Quântica de Campos fazer. Quanto a anomalia da super-simetria, não é possível haver anomalia de super-simetria, 
pois do contrário, seria necessário modificar a relação de comutação da \cref{qp}, o que já foi mostrado que é impossível, devido as identidades de Jacobi.} é $\Tr\qty[\qty(-1)^{\mathsf F}]=0$, neste caso, todos os vácuos 
bosônicos possuem parceiros fermiônicos, então é em princípio \textbf{possível}, mas não necessário, que\[Q^A_a\Psi^n_{\textnormal{vac bosônico}}\propto\Psi^m_{\textnormal{vac fermiônico}}\]De forma que o vácuo não seja super-simétrico\footnote{Na presença 
de quebra espontânea de super-simetria, ao contrário de quebra espontânea de simetrias globais, temos a presença de um \textit{férmion} de Nambu-Goldstone. Neste caso, como uma das direções do potencial é plana --- 
a direção do férmion sem massa que sinaliza a quebra espontânea de super-simetria ---, e a outra direção é não plana, possuímos uma disparidade nas energias dos estados, isto é, a energia do estado de vácuo é positiva não nula, do 
contrário da energia do vácuo super-simétrico que é zero. 
De fato, a energia do estado de vácuo tem o papel de parâmetro de ordem da quebra espontânea de super-simetria.}.

\section{Construção de Ações Super-Simétricas}
A construção de teorias que respeitem super-simetria é um trabalho não trivial, a princípio, se estivermos 
interessados em uma teoria livre, poderíamos obter uma teoria super-simétrica apenas incluindo o mesmo número de 
graus de liberdade fermiônico e bosônico, porém, é difícil de obter qual é a exata transformação entre os 
campos bosônicos e fermiônicos que corresponde a transformação de super-simetria, ou então, analogamente, obter 
uma corrente fermiônica conservada. Mesmo se em posse da virtude de visão além do alcance todos esses passos 
forem corretamente concluídos, obtemos uma teoria super-simétrica, mas esta não será \textbf{manifestamente} 
super-simétrica. Vamos aqui adotar o formalismo do \textbf{Super-Espaço}, no qual a super-simetria da Ação é 
manifesta, e a inclusão de interações nas teorias é direta. Infelizmente este formalismo funciona apenas para 
super-simetria mínima, $\mathcal N=1$, para a construção de ações super-simétricas com $\mathcal N=2,4,8$ é 
necessário utilizar-se de outros métodos\footnote{É possível construir o espectro de teorias com super-simetria estendida, $\mathcal N>1$, como produtos diretos de multipletos de super-simetria simples.}, portanto vamos nos limitar a analisar a super-simetria mínima.
\subsection{Formalismo de Super-Espaço}
Sabemos que o gerador das translações no espaço-tempo é o momentum $P^\mu$, isto é, uma translação pode ser vista como um 
elemento do grupo de Poincaré da forma\[g\qty(a)=\exp\qty(-\im a_\mu P^\mu)\] Vamos fazer um análogo para a 
Álgebra\footnote{Sabemos na verdade que estamos lidando com uma Álgebra de Lie $\mathbb Z_2$-graduada, ou, Super-Álgebra de Lie. Mas não iremos nos preocupar em explicitar estes nomes.} de Super-Poincaré, como esta contém os geradores $Q_a,Q^\dagger_{\dot b}$, necessariamente temos que 
o grupo\footnote{Super-Grupo, mas não iremos nos preocupar com isso.} de Super-Poincaré --- Aquele gerado pela Álgebra de Super-Poincaré ---, terá elementos da forma\footnote{Devido a estatística fermiônica 
dos geradores $Q^{\qty(\dagger)}$ é necessário que $\theta^{\qty(\dagger)}$ sejam variáveis grassmanianas.},\[g\qty(\theta,\theta^\dagger)=\exp\qty(-\im\theta Q-\im\theta^\dagger Q^\dagger)\]A ideia do formalismo de Super-Espaço é 
considerar que este elemento do grupo de Super-Poincaré simboliza uma \textit{translação} nos parâmetros $\theta^{\qty(\dagger)}$, assim, consideramos como 
pontos do Super-Espaço o conjunto de valores $\qty(x,\theta,\theta^\dagger)$. Outro modo mais formal de definir o Super-Espaço é 
como espaço quociente de um grupo com um sub-grupo normal, no caso de Minkowski,\[\mathbb R^{3,1}=ISO^+\qty(3,1)/SO^+\qty(3,1)\] Podemos 
definir analogamente também para o Super-Espaço\footnote{Em geral, o grupo de Super-Poincaré é denotado como $ISO^+\qty(3,1|\mathcal N)$.},\[\textnormal{Super-Espaço}=ISO^+\qty(3,1|1)/SO^+\qty(3,1)\] Isto é 
apenas um modo mais matemático de dizer que, assim como o espaço de Minkowski é gerado pelo conjunto de todas as 
translações, o Super-Espaço é gerado pelo conjunto de todas as \textit{translações} da forma,\[g\qty(x,\theta,\theta^\dagger)=\exp\qty(-\im x_\mu P^\mu-\im\theta Q-\im\theta^\dagger Q^\dagger)\] O principal 
motivo de se formular a teoria no Super-Espaço é, assim como existe uma representação infinito-dimensional para $P^\mu$ 
no espaço de Minkowski, $-\im \partial^\mu$, iremos obter uma representação infinito-dimensional para $Q^{\qty(\dagger)}$ em formato 
de operador diferencial sobre o Super-Espaço, isso facilita a obtenção das transformações super-simétricas nos campos da teoria.
\subsection{Super-Campos}Definimos como \textbf{Super-Campo} uma função sobre o Super-Espaço que toma valores em $\mathbb C$, o exemplo 
mais simples de tal função é a de um escalar,\[\Phi\qty(x,\theta,\theta^\dagger)\] A princípio poderíamos escolher 
qualquer tipo de representação por Lorentz para um Super-Campo, mas, como veremos, a não ser que estejamos interessados 
em construir teorias com partículas de spin $\frac32$ ou $2$, a representação escalar é suficiente. Assim como foi mencionado 
anteriormente, conhecemos a representação de $P^\mu$ nos Super-Campos,\[\comm{\Phi\qty(x,\theta,\theta^\dagger)}{P^\mu}=-\im\partial^\mu\Phi\qty(x,\theta,\theta^\dagger)\] O que torna crível a 
existência de um operador diferencial $\mathcal Q_a$ tal que,\[\comm{\Phi\qty(x,\theta,\theta^\dagger)}{Q_a}=-\im\mathcal Q_a\Phi\qty(x,\theta,\theta^\dagger)\] A tentativa mais direta é a de $\mathcal Q_a=\frac{\partial}{\partial\theta^a}=\partial_a$ e $\mathcal Q^\dagger_{\dot a}=-\frac{\partial}{\partial\theta^{\dagger\dot a}}=-\partial^\dagger_{\dot a}$\footnote{O 
sinal de menos aqui é devido as variáveis serem grassmanianas.}. Isto não funciona, pois a \cref{qqdagger} não é respeitada. Para fazer estes operadores 
uma representação da álgebra será necessário a inclusão de derivadas em $x^\mu$. A inclusão que de fato é representação da álgebra é,\[\mathcal Q_a=\partial_a+\im\sigma_{\mu a\dot c}\theta^{\dagger \dot c}\partial^\mu,\ \ \ \mathcal Q^\dagger_{\dot b}=-\partial^\dagger_{\dot b}-\im\theta^c\sigma_{\mu c\dot b}\partial^\mu\rightarrow \comm{\mathcal Q_a}{\mathcal Q^\dagger_{\dot b}}=-2\im\sigma_{\mu a\dot b}\partial^\mu\] A 
qual pode ser verificada de respeitar todas as relações de comutação e identidades de Jacobi. A vantagem de ter esta representação é que dada uma 
transformação infinitesimal super-simétrica, sabemos que os campos vao se transformar como,\[\Phi\rightarrow\Phi+\epsilon\mathcal Q\Phi+\epsilon^\dagger\mathcal Q^\dagger\Phi\] Sabemos então checar se uma 
determinada ação possui super-simetria.

Nossa abordagem para a construção de uma Ação Super-Simétrica é similar a construção de Ações invariantes por Poincaré. Tomamos uma função escalar e real, Lagrangiana, 
e integramos por todo o espaço de Minkowski, similarmente, iremos partir de um Super-Campo escalar e real, e integrá-lo por todo o Super-Espaço,\[K^\dagger\qty(x,\theta,\theta^\dagger)=K\qty(x,\theta,\theta^\dagger)\rightarrow S=\int\dd[4]{x}\dd[2]{\theta}\dd[2]{\theta^\dagger}K\qty(x,\theta,\theta^\dagger)\] Dessa 
forma, segue naturalmente que a Ação obtida é Super-Simetricamente invariante, pois, \[\delta S=\int\dd[4]{x}\dd[2]{\theta}\dd[2]{\theta^\dagger}\qty(\epsilon\mathcal QK+\epsilon^\dagger\mathcal Q^\dagger K)\] Como 
os $\mathcal Q^{\qty(\dagger)}$ são operadores diferenciais nas variáveis que estão sendo integradas sobre, a variação da Ação é um mero termo de borda, para o qual 
com devidas condições de contorno é zero. Portanto desde que obtemos um Super-Campo real e escalar teremos uma possível Ação, porém, 
assim como as super-cargas não preservam o spin, \cref{qlorentz}, um Super-Campo não descreve uma partícula de spin definido, de fato, um 
Super-Campo não descreve \textbf{uma} partícula\footnote{No sentido usual, uma partícula está associada a um auto-estado do operador massa $P_\mu P^\mu$ e do operador spin $W_\mu W^\mu$, $W_\mu=\frac12\epsilon_{\nu\mu\alpha\beta}P^\nu M^{\alpha\beta}$. Poderíamos definir uma super-partícula como descrita por um Super-Campo, mas o 
quanto desta definição é útil pode ser questionada, uma vez que o que é mensurável são o spin e a massa.}, mas sim uma coleção de partículas de spin diferente e de mesma massa\footnote{Segue da \cref{qp}, $\comm{Q_a}{P^\mu}=0\rightarrow \comm{Q_a}{P^\mu P_\mu}=0$.}. Isso 
pode ser visto se utilizar-mos das propriedades das variáveis grassmanianas\footnote{$\theta_a^{\qty(\dagger)}\theta_b^{\qty(\dagger)}=-\theta_b^{\qty(\dagger)}\theta_a^{\qty(\dagger)}$. Por completude vamos também relembrar a notação de $\theta\theta=\theta^a\theta_a$, $\theta^\dagger\theta^\dagger=\theta^\dagger_{\dot a}\theta^{\dagger\dot a}$} para expandir o Super-Campo 
como,\[\Phi\qty(x,\theta,\theta^\dagger)=\phi\qty(x)+\theta\psi\qty(x)+\theta^\dagger\chi^\dagger\qty(x)+\theta\theta M\qty(x)+\theta^\dagger\theta^\dagger N\qty(x)+\theta\sigma^\mu\theta^\dagger v_\mu\qty(x)+\theta\theta\theta^\dagger\lambda^\dagger\qty(x)+\theta^\dagger\theta^\dagger\theta\xi\qty(x)+\theta\theta\theta^\dagger\theta^\dagger D\qty(x)\] Um único 
Super-Campo possui 4 escalares complexos, 2 férmions de mão esquerda, 2 férmions de mão direita e um vetor complexo. Gostaríamos de reduzir esta quantidade de campos, logo, precisamos aplicar alguma restrição sobre estes, mas, 
de forma a não destruir a Super-Simetria. Um método manifesto de aplicar restrições é introduzindo uma \textit{derivada super-covariante}\footnote{O porque de se não aplicar uma condição como $\partial_\mu\Phi=0$ é porque, apesar de diminuir o número de graus de liberdade total, esta restrição não elimina explicitamente um dos campos 
da decomposição. Por exemplo, para eliminar completamente $N, D,\xi$ poderíamos multiplicar $\Phi$ por $\theta^{\dagger\dot a}$, porém, isto não é invariante super-simetricamente. É necessário obter um operador diferencial nos $\theta,\theta^\dagger$ que 
mantenha a Super-Simetria.}, $\mathcal D_a^\dagger$, tal que,\[\comm{\mathcal D_a}{\mathcal Q_b}=\comm{\mathcal D_a}{\mathcal Q^\dagger_{\dot b}}=0\] Pelas 
formas das relações de comutação esperamos que $\mathcal D_a$ seja similar a $\mathcal Q_a$, e de fato, a escolha,\[\mathcal D_a=\partial_a-\im\sigma_{\mu a\dot c}\theta^{\dagger \dot c}\partial^\mu,\ \ \ \mathcal D^\dagger_{\dot b}=-\partial^\dagger_{\dot b}+\im\theta^c\sigma_{\mu c\dot b}\partial^\mu\rightarrow \comm{\mathcal D_a}{\mathcal D^\dagger_{\dot b}}=2\im\sigma_{\mu a\dot b}\partial^\mu\] é a correta a ser tomada. 
Assim podemos impor uma condição como, \[\mathcal D_{\dot a}^\dagger\Phi=0,\ \ \ \textnormal{ou, }\mathcal D_a\Phi^\dagger=0\] E devido aos motivos mencionados anteriormente, 
essas condições continuam valendo após uma transformação super-simétrica. Super-Campos satisfazendo essas respectivas condições são denominados de \textbf{Chirais} e \textbf{Anti-Chirais}\footnote{Sempre 
vamos denotar um Super-Campo Chiral por uma letra grega maiúscula e um Anti-Chiral por uma letra grega maiúscula com $\dagger$.},
o motivo desses nomes será explicado posteriormente. Vamos então obter o que estas restrições impõem sobre o Super-Campo.
\subsubsection{Super-Campo Chiral} \label{secchiral}Notemos primeiramente que,
\begin{align*}
    \mathcal D^\dagger_{\dot a}\theta_b&=-\partial^\dagger_{\dot a}\theta_b+\im\theta^c\sigma_{\mu c\dot a}\partial^\mu\theta_b=0\\
    \mathcal D^\dagger_{\dot a}x_\nu&=-\partial^\dagger_{\dot a}x_\nu+\im\theta^c\sigma_{\mu c\dot a}\partial^\mu x_\nu=\im\theta^c\sigma_{\nu c\dot a}\\
    \mathcal D^\dagger_{\dot a}\theta^{\dagger\dot b}&=-\partial^\dagger_{\dot a}\theta^{\dagger\dot b}+\im\theta^c\sigma_{\mu c\dot a}\partial^\mu\theta^{\dagger \dot b}=-\tensor{\delta}{_{\dot a}^{\dot b}}
\end{align*}
Logo, se definirmos $y_\nu=x_\nu-\im\theta\sigma_\nu\theta^\dagger$, \[\mathcal D^\dagger_{\dot a}y_\nu=\mathcal D^\dagger_{\dot a}x_\nu+\im\theta^c\sigma_{\nu c\dot b}\mathcal D^\dagger_{\dot a}\theta^{\dagger\dot b}=\im\theta^c\sigma_{\nu c \dot a}-\im\theta^c\sigma_{\nu c \dot a}=0\] Portanto, 
qualquer função que dependa apenas de $y,\theta$ é naturalmente um Super-Campo Chiral, assim, podemos expandir naturalmente primeiro em $\theta$, para somente ao final expandir em $\theta^\dagger$, \begin{align*}
\Phi\qty(y,\theta)&=\phi\qty(y)+\sqrt 2\theta\psi\qty(y)+\theta\theta F\qty(y)\\
&=\phi\qty(x)+\sqrt2\theta\psi\qty(x)+\theta\theta F\qty(x)-\im\theta\sigma^\mu\theta^\dagger\partial_\mu\phi\qty(x)-\frac{\im}{\sqrt2}\theta\theta\theta^\dagger{\bar\sigma}^\mu\partial_\mu\psi\qty(x)+\frac14\theta\theta\theta^\dagger\theta^\dagger\partial^2\phi\qty(x)\numberthis\label{chiral}
\end{align*}
Com sucesso conseguimos eliminar graus de liberdade! Obtemos 2 campos complexos escalares e um campo fermiônico de mão esquerda\footnote{Esta é a razão deste Super-Campo ser chamado de Chiral, pois ele contêm um campo fermiônico de 
mão esquerda, naturalmente, o Super-Campo Anti-Chiral irá possuir um campo fermiônico de mão direita. Esta é a menor coleção de campos com a qual é possível se construir uma teoria Super-Simétrica não trivial, é necessário haver o mesmo número de graus 
de liberdade on-shell fermiônicos e bosônicos, embora não seja necessário que os números de graus de liberdade off-shell sejam iguais, esta é a única maneira de se 
obter uma lagrangiana manifestamente Super-Simétrica. Este será o papel desempenhado pelo campo $F\qty(x)$, como será visto a seguir.}. Porém, este campo não é real, e a restrição de ser real e ser Chiral resulta em um campo trivial $\Phi=0$, logo, 
não é possível construir uma ação real somente com campos chirais.
\subsubsection{Super-Campo Anti-Chiral} Seguindo o mesmo raciocínio anterior, podemos calcular, \[\mathcal D_a y^\dagger_\nu=0,\ \ \ \mathcal D_a\theta^{\dagger\dot b}=0\] O que indica novamente que qualquer função de somente $y^\dagger,\theta^\dagger$ é naturalmente um Super-Campo Anti-Chiral, porém, note que,\[\qty[\Phi\qty(y,\theta)]^\dagger=\Phi^\dagger\qty(y^\dagger,\theta^\dagger)\] Isto é, todo Super-Campo Anti-Chiral é o conjugado de um Super-Campo Chiral! Assim,
\begin{align*}
    \Phi^\dagger\qty(y^\dagger,\theta^\dagger)&=\phi^\dagger\qty(x)+\sqrt2\theta^\dagger\psi^\dagger\qty(x)+\theta^\dagger\theta^\dagger F^\dagger\qty(x)+\im\theta\sigma^\mu\theta^\dagger\partial_\mu\phi^\dagger\qty(x)+\frac{\im}{\sqrt2}\theta^\dagger\theta^\dagger\partial_\mu\psi^\dagger\qty(x){\bar\sigma}^\mu\theta+\frac14\theta^\dagger\theta^\dagger\theta\theta\partial^2\phi^\dagger\qty(x)\numberthis\label{achiral}
\end{align*}
Análogo ao multipleto chiral, o multipleto anti-chiral possui 2 campos escalares complexos e um campo fermiônico de mão direita. O fato de todo multipleto anti-chiral ser o conjugado de um chiral nos fornece naturalmente um termo para a Ação,\[S=\int\dd[4]{x}\dd[2]{\theta}\dd[2]{\theta^\dagger}\Phi^\dagger\Phi\] A integral grassmaniana é trivial, pois ela apenas 
seleciona o termo com coeficiente $\theta\theta\theta^\dagger\theta^\dagger$, vamos obtê-lo\footnote{Utilizando-se de identidades de Fierz e desprezando derivadas totais.}:
\begin{align*}
    \Phi^\dagger\Phi&\supset\theta\theta\theta^\dagger\theta^\dagger\frac14\phi^\dagger\partial^2\phi+\theta\theta\theta^\dagger\theta^\dagger\frac14\phi\partial^2\phi^\dagger-\im\theta^\dagger\psi^\dagger\theta\theta\theta^\dagger{\bar\sigma}^\mu\partial_\mu\psi+\im\theta\psi\theta^\dagger\theta^\dagger\partial_\mu\psi^\dagger{\bar\sigma}^\mu\theta+\theta\theta\theta^\dagger\theta^\dagger F^\dagger F+\theta\sigma^\mu\theta^\dagger\theta\sigma^\mu\theta^\dagger\partial_\mu\phi^\dagger\partial^\mu\phi\\
    \Phi^\dagger\Phi&\supset\theta\theta\theta^\dagger\theta^\dagger\qty[F^\dagger F-\partial_\mu\phi^\dagger\partial^\mu\phi+\im\psi^\dagger{\bar\sigma}^\mu\partial_\mu\psi]
\end{align*}
Assim nossa Ação manifestamente Super-Simétrica é,\[S=\int\dd[4]{x}\qty[F^\dagger F-\partial_\mu\phi^\dagger\partial^\mu\phi+\im\psi^\dagger{\bar\sigma}^\mu\partial_\mu\psi]\] Onde claramente podemos ver a presença de $F$ como um campo auxiliar, isto é, está presente 
somente para garantir a igualdade de graus de liberdade entre bosons e férmions off-shell --- 4 graus de liberdade bosônicos e 4 fermiônicos --- devido a não possuir termos cinéticos, on-shell apenas $\phi$ contribui e $\psi$ e $\psi^\dagger$ deixam de ser independentes --- 2 graus 
de liberdade bosônicos e 2 fermiônicos ---. Fora este fato, a teoria obtida aqui é incrivelmente simples, uma teoria livre de duas partículas, uma bosônica e outra fermiônica. Gostaríamos de introduzir interações, a princípio renormalizáveis, porém, como $\Phi^\dagger\Phi$ já possui dimensão de 
massa\footnote{Sabemos que $\qty[P^\mu]=+1,\qty[M^{\mu\nu}]=0$ e com a \cref{qp} podemos inferir que $\qty[Q^{\qty(\dagger)}]=+\frac12$, assim $\qty[\theta^{\qty(\dagger)}]=-\frac12$.} 2, e $\dd[4]{x}\dd[2]{\theta}\dd[2]{\theta^\dagger}$ apenas possui dimensão de massa 2, portanto não é possível introduzir termos da forma $\qty(\Phi^\dagger\Phi)^n$. O 
método para obter uma interação é utilizar-se da propriedade de um campo Chiral, $\Phi\qty(x,\theta,\theta^\dagger)=\Phi\qty(y,\theta)$, para integrar apenas em metade do Super-Espaço de forma como se $y$ fosse a coordenada do espaço-tempo, isto é,\[\int\dd[4]{y}\dd[2]{\theta}\Phi\qty(y,\theta)\] Podemos ir além, 
como qualquer função \textit{holomorfa} de um campo (Anti-)Chiral é ainda um campo (Anti-)Chiral, podemos introduzir uma interação como um termo\footnote{$W\qty(\Phi)$ aqui é uma função holomorfa qualquer, para garantir a realidade da ação é necessário adicionar o termo conjugado.},\[S_{\textnormal{int}}=\int\dd[4]{x}\dd[2]{\theta}W\qty(\Phi\qty(x,\theta))+\int\dd[4]{x}\dd[2]{\theta^\dagger}W^\dagger\qty(\Phi^\dagger\qty(x,\theta^\dagger))\] Sobre se este termo quebra ou não explicitamente a Super-Simetria, 
podemos calcular,\[\delta S_{\textnormal{int}}=\int\dd[4]{x}\dd[2]{\theta}\qty{\epsilon\mathcal Q W+\epsilon^\dagger\mathcal Q^\dagger W}+\textnormal{h.c.}=\int\dd[4]{x}\dd[2]{\theta}\qty{\epsilon^a\partial_a W-\im\theta\sigma^\mu\epsilon^\dagger\partial_\mu W}+\textnormal{h.c}=0\] A igualdade a zero é garantida pois todos os termos da variação da 
Ação são termos de borda, que são desprezados. Assim, obtermos um termo de interação Super-Simétrico, se requeremos que sejam renormalizáveis, ao menos por \textit{power-counting}, o \textit{Super-Potencial} mais geral é,\[W\qty(\Phi)=\frac m2\Phi^2+\frac{\lambda}{3!}\Phi^3\] O qual podemos ainda expandir em $\theta$,
\begin{align*}
    W\qty(\Phi)&=W\qty(\phi+\sqrt 2\theta\psi+\theta\theta F)\\
    W\qty(\Phi)&=W\qty(\phi)+\sqrt 2\pdv{W}{\phi}\theta\psi+\theta\theta\qty(\pdv {W}{\phi} F-\frac12\pdv[2]{W}{\phi}\psi\psi)
\end{align*}
Obtemos assim então o famoso modelo de Wess-Zumino! Vamos explicitar os termos da ação,
\begin{align*}
    S&=\int\dd[4]{x}\qty[\int\dd[2]{\theta}\dd[2]{\theta^\dagger}\Phi^\dagger\Phi+\int\dd[2]{\theta}W\qty(\Phi)+\int\dd[2]{\theta^\dagger}W^\dagger\qty(\Phi^\dagger)]\\
    &=\int\dd[4]{x}\qty[-\partial_\mu\phi^\dagger\partial^\mu\phi+\im\psi^\dagger{\bar\sigma}^\mu\partial_\mu\psi+F^\dagger F+F\pdv{W}{\phi}-\frac12\pdv[2]{W}{\phi}\psi\psi+F^\dagger\pdv{{W^\dagger}}{{\phi^\dagger}}-\frac12\pdv[2]{{W^\dagger}}{{\phi^\dagger}}\psi^\dagger\psi^\dagger]
\end{align*}
As equações de movimento para $F$ são triviais e podem ser resolvidas para obter,
\begin{align*}
    S&=\int\dd[4]{x}\qty[-\partial_\mu\phi^\dagger\partial^\mu\phi+\im\psi^\dagger{\bar\sigma}^\mu\partial_\mu\psi+\norm{\pdv{W}{\phi}}^2-\frac12\pdv[2]{W}{\phi}\psi\psi-\frac12\pdv[2]{{W^\dagger}}{{\phi^\dagger}}\psi^\dagger\psi^\dagger]\\
    S&=\int\dd[4]{x}\qty[-\partial_\mu\phi^\dagger\partial^\mu\phi+\im\psi^\dagger{\bar\sigma}^\mu\partial_\mu\psi+\norm{m\phi+\frac\lambda2\phi^2}^2-\frac12\qty(m+\lambda\phi)\psi\psi-\frac12\qty(m^\dagger+\lambda^\dagger\phi^\dagger)\psi^\dagger\psi^\dagger]\\
    S&=\int\dd[4]{x}\qty[-\partial_\mu\phi^\dagger\partial^\mu\phi+\im\psi^\dagger{\bar\sigma}^\mu\partial_\mu\psi+m^\dagger m\phi^\dagger\phi+\frac{m^\dagger\lambda}{2}\phi^\dagger\phi^2+\frac{m\lambda^\dagger}{2}\phi{\phi^\dagger}^2+\frac{\lambda^\dagger\lambda}{4}\qty(\phi^\dagger\phi)^2-\frac12\qty(m+\lambda\phi)\psi\psi-\frac12\qty(m^\dagger+\lambda^\dagger\phi^\dagger)\psi^\dagger\psi^\dagger]
\end{align*}
Agora podemos apreciar a beleza do formalismo de Super-Campos, onde todos esses termos estão contidos em $\Phi^\dagger\Phi$, e também podemos apreciar a não trivialidade dos termos de interação. 
Se desejamos construir uma teoria Super-Simétrica, não é possível escolher a belo prazer os termos de interação, há restrições não triviais sobre estes imposta pela Super-Simetria, 
assim, tentar construir um modelo interagente Super-Simétrico sem o formalismo de Super-Campos é uma tarefa de dificuldade colossal. Note que, a princípio apenas a nível árvore, o férmion e o bóson possuem 
massas iguais, $\norm{m}$, isto é requerido pelo Índice de Witten, e portanto é válido não perturbativamente também. 

Ao final, temos uma teoria, digamos, estritamente de matéria, como oposto a radiação, ou então, a portadores de força. Se a Super-Simetria tem algum valor à ser estudado, é necessário que esta consiga acomodar 
uma descrição de uma teoria de calibre, vamos então aproveitar este gancho para apresentar outra possível restrição nos Super-Campos, uma que aparenta ser mais simples, mas que como veremos, dará origem a 
campos vetoriais de calibre.
\subsubsection{Super-Campos Reais/Vetoriais} A restrição imposta para um Super-Campos Real é, ser Real, \[V\qty(x,\theta,\theta^\dagger)=V^\dagger\qty(x,\theta,\theta^\dagger)\] Vamos obter qual restrição isso implica utilizando a expansão usual geral,
\begin{align*}
    V\qty(x,\theta,\theta^\dagger)&=C\qty(x)+\theta\chi\qty(x)+\theta^\dagger\psi^\dagger\qty(x)+\theta\theta M\qty(x)+\theta^\dagger\theta^\dagger N\qty(x)+\theta\sigma^\mu\theta^\dagger v_\mu\qty(x)+\theta\theta\theta^\dagger\lambda^\dagger\qty(x)+\theta^\dagger\theta^\dagger\theta\xi\qty(x)+\theta\theta\theta^\dagger\theta^\dagger D\qty(x)\\
    V^\dagger\qty(x,\theta,\theta^\dagger)&=C^\dagger\qty(x)+\theta^\dagger\chi^\dagger\qty(x)+\theta\psi\qty(x)+\theta^\dagger\theta^\dagger M^\dagger\qty(x)+\theta\theta N^\dagger\qty(x)+\theta\sigma^\mu\theta^\dagger v^\dagger_\mu\qty(x)+\theta^\dagger\theta^\dagger\theta\lambda\qty(x)+\theta\theta\theta^\dagger\xi^\dagger\qty(x)+\theta\theta\theta^\dagger\theta^\dagger D^\dagger\qty(x)
\end{align*}
Portanto as restrições são, \[C\qty(x)=C^\dagger\qty(x),\ \ \psi\qty(x)=\chi\qty(x),\ \ N\qty(x)=M^\dagger\qty(x),\ \ v_\mu\qty(x)=v_\mu^\dagger\qty(x),\ \ \xi\qty(x)=\lambda\qty(x),\ \ D\qty(x)=D^\dagger\qty(x)\] Isto é, 
\[V\qty(x,\theta,\theta^\dagger)=C\qty(x)+\theta\psi\qty(x)+\theta^\dagger\psi^\dagger\qty(x)+\theta\theta M\qty(x)+\theta^\dagger\theta^\dagger M^\dagger\qty(x)+\theta\sigma^\mu\theta^\dagger v_\mu\qty(x)+\theta\theta\theta^\dagger\lambda^\dagger\qty(x)+\theta^\dagger\theta^\dagger\theta\lambda\qty(x)+\theta\theta\theta^\dagger\theta^\dagger D\qty(x)\] O 
conteúdo do multipleto real/vetorial é portanto, 2 campos reais escalares, 1 campo complexo escalar, 2 férmions de Weyl e um campo real vetorial. A presença de um campo vetorial sinaliza a possibilidade de 
descrever bósons de calibre, assim como veremos, parte dos escalares está associado a campos auxiliares, enquanto parte está associado a liberdade de calibre, o mesmo vale para os 2 férmions. Perceba que, é possível 
construir um campo real/vetorial apenas em posse de um campo chiral $\Xi$, pois a combinação $\im\qty(\Xi^\dagger-\Xi)$ é real! Assim, sendo \begin{align*}
    \Xi&=B+\theta\xi+\theta\theta G-\im\theta\sigma^\mu\theta^\dagger\partial_\mu B-\frac\im 2\theta\theta\theta^\dagger{\bar\sigma}\partial_\mu\xi+\frac14\theta\theta\theta^\dagger\theta^\dagger\partial^2B\\
    \im\qty(\Xi^\dagger-\Xi)&=2\textnormal{Im}B+\im\theta^\dagger\xi^\dagger-\im\theta\xi+\im\theta^\dagger\theta^\dagger G^\dagger-\im\theta\theta G-2\textnormal{Re}\theta\sigma^\mu\theta^\dagger\partial_\mu B-\frac12\theta^\dagger\theta^\dagger\partial_\mu\xi^\dagger{\bar\sigma}^\mu\theta-\frac12\theta\theta\theta^\dagger{\bar\sigma^\mu}\partial_\mu\xi+\theta\theta\theta^\dagger\theta^\dagger\frac12\textnormal{Im}\partial^2B
\end{align*}
Logo é sim possível definir uma transformação de calibre por, $V\rightarrow V+\im\qty(\Xi^\dagger-\Xi)$, que, além de outras mudanças, tem o efeito de, $v_\mu\rightarrow v_\mu-2\textnormal{Re}\partial_\mu B$. Como a transformação de calibre que estamos interessados 
envolve apenas a parte real do campo $B$, podemos utilizar dos restantes graus de liberdade para escolher um calibre que elimine a mencionada anteriormente liberdade de calibre contida nos campos escalar e de Weyl do Super-Campo Vetorial. Uma das 
escolhas mais úteis é o chamado \textbf{Calibre de Wess-Zumino}, caracterizado pela escolha de,\[\textnormal{Im}B=-\frac12 C,\ \ \xi=-\im\psi,\ \ G=-\im M\] No qual temos, 
\[V\qty(x,\theta,\theta^\dagger)=\theta\sigma^\mu\theta^\dagger v_\mu\qty(x)+\theta\theta\theta^\dagger\qty(\lambda^\dagger\qty(x)-\frac12{\bar\sigma}^\mu\partial_\mu\xi\qty(x))+\theta^\dagger\theta^\dagger\qty(\lambda\qty(x)-\frac12\partial_\mu\xi^\dagger\qty(x){\bar\sigma}^\mu)\theta+\theta\theta\theta^\dagger\theta^\dagger\qty( D\qty(x)+\frac12\textnormal{Im}\partial^2 B\qty(x))\] Renomeando as variáveis podemos reescrever como,
\[V\qty(x,\theta,\theta^\dagger)=\theta\sigma^\mu\theta^\dagger v_\mu\qty(x)+\theta\theta\theta^\dagger\lambda^\dagger\qty(x)+\theta^\dagger\theta^\dagger\theta\lambda\qty(x)+\theta\theta\theta^\dagger\theta^\dagger\frac12D\qty(x)\] Aqui é claro os papeis de $\lambda$ como o super-parceiro de $v_\mu$, e de $D$ como o campo auxiliar. Necessitamos portanto agora de escrever um termo cinético, porém, não é suficiente de ser qualquer 
termo cinético, uma vez que este precisa ser invariante por uma transformação de calibre, aqui entra em jogo a importância da definição de um Super-Campo (Anti-)Chiral, note que $\partial_\mu V$ não tem nenhuma chance de ser invariante por calibre, porém, 
algo como $\mathcal D_a V$ tem a possibilidade de ser invariante, pois, $\mathcal D_a\Xi^\dagger=0=\mathcal D^\dagger_{\dot a}\Xi$. Contudo, como as derivadas super-covariantes não anti-comutam, $\mathcal D^\dagger_{\dot a}\mathcal D_b V$ não é invariante por calibre, mas analisemos,
\begin{align*}
    \mathcal D^\dagger_{\dot a}\mathcal D^{\dagger \dot a}\mathcal D_b V\rightarrow\mathcal D^\dagger_{\dot a}\mathcal D^{\dagger \dot a}\mathcal D_b \qty{V+\im\qty(\Xi^\dagger-\Xi)}&=\mathcal D^\dagger_{\dot a}\mathcal D^{\dagger \dot a}\mathcal D_b V-\im\mathcal D^\dagger_{\dot a}\mathcal D^{\dagger \dot a}\mathcal D_b \Xi\\
    &=\mathcal D^\dagger_{\dot a}\mathcal D^{\dagger \dot a}\mathcal D_b V-\im\mathcal D^\dagger_{\dot a}\qty[-\mathcal D_b\mathcal D^{\dagger \dot a} +2\im\epsilon^{\dot a\dot c}\sigma^{\mu}_{b\dot c}\partial_\mu]\Xi\\
    &=\mathcal D^\dagger_{\dot a}\mathcal D^{\dagger \dot a}\mathcal D_b V+\im\qty[\mathcal D^\dagger_{\dot a}\mathcal D_b +2\im\sigma^{\mu}_{b\dot a}\partial_\mu]\mathcal D^{\dagger \dot a}\Xi\\
    \mathcal D^\dagger_{\dot a}\mathcal D^{\dagger \dot a}\mathcal D_b V\rightarrow\mathcal D^\dagger_{\dot a}\mathcal D^{\dagger \dot a}\mathcal D_b \qty{V+\im\qty(\Xi^\dagger-\Xi)}&=\mathcal D^\dagger_{\dot a}\mathcal D^{\dagger \dot a}\mathcal D_b V
\end{align*}
De fato é invariante! Portanto este pode ser um possível componente do termo cinético, para conferir exatamente qual é o conteúdo deste campo vamos fazer alguns truques, primeiramente, este novo campo é Chiral! Isto é devido as 
derivadas super-covariantes serem anti-comutativas,
\begin{align*}
    W_a=\mathcal D^\dagger_{\dot b}\mathcal D^{\dagger \dot b}\mathcal D_a V\rightarrow \mathcal D^\dagger_{\dot c}W_a=0
\end{align*}
Para usar-se so fato de $W_a$ ser Chiral, vamos substituir já em $V$, $x=y+\im\theta\sigma\theta^\dagger$, expandir e usar identidades de Fierz,
\begin{align*}
    V&=\theta\sigma^\mu\theta^\dagger v_\mu\qty(y+\im\theta\sigma\theta^\dagger)+\theta\theta\theta^\dagger\lambda^\dagger\qty(y+\theta\sigma\theta^\dagger)+\theta^\dagger\theta^\dagger\theta\lambda\qty(y+\im\theta\sigma\theta)+\theta\theta\theta^\dagger\theta^\dagger\frac12 D\qty(y+\theta\sigma\theta^\dagger)\\
    V&=\theta\sigma^\mu\theta^\dagger v_\mu\qty(y)+\theta\theta\theta^\dagger\lambda^\dagger\qty(y)+\theta^\dagger\theta^\dagger\theta\lambda\qty(y)+\theta\theta\theta^\dagger\theta^\dagger\frac12\qty[ D\qty(y)-\im\partial^\mu v_\mu\qty(y)]\\
    \mathcal D_aV&=\theta^\dagger\theta^\dagger\qty[\lambda_a\qty(y)+\theta_a\qty(D\qty(y)-\im\partial^\mu v_\mu\qty(y))-\im\qty(\sigma^\mu{\bar\sigma^\nu}\theta)_a\partial_\mu v_\nu\qty(y)+\im\theta\theta\qty(\sigma^\mu\partial_\mu\lambda^\dagger)_a]+\cdots
\end{align*}
Mantemos aqui apenas os termos com $\theta^\dagger\theta^\dagger$, pois, assim como foi visto no começo da \cref{secchiral}, $\mathcal D^\dagger_{\dot a}$ atua em funções de $y,\theta,\theta^\dagger$ como $-\partial^\dagger_{\dot a}$, 
logo, $\mathcal D^\dagger_{\dot a}\mathcal D^{\dagger\dot a}=\partial^\dagger_{\dot a}\partial^{\dagger\dot a}$, portanto apenas contribuem para $W_a$ os termos com $\theta^\dagger\theta^\dagger$. Assim\footnote{Aqui utilizamos algumas igualdades dos 
geradores da algebra de momentum angular, $\sigma^\mu{\bar\sigma}^\nu=-g^{\mu\nu}-2\im\sigma^{\mu\nu}$. Também aqui tomamos a definição usual de $F_{\mu\nu}=\partial_\mu v_\nu-\partial_\nu v_\mu$.},
\begin{align*}
    W_a&=\mathcal D^\dagger_{\dot b}\mathcal D^{\dagger \dot b}\mathcal D_a V\\
    W_a&=\mathcal D^\dagger_{\dot b}\mathcal D^{\dagger \dot b}\qty{\theta^\dagger\theta^\dagger\qty[\lambda_a\qty(y)+\theta_a\qty(D\qty(y)-\im\partial^\mu v_\mu\qty(y))-\im\qty(\sigma^\mu{\bar\sigma^\nu}\theta)_a\partial_\mu v_\nu\qty(y)+\im\theta\theta\qty(\sigma^\mu\partial_\mu\lambda^\dagger)_a]+\cdots}\\
    W_a&=\lambda_a\qty(y)+\theta_a\qty(D\qty(y)-\im\partial^\mu v_\mu\qty(y))-\im\qty(\sigma^\mu{\bar\sigma^\nu}\theta)_a\partial_\mu v_\nu\qty(y)+\im\theta\theta\qty(\sigma^\mu\partial_\mu\lambda^\dagger)_a\\
    W_a&=\lambda_a\qty(y)+\theta_aD\qty(y)-\qty(\sigma^{\mu\nu}\theta)_aF_{\mu\nu}\qty(y)+\im\theta\theta\qty(\sigma^\mu\partial_\mu\lambda^\dagger)_a
\end{align*}
Exatamente o que precisávamos, note o aparecimento do termo cinético padrão, $F_{\mu\nu}$. Para construir um termo válido para ação, como $W_a$ é Chiral, necessitamos de 
integrar somente pela metade do Super-Espaço, assim como já fizemos anteriormente, portanto, esperamos que a seguinte ação gere um termo 
válido cinético\footnote{Aqui já adicionamos o termo conjugado para garantir a realidade da Ação, e, introduzimos o fator de $\frac14$ para garantir que os campos estejam normalizados canonicamente.},
\begin{align*}
    S&=\int\dd[4]{x}\dd[2]{\theta}\frac14W^aW_a+\int\dd[4]{x}\dd[2]{\theta^\dagger}\frac14W^\dagger_{\dot a}W^{\dagger\dot a}
\end{align*}
Vamos então obter os termos da ação, começamos notando que só precisamos dos termos com $\theta\theta$ em $W^aW_a$\footnote{Novamente usamos as identidades de Fierz, $\theta^a\theta^b=-\frac12\theta\theta\epsilon^{ab}$, e também a relação, $\Tr\qty[\sigma^{\mu\nu}\sigma^{\alpha\beta}]=\frac12\qty(g^{\mu\alpha}g^{\nu\beta}-g^{\mu\beta}g^{\nu\alpha})-\frac\im2\epsilon^{\mu\nu\alpha\beta}$. Aqui ocorre a aparição do 
dual, $\star F_{\mu\nu}=\frac12\epsilon_{\mu\nu\alpha\beta}F^{\alpha\beta}$.},
\begin{align*}
    W^aW_a&\supset\theta\theta\qty[ D^2+2\im\lambda\sigma^\mu\partial_\mu\lambda^\dagger-\frac12\Tr\qty[\sigma^{\mu\nu}\sigma^{\alpha\beta}]F_{\mu\nu}F_{\alpha\beta}]\\
    W^aW_a&\supset\theta\theta\qty[ D^2+2\im\lambda\sigma^\mu\partial_\mu\lambda^\dagger-\frac12F_{\mu\nu}F^{\mu\nu}-\frac\im2\star F_{\mu\nu}F^{\mu\nu}]\\
    W_{\dot a}^\dagger W^{\dagger \dot a}&\supset\theta^\dagger\theta^\dagger\qty[ D^2-2\im\partial_\mu\lambda\sigma^\mu\lambda^\dagger-\frac12F_{\mu\nu}F^{\mu\nu}+\frac\im2\star F_{\mu\nu}F^{\mu\nu}]\\
\end{align*}
Portanto a Ação é,
\[S=\int\dd[4]{x}\qty[-\frac14 F_{\mu\nu}F^{\mu\nu}+\im\lambda^\dagger{\bar\sigma}^\mu\partial^\mu\lambda+\frac12 D^2]\] No qual temos a ação cinética padrão de um bóson de calibre de $\mathfrak u\qty(1)$, mais o seu parceiro super-simétrico e um 
campo auxiliar. Finalmente, precisamos construir como introduzir interações providas pelo multipleto vetorial em um multipleto de matéria, isto é, (Anti-)Chiral. Sabemos que em teorias de 
calibre, o acoplamento entre os portadores de forças e os integrantes da matéria é descrito por uma modificação no termo cinético dos campos de matéria, portanto, devemos buscar uma modificação em $\Phi^\dagger \Phi$ que 
possua invariância de calibre sobre\footnote{Aqui estamos utilizando o parâmetro de acoplamento $g$ na forma \textit{física}, isto é, na forma em que tem mais utilidade para teoria de perturbação e espalhamento. Mas, poderíamos usar em uma outra forma, a \textit{holomorfa}, na qual substituímos $gV\rightarrow V$, e o parâmetro de acoplamento aparece apenas em $S=\int\dd[4]{x}\dd[2]{\theta}\frac{1}{4g^2}W^aW_a+\textnormal{h.c.}$, esta forma é mais útil para análises de efeitos não perturbativos.} \[\Phi\rightarrow\exp\qty(-2\im g\Xi)\Phi,\ \ \ \Phi^\dagger\rightarrow\Phi\exp\qty(2\im g\Xi^\dagger),\ \ \ V\rightarrow V+\im\qty(\Xi^\dagger-\Xi)\] Como a transformação é linear em $V$ e exponencial em $\Phi^{\qty(\dagger)}$, 
existe apenas uma função em que a multiplicação se torna linear, e esta é a exponencial, portanto, a variação do termo cinético mais simples que podemos escrever é,\[S=\int\dd[4]{x}\dd[2]{\theta}\dd[2]{\theta^\dagger}\Phi^\dagger\exp\qty(-2g V)\Phi\] Onde é claro, $g$ é o parâmetro de acoplamento adimensional. Tudo isto esta bem definido por que de fato 
$V$ é adimensional, apesar de $v_\mu$ e $\lambda$ possuírem dimensões de massa usuais. Vamos obter que tipo de interações são geradas por este termo cinético, basta expandir em potências a exponencial e utilizar-se das propriedades das quantidades grassmanianas\footnote{Usamos $\theta\sigma^\mu\theta^\dagger\theta\sigma^\nu\theta^\dagger=-\frac12\theta\theta\theta^\dagger\theta^\dagger g^{\mu\nu}$.},
\begin{align*}
    V&=\theta\sigma^\mu\theta^\dagger v_\mu+\theta\theta\theta^\dagger\lambda^\dagger+\theta^\dagger\theta^\dagger\theta\lambda+\frac12\theta\theta\theta^\dagger\theta^\dagger D\\
    V^2&=-\frac12\theta\theta\theta^\dagger\theta^\dagger v^\mu v_\mu\\
    V^3&=0\\
    \exp\qty(-2gV)&=1-2g\theta\sigma^\mu\theta^\dagger v_\mu-2g\theta\theta\theta^\dagger\lambda^\dagger-2g\theta^\dagger\theta^\dagger\theta\lambda-\theta\theta\theta^\dagger\theta^\dagger\qty(g D+g^2v_\mu v^\mu)
\end{align*}
Assim, como a integral grassmaniana seleciona apenas os termos com $\theta\theta\theta^\dagger\theta^\dagger$, o que resta é calcular as contribuições para este tipo de termo. O que é um trabalho direto, porém demorado, vamos 
apenas citar a forma final sendo,
\begin{align*}
    S=\int\dd[4]{x}\qty[-D_\mu\phi^\dagger D^\mu\phi+\im\psi^\dagger{\bar\sigma}^\mu D_\mu\psi+F^\dagger F+\sqrt 2 g\psi^\dagger\lambda^\dagger\phi+\sqrt 2 g\phi^\dagger\lambda\psi-g\phi^\dagger\phi D]
\end{align*}
Nota-se aqui o aparecimento da derivada covariante $D_\mu=\partial_\mu-\im g v_\mu$ a qual atua igualmente nas partes bosônicas e fermiônicas do multipleto chiral. Como o campo auxiliar não possui termo cinético ele não é modificado. 
Algo de interessante a se notar é, além da interação com o portador de força bosônico via derivada covariante, há termos de interação não triviais com o portador de força fermiônico. A extensão para teorias de calibre não abelianas pode ser feita, 
basta-se tomar índices de grupo em $V$ e $\Xi$ contraídos com os seus respectivos geradores, para se manter a super-simetria, algo que deve ser respeitado é: Todos os integrantes de um mesmo multipleto devem possuir as mesmas cargas e as mesmas transformações 
com relação a simetrias internas, portanto, assim como um bóson vetorial de calibre deve necessariamente se transformar na representação adjunta do grupo de calibre, seu super-parceiro fermiônico deve-se transformar na mesma representação, a adjunta.
Muitos outros tópicos aqui poderiam ser abordados, como a existência de propriedades de não renormalização do super-potencial, teorias com Super-Simetria estendida e até um dos tópicos que conferem à Super-Simetria um interesse mor, a calibração da 
Super-Simetria, isto é fazer da Super-Simetria uma simetria local, isto por sí é um tópico extremamente complexo, porém de altíssima beleza, uma vez que se a atuação de $Q_a$ é local, assim é a da \cref{qqdagger}, isto é, translações são locais. Portanto necessariamente uma teoria de Super-Simetria local é uma teoria de Gravitação.

%%%%%%%%%%%%%%%%%%%%%%%%%%%%%%%%%%%%%%%%%%%%%%%%%%%%%%%%%%%%%

\newpage

\printbibliography

\end{document}