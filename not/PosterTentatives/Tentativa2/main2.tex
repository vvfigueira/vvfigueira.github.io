% LaTeX poster using tikzposter class
% Paper size: default a0paper (we removed the explicit geometry to avoid timing issues)
% Times-like fonts, two columns, large fonts, centered body text

\documentclass[25pt,a0paper,portrait]{tikzposter}

% NOTE: we removed a direct \geometry{...} call that earlier caused a calc / TP@visibletextwidth timing error.
% If you absolutely need custom paper dimensions (100cm x 150cm), set them via your TeX engine or uncomment and
% move geometry settings into a proper location after tikzposter initialization.

% Fonts: Times-like (no amssymb to avoid \Bbbk clash)
\usepackage{newtxtext,newtxmath}

% Graphics and math
\usepackage{graphicx}
\usepackage{amsmath}
\usepackage{microtype}
\usepackage{enumitem}
\usepackage{hyperref}

% Tikzposter theme
\usetheme{Default}
\usecolorstyle{SolarizedDark}
\title{2+1 Gravity as a Toy Model for a Gauge Formulation of Gravity}
\author{Vicente V. Figueira}
\institute{}
\date{}

% Title formatting
\makeatletter
\renewcommand{\TP@maketitle}{{%
  \centering
  {\bfseries\Huge\sffamily\@title}\par\vskip1em
  {\Large\@author}\par\vskip1em}}
\makeatother

\begin{document}

\maketitle

\begin{columns}

\column{0.5\textwidth}
\block{Introduction}{\centering\Large
General Relativity in $3+1$ dimensions is highly non-linear and non-polynomial, making exact solutions and quantization difficult. In $2+1$ dimensions the metric has no local degrees of freedom; the theory becomes topological and amenable to gauge-theoretic reformulation.}

\block{Motivation \& Key Observations}{\centering\Large
\begin{itemize}[left=0pt,label={--},itemsep=6pt]
  \item Degree-of-freedom counting: $\tfrac12D(D-3)$ → vanishes for $D=3$.
  \item Riemann tensor determined algebraically by the Ricci tensor in $2+1$.
  \item Expectation: classical solvability and topological quantization.
\end{itemize}}

\block{Vielbein and Spin Connection}{\centering\Large
Vielbein $e^{\mu}$ and spin connection $\omega^{\mu}_{\;\nu}$ recast gravity in form language. The curvature 2-form:
\[R^{\mu}_{\;\nu}=d\omega^{\mu}_{\;\nu}+\omega^{\mu}_{\;\rho}\wedge\omega^{\rho}_{\;\nu}\]
Einstein--Hilbert (Einstein--Cartan) action:
\[S_{EH}=\frac{1}{2\kappa}\int\star R_{\beta\rho}\wedge e^{\beta}\wedge e^{\rho}.\]}

\block{Gauge Interpretation}{\centering\Large
\begin{itemize}[left=0pt,label={--},itemsep=6pt]
  \item $\omega$ transforms as an $\mathfrak{so}(D-1,1)$ connection.
  \item $e$ behaves like a translation-valued 1-form but not in adjoint rep.
  \item Direct gauge interpretation works nicely only for $D=2+1$.
\end{itemize}}

\column{0.5\textwidth}
\block{2+1 Gravity as Chern--Simons}{\centering\Large
In $2+1$ dimensions:
\[S_{CS}[A]=\frac{k}{4\pi}\int\langle A\wedge dA+\tfrac23A\wedge A\wedge A\rangle, \quad A=\omega+e.\]
This reproduces $S_{EH}$ up to a boundary term.}

\block{Bulk vs Boundary Dynamics}{\centering\Large
\begin{itemize}[left=0pt,label={--},itemsep=6pt]
  \item Bulk equations: $F=0$ ⇒ locally pure gauge.
  \item Physical degrees of freedom appear on boundaries / topologically non-trivial manifolds (BTZ black hole).
  \item Provides a topological quantum field theory framework for quantization.
\end{itemize}}

\block{Including Cosmological Constant}{\centering\Large
Requires gauging SO(3,1) (dS) or SO(2,2) (AdS). The action becomes difference of two Chern--Simons actions with $A_{\pm}=\omega\pm e$.}

\block{Conclusions}{\centering\Large
\begin{itemize}[left=0pt,label={--},itemsep=6pt]
  \item $2+1$ gravity is an exactly solvable topological gauge theory.
  \item Offers tractable setting to explore quantum gravity, renormalization.
  \item Boundary dynamics encode non-trivial physics.
\end{itemize}}

\end{columns}

\block{References}{\centering\Large
[1] Carlip (1995), Lectures in (2+1)-D Gravity.\\
[2] Deser, Jackiw (1984).\\
[3] Witten (1988).\\
Contact: Vicente V. Figueira}

\end{document}
