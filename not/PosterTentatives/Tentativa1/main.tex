\documentclass[final]{beamer}

% beamerposter with custom size: width=100cm, height=150cm
\usepackage[orientation=portrait,size=custom,width=100,height=150,scale=1.2]{beamerposter}

% Built-in beamer theme (works without extra files)
\usetheme{default}
\usecolortheme{default}

% Title info
\title{\textbf{2+1 Dimensional Gravity as a Gauge Theory}}
\author{Vicente Viater Figueira}
\institute{Institute of Physics (IFUSP)}
\usefonttheme{serif}
\begin{document}
\begin{frame}[t]

% ===== Title area =====
\begin{center}
  {\Huge \bfseries \inserttitle \par}
  \vspace{1cm}
  {\Large \insertauthor \par}
  \vspace{0.5cm}
  {\large \insertinstitute \par}
\end{center}

\vspace{1.5cm}

% ===== Two-column layout =====
\begin{columns}[t,totalwidth=\textwidth]

% ==== Column 1 ====
\begin{column}{0.48\textwidth}
  \begin{block}{Introduction}
    Write your introduction here. 
    Add context, motivation, and objectives.
  \end{block}

  \begin{block}{Theory}
    Your theoretical framework and equations go here.
    You can include math:
    \[
      E = mc^2
    \]
  \end{block}

  \begin{block}{Methods}
    Describe your methodology, experimental setup, 
    simulations, and data analysis process.
  \end{block}
\end{column}

% ==== Column 2 ====
\begin{column}{0.48\textwidth}
  \begin{block}{Results}
    Present figures, plots, and tables here.
    \begin{itemize}
      \item Use bullet points for key results
      \item Include high-resolution images
      \item Make plots large enough to be readable from 1–2 m away
    \end{itemize}
  \end{block}

  \begin{block}{Discussion}
    Interpret your results, compare with theory,
    highlight implications and limitations.
  \end{block}

  \begin{block}{Conclusions}
    Summarize findings and outline future work.
  \end{block}

  \begin{block}{Acknowledgements}
    This work was supported by \textbf{CAPES}.  
    Special thanks to \textbf{IFUS} for infrastructure and support.
  \end{block}
\end{column}

\end{columns}
\end{frame}
\end{document}