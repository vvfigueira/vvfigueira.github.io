\documentclass{beamer}

\AtBeginDocument{\newcommand{\im}{\textnormal{i}}}

% Tema da apresentação
\usetheme{Madrid} % Outros temas: AnnArbor, Copenhagen, Dresden, Warsaw, etc.

\newcommand{\lastframetitle}{}

% Atualiza o título do último slide em cada frame
\addtobeamertemplate{frametitle}{}{\xdef\lastframetitle{\insertframetitle}}

\setbeamercolor{title}{fg=black,bg=red!30!black!80!white}
\setbeamercolor{title in head/foot}{fg=black,bg=red!30!black!80!white}
% \setbeamercolor{block title}{fg=white,bg=green!60!black}

% \AtBeginSection[]{
%     \addtocounter{framenumber}{-1}
%   \begin{frame}{Sumário}
%     \tableofcontents[currentsection] % Destaca a seção atual
%   \end{frame}
    
% }

\useoutertheme{split}
% \setbeamertemplate{headline}{%
%   \leavevmode%
%   \hbox{%
%     \begin{beamercolorbox}[wd=0.5\paperwidth,ht=2.5ex,dp=1ex,left]{section in head/foot}%
%       \hspace{1em}\insertsectionhead
%     \end{beamercolorbox}%
%     \begin{beamercolorbox}[wd=0.5\paperwidth,ht=2.5ex,dp=1ex,left]{subsection in head/foot}%
%       \hspace{1em}\ifx\insertframetitle\empty
%       \lastframetitle % Mostra o último título válido
%     \else
%     \fi
%     \end{beamercolorbox}%
%   }
%   \vskip0pt%
% }

% Pacotes adicionais
\usepackage[utf8]{inputenc} % Codificação UTF-8
\usepackage[english]{babel} % Língua portuguesa do Brasil
\usepackage{graphicx} % Para inserir imagens
\usepackage{stackengine}
\usepackage{amsmath,amssymb} % Para matemática avançada
\usepackage{physics}
\usepackage{tensor}
\usepackage{bbm}
\usepackage{mathtools}
\usepackage[style=numeric-comp, backend=bibtex]{biblatex}
\usepackage{tikz}

\usefonttheme{serif}

\addbibresource{refs.bib}

\newcommand{\wedgecomma}{\stackon[1pt]{,}{\smash{\scriptsize$\wedge$}}}
\newcommand{\wedgecomm}[2]{\qty[ #1\ \wedgecomma\ #2 ]}

% Informações da apresentação
\title{2+1 Gravity as a Gauge Theory}
\author{Vicente V. Figueira}
\institute{QCD Meets Gravity XI School}
\date{December 8, 2025}

% \usebackgroundtemplate{%
%   \tikz\node[opacity=0.2,inner sep=0pt]{%
%     \includegraphics[width=\paperwidth,height=\paperheight]{bgimage.png}%
%   };
% }

% \usebackgroundtemplate{%
%   \parbox[c][\paperheight][c]{\paperwidth}{%
%     \centering
%     \includegraphics[height=\paperheight,keepaspectratio]{bg2.png}%
%   }%
% }

\addtobeamertemplate{background}{}{% append to background template
  \begin{tikzpicture}[remember picture,overlay]
    % compute the true page center and shift the scope there
    \coordinate (pgcenter) at (current page.center);
    \begin{scope}[shift={(pgcenter)}]
      % inner/outer sep = 0 avoids node padding; anchor=center aligns image center
      \node[opacity=0.3,anchor=center,inner sep=0,outer sep=0] {
        \includegraphics[height=\paperheight,keepaspectratio]{bg2.png}%
      };
    \end{scope}
  \end{tikzpicture}%
}


\begin{document}

\nocite{*}

% \begin{frame}[plain]
% \begin{tikzpicture}[remember picture,overlay]
%     \node at (current page.center) {
%         \includegraphics[height=\paperheight, keepaspectratio]{bg2.png}
%     };
% \end{tikzpicture}
% \end{frame}


% Slide de título
\begin{frame}
    \vspace{2cm}
    \titlepage%\centering%\includegraphics[width=0.8\textwidth]{agorameeting.jpg}
\end{frame}

\begin{frame}
    Gravity--Gauge comparison:
    \begin{align*}
        &S_{\textnormal{EH}}=\frac{1}{\kappa^2}\int\dd[D]{x}\sqrt{\abs{g}}R\ \ \ \ \ \ \textnormal{v.s.} &S_{\textnormal{YM}}=\frac{1}{g^2}\int\Tr\qty[\vb F\wedge\star\vb F]\\
        &\vb g\sim\phi_\ast \vb g &\vb A\sim U\qty(\vb A+\vb d)U^{-1}
    \end{align*}
    \begin{center} Using vielbein/spin-connection is possible to \textit{polynomialize} gravity:\end{center}
    \[S_{\textnormal{EH}}\stackrel{D=3}{=}\frac1\lambda\int\Tr\qty[\vb R\wedge\vb e]\stackrel{!}{=}S_{\textnormal{CS}}\]
    \begin{center} Diff $\cong$ local ISO$\qty(2,1)$ \end{center}
    \begin{center} $2+1$ dimensional gravity is a CS gauge theory of ISO$\qty(2,1)$ \end{center}
\end{frame}

% Slide de sumário
% \begin{frame}{Sumário}
%     \tableofcontents
% \end{frame}

% Seção: Introdução
% \section{Motivation}

% \begin{frame}{Yang-Mills}
%     $$S_{\textnormal{YM}}= \frac{1}{g^2}\int\limits_M\Tr\qty[\vb F\wedge \star \vb F],\ \ \ \vb F = \vb d\vb A+\wedgecomm{\vb A}{\vb A}$$
%     \begin{itemize}
%         \item Polynomial.\pause
%         \item Dimensionless coupling $\rightarrow$ Renormalizable.\pause
%         \item Has local Lie group-valued redundancies, $\vb A\cong U\qty(\vb A+\vb d)U^{-1}$.\pause
%     \end{itemize}

%     \vspace{1.5cm}

%     It's possible to formulate GR as YM theory?

% \end{frame}

% \begin{frame}{Why $D=2+1$?}
%     \begin{itemize}
%         \item No local propagating modes:
%         $$\textnormal{d.o.f.} = \frac12 D\qty(D-3)$$
%     \end{itemize}
%     Simpler but non-trivial toy model.
% \end{frame}

% \section{Reformulation}
% \begin{frame}{Recover of polynomiality}
%     \begin{itemize}
%         \item Change of variables: $\vb g\rightarrow \begin{cases}\vb e^\mu,&\textnormal{ Vielbein} \\\boldsymbol\omega_{\alpha\beta},&\textnormal{ Spin connection} \end{cases}$\pause
%         \item Relax torsionless condition $\rightarrow$ $\vb e^\mu$ and $\boldsymbol\omega_{\alpha\beta}$ are independent.\pause
%     \end{itemize}

%     \vspace{0.7cm}

%     $$S_{\textnormal{EH}}= \frac{1}{2\kappa}\int\limits_M\star\vb R^{\alpha\beta}\wedge \vb e_\alpha\wedge \vb e_\beta\overset{D=3}{=}\frac{1}{2\kappa}\epsilon_{\alpha\beta\mu}\int\limits_M \vb R^{\alpha\beta}\wedge \vb e^\mu$$\pause
    
%     \vspace{0.7cm}
    
%     Can we write the integrand as a trace?
% \end{frame}

% \begin{frame}{Recover of dimensionless coupling constant}
%     \begin{itemize}
%         \item For $\mathfrak{iso}\qty(2,1)$: $$\Tr\qty[J_{\alpha\beta}P_\mu]=\frac{\lambda}{\kappa}\epsilon_{\alpha\beta\mu},\ \ \ \Tr\qty[P_\nu P_\mu]=0,\ \ \ \Tr\qty[J_{\alpha\beta}J_{\mu\nu}]=0$$\pause
%         \item Dressing the fields with the algebra,
%         $$S_{\textnormal{EH}}=\frac{1}{\lambda}\int\limits_M \Tr\qty[\vb R\wedge \vb e],\ \ \ \vb R=\frac12\vb R^{\alpha\beta}J_{\alpha\beta},\ \vb e=\vb e^\mu P_\mu$$\pause
%     \end{itemize}

%     \vspace{0.6cm}

%     Not standard YM form...
% \end{frame}

% \section{Identification}

% \begin{frame}{Chern-Simons}
%     \begin{itemize}
%         \item d.o.f. mismatch: $0$ v.s. $2N_c$\pause
%         \item Available gauge theory with $0$ d.o.f.: Chern-Simons!
%         $$S_{\textnormal{CS}}=\frac{1}{2\lambda}\int\limits_M\Tr\qty[\vb A \wedge \qty(\vb d{\vb A}+\frac13\wedgecomm{\vb A}{\vb A})]$$\pause
%         \item Tentative: CS theory of ISO(2,1), $\vb A =\boldsymbol\omega+\vb e$,\pause
%         $$S_{\textnormal{CS}}=\frac1\lambda\int\limits_M\expval{\vb e\ \wedgecomma\ \qty(\vb d{\boldsymbol\omega}+\frac12\wedgecomm{\boldsymbol\omega}{\boldsymbol\omega})}=S_{\textnormal{EH}}$$
%     \end{itemize}
%     Match!
% \end{frame}

% \section{Conclusions}

% \begin{frame}{Conclusions}
%     \begin{itemize}
%         \item $2+1$ dimensional gravity is a CS gauge theory of ISO(2,1).\pause
%         \item Opens path to quantization via well known CS methods.\pause
%         \item Inclusion of cosmological constant change gauge group, $\begin{cases}\Lambda > 0: & \textnormal{SO}\qty(3,1),\ (\textnormal{dS})\\\Lambda < 0: & \textnormal{SO}\qty(2,2), \ (\textnormal{AdS})\end{cases}$\pause
%         \item Offers a bridge between GR and usual gauge theory intuition.\pause
%         \item Pose some suggestions about $D=3+1$.
%     \end{itemize}
% \end{frame}

% \begin{frame}
%     %\addtocounter{framenumber}{-1}
%     \centering{Thank You!}
% \end{frame}

% \section{References}

% \begin{frame}{References}
%     \printbibliography
% \end{frame}

% \addtocounter{framenumber}{2000}

\end{document}