\documentclass[final]{beamer}

% Poster size: 1 m x 1.5 m (portrait)
\usepackage[orientation=portrait,size=custom,width=100,height=150,scale=1.3]{beamerposter}

% Font: Times New Roman
\usefonttheme{serif}

% Built-in beamer theme
\usetheme{default}
\usecolortheme{default}

% ===== Style Adjustments =====
% Much bigger fonts for visibility
\setbeamerfont{block title}{size=\huge,series=\bfseries}
\setbeamerfont{block body}{size=\Large}

% Center everything inside blocks
\addtobeamertemplate{block begin}{\centering}{}

% Add line spacing for better readability
\renewcommand{\baselinestretch}{1.3}

% Title info
\title{\textbf{Title of Your Physics Presentation}}
\author{Your Name}
\institute{Institute of Physics (IFUS)}

\begin{document}
\begin{frame}[t]

% ===== Title area =====
\begin{center}
  {\fontsize{60}{70}\selectfont \bfseries \inserttitle \par}
  \vspace{1cm}
  {\fontsize{40}{48}\selectfont \insertauthor \par}
  \vspace{0.5cm}
  {\fontsize{36}{44}\selectfont \insertinstitute \par}
\end{center}

\vspace{1.5cm}

% ===== Two-column layout =====
\begin{columns}[t,totalwidth=\textwidth]

% ==== Column 1 ====
\begin{column}{0.48\textwidth}
  \begin{block}{Introduction}
    Write your introduction here. 
    Add context, motivation, and objectives.
  \end{block}

  \begin{block}{Theory}
    Your theoretical framework and equations go here.
    Example equation:
    \[
      E = mc^2
    \]
  \end{block}

  \begin{block}{Methods}
    Describe your methodology, experimental setup, 
    simulations, and data analysis process.
  \end{block}
\end{column}

% ==== Column 2 ====
\begin{column}{0.48\textwidth}
  \begin{block}{Results}
    Present figures, plots, and tables here.
    \begin{center}
    \begin{itemize}
      \item Use bullet points for key results
      \item Include high-resolution images
      \item Make plots large enough to be readable from 1–2 m away
    \end{itemize}
    \end{center}
  \end{block}

  \begin{block}{Discussion}
    Interpret your results, compare with theory,
    highlight implications and limitations.
  \end{block}

  \begin{block}{Conclusions}
    Summarize findings and outline future work.
  \end{block}

  \begin{block}{Acknowledgements}
    This work was supported by \textbf{CAPES}.  
    Special thanks to \textbf{IFUS} for infrastructure and support.
  \end{block}
\end{column}

\end{columns}
\end{frame}
\end{document}
