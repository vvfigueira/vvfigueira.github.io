\documentclass[twoside]{amsart}

\usepackage[brazilian]{babel}
\usepackage{csquotes}
%\usepackage[sorting=none, style=verbose-inote, backend=biber]{biblatex}
\usepackage{amsmath}
\usepackage{amssymb}
\usepackage{bbm}
\usepackage{graphics}
\usepackage{mathtools}
\usepackage[hidelinks]{hyperref}
\usepackage{physics}
\usepackage{enumitem}
\usepackage{slashed}
\usepackage{tensor}
\usepackage[lmargin=0.5cm,rmargin=0.5cm, tmargin =1cm,bmargin =1cm]{geometry}

\AtBeginDocument{\renewcommand*{\hbar}{{\mkern-1mu\mathchar'26\mkern-8mu\textnormal{h}}}}
\AtBeginDocument{\newcommand{\e}{\textnormal{e}}}
\AtBeginDocument{\newcommand{\im}{\textnormal{i}}}
\AtBeginDocument{\newcommand{\luz}{\textnormal{c}}}
\AtBeginDocument{\newcommand{\grav}{\textnormal{G}}}
\AtBeginDocument{\newcommand{\kb}{{\textnormal{k}_{\textnormal{B}}}}}
\newcommand{\Dd}[1]{\mathcal D #1}
\newcommand{\Det}[1]{\textup{Det} #1}
\newcommand{\sgn}[1]{\mbox{sgn}\qty(#1)}
\newcommand{\cqd}{\hfill$\blacksquare$}

\numberwithin{equation}{section}

\newtheorem{teo}{Teorema}[section]
\newtheorem{defi}{Definição}[section]
\newtheorem{lem}{Lema}[section]
\newtheorem{hip}{Hipótese}[subsection]

\pagestyle{plain}

\AddToHook{cmd/section/before}{\clearpage}

%\addbibresource{ref.bib}

\title{
Notas de Mecânica Relativística
}
\author{
  Vicente V. Figueira
       }
\date{\today}

\begin{document}

\maketitle

\tableofcontents

%%%%%%%%%%%%%%%%%%%%%%%%%%%%%%%%%%%%%%%%%%%%%%%%%%%%%%%%%%%%%

%\begin{refsection}
\section{Introdução}

%\printbibliography[heading=subbibliography]
%\end{refsection}

%%%%%%%%%%%%%%%%%%%%%%%%%%%%%%%%%%%%%%%%%%%%%%%%%%%%%%%%%%%%%

%\begin{refsection}
\section{Mecânica Relativística}

\subsection{Hipóteses da Teoria Relativística}


Podemos pensar na Física como um conjunto de várias teorias, cada qual com seu limite de 
validade, e, estas historicamente foram nomeadas segundo estes, ou, segundo o fenômeno 
característico que se propõem a modelar. A Teoria Clássica, ou Mecânica Clássica como 
alguns chamam, está entre as primeiras teorias física a ser proposta, e como seu nome 
sugere, está intencionada a descrever fenômenos \emph{não extremos}, seja do ponto de 
vista de energia, número de constituintes ou dimensões do sistema. Questões como qual 
os valores numéricos dos reais limites da teoria só podem serem sanadas com comparação 
experimental dos valores teóricos.

A formulação da Teoria Clássica tem início com observações experimentais e experimentos 
mentais, que são \emph{confiáveis} para extrair informações \emph{médias} em sistemas de 
energia, número de constituintes e dimensões \emph{médias}. O primeiro ápice da formulação 
Clássica foi devido a Newton, seguido por Lagrange e Hamilton. Porém, estamos aqui interessados 
em desenvolver uma abordagem via postulados para sedimentar quais informações estão sendo 
assumidas em nossa teoria.

Ao tentar dar início sobre quais são as hipóteses ocultas feitas sobre uma teoria, devemos 
pensar primeiramente qual são os principais agentes dinâmicos, que por experiência devem ser 
posições e velocidades, estes que são vetores, logo, deve haver um espaço vetorial abaixo, 
e via experiência, este é um espaço de três dimensões. Certamente, novamente por experiência,  
também há uma quantidade que flui naturalmente, chamada comumente de \emph{tempo}. 
Naturalmente, eu, em meu referencial, utilizando de aparatos adequados posso medir intervalos 
espaciais e temporais à vontade, porém, outra pessoa, distante de mim, poderia por sua vez 
também estar interessada em realizar as mesmas medições que eu estou fazendo, se ela realiza 
as medições do seu referencial, e eu realizo as medições do meu referencial, há alguma 
correlação entre estas? Bem, naturalmente se para referenciais distintos não houvesse nenhum 
tipo de correlação entre os valores medidos toda a física cairia por terra, visto que não 
haveria nenhum modo de comparar medidas e testar teorias físicas, deve portanto haver ao menos 
alguma classe de referenciais tais que as coordenadas espaciais e temporais possuam correlação 
entre os dois observadores. Observadores pertencentes a esta classe especial de referenciais que concordam com a física descrita são chamados de observadores \emph{inerciais}, devemos portanto descrever qual propriedades queremos que um referencial inercial possua. O requisito mais intuitivo é de que observadores inerciais percebem objetos que não interagem movem-se com velocidade constante, a princípio mudanças de coordenadas do tipo de translação são triviais de serem tratadas, apenas se fazendo, $${\bar x}^\mu=x^\mu+a^\mu$$
Nosso interesse esta em mudanças de referenciais não dadas por translações, como boost e rotações, essas atuam de forma linear como, $${\bar x}^\mu={\textup L}^\mu_{\ \nu}x^\nu$$
No caso de uma rotação pura ao redor de um eixo $\boldsymbol\theta$ com módulo $\theta$, assumimos que a componente temporal não é alterada, assim, o vetor mais genérico linear capaz de ser formado com $\vb x,\boldsymbol\theta$ e $\theta$ de modo a preservar a norma $\vb{\bar x}=\vb x$ é,

\begin{align}
    \vb{\bar x}&=\frac{\boldsymbol\theta\cdot\vb x}{\theta^2}\boldsymbol\theta+\qty(\vb x-\frac{\boldsymbol\theta\cdot\vb x}{\theta^2}\boldsymbol\theta)\cos\theta-\frac{\boldsymbol\theta\times \vb x}{\theta}\sin\theta
\end{align}

Logo a matriz $\textup L$ é $${\textup L}^{0}_{\ 0}=1,\ {\textup L}^0_{\ i}={\textup L}^{i}_{\ 0}=0,\ {\textup L}^{i}_{\ j}={\textup R}^{i}_{\ j}$$
Com,
\begin{align}
    {\textup R}^{i}_{\ j}&=\frac{\theta_i\theta_{j}}{\theta^2}+\qty(\delta_{ij}-\frac{\theta_{i}\theta_j}{\theta^2})\cos\theta+\frac{\epsilon_{ijk}\theta_k}{\theta}\sin\theta
\end{align}

Para um boost, temos que construir escalares e vetores com $\vb v$ e $\vb x$, a construção linear mais geral é,

\begin{align}
    {\bar t}&=a\qty(v)t+b\qty(v)\vb v\cdot \vb x\\
    \vb {\bar x}&=c\qty(v)\vb x+d\qty(v)\frac{\vb v\cdot \vb x}{v^2}\vb v+e\qty(v)t\vb v
\end{align}

A condição $\vb x=\vb v t$ deve implicar $\vb{\bar x}=0$, 

\begin{align}
    \vb 0&=c\qty(v)\vb v t+d\qty(v)t\frac{\vb v\cdot \vb v}{v^2}\vb v+e\qty(v)t\vb v\\
    \vb 0&=c\qty(v)\vb v t+d\qty(v)\vb v t+e\qty(v)\vb v t
\end{align}

logo a restrição é,

\begin{align}
    c\qty(v)+d\qty(v)+e\qty(v)&=0
\end{align}

A transformação inversa deve valer para $\vb {\bar v}=-\vb v$, 

\begin{align}
    &\begin{cases}
        {t}&=a\qty(v){\bar t}-b\qty(v)\vb v\cdot \vb{\bar x}\\
    \vb { x}&=c\qty(v)\vb{\bar x}+d\qty(v)\frac{\vb v\cdot \vb{\bar x}}{v^2}\vb v-e\qty(v){\bar t}\vb v
    \end{cases}\\
    &\begin{cases}
        {t}&=a\qty(v)\qty[a\qty(v)t+b\qty(v)\vb v\cdot \vb x]-b\qty(v)\vb v\cdot \qty[c\qty(v)\vb x+\frac{1}{v^2}d\qty(v)\qty(\vb v\cdot \vb x)\vb v+e\qty(v)t\vb v]\\
    \vb { x}&=c\qty(v)\qty[c\vb x+\frac{1}{v^2}d\qty(v)\qty(\vb v\cdot\vb x)\vb v+e\qty(v)t\vb v]+\frac{1}{v^2}d\qty(v)\qty(\vb v\cdot \qty[c\qty(v)\vb x+\frac{1}{v^2}d\qty(v)\qty(\vb v\cdot \vb x)\vb v+e\qty(v)t\vb v])\vb v-e\qty(v)\qty[a\qty(v)t+b\qty(v)\qty(\vb v\cdot\vb x)]\vb v
    \end{cases}
\end{align}

o que implica em

\begin{align}
    c^2&=1\\
    a^2-ebv^2&=1\\
    e^2-ebv^2&=1\\
    ea+e^2&=0\\
    ba+be&=0
\end{align}

Tomamos $c=1$ pois $c=-1$ corresponde a uma rotação. Como $e\neq 0$ segue que $a=-e$, logo,

\begin{align}
    b&=\frac{1-a^2}{av^2}\\
    c&=1\\
    d&=a-1\\
    e&=-a
\end{align}

Como dois boost seguidos deve corresponder a um único boost, chegamos em

\begin{align}
    \frac{va\qty(v)}{\bar wa\qty(\bar w)}\qty(1-a^2\qty(\bar w))&=\frac{\bar wa\qty(\bar w)}{va\qty(v)}\qty(1-a^2\qty(v))\\
    \frac{1-a^2\qty(v)}{v^2a^2\qty(v)}&=\frac{1-a^2\qty(\bar w)}{{\bar w}^2a^2\qty({\bar w})}=K\\
    a\qty(v)&=\frac{\pm 1}{\sqrt{1+Kv^2}}=\pm \gamma\qty(v),\ \ b\qty(v)=\frac{\pm K}{\sqrt{1+Kv^2}}=\pm K\gamma\qty(v) 
\end{align}

Logo a transformação é,

\begin{align}
    {\bar t}&=\pm\gamma\qty(v)\qty(t+K\vb v \cdot \vb x)\\
    \vb {\bar x}&= \vb x+\frac{\pm\gamma\qty(v)-1}{v^2}\qty(\vb v \cdot \vb x)\vb v\mp\gamma\qty(v)t \vb v
\end{align}

Resta apenas fixar o sinal de $a\qty(v)$ e o valor de $K$, o caso $K=0$ faz com que $\gamma=1$ e retorna as relações da relatividade de Galileu. Para casos $K\neq 0$, apenas é importante o sinal de $K$ e não seu valor absoluto, visto que podemos sempre mudar as unidades de tempo para obter,

\begin{align}
    {\bar t}&=\pm\gamma\qty(v)\qty(t+\sgn{K}\norm{K}\vb v \cdot \vb x)\\
    \vb {\bar x}&= \vb x+\frac{\pm\gamma\qty(v)-1}{v^2}\qty(\vb v \cdot \vb x)\vb v\mp\gamma\qty(v)t \vb v\\
    \norm{K}^{-\frac12}{\bar t}&=\pm\gamma\qty(v)\qty(\norm{K}^{-\frac12}t+\sgn{K}\norm{K}^\frac12\vb v \cdot \vb x)\\
    \vb {\bar x}&= \vb x+\frac{\pm\gamma\qty(v)-1}{\norm{K}v^2}\qty(\norm{K}^\frac12\vb v \cdot \vb x)\norm{K}^\frac12\vb v\mp\gamma\qty(v)\norm{K}^{-\frac12}t \norm{K}^\frac12\vb v 
\end{align}

Redefinindo $t\norm{K}^{-\frac12}\rightarrow t$,

\begin{align}
    {\bar t}&=\pm\gamma\qty(v)\qty(t+\sgn{K}\vb v \cdot \vb x)\\
    \vb {\bar x}&= \vb x+\frac{\pm\gamma\qty(v)-1}{v^2}\qty(\vb v \cdot \vb x)\vb v\mp\gamma\qty(v)t \vb v
\end{align}

Para ter uma intuição sobre o que o sinal de $K$ significa vamos fazer dois boost seguidos,

\begin{align}
    {\bar{\bar t}}&=\pm\gamma\qty(w)\qty({\bar t}+\sgn{K}\vb w\cdot\vb{\bar x})\\
    &=\pm\gamma\qty(w)\qty(\pm\gamma\qty(v)\qty{t+\sgn{K}\vb v\cdot\vb x}+\sgn{K}\vb w\cdot\qty[\vb x+\frac{\pm\gamma\qty(v)-1}{v^2}\qty(\vb v \cdot \vb x)\vb v\mp\gamma\qty(v)t \vb v])\\
    &=\gamma\qty(w)\gamma\qty(v)\qty(t+\sgn{K}\vb v\cdot\vb x\pm\frac{\sgn K}{\gamma\qty(v)}\vb w\cdot \vb x+\sgn K\frac{\gamma\qty(v)\mp1}{v^2\gamma\qty(v)}\qty(\vb w\cdot \vb v)\vb v\cdot \vb x-\sgn K t\vb w \cdot \vb v)\\
    &=\gamma\qty(w)\gamma\qty(v)\qty(1-\sgn K\vb w\cdot \vb v)\qty(t+\frac{\sgn K}{v^2\gamma\qty(v)\qty(1-\sgn K\vb w\cdot \vb v)} \vb x\cdot \qty{v^2\gamma\qty(v)\vb v\pm v^2\vb w+\qty(\gamma\qty(v)\mp 1)\qty(\vb w\cdot \vb v)\vb v})
\end{align}

Fazendo as identificações,

\begin{align}
    \pm\gamma\qty(u)&=\gamma\qty(w)\gamma\qty(v)\qty(1-\sgn K\vb w\cdot \vb v)\\
    \vb u &=\frac{v^2\gamma\qty(v)+\qty(\gamma\qty(v)\mp1)\vb w\cdot\vb v}{v^2\gamma\qty(v)\qty(1-\sgn K\vb w\cdot\vb v)}\vb v\pm\frac{1}{\gamma\qty(v)\qty(1-\sgn K \vb w\cdot \vb v)}\vb w
\end{align}

Para $a\qty(v)=\pm\gamma\qty(v)$ ser real com $\sgn K=-1$ somos obrigados a ter $\norm{v}< 1$ para todos os referenciais inerciais, assim $1-\sgn K \vb w\cdot \vb v>0$, que por sua vez nos exige tomar $a\qty(v)=+\gamma\qty(v)$. Analogamente $\sgn K=0$ também fixa $a\qty(v)=+\gamma\qty(v)$, outra opção restante é $\sgn K=+1$ que não restringe os valores de $\norm{v}$, assim $a\qty(v)$ pode assumir valores tanto positivos quanto negativos, valor de $a\qty(v)$ negativo implica em reversão temporal, como estamos assumindo uma transformação passiva não é esperado que resulte em uma inversão na direção temporal, logo, fixamos $K\leq 0$ como casos \emph{físicos}, como nesses casos a dimensão de $K^{-1}$ é de velocidade quadrada, chamamos o parâmetro $c=\qty(-K)^{-\frac12}$ como sendo uma escala de velocidade universal que fixa o máximo (não atingível) de velocidade de um referencial inercial, no limite $c\rightarrow \infty$ recuperamos a relatividade Galileana que é compatível com o conceito de que não há limite de velocidades. Com essas convenções temos,

\begin{align}
    \gamma&=\frac{1}{\sqrt{1-\frac{v^2}{c^2}}}\\
    {\bar t}&=\gamma\qty(t-\frac{\vb v\cdot\vb x}{c^2})\\
    \vb {\bar x}&=\vb x+\frac{\gamma-1}{v^2}\qty(\vb v \cdot \vb x)\vb v-\gamma t\vb v\\
    \gamma\qty(u)&=\gamma\qty(w)\gamma\qty(v)\qty(1+\frac{\vb v\cdot\vb w}{c^2})\\
    \vb u&=\frac1\gamma\frac{\gamma \vb v+\vb w}{1+\frac{\vb w \cdot\vb v}{c^2}}+\frac{\gamma-1}{v^2\gamma}\frac{\vb v\cdot \vb w}{1+\frac{\vb v\cdot \vb w}{c^2}}\vb v
\end{align}

As transformações de referenciais devem por sua vez satisfazer alguma condição de quantidade conservada em um espaço métrico, note que,

\begin{align}
    \dd{{\bar t}}&=\gamma\qty(\dd{t}-\frac{\vb v}{c^2}\cdot \dd{\vb x})\\
    \dd{\vb{\bar x}}&=\dd{\vb x}+\frac{\gamma-1}{v^2}\qty(\vb v\cdot \dd{\vb x})\vb v-\gamma\dd{t}\vb v\\
    \qty(\dd{{\bar t}})^2&=\gamma^2\qty(\dd{t})^2-2\frac{\gamma^2}{c^2}\vb v\cdot \dd{\vb x}\dd{t}+\frac{\gamma^2}{c^4}\qty(\vb v\cdot \dd{\vb x})^2\\
    \qty(\dd{\vb{\bar x}})^2&=\gamma^2v^2\qty(\dd{t})^2+\qty(\dd{\vb x})^2+\qty[\frac{\qty(\gamma-1)^2}{v^2}+\frac{2\qty(\gamma-1)}{v^2}]\qty(\vb v\cdot\dd{\vb x})^2-\qty[2\gamma+2\gamma\qty(\gamma-1)]\dd{t}\vb v\cdot\dd{\vb x}\\
    -c^2\qty(\dd{{\bar t}})^2+\qty(\dd{\vb{\bar x}})^2&=-c^2\qty(\dd{t})^2+\qty(\dd{\vb x})^2+\qty(\vb v\cdot\dd{\vb x})^2\qty[-\frac{\gamma^2}{c^2}+\frac{\qty(\gamma-1)^2}{v^2}+2\frac{\gamma+1}{v^2}]+\dd{t}\vb v\cdot\dd{\vb x}\qty[2\gamma^2-2\gamma-2\gamma\qty(\gamma-1)]
\end{align}

Logo, concluímos que,

\begin{align}
    -c^2\qty(\dd{{\bar t}})^2+\qty(\dd{\vb{\bar x}})^2=-c^2\qty(\dd{{ t}})^2+\qty(\dd{\vb{ x}})^2
\end{align}

Ou seja, os boost respeitam uma regra de preservação de uma \emph{métrica} no espaço-tempo, note que uma rotação arbitrária também respeita a mesma condição, assim como translações arbitrárias, para rotações e boost arbitrários temos uma condição ainda mais forte,

\begin{align}
    -c^2\qty({\bar t})^2+\qty(\vb{\bar x})^2&=-c^2\qty(t)^2+\qty(\vb x)^2
\end{align}

Ou seja, boost são análogos a rotações no plano $\vb x-t$, introduzindo a notação, $x=\mqty(ct & \vb x)^\textup T$ e 

\begin{align}
    g&=\mqty(-1 & 0 & 0& 0\\ 0&1&0&0\\0&0&1&0\\0&0&0&1)
\end{align}

Temos que a relação da \emph{métrica no espaço-tempo} é,

\begin{align}
    {\bar x}^\textup T g \bar x&=x^\textup T g x
\end{align}

Ou em notação mais conveniente com a convenção de soma,

\begin{align}
    \bar x\cdot \bar x=g_{\mu\nu}{\bar x}^\mu {\bar x}^\nu&=g_{\mu\nu}x^\mu x^\nu=x\cdot x
\end{align}

Chamamos o invariante \emph{elemento de linha} de,

\begin{align}
    -c^2\dd{s}^2&=g_{\mu\nu}\dd{x^\mu}\dd{x^\nu}
\end{align}

Que é preservado entre referenciais inerciais, os boost escritos na nova notação será,

\begin{align}
    {\bar x}^0&=\gamma x^0-\gamma\frac{\vb v\cdot\vb x}{c}\\
    \vb {\bar x}&=\vb x+\frac{\gamma-1}{v^2}\qty(\vb v \cdot \vb x)\vb v-\frac{\gamma}{c} x^0\vb v\\
    {\bar x}^0&={\textup B}^0_{\ 0}x^0+{\textup B}^0_{\ i}x^i\\
    {\bar x}^j&={\textup B}^j_{\ 0}x^0+{\textup B}^j_{\ i}x^i\\
    {\textup B}^0_{\ 0}&=\gamma,\ {\textup B}^0_{\ i}=-\frac{\gamma}{c}v_i\\
    {\textup B}^j_{\ 0}&=-\frac{\gamma}{c}v_j,\ {\textup B}^j_{\ i}=\delta_{ji}+\frac{\gamma-1}{v^2}v_jv_i
\end{align}

Assim podemos escrever uma transformação de Boost como,

\begin{align}
    \bar x={\textup B} x\\
    {\bar x}^\mu={\textup B}^\mu_{\ \nu}x^\nu
\end{align}

Uma combinação arbitrária de Boosts e Rotações constitui o que chamamos de \emph{transformações de Lorentz}, geralmente representadas por $\Lambda$, que são o tipo mais geral de transformações lineares que preservam a métrica do espaço-tempo, pois,

\begin{align}
    x^{\textup T}g x&={\bar x}^{\textup T}g{\bar x}\\
    &=\qty(\Lambda x)^{\textup T}g\qty(\Lambda x)\\
    &=x^{\textup T}\Lambda^{\textup T}g\Lambda x
\end{align}

Portanto,

\begin{align}
    g&=\Lambda^{\textup T}g\Lambda\\
    g_{\mu\nu}&=\Lambda^\alpha_{\ \mu}g_{\alpha\beta}\Lambda^\beta_{\ \nu}\\
    g_{\mu\nu}&=g_{\alpha\beta}\Lambda^\alpha_{\ \mu}\Lambda^\beta_{\ \nu}
\end{align}

Se temos um objeto movendo-se descrito por um vetor $\vb x$, a velocidade que medimos é da forma $\dv{\vb x}{t}$, porém esse não constitui como um vetor que se transforma corretamente entre referenciais inerciais, para isso, necessitamos de uma quantidade análoga a $t$ que seja invariante, uma quantidade disponível é o elemento de linha, fazemos um boost com a mesma velocidade instantânea que o objeto que queremos descrever e,

\begin{align}
    -c^2\dd{s}^2&=-c^2\dd{{\bar t}}^2=-c^2\dd{t}^2+\dd{\vb x}^2\\
    -c^2&=-c^2\dv{t}{s}^2+\dv{\vb x}{s}^2=g_{\mu\nu}u^\mu u^\nu=u\cdot u\\
    u&=\dv{x}{s}
\end{align}

Note que,

\begin{align}
    \dd{{\bar t}}&=\gamma\dd{t}-\frac{\gamma}{c^2}\vb v\cdot \dd{\vb x}\\
    \dv{{\bar t}}{t}&=\gamma\qty(1-\frac{v^2}{c^2})\\
    \dv{{\bar t}}{t}&=\frac{1}{\gamma}
\end{align}

Como $\dd{s}=\dd{{\bar t}}$,

\begin{align}
    \dv{x}{s}&=\dv{x}{{\bar t}}\\
    &=\dv{x}{t}\dv{t}{{\bar t}}\\
    &=\gamma \dv{x}{t}=\mqty(\gamma c&\gamma\vb v)^{\textup T}
\end{align}

No limite de $c\rightarrow \infty$ a parte espacial concorda com a velocidade galileana.

O grande ponto de se definir referenciais é de que em referenciais inerciais, é que sobre corpos que não interagem estes concordam com a descrição do movimento deles. Em um referencial, corpos não interagentes movem-se de forma que sua velocidade é constante, ou seja,

\begin{align}
    \dv[2]{x}{s}=0
\end{align}

Associamos uma mudança da trajetória não interagente com a presença de uma interação, chamamos de \emph{força} a quantidade que descreve essa deviação, para levar em conta as diferentes propriedades de objetos introduzimos um parâmetro de proporcionalidade entre a derivada segunda de $x$ com $F$ chamado de \emph{massa},

\begin{align}
    m\dv[2]{x}{s}=F
\end{align}

Para levar em conta objetos que possam mudar de propriedades com o passar de tempo, podemos incluir uma dependência de $m$ com $x$, e fazer,

\begin{align}
    \dv{}{s}\qty[m\dv{x}{s}]=F
\end{align}

Definimos a quantidade 

\begin{align}
    p=m\dv{x}{s}=\mqty(m\gamma c& m\gamma \vb v)^{\textup T}
\end{align}

Que no limite $c\rightarrow \infty$ a parte espacial é levada em $m\vb v$, que é o momento galileano, note que, 

\begin{align}
    p\cdot p&=-m^2\gamma^2c^2+m^2\gamma^2v^2\\
    &=-m^2c^2
\end{align}

E,

\begin{align}
    F\cdot p&=\dv{p}{s}\cdot p\\
    &=\frac12\dv{p\cdot p}{s}=-c^2m\dv{m}{s}=0\\
    -F^0m\gamma c+m\gamma \vb F\cdot \vb v&=0\\
    F^0&=\frac{\vb F\cdot \vb v}{c}
\end{align}

Se analisamos a força no referencial de repouso da partícula $F=\mqty(0&\vb f)^{\textup T}$, no referencial do laboratório,

\begin{align}
    F=\mqty(\frac\gamma c\vb v\cdot \vb f&\vb f+\frac{\gamma - 1}{v^2}\qty(\vb v\cdot \vb f)\vb v)^{\textup T}
\end{align}

Chamamos a quantidade $cp^0$ de energia do sistema, motivos desse nome serão dados posteriormente. Note que,

\begin{align}
    cp^0&=m\gamma c^2\\
    &=\frac{mc^2}{\sqrt{1-\frac{v^2}{c^2}}}\\
    &\approx mc^2+\frac12 mv^2+\mathcal O\qty(\frac{v^4}{c^2})
\end{align}

Chamamos a quantidade $mc^2$ de energia de repouso e $cp^0-mc^2$ de energia cinética, no limite $c\rightarrow \infty$ a energia de repouso é tomada como zero, e a energia cinética tem limite como $m\frac{v^2}{2}$. Podemos escrever ainda,

\begin{align}
    -\frac{E^2}{c^2}+{\vb p}^2&=-m^2c^2\\
    E&=\sqrt{m^2c^4+{\vb p}^2c^2}
\end{align}

Note que a parte temporal da equação diz,

\begin{align}
    \frac1c\dv{E}{s}&=\frac\gamma c\vb f\cdot \vb v\\
    \frac\gamma c\dv{E}{t}&=\frac\gamma c\vb f\cdot \vb v\\
    \dv{E}{t}&=\vb f\cdot \vb v
\end{align}

Logo o observador inercial percebe uma variação da energia da partícula que depende da força e da velocidade, note que poderíamos escrever,

\begin{align}
    \dv{E}{s}&=\gamma\vb f \cdot \dv{\vb x}{t}\\
    \dv{E}{s}&=\vb f\cdot \vb u\\
    \dv{E}{s}&=\vb f\cdot \dv{\vb x}{s}
\end{align}

Isso sugere que há uma mudança de energia associada a uma mudança da posição perante forças, chamamos essa quantidade de \emph{trabalho}

\begin{align}
    \Delta E&=\Delta W=\vb f\cdot \Delta \vb x
\end{align}

Note que se fosse verdade o fato, $\vb f =-\grad V$, teríamos,

\begin{align}
    \Delta W&=-\pdv{V}{\vb x}\cdot \Delta \vb x\\
    \Delta W&=-\Delta V\\
    \Delta W+\Delta V&=0
\end{align}

Podemos pensar nesse caso que há dois tipos diferentes de energia, a energia mecânica da partícula $cp^0$, e a energia devido a interações por causa das forças, assim, chamamos este outro tipo de energia de \emph{energia potencial}, no caso em que $\vb f=-\grad V$, temos que a soma dessas duas partes é constante. Estamos principalmente interessados em forças que possam surgir de tais funções de potencial, note que,

\begin{align}
    F&=\mqty(\frac\gamma c\vb f\cdot \vb v&\vb f+\frac{1}{c^2}\frac{\gamma -1}{v^2}\qty(\vb f\cdot \vb v)\vb v)^{\textup T}\\
    &=\mqty(\frac1c\vb f\cdot \vb u&\vb f+\frac{1}{c^2}\frac{\gamma^2}{\gamma+1}\qty(\vb f\cdot \vb v)\vb v)^{\textup T}\\
    &=\mqty(-\frac1c\dv{V}{s}&\vb f+\frac{1}{c^2}\frac{1}{\gamma+1}\qty(\vb f\cdot\vb u)\vb u)^{\textup T}\\
    &=\mqty(-\frac1c\dv{V}{s}&\vb f-\frac{1}{c^2}\frac{1}{\gamma+1}\dv{V}{s}\vb u)^{\textup T}
\end{align}

De modo a confirmar que,

\begin{align}
    \dv{}{s}\qty(E+V)&=0
\end{align}

Note que, escrevendo,

\begin{align}
    m\dv{u_\mu}{s}&=F_{\mu}\qty(x,u)\\
    m\dv{u_\mu}{s}&=F_{\mu}\qty(x)+\frac1cF_{\mu \nu}\qty(x)u^\nu+\frac{1}{c^2}F_{\mu\nu\sigma}u^\nu u^\sigma+\mathcal O\qty(\frac{u^3}{c^3})
\end{align}

Mas pela relação de ortogonalidade,

\begin{align}
    0&=\frac1cF_{\mu}u^\mu+\frac{1}{c^2}F_{\mu\nu}u^\mu u^\nu+\frac{1}{c^3}F_{\mu\nu\sigma}u^\mu u^\nu u^\sigma+\mathcal O\qty(\frac{u^4}{c^4})
\end{align}

Que implica em $F_\mu\equiv0$ e que $F_{\mu\nu}$ é antissimétrico. Se estamos interessados em forças que são lineares na quadrivelocidade,

\begin{align}
    m\dv{u_\mu}{s}&=\frac1cF_{\mu\nu}\qty(x)u^\nu
\end{align}

Um tensor antissimétrico genérico é escrito como,

\begin{align}
    F_{i0}=f_i\qty(x),\ \ F_{ij}=\epsilon_{ijk}g^k\qty(x)
\end{align}

Note que $\vb f$ é um vetor polar e $\vb g$ é um pseudo-vetor. As equações de movimento são,

\begin{align}
    m\dv{u_0}{s}&=\frac1cF_{0i}u^i\\
    -\frac1c\dv{E}{s}&=-\frac1cf_i\dv{x^i}{s}\\
    \dv{E}{s}&=\vb f\cdot\dv{\vb x}{s}
\end{align}

Para a parte espacial,

\begin{align}
    m\dv{u_i}{s}&=\frac1cF_{ij}u^j+\frac1cF_{i0}u^0\\
    m\dv{u_i}{s}&=\gamma c \frac1cf_i+\frac1c\epsilon_{ijk}u^jg^k\\
    m\dv{\vb u}{s}&=\gamma\vb f+\frac1c\vb u\times\vb g
\end{align}

Ou seja, o quadrivetor força é neste caso,

\begin{align}
    F^\mu\qty(x,u)=\mqty(\frac1c\vb f\qty(x)\cdot \vb u&\gamma\qty(\vb u) \vb f\qty(x)+\frac1c\vb u\times\vb g\qty(x))
\end{align}

Estamos principalmente interessados em casos \emph{conservativos}, que seriam equivalentes à,

\begin{align}
    \vb f\qty(x)&=-\grad V\qty(\vb x)
\end{align}

Logo, 

\begin{align}
    \dv{E}{s}&=-\grad V\qty(\vb x)\cdot\dv{\vb x}{s}\\
    \dv{E}{s}&=-\dv{V}{s}\\
    \dv{E+V}{s}&=0
\end{align}

Se $\vb g\qty(x)=-\grad\times \vb A\qty(\vb x)$, segue que,

\begin{align}
    m\dv{\vb u}{s}&=\gamma\vb f+\frac1c\vb u\times\vb g\\
    m\dv{\vb u}{s}&=-\gamma\grad V-\frac1c\vb u\times\qty(\grad\times \vb A)\\
    m\dv{\vb u}{s}&=-\gamma\grad V+\frac1c\qty[\qty(\vb u\cdot\grad) \vb A-\grad\qty(\vb u\cdot \vb A)]\\
    m\dv{\vb u}{s}&=-\grad \qty[\gamma V\qty(\vb x)+\frac1c\vb u\cdot \vb A\qty(\vb x)]+\frac1c\qty(\vb u\cdot\grad) \vb A\qty(\vb x)\\
    m\dv{\vb u}{s}&=-\grad \qty[\gamma V\qty(\vb x)+\frac1c\vb u\cdot \vb A\qty(\vb x)]+\frac1c\dv{\vb A}{s}\qty(\vb x)\\
    m\dv{\vb u}{s}&=-\grad \qty(\frac1cu^0 V+\frac1c\vb u\cdot \vb A)+\dv{}{s}\qty[\pdv{}{\vb u}\qty(\frac1cu^0V+\frac1c\vb u\cdot \vb A)]\\
    m\dv{\vb u}{s}&=-\grad\qty( \frac{u^\nu}{c} A_\nu)+\dv{}{s}\qty[\pdv{}{\vb u}\qty(\frac{u^\nu}{c} A_\nu)]
\end{align}

E,

\begin{align}
    \frac1c\dv{E}{s}&=\qty(\frac{\vb u}{c}\cdot \grad)A^0\\
    m\dv{u^0}{s}&=\dv{}{s}\qty[\pdv{}{u_0}\qty(\frac{u_\nu}{c} A^\nu)]-\partial^0\qty(\frac{u_\nu}{c} A^\nu)
\end{align}

Juntando,

\begin{align}
    m\dv{u^\mu}{s}&=-\partial^\mu\qty(\frac{u^\nu}{c} A_\nu)+\dv{}{s}\qty[\pdv{}{u_\mu}\qty(\frac{u^\nu}{c} A_\nu)]
\end{align}

Com é claro,

\begin{align}
    A_\mu=\mqty( V&\vb A),\ \ \Phi\qty(x,u)=\frac{u_\nu}{c} A^\nu\qty(x)
\end{align}

Se $A^\mu$ é constante não há forças, o próximo caso é um quadripotencial da forma,

\begin{align}
    A^\mu\qty(x)=kf\qty(x)a^\mu
\end{align}

Então,

\begin{align}
    m\dv{u^\mu}{s}&=-\frac{ka^\nu u_\nu}{c}\partial^\mu f\qty(x)+\frac {ka^\nu}{c}\dv{}{s}\qty[\pdv{}{u_\mu}\qty(u_\nu f\qty(x))]\\
    &=-\frac{ka^\nu u_\nu}{c} \partial^\mu f\qty(x)+\frac {ka^\mu}{c}\dv{}{s}\qty[f\qty(x)]\\
    &=-\frac{ka^\nu u_\nu}{c} \partial^\mu f\qty(x)+\frac {ka^\mu u_\nu}{c}\partial^\nu f\qty(x)\\
    &=\frac{ku^\nu}{c}a^\alpha \partial^\beta f\qty(x)\qty[g_{\nu\beta}g_\alpha^{\ \mu}-g_{\nu\alpha}g_\beta^{\ \mu}]
\end{align}

Vamos analisar o caso especial, $f=\frac12\qty(x-r)^2$, no qual $r$ é um vetor fixo simultâneo à $x$,

\begin{align}
    A^\mu\qty(x)=\frac k2\qty(x^\sigma-r^\sigma)\qty(x_\sigma-r_\sigma)a^\mu
\end{align}

\begin{align}
    m\dv{u^\mu}{s}&=\frac{ku^\nu}{c}a^\alpha \qty(x^\beta-r^\beta)\qty[g_{\nu\beta}g_\alpha^{\ \mu}-g_{\nu\alpha}g_\beta^{\ \mu}]
\end{align}

A parte temporal é,

\begin{align}
    m\dv{u^0}{s}&=-\frac {ka^\nu u_\nu}{c}\qty(x^0-r^0)+\frac{ka^0 u^\nu}{c}\qty(x_\nu-r_\nu)\\
    \frac1c\dv{E}{s}&=\frac{ka^0}{c}\vb u\cdot \qty(\vb x-\vb r)
\end{align}

A parte espacial,

\begin{align}
    m\dv{\vb u}{s}&=-\frac {ka^\nu u_\nu}{c}\qty(\vb x-\vb r)+\frac{k\vb a u^\nu}{c}\qty(x_\nu-r_\nu)\\
    &=ka^0\gamma\qty(\vb x-\vb r)-\frac kc \qty(\vb a\cdot \vb u)\qty(\vb x-\vb r)+\frac{k\vb a }{c}\qty[\vb u\cdot \qty(\vb x-\vb r)]\\
\end{align}

Os três principais casos são $a^2=0,\pm 1$. O caso $a^2=-1$ resume-se à $a^\mu=\mqty(\pm 1&\vb 0)$, que é

\begin{align}
    \dv{E}{s}&=\pm k\vb u\cdot \qty(\vb x-\vb r)=\pm\dv{}{s}\qty[\frac k2\qty(\vb x-\vb r)^2]\\
    \dv{}{s}\qty[E\mp\frac k2\qty(\vb x-\vb r)^2]&=0\\
     m\dv{\vb u}{s}&=\pm k\gamma\qty(\vb x-\vb r)
\end{align}

O caso $a^\mu=\mqty(- 1&\vb 0)$ trata-se de um oscilador harmônico. O caso $a^2=1$ pode ser tomado como $a^\mu=\mqty(0&\vb {\hat n})$,

\begin{align}
    \dv{E}{s}&=0\\
     m\dv{\vb u}{s}&=-\frac kc \qty(\vb {\hat n}\cdot \vb u)\qty(\vb x-\vb r)+\frac{k}{c}\vb{\hat n}\qty[\vb u\cdot \qty(\vb x-\vb r)]\\
     m\dv{\vb u}{s}&=-\frac kc\vb u\times\qty[\qty(\vb x-\vb r)\times \vb {\hat n}]
\end{align}

Note que em uma dimensão espacial essa força é trivial, em duas dimensões, supomos que $\vb{\hat  n} =\pm\vb{\hat y}$ e $\vb r=\vb 0$

\begin{align}
     m\dv{u_x}{s}&=\mp\frac kc xu_y\\
     m\dv{u_y}{s}&=\pm\frac{k}{c}xu_x\\
     m\dv{\cos\theta}{s}&=\mp\frac kc x\sin\theta\\
     m\dv{\sin\theta}{s}&=\pm\frac{k}{c}x\cos\theta\\
     \dv{\theta}{s}&=\pm\frac {k}{mc} x\\
     \dv[2]{\theta}{s}&=\pm\frac {\norm{\vb u}k}{mc}\cos\theta
\end{align}

Resta apenas o caso de $a^2=0$, no qual podemos tomar $a^\mu=\mqty(\pm 1&\vb{\hat n})$,

\begin{align}
    \dv{E}{s}&=\pm k\vb u\cdot \qty(\vb x-\vb r)\\
    m\dv{\vb u}{s}&=\pm k\gamma\qty(\vb x-\vb r)-\frac kc\vb u\times\qty[\qty(\vb x-\vb r)\times\vb {\hat n}]
\end{align}

Todos os casos são equivalentes no limite $c\rightarrow\infty$.

Outro potencial de interesse é,

\begin{align}
    A^\mu&=ka^\mu\qty[\qty(x^\sigma-r^\sigma)\qty(x_\sigma-r_\sigma)]^{-\frac12}
\end{align}

\begin{align}
    m\dv{u^\mu}{s}&=-\frac {ka^\nu u_\nu}{c}\partial^\mu\qty(\qty[\qty(x^\sigma-r^\sigma)\qty(x_\sigma -r_\sigma)]^{-\frac12})+\frac{ka_\nu}{c}\dv{}{s}\qty[\pdv{}{u_\mu}\qty(u^\nu \qty[\qty(x^\sigma-r^\sigma)\qty(x_\sigma-r_\sigma))]^{-\frac12}]\\
    &=\frac{ka^\nu u_\nu}{c}\frac{\qty(x^\mu-r^\mu)}{\qty[\qty(x^\sigma-r^\sigma)\qty(x_\sigma -r_\sigma)]^{\frac32}}-\frac{ka^\mu u_\nu}{c}\frac{\qty(x^\nu-r^\nu)}{\qty[\qty(x^\sigma-r^\sigma)\qty(x_\sigma -r_\sigma)]^{\frac32}}
\end{align}

A parte temporal é,

\begin{align}
    m\dv{u^0}{s}&=-\frac{k}{c}\frac{a^0}{\norm{\vb x-\vb r}^3}\vb u\cdot \qty(\vb x-\vb r)\\
    \dv{E}{s}&=-\frac{ka^0}{\norm{\vb x-\vb r}^3}\vb u\cdot \qty(\vb x-\vb r)
\end{align}

A parte espacial,

\begin{align}
    m\dv{\vb u}{s}&=-ka^0 \gamma\frac{\qty(\vb x-\vb r)}{\norm{\vb x-\vb r}^{3}}+\frac{k\qty(\vb a\cdot \vb u)}{c}\frac{\qty(\vb x-\vb r)}{\norm{\vb x-\vb r}^{3}}-\frac{k\vb a}{c}\vb u \cdot \frac{\qty(\vb x-\vb r)}{\norm{\vb x-\vb r}^{3}}\\
    &=-ka^0\gamma\frac{\qty(\vb x-\vb r)}{\norm{\vb x-\vb r}^{3}}+\frac kc\vb u\times\qty[\frac{\qty(\vb x-\vb r)}{\norm{\vb x-\vb r}^3}\times \vb a]
\end{align}

Novamente vamos analisar os casos $a^\mu=\mqty(\pm1&\vb 0)$,

\begin{align}
    \dv{E}{s}&=\mp k\vb u\cdot\frac{\qty(\vb x-\vb r)}{\norm{\vb x-\vb r}^3}=\mp \dv{}{s}\qty[\frac{-k}{\norm{\vb x-\vb r}}] \\
    \dv{}{s}\qty[E\pm \qty(\frac{-k}{\norm{\vb x-\vb r}})]&=0\\
    m\dv{\vb u}{s}&=\mp k \gamma\frac{\qty(\vb x-\vb r)}{\norm{\vb x-\vb r}^{3}}
\end{align}

Caso $a^\mu =\mqty(0&\vb{\hat n})$,

\begin{align}
    \dv{E}{s}&=0\\
    m\dv{\vb u}{s}&=\frac{k\qty(\vb {\hat n}\cdot \vb u)}{c}\frac{\qty(\vb x-\vb r)}{\norm{\vb x-\vb r}^{3}}-\frac{k\vb {\hat n}}{c}\vb u \cdot \frac{\qty(\vb x-\vb r)}{\norm{\vb x-\vb r}^{3}}\\
    &=\frac kc\vb u\times\qty[\frac{\qty(\vb x-\vb r)}{\norm{\vb x-\vb r}^3}\times \vb {\hat n}]
\end{align}

E o caso $a^\mu=\mqty(\pm 1&\vb {\hat n})$,

\begin{align}
    \dv{E}{s}&=\mp k\vb u\cdot\frac{\qty(\vb x-\vb r)}{\norm{\vb x-\vb r}^3}=\mp \dv{}{s}\qty[\frac{-k}{\norm{\vb x-\vb r}}]\\
    m\dv{\vb u}{s}&=\mp k \gamma\frac{\qty(\vb x-\vb r)}{\norm{\vb x-\vb r}^{3}}+\frac{k\qty(\vb {\hat n}\cdot \vb u)}{c}\frac{\qty(\vb x-\vb r)}{\norm{\vb x-\vb r}^{3}}-\frac{k\vb {\hat n}}{c}\vb u \cdot \frac{\qty(\vb x-\vb r)}{\norm{\vb x-\vb r}^{3}}\\
    &=\mp k \gamma\frac{\qty(\vb x-\vb r)}{\norm{\vb x-\vb r}^{3}}+\frac kc\vb u\times\qty[\frac{\qty(\vb x-\vb r)}{\norm{\vb x-\vb r}^3}\times \vb {\hat n}]
\end{align}

Além do movimento translacional, temos também o movimento rotacional, descrito por,

\begin{align}
    \dv{mu^\sigma}{s}&=F^\sigma\\
    \epsilon_{\mu\nu\rho\sigma}x^\rho\dv{mu^\sigma}{s}&=\epsilon_{\mu\nu\rho\sigma}x^\rho F^\sigma\\
    \dv{}{s}\qty[\epsilon_{\mu\nu\rho\sigma}x^\rho mu^\sigma]&=\epsilon_{\mu\nu\rho\sigma}x^\rho F^\sigma\\
    \dv{}{s}\qty[\epsilon_{\mu\nu\rho\sigma}x^\rho p^\sigma]&=\epsilon_{\mu\nu\rho\sigma}x^\rho F^\sigma\\
    \dv{}{s}\qty[L_{\mu\nu}]&=\tau_{\mu\nu}
\end{align}

No qual $L_{\mu\nu}$ é dito o tensor momento angular e $\tau_{\mu\nu}$ é o tensor de torque. Note que,

\begin{align}
    \tau_{\mu\nu}&=\frac1c\epsilon_{\mu\nu\rho\sigma}x^\rho F^{\sigma\alpha}u_\alpha\\
    &=\frac1c\epsilon_{\mu\nu\rho\sigma}x^\rho\qty(\partial^\alpha A^\sigma-\partial^\sigma A^\alpha)u_\alpha
\end{align}

Definimos, $L_{0i}=L_i$, e $L_{ij}=\epsilon_{ijk}K^k$, $\tau_{0i}=\tau_i$, e $\tau_{ij}=\epsilon_{ijk}\kappa^k$ 

\begin{align}
    \tau_{0i}&=\frac1c\epsilon_{0ijk}x^j\qty(\partial^\alpha A^k-\partial^k A^\alpha)u_\alpha\\
    &=\frac1c\epsilon_{0ijk}x^j\qty(a^k\partial^\alpha f\qty(x)-a^\alpha\partial^k f\qty(x))u_\alpha\\
    &=\frac{kg\qty(x)}{c}\epsilon_{0ijk}x^j\qty(a^k\qty(x^\alpha-r^\alpha)-a^\alpha\qty(x^k-r^k))u_\alpha\\
    &=\frac{kg\qty(x)}{c}\epsilon_{0ijk}x^j\qty(a^k\qty(\vb x-\vb r)\cdot \vb u-\qty(a\cdot u)\qty(x^k-r^k))\\
    &=\frac{kg\qty(x)}{c}\vb x\times \vb a \qty[\qty(\vb x-\vb r)\cdot \vb u]-\frac{kg\qty(x)}{c}\vb x\times\qty(\vb x-\vb r)\qty(a\cdot u)\\
    &=\frac{kg\qty(x)}{c}\vb x\times \vb a \qty[\qty(\vb x-\vb r)\cdot \vb u]+a^0\gamma kg\qty(x)\vb x\times\qty(\vb x-\vb r)-\qty(\vb a\cdot \vb u)\frac{kg\qty(x)}{c}\vb x\times \qty(\vb x-\vb r)\\
    &=kg\qty(x)\vb x\times\qty{a^0\gamma\qty(\vb x-\vb r)+\frac{\qty(\qty(\vb x-\vb r)\cdot \vb u)}{c}\vb a-\frac{\qty(\vb a\cdot \vb u)}{c}\qty(\vb x -\vb r)}\\
    \boldsymbol \tau&=kg\qty(x)\vb x\times\qty{a^0\gamma\qty(\vb x-\vb r)+\frac{\vb u}{c}\times \qty[\vb a\times\qty(\vb x-\vb r)]}
\end{align}

Suponha que possuímos mais de um corpo, descreveremos as coordenadas de cada um destes por $\tensor[_i]{x}{^\mu}$, no qual $i$ enumera as partículas. Nessa situação não temos mais o tempo próprio para usar como parâmetro de trajetória, se $\lambda$ é um parâmetro que foi utilizado para a parametrização da trajetória, então temos que,

\begin{align}
    \tensor{\dot{x}}{^\mu}=\dv{\tensor{x}{^\mu}}{\lambda}=\mqty(c\dv{t}{\lambda}&\dv{\vb x}{\lambda})
\end{align}

Definimos então o fator gamma generalizado como,

\begin{align}
    \gamma=\frac{1}{\sqrt{-\frac{1}{c^2}\dv{\tensor{x}{_\mu}}{\lambda}\dv{\tensor{x}{^\mu}}{\lambda}}}
\end{align}

Certamente, se lambda é o tempo próprio da partícula temos que,

\begin{align}
    u^\mu=\dot{x}^\mu
\end{align}

E,

\begin{align}
    \gamma=1
\end{align}

Outro caso de interesse é claro, $\lambda=t$, para o qual,

\begin{align}
    \gamma=\frac{1}{\sqrt{1-\frac{v^2}{c^2}}}
\end{align}

Note que,

\begin{align}
    \gamma \dot{x}^\mu&=\frac{1}{\sqrt{-\frac{1}{c^2}\dv{x_\nu}{\lambda}\dv{x^\nu}{\lambda}}}\dv{x^\mu}{\lambda}\\
    &=\dv{\lambda}{t}\frac{1}{\sqrt{-\frac{1}{c^2}\dv{x_\nu}{t}\dv{x^\nu}{t}}}\dv{x^\mu}{\lambda}\\
    &=\frac{1}{\sqrt{1-\frac{v^2}{c^2}}}\dv{x^\mu}{t}\\
    &=u^\mu
\end{align}

Obtemos então uma maneira genérica de tratar a quadrivelocidade.

%%\printbibliography[heading=subbibliography]
%%\end{refsection}

%%%%%%%%%%%%%%%%%%%%%%%%%%%%%%%%%%%%%%%%%%%%%%%%%%%%%%%%%%%%%

%\begin{refsection}
\section{Formalismo Lagrangiano}

Gostaríamos agora de expressar nossas equações de movimento com notação não vetorial, para isto precisamos recorrer ao cálculo variacional, gostaríamos de obter a expressão de movimento de uma partícula livre, $$\dv{p}{s}=0$$ como resultado da extremização de alguma quantidade que seja invariante entre observadores inerciais (motivação?), caso o movimento da partícula se dê entre $t_1$ e $t_2$, a expressão que procuramos deve ser da forma, $$S[x,\cdots]=\int\limits_{t_1}^{t_2}\dd{t'}L\qty[x\qty(t'),\cdots]$$ O processo do cálculo variacional,

\begin{align}
    \fdv{S}{x^\mu\qty(t)}&=\int\limits_{t_1}^{t_2}\dd{t'}\qty{\pdv{L}{x^\mu\qty(t')}\delta\qty(t'-t)+\pdv{L}{{\dot x}^\mu\qty(t')}\delta'\qty(t'-t)+\cdots}\\
    &=\int\limits_{t_1}^{t_2}\dd{t'}\qty{\pdv{L}{x^\mu\qty(t')}\delta\qty(t'-t)+\dv{}{t'}\qty(\pdv{L}{{\dot x}^\mu\qty(t')}\delta\qty(t'-t))-\dv{}{t'}\pdv{L}{{\dot x}^\mu\qty(t')}\delta\qty(t'-t)+\cdots}\\
    &=\pdv{L}{x^\mu\qty(t)}-\dv{}{t}\pdv{L}{{\dot x}^\mu\qty(t)}=0
\end{align}

A resposta mais trivial seria $L\equiv \textup{cte}$ que não possui nenhuma dinâmica, a segunda opção é, eliminando ${\dot x}^0\equiv c$,

\begin{align}
    \dv{}{t}\pdv{L}{\dot{\vb x}}&=0=\dv{}{s}\qty(m\gamma\qty(\dot{\vb x})\dot{\vb x})\\
    \pdv{L}{\dot{\vb x}}&=m\gamma\qty(\dot {\vb x})\dot{\vb x}\\
    \pdv{L}{\dot{\vb x}}&=-\dv{}{\dot{\vb x}}\qty(\frac{mc^2}{\gamma\qty(\dot{\vb x})})
\end{align}

Podemos então descrever uma partícula livre por,

\begin{align}
    S[\dot x]&=-mc^2\int\limits_{t_1}^{t_2}\dd{t}\frac{1}{\gamma\qty(\dot{\vb x})}\\
    &=-mc^2\int\limits_{t_1}^{t_2}\dd{t}\sqrt{1-\frac{1}{c^2}\dv{\vb x}{t}\cdot\dv{\vb x}{t}}\\
    &=-mc\int\limits_{t_1}^{t_2}\dd{t}\sqrt{c^2\dv{t}{t}\dv{t}{t}-\dv{\vb x}{t}\cdot\dv{\vb x}{t}}\\
    &=-mc\int\limits_{t_1}^{t_2}\dd{t}\sqrt{\dv{x^0}{t}\dv{x^0}{t}-\dv{\vb x}{t}\cdot\dv{\vb x}{t}}\\
    &=-mc\int\limits_{t_1}^{t_2}\dd{t}\sqrt{-\dv{x_\mu}{t}\dv{x^\mu}{t}}\\
    &=-mc\int\limits_{\lambda_1}^{\lambda_2}\dd{\lambda}\sqrt{-\dv{x_\mu}{\lambda}\dv{x^\mu}{\lambda}}
\end{align}

Onde $\lambda$ é um parâmetro introduzido para parametrizar a trajetória da partícula, após a variação da ação podemos tomar este como sendo $t$ ou $s$. Note que esta ação é invariante pela reparametrização da linha de mundo,

\begin{align}
    S&=-mc\int\limits_{\lambda_1'}^{\lambda_2'}\dd{\lambda'}\dv{\lambda}{\lambda'}\sqrt{-\dv{x_\mu}{\lambda'}\dv{x^\mu}{\lambda'}\qty(\dv{\lambda'}{\lambda})^2}\\
    &=-mc\int\limits_{\lambda_1'}^{\lambda_2'}\dd{\lambda'}\sqrt{-\dv{x_\mu}{\lambda'}\dv{x^\mu}{\lambda'}}=S'
\end{align}

Para tornar mais simples a expressão da ação e poder introduzir a noção de partículas com $m=0$, vamos introduzir um artifício novo, uma nova variável $\alpha\qty(\lambda)$,

\begin{align}
    \tilde S&=\frac12\int\limits_{\lambda_1}^{\lambda_2}\dd{\lambda}\qty[\alpha\dv{x_\mu}{\lambda}\dv{x^\mu}{\lambda}-\frac{m^2c^2}{\alpha}]
\end{align}

As equações de movimento são,

\begin{align}
    \fdv{\tilde S}{\alpha}&=\frac12\dv{x_\mu}{\lambda}\dv{x^\mu}{\lambda}+\frac12 \frac{m^2c^2}{\alpha^2}=0\\
    \fdv{\tilde S}{x_\mu}&=-\dv{}{\lambda}\qty[\alpha\dv{x^\mu}{\lambda}]=0\\
    \alpha&=\frac{mc}{\sqrt{-\dv{x_\mu}{\lambda}\dv{x^\mu}{\lambda}}}
\end{align}

Note que,

\begin{align}
    \tilde S&=-\frac12\int\limits_{\lambda_1}^{\lambda_2}\dd{\lambda}\qty[-\frac{mc}{\sqrt{-\dv{x_\mu}{\lambda}\dv{x^\mu}{\lambda}}}\dv{x_\nu}{\lambda}\dv{x^\nu}{\lambda}+mc\sqrt{-\dv{x_\nu}{\lambda}\dv{x^\nu}{\lambda}}]\\
    &=-mc\int\limits_{\lambda_1}^{\lambda_2}\dd{\lambda}\sqrt{-\dv{x_\mu}{\lambda}\dv{x^\mu}{\lambda}}
\end{align}

Para incluir forças podemos notar o que foi desenvolvido no capítulo anterior para que,

\begin{align}
    m\dv{u^\mu}{s}&=-\partial^\mu\Phi+\dv{}{s}\qty[\pdv{}{u_\mu}\Phi]\\
    \dv{}{t}\qty[m\gamma\qty(\dot{\vb x})\dot{\vb x}]&+\grad\frac{\Phi}{\gamma\qty(\dot{\vb x})}-\dv{}{t}\qty[\pdv{}{\dot{\vb x}}\frac{\Phi}{\gamma\qty(\dot{\vb x})}]=0\\
    \dv{}{t}\qty[\pdv{}{\dot{\vb x}}\qty(-\frac{mc^2}{\gamma\qty(\dot{\vb x})})]&+\grad\frac{\Phi}{\gamma\qty(\dot{\vb x})}-\dv{}{t}\qty[\pdv{}{\dot{\vb x}}\frac{\Phi}{\gamma\qty(\dot{\vb x})}]=0
\end{align}

Que implica em,

\begin{align}
    L&=-\frac{mc^2}{\gamma\qty(\dot{\vb x})}-\frac{\Phi\qty(\vb x,\dot{\vb x})}{\gamma\qty(\dot{\vb x})}\\
    S&=\int\limits_{t_1}^{t_2}\dd{t}\qty(-\frac{mc^2}{\gamma\qty(\dot{\vb x})}-\frac{\Phi\qty(\vb x,\dot{\vb x})}{\gamma\qty(\dot{\vb x})})=\int\limits_{t_1}^{t_2}\dd{t}\qty(-\frac{mc^2}{\gamma\qty(\dot{\vb x})}-\frac{\gamma\qty(\dot{\vb x})}{\gamma\qty(\dot{\vb x})}\frac1c\dv{x^\mu}{t}A_\mu)\\
    &=-\int\limits_{\lambda_1}^{\lambda_2}\dd{\lambda}\qty(mc\sqrt{-\dv{x_\mu}{\lambda}\dv{x^\mu}{\lambda}}+\frac1c\dv{x^\mu}{\lambda}A_\mu)\\
    &=-\int\limits_{\lambda_1}^{\lambda_2}\dd{\lambda}\qty(-\frac12\frac{mc}{\sqrt{-\dv{x_\mu}{\lambda}\dv{x^\mu}{\lambda}}}\dv{x_\nu}{\lambda}\dv{x^\nu}{\lambda}+\frac12 m^2c^2\frac{\sqrt{-\dv{x_\mu}{\lambda}\dv{x^\mu}{\lambda}}}{mc}+\frac1c\dv{x^\mu}{\lambda}A_\mu)\\
    &=\int\limits_{\lambda_1}^{\lambda_2}\dd{\lambda}\qty(\frac12\alpha\dv{x_\nu}{\lambda}\dv{x^\nu}{\lambda}-\frac12 \frac{m^2c^2}{\alpha}-\frac1c\dv{x^\mu}{\lambda}A_\mu)
\end{align}

Note que temos portanto equações de movimento como,

\begin{align}
    \fdv{S}{x_\mu}&=0\\
    -\dv{}{\lambda}\qty(\alpha\dv{x^\mu}{\lambda})-\frac1c\dv{x^\nu}{\lambda}\partial^\mu A_\nu+\frac1c\dv{}{\lambda}A^\mu&=0
\end{align}

Que é equivalente à o que foi discutido no capítulo anterior se junta da segunda equação,

\begin{align}
    \fdv{S}{\alpha}&=0\\
    \frac12\dv{x_\mu}{\lambda}\dv{x^\mu}{\lambda}+\frac12\frac{m^2c^2}{\alpha^2}&=0
\end{align}

Note que $\alpha$ resulta da arbitrariedade da parametrização por $\lambda$. Podemos agora analisar o caso especial de $m=0$, suponha que $A_\mu\equiv 0$, as equações de movimento são então,

\begin{align}
    \dv{}{\lambda}\qty[\alpha\dv{x^\mu}{\lambda}]&=0\\
    \dv{x_\mu}{\lambda}\dv{x^\mu}{\lambda}&=0
\end{align}

Da segunda equação obtemos que,

\begin{align}
    c\dv{t}{\lambda}&=\norm{\dv{\vb x}{\lambda}}
\end{align}

Como podemos escolher parametrizar por $\lambda=t$, segue que,

\begin{align}
    c&=\norm{\dv{\vb x}{t}}
\end{align}

Podemos então tomar $\alpha\qty(t) =1$ nessa parametrização. De forma que uma partícula com massa nula deve necessariamente mover-se com velocidade da luz. Se $m=0$ e $A^\mu=\mqty(A^0&\vb 0) $ com $\partial_0A^\mu=0$,

\begin{align}
    \dv{}{\lambda}\qty(\alpha\dv{x^\mu}{\lambda})&=\frac1c\dv{x_\nu}{\lambda}\qty(\partial^\nu A^\mu-\partial^\mu A^\nu)\\
    \dv{x_\mu}{\lambda}\dv{x^\mu}{\lambda}&=0
\end{align}

Novamente, trabalhando na parametrização de $\lambda = t$, temos novamente que $\norm{\dv{\vb x}{t}}=c$, a parte temporal da primeira equação resulta em,

\begin{align}
    \dv{}{t}\qty(\alpha c)&=\frac1c\dv{x_nu}{t}\qty(\partial^\nu A^0-\partial^0 A^\nu)\\
    \dv{\alpha}{t}&=\frac{1}{c^2}\dv{x_\nu}{t}\partial^\nu A^0\\
    \dv{\alpha}{t}&=-\frac{1}{c^2}c\partial^0 A^0+\frac{1}{c^2}\dv{\vb x}{t}\cdot \grad A^0\\
    \dv{\alpha}{t}&=\frac1c\vb {\hat n}\cdot\grad A^0
\end{align}

A parte espacial é,

\begin{align}
    \dv{}{t}\qty(\alpha\dv{\vb x}{t})&=\frac1c\dv{x_\nu}{t}\qty(\partial^\nu\vb A-\grad A^\nu)\\
    \dv{}{t}\qty(\alpha\vb {\hat n})&=-\frac{1}{c^2}\dv{x_0}{t}\grad A^0\\
    \dv{}{t}\qty(\alpha\vb {\hat n})&=\frac{1}{c}\grad A^0
\end{align}

Combinando as duas,

\begin{align}
    \alpha\dv{\vb{\hat n}}{t}&=\frac1c\grad A^0-\frac{\vb {\hat n}}{c}\qty(\vb{\hat n}\cdot\grad A^0)
\end{align}

Se desejamos descrever a dinâmica de mais de uma partículas interagentes fazemos,

\begin{align}
    S&=\frac12\sum\limits_i\int\limits_{\lambda_1}^{\lambda_2}\dd{\lambda}\qty[\tensor[_i]{\alpha}{}\dv{\tensor[_i]{x}{_\mu}}{\lambda}\dv{\tensor[_i]{x}{^\mu}}{\lambda}-\frac{\tensor[_i]{m}{^2}c^2}{\tensor[_i]{\alpha}{}}]
\end{align}

Para partículas interagentes certamente cada uma interage com outra segundo um potencial $A^\mu$,

\begin{align}
    S&=\sum\limits_i\int\limits_{\lambda_1}^{\lambda_2}\dd{\lambda}\qty(\frac12\tensor[_i]{\alpha}{}\dv{\tensor[_i]{x}{_\nu}}{\lambda}\dv{\tensor[_i]{x}{^\nu}}{\lambda}-\frac12 \frac{\tensor[_i]{m}{^2}c^2}{\tensor[_i]{\alpha}{}}-\frac1c\dv{\tensor[_i]{x}{^\mu}}{\lambda}\tensor[_i]{A}{_\mu}\qty(\tensor[_l]{x}{}))
\end{align}

Supomos que $\tensor[_i]{m}{}>0$, de modo que a variação de $S$ gera,

\begin{align}
    \fdv{S}{\tensor[_i]{\alpha}{}}&=\frac12\dv{\tensor[_i]{x}{_\mu}}{\lambda}\dv{\tensor[_i]{x}{^\mu}}{\lambda}+\frac12\frac{\tensor[_i]{m}{^2}c^2}{\tensor[_i]{\alpha}{^2}}=0\\
    \tensor[_i]{\alpha}{}&=\frac{\tensor[_i]{m}{}c}{\sqrt{-\dv{\tensor[_i]{x}{_\mu}}{\lambda}\dv{\tensor[_i]{x}{^\mu}}{\lambda}}}\\
    \tensor[_i]{\alpha}{}&=\frac{\tensor[_i]{m}{}}{\sqrt{-\frac{1}{c^2}\dv{\tensor[_i]{x}{_\mu}}{\lambda}\dv{\tensor[_i]{x}{^\mu}}{\lambda}}}
\end{align}

E,

\begin{align}
    \fdv{S}{\tensor[_i]{x}{_\mu}}&=-\frac1c\sum\limits_l\tensor[_l]{\dot{x}}{^\nu}\tensor[_i]{\partial}{^\mu}\tensor[_l]{A}{_\nu}+\frac1c\dv{\tensor[_i]{A}{^\mu}}{\lambda}-\dv{}{\lambda}\qty[\tensor[_i]{\alpha}{}\tensor[_i]{\dot{x}}{^\mu}]=0\\
    \dv{}{\lambda}\qty[\tensor[_i]{\alpha}{}\tensor[_i]{\dot{x}}{^\mu}]&=-\frac1c\sum\limits_l\tensor[_l]{\dot{x}}{^\nu}\tensor[_i]{\partial}{^\mu}\tensor[_l]{A}{_\nu}+\frac1c\dv{\tensor[_i]{A}{^\mu}}{\lambda}\\
    \dv{}{\lambda}\qty[\tensor[_i]{\alpha}{}\tensor[_i]{\dot{x}}{^\mu}]&=-\frac1c\sum\limits_l\tensor[_l]{\dot{x}}{^\nu}\tensor[_i]{\partial}{^\mu}\tensor[_l]{A}{_\nu}+\frac1c\sum\limits_l\tensor[_l]{\dot{x}}{^\nu}\tensor[_l]{\partial}{_\nu}\tensor[_i]{A}{^\mu}\\
    \dv{}{\lambda}\qty[\tensor[_i]{m}{}\tensor[_i]{u}{^\mu}]&=-\frac1c\sum\limits_l\tensor[_l]{\dot{x}}{^\nu}\tensor[_i]{\partial}{^\mu}\tensor[_l]{A}{_\nu}+\frac1c\sum\limits_l\tensor[_l]{\dot{x}}{^\nu}\tensor[_l]{\partial}{_\nu}\tensor[_i]{A}{^\mu}\\
    \dv{}{\lambda}\qty[\tensor[_i]{m}{}\tensor[_i]{u}{^\mu}]&=\frac1c\sum\limits_l\tensor[_l]{\dot{x}}{^\nu}\qty[\tensor[_l]{\partial}{_\nu}\tensor[_i]{A}{^\mu}-\tensor[_i]{\partial}{^\mu}\tensor[_l]{A}{_\nu}]
\end{align}

Supomos que,

\begin{align}
    \tensor[_i]{A}{^\mu}&=\sum\limits_j\tensor[_{ji}]{A}{^\mu}\qty(\tensor[_i]{x}{}-\tensor[_j]{x}{})=\sum\limits_j\tensor[_{ji}]{f}{}\qty(\tensor[_i]{x}{}-\tensor[_j]{x}{})\tensor[_{ji}]{a}{^\mu}
\end{align}

Então,

\begin{align}
    \dv{}{\lambda}\qty[\tensor[_i]{m}{}\tensor[_i]{u}{^\mu}]&=\frac1c\sum\limits_{l,k}\tensor[_l]{\dot{x}}{^\nu}\qty[\tensor[_l]{\partial}{_\nu}\tensor[_{ki}]{A}{^\mu}-\tensor[_i]{\partial}{^\mu}\tensor[_{kl}]{A}{_\nu}]\\
    &=\frac2c\sum\limits_{l,k}\tensor[_l]{\dot{x}}{^\nu}\qty[\tensor[_{ki}]{a}{^\mu}\tensor[_{ki}]{{f'}}{}\qty{\delta_{li}\qty(\tensor[_i]{x}{_\nu}-\tensor[_k]{x}{_\nu})-\delta_{lk}\qty(\tensor[_i]{x}{_\nu}-\tensor[_k]{x}{_\nu})}-\tensor[_{kl}]{a}{_\nu}\tensor[_{kl}]{{f'}}{}\qty{\delta_{il}\qty(\tensor[_l]{x}{^\mu}-\tensor[_k]{x}{^\mu})-\delta_{ik}\qty(\tensor[_l]{x}{^\mu}-\tensor[_k]{x}{^\mu})}]\\
    &=\frac2c\sum\limits_{k}\qty{\tensor[_{ki}]{a}{^\mu}\tensor[_{ki}]{{f'}}{}\qty(\tensor[_i]{x}{_\nu}-\tensor[_k]{x}{_\nu})\qty[\tensor[_i]{\dot{x}}{^\nu}-\tensor[_k]{\dot{x}}{^\nu}]}-\frac2c\sum\limits_k \tensor[_{ki}]{{f'}}{}\qty(\tensor[_i]{x}{^\mu}-\tensor[_k]{x}{^\mu})\qty(\tensor[_{ki}]{a}{_\nu}\tensor[_i]{\dot{x}}{^\nu}+\tensor[_{ik}]{a}{_\nu}\tensor[_k]{\dot{x}}{^\nu})
\end{align}

A parte temporal é,

\begin{align}
    \frac1c\dv{\tensor[_i]{E}{}}{\lambda}&=\frac2c\sum\limits_{k}\qty{\tensor[_{ki}]{a}{^0}\tensor[_{ki}]{{f'}}{}\qty(\tensor[_i]{x}{^\mu}-\tensor[_k]{x}{^\mu})\qty[\tensor[_i]{\dot{x}}{_\mu}-\tensor[_k]{\dot{x}}{_\mu}]}-2\sum\limits_k \tensor[_{ki}]{{f'}}{}\qty(\tensor[_i]{t}{}-\tensor[_k]{t}{})\qty(\tensor[_{ki}]{a}{_\nu}\tensor[_i]{\dot{x}}{^\nu}+\tensor[_{ik}]{a}{_\nu}\tensor[_k]{\dot{x}}{^\nu})\\
    \dv{\tensor[_i]{E}{}}{\lambda}&=2\sum\limits_{k}\qty{\tensor[_{ki}]{a}{^0}\tensor[_{ki}]{{f'}}{}\qty(\tensor[_i]{x}{^\mu}-\tensor[_k]{x}{^\mu})\cdot\dv{}{\lambda}\qty(\tensor[_i]{x}{_\mu}-\tensor[_k]{x}{_\mu})}-2c\sum\limits_k \tensor[_{ki}]{{f'}}{}\qty(\tensor[_i]{t}{}-\tensor[_k]{t}{})\qty(\tensor[_{ki}]{a}{_\nu}\tensor[_i]{\dot{x}}{^\nu}+\tensor[_{ik}]{a}{_\nu}\tensor[_k]{\dot{x}}{^\nu})\\
    \dv{\tensor[_i]{E}{}}{\lambda}&=\sum\limits_{k}\qty{\tensor[_{ki}]{a}{^0}\tensor[_{ki}]{{f'}}{}\dv{}{\lambda}\qty(\tensor[_i]{x}{}-\tensor[_k]{x}{})^2}-2c\sum\limits_k \tensor[_{ki}]{{f'}}{}\qty(\tensor[_i]{t}{}-\tensor[_k]{t}{})\qty(\tensor[_{ki}]{a}{_\nu}\tensor[_i]{\dot{x}}{^\nu}+\tensor[_{ik}]{a}{_\nu}\tensor[_k]{\dot{x}}{^\nu})
\end{align}

Se $\tensor[_{ki}]{a}{^\mu}$ for antissimétrico, certamente vale que,

\begin{align}
    \sum\limits_i\dv{\tensor[_i]{E}{}}{\lambda}&=\sum\limits_{k,i}\qty{\tensor[_{ki}]{a}{^0}\tensor[_{ki}]{{f'}}{}\dv{}{\lambda}\qty(\tensor[_i]{x}{}-\tensor[_k]{x}{})^2}-2c\sum\limits_{k,i}\tensor[_{ki}]{a}{_\nu} \tensor[_{ki}]{{f'}}{}\qty(\tensor[_i]{t}{}-\tensor[_k]{t}{})\qty(\tensor[_i]{\dot{x}}{^\nu}-\tensor[_k]{\dot{x}}{^\nu})=0
\end{align}

A parte espacial,

\begin{align}
    \dv{}{\lambda}\qty[\tensor[_i]{m}{}\tensor[_i]{\vb {u}}{}]&=\frac2c\sum\limits_{k}\qty{\tensor[_{ki}]{\vb{a}}{}\tensor[_{ki}]{{f'}}{}\qty(\tensor[_i]{x}{_\nu}-\tensor[_k]{x}{_\nu})\qty[\tensor[_i]{\dot{x}}{^\nu}-\tensor[_k]{\dot{x}}{^\nu}]}-\frac2c\sum\limits_k\qty{ \tensor[_{ki}]{a}{_\nu}\tensor[_{ki}]{{f'}}{}\qty(\tensor[_i]{\vb{x}}{}-\tensor[_k]{\vb{x}}{})\qty[\tensor[_i]{\dot{x}}{^\nu}-\tensor[_k]{\dot{x}}{^\nu}]}\\
    &=\frac2c\sum\limits_k\tensor[_{ki}]{{f'}}{}\qty(\tensor[_i]{\dot{x}}{^\nu}-\tensor[_k]{\dot{x}}{^\nu})\qty{\tensor[_{ki}]{\vb{a}}{}\qty(\tensor[_i]{x}{_\nu}-\tensor[_k]{x}{_\nu})- \tensor[_{ki}]{a}{_\nu}\qty(\tensor[_i]{\vb{x}}{}-\tensor[_k]{\vb{x}}{})}\\
    &=\frac2c\sum\limits_k\tensor[_{ki}]{{f'}}{}\qty(\tensor[_i]{\dot{\vb{x}}}{}-\tensor[_k]{\dot{\vb{x}}}{})\times\qty{\tensor[_{ki}]{\vb{a}}{}\times\qty(\tensor[_i]{\vb{x}}{}-\tensor[_k]{\vb{x}}{})}-2c\sum\limits_k\tensor[_{ki}]{{f'}}{}\qty(\tensor[_i]{\dot{t}}{}-\tensor[_k]{\dot{t}}{})\qty{\tensor[_{ki}]{\vb{a}}{}\qty(\tensor[_i]{t}{}-\tensor[_k]{t}{})- \tensor[_{ki}]{a}{^0}\qty(\tensor[_i]{\vb{x}}{}-\tensor[_k]{\vb{x}}{})}
\end{align}

Como $\tensor[_{ki}]{{f'}}{}$ é simétrico, e $\tensor[_{ki}]{\vb{a}}{}$ antissimétrico, temos que,

\begin{align}
    \sum\limits_i\dv{}{\lambda}\qty[\tensor[_i]{m}{}\tensor[_i]{\vb {u}}{}]&=\frac2c\sum\limits_{k,i}\tensor[_{ki}]{{f'}}{}\qty(\tensor[_i]{\dot{\vb{x}}}{}-\tensor[_k]{\dot{\vb{x}}}{})\times\qty{\tensor[_{ki}]{\vb{a}}{}\times\qty(\tensor[_i]{\vb{x}}{}-\tensor[_k]{\vb{x}}{})}-2c\sum\limits_{k,i}\tensor[_{ki}]{{f'}}{}\qty(\tensor[_i]{\dot{t}}{}-\tensor[_k]{\dot{t}}{})\qty{\tensor[_{ki}]{\vb{a}}{}\qty(\tensor[_i]{t}{}-\tensor[_k]{t}{})- \tensor[_{ki}]{a}{^0}\qty(\tensor[_i]{\vb{x}}{}-\tensor[_k]{\vb{x}}{})}=0
\end{align}

Note que,

\begin{align}
    \sum\limits_i\tensor[_i]{m}{}\tensor[_i]{\vb{u}}{}&=\sum\limits_i\tensor[_i]{m}{}\tensor[_i]{\gamma}{}\tensor[_i]{\dot{\vb{x}}}{}\\
    &=\sum\limits_i\frac{\tensor[_i]{E}{}}{c^2}\dv{\lambda}{\tensor[_i]{t}{}}\tensor[_i]{\dot{\vb{x}}}{}\\
    &=\sum\limits_i\frac{\tensor[_i]{E}{}}{c^2}\tensor[_i]{\vb{v}}{}={\vb V}\sum\limits_i\frac{\tensor[_i]{E}{}}{c^2}\equiv \mbox{cte}
\end{align}

No qual interpretamos $\vb V$ como a velocidade do centro de massa. Podemos tentar definir um conceito análogo de \emph{centro de massa}, como,

\begin{align}
    \vb{R}&=\frac{\sum\limits_i\tensor[_i]{E}{}\tensor[_i]{\vb{x}}{}}{\sum\limits_i\tensor[_i]{E}{}}
\end{align}

Porém este não é de muita utilidade visto que,

\begin{align}
    \dot{\vb R}&=\frac{\sum\limits_i\tensor[_i]{E}{}\tensor[_i]{\dot{\vb{x}}}{}}{\sum\limits_i\tensor[_i]{E}{}}+\frac{\sum\limits_i\dv{\tensor[_i]{E}{}}{\lambda}\tensor[_i]{\vb{x}}{}}{\sum\limits_i\tensor[_i]{E}{}}
\end{align}

Exceto no limite $c\rightarrow\infty$ para qual $\dot{\vb R}=\vb V$.

Voltemos à um caso genérico,

\begin{align}
    \dv{}{\lambda}\qty[\tensor[_i]{m}{}\tensor[_i]{u}{^\mu}]&=\frac{1}{c}\sum\limits_k\tensor[_{ki}]{a}{^\mu}\tensor[_{ki}]{{f'}}{}\dv{}{\lambda}\qty[\qty(\tensor[_i]{x}{}-\tensor[_k]{x}{})^2]-\frac2c\sum\limits_k\tensor[_{ki}]{a}{_\nu}\tensor[_{ki}]{{f'}}{}\qty(\tensor[_i]{x}{^\mu}-\tensor[_k]{x}{^\mu})\qty(\tensor[_i]{\dot{x}}{^\nu}-\tensor[_k]{\dot{x}}{^\nu})
\end{align}

Supomos que possuímos apenas 2 partículas e, $\tensor[_{21}]{a}{^\mu}=-\tensor[_{12}]{a}{^\mu}=a^\mu$, $\tensor[_{11}]{a}{^\mu}=\tensor[_{22}]{a}{^\mu}=0$ e $\tensor[_{ki}]{{f'}}{}={f'}\qty[\qty(\tensor[_i]{x}{}-\tensor[_k]{x}{})^2]$ para todos $k,i$,

\begin{align}
    \dv{}{\lambda}\qty[\tensor[_i]{m}{}\tensor[_i]{u}{^\mu}]&=\frac{1}{c}\tensor[]{a}{^\mu}\sum\limits_k\tensor[]{{f'}}{}\dv{}{\lambda}\qty[\qty(\tensor[_i]{x}{}-\tensor[_k]{x}{})^2]-\frac2c\tensor[]{a}{_\nu}\sum\limits_k\tensor[]{{f'}}{}\qty(\tensor[_i]{x}{^\mu}-\tensor[_k]{x}{^\mu})\qty(\tensor[_i]{\dot{x}}{^\nu}-\tensor[_k]{\dot{x}}{^\nu})\\
    \dv{}{\lambda}\qty[\tensor[_i]{m}{}\tensor[_i]{u}{^\mu}]&=\frac{1}{c}\tensor[]{a}{^\mu}\sum\limits_k\dv{f}{\lambda}-\frac2c\sum\limits_k{f'}\qty(\tensor[_i]{x}{^\mu}-\tensor[_k]{x}{^\mu})\vb{a}\cdot\qty(\tensor[_i]{\dot{\vb{x}}}{}-\tensor[_k]{\dot{\vb{x}}}{})+2\sum\limits_k{f'}\qty(\tensor[_i]{x}{^\mu}-\tensor[_k]{x}{^\mu})a^0\qty(\tensor[_i]{\dot{t}}{}-\tensor[_k]{\dot{t}}{})\\
    &\begin{cases}
        \dv{}{\lambda}\qty[\tensor[_1]{m}{}\tensor[_1]{u}{^\mu}]&=\frac{1}{c}\tensor[_{21}]{a}{^\mu}\dv{f}{\lambda}-\frac2c{f'}\qty(\tensor[_1]{x}{^\mu}-\tensor[_2]{x}{^\mu})\tensor[_{21}]{\vb{a}}{}\cdot\qty(\tensor[_1]{\dot{\vb{x}}}{}-\tensor[_2]{\dot{\vb{x}}}{})+2{f'}\qty(\tensor[_1]{x}{^\mu}-\tensor[_2]{x}{^\mu})\tensor[_{21}]{a}{^0}\qty(\tensor[_1]{\dot{t}}{}-\tensor[_2]{\dot{t}}{})\\
        \dv{}{\lambda}\qty[\tensor[_2]{m}{}\tensor[_2]{u}{^\mu}]&=\frac{1}{c}\tensor[_{12}]{a}{^\mu}\dv{f}{\lambda}-\frac2c{f'}\qty(\tensor[_2]{x}{^\mu}-\tensor[_1]{x}{^\mu})\tensor[_{12}]{\vb{a}}{}\cdot\qty(\tensor[_2]{\dot{\vb{x}}}{}-\tensor[_1]{\dot{\vb{x}}}{})+2{f'}\qty(\tensor[_2]{x}{^\mu}-\tensor[_1]{x}{^\mu})\tensor[_{12}]{a}{^0}\qty(\tensor[_2]{\dot{t}}{}-\tensor[_1]{\dot{t}}{})
    \end{cases}\\
    &\begin{cases}
        \dv{}{\lambda}\qty[\tensor[_1]{m}{}\tensor[_1]{u}{^\mu}]&=\frac{1}{c}\tensor[]{a}{^\mu}\dv{f}{\lambda}-\frac2c{f'}\qty(\tensor[_1]{x}{^\mu}-\tensor[_2]{x}{^\mu})\tensor[]{\vb{a}}{}\cdot\qty(\tensor[_1]{\dot{\vb{x}}}{}-\tensor[_2]{\dot{\vb{x}}}{})+2{f'}\qty(\tensor[_1]{x}{^\mu}-\tensor[_2]{x}{^\mu})\tensor[]{a}{^0}\qty(\tensor[_1]{\dot{t}}{}-\tensor[_2]{\dot{t}}{})\\
        \dv{}{\lambda}\qty[\tensor[_2]{m}{}\tensor[_2]{u}{^\mu}]&=-\frac{1}{c}\tensor[]{a}{^\mu}\dv{f}{\lambda}+\frac2c{f'}\qty(\tensor[_1]{x}{^\mu}-\tensor[_2]{x}{^\mu})\tensor[]{\vb{a}}{}\cdot\qty(\tensor[_1]{\dot{\vb{x}}}{}-\tensor[_2]{\dot{\vb{x}}}{})-2{f'}\qty(\tensor[_1]{x}{^\mu}-\tensor[_2]{x}{^\mu})\tensor[]{a}{^0}\qty(\tensor[_1]{\dot{t}}{}-\tensor[_2]{\dot{t}}{})
    \end{cases}\\
    &\begin{cases}
        \frac1c\dv{}{\lambda}\qty[\tensor[_1]{E}{}]&=\frac{1}{c}\tensor[]{a}{^0}\dv{f}{\lambda}-2{f'}\qty(\tensor[_1]{t}{}-\tensor[_2]{t}{})\qty[\tensor[]{\vb{a}}{}\cdot\qty(\tensor[_1]{\dot{\vb{x}}}{}-\tensor[_2]{\dot{\vb{x}}}{})-c\tensor[]{a}{^0}\qty(\tensor[_1]{\dot{t}}{}-\tensor[_2]{\dot{t}}{})]\\
        \frac1c\dv{}{\lambda}\qty[\tensor[_2]{E}{}]&=-\frac{1}{c}\tensor[]{a}{^0}\dv{f}{\lambda}+2{f'}\qty(\tensor[_1]{t}{}-\tensor[_2]{t}{})\qty[\tensor[]{\vb{a}}{}\cdot\qty(\tensor[_1]{\dot{\vb{x}}}{}-\tensor[_2]{\dot{\vb{x}}}{})-c\tensor[]{a}{^0}\qty(\tensor[_1]{\dot{t}}{}-\tensor[_2]{\dot{t}}{})]
    \end{cases}\\
    &\begin{cases}
        \dv{}{\lambda}\qty[\tensor[_1]{m}{}\tensor[_1]{\vb{u}}{}]&=\frac{2}{c}{f'}\tensor[]{\vb{a}}{}\qty(\tensor[_1]{\vb{x}}{}-\tensor[_2]{\vb{x}}{})\cdot\qty(\tensor[_1]{\dot{\vb{x}}}{}-\tensor[_2]{\dot{\vb{x}}}{})-\frac2c{f'}\qty(\tensor[_1]{\vb{x}}{}-\tensor[_2]{\vb{x}}{})\qty[\tensor[]{\vb{a}}{}\cdot\qty(\tensor[_1]{\dot{\vb{x}}}{}-\tensor[_2]{\dot{\vb{x}}}{})-ca^0\qty(\tensor[_1]{\dot{t}}{}-\tensor[_2]{\dot{t}}{})]\\
        \dv{}{\lambda}\qty[\tensor[_2]{m}{}\tensor[_2]{\vb{u}}{}]&=-\frac{2}{c}{f'}\tensor[]{\vb{a}}{}\qty(\tensor[_1]{\vb{x}}{}-\tensor[_2]{\vb{x}}{})\cdot\qty(\tensor[_1]{\dot{\vb{x}}}{}-\tensor[_2]{\dot{\vb{x}}}{})+\frac2c{f'}\qty(\tensor[_1]{\vb{x}}{}-\tensor[_2]{\vb{x}}{})\qty[\tensor[]{\vb{a}}{}\cdot\qty(\tensor[_1]{\dot{\vb{x}}}{}-\tensor[_2]{\dot{\vb{x}}}{})-ca^0\qty(\tensor[_1]{\dot{t}}{}-\tensor[_2]{\dot{t}}{})]
    \end{cases}\\
    &\begin{cases}
        \dv{}{\lambda}\qty[\tensor[_1]{E}{}]&=\tensor[]{a}{^0}\dv{f}{\lambda}-2c{f'}\qty(\tensor[_1]{t}{}-\tensor[_2]{t}{})\qty[\tensor[]{\vb{a}}{}\cdot\qty(\tensor[_1]{\dot{\vb{x}}}{}-\tensor[_2]{\dot{\vb{x}}}{})-c\tensor[]{a}{^0}\qty(\tensor[_1]{\dot{t}}{}-\tensor[_2]{\dot{t}}{})]\\
        \dv{}{\lambda}\qty[\tensor[_2]{E}{}]&=-\tensor[]{a}{^0}\dv{f}{\lambda}+2c{f'}\qty(\tensor[_1]{t}{}-\tensor[_2]{t}{})\qty[\tensor[]{\vb{a}}{}\cdot\qty(\tensor[_1]{\dot{\vb{x}}}{}-\tensor[_2]{\dot{\vb{x}}}{})-c\tensor[]{a}{^0}\qty(\tensor[_1]{\dot{t}}{}-\tensor[_2]{\dot{t}}{})]
    \end{cases}\\
    &\begin{cases}
        \dv{}{\lambda}\qty[\tensor[_1]{m}{}\tensor[_1]{\gamma}{}\tensor[_1]{\dot{\vb{x}}}{}]&=\frac{2}{c}{f'}\qty(\tensor[_1]{\dot{\vb{x}}}{}-\tensor[_2]{\dot{\vb{x}}}{})\times\qty[\vb a\times\qty(\tensor[_1]{\vb{x}}{}-\tensor[_2]{\vb{x}}{})]+2{f'}\qty(\tensor[_1]{\vb{x}}{}-\tensor[_2]{\vb{x}}{})a^0\qty(\tensor[_1]{\dot{t}}{}-\tensor[_2]{\dot{t}}{})]\\
        \dv{}{\lambda}\qty[\tensor[_2]{m}{}\tensor[_2]{\gamma}{}\tensor[_2]{\dot{\vb{x}}}{}]&=-\frac{2}{c}{f'}\qty(\tensor[_1]{\dot{\vb{x}}}{}-\tensor[_2]{\dot{\vb{x}}}{})\times\qty[\vb a\times\qty(\tensor[_1]{\vb{x}}{}-\tensor[_2]{\vb{x}}{})]-2{f'}\qty(\tensor[_1]{\vb{x}}{}-\tensor[_2]{\vb{x}}{})a^0\qty(\tensor[_1]{\dot{t}}{}-\tensor[_2]{\dot{t}}{})]
    \end{cases}\\
\end{align}

Supomos que $\vb a =0$, e que o sinal da interação entre as partículas viaja com velocidade $c$,

\begin{align}
    &\begin{cases}
        \dv{}{\lambda}\qty[\tensor[_1]{E}{}]&=\tensor[]{a}{^0}\dv{f}{\lambda}+2c^2{f'}\qty(\tensor[_1]{t}{}-\tensor[_2]{t}{})\tensor[]{a}{^0}\qty(\tensor[_1]{\dot{t}}{}-\tensor[_2]{\dot{t}}{})\\
        \dv{}{\lambda}\qty[\tensor[_2]{E}{}]&=-\tensor[]{a}{^0}\dv{f}{\lambda}-2c^2{f'}\qty(\tensor[_1]{t}{}-\tensor[_2]{t}{})\tensor[]{a}{^0}\qty(\tensor[_1]{\dot{t}}{}-\tensor[_2]{\dot{t}}{})
    \end{cases}\\
    &\begin{cases}
        \dv{}{\lambda}\qty[\tensor[_1]{m}{}\tensor[_1]{\gamma}{}\tensor[_1]{\dot{\vb{x}}}{}]&=2{f'}\qty(\tensor[_1]{\vb{x}}{}-\tensor[_2]{\vb{x}}{})a^0\qty(\tensor[_1]{\dot{t}}{}-\tensor[_2]{\dot{t}}{})]\\
        \dv{}{\lambda}\qty[\tensor[_2]{m}{}\tensor[_2]{\gamma}{}\tensor[_2]{\dot{\vb{x}}}{}]&=-2{f'}\qty(\tensor[_1]{\vb{x}}{}-\tensor[_2]{\vb{x}}{})a^0\qty(\tensor[_1]{\dot{t}}{}-\tensor[_2]{\dot{t}}{})]
    \end{cases}\\
    &\begin{cases}
        \dv{}{\lambda}\qty[\tensor[_1]{E}{}]&=\tensor[]{a}{^0}\dv{f}{\lambda}+c^2\tensor[]{a}{^0}{f'}\dv{}{\lambda}\qty[\qty(\tensor[_1]{t}{}-\tensor[_2]{t}{})^2]\\
        \dv{}{\lambda}\qty[\tensor[_2]{E}{}]&=-\tensor[]{a}{^0}\dv{f}{\lambda}-c^2\tensor[]{a}{^0}{f'}\dv{}{\lambda}\qty[\qty(\tensor[_1]{t}{}-\tensor[_2]{t}{})^2]
    \end{cases}\\
    &\begin{cases}
        \dv{}{\lambda}\qty[\tensor[_1]{m}{}\tensor[_1]{\gamma}{}\tensor[_1]{\dot{\vb{x}}}{}]&=2{f'}\qty(\tensor[_1]{\vb{x}}{}-\tensor[_2]{\vb{x}}{})a^0\dv{}{\lambda}\qty(\tensor[_1]{{t}}{}-\tensor[_2]{{t}}{})\\
        \dv{}{\lambda}\qty[\tensor[_2]{m}{}\tensor[_2]{\gamma}{}\tensor[_2]{\dot{\vb{x}}}{}]&=-2{f'}\qty(\tensor[_1]{\vb{x}}{}-\tensor[_2]{\vb{x}}{})a^0\dv{}{\lambda}\qty(\tensor[_1]{{t}}{}-\tensor[_2]{{t}}{})
    \end{cases}\\
    &\begin{cases}
        \dv{}{\lambda}\qty[\tensor[_1]{E}{}]&=\tensor[]{a}{^0}\dv{f}{\lambda}+\tensor[]{a}{^0}{f'}\dv{}{\lambda}\qty[\norm{\tensor[_1]{\vb{x}}{}-\tensor[_2]{\vb{x}}{}}^2]\\
        \dv{}{\lambda}\qty[\tensor[_2]{E}{}]&=-\tensor[]{a}{^0}\dv{f}{\lambda}-\tensor[]{a}{^0}{f'}\dv{}{\lambda}\qty[\norm{\tensor[_1]{\vb{x}}{}-\tensor[_2]{\vb{x}}{}}^2]
    \end{cases}\\
    &\begin{cases}
        \dv{}{\lambda}\qty[\tensor[_1]{m}{}\tensor[_1]{\gamma}{}\tensor[_1]{\dot{\vb{x}}}{}]&=2{f'}\qty(\tensor[_1]{\vb{x}}{}-\tensor[_2]{\vb{x}}{})a^0\dv{}{\lambda}\qty[\norm{\tensor[_1]{\vb{x}}{}-\tensor[_2]{\vb{x}}{}}]\\
        \dv{}{\lambda}\qty[\tensor[_2]{m}{}\tensor[_2]{\gamma}{}\tensor[_2]{\dot{\vb{x}}}{}]&=-2{f'}\qty(\tensor[_1]{\vb{x}}{}-\tensor[_2]{\vb{x}}{})a^0\dv{}{\lambda}\qty[\norm{\tensor[_1]{\vb{x}}{}-\tensor[_2]{\vb{x}}{}}]
    \end{cases}
\end{align}

%\printbibliography[heading=subbibliography]
%\end{refsection}

%%%%%%%%%%%%%%%%%%%%%%%%%%%%%%%%%%%%%%%%%%%%%%%%%%%%%%%%%%%%%

%\begin{refsection}
\section{Formalismo Hamiltoniano}

Vamos fazer uma transformação de variáveis para expressar as equações de movimento na forma dita Hamiltoniana, para isso vamos definir o \emph{momento}, como sendo, utilizando a notação, ${\dot x}^\mu=\dv{x^\mu}{\lambda}$,

\begin{align}
    \pdv{L}{{\dot x}^\mu}&=p_\mu
\end{align}

Expressando a Lagrangiana em função de $x^\mu$ e $p^\mu$,

\begin{align}
    L&=\frac\alpha2{\dot x}_\mu{\dot x}^\mu-\frac{m^2c^2}{2\alpha}-\frac1c{\dot x}^\mu A_\mu\\
    &=\frac\alpha2\qty[\frac1\alpha\qty(p_\mu+\frac1c A_\mu)]\qty[\frac1\alpha\qty(p^\mu+\frac1c A^\mu)]-\frac{m^2c^2}{2\alpha}-\frac{1}{\alpha c}\qty(p^\mu +\frac1c A^\mu)A_\mu\\
    &=\frac{1}{2\alpha}p_\mu p^\mu+\frac{1}{\alpha c}p_\mu A^\mu+\frac{1}{2\alpha c^2}A^\mu A_\mu-\frac{m^2c^2}{2\alpha}-\frac{1}{\alpha c}p_\mu A^\mu -\frac{1}{\alpha c^2}A_\mu A^\mu\\
    &=\frac{1}{2\alpha}p_\mu p^\mu-\frac{1}{2\alpha c^2}A_\mu A^\mu-\frac{m^2c^2}{2\alpha}
\end{align}

Fazemos a mudança de variáveis definindo o \emph{Hamiltoniano} como,

\begin{align}
    H&={\dot x}^\mu p_\mu-L\\
    &=\frac1\alpha\qty(p_\mu+\frac1c A_\mu)p^\mu-\frac{1}{2\alpha}p_\mu p^\mu+\frac{1}{2\alpha c^2}A_\mu A^\mu +\frac{m^2c^2}{2\alpha}\\
    &=\frac{1}{2\alpha}\qty(p+\frac1c A)^2+\frac{m^2c^2}{2\alpha}
\end{align}

Note que,

\begin{align}
    \dd{H}&=p_\mu \dd{\dot{x}^\mu}+\dot{x}^\mu\dd{p_\mu}-\dd{L}\\
    \pdv{H}{\alpha}\dd{\alpha}+\pdv{H}{x^\mu}\dd{x^\mu}+\pdv{H}{p_\mu}\dd{p_\mu}&=p_\mu \dd{\dot{x}^\mu}+\dot{x}^\mu\dd{p_\mu}-\pdv{L}{\alpha}\dd{\alpha}-\pdv{L}{x^\mu}\dd{x^\mu}-\pdv{L}{\dot{x}^\mu}\dd{\dot{x}^\mu}
\end{align}

Temos então as identificações,

\begin{align}
    \pdv{H}{\alpha}&=-\pdv{L}{\alpha}=-\dv{}{\lambda}\pdv{L}{\dot\alpha}=0\\
    \pdv{H}{x^\mu}&=-\pdv{L}{x^\mu}=-\dv{}{\lambda}\pdv{L}{x^\mu}=-\dot{p}_\mu\\
    \pdv{H}{p_\mu}&=\dot{x}^\mu
\end{align}

Vamos aplicar as equações do movimento,

\begin{align}
    \pdv{H}{\alpha}&=-\frac{1}{2\alpha^2}\qty(p+\frac1c A)^2-\frac{m^2c^2}{2\alpha^2}=0\\
    \pdv{H}{x^\mu}&=\frac{1}{\alpha c}\qty(p_\nu+\frac1c A_\nu)\partial_\mu A^\nu=-\dot{p}_\mu\\
    \pdv{H}{p_\mu}&=\frac{1}{\alpha}\qty(p^\mu+\frac1c A^\mu)=\dot{x}^\mu
\end{align}

%\printbibliography[heading=subbibliography]
%\end{refsection}

%%%%%%%%%%%%%%%%%%%%%%%%%%%%%%%%%%%%%%%%%%%%%%%%%%%%%%%%%%%%%

%\begin{refsection}
\section{Formalismo de Hamilton-Jacobi}

%\printbibliography[heading=subbibliography]
%\end{refsection}

%%%%%%%%%%%%%%%%%%%%%%%%%%%%%%%%%%%%%%%%%%%%%%%%%%%%%%%%%%%%%

%\begin{refsection}
\section{Teoria Clássica de Campos}

\subsection{Teorema de Noether}

Correntes conservadas devido a simetria do Espaço-Tempo, caso de 
translação, $$x\rightarrow x+\delta a$$

\begin{align}
    \mathcal L\qty(x)\rightarrow \mathcal L'\qty(x+\delta a)\\
    \mathcal L'\qty(x+\delta a)&=\mathcal L\qty(x)+\partial_\mu \mathcal L\qty(x)\delta a^\mu
\end{align}

Mas, 

\begin{align}
    \phi\rightarrow \phi'\qty(x+\delta a)\\
    \phi'\qty(x+\delta a)&=\phi(x)+\partial_\mu\phi\qty(x)\delta a^\mu\\
    \partial_\nu'\phi'\qty(x+\delta a)&=\partial_\nu\phi\qty(x)+\partial_\nu\partial_\mu\phi\qty(x)\delta a^\mu
\end{align}

Assim,

\begin{align}
    \mathcal L'\qty(x+\delta a)&=-\frac12\partial_\mu'\phi'\qty(x+\delta a){\partial'}^\mu \phi'\qty(x+\delta a) -
    \frac{m^2}{2}\phi'\qty(x+\delta a)\phi'\qty(x+\delta a)\\
    &=-\frac12\partial_\mu\phi\qty(x)\partial^\mu\phi\qty(x)-\frac{m^2}{2}\phi\qty(x)\phi\qty(x)-
    \partial_\nu\partial_\mu\phi\qty(x)\delta a^\nu\partial^\mu\phi\qty(x)-m^2\partial_\nu\phi\qty(x)\delta a^\nu\phi\qty(x)\\
    &=\mathcal L\qty(x)-\delta a^\nu\qty{\partial^\mu\qty[\partial_\nu\partial_\mu\phi\qty(x)\phi\qty(x)]-\partial_\nu\partial^\mu\partial_\mu\phi\qty(x)\phi\qty(x)+m^2\phi\qty(x)\partial_\nu\phi\qty(x)}\\
    &=\mathcal L\qty(x)-\delta a^\nu\qty{\partial^\mu\qty[\partial_\nu\partial_\mu\phi\qty(x)\phi\qty(x)]-m^2\partial_\nu\phi\qty(x)\phi\qty(x)+m^2\phi\qty(x)\partial_\nu\phi\qty(x)}\\
    &=\mathcal L\qty(x)-\partial_\mu\qty{\delta a^\nu\phi\qty(x)\partial_\nu\partial^\mu\phi\qty(x)}
\end{align}

Poderíamos ter calculado,

\begin{align}
    \mathcal L'\qty[\phi',\partial_\mu'\phi ']&=\mathcal L\qty[\phi,\partial_\mu\phi]+\pdv{\mathcal L}{\phi}\partial_\mu\phi\qty(x)\delta a^\mu
    +\pdv{\mathcal L}{\partial_\nu\phi}\partial_\nu\partial_\mu\phi\qty(x)\delta a^\mu\\
    &=\mathcal L\qty[\phi,\partial_\mu\phi]+\partial_\nu\pdv{\mathcal L}{\partial_\nu\phi}\partial_\mu\phi\qty(x)\delta a^\mu+
    \pdv{\mathcal L}{\partial_\nu\phi}\partial_\nu\partial_\mu\phi\qty(x)\delta a^\mu\\
    &=\mathcal L\qty[\phi,\partial_\mu\phi]+\delta a^\mu\partial_\nu\qty{\frac{\mathcal L}{\partial_\nu\phi}\partial_\mu\phi\qty(x)}
\end{align}

Igualando,

\begin{align}
    \partial_\nu\qty{\pdv{\mathcal L}{\partial_\nu\phi}\partial_\mu\phi-\delta^\nu_{\ \mu}\mathcal L}\delta a^\mu&=0
\end{align}

Logo a quantidade conservada é o tensor de energia momento, $$T^\nu_{\ \mu}=\pdv{\mathcal L}{\partial_\nu\phi}\partial_\mu\phi-\delta^\nu_{\ \mu}\mathcal L$$

Outra simetria é a de transformação de Lorentz, $$x\rightarrow \Lambda x=x^\mu+\delta\omega^\mu_{\ \nu}x^\nu,\ \delta\omega_{\mu\nu}=-\delta\omega_{\nu\mu}$$

\begin{align}
    \mathcal L\qty(x)\rightarrow\mathcal L'\qty(x+\delta\omega x)\\
    \mathcal L'\qty(x+\delta\omega x)&=\mathcal L\qty(x)+\partial_\mu\mathcal L\qty(x)\delta\omega^\mu_{\ \nu}x^\nu
\end{align}

\begin{align}
    \phi\rightarrow\phi'\qty(x+\delta\omega x)\\
    \phi'\qty(x+\delta\omega x)&=\phi\qty(x)+\partial_\mu\phi\qty(x)\delta\omega^{\mu}_{\ \nu}x^\nu\\
    \partial_\rho'\phi'\qty(x+\delta\omega x)&=\partial_\rho\phi\qty(x)-\delta\omega_{\rho}^{\ \mu}\partial_\mu\phi\qty(x)+\partial_\rho\partial_\mu\phi\qty(x)\delta\omega^{\mu}_{\ \nu}x^\nu
\end{align}

\begin{align}
    \mathcal L'\qty(x+\delta\omega x)&=-\frac12\partial_\mu'\phi\qty(x+\delta \omega x){\partial'}^\mu\phi'\qty(x+\delta\omega x)-\frac{m^2}{2}\phi'\qty(x+\delta\omega x)\phi'\qty(x+\delta \omega x)\\
    &=-\frac12\partial_\mu\phi\qty(x)\partial^\mu\phi\qty(x)-\frac{m^2}{2}\phi\qty(x)\phi\qty(x)-\partial_\nu\partial_\mu\phi\qty(x)\delta\omega^{\nu\rho}x_{\rho}\partial^\mu\phi\qty(x)-m^2\phi\qty(x)\partial_\nu\phi\qty(x)\delta\omega^{\nu\rho}x_{\rho}+\delta\omega^{\nu\rho}\partial_\nu\phi\qty(x)\partial_\rho\phi\qty(x)\\
    &=\mathcal L\qty(x)-\delta\omega^{\nu\rho}\qty{\partial^\mu\qty[\phi\qty(x)x_\rho\partial_\nu\partial_\mu\phi\qty(x)]-\phi\qty(x)\delta^{\mu}_{\ \rho}\partial_\nu\partial_\mu\phi\qty(x)-x_\rho\phi\qty(x)\partial_\nu\partial^\mu\partial_\mu\phi\qty(x)+m^2x_\rho\phi\qty(x)\partial_\nu\phi\qty(x)}\\
    &=\mathcal L\qty(x)-\delta\omega^{\nu\rho}\qty{\partial^\mu\qty[\phi\qty(x)x_\rho\partial_\nu\partial_\mu\phi\qty(x)]-\phi\qty(x)\partial_\nu\partial_\rho\phi\qty(x)-m^2x_\rho\phi\qty(x)\partial_\nu\phi\qty(x)+m^2x_\rho\phi\qty(x)\partial_\nu\phi\qty(x)}\\
    &=\mathcal L\qty(x)-\partial_\mu\qty{\delta\omega^{\nu\rho}x_\rho\phi\qty(x)\partial_\nu\partial^\mu\phi\qty(x)}
\end{align}

Logo é uma simetria da ação, poderíamos ter escrito,

\begin{align}
    \mathcal L'\qty[\phi',\partial_\mu'\phi']&=\mathcal L\qty[\phi,\partial_\mu\phi]+\pdv{\mathcal L}{\phi}\partial_\mu\phi\qty(x)\delta\omega^{\mu\nu}x_\nu+\pdv{\mathcal L}{\partial_\rho\phi\qty(x)}\qty{\partial_\rho\partial_\mu\phi\qty(x)\delta\omega^{\mu\nu}x_\nu-\delta\omega_{\rho}^{\ \nu}\partial_\nu\phi\qty(x)}\\
    &=\mathcal L\qty[\phi,\partial_\mu\phi]+\partial_\rho\pdv{\mathcal L}{\partial_\rho\phi}\partial_\mu\phi\qty(x)\delta\omega^{\mu\nu}x_\nu+\pdv{\mathcal L}{\partial_\rho\phi\qty(x)}\partial_\rho\partial_\mu\phi\qty(x)\delta\omega^{\mu\nu}x_\nu-\pdv{\mathcal L}{\partial^\rho\phi}\partial_\nu\phi\qty(x)\delta\omega^{\rho\nu}\\
    &=\mathcal L\qty[\phi,\partial_\mu\phi]+\partial_\rho\qty{\pdv{\mathcal L}{\partial_\rho\phi}\partial_\mu\phi\qty(x)\delta\omega^{\mu\nu}x_\nu}-\pdv{\mathcal L}{\partial_\rho\phi}\partial_\mu\phi\qty(x)\delta\omega^{\mu\nu}g_{\rho\nu}-\pdv{\mathcal L}{\partial^\mu\phi}\partial_\nu\phi\qty(x)\delta\omega^{\mu\nu}\\
    &=\mathcal L\qty[\phi,\partial_\mu\phi]+\frac12\delta\omega^{\mu\nu}\qty{\pdv{\mathcal L}{\partial^\mu\phi}\partial_\nu\phi-\pdv{\mathcal L}{\partial^\nu\phi}\partial_\mu\phi+\partial_\rho\qty{\pdv{\mathcal L}{\partial_\rho\phi}\qty(\partial_\mu\phi x_\nu-\partial_\nu\phi x_\mu)}}\\
    &-\frac12\delta\omega^{\mu\nu}\qty{\pdv{\mathcal L}{\partial^\mu\phi}\partial_\nu\phi\qty(x)-\pdv{\mathcal L}{\partial^\nu\phi}\partial_\mu\phi\qty(x)}\\
    &=\mathcal L[\phi,\partial_\mu\phi]+\frac12\delta\omega^{\mu\nu}\partial_\rho\qty{x_\nu T^\rho_{\ \mu}+x_\nu\delta^\rho_{\ \mu}\mathcal L-x_\mu T^\rho_{\ \nu}-x_\mu\delta^\rho_{\ \nu}\mathcal L}
\end{align}

Mas, $$\mathcal L'\qty(x+\delta\omega x)=\mathcal L\qty(x)+\frac12\delta\omega^{\mu\nu}\qty{\partial_\mu\qty(x_\nu\mathcal L)-\partial_\nu\qty(x_\mu\mathcal L)}$$

Igualando,

\begin{align}
    \delta\omega^{\mu\nu}\qty{\partial_\mu\qty(x_\nu\mathcal L)-\partial_\nu\qty(x_\mu\mathcal L)}&=\delta\omega^{\mu\nu}\partial_\rho\qty{x_\nu T^\rho_{\ \mu}+x_\nu\delta^\rho_{\ \mu}\mathcal L-x_\mu T^\rho_{\ \nu}-x_\mu\delta^\rho_{\ \nu}\mathcal L}\\
    \delta\omega^{\mu\nu}\partial_\rho\qty(T^\rho_{\ \mu}x_\nu-T^\rho_{\ \nu}x_\mu)&=0\\
    \partial_\rho\qty{T^\rho_{\ \mu}x_\nu-T^\rho_{\ \nu}x_\mu}&=0
\end{align}

Para o campo escalar sempre temos o tensor energia momento simétrico, logo, a quantidade conservada é o tensor de momento angular, $$\mathcal M^{\rho\mu\nu}=T^{\rho\mu}x^\nu-T^{\rho\nu}x^\mu$$

Para um campo com índices vetoriais, a simetria de translações é mantida, resultando em um tensor de energia momento associado como, $$T^{\mu}_{\ \nu}=\pdv{\mathcal L}{\partial_\mu A^\rho}\partial_\nu A^\rho-\delta^{\mu}_{\ \nu}\mathcal L$$

Mas para a simetria de transformações de Lorentz, 

\begin{align}
    {A'}^\mu\qty(x+\delta \omega x)&=A^\mu\qty(x)-\delta\omega^{\mu}_{\ \nu}A^\nu\qty(x)+\partial_\rho A^\mu\qty(x)\delta\omega^{\rho \nu}x_\nu\\
    \partial_\sigma'{A'}^\mu\qty(x+\delta\omega x)&=\partial_\sigma A^\mu\qty(x)-\delta\omega_{\sigma}^{\ \nu}\partial_\nu A^\mu\qty(x)-\delta\omega^{\mu}_{\ \nu}\partial_\sigma A^\nu\qty(x)+\partial_\nu\partial_\sigma A^\mu\qty(x)\delta\omega^{\nu\rho}x_\rho
\end{align}

Temos,

\begin{align}
    \mathcal L'\qty[{A'}^\mu,\partial_\nu' {A'}^\mu]&=\mathcal L\qty[A^\mu,\partial_\nu A^\mu]+\pdv{\mathcal L}{A^\mu}\qty{\partial_\rho A^\mu\delta\omega^{\rho\nu}x_\nu-\delta\omega^{\mu\nu}A_\nu\qty(x)}\\
    &+\pdv{\mathcal L}{\partial_\sigma A^\mu}\qty{-\partial_\sigma A_\nu\qty(x)\delta\omega^{\mu\nu}-\partial_\nu A^\mu\qty(x)\delta\omega_{\sigma}^{\ \nu}+\partial_\rho\partial_\sigma A^\mu\qty(x)\delta\omega^{\rho\nu}x_\nu}\\
    \delta\mathcal L&=\partial_\sigma\pdv{\mathcal L}{\partial_\sigma A^\mu}\qty{\partial_\rho A^\mu\delta\omega^{\rho\nu}x_\nu-\delta\omega^{\mu\nu}A_\nu\qty(x)}\\
    &+\pdv{\mathcal L}{\partial_\sigma A^\mu}\qty{-\partial_\sigma A_\nu\qty(x)\delta\omega^{\mu\nu}-\partial_\nu A^\mu\qty(x)\delta\omega_{\sigma}^{\ \nu}+\partial_\rho\partial_\sigma A^\mu\qty(x)\delta\omega^{\rho\nu}x_\nu}\\
    \delta\mathcal L&=\partial_\sigma\qty{\pdv{\mathcal L}{\partial_\sigma A^\mu}\qty{\partial_\rho A^\mu\delta\omega^{\rho\nu}x_\nu-\delta\omega^{\mu\nu}A_\nu}}\\
    &-\pdv{\mathcal L}{\partial_\sigma A^\mu}\partial_\sigma\partial_\rho A^\mu\delta\omega^{\rho\nu}x_\nu-\pdv{\mathcal L}{\partial_\sigma A^\mu}\partial_\rho A^\mu\delta\omega^{\rho\nu}g_{\sigma\nu}\\
    &+\pdv{\mathcal L}{\partial_\sigma A^\mu}\partial_\sigma A_\nu\delta\omega^{\mu\nu}\\
    &+\pdv{\mathcal L}{\partial_\sigma A^\mu}\qty{-\partial_\sigma A_\nu\qty(x)\delta\omega^{\mu\nu}-\partial_\nu A^\mu\qty(x)\delta\omega_{\sigma}^{\ \nu}+\partial_\rho\partial_\sigma A^\mu\qty(x)\delta\omega^{\rho\nu}x_\nu}\\
    \delta\mathcal L&=\partial_\sigma\qty{\delta\omega^{\mu\nu}\qty{\pdv{\mathcal L}{\partial_\sigma A^\mu}A_\nu+\pdv{\mathcal L}{\partial_\sigma A^\alpha}\partial_\mu A^\alpha x_\nu}}\\
    \delta\mathcal L&=\frac12\delta\omega^{\mu\nu}\partial_\sigma\qty{\pdv{\mathcal L}{\partial_\sigma A^\mu}A_\nu-\pdv{\mathcal L}{\partial_\sigma A^\nu}A_\mu+\pdv{\mathcal L}{\partial_\sigma A^\alpha}\partial_\mu A^\alpha x_\nu-\pdv{\mathcal L}{\partial_\sigma A^\alpha}\partial_\nu A^\alpha x_\mu}\\
    \delta\mathcal L&=\frac12\delta\omega^{\mu\nu}\partial_\sigma\qty{\pdv{\mathcal L}{\partial_\sigma A^\mu}A_\nu-\pdv{\mathcal L}{\partial_\sigma A^\nu}A_\mu+T^{\sigma}_{\ \mu}x_\nu+\delta^{\sigma}_{\ \mu}x_\nu\mathcal L-T^{\sigma}_{\ \nu}x_\mu-\delta^{\sigma}_{\ \nu}x_\mu\mathcal L}
\end{align}

Definindo,$$\mathcal S^{\sigma\mu\nu}=\pdv{\mathcal L}{\partial_\sigma A_\mu}A^\nu-\pdv{\mathcal L}{\partial_\sigma A_\nu}A^\mu$$ $$\mathcal M^{\sigma\mu\nu}=T^{\sigma\mu}x^\nu-T^{\sigma\nu}x^\mu$$ $$\mathcal J^{\sigma\mu\nu}=\mathcal S^{\sigma\mu\nu}+\mathcal M^{\sigma\mu\nu}$$

Concluímos que,

\begin{align}
    \delta\mathcal L&=\frac12\delta\omega^{\mu\nu}\partial_\sigma\qty{\mathcal S^{\sigma}_{\ \mu\nu}+\mathcal M^{\sigma}_{\ \mu\nu}+\delta^\sigma_{\ \mu}x_\nu\mathcal L-\delta^\sigma_{\ \nu}x_\mu\mathcal L}
\end{align}

Igualando as variações, concluímos que 

\begin{align}
    \partial_\sigma\mathcal J^{\sigma}_{\ \mu\nu}&=0
\end{align}

Para um campo geral com $n$ índices, $A^{\mu_1\cdots\mu_n}$,

\begin{align}
    {A'}^{\mu_1\cdots\mu_n}\qty(x+\delta \omega x)&=A^{\mu_1\cdots\mu_n}\qty(x)-\delta\omega^{\mu_i}_{\ \nu_i}A^{\mu_1\cdots\nu_i\cdots\mu_n}\qty(x)+\partial_\rho A^{\mu_1\cdots\mu_n}\qty(x)\delta\omega^{\rho \nu}x_\nu\\
    \partial_\sigma'{A'}^{\mu_1\cdots\mu_n}\qty(x+\delta\omega x)&=\partial_\sigma A^{\mu_1\cdots\mu_n}\qty(x)-\delta\omega_{\sigma}^{\ \nu}\partial_\nu A^{\mu_1\cdots\mu_n}\qty(x)-\delta\omega^{\mu_i}_{\ \nu_i}\partial_\sigma A^{\mu_1\cdots\nu_i\cdots\mu_n}\qty(x)+\partial_\nu\partial_\sigma A^{\mu_1\cdots\mu_n}\qty(x)\delta\omega^{\nu\rho}x_\rho
\end{align}

Temos,

\begin{align}
    \mathcal L'\qty[{A'}^{\mu_1\cdots\mu_n},\partial_\nu' {A'}^{\mu_1\cdots\mu_n}]&=\mathcal L\qty[A^{\mu_1\cdots\mu_n},\partial_\nu A^{\mu_1\cdots\mu_n}]+\pdv{\mathcal L}{A^{\mu_1\cdots\mu_n}}\qty{\partial_\rho A^{\mu_1\cdots\mu_n}\delta\omega^{\rho\nu}x_\nu-\delta\omega^{\mu_i\nu_i}A_{\mu_1\cdots\nu_i\cdots\mu_n}\qty(x)}\\
    &+\pdv{\mathcal L}{\partial_\sigma A^{\mu_1\cdots\mu_n}}\qty{-\partial_\sigma A_{\mu_1\cdots\nu_i\cdots\mu_n}\qty(x)\delta\omega^{\mu_i\nu_i}-\partial_{\nu} A^{\mu_1\cdots\mu_n}\qty(x)\delta\omega_{\sigma}^{\ \nu}+\partial_\rho\partial_\sigma A^{\mu_1\cdots\mu_n}\qty(x)\delta\omega^{\rho\nu}x_\nu}\\
    \delta\mathcal L&=\partial_\sigma\pdv{\mathcal L}{\partial_\sigma A^{\mu_1\cdots\mu_n}}\qty{\partial_\rho A^{\mu_1\cdots\mu_n}\delta\omega^{\rho\nu}x_\nu-\delta\omega^{\mu_i\nu_i}A_{\mu_1\cdots\nu_i\cdots\mu_n}\qty(x)}\\
    &+\pdv{\mathcal L}{\partial_\sigma A^{\mu_1\cdots\mu_n}}\qty{-\partial_\sigma A_{\mu_1\cdots\nu_i\cdots\mu_n}\qty(x)\delta\omega^{\mu_i\nu_i}-\partial_\nu A^{\mu_1\cdots\mu_n}\qty(x)\delta\omega_{\sigma}^{\ \nu}+\partial_\rho\partial_\sigma A^{\mu_1\cdots\mu_n}\qty(x)\delta\omega^{\rho\nu}x_\nu}\\
    \delta\mathcal L&=\partial_\sigma\qty{\pdv{\mathcal L}{\partial_\sigma A^{\mu_1\cdots\mu_n}}\qty{\partial_\rho A^{\mu_1\cdots\mu_n}\delta\omega^{\rho\nu}x_\nu-\delta\omega^{\mu_i\nu_i}A_{\mu_1\cdots\nu_i\cdots\mu_n}}}\\
    &-\pdv{\mathcal L}{\partial_\sigma A^{\mu_1\cdots\mu_n}}\partial_\sigma\partial_\rho A^{\mu_1\cdots\mu_n}\delta\omega^{\rho\nu}x_\nu-\pdv{\mathcal L}{\partial_\sigma A^{\mu_1\cdots\mu_n}}\partial_\rho A^{\mu_1\cdots\mu_n}\delta\omega^{\rho\nu}g_{\sigma\nu}\\
    &+\pdv{\mathcal L}{\partial_\sigma A^{\mu_1\cdots\mu_n}}\partial_\sigma A_{\mu_1\cdots\nu_i\cdots\mu_n}\delta\omega^{\mu_i\nu_i}\\
    &+\pdv{\mathcal L}{\partial_\sigma A^{\mu_1\cdots\mu_n}}\qty{-\partial_\sigma A_{\mu_1\cdots\nu_i\cdots\mu_n}\qty(x)\delta\omega^{\mu_i\nu_i}-\partial_\nu A^{\mu_1\cdots\mu_n}\qty(x)\delta\omega_{\sigma}^{\ \nu}+\partial_\rho\partial_\sigma A^{\mu_1\cdots\mu_n}\qty(x)\delta\omega^{\rho\nu}x_\nu}\\
    \delta\mathcal L&=\partial_\sigma\qty{\delta\omega^{\rho\nu}\pdv{\mathcal L}{\partial_\sigma A^{\mu_1\cdots\mu_n}}\partial_\rho A^{\mu_1\cdots\mu_n}x_\nu}+\partial_\sigma\qty{\delta\omega^{\mu_i\nu_i}\pdv{\mathcal L}{\partial_\sigma A^{\mu_1\cdots\mu_i\cdots\mu_n}}A_{\mu_1\cdots\nu_i\cdots\mu_n}}\\
    \delta\mathcal L&=\delta\omega^{\rho\nu}\qty{\frac12\partial_\sigma\qty{T^{\sigma}_{\ \rho}x_\nu-T^\sigma_{\ \nu}x_\rho}+\partial_\sigma\qty{\pdv{\mathcal L}{\partial_\sigma A^{[\rho|\cdots\mu_n}}A_{|\nu]\cdots\mu_n}+\cdots+\pdv{\mathcal L}{\partial_\sigma A^{\mu_1\cdots[\rho|}}A_{\mu_1\cdots|\nu]}}}\\
    &+\frac12\delta\omega^{\rho\nu}\partial_\sigma\qty{\delta^\sigma_{\ \rho}x_\nu\mathcal L-\delta^\sigma_{\ \nu}x_\rho\mathcal L}
\end{align}

Definindo,$$\mathcal S^{\sigma\mu\nu}=2\qty{\pdv{\mathcal L}{\partial_\sigma A_{[\mu|\cdots\mu_n}}A^{|\nu]\cdots\mu_n}+\cdots+\pdv{\mathcal L}{\partial_\sigma A_{\mu_1\cdots[\mu|}}A^{\mu_1\cdots|\nu]}}$$ $$\mathcal M^{\sigma\mu\nu}=T^{\sigma\mu}x^\nu-T^{\sigma\nu}x^\mu$$ $$\mathcal J^{\sigma\mu\nu}=\mathcal S^{\sigma\mu\nu}+\mathcal M^{\sigma\mu\nu}$$

Concluímos que,

\begin{align}
    \delta\mathcal L&=\frac12\delta\omega^{\mu\nu}\partial_\sigma\qty{\mathcal S^{\sigma}_{\ \mu\nu}+\mathcal M^{\sigma}_{\ \mu\nu}+\delta^\sigma_{\ \mu}x_\nu\mathcal L-\delta^{\sigma}_{\ \nu}x_\mu\mathcal L}
\end{align}

Igualando as variações, concluímos que 

\begin{align}
    \partial_\sigma\mathcal J^{\sigma}_{\ \mu\nu}&=0
\end{align}

%\printbibliography[heading=subbibliography]
%\end{refsection}

%%%%%%%%%%%%%%%%%%%%%%%%%%%%%%%%%%%%%%%%%%%%%%%%%%%%%%%%%%%%%o  

\end{document}