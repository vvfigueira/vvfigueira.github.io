\subsection{Comportamento de amplitudes na teoria teste}

A teoria teste com que iniciaremos nossos estudos é dada pelo lagrangiano,\[\mathcal L = -\frac12\qty(\Box \phi+\frac g2\phi^2)^2-\frac {m^2}{2}\qty(\partial_\mu\phi\partial^\mu\phi+\frac{g}{3!}\phi^3),\]
note a forma sugestiva em que escrevemos, o lagrangiano usual de uma teoria $\phi^3$ gera como equações de movimento $\Box\phi+\frac g2\phi^2=0$, ou seja, o primeiro termo é apenas a equação de 
movimento do segundo termo ao quadrado, assim, as amplitudes geradas a nível árvore por esse lagrangiano são idênticas às geradas por uma teoria usual de $\phi^3$. É interessante analisar a dependência 
dessa teoria com as dimensões do espaço tempo, em $D$ dimensões temos: $\qty[\phi]=\frac D2-2,\qty[m]=1,\qty[g]=4-\frac 12D$, porém, se estamos interessados em analisar espalhamentos 
na teoria quântica a dimensão do campo $\qty[\phi]$ é ``fixa'', o requisito vem da dimensão dos operadores de criação e aniquilação, \[\comm{a_{\vb k}}{a^\dagger_{\vb q}}=\qty(2\pi)^32\omega_{\vb k}\delta^{\qty(D-1)}\qty(\vb k-\vb q),\ \ \ \phi\qty(x)=\int\frac{\dd[D-1]{\vb k}}{\qty(2\pi)^D2\omega_{\vb k}}a_{\vb k}e^{\im k\cdot x}+\textnormal{h.c.},\] 
que fixa $\qty[a_{\vb k}]=1-\frac D2$ e $\qty[\phi]=\frac D2-1$, obviamente ambas as condições são inconsistentes. A solução é que deveríamos utilizar de $\frac1m \phi$ ao invés de $\phi$, porém, isso somente 
irá adicionar fatores de $m^2$ nas amplitudes, que podemos retirar se quisermos. Conforme a análise dimensional, uma escolha sensata seria $D=6$ --- escolha usual para teorias $\phi^3$ --- na qual o acoplamento $m^2g\phi^3$ possui 
dimensão zero, porém há um acoplamento de $g^2\phi^4$, que possuiria dimensão de massa negativa, indicando um caráter não renormalizável, assim, seguiremos em $D=4$, por esse motivo e 
por nos proporcionar o poderoso método \textit{on-shell}. As regras de Feynman geradas por essa teoria são,
\begin{itemize}
    \item \ \ \raisebox{-6pt}{\feynmandiagram [horizontal=a to b] {
		a [particle=\(\phi\)] -- [scalar] b [particle=\(\phi\)],
		};} $=-\im\qty(m^2p^2+p^4)^{-1}$
	\item \ \ \raisebox{3pt}{\feynmandiagram [small,baseline = (b.base),horizontal=a to b] {
		a [particle=\(\phi_1\)] -- [scalar] b  ,
		b -- [scalar] c [particle=\(\phi_2\)],
		b -- [scalar] d [particle=\(\phi_3\)],
		};} $=\im g\qty(p_1^2+p_2^2+p_3^2+m^2)$
	\item \ \ \raisebox{3pt}{\feynmandiagram [small,baseline = (b.base)] {
		a [particle=\(\phi_1\)] -- [scalar] b  ,
		e [particle=\(\phi_3\)] -- [scalar] b  ,
		b -- [scalar] c [particle=\(\phi_2\)],
		b -- [scalar] d [particle=\(\phi_4\)],
		};} $=-3\im g^2$
\end{itemize}
Note a estrutura não usual do propagador devido a termos de derivadas superiores, claramente há dois polos,\[\frac{1}{\im}\frac{1}{m^2p^2+p^4}=\frac{1}{\im m^2}\frac{1}{p^2}\mathbin{\textcolor{red}{-}}\frac{1}{\im m^2}\frac{1}{p^2+m^2}, \] 
o polo massivo também possui um sinal contrário do polo sem massa, isso é um indício da presença de um fantasma e de possível quebra de unitariedade. Conforme mencionamos anteriormente, com correções de loops 
este polo massivo se torna complexo, desacoplando o estado do espectro físico; além disso, todas as amplitudes físicas --- estados externos sem massa --- a nível árvore são completamente iguais a de uma teoria 
$\phi^3$ normal, apenas com a diferença das amplitudes possuírem um fator multiplicativo de $m^2$ adicional\autocite{johanssonUnravelingConformalGravity2018}, também, amplitudes com apenas uma perna externa massiva 
a nível árvore são identicamente nulas \autocite{johanssonUnravelingConformalGravity2018}.

Vamos calcular agora os cortes da função de quatro pontos a um \textit{loop} para analisar a presença ou não de cancelamentos --- o que significaria que as divergências possuem um comportamento mais ameno do que 
esperado por \textit{power--counting} ---, em $D=4$ um corte quádruplo não possui nenhuma informação sobre possíveis cancelamentos, assim, começamos por um corte triplo, isto é,
\begin{align*}
    \mathcal I^1_4\eval_{3\textnormal{ Cuts}} &=\raisebox{+3pt}{\begin{tikzpicture}[baseline=(b.base)]
        \begin{feynman}
            \vertex (i1) at (-1,0.5) {3};
            \vertex (i2) at (-1,-0.5) {4};
            \vertex [solid blob=gray, minimum size=0.3cm] (b1) at (-0.5,0.5) {};
            \vertex [solid blob=gray, minimum size=0.3cm] (b2) at (-0.5,-0.5) {};
            \vertex [solid blob=gray, minimum size=0.3cm] (b3) at (0.3,0) {};
            \vertex (b) at (0,0) {};
            \vertex (f1) at (1,0.5) {2};
            \vertex (f2) at (1,-0.5) {1};
            \diagram*{
                (i1) -- [scalar] (b1),
                (i2) -- [scalar] (b2),
                (b1) -- [scalar, cut2] (b3),
                (b3) -- [scalar, cut2] (b2),
                (b2) -- [scalar, cut2] (b1),
                (b3) -- [scalar] (f1),
                (b3) -- [scalar] (f2),
            };
        \end{feynman}
    \end{tikzpicture}} 
\end{align*}
Para isto necessitamos saber sobre as funções de quatro pontos com até duas pernas externas massivas, estas são dadas por,
\begin{align*}
    \raisebox{+3pt}{\begin{tikzpicture}[baseline=(b.base)]
        \begin{feynman}
            \vertex (i1) at (-1,1){2};
            \vertex (i2) at (-1,-1){1};
            \vertex [solid blob=gray, minimum size=1cm] (b) at (0,0) {};
            \vertex (f1) at (1,1){3};
            \vertex (f2) at (1,-1){4};
            \diagram*{
                (i1) -- [scalar] (b) [blob],
                (i2) -- [scalar] (b),
                (b) -- [scalar] (f1),
                (b) -- [scalar] (f2),
            };
        \end{feynman}
    \end{tikzpicture}} = -\im g^2m^2\qty(\frac1s+\frac1t+\frac1u), \ \ \ \raisebox{+3pt}{\begin{tikzpicture}[baseline=(b.base)]
        \begin{feynman}
            \vertex (i1) at (-1,1){2};
            \vertex (i2) at (-1,-1){1};
            \vertex [solid blob=gray, minimum size=1cm] (b) at (0,0) {};
            \vertex (f1) at (1,1){\(\vb 3\)};
            \vertex (f2) at (1,-1){\(\vb 4\)};
            \diagram*{
                (i1) -- [scalar] (b) [blob],
                (i2) -- [scalar] (b),
                (b) -- [scalar] (f1),
                (b) -- [scalar] (f2),
            };
        \end{feynman}
    \end{tikzpicture}} = \im g^2m^2\qty(\frac1s+\frac1{t-m^2}+\frac1{u-m^2})
\end{align*}
além de resolver os dois possíveis cortes,
\begin{align*}
    \begin{cases}
        \ell^2=\qty(\ell+4)^2=\qty(\ell+3+4)^2=0\\
        \ell^2=\qty(\ell+4)^2=\qty(\ell+3+4)^2=-m^2
    \end{cases}\Rightarrow \begin{cases}
        \ell_0 = |4]\langle 4|+z|3]\langle4|\\
        \ell_m = |4]\langle 4|+z|3]\langle4|+\frac{m^2|4]\langle 3|}{z\langle43\rangle[43]}
    \end{cases}
\end{align*}
Ambos cortes capturam o limite $\ell\rightarrow\infty$ com $z\rightarrow\infty$, e de fato $\ell_0\sim\ell_m$ no mesmo. Assim, podemos calcular o valor do corte neste limite,
\begin{align*}
    \mathcal I^1_4\eval_{3\textnormal{ Cuts}} =\raisebox{+3pt}{\begin{tikzpicture}[baseline=(b.base)]
        \begin{feynman}
            \vertex (i1) at (-1,0.5) {3};
            \vertex (i2) at (-1,-0.5) {4};
            \vertex [solid blob=gray, minimum size=0.3cm] (b1) at (-0.5,0.5) {};
            \vertex [solid blob=gray, minimum size=0.3cm] (b2) at (-0.5,-0.5) {};
            \vertex [solid blob=gray, minimum size=0.3cm] (b3) at (0.3,0) {};
            \vertex (b) at (0,0) {};
            \vertex (f1) at (1,0.5) {2};
            \vertex (f2) at (1,-0.5) {1};
            \diagram*{
                (i1) -- [scalar] (b1),
                (i2) -- [scalar] (b2),
                (b1) -- [scalar, cut2] (b3),
                (b3) -- [scalar, cut2] (b2),
                (b2) -- [scalar, cut2] (b1),
                (b3) -- [scalar] (f1),
                (b3) -- [scalar] (f2),
            };
        \end{feynman}
    \end{tikzpicture}}&=\frac{\qty(\im gm^2)^2\qty(-\im g^2 m^2)}{\qty(\im m^2)^3}\qty(\frac1s+\frac{1}{2\ell_0\cdot 1}+\frac{1}{2\ell_0\cdot 2})+\frac{\qty(-\im gm^2)^2\qty(\im g^2m^2)}{\qty(-\im m^2)^3}\qty(\frac1s+\frac{1}{2\ell_m\cdot 1}+\frac{1}{2\ell_m\cdot 2})\\
    &=-g^4\qty(\frac2s+\frac{1}{2\ell_0\cdot 1}+\frac{1}{2\ell_0\cdot 2}+\frac{1}{2\ell_m\cdot 1}+\frac{1}{2\ell_m\cdot 2})
\end{align*}
O termo $\frac2s$ representa que não existem cancelamentos, isto é, o termo com 3 polos é presente neste integrando. Este é o comportamento esperado por \textit{power--counting} e não é 
necessariamente um problema, visto que essa contribuição de 3 polos no integrando de um \textit{loop} não gera divergências. Analisamos também o corte duplo, que supostamente demonstraria o 
melhor comportamento possível, porém, novamente há um polo duplo sem cancelamento conforme dita argumentos de \textit{power--counting},
\begin{align*}
    \mathcal I^1_4\eval_{3\textnormal{ Cuts}} =\raisebox{+3pt}{\begin{tikzpicture}[baseline=(b.base)]
        \begin{feynman}
            \vertex (i1) at (-1.3,0.5) {3};
            \vertex (i2) at (-1.3,-0.5) {4};
            \vertex [solid blob=gray, minimum size=0.3cm] (b1) at (-0.5,0) {};
            \vertex [solid blob=gray, minimum size=0.3cm] (b2) at (0.5,0) {};
            \vertex (b) at (0,0) {};
            \vertex (f1) at (1.3,0.5) {2};
            \vertex (f2) at (1.3,-0.5) {1};
            \diagram*{
                (i1) -- [scalar] (b1),
                (i2) -- [scalar] (b1),
                (b1) -- [scalar, bend left=45, looseness = 1,cut2] (b2),
                (b2) -- [scalar, bend left=45, looseness = 1,cut2] (b1),
                (b2) -- [scalar] (f1),
                (b2) -- [scalar] (f2),
            };
        \end{feynman}
    \end{tikzpicture}}&=g^4\frac{2}{s^2}+\mathcal O\qty(z^{-1})
\end{align*}
O próximo passo é analisar cortes de maior número de \textit{loops}, começando pela topologia:
\begin{align*}
    \mathcal I^2_4\eval_{5\textnormal{ Cuts}} =\raisebox{+3pt}{\begin{tikzpicture}[baseline=(b.base)]
        \begin{feynman}
            \vertex (i3) at (-2,2) {3};
            \vertex (i2) at (-1,-1) {2};
            \vertex (i1) at (1,-1) {1};
            \vertex (i4) at (2,2) {4};
            \vertex [solid blob=gray, minimum size=0.3cm] (b1) at (-1,1) {};
            \vertex [solid blob=gray, minimum size=0.3cm] (b2) at (1,1) {};
            \vertex [solid blob=gray, minimum size=0.3cm] (b3) at (0,1) {};
            \vertex [solid blob=gray, minimum size=0.3cm] (b) at (0,0) {};
            \diagram*{
                (i1) -- [scalar] (b),
                (i2) -- [scalar] (b),
                (b) -- [scalar, cut2] {(b1),(b2),(b3)},
                (b1) -- [scalar,cut2] (b3),
                (b3) -- [scalar,cut2] (b2),
                (b1) -- [scalar] (i3),
                (b2) -- [scalar] (i4),
            };
        \end{feynman}
    \end{tikzpicture}}
\end{align*}
Para isto é necessário resolver 5 topologias diferentes de cortes,
\begin{align*}
    \begin{cases}
        \ell_1^2=\ell_2^2=\qty(3-\ell_1)^2=\qty(3-\ell_1-\ell_2)^2=\qty(3+4-\ell_1-\ell_2)^2=0\\
        \ell_1^2=\ell_2^2=\qty(3-\ell_1)^2=-m^2,\ \ \qty(3-\ell_1-\ell_2)^2=\qty(3+4-\ell_1-\ell_2)^2=0\\
        \ell_1^2=\ell_2^2=0,\ \ \ \qty(3-\ell_1)^2=\qty(3-\ell_1-\ell_2)^2=\qty(3+4-\ell_1-\ell_2)^2=-m^2\\
        \ell_1^2=0,\ \ \ \ell_2^2=\qty(3-\ell_1)^2=\qty(3-\ell_1-\ell_2)^2=\qty(3+4-\ell_1-\ell_2)^2=-m^2\\
        \ell_1^2=\ell_2^2=\qty(3-\ell_1)^2=\qty(3-\ell_1-\ell_2)^2=\qty(3+4-\ell_1-\ell_2)^2=-m^2
    \end{cases}
\end{align*}
o que foi realizado, e somando-se as contribuições não foi encontrado nenhum cancelamento que gerasse disparidade com a análise de \textit{power-counting}. Acreditamos que 
este modelo escalar seja simplório demasiado para demonstrar qualquer tipo de cancelamento, principalmente devido absência de numeradores nas regras de feynman, assim, iremos continuar a análise 
em um modelo mais intrincado, outra teoria proposta em \autocite{johanssonUnravelingConformalGravity2018}, que possui propriedades semelhantes à nosso modelo escalar e à gravitação quadrática, 
a teoria $\qty(DF)^2$ mínima,\[\mathcal L = -\frac12 D_\mu F^{\mu\nu}D_{\rho}\tensor{F}{^\rho_\nu}-\frac{m^2}{4}F_{\mu\nu}F^{\mu\nu}\]