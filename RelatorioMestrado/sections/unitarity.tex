\subsection{Unitariedade em TQC}

Em teoria quântica de campos existem vários conceitos fundamentais, entre eles: Unitariedade, Localidade, Causalidade. Apesar de conceitos 
completamente independentes, suas aparições em TQC apresentam correlações. Estamos aqui interessados especificamente no primeiro, Unitariedade. 
Em mecânica quântica, o conceito de unitariedade está associado à evolução temporal e conservação de probabilidades, isto é, quando dizemos que 
uma determinada teoria quântica é unitária, estamos afirmando que o operador de evolução temporal $U\qty(t_1,t_0)$ satisfaz $U^\dagger\qty(t_1,t_0)U\qty(t_1,t_0)=\mathbb 1$. 
Isto implica na conservação de probabilidades em relação a evolução temporal, \[\Psi(t_0)\rightarrow \qty(\Psi\qty(t_1),\Psi\qty(t_1))=\qty(U\qty(t_1,t_0)\Psi\qty(t_0),U\qty(t_1,t_0)\Psi\qty(t_0))=\qty(\Psi\qty(t_0),U^\dagger\qty(t_1,t_0)U\qty(t_1,t_0)\Psi\qty(t_0))=\qty(\Psi\qty(t_0),\Psi\qty(t_0))\]