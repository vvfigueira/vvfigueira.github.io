\subsection{Comportamento no UV}

Ao introduzir diagramas de \textit{loops}, naturalmente estes introduzem \textbf{divergências}, porém, felizmente, atualmente está bem consolidado tanto conceitualmente, quanto algoritmicamente, 
procedimentos para ``dar cabo'' à esses ``problemas''. Teorias nas quais é possível de se controlar as divergências, \textbf{sem} perder o caráter preditivo da teoria, são 
ditas \textbf{renormalizáveis}, no caso contrário, \textbf{não--renormalizáveis}. Como exemplo de diagrama divergente podemos voltar à nossa teoria teste,
\begin{align*}
    \raisebox{+3pt}{\begin{tikzpicture}[baseline=(a.base)]
        \begin{feynman}\diagram* {
        a -- [scalar] b -- [ half left,scalar] c-- [scalar] d,c --[half left,scalar] b
        };
        \end{feynman}
    \end{tikzpicture}}&=\im\frac{g^2}{2}\int\frac{\dd[4]{\ell}}{\qty(2\pi)^4}\frac{1}{\qty(\ell^2+m^2)\qty(\qty(\ell-p)^2+m^2)}\sim\lim\limits_{\Lambda\rightarrow\infty}\int\limits_{0}^\Lambda\dd{\ell}\frac{1}{\ell}\sim\lim\limits_{\Lambda\rightarrow \infty}\ln\qty[\Lambda],
\end{align*}
obviamente, essa análise da divergência é apenas pictórica e altamente ingênua. ``Sabemos'' que deve haver divergências na região de integração $\ell\rightarrow \infty$, o grande problema em analisar essa 
região é a métrica indefinida do espaço de Minkowski que força $\ell\rightarrow \infty \not\Rightarrow\ell^2\rightarrow\infty$, tome como exemplo a região $\ell^\mu=\mqty(\Lambda&0&0&\Lambda)$, que respeita 
$\ell\rightarrow\infty$, mas, $\ell^2\equiv0$. A análise é ainda mais dificultada quanto maior o número de loops presentes no diagrama.

A ``única'' maneira de se analisar fielmente o comportamento no UV é realizar uma rotação de Wick para o espaço euclidiano. Isto é, tomando em conta a prescrição de $\im\epsilon$, rotacionamos 
o contorno da integral de $\dd{\ell^0}$ para o eixo imaginário, desviando dos polos, assim, $\ell^2\rightarrow -\ell^2<0$, onde agora $\ell^2$ é definido com a métrica euclidiana. Neste caso, 
o limite $\ell\rightarrow\infty\Rightarrow \ell^2\rightarrow\infty$ faz sentido. Há apenas uma obstrução para este procedimento, que pode ser vista no exemplo de integral divergente que dêmos: 
em geral o integrando não é uma função somente de $\ell^2$, mas também depende de contrações com momentos externos da forma $\qty(\ell-p)^2$, portanto, é necessário realizar uma reparametrização 
antes de aplicar a rotação de Wick. Esses procedimentos não são em sua totalidade difíceis, porém, para grande número de \textit{loops} e pernas externas, se tornam demasiado longas as contas, 
além de que esta análise é para somente um diagrama. Para topologias que permitem vários diagramas se torna impraticável a análise, além de ser obscuro a presença ou não de cancelamentos entre 
contribuições de diagramas. Este é o principal ponto que estamos interessados em analisar, em \autocite{herrmannUVCancelationsGravity2019} foi de fato mostrado que cancelamentos entre certos diagramas 
divergentes ocorrem, gerando amplitudes que não possuem algumas das divergências esperadas por análise ``ingênua'' de \textit{power--counting}.

Conforme mencionamos, é impraticável realizar uma análise de divergência para um número de pernas externas e \textit{loops} arbitrário, ao menos utilizando-se o método usual. O algorítimo 
utilizado em \autocite{herrmannUVCancelationsGravity2019} para se mostrar os cancelamentos foi de analisar a existência ou não de determinados polos no infinito de \textbf{cortes} de diagramas. 
O motivo dos cortes é: cortes eliminam parte da liberdade dos momentos de \textit{loop}, assim, com uma escolha esperta de cortes podemos restringir a região dos momentos de \textit{loop} para uma 
região do espaço de integração na qual o limite $\ell\rightarrow \infty$ faça sentido. Nossa teoria teste é muito simples para esperarmos qualquer tipo de cancelamento em cortes, porém, ela 
servirá como uma arena de aprendizagem das técnicas necessárias para se realizar o cálculo em teorias mais complexas.