\subsection{Unitariedade generalizada}

No item anterior, demos uma breve descrição de como a unitariedade da matrix S --- que nos permite calcular 
a descontinuidade de uma amplitude partindo de amplitudes de menor ordem ---, suplementada por hipóteses 
adicionais da estrutura analítica, nos permite de obter a amplitude completa, à nível árvore, partindo de amplitudes de menor 
ordem. Vamos agora descrever como esse método de unitariedade pode ser estendido para \textbf{unitariedade 
generalizada}, que nos permite obter amplitudes de \textit{loop} partindo de amplitudes de menor ordem. 
Claro que isso só é possível suplementando com condições da estrutura analítica.

Para isso, começamos partindo de uma amplitude $\mathcal A^L_{\gamma\alpha}$ de $L$--\textit{loops}. Sabemos da estrutura das regras de Feynman 
que tal amplitude pode ser escrita como,
\[\mathcal A^L_{\gamma\alpha}=\sum\limits_a\int\prod\limits_{n=0}^L\frac{\dd[4]{\ell_n}}{\qty(2\pi)^4}\mathcal I^{L,a}_{\gamma\alpha},\]
no qual $a$ indexa diferentes topologias que contribuem para este processo, e chamamos o objeto $\mathcal I^{L,a}_{\gamma\alpha}$ de integrando 
da amplitude. Ainda mais, o integrando possui forma bem definida, provindas das regras de Feynman,
\[\mathcal I^{L,a}_{\gamma\alpha}=\frac{1}{S_a}\frac{n_ac_a}{\prod\limits_{\beta_a}\qty(p_{\beta_a}^2+m^2_\beta)},\]
$n_a$ engloba toda a dependência cinemática do numerador, $c_a$ são fatores puramente numéricos dependentes dos acoplamentos e 
dos grupos de calibre, e $S_a$ são apenas fatores de simetria. O caso ideal seria se a soma sobre topologias comutasse com a integral dos 
momentos de \textit{loop}, e pudéssemos definir o seguinte objeto,
\[\mathcal A^L_{\gamma\alpha}=\sum\limits_a\int\prod\limits_{n=0}^L\frac{\dd[4]{\ell_n}}{\qty(2\pi)^4}\mathcal I^{L,a}_{\gamma\alpha}\stackrel{\text{?}}{=}\int\prod\limits_{n=0}^L\frac{\dd[4]{\ell_n}}{\qty(2\pi)^4}\sum\limits_a\mathcal I^{L,a}_{\gamma\alpha}\stackrel{\text{?}}{=}\int\prod\limits_{n=0}^L\frac{\dd[4]{\ell_n}}{\qty(2\pi)^4}\mathcal I^{L}_{\gamma\alpha},\]
a obstrução para se fazer isso é que não existe uma maneira de se definir globalmente como os momentos de loop são definidos no objeto $\mathcal I^{L,a}_{\gamma\alpha}$. 
Existem teorias em que isso sim é possível, porém, no caso geral não é verdade. Assim como mostramos no conceito de unitariedade da matrix S, 
a amplitude completa á nível árvore pode ser reconstruída em sua totalidade sabendo apenas o valor de seu resíduo em,
\[\qty(p_\alpha^2+m_\beta^2)\mathcal A_{\gamma\alpha}\eval_{p_\alpha^2=-m_\beta^2}=\mathfrak{Res}\qty[\mathcal A_{\gamma\alpha}]=\tilde{\mathcal A}^\ast_{\beta\gamma}\tilde{\mathcal A}_{\beta\alpha}\eval_{p_\alpha^2=-m_\beta^2},\]
com o mesmo espírito, propomos que o integrando de amplitudes de \textit{loops} possa ser reconstruído fazendo 
a igualdade de seus resíduos em \textbf{cortes} com o produto de amplitudes de menor ordem. Por exemplo, a unitariedade por sí é capaz de nos garantir, 
\[\qty(\ell_i^2+m_{\beta_1}^2)\qty(\qty(p_\alpha-\ell_i)^2+m^2_{\beta_2})\mathcal I^{L}_{\gamma\alpha}\eval_{\ell_i^2=-m_{\beta_1}^2,\qty(p_\alpha-\ell_i)^2=-m^2_{\beta_2}}=\sum\limits_{n=0}^L\sum\limits_{\beta}\mathcal I^{L-n\ast}_{\beta\gamma}\mathcal I^{n}_{\beta\alpha}\eval_{\ell_i^2=-m_{\beta_1}^2,\qty(p_\alpha-\ell_i)^2=-m^2_{\beta_2}},\]
contudo, ela somente nos é capaz de nos dizer algo sobre \textbf{cortes} verticais do diagrama, isto é, \textbf{cortes} 
da forma, 
\begin{align*}
    \mathfrak{Res}\qty[\raisebox{+3pt}{\begin{tikzpicture}[baseline=(b.base)]
        \begin{feynman}
            \vertex (i1) at (-1,1);
            \vertex (i2) at (-1,-1);
            \vertex [blob, minimum size=1cm] (b) at (0,0) {};
            \placedotsbetween{b}{i2}{i1}{0.7cm}{4}
            \vertex (f1) at (1,1);
            \vertex (f2) at (1,-1);
            \placedotsbetween{b}{f1}{f2}{0.7cm}{4}
            \diagram*{
                (i1) -- [fermion] (b) [blob],
                (i2) -- [fermion] (b),
                (b) -- [fermion] (f1),
                (b) -- [fermion] (f2),
            };
        \end{feynman}
    \end{tikzpicture}}]&=\sum\raisebox{+3pt}{\begin{tikzpicture}[baseline=(b.base)]
        \begin{feynman}
            \vertex (i1) at (-1,1);
            \vertex (i2) at (-1,-1);
            \vertex [blob, minimum size=1cm] (b) at (0,0) {};
            \placedotsbetween{b}{i1}{i2}{0.7cm}{4};
            \vertex [blob, minimum size=1cm] (B) at (2,0) {};
            \vertex (d1) at (1,0) {$\raisebox{6pt}{\vdots}$};
            \vertex (F1) at (2.8,0.6);
            \vertex (F2) at (2.8,-0.6);
            \placedotsbetween{B}{F1}{F2}{0.7cm}{4};
            \vertex (mid1) at (3.3,-0.6) {$\cdots$};
            \vertex (mid2) at (3.3,0) {$\cdots$};
            \vertex (mid3) at (3.3,0.6) {$\cdots$};
            \vertex (II1) at (3.8,0.6);
            \vertex (II2) at (3.8,-0.6);
            % \vertex (DD2) at (3.9,0) {$\raisebox{6pt}{\vdots}$};
            \vertex [blob, minimum size=1cm] (BB) at (4.6,0) {};
            \placedotsbetween{BB}{II1}{II2}{0.7cm}{4};
            \vertex (dd1) at (5.6,0) {$\raisebox{6pt}{\vdots}$};
            \vertex [blob, minimum size=1cm] (bb) at (6.6,0) {};
            \vertex (ii1) at (7.6,1);
            \vertex (ii2) at (7.6,-1);
            \placedotsbetween{bb}{ii1}{ii2}{0.7cm}{4};
            \diagram*{
                (i1) -- [fermion] (b),
                (i2) -- [fermion] (b),
                (b) -- [fermion, bend left=45, looseness=1, cut2] (B),
                (b) -- [fermion, bend right=45, looseness=1, cut2] (B),
                (B) -- [fermion, bend left=10, looseness=1] (F1),
                (B) -- [fermion, bend right=10, looseness=1] (F2),
                (II1) -- [fermion, bend left=10, looseness=1] (BB),
                (II2) -- [fermion, bend right=10, looseness=1] (BB),
                (BB) -- [fermion, bend left=45, looseness=1, cut] (bb),
                (BB) -- [fermion, bend right=45, looseness=1, cut] (bb),
                (bb) -- [fermion] (ii1),
                (bb) -- [fermion] (ii2),
            };
        \end{feynman}
    \end{tikzpicture}}
\end{align*}
Apesar desta topologia de resíduos nos proporcionar muita informação sobre a amplitude, para $L\geq 1$, essa 
informação apenas não é suficiente para conseguirmos reconstruir a amplitude, pois, há outros resíduos possíveis 
de serem tomados que não são levados em conta na aplicação da unitariedade convencional. Um exemplo é, 
\begin{align*}
    \mathfrak{Res}\qty[\raisebox{+3pt}{\begin{tikzpicture}[baseline=(b.base)]
        \begin{feynman}
            \vertex (i1) at (-1,1);
            \vertex (i2) at (-1,-1);
            \vertex [blob, minimum size=1cm] (b) at (0,0) {};
            \placedotsbetween{b}{i1}{i2}{0.7cm}{4}
            \vertex (f1) at (1,1);
            \vertex (f2) at (1,-1);
            \placedotsbetween{b}{f1}{f2}{0.7cm}{4};
            \diagram*{
                (i1) -- [fermion] (b) [blob],
                (i2) -- [fermion] (b),
                (b) -- [fermion] (f1),
                (b) -- [fermion] (f2),
            };
        \end{feynman}
    \end{tikzpicture}}]&=\sum\raisebox{+3pt}{\begin{tikzpicture}[baseline=(b.base)]
        \begin{feynman}
            \vertex (i1) at (-1,1);
            \vertex (i2) at (-1,-1);
            \vertex [blob, minimum size=1cm] (b) at (0,0) {};
            \placedotsbetween{b}{i1}{i2}{0.7cm}{4};
            % \vertex (d2) at (1.04,0) {$\raisebox{0pt}{\rotatebox[origin=c]{90}{\therefore}}$};
            \vertex [blob, minimum size=1cm] (B1) at (1.7,1) {};
            \vertex [blob, minimum size=1cm] (B2) at (1.7,-1) {};
            \vertex (f1) at (2.7,1);
            \vertex (f2) at (2.7,-1);
            \placedotsbetween{B1}{B2}{f1}{0.7cm}{4};
            \placedotsbetween{B2}{f2}{B1}{0.7cm}{4};
            \placedotsbetween{b}{B1}{B2}{0.7cm}{3};
            \placedotsbetween{B1}{b}{B2}{0.7cm}{3};
            \placedotsbetween{B2}{b}{B1}{0.7cm}{3};
            \diagram*{
                (i1) -- [fermion] (b),
                (i2) -- [fermion] (b),
                (b) -- [fermion, cut2] (B1),
                (B2) -- [fermion, cut2] (b),
                (B1) -- [fermion, cut2] (B2),
                (B1) -- [fermion] (f1),
                (B2) -- [fermion] (f2),
            };
        \end{feynman}
    \end{tikzpicture}}
\end{align*}
que corresponderia à,
\[\qty(\ell^2+m_{\beta_3}^2)\qty(\qty(p_{\gamma_1}-\ell)^2+m^2_{\beta_1})\qty(\qty(p_{\gamma_2}-\ell)^2+m^2_{\beta_2})\mathcal I^{L}_{\gamma\alpha}\eval_{\textnormal{Cuts}}\stackrel{?}{=}\sum\limits_{n=0}^L\sum\limits_{m=0}^{L-n}\sum\limits_{\beta_1,\beta_2,\beta_3}\mathcal I^{L-n-m\ast}_{\qty{\beta_2,\beta_3}\gamma_2}\mathcal I^{m\ast}_{\qty{\beta_1,\beta_2}\gamma_1}\mathcal I^{n}_{\qty{\beta_1,\beta_2}\alpha}\eval_{\textnormal{Cuts},\gamma=\gamma_1\cup\gamma_2}\]
o grande problema com esse tipo de corte é: nos assumimos que os estados iniciais e finais em integrandos $\mathcal I_{\beta\alpha}^L$ são tais que 
satisfazem $p_\alpha^0,p_\beta^0>0$, porém, neste caso é fácil de se mostrar que não é possível de se satisfazer essa condição para a solução dos 
cortes, assim, as amplitudes que aparecem nos resíduos dos cortes não possuem de fato uma relação rígida de partículas iniciais e finais, por conta do 
sinal da componente temporal do momento não ser positivo --- podendo ser complexo ---. Então, se estamos interessados em obter o máximo de 
informação sobre amplitudes de ordens superiores partindo de ordens inferiores, é necessário obtermos um formalismo que trate partículas externas 
de maneira uniforme, isto é, um formalismo que seja indiferente de $p^0>0$ e trate igualmente partículas internas e externas. 

Para isso, ao invés de utilizarmos como amplitude $\mathcal A_{\beta \alpha}$. que diferencia os estados iniciais $\alpha$ dos finais $\beta$, vamos fazer $p_{\alpha_i}\rightarrow -p_{\alpha_i}$, 
de modo que a conservação de momento passa a ser $\sum\limits_i^np_i=0$. A amplitude $\mathcal A_{\beta \alpha}$ vista como função dos momentos $p_{\beta_i},-p_{\alpha_j}$ para $n$ partículas 
externas será denominada $\mathcal A_n\qty(1^{h_1},\cdots,n^{h_n})$, no qual $i$ refere-se ao momento da $i$--ésima partícula e $h_i$ à helicidade/spin da $i$--ésima partícula. Nota-se a 
uniformidade do tratamento de partículas iniciais e finais, a única diferença é que partículas com $p^0_i>0$ devem serem interpretadas como finais e partículas com $p^0_i<0$ como iniciais. De certa 
forma, podemos interpretar essa nova função das variáveis cinemáticas como sendo uma abreviação de todas as amplitudes possíveis de $n$ pontos, isto é,
\[\mathcal A_n\qty(1^{h_1},\cdots, n^{h_n})=\sum\limits_{\alpha,\beta}\prod\limits_{i\in\alpha,j\in\beta}\theta\qty(-p^0_{\alpha_i}>0)\theta\qty(p^0_{\beta_j}>0)\mathcal A_{\beta\alpha},\]
devido às funções de Heaviside presentes, produtos dessas amplitudes fatorizam, e portanto, podemos utilizar do resultado da \eqref{residuotreelevelsmatrix} e obter uma expressão ainda mais geral,
\begin{align*}
    \mathcal A_n&=\sum\limits_{I,m}\frac{1}{P_I^2+m_\beta^2}\tilde{\mathcal A}_{n-m+2}\tilde{\mathcal A}_{m}\eval_{z=z_\beta}+\lim\limits_{\abs{z}\rightarrow \infty}\tilde{\mathcal A}_{n}\\
    \qty(P_I^2+m_I^2)\mathcal A_n\eval_{P_I^2=-m_\beta^2}&=\mathfrak{Res}\qty[\mathcal A_n]=\sum\limits_{m=3}^{n-1}\tilde{\mathcal A}_{n-m+2}\tilde{\mathcal A}_{m}\eval_{P_I^2=-m_I^2}
\end{align*}
É claro, para teorias nas quais o termo de borda é zero. Assim, podemos utilizar este formalismo para calcular cortes do tipo triângulo mostrados anteriormente, devido a não ser necessário se preocupar sobre o 
sinal dos momentos. Escrevemos então analogamente,
\[\mathcal A_n^{L}=\int\prod\limits_{j=1}^L\frac{\dd[4]{\ell_j}}{\qty(2\pi)^4}\mathcal I^L_n,\]
e agora nos voltamos sobre a construibilidade do integrando por meio de cortes. Bem, termos não construíveis são analíticos em $\mathbb C/\infty$ e estão relacionados com \textit{tadpoles}, 
por hora iremos ignorar estes. 

Comecemos com um exemplo à $1$--\textit{loop}, neste caso há apenas uma única integral de uma única variável de momento de \textit{loop}, $\ell$. Assim, como este possui apenas $4$ componentes, 
podemos ao máximo realizar um corte quádruplo, sendo qualquer outro número superior de cortes zero. Assim, olhemos para a maior ordem de corte, 
\[\qty(\ell^2+m_1^2)\qty(\qty(\ell-P_I)^2+m_2^2)\qty(\qty(\ell-P_J)^2+m_3^2)\qty(\qty(\ell-P_K)^2+m_4^2)\mathcal I^1_n\eval_{\textnormal{Cuts}}=\sum\limits_{k,l,m,p}^{k+l+m+p=n}\mathcal A_{k+2}\mathcal A_{l+2}\mathcal A_{m+2}\mathcal A_{p+2},\]
as escolhas de $P_{I,J,K}$ ditam as topologias que aparecerão do lado direito, porém, note que o lado direito não possui nenhuma menção à variável de integração $\ell$, assim,
\[\mathcal I^1_n\supseteq\sum\limits_{m_i}\sum\limits_{k,l,m,p}^{k+l+m+p=n}\mathcal A_{k+2}\mathcal A_{l+2}\mathcal A_{m+2}\mathcal A_{p+2}\int\frac{\dd[4]{\ell}}{\qty(2\pi)^4}\frac{1}{\qty(\ell^2+m_1^2)\qty(\qty(\ell-P_I)^2+m_2^2)\qty(\qty(\ell-P_J)^2+m_3^2)\qty(\qty(\ell-P_K)^2+m_4^2)},\]
isto é um exemplo de reconstrução, em geral, podemos fatorizar $\mathcal I^L_n$ em uma base de integrais escalares com coeficientes que englobam dependências cinemáticas, vetores de polarizações e 
elementos de álgebras. Claro, obtemos aqui somente um dos termos de $\mathcal I^1_n$, é possível que haja contribuições com menor número de polos, e para obter estas seria necessário tomar 
mais resíduos. Vamos demonstrar como isso pode ser feito na teoria teste, $g\phi^3$,
\begin{align*}
    \mathcal I^1_4\eval_{4\textnormal{ Cuts}} &=\raisebox{+3pt}{\begin{tikzpicture}[baseline=(b.base)]
        \begin{feynman}
            \vertex (i1) at (-1,0.5);
            \vertex (i2) at (-1,-0.5);
            \vertex [solid blob=gray, minimum size=0.3cm] (b1) at (-0.5,0.5) {};
            \vertex [solid blob=gray, minimum size=0.3cm] (b2) at (-0.5,-0.5) {};
            \vertex [solid blob=gray, minimum size=0.3cm] (b3) at (0.5,0.5) {};
            \vertex [solid blob=gray, minimum size=0.3cm] (b4) at (0.5,-0.5) {};
            \vertex (b) at (0,0) {};
            \vertex (f1) at (1,0.5);
            \vertex [particle =\(4\)] (f2) at (1,-0.5);
            \diagram*{
                (i1) [particle =\(2\)]-- [scalar] (b1),
                (i2) [particle =\(1\)]-- [scalar] (b2),
                (b1) -- [scalar, cut2, momentum = \(\ell\)] (b3),
                (b3) -- [scalar, cut2, momentum = \(\ell-3\)] (b4),
                (b4) -- [scalar, cut2, momentum = \(\ell-3-4\)] (b2),
                (b2) -- [scalar, cut2, momentum = \(\ell+2\)] (b1),
                (b3) -- [scalar] (f1)[particle =\(3\)],
                (b4) -- [scalar] (f2),
            };
        \end{feynman}
    \end{tikzpicture}}=\mathcal A_3\qty(2,\ell, -\ell-2)\mathcal A_3\qty(-\ell,\ell-3,3)\mathcal A_3\qty(4,3-\ell, \ell-3-4)\mathcal A_3\qty(1, \ell+2,3+4-\ell)\\
    \mathcal I^1_4\eval_{4\textnormal{ Cuts}}&=g^4, 
\end{align*}
que implica,
\[\mathcal I^1_4 \supseteq g^4\int\frac{\dd[4]{\ell}}{\qty(2\pi)^4}\frac{1}{\qty(\ell^2+m^2)\qty(\qty(\ell+2)^2+m^2)\qty(\qty(\ell-3)^2+m^2)\qty(\qty(\ell-3-4)^2+m^2)}.\]
Obviamente esse resulta é somente para uma única contribuição de topologia no corte de $4$ partículas, há mais outras cinco contribuições que podem ser obtidas por permutação das pernas externas. 
Procedemos para o corte de $3$ partículas,
\begin{align*}
    \mathcal I^1_4\eval_{3\textnormal{ Cuts}} &=\raisebox{+3pt}{\begin{tikzpicture}[baseline=(b.base)]
        \begin{feynman}
            \vertex (i1) at (-1,0.5);
            \vertex (i2) at (-1,-0.5);
            \vertex [solid blob=gray, minimum size=0.3cm] (b1) at (-0.5,0.5) {};
            \vertex [solid blob=gray, minimum size=0.3cm] (b2) at (-0.5,-0.5) {};
            \vertex [solid blob=gray, minimum size=0.3cm] (b3) at (0.3,0) {};
            \vertex (b) at (0,0) {};
            \vertex (f1) at (1,0.5);
            \vertex [particle =\(4\)] (f2) at (1,-0.5);
            \diagram*{
                (i1) [particle =\(2\)]-- [scalar] (b1),
                (i2) [particle =\(1\)]-- [scalar] (b2),
                (b1) -- [scalar, cut2, momentum = \(\ell\)] (b3),
                (b3) -- [scalar, cut2, momentum = \(\ell-3-4\)] (b2),
                (b2) -- [scalar, cut2, momentum = \(\ell+2\)] (b1),
                (b3) -- [scalar] (f1)[particle =\(3\)],
                (b3) -- [scalar] (f2),
            };
        \end{feynman}
    \end{tikzpicture}}=\mathcal A_3\qty(2,\ell, -\ell-2)\mathcal A_3\qty(1, \ell+2,3+4-\ell)\mathcal A_4\qty(-\ell,\ell-3-4,3,4)\\
    \mathcal I^1_4\eval_{3\textnormal{ Cuts}}&=g^4\qty[\frac{1}{m^2+\qty(3+4)^2}+\frac{1}{m^2+\qty(3-\ell)^2}+\frac{1}{m^2+\qty(4-\ell)^2}], 
\end{align*}
Certamente esta é apenas uma contribuição de outras duas, note que os dois últimos termos correspondem ao mesmo resíduo da expressão com $4$ polos, portanto apenas o primeiro termo 
nos diz algo não trivial sobre a parcela com $3$ polos,
\[\mathcal I^1_3\supseteq \frac{g^4}{m^2+\qty(3+4)^2}\int\frac{\dd[4]{\ell}}{\qty(2\pi)^4}\frac{1}{\qty(\ell^2+m^2)\qty(\qty(\ell+2)^2+m^2)\qty(\qty(\ell-3-4)^2+m^2)}.\]
O corte duplo é,
\begin{align*}
    \mathcal I^1_4\eval_{2\textnormal{ Cuts}} &=\raisebox{+3pt}{\begin{tikzpicture}[baseline=(b.base)]
        \begin{feynman}
            \vertex (i1) at (-1,0.5);
            \vertex (i2) at (-1,-0.5);
            \vertex [solid blob=gray, minimum size=0.3cm] (b1) at (-0.5,0) {};
            \vertex [solid blob=gray, minimum size=0.3cm] (b2) at (0.5,0) {};
            \vertex (b) at (0,0) {};
            \vertex (f1) at (1,0.5);
            \vertex [particle =\(4\)] (f2) at (1,-0.5);
            \diagram*{
                (i1) [particle =\(2\)]-- [scalar] (b1),
                (i2) [particle =\(1\)]-- [scalar] (b1),
                (b1) -- [scalar, cut2, momentum = \(\ell\), bend left=45, looseness=1] (b2),
                (b2) -- [scalar, cut2, momentum = \(\ell-3-4\), bend left=45, looseness=1] (b1),
                (b2) -- [scalar] (f1)[particle =\(3\)],
                (b2) -- [scalar] (f2),
            };
        \end{feynman}
    \end{tikzpicture}}=\frac12\mathcal A_4\qty(1,2,\ell, -\ell-2)\mathcal A_4\qty(-\ell,\ell-3-4,3,4)\\
    \mathcal I^1_4\eval_{3\textnormal{ Cuts}}&=\frac12g^4\qty[\frac{1}{m^2+\qty(1+2)^2}+\frac{1}{m^2+\qty(1+\ell)^2}+\frac{1}{m^2+\qty(2+\ell)^2}]\qty[\frac{1}{m^2+\qty(3+4)^2}+\frac{1}{m^2+\qty(3-\ell)^2}+\frac{1}{m^2+\qty(4-\ell)^2}], 
\end{align*}
realizando a mesma análise anterior, existe apenas um termo ligado a contribuição de $2$ polos,
\[\mathcal I^1_3\supseteq \frac{g^4}{2\qty(m^2+\qty(3+4)^2)\qty(m^2+\qty(1+2)^2)}\int\frac{\dd[4]{\ell}}{\qty(2\pi)^4}\frac{1}{\qty(\ell^2+m^2)\qty(\qty(\ell-3-4)^2+m^2)}.\]
Nós poderíamos ir ainda mais além e tentar calcular o corte simples,
\[\mathcal I^1_4\eval_{1\textnormal{ Cuts}}=\raisebox{+3pt}{\begin{tikzpicture}[baseline=(b.base)]
        \begin{feynman}
            \vertex (i1) at (-1,0.5);
            \vertex (i2) at (-1,-0.5);
            \vertex [solid blob=gray, minimum size=0.3cm] (b1) at (0,0) {};
            \vertex (c) at (0,0.7) {};
            \vertex (b) at (0,0) {};
            \vertex (f1) at (1,0.5);
            \vertex [particle =\(4\)] (f2) at (1,-0.5);
            \diagram*{
                (i1) [particle =\(2\)]-- [scalar] (b1),
                (i2) [particle =\(1\)]-- [scalar] (b1),
                % (b1) -- [scalar, cut2, momentum = \(\ell\), out=60, in=120, looseness=8] (b1),
                (b1) -- [scalar] (f1)[particle =\(3\)],
                (b1) -- [scalar] (f2),
            };
            \draw[scalar,cut2](b1) .. controls +(0.8,1) and +(-0.8,1) .. (b1);
        \end{feynman}
    \end{tikzpicture}}=\mathcal A_6\qty(1,2,3,4,\ell,-\ell),\]
porém, este tipo de corte não está bem definido em TQC no geral, principalmente devido a divergências no limite colinear de amplitudes, contudo, não há problema nisto, cortes como esses 
iriam apenas contribuir para \textit{tadpoles}, os quais após renormalização devem ser zero, portanto não iremos nos preocupar. Note que para o caso desta teoria teste é sim 
possível obter o integrando $\mathcal I^1_4$ completo partindo da unitariedade generalizada.
% Para obter tal formalismo precisamos supor várias coisas, primeiramente, LSZ. Isto é, um elemento de matriz S é obtido por,
% \[S_{\beta\alpha}=\im^n\int\prod\limits_{a=1}^n\dd[4]{x_a}\qty(\textnormal{K.O.})_a\epsilon_{\sigma_a}^{I_a}\qty(k_a)\exp\qty((\pm)_a\im k_a\cdot x_a)\qty(\Omega,\textnormal{T}\qty{\Phi^{I_a}_a\qty(x_a)\cdots}\Omega)\]
% como de costume, o sinal $\pm$ dentro da exponencial nos indica se determinada partícula faz parte da coleção inicial ou final, mas claro, isso só é verdade se $k_a^0>0$. Podemos 
% absorver este fator dentro dos momentos, assim, alguns possuirão $k_a<0$, nisto