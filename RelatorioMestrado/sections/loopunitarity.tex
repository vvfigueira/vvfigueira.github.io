\subsection{Unitariedade generalizada}

No item anterior, demos uma breve descrição de como a unitariedade da matrix S --- que nos permite calcular 
a descontinuidade de uma amplitude partindo de amplitudes de menor ordem ---, suplementada por hipóteses 
adicionais da estrutura analítica, nos permite de obter a amplitude completa, à nível árvore, partindo de amplitudes de menor 
ordem. Vamos agora descrever como esse método de unitariedade pode ser estendido para \textbf{unitariedade 
generalizada}, que nos permite obter amplitudes de \textit{loop} partindo de amplitudes de menor ordem. 
Claro que isso só é possível suplementando com condições da estrutura analítica.

Para isso, começamos partindo de uma amplitude $\mathcal A^L_{\gamma\alpha}$ de $L$--\textit{loops}. Sabemos da estrutura das regras de Feynman 
que tal amplitude pode ser escrita como,
\[\mathcal A^L_{\gamma\alpha}=\sum\limits_a\int\prod\limits_{n=0}^L\frac{\dd[4]{\ell_n}}{\qty(2\pi)^4}\mathcal I^{L,a}_{\gamma\alpha},\]
no qual $a$ indexa diferentes topologias que contribuem para este processo, e chamamos o objeto $\mathcal I^{L,a}_{\gamma\alpha}$ de integrando 
da amplitude. Ainda mais, o integrando possui forma bem definida, provindas das regras de Feynman,
\[\mathcal I^{L,a}_{\gamma\alpha}=\frac{1}{S_a}\frac{n_ac_a}{\prod\limits_{\beta_a}\qty(p_{\beta_a}^2+m^2_\beta)},\]
$n_a$ engloba toda a dependência cinemática do numerador, $c_a$ são fatores puramente numéricos dependentes dos acoplamentos e 
dos grupos de calibre, e $S_a$ são apenas fatores de simetria. O caso ideal seria se a soma sobre topologias comutasse com a integral dos 
momentos de \textit{loop}, e pudéssemos definir o seguinte objeto,
\[\mathcal A^L_{\gamma\alpha}=\sum\limits_a\int\prod\limits_{n=0}^L\frac{\dd[4]{\ell_n}}{\qty(2\pi)^4}\mathcal I^{L,a}_{\gamma\alpha}\stackrel{\text{?}}{=}\int\prod\limits_{n=0}^L\frac{\dd[4]{\ell_n}}{\qty(2\pi)^4}\sum\limits_a\mathcal I^{L,a}_{\gamma\alpha}\stackrel{\text{?}}{=}\int\prod\limits_{n=0}^L\frac{\dd[4]{\ell_n}}{\qty(2\pi)^4}\mathcal I^{L}_{\gamma\alpha},\]
a obstrução para se fazer isso é que não existe uma maneira de se definir globalmente como os momentos de loop são definidos no objeto $\mathcal I^{L,a}_{\gamma\alpha}$. 
Existem teorias em que isso sim é possível, porém, no caso geral não é verdade. Assim como mostramos no conceito de unitariedade da matrix S, 
a amplitude completa á nível árvore pode ser reconstruída em sua totalidade sabendo apenas o valor de seu resíduo em,
\[\qty(p_\alpha^2+m_\beta^2)\mathcal A_{\gamma\alpha}=\mathfrak{Res}\qty[\mathcal A_{\gamma\alpha}]=\tilde{\mathcal A}^\ast_{\beta\gamma}\tilde{\mathcal A}_{\beta\alpha}\eval_{p_\alpha^2=-m_\beta^2},\]
com o mesmo espírito, propomos que o integrando de amplitudes de \textit{loops} possa ser reconstruído fazendo 
a igualdade de seus resíduos em \textbf{cortes} com o produto de amplitudes de menor ordem. Por exemplo, a unitariedade por sí é capaz de nos garantir, 
\[\qty(\ell_i^2+m_{\beta_1}^2)\qty(\qty(p_\alpha-\ell_i)^2+m^2_{\beta_2})\mathcal I^{L}_{\gamma\alpha}=\sum\limits_{n=0}^L\sum\limits_{\beta}\mathcal I^{L-n\ast}_{\beta\gamma}\mathcal I^{n}_{\beta\alpha},\]
contudo, ela somente nos é capaz de nos dizer algo sobre \textbf{cortes} verticais do diagrama, isto é, \textbf{cortes} 
da forma, 
\begin{align*}
    \mathfrak{Res}\qty[\raisebox{+3pt}{\begin{tikzpicture}[baseline=(b.base)]
        \begin{feynman}
            \vertex (i1) at (-1,1);
            \vertex (i2) at (-1,-1);
            \vertex [blob, minimum size=1cm] (b) at (0,0) {};
            \vertex (d1) at (-1,0) {$\raisebox{6pt}{\vdots}$};
            \vertex (f1) at (1,1);
            \vertex (d2) at (1,0) {$\raisebox{6pt}{\vdots}$};
            \vertex (f2) at (1,-1);
            \diagram*{
                (i1) -- [fermion] (b) [blob],
                (i2) -- [fermion] (b),
                (b) -- [fermion] (f1),
                (b) -- [fermion] (f2),
            };
        \end{feynman}
    \end{tikzpicture}}]&=\sum\raisebox{+3pt}{\begin{tikzpicture}[baseline=(b.base)]
        \begin{feynman}
            \vertex (i1) at (-1,1);
            \vertex (i2) at (-1,-1);
            \vertex [blob, minimum size=1cm] (b) at (0,0) {};
            \vertex (d1) at (-1,0) {$\raisebox{6pt}{\vdots}$};
            \vertex (d2) at (1,0) {$\raisebox{6pt}{\vdots}$};
            \vertex [blob, minimum size=1cm] (B) at (2,0) {};
            \vertex (D2) at (2.7,0) {$\raisebox{6pt}{\vdots}$};
            \vertex (F1) at (2.8,0.6);
            \vertex (F2) at (2.8,-0.6);
            \vertex (mid1) at (3.3,-0.6) {$\cdots$};
            \vertex (mid2) at (3.3,0) {$\cdots$};
            \vertex (mid3) at (3.3,0.6) {$\cdots$};
            \vertex (II1) at (3.8,0.6);
            \vertex (II2) at (3.8,-0.6);
            \vertex (DD2) at (3.9,0) {$\raisebox{6pt}{\vdots}$};
            \vertex [blob, minimum size=1cm] (BB) at (4.6,0) {};
            \vertex (dd1) at (5.6,0) {$\raisebox{6pt}{\vdots}$};
            \vertex (dd2) at (7.6,0) {$\raisebox{6pt}{\vdots}$};
            \vertex [blob, minimum size=1cm] (bb) at (6.6,0) {};
            \vertex (ii1) at (7.6,1);
            \vertex (ii2) at (7.6,-1);
            \diagram*{
                (i1) -- [fermion] (b),
                (i2) -- [fermion] (b),
                (b) -- [fermion, bend left=45, looseness=1, cut] (B),
                (b) -- [fermion, bend right=45, looseness=1, cut] (B),
                (B) -- [fermion, bend left=10, looseness=1] (F1),
                (B) -- [fermion, bend right=10, looseness=1] (F2),
                (II1) -- [fermion, bend left=10, looseness=1] (BB),
                (II2) -- [fermion, bend right=10, looseness=1] (BB),
                (BB) -- [fermion, bend left=45, looseness=1, cut] (bb),
                (BB) -- [fermion, bend right=45, looseness=1, cut] (bb),
                (bb) -- [fermion] (ii1),
                (bb) -- [fermion] (ii2),
            };
        \end{feynman}
    \end{tikzpicture}}.
\end{align*}
Apesar desta topologia de resíduos nos proporcionar muita informação sobre a amplitude, para $L\geq 1$, essa 
informação apenas não é suficiente para conseguirmos reconstruir a amplitude, pois, há outros resíduos possíveis 
de serem tomados que não são levados em conta na aplicação da unitariedade convencional. Um exemplo é, 
\begin{align*}
    \mathfrak{Res}\qty[\raisebox{+3pt}{\begin{tikzpicture}[baseline=(b.base)]
        \begin{feynman}
            \vertex (i1) at (-1,1);
            \vertex (i2) at (-1,-1);
            \vertex [blob, minimum size=1cm] (b) at (0,0) {};
            \vertex (d1) at (-1,0) {$\raisebox{6pt}{\vdots}$};
            \vertex (f1) at (1,1);
            \vertex (d2) at (1,0) {$\raisebox{6pt}{\vdots}$};
            \vertex (f2) at (1,-1);
            \diagram*{
                (i1) -- [fermion] (b) [blob],
                (i2) -- [fermion] (b),
                (b) -- [fermion] (f1),
                (b) -- [fermion] (f2),
            };
        \end{feynman}
    \end{tikzpicture}}]&=\sum\raisebox{+3pt}{\begin{tikzpicture}[baseline=(b.base)]
        \begin{feynman}
            \vertex (i1) at (-1,1);
            \vertex (i2) at (-1,-1);
            \vertex [blob, minimum size=1cm] (b) at (0,0) {};
            \vertex (d1) at (-1,0) {$\raisebox{6pt}{\vdots}$};
            \vertex (d2) at (1,0) {$\raisebox{6pt}{\vdots}$};
            \vertex [blob, minimum size=1cm] (B) at (2,0) {};
            \vertex (D2) at (2.7,0) {$\raisebox{6pt}{\vdots}$};
            \vertex (F1) at (2.8,0.6);
            \vertex (F2) at (2.8,-0.6);
            \vertex (mid1) at (3.3,-0.6) {$\cdots$};
            \vertex (mid2) at (3.3,0) {$\cdots$};
            \vertex (mid3) at (3.3,0.6) {$\cdots$};
            \vertex (II1) at (3.8,0.6);
            \vertex (II2) at (3.8,-0.6);
            \vertex (DD2) at (3.9,0) {$\raisebox{6pt}{\vdots}$};
            \vertex [blob, minimum size=1cm] (BB) at (4.6,0) {};
            \vertex (dd1) at (5.6,0) {$\raisebox{6pt}{\vdots}$};
            \vertex (dd2) at (7.6,0) {$\raisebox{6pt}{\vdots}$};
            \vertex [blob, minimum size=1cm] (bb) at (6.6,0) {};
            \vertex (ii1) at (7.6,1);
            \vertex (ii2) at (7.6,-1);
            \diagram*{
                (i1) -- [fermion] (b),
                (i2) -- [fermion] (b),
                (b) -- [fermion, bend left=45, looseness=1, cut] (B),
                (b) -- [fermion, bend right=45, looseness=1, cut] (B),
                (B) -- [fermion, bend left=10, looseness=1] (F1),
                (B) -- [fermion, bend right=10, looseness=1] (F2),
                (II1) -- [fermion, bend left=10, looseness=1] (BB),
                (II2) -- [fermion, bend right=10, looseness=1] (BB),
                (BB) -- [fermion, bend left=45, looseness=1, cut] (bb),
                (BB) -- [fermion, bend right=45, looseness=1, cut] (bb),
                (bb) -- [fermion] (ii1),
                (bb) -- [fermion] (ii2),
            };
        \end{feynman}
    \end{tikzpicture}}.
\end{align*}