\section{Realizações no Período}

As realizações durante o período de Março até Dezembro estão organizadas na \cref{tab:realizacoes}.

\begin{table}[h!]
    \centering
    \caption{Resumo das Realizações do Período.}
    \label{tab:realizacoes}
    \begin{tabular}{@{}llp{7cm}@{}}
        \toprule
        \textbf{Data} & \textbf{Categoria} & \textbf{Descrição} \\
        \midrule
        Mar–Jul 2025 & Disciplinas Cursadas & Teoria de Cordas.\newline Tópicos Avançados em Relatividade Geral. \\[2mm]

        Mar-Jul 2025 & Revisão Bibliográfica & Estudo do formalismo \textit{on-shell} \autocite{arkani-hamedScatteringAmplitudesAll2021,elvangScatteringAmplitudesGauge2015}, bem como 
        dos métodos e teorias utilizadas em \autocite{herrmannUVCancelationsGravity2019,johanssonConformalGravityGauge2017,johanssonUnravelingConformalGravity2018}.\\[2mm]

        Jul 2025 & Participação em Evento & XV Escola do CBPF. \\[2mm]

        Ago-Dez 2025 & Cálculos & Cálculos preliminares com a teoria teste proposta em \autocite{johanssonUnravelingConformalGravity2018}. \\[2mm]

        Nov-Dez 2025 & Escrita & Redação de capítulos preliminares sobre métodos \textit{on-shell} e unitariedade generalizada. \\[2mm]

        Out 2025 & Participação em Evento & XLVIII CPLF.\newline Agorá II. \\[2mm]

        Dez 2025 & Participação em Evento & QCD Meets Gravity School.\newline QCD Meets Gravity Conference.\\
        \bottomrule
    \end{tabular}
\end{table}

Vale ressaltar que ambas as disciplinas cursadas foram concluídas com conceito A. Nos eventos XLVIII CPLF e QCD Meets Gravity School foram apresentados um pôster de 
título ``\textit{2+1 Dimensional Gravity as a Gauge Theory}'', enquanto no evento Agorá II foi apresentado um \textit{short talk} de título ``\textit{2+1 Gravity as a Gauge Theory}''.

O restante desta secção retrata os métodos e técnicas estudadas, tanto como os cálculos preliminares realizados e excertos da dissertação escritos.