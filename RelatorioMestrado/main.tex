\documentclass[12pt]{amsart}

\usepackage[brazilian]{babel}
\usepackage{csquotes}
\usepackage[style=numeric-comp, backend=bibtex]{biblatex}
\usepackage{amsmath}
\usepackage{amssymb}
% \usepackage{bbm}
\usepackage{graphics, setspace}
\usepackage{mathtools}
\usepackage[hidelinks]{hyperref}
\usepackage{physics}
\usepackage{enumitem}
\usepackage{slashed}
\usepackage{tensor}
\usepackage[lmargin=1cm,rmargin=1cm, tmargin =2cm,bmargin =2cm]{geometry}
\usepackage{tensor}
\usepackage[brazilian]{cleveref}
\usepackage{tikz}
\usetikzlibrary{positioning,decorations.text}
\usetikzlibrary{math}
\usepackage{tikz-feynman}
\usepackage{bbold}

\AtBeginDocument{\renewcommand*{\hbar}{{\mkern-1mu\mathchar'26\mkern-8mu\textnormal{h}}}}
\AtBeginDocument{\newcommand{\e}{\textnormal{e}}}
\AtBeginDocument{\newcommand{\im}{\textnormal{i}}}
\AtBeginDocument{\newcommand{\luz}{\textnormal{c}}}
\AtBeginDocument{\newcommand{\grav}{\textnormal{G}}}
\AtBeginDocument{\newcommand{\kb}{{\textnormal{k}_{\textnormal{B}}}}}
\newcommand{\Dd}[1]{\mathcal D #1}
\newcommand{\Det}[1]{\textup{Det} #1}
\newcommand{\sgn}[1]{\mbox{sgn}\qty(#1)}
\newcommand{\cqd}{\hfill$\blacksquare$}
\newcommand{\dbar}{\mbox{\dj}}
\newcommand{\HRule}[1]{\rule{\linewidth}{#1}}
\newcommand{\sumint}[1]{\ \ \mathclap{\displaystyle\int\limits_{#1}}\mathclap{\textstyle\sum}\ \ }

\numberwithin{equation}{section}

\newtheorem{teo}{Teorema}[section]
\newtheorem{defi}{Definição}[section]
\newtheorem{lem}{Lema}[section]
\newtheorem{hip}{Hipótese}[subsection]

\newcommand{\numberthis}{\addtocounter{equation}{1}\tag{\theequation}}

\newcommand{\placedots}[3]{%
  \foreach \a in #3 {
    \node[draw=none, inner sep=0pt] at ($ (#1) + (\a:#2) $) {\(\cdot\)};
  }%
}

\makeatletter
\newcommand{\placedotsbetween}[5]{%
  \begin{scope}
    % blob center
    \pgfpointanchor{#1}{center}%
    \pgfmathsetmacro{\bxpt}{\pgf@x/1pt}%
    \pgfmathsetmacro{\bypt}{\pgf@y/1pt}%

    % start vertex
    \pgfpointanchor{#2}{center}%
    \pgfmathsetmacro{\axpt}{\pgf@x/1pt}%
    \pgfmathsetmacro{\aypt}{\pgf@y/1pt}%

    % end vertex
    \pgfpointanchor{#3}{center}%
    \pgfmathsetmacro{\cxpt}{\pgf@x/1pt}%
    \pgfmathsetmacro{\cypt}{\pgf@y/1pt}%

    % vectors
    \pgfmathsetmacro{\dxA}{\axpt - \bxpt}%
    \pgfmathsetmacro{\dyA}{\aypt - \bypt}%
    \pgfmathsetmacro{\dxC}{\cxpt - \bxpt}%
    \pgfmathsetmacro{\dyC}{\cypt - \bypt}%

    % angles
    \pgfmathsetmacro{\angA}{atan2(\dyA,\dxA)}%
    \pgfmathsetmacro{\angC}{atan2(\dyC,\dxC)}%

    % shortest arc
    \pgfmathsetmacro{\rawdelta}{\angC - \angA}%
    \pgfmathsetmacro{\delta}{\rawdelta}%
    \ifdim\delta pt>180pt \pgfmathsetmacro{\delta}{\delta - 360}\fi
    \ifdim\delta pt<-180pt \pgfmathsetmacro{\delta}{\delta + 360}\fi

    % spacing
    \pgfmathsetmacro{\step}{\delta/(#5+1)}%

    % place dots
    \foreach \k in {1,...,#5} {
      \pgfmathsetmacro{\angk}{\angA + \k*\step}%
      \node[inner sep=0pt, draw=none] 
        at ($ (#1) + (\angk:#4) $) {\(\cdot\)};
    }
  \end{scope}
}
\makeatother

\pagestyle{plain}

%\AddToHook{cmd/section/before}{\clearpage}

\addbibresource{referencias.bib}

\newtheorem{teorema}{Teorema}[section]
\newtheorem{definicao}{Definição}[section]
\newtheorem{lema}{Lema}[section]
\newtheorem{hipotese}{Hipótese}[section]
\newtheorem{postulado}{Postulado}[section]

\tikzset{
  cut/.style={
    postaction={
      decorate,
      decoration={
        markings,
        mark= at position 0.5 with {
          % draw a short line perpendicular to the path
          \pgftransformrotate{\pgfdecoratedangle} % align to tangent
          \pgftransformrotate{90}                   % rotate to perpendicular
          \draw[thick, red, dashed, dash pattern=on 3pt off 1pt] (-0.2cm,0) -- (0.2cm,0);
        }
      }
    }
  }
}

\tikzset{
  cut2/.style={
    postaction={
      decorate,
      decoration={
        markings,
        mark= at position 0.5 with {
          % compute angle (normal to path = decoratedangle + 90)
          \pgfmathsetmacro{\ang}{\pgfdecoratedangle + 90}
          % convert half-length and gap to pt lengths for pgf math
          \pgfmathsetmacro{\L}{0.2}   % half-length
          % compute dx,dy for half-length
          \pgfmathsetmacro{\dx}{\L*cos(\ang)}
          \pgfmathsetmacro{\dy}{\L}
          % draw two short segments leaving a central gap
          \draw[thick, red, dashed, dash pattern=on 3pt off 1pt] (0 cm,-\dy cm) -- (0 cm,\dy cm);
        }
      }
    }
  }
}

\begin{document}

%\tableofcontents

%%%%%%%%%%%%%%%%%%%%%%%%%%%%%%%%%%%%%%%%%%%%%%%%%%%%%%%%%%%%%

\begin{titlepage}
    \begin{center}
        \vspace*{-1cm}{
            \setstretch{.5} 
            \textsc{Universidade de São Paulo} \\
            \HRule{.2pt}\\
            \textsc{Instituto de Física}
        }

        \vspace{5.5cm}
        
        \Large \textbf{\textsc{Amplitudes de espalhamento em teorias com derivadas de
            ordem superior}}
 	    \HRule{1.5pt} \\ [0.5cm]
        \linespread{1}
        \large Relatório de atividades anual de Mestrado. \\ 
   	    \HRule{1.5pt} \\ [0.5cm]       
        Projeto sob fomento da CAPES
        \\ [0.5cm]
        Pesquisador Responsável: Gabriel Santos Menezes \\
        Aluno: Vicente Viater Figueira
        \\ [0.5cm]
        Vigência: 01/03/2025 a 01/03/2027\\
        Período Coberto pelo Relatório: 01/03/2025 a 01/12/2025
        \vfill
        {\normalsize  São Paulo, \today}
    \end{center}
\end{titlepage}

\nocite{*}

\section{Resumo do projeto proposto}

Este plano de atividades se propõe a realizar um estudo de amplitudes de espalhamento na chamada
teoria $\qty(DF)^2$ e, utilizando-se do método da cópia dupla, estender esses resultados para o caso da super-
gravidade conforme do tipo Berkovits-Witten. Estes estudos também preveem uma maior compreensão
do método da unitariedade generalizada para o caso de partículas instáveis. Além disso, também nos
permitiria uma abordagem sistemática no estudo do comportamento a altas energias das amplitudes de
espalhamento em gravidade quadrática.

\section{Realizações no Período}

Disciplinas feitas no primeiro semestre: Teoria de Cordas, Tópicos Avançados em Relatividade Geral

Eventos participados: CBPF, CPLF e II Agorá Meeting

\subsection{Métodos On-Shell}

O principal ponto da abordagem relativamente moderna de métodos on-shell para o cálculo de amplitudes em teorias de campo 
é utilizar-se de uma informação subutilizada em Teoria Quântica de Campos (TQC) usual, transformações pelo \textit{Little-Group}. 
É de conhecimento geral que a álgebra de Poincaré admite dois invariantes de Casimir, a massa quadrada $-P^\mu P_\mu$ e o spin 
$W^\mu W_\mu$, para estados fisicamente aceitáveis é necessário $P^\mu P_\mu\leq 0$, desses dois casos
\subsubsection{teste}

\subsection{Unitariedade em TQC}

Unitariedade em TQC se refere a unitariedade da matrix $S$. Como revisão, a matrix $S$ é a amplitude de transição entre um estado \textit{in}, $\Psi^+_\alpha$, para um estado 
\textit{out}, $\Psi^-_\beta$, \[S_{\beta\alpha}=\qty(\Psi^-_\beta,\Psi^+_\alpha),\] aqui $\alpha$ e $\beta$ são índices que condensam toda a informação contida em seu respectivo 
estado do espaço de Hilbert. É assumido que tanto os estados \textit{in}, quanto os \textit{out}, sejam uma base completa do espaço de Hilbert, de forma que se a matrix $S$ 
é um mapa entre essas duas bases, é necessário ela ser um mapa unitário, e de fato, manipulando formalmente essa expressão, \[\int\dd{\beta}S^\ast_{\beta\gamma}S_{\beta\alpha}=\int\dd{\beta}\qty(\Psi^-_\beta,\Psi^+_\gamma)^\ast\qty(\Psi^-_\beta,\Psi^+_\alpha)=\int\dd{\beta}\qty(\Psi^+_\gamma,\Psi^-_\beta)\qty(\Psi^-_\beta,\Psi^+_\alpha)=\qty(\Psi^+_\gamma,\Psi^+_\alpha)=\delta_{\gamma\alpha}\]
Há fortes consequências dessa propriedades, a principal é chamada por motivos históricos de \textbf{Teorema Óptico}, primeiro, é necessário expandir, \[S_{\beta\alpha}=\delta_{\beta\alpha}+\im \qty(2\pi)^4\delta^{\qty(4)}\qty(p_\beta-p_\alpha)\mathcal A_{\beta\alpha}\] nessa forma, 
a condição de unitariedade implica,
\begin{align*}
    \delta_{\gamma\alpha}&=\int\dd{\beta}S^\ast_{\beta\gamma}S_{\beta\alpha}=\int\dd{\beta}\qty(\delta_{\beta\gamma}+\im \qty(2\pi)^4\delta^{\qty(4)}\qty(p_\beta-p_\gamma)\mathcal A_{\beta\gamma})^\ast\qty(\delta_{\beta\alpha}+\im \qty(2\pi)^4\delta^{\qty(4)}\qty(p_\beta-p_\alpha)\mathcal A_{\beta\alpha})\\
    \delta_{\gamma\alpha}&=\delta_{\gamma\alpha}-\im \qty(2\pi)^4\delta^{\qty(4)}\qty(p_\alpha-p_\gamma)\mathcal A^\ast_{\alpha\gamma}+\im \qty(2\pi)^4\delta^{\qty(4)}\qty(p_\gamma-p_\alpha)\mathcal A_{\gamma\alpha}+\qty(2\pi)^8\int\dd{\beta}\delta^{\qty(4)}\qty(p_\beta-p_\gamma)\delta^{\qty(4)}\qty(p_\beta-p_\alpha)\mathcal A^\ast_{\beta\gamma}\mathcal A_{\beta\alpha}\\
    0&=-\im \mathcal A^\ast_{\alpha\gamma}+\im \mathcal A_{\gamma\alpha}+\qty(2\pi)^4\int\dd{\beta}\delta^{\qty(4)}\qty(p_\beta-p_\alpha)\mathcal A^\ast_{\beta\gamma}\mathcal A_{\beta\alpha}
\end{align*}
A maior utilidade deste resultado é do ponto de vista de teoria de perturbação, certamente calculamos uma amplitude de espalhamento $\mathcal A_{\beta\alpha}$ em uma determinada 
ordem $\mathcal O(g^n)$ do parâmetro de acoplamento, porém, o resultado acima promove uma relação entre $\mathcal A$ e $\mathcal A^2$, ou seja, há relações entre amplitudes em 
diferentes ordens na expansão do parâmetro de acoplamento. A versão mais famosa deste resultado é para $\alpha=\gamma$, 
\begin{align*}
    \im \mathcal A^\ast_{\alpha\alpha}-\im \mathcal A_{\alpha\alpha}&=\qty(2\pi)^4\int\dd{\beta}\delta^{\qty(4)}\qty(p_\beta-p_\alpha)\mathcal A^\ast_{\beta\alpha}\mathcal A_{\beta\alpha}\\
    2\mathfrak{Im}\qty[\mathcal A_{\alpha\alpha}]&=\qty(2\pi)^4\int\dd{\beta}\delta^{\qty(4)}\qty(p_\beta-p_\alpha)\abs{\mathcal A_{\beta\alpha}}^2
\end{align*}
Trabalhando do ponto de vista de teoria de perturbação, podemos calcular a parte imaginária da contribuição de 1--\textit{loop} de $\mathcal A_{\alpha\alpha}$ apenas sabendo a contribuição 
de nível arvore para $\mathcal A_{\beta\alpha}$. Parte deste fato está relacionado ao teorema de Sokhotski–Plemelj, \[\frac{1}{p^2+m^2-\im\epsilon}=\im\pi\delta\qty(p^2+m^2)+\textnormal{P.V.}\frac{1}{p^2+m^2}.\] Que nos confirma que o propagador 
apenas possui parte imaginária para uma partícula \textit{on-shell}, porém, para diagramas a nível árvore não é cinematicamente permitido de uma partícula virtual 
interna ao diagrama entrar \textit{on-shell}, o que é compatível com o senso comum de contribuições à nível árvore serem polinômios de propagadores e numeradores cinemáticos, que certamente não 
possuem parte imaginária para partículas \textit{off-shell}. Agora, para contribuições de \textit{loop}, partículas virtuais internas podem ficarem \textit{on-shell}, e portanto, os diagramas 
podem possuírem parte imaginária.

Como exemplo tomemos a teoria $g\phi^3$,\[\mathcal L = -\frac12\phi\qty(-\Box+m^2)\phi+\frac1{3!}g\phi^3,\] A contribuição de 1--\textit{loop} para o processo $1\rightarrow 1$ é,
\begin{align*}
    \im\mathcal A^{\textnormal{1--loop}}_{1\rightarrow 1}&=\raisebox{+3pt}{\feynmandiagram [baseline = (b.base),horizontal=a to b,layered layout] {
        a -- [scalar,momentum =\(p\)] b -- [ half left,scalar, momentum=\(\ell\)] c --[half left,scalar, momentum=\(\ell-p\)] b,
        c -- [scalar,momentum =\(p\)] d,
    };}=\frac12 \qty(\im g)^2\frac{1}{\im^2}\int\frac{\dd[4]{\ell}}{\qty(2\pi)^4}\frac{1}{\ell^2+m^2-\im\epsilon}\frac{1}{\qty(\ell - p)^2+m^2-\im\epsilon}\\
    \mathcal A^{\textnormal{1--loop}}_{1\rightarrow 1}&=-\im\frac12g^2\int\frac{\dd[4]{\ell}}{\qty(2\pi)^4}\qty(\im\pi\delta\qty(\ell^2+m^2)+\textnormal{P.V.}\frac{1}{\ell^2+m^2})\qty(\im\pi\delta\qty(\qty(\ell-p)^2+m^2)+\textnormal{P.V.}\frac{1}{\qty(\ell-p)^2+m^2})\\
    \mathfrak{Im}\qty[\mathcal A^{\textnormal{1--loop}}_{1\rightarrow 1}]&=-\frac12g^2\int\frac{\dd[4]{\ell}}{\qty(2\pi)^4}\qty(-\pi^2\delta\qty(\ell^2+m^2)\delta\qty(\qty(\ell-p)^2+m^2)+\textnormal{P.V.}\frac{1}{\ell^2+m^2}\textnormal{P.V.}\frac{1}{\qty(\ell-p)^2+m^2})
\end{align*}
A parte dependente do valor principal resultará em zero, e portanto,
\begin{align*}
    \mathfrak{Im}\qty[\mathcal A^{\textnormal{1--loop}}_{1\rightarrow 1}]&=\frac12\pi^2g^2\int\frac{\dd[4]{\ell}}{\qty(2\pi)^4}\delta\qty(\ell^2+m^2)\delta\qty(\qty(\ell-p)^2+m^2)\\
    \mathfrak{Im}\qty[\mathcal A^{\textnormal{1--loop}}_{1\rightarrow 1}]&=\frac12\pi^2g^2\int\frac{\dd[4]{q}\dd[4]{\ell}}{\qty(2\pi)^4}\delta\qty(\ell^2+m^2)\delta\qty(q^2+m^2)\delta^{\qty(4)}\qty(q+\ell-p)\\
    \mathfrak{Im}\qty[\mathcal A^{\textnormal{1--loop}}_{1\rightarrow 1}]&=\frac12\pi^2g^2\int\frac{\dd[4]{q}\dd[4]{\ell}}{\qty(2\pi)^42\omega_{\boldsymbol \ell}2\omega_{\vb q}}\qty(\delta\qty(\ell^0-\omega_{\boldsymbol\ell})+\delta\qty(\ell^0+\omega_{\boldsymbol\ell}))\qty(\delta\qty(q^0-\omega_{\vb q})+\delta\qty(q^0+\omega_{\vb q}))\delta^{\qty(4)}\qty(q+\ell-p)\\
    \mathfrak{Im}\qty[\mathcal A^{\textnormal{1--loop}}_{1\rightarrow 1}]&=\frac12\pi^2g^2\int\frac{\dd[3]{\vb q}\dd[3]{\boldsymbol\ell}}{\qty(2\pi)^42\omega_{\boldsymbol \ell}2\omega_{\vb q}}\qty(\delta\qty(\omega_{\vb q}+\omega_{\boldsymbol\ell}-p^0)+\delta\qty(\omega_{\vb q}-\omega_{\boldsymbol\ell}-p^0)+\delta\qty(-\omega_{\vb q}+\omega_{\boldsymbol\ell}-p^0))\delta^{\qty(3)}\qty(\vb q+\boldsymbol\ell-\vb p)\\
    \mathfrak{Im}\qty[\mathcal A^{\textnormal{1--loop}}_{1\rightarrow 1}]&=\frac18\qty(2\pi)^4g^2\int\frac{\dd[3]{\vb q}\dd[3]{\boldsymbol\ell}}{\qty(2\pi)^62\omega_{\boldsymbol \ell}2\omega_{\vb q}}\delta^{\qty(4)}\qty(q+\ell- p)\\
    \mathfrak{Im}\qty[\mathcal A^{\textnormal{1--loop}}_{1\rightarrow 1}]&=\frac18\qty(2\pi)^4\int\frac{\dd[3]{\vb q}\dd[3]{\boldsymbol\ell}}{\qty(2\pi)^62\omega_{\boldsymbol \ell}2\omega_{\vb q}}\delta^{\qty(4)}\qty(q+\ell- p)\abs{\mathcal A^{\textnormal{tree}}_{1\rightarrow 2}}^2, \ \ \ \mathcal A^{\textnormal{tree}}_{1\rightarrow 2}=g
\end{align*}
Neste \textit{toy-model} podemos apreciar claramente a parte imaginária da amplitude $1\rightarrow 1$ a 1--\textit{loop} ser expressável em termos da amplitude nível árvore $1\rightarrow 2$. A integral que aparece, 
\[\int\dd{\beta}=\frac14\int\frac{\dd[3]{\boldsymbol\ell}\dd[3]{\vb q}}{\qty(2\pi)^62\omega_{\boldsymbol\ell}2\omega_{\vb q}},\] 
nada é além da medida Lorentz invariante do espaço de fase. Há uma maneira diagramática de obter essa igualdade entre a parte imaginária e produtos 
de amplitudes em menor ordem, elas vão pelo nome de \textbf{regras de corte de Cutkosky}, o procedimento é simples, escrevemos um diagrama de Feynman de n--\textit{loops} que 
contribua para o processo em análise, disto, \textit{cortamos} propagadores deste diagrama de forma a separar o diagrama inicial em dois diagramas de ordem menor. 
O procedimento de \textit{cortar} um propagador corresponde a substituir $\qty(p^2+m^2-\im\epsilon)^{-1}$ por $\im\pi\theta\qty(p^0)\delta\qty(p^2+m^2)$, diagramaticamente, 
representamos um propagador cortado por uma linha perpendicular passando por seu propagador, ao fim, multiplicamos as duas amplitudes restantes, com a da direita sendo 
conjugada, ao fim, integramos sobre o espaço de fase Lorentz invariante. Note que neste processo obtemos duas amplitudes \textit{on-shell}. Como exemplo,
\begin{align*}
    \mathfrak{Im}\qty[\mathcal A^{\textnormal{1--loop}}_{1\rightarrow 1}]&=\frac12
    \raisebox{+3pt}{\begin{tikzpicture}[baseline=(a.base)]
        \begin{feynman}\diagram* {
        a -- [scalar,momentum =\(p\)] b -- [ half left,scalar,momentum=\(\ell\), cut] c-- [scalar,momentum =\(p\)] d,c --[half left,scalar,reversed momentum=\(q\), cut] b
        };
        \end{feynman}
    \end{tikzpicture}}=\frac12\qty(2\pi)^4\int\dd{\beta}\delta^{\qty(4)}\qty(q+\ell-p)\mathcal A^{\textnormal{tree}}_{1\rightarrow 2}\qty(p;\ell,q)\mathcal A^{\textnormal{tree}\ast}_{1\rightarrow 2}\qty(p;\ell,q)
\end{align*}
Claramente, para esse exemplo simples, há apenas uma única maneira de se \textit{cortar} o diagrama de 1--\textit{loop} em 
duas partes de menor ordem. Porém, para um número de pernas externas maior, ou maior número de \textit{loops}, é necessário somar 
sobre todas as maneiras de se separar as amplitudes. Este conceito de unitariedade possui algumas limitações, 
primeiro, é possível apenas determinar a parte imaginária das amplitudes via amplitudes de ordem inferior, 
segundo, somente conseguimos aplicar este resultado para um espalhamento da forma $\alpha\rightarrow \alpha$, que 
é longe de ser a forma de espalhamento mais geral. Contudo, é possível obter um resultado mais geral, para isto, temos 
que relembrar a definição de estados \textit{in/out}. Dado um hamiltoniano $H=H_0+V$, e sendo $\Phi_\alpha$ autoestado de $H_0$ com autovalor $E_\alpha$, definimos 
$\Psi^\pm_\alpha$ autoestado de $H$ com autovalor $E_\alpha$ por \[\Psi^\pm_\alpha=\Phi_\alpha+\qty(E_\alpha-H_0\pm\im\epsilon)^{-1}V\Psi^\pm_\alpha,\ \ \ \epsilon>0.\]

Note a imposição $\epsilon>0$, e portanto, a troca $\epsilon\leftrightarrow-\epsilon$ corresponde a: $\Psi^-_\alpha\leftrightarrow\Psi^+_\alpha$. 
Assim, podemos retornar a expressão,
\[S_{\alpha\gamma}^\ast=\qty(\Psi^-_\alpha,\Psi^+_\gamma)^\ast = \qty(\Psi^+_\gamma,\Psi^-_\alpha)= \qty(\Psi^-_\gamma,\Psi^+_\alpha)\eval_{\epsilon<0}=S_{\gamma\alpha}\eval_{\epsilon<0},\]
portanto,\[\mathcal A_{\alpha\gamma}^\ast=\mathcal A_{\gamma\alpha}\eval_{\epsilon<0},\]
utilizando esse resultado podemos concluir,
\begin{align*}
    -\im\qty(\mathcal A_{\gamma\alpha}-\mathcal A^\ast_{\alpha\gamma})&=\int\dd{\beta}\qty(2\pi)^4\delta^{\qty(4)}\qty(p_\beta-p_\alpha)\mathcal A^\ast_{\beta\gamma}\mathcal A_{\beta\alpha}\\
    -\im\qty(\mathcal A_{\gamma\alpha}\eval_{\epsilon>0}-\mathcal A_{\gamma\alpha}\eval_{\epsilon<0})&=\int\dd{\beta}\qty(2\pi)^4\delta^{\qty(4)}\qty(p_\beta-p_\alpha)\mathcal A^\ast_{\beta\gamma}\mathcal A_{\beta\alpha}
\end{align*}
O lado esquerdo desta igualdade deve ser entendido como sendo o limite $\epsilon\rightarrow 0$, claramente, se $\mathcal A_{\gamma\alpha}$ fosse 
uma função contínua em $\epsilon$, o resultado seria zero, e como o lado direito da igualdade é não necessariamente zero, podemos apenas concluir que: 
Em geral $\mathcal A_{\gamma\alpha}$ é descontínuo em $\epsilon$, porém, como $\epsilon$ contribui para a amplitude somente dentro do propagador 
$-\im\qty(p^2+m^2-\im\epsilon)^{-1}$, e é acompanhado por um fator de $\im$, concluímos que genericamente as amplitudes, vistas como 
funções dos invariantes cinemáticos, possuem um \textit{branch cut} num subconjunto do eixo real, quando interpretamos os momentos podendo tomar 
valores em números complexos. 

Seja uma função $f:\mathbb C\rightarrow \mathbb C$ analítica em todo plano, exceto por possíveis polos e um \textit{branch cut} no eixo real, naturalmente 
isso significa que, 
\begin{align*}
    \exists s\in\mathbb R,\ \ 0&\neq\lim\limits_{\epsilon\rightarrow 0^+}\qty[f\qty(s-\im\epsilon)-f\qty(s+\im\epsilon)]= \mathfrak{Disc}[f],
\end{align*}
no qual já definimos o que chamamos de descontinuidade de uma função. Assim,
\begin{align*}
    -\im \mathfrak{Disc}\qty[\mathcal A_{\gamma\alpha}]&=\int\dd{\beta}\qty(2\pi)^4\delta^{\qty(4)}\qty(p_\beta-p_\alpha)\mathcal A^\ast_{\beta\gamma}\mathcal A_{\beta\alpha}.
\end{align*}

Agora, se supormos que $\mathcal A_{\gamma\alpha}$ é uma amplitude de nível árvore, claramente as sub-amplitudes 
$\mathcal A_{\beta\gamma},\mathcal A_{\beta\alpha}$ devem ser também de nível árvore, do contrário, seriam contribuições de ordem superior 
para $\mathcal A_{\gamma\alpha}$. Assim, necessariamente $\Phi_\beta$ é um estado de uma única partícula, portanto, $\dd{\beta}$ é a 
medida invariante de Lorentz do espaço de fase de uma única partícula,
\begin{align*}
    -\im \mathfrak{Disc}\qty[\mathcal A^{\textnormal{Tree}}_{\gamma\alpha}]&=\int\dd{\beta}\qty(2\pi)^4\delta^{\qty(4)}\qty(p_\beta- p_\alpha)\mathcal A^\ast_{\beta\gamma}\mathcal A_{\beta\alpha}\\
    -\im \mathfrak{Disc}\qty[\mathcal A^{\textnormal{Tree}}_{\gamma\alpha}]&=\sum\limits_\beta\int\frac{\dd[3]{\vb q}}{\qty(2\pi)^32\omega_{\vb q}}\qty(2\pi)^4\delta^{\qty(4)}\qty(q- p_\alpha)\mathcal A^\ast_{\beta\gamma}\mathcal A_{\beta\alpha}\\
    -\im \mathfrak{Disc}\qty[\mathcal A^{\textnormal{Tree}}_{\gamma\alpha}]&=\sum\limits_\beta2\pi\frac{1}{2\omega_{\vb q}}\delta\qty(q^0- p^0_\alpha)\mathcal A^\ast_{\beta\gamma}\mathcal A_{\beta\alpha}\eval_{\vb q = \vb p_\alpha=\vb p_\gamma}\\
    -\im \mathfrak{Disc}\qty[\mathcal A^{\textnormal{Tree}}_{\gamma\alpha}]&=\sum\limits_\beta2\pi\theta\qty(p_\alpha^0)\delta\qty(\qty(q^0)^2-\qty(p_\alpha^0)^2)\mathcal A^\ast_{\beta\gamma}\mathcal A_{\beta\alpha}\eval_{\vb q = \vb p_\alpha=\vb p_\gamma}\\
    \mathfrak{Disc}\qty[\mathcal A^{\textnormal{Tree}}_{\gamma\alpha}]&=\sum\limits_\beta2\pi\im\theta\qty(p_\alpha^0)\delta\qty(p_\alpha^2+m_\beta^2)\mathcal A^\ast_{\beta\gamma}\mathcal A_{\beta\alpha}.
\end{align*}

Note que $p_\alpha^2+m_\beta^2 = 0$ é impossível de ser satisfeito para diagramas nível árvore, pois, isso força $m_\beta^2> \sum\limits_\alpha m_\alpha^2$, 
e caso exista tal estado em nossa teoria, este não é estável, logo, não é possível de estar em um espalhamento de estados assintóticos. 
Claramente, essa expressão somente toma total sentido se estamos dispostos a interpretar $\mathcal A_{\gamma\alpha}$ como função dos momentos externos, 
e admitirmos estes a poderem tomar valores complexos. Nesse ponto de vista, olhemos para as amplitudes como funções de cada momento das partículas iniciais e 
finais: $p_{\alpha_i},p_{\gamma_j}$, e estudemos sua extensão no plano complexo definida por,
\begin{align*}
    \mathcal A_{\gamma \alpha} = \mathcal A_{\gamma \alpha}\qty(\qty{p_\alpha},\qty{p_\gamma})\rightarrow\tilde{\mathcal A}_{\gamma\alpha}\qty(\qty{p_\alpha+k_\alpha z},\qty{p_\gamma+k_\gamma z})=\tilde{\mathcal A}_{\gamma \alpha}\qty(z),
\end{align*}
no qual $z\in\mathbb C$ e $k_{\alpha_i},k_{\gamma_j}$ são momentos arbitrários satisfazendo:\[k_\alpha=k_\gamma,\ \ \ k_{\alpha_i}\cdot k_{\alpha_j}=k_{\alpha_i}\cdot k_{\gamma_j}=k_{\gamma_i}\cdot k_{\gamma_j}=0,\ \ \ p_{\alpha_i}\cdot k_{\alpha_i}=p_{\gamma_i}\cdot k_{\gamma_i}=0.\]

Desta forma temos que $\tilde{\mathcal A}_{\gamma \alpha}\qty(0)=\mathcal A_{\gamma \alpha}$, olhemos para expressão anterior quando sujeita a esta extensão complexa,
\begin{align*}
    \mathfrak{Disc}\qty[\tilde{\mathcal A}_{\gamma\alpha}]&=\sum\limits_\beta2\pi\im\theta\qty({\tilde{p}}_\alpha^0)\delta\qty({\tilde{p}}_\alpha^2+m_\beta^2)\tilde{\mathcal A}^\ast_{\beta\gamma}\tilde{\mathcal A}_{\beta\alpha}\\
    \mathfrak{Disc}\qty[\tilde{\mathcal A}_{\gamma\alpha}]&=\sum\limits_\beta2\pi\im\theta\qty(p_\alpha^0+k_\alpha^0 z)\delta\qty(p_\alpha^2+m_\beta^2+2p_\alpha\cdot k_\alpha z)\tilde{\mathcal A}^\ast_{\beta\gamma}\tilde{\mathcal A}_{\beta\alpha}\\
    \mathfrak{Disc}\qty[\tilde{\mathcal A}_{\gamma\alpha}]&=\sum\limits_\beta\frac{2\pi\im}{2\abs{p_\alpha\cdot k_\alpha}}\theta\qty(p_\alpha^0+k_\alpha^0 z)\delta\qty(z+\frac{p_\alpha^2+m_\beta^2}{2p_\alpha\cdot k_\alpha})\tilde{\mathcal A}^\ast_{\beta\gamma}\tilde{\mathcal A}_{\beta\alpha},\ \ \ z_\beta = -\frac{p_\alpha^2+m_\beta^2}{2p_\alpha\cdot k_\alpha}\\
    \mathfrak{Disc}\qty[\tilde{\mathcal A}_{\gamma\alpha}]&=\sum\limits_\beta \abs{z_\beta}\frac{2\pi\im}{\abs{p_\alpha^2+m_\beta^2}}\theta\qty(p_\alpha^0+k_\alpha^0 z_\beta)\delta\qty(z-z_\beta)\tilde{\mathcal A}^\ast_{\beta\gamma}\tilde{\mathcal A}_{\beta\alpha}\\
    \frac1z\mathfrak{Disc}\qty[\tilde{\mathcal A}_{\gamma\alpha}]&=\sum\limits_\beta \frac{\abs{z_\beta}}{z_\beta}\frac{2\pi\im}{\abs{p_\alpha^2+m_\beta^2}}\theta\qty(p_\alpha^0+k_\alpha^0 z_\beta)\delta\qty(z-z_\beta)\tilde{\mathcal A}^\ast_{\beta\gamma}\tilde{\mathcal A}_{\beta\alpha}.
\end{align*}

Sabemos que as amplitudes $\mathcal A_{\beta\gamma},\mathcal A_{\beta\alpha}$ são de nível árvore, portanto, pelo argumento anterior, não possuem singularidades em $z=z_\beta$, 
assim, podemos integrar a expressão anterior em um intervalo simplesmente conexo fechado qualquer que contenha $z_\beta$ e não passe por $z=0$. Chamemos este intervalo de $I_\beta$,
\begin{align*}
    \int\limits_{I_\beta}\dd{z}\frac1z\mathfrak{Disc}\qty[\tilde{\mathcal A}_{\gamma\alpha}]&=\int\limits_{I_\beta}\dd{z}\sum\limits_\beta \frac{\abs{z_\beta}}{z_\beta}\frac{2\pi\im}{\abs{p_\alpha^2+m_\beta^2}}\theta\qty(p_\alpha^0+k_\alpha^0 z_\beta)\delta\qty(z-z_\beta)\tilde{\mathcal A}^\ast_{\beta\gamma}\tilde{\mathcal A}_{\beta\alpha}\\
    \int\limits_{I_\beta}\dd{z}\frac1z\mathfrak{Disc}\qty[\tilde{\mathcal A}_{\gamma\alpha}]&=\sum\limits_\beta \frac{\abs{z_\beta}}{z_\beta}\frac{2\pi\im}{\abs{p_\alpha^2+m_\beta^2}}\theta\qty(p_\alpha^0+k_\alpha^0 z_\beta)\tilde{\mathcal A}^\ast_{\beta\gamma}\tilde{\mathcal A}_{\beta\alpha}\eval_{z=z_\beta},
\end{align*}
agora, utilizando a definição de descontinuidade, e sabendo que a descontinuidade de $z$ é zero neste domínio, podemos deformar o contorno $I_\beta$ para duas versões, uma passando por baixo da reta real, e outra 
por cima da reta real com orientação oposta, e devido à descontinuidade é possível unir estes contornos formando uma curva fechada no sentido anti-horário circundando o ponto $z_\beta$, 
chamemos este contorno de $C_\beta$,
\begin{align*}
    \oint\limits_{C_\beta}\dd{z}\frac1z\tilde{\mathcal A}_{\gamma\alpha}&=\sum\limits_\beta \frac{\abs{z_\beta}}{z_\beta}\frac{2\pi\im}{\abs{p_\alpha^2+m_\beta^2}}\theta\qty(p_\alpha^0+k_\alpha^0 z_\beta)\tilde{\mathcal A}^\ast_{\beta\gamma}\tilde{\mathcal A}_{\beta\alpha}\eval_{z=z_\beta}.
\end{align*}
Podemos então agora deformar o contorno $C_\beta$ em outros dois, um circundando o ponto $z=0$ no sentido horário, e outro no sentido anti-horário circundando $z=\infty$, assim,
\begin{align*}
    -\frac{1}{2\pi\im}\oint\limits_{C_0}\dd{z}\frac1z\tilde{\mathcal A}_{\gamma\alpha}-\frac{1}{2\pi\im}\oint\limits_{C_\infty}\dd{z}\frac1z\tilde{\mathcal A}_{\gamma\alpha}&=\sum\limits_\beta \frac{\abs{z_\beta}}{z_\beta}\frac{1}{\abs{p_\alpha^2+m_\beta^2}}\theta\qty(p_\alpha^0+k_\alpha^0 z_\beta)\tilde{\mathcal A}^\ast_{\beta\gamma}\tilde{\mathcal A}_{\beta\alpha}\eval_{z=z_\beta}\\
    -\mathfrak{Res}_{z=0}\qty[\frac1z\tilde{\mathcal A}_{\gamma\alpha}]-\mathfrak{Res}_{z=\infty}\qty[\frac1z\tilde{\mathcal A}_{\gamma\alpha}]&=\sum\limits_\beta \frac{\abs{z_\beta}}{z_\beta}\frac{1}{\abs{p_\alpha^2+m_\beta^2}}\theta\qty(p_\alpha^0+k_\alpha^0 z_\beta)\tilde{\mathcal A}^\ast_{\beta\gamma}\tilde{\mathcal A}_{\beta\alpha}\eval_{z=z_\beta}\\
    \mathcal A_{\gamma\alpha}&=-\sum\limits_\beta \frac{\abs{z_\beta}}{z_\beta}\frac{1}{\abs{p_\alpha^2+m_\beta^2}}\theta\qty(p_\alpha^0+k_\alpha^0 z_\beta)\tilde{\mathcal A}^\ast_{\beta\gamma}\tilde{\mathcal A}_{\beta\alpha}\eval_{z=z_\beta}-\mathfrak{Res}_{z=\infty}\qty[\frac1z\tilde{\mathcal A}_{\gamma\alpha}]\\
    \mathcal A_{\gamma\alpha}&=-\sum\limits_\beta \frac{\abs{z_\beta}}{z_\beta}\frac{1}{\abs{p_\alpha^2+m_\beta^2}}\theta\qty(p_\alpha^0+k_\alpha^0 z_\beta)\tilde{\mathcal A}^\ast_{\beta\gamma}\tilde{\mathcal A}_{\beta\alpha}\eval_{z=z_\beta}+\lim\limits_{\abs{z}\rightarrow \infty}\tilde{\mathcal A}_{\gamma\alpha},
\end{align*}
finalmente, sempre podemos tomar $z_\beta<0$, pois, $m_\beta^2>-p_\alpha^2$ e o sinal de $k_\alpha$ é fixado por se fazer $\theta\qty(p_\alpha^0+k_\alpha^0z_\beta)$, assim,
\begin{align*}
    \mathcal A_{\gamma\alpha}&=\sum\limits_\beta\frac{1}{p_\alpha^2+m_\beta^2}\tilde{\mathcal A}^\ast_{\beta\gamma}\tilde{\mathcal A}_{\beta\alpha}\eval_{z=z_\beta}+\lim\limits_{\abs{z}\rightarrow \infty}\tilde{\mathcal A}_{\gamma\alpha}\numberthis\label{residuotreelevelsmatrix}.
\end{align*}
Esta forma da condição de unitariedade é também conhecida como \textbf{relações de recursão à nível árvore}. Escolhas específicas de $k_\alpha,k_\gamma$ 
caracterizam diferentes métodos de recursão, os principais métodos são BCFW --- Britto, Cachazo, Feng e Witten --- e CSW --- Cachazo, Svrcek e Witten ---. 
Esses métodos de recursão, unidos com a simplicidade do formalismo de amplitudes \textit{on-shell}, providenciam uma abordagem robusta e --- comparativamente ao método usual --- simples para 
se obter amplitudes de $n\geq4$ pontos à nível árvore. Claramente, essas relações somente facilitam a obtenção de amplitudes se o requisito \[\lim\limits_{\abs{z}\rightarrow \infty}\tilde{\mathcal A}_{\gamma\alpha}=0,\] for satisfeito. Esta 
condição é altamente não trivial e não é válida para várias teorias, porém, para teorias do tipo Yang-Mills foi-se provado que é de fato verdadeira.

\subsection{Unitariedade generalizada}

No item anterior, demos uma breve descrição de como a unitariedade da matrix S --- que nos permite calcular 
a descontinuidade de uma amplitude partindo de amplitudes de menor ordem ---, suplementada por hipóteses 
adicionais da estrutura analítica, nos permite de obter a amplitude completa, à nível árvore, partindo de amplitudes de menor 
ordem. Vamos agora descrever como esse método de unitariedade pode ser estendido para \textbf{unitariedade 
generalizada}, que nos permite obter amplitudes de \textit{loop} partindo de amplitudes de menor ordem. 
Claro que isso só é possível suplementando com condições da estrutura analítica.

Para isso, começamos partindo de uma amplitude $\mathcal A^L_{\gamma\alpha}$ de $L$--\textit{loops}. Sabemos da estrutura das regras de Feynman 
que tal amplitude pode ser escrita como,
\[\mathcal A^L_{\gamma\alpha}=\sum\limits_a\int\prod\limits_{n=0}^L\frac{\dd[4]{\ell_n}}{\qty(2\pi)^4}\mathcal I^{L,a}_{\gamma\alpha},\]
no qual $a$ indexa diferentes topologias que contribuem para este processo, e chamamos o objeto $\mathcal I^{L,a}_{\gamma\alpha}$ de integrando 
da amplitude. Ainda mais, o integrando possui forma bem definida, provindas das regras de Feynman,
\[\mathcal I^{L,a}_{\gamma\alpha}=\frac{1}{S_a}\frac{n_ac_a}{\prod\limits_{\beta_a}\qty(p_{\beta_a}^2+m^2_\beta)},\]
$n_a$ engloba toda a dependência cinemática do numerador, $c_a$ são fatores puramente numéricos dependentes dos acoplamentos e 
dos grupos de calibre, e $S_a$ são apenas fatores de simetria. O caso ideal seria se a soma sobre topologias comutasse com a integral dos 
momentos de \textit{loop}, e pudéssemos definir o seguinte objeto,
\[\mathcal A^L_{\gamma\alpha}=\sum\limits_a\int\prod\limits_{n=0}^L\frac{\dd[4]{\ell_n}}{\qty(2\pi)^4}\mathcal I^{L,a}_{\gamma\alpha}\stackrel{\text{?}}{=}\int\prod\limits_{n=0}^L\frac{\dd[4]{\ell_n}}{\qty(2\pi)^4}\sum\limits_a\mathcal I^{L,a}_{\gamma\alpha}\stackrel{\text{?}}{=}\int\prod\limits_{n=0}^L\frac{\dd[4]{\ell_n}}{\qty(2\pi)^4}\mathcal I^{L}_{\gamma\alpha},\]
a obstrução para se fazer isso é que não existe uma maneira de se definir globalmente como os momentos de loop são definidos no objeto $\mathcal I^{L,a}_{\gamma\alpha}$. 
Existem teorias em que isso sim é possível, porém, no caso geral não é verdade. Assim como mostramos no conceito de unitariedade da matrix S, 
a amplitude completa á nível árvore pode ser reconstruída em sua totalidade sabendo apenas o valor de seu resíduo em,
\[\qty(p_\alpha^2+m_\beta^2)\mathcal A_{\gamma\alpha}\eval_{p_\alpha^2=-m_\beta^2}=\mathfrak{Res}\qty[\mathcal A_{\gamma\alpha}]=\tilde{\mathcal A}^\ast_{\beta\gamma}\tilde{\mathcal A}_{\beta\alpha}\eval_{p_\alpha^2=-m_\beta^2},\]
com o mesmo espírito, propomos que o integrando de amplitudes de \textit{loops} possa ser reconstruído fazendo 
a igualdade de seus resíduos em \textbf{cortes} com o produto de amplitudes de menor ordem. Por exemplo, a unitariedade por sí é capaz de nos garantir, 
\[\qty(\ell_i^2+m_{\beta_1}^2)\qty(\qty(p_\alpha-\ell_i)^2+m^2_{\beta_2})\mathcal I^{L}_{\gamma\alpha}\eval_{\ell_i^2=-m_{\beta_1}^2,\qty(p_\alpha-\ell_i)^2=-m^2_{\beta_2}}=\sum\limits_{n=0}^L\sum\limits_{\beta}\mathcal I^{L-n\ast}_{\beta\gamma}\mathcal I^{n}_{\beta\alpha}\eval_{\ell_i^2=-m_{\beta_1}^2,\qty(p_\alpha-\ell_i)^2=-m^2_{\beta_2}},\]
contudo, ela somente nos é capaz de nos dizer algo sobre \textbf{cortes} verticais do diagrama, isto é, \textbf{cortes} 
da forma, 
\begin{align*}
    \mathfrak{Res}\qty[\raisebox{+3pt}{\begin{tikzpicture}[baseline=(b.base)]
        \begin{feynman}
            \vertex (i1) at (-1,1);
            \vertex (i2) at (-1,-1);
            \vertex [blob, minimum size=1cm] (b) at (0,0) {};
            \placedotsbetween{b}{i2}{i1}{0.7cm}{4}
            \vertex (f1) at (1,1);
            \vertex (f2) at (1,-1);
            \placedotsbetween{b}{f1}{f2}{0.7cm}{4}
            \diagram*{
                (i1) -- [fermion] (b) [blob],
                (i2) -- [fermion] (b),
                (b) -- [fermion] (f1),
                (b) -- [fermion] (f2),
            };
        \end{feynman}
    \end{tikzpicture}}]&=\sum\raisebox{+3pt}{\begin{tikzpicture}[baseline=(b.base)]
        \begin{feynman}
            \vertex (i1) at (-1,1);
            \vertex (i2) at (-1,-1);
            \vertex [blob, minimum size=1cm] (b) at (0,0) {};
            \placedotsbetween{b}{i1}{i2}{0.7cm}{4};
            \vertex [blob, minimum size=1cm] (B) at (2,0) {};
            \vertex (d1) at (1,0) {$\raisebox{6pt}{\vdots}$};
            \vertex (F1) at (2.8,0.6);
            \vertex (F2) at (2.8,-0.6);
            \placedotsbetween{B}{F1}{F2}{0.7cm}{4};
            \vertex (mid1) at (3.3,-0.6) {$\cdots$};
            \vertex (mid2) at (3.3,0) {$\cdots$};
            \vertex (mid3) at (3.3,0.6) {$\cdots$};
            \vertex (II1) at (3.8,0.6);
            \vertex (II2) at (3.8,-0.6);
            % \vertex (DD2) at (3.9,0) {$\raisebox{6pt}{\vdots}$};
            \vertex [blob, minimum size=1cm] (BB) at (4.6,0) {};
            \placedotsbetween{BB}{II1}{II2}{0.7cm}{4};
            \vertex (dd1) at (5.6,0) {$\raisebox{6pt}{\vdots}$};
            \vertex [blob, minimum size=1cm] (bb) at (6.6,0) {};
            \vertex (ii1) at (7.6,1);
            \vertex (ii2) at (7.6,-1);
            \placedotsbetween{bb}{ii1}{ii2}{0.7cm}{4};
            \diagram*{
                (i1) -- [fermion] (b),
                (i2) -- [fermion] (b),
                (b) -- [fermion, bend left=45, looseness=1, cut2] (B),
                (b) -- [fermion, bend right=45, looseness=1, cut2] (B),
                (B) -- [fermion, bend left=10, looseness=1] (F1),
                (B) -- [fermion, bend right=10, looseness=1] (F2),
                (II1) -- [fermion, bend left=10, looseness=1] (BB),
                (II2) -- [fermion, bend right=10, looseness=1] (BB),
                (BB) -- [fermion, bend left=45, looseness=1, cut] (bb),
                (BB) -- [fermion, bend right=45, looseness=1, cut] (bb),
                (bb) -- [fermion] (ii1),
                (bb) -- [fermion] (ii2),
            };
        \end{feynman}
    \end{tikzpicture}}
\end{align*}
Apesar desta topologia de resíduos nos proporcionar muita informação sobre a amplitude, para $L\geq 1$, essa 
informação apenas não é suficiente para conseguirmos reconstruir a amplitude, pois, há outros resíduos possíveis 
de serem tomados que não são levados em conta na aplicação da unitariedade convencional. Um exemplo é, 
\begin{align*}
    \mathfrak{Res}\qty[\raisebox{+3pt}{\begin{tikzpicture}[baseline=(b.base)]
        \begin{feynman}
            \vertex (i1) at (-1,1);
            \vertex (i2) at (-1,-1);
            \vertex [blob, minimum size=1cm] (b) at (0,0) {};
            \placedotsbetween{b}{i1}{i2}{0.7cm}{4}
            \vertex (f1) at (1,1);
            \vertex (f2) at (1,-1);
            \placedotsbetween{b}{f1}{f2}{0.7cm}{4};
            \diagram*{
                (i1) -- [fermion] (b) [blob],
                (i2) -- [fermion] (b),
                (b) -- [fermion] (f1),
                (b) -- [fermion] (f2),
            };
        \end{feynman}
    \end{tikzpicture}}]&=\sum\raisebox{+3pt}{\begin{tikzpicture}[baseline=(b.base)]
        \begin{feynman}
            \vertex (i1) at (-1,1);
            \vertex (i2) at (-1,-1);
            \vertex [blob, minimum size=1cm] (b) at (0,0) {};
            \placedotsbetween{b}{i1}{i2}{0.7cm}{4};
            % \vertex (d2) at (1.04,0) {$\raisebox{0pt}{\rotatebox[origin=c]{90}{\therefore}}$};
            \vertex [blob, minimum size=1cm] (B1) at (1.7,1) {};
            \vertex [blob, minimum size=1cm] (B2) at (1.7,-1) {};
            \vertex (f1) at (2.7,1);
            \vertex (f2) at (2.7,-1);
            \placedotsbetween{B1}{B2}{f1}{0.7cm}{4};
            \placedotsbetween{B2}{f2}{B1}{0.7cm}{4};
            \placedotsbetween{b}{B1}{B2}{0.7cm}{3};
            \placedotsbetween{B1}{b}{B2}{0.7cm}{3};
            \placedotsbetween{B2}{b}{B1}{0.7cm}{3};
            \diagram*{
                (i1) -- [fermion] (b),
                (i2) -- [fermion] (b),
                (b) -- [fermion, cut2] (B1),
                (B2) -- [fermion, cut2] (b),
                (B1) -- [fermion, cut2] (B2),
                (B1) -- [fermion] (f1),
                (B2) -- [fermion] (f2),
            };
        \end{feynman}
    \end{tikzpicture}}
\end{align*}
que corresponderia à,
\[\qty(\ell^2+m_{\beta_3}^2)\qty(\qty(p_{\gamma_1}-\ell)^2+m^2_{\beta_1})\qty(\qty(p_{\gamma_2}-\ell)^2+m^2_{\beta_2})\mathcal I^{L}_{\gamma\alpha}\eval_{\textnormal{Cuts}}\stackrel{?}{=}\sum\limits_{n=0}^L\sum\limits_{m=0}^{L-n}\sum\limits_{\beta_1,\beta_2,\beta_3}\mathcal I^{L-n-m\ast}_{\qty{\beta_2,\beta_3}\gamma_2}\mathcal I^{m\ast}_{\qty{\beta_1,\beta_2}\gamma_1}\mathcal I^{n}_{\qty{\beta_1,\beta_2}\alpha}\eval_{\textnormal{Cuts},\gamma=\gamma_1\cup\gamma_2}\]
o grande problema com esse tipo de corte é: nos assumimos que os estados iniciais e finais em integrandos $\mathcal I_{\beta\alpha}^L$ são tais que 
satisfazem $p_\alpha^0,p_\beta^0>0$, porém, neste caso é fácil de se mostrar que não é possível de se satisfazer essa condição para a solução dos 
cortes, assim, as amplitudes que aparecem nos resíduos dos cortes não possuem de fato uma relação rígida de partículas iniciais e finais, por conta do 
sinal da componente temporal do momento não ser positivo --- podendo ser complexo ---. Então, se estamos interessados em obter o máximo de 
informação sobre amplitudes de ordens superiores partindo de ordens inferiores, é necessário obtermos um formalismo que trate partículas externas 
de maneira uniforme, isto é, um formalismo que seja indiferente de $p^0>0$ e trate igualmente partículas internas e externas. 

Para isso, ao invés de utilizarmos como amplitude $\mathcal A_{\beta \alpha}$. que diferencia os estados iniciais $\alpha$ dos finais $\beta$, vamos fazer $p_{\alpha_i}\rightarrow -p_{\alpha_i}$, 
de modo que a conservação de momento passa a ser $\sum\limits_i^np_i=0$. A amplitude $\mathcal A_{\beta \alpha}$ vista como função dos momentos $p_{\beta_i},-p_{\alpha_j}$ para $n$ partículas 
externas será denominada $\mathcal A_n\qty(1^{h_1},\cdots,n^{h_n})$, no qual $i$ refere-se ao momento da $i$--ésima partícula e $h_i$ à helicidade/spin da $i$--ésima partícula. Nota-se a 
uniformidade do tratamento de partículas iniciais e finais, a única diferença é que partículas com $p^0_i>0$ devem serem interpretadas como finais e partículas com $p^0_i<0$ como iniciais. De certa 
forma, podemos interpretar essa nova função das variáveis cinemáticas como sendo uma abreviação de todas as amplitudes possíveis de $n$ pontos, isto é,
\[\mathcal A_n\qty(1^{h_1},\cdots, n^{h_n})=\sum\limits_{\alpha,\beta}\prod\limits_{i\in\alpha,j\in\beta}\theta\qty(-p^0_{\alpha_i}>0)\theta\qty(p^0_{\beta_j}>0)\mathcal A_{\beta\alpha},\]
devido às funções de Heaviside presentes, produtos dessas amplitudes fatorizam, e portanto, podemos utilizar do resultado da \eqref{residuotreelevelsmatrix} e obter uma expressão ainda mais geral,
\begin{align*}
    \mathcal A_n&=\sum\limits_{I,m}\frac{1}{P_I^2+m_\beta^2}\tilde{\mathcal A}_{n-m+2}\tilde{\mathcal A}_{m}\eval_{z=z_\beta}+\lim\limits_{\abs{z}\rightarrow \infty}\tilde{\mathcal A}_{n}\\
    \qty(P_I^2+m_I^2)\mathcal A_n\eval_{P_I^2=-m_\beta^2}&=\mathfrak{Res}\qty[\mathcal A_n]=\sum\limits_{m=3}^{n-1}\tilde{\mathcal A}_{n-m+2}\tilde{\mathcal A}_{m}\eval_{P_I^2=-m_I^2}
\end{align*}
É claro, para teorias nas quais o termo de borda é zero. Assim, podemos utilizar este formalismo para calcular cortes do tipo triângulo mostrados anteriormente, devido a não ser necessário se preocupar sobre o 
sinal dos momentos. Escrevemos então analogamente,
\[\mathcal A_n^{L}=\int\prod\limits_{j=1}^L\frac{\dd[4]{\ell_j}}{\qty(2\pi)^4}\mathcal I^L_n,\]
e agora nos voltamos sobre a construibilidade do integrando por meio de cortes. Bem, termos não construíveis são analíticos em $\mathbb C/\infty$ e estão relacionados com \textit{tadpoles}, 
por hora iremos ignorar estes. 

Comecemos com um exemplo à $1$--\textit{loop}, neste caso há apenas uma única integral de uma única variável de momento de \textit{loop}, $\ell$. Assim, como este possui apenas $4$ componentes, 
podemos ao máximo realizar um corte quádruplo, sendo qualquer outro número superior de cortes zero. Assim, olhemos para a maior ordem de corte, 
\[\qty(\ell^2+m_1^2)\qty(\qty(\ell-P_I)^2+m_2^2)\qty(\qty(\ell-P_J)^2+m_3^2)\qty(\qty(\ell-P_K)^2+m_4^2)\mathcal I^1_n\eval_{\textnormal{Cuts}}=\sum\limits_{k,l,m,p}^{k+l+m+p=n}\mathcal A_{k+2}\mathcal A_{l+2}\mathcal A_{m+2}\mathcal A_{p+2},\]
as escolhas de $P_{I,J,K}$ ditam as topologias que aparecerão do lado direito, porém, note que o lado direito não possui nenhuma menção à variável de integração $\ell$, assim,
\[\mathcal I^1_n\supseteq\sum\limits_{m_i}\sum\limits_{k,l,m,p}^{k+l+m+p=n}\mathcal A_{k+2}\mathcal A_{l+2}\mathcal A_{m+2}\mathcal A_{p+2}\int\frac{\dd[4]{\ell}}{\qty(2\pi)^4}\frac{1}{\qty(\ell^2+m_1^2)\qty(\qty(\ell-P_I)^2+m_2^2)\qty(\qty(\ell-P_J)^2+m_3^2)\qty(\qty(\ell-P_K)^2+m_4^2)},\]
isto é um exemplo de reconstrução, em geral, podemos fatorizar $\mathcal I^L_n$ em uma base de integrais escalares com coeficientes que englobam dependências cinemáticas, vetores de polarizações e 
elementos de álgebras. Claro, obtemos aqui somente um dos termos de $\mathcal I^1_n$, é possível que haja contribuições com menor número de polos, e para obter estas seria necessário tomar 
mais resíduos. Vamos demonstrar como isso pode ser feito na teoria teste, $g\phi^3$,
\begin{align*}
    \mathcal I^1_4\eval_{4\textnormal{ Cuts}} &=\raisebox{+3pt}{\begin{tikzpicture}[baseline=(b.base)]
        \begin{feynman}
            \vertex (i1) at (-1,0.5);
            \vertex (i2) at (-1,-0.5);
            \vertex [solid blob=gray, minimum size=0.3cm] (b1) at (-0.5,0.5) {};
            \vertex [solid blob=gray, minimum size=0.3cm] (b2) at (-0.5,-0.5) {};
            \vertex [solid blob=gray, minimum size=0.3cm] (b3) at (0.5,0.5) {};
            \vertex [solid blob=gray, minimum size=0.3cm] (b4) at (0.5,-0.5) {};
            \vertex (b) at (0,0) {};
            \vertex (f1) at (1,0.5);
            \vertex [particle =\(4\)] (f2) at (1,-0.5);
            \diagram*{
                (i1) [particle =\(2\)]-- [scalar] (b1),
                (i2) [particle =\(1\)]-- [scalar] (b2),
                (b1) -- [scalar, cut2, momentum = \(\ell\)] (b3),
                (b3) -- [scalar, cut2, momentum = \(\ell-3\)] (b4),
                (b4) -- [scalar, cut2, momentum = \(\ell-3-4\)] (b2),
                (b2) -- [scalar, cut2, momentum = \(\ell+2\)] (b1),
                (b3) -- [scalar] (f1)[particle =\(3\)],
                (b4) -- [scalar] (f2),
            };
        \end{feynman}
    \end{tikzpicture}}=\mathcal A_3\qty(2,\ell, -\ell-2)\mathcal A_3\qty(-\ell,\ell-3,3)\mathcal A_3\qty(4,3-\ell, \ell-3-4)\mathcal A_3\qty(1, \ell+2,3+4-\ell)\\
    \mathcal I^1_4\eval_{4\textnormal{ Cuts}}&=g^4, 
\end{align*}
que implica,
\[\mathcal I^1_4 \supseteq g^4\int\frac{\dd[4]{\ell}}{\qty(2\pi)^4}\frac{1}{\qty(\ell^2+m^2)\qty(\qty(\ell+2)^2+m^2)\qty(\qty(\ell-3)^2+m^2)\qty(\qty(\ell-3-4)^2+m^2)}.\]
Obviamente esse resulta é somente para uma única contribuição de topologia no corte de $4$ partículas, há mais outras cinco contribuições que podem ser obtidas por permutação das pernas externas. 
Procedemos para o corte de $3$ partículas,
\begin{align*}
    \mathcal I^1_4\eval_{3\textnormal{ Cuts}} &=\raisebox{+3pt}{\begin{tikzpicture}[baseline=(b.base)]
        \begin{feynman}
            \vertex (i1) at (-1,0.5);
            \vertex (i2) at (-1,-0.5);
            \vertex [solid blob=gray, minimum size=0.3cm] (b1) at (-0.5,0.5) {};
            \vertex [solid blob=gray, minimum size=0.3cm] (b2) at (-0.5,-0.5) {};
            \vertex [solid blob=gray, minimum size=0.3cm] (b3) at (0.3,0) {};
            \vertex (b) at (0,0) {};
            \vertex (f1) at (1,0.5);
            \vertex [particle =\(4\)] (f2) at (1,-0.5);
            \diagram*{
                (i1) [particle =\(2\)]-- [scalar] (b1),
                (i2) [particle =\(1\)]-- [scalar] (b2),
                (b1) -- [scalar, cut2, momentum = \(\ell\)] (b3),
                (b3) -- [scalar, cut2, momentum = \(\ell-3-4\)] (b2),
                (b2) -- [scalar, cut2, momentum = \(\ell+2\)] (b1),
                (b3) -- [scalar] (f1)[particle =\(3\)],
                (b3) -- [scalar] (f2),
            };
        \end{feynman}
    \end{tikzpicture}}=\mathcal A_3\qty(2,\ell, -\ell-2)\mathcal A_3\qty(1, \ell+2,3+4-\ell)\mathcal A_4\qty(-\ell,\ell-3-4,3,4)\\
    \mathcal I^1_4\eval_{3\textnormal{ Cuts}}&=g^4\qty[\frac{1}{m^2+\qty(3+4)^2}+\frac{1}{m^2+\qty(3-\ell)^2}+\frac{1}{m^2+\qty(4-\ell)^2}], 
\end{align*}
Certamente esta é apenas uma contribuição de outras duas, note que os dois últimos termos correspondem ao mesmo resíduo da expressão com $4$ polos, portanto apenas o primeiro termo 
nos diz algo não trivial sobre a parcela com $3$ polos,
\[\mathcal I^1_3\supseteq \frac{g^4}{m^2+\qty(3+4)^2}\int\frac{\dd[4]{\ell}}{\qty(2\pi)^4}\frac{1}{\qty(\ell^2+m^2)\qty(\qty(\ell+2)^2+m^2)\qty(\qty(\ell-3-4)^2+m^2)}.\]
O corte duplo é,
\begin{align*}
    \mathcal I^1_4\eval_{2\textnormal{ Cuts}} &=\raisebox{+3pt}{\begin{tikzpicture}[baseline=(b.base)]
        \begin{feynman}
            \vertex (i1) at (-1,0.5);
            \vertex (i2) at (-1,-0.5);
            \vertex [solid blob=gray, minimum size=0.3cm] (b1) at (-0.5,0) {};
            \vertex [solid blob=gray, minimum size=0.3cm] (b2) at (0.5,0) {};
            \vertex (b) at (0,0) {};
            \vertex (f1) at (1,0.5);
            \vertex [particle =\(4\)] (f2) at (1,-0.5);
            \diagram*{
                (i1) [particle =\(2\)]-- [scalar] (b1),
                (i2) [particle =\(1\)]-- [scalar] (b1),
                (b1) -- [scalar, cut2, momentum = \(\ell\), bend left=45, looseness=1] (b2),
                (b2) -- [scalar, cut2, momentum = \(\ell-3-4\), bend left=45, looseness=1] (b1),
                (b2) -- [scalar] (f1)[particle =\(3\)],
                (b2) -- [scalar] (f2),
            };
        \end{feynman}
    \end{tikzpicture}}=\frac12\mathcal A_4\qty(1,2,\ell, -\ell-2)\mathcal A_4\qty(-\ell,\ell-3-4,3,4)\\
    \mathcal I^1_4\eval_{3\textnormal{ Cuts}}&=\frac12g^4\qty[\frac{1}{m^2+\qty(1+2)^2}+\frac{1}{m^2+\qty(1+\ell)^2}+\frac{1}{m^2+\qty(2+\ell)^2}]\qty[\frac{1}{m^2+\qty(3+4)^2}+\frac{1}{m^2+\qty(3-\ell)^2}+\frac{1}{m^2+\qty(4-\ell)^2}], 
\end{align*}
realizando a mesma análise anterior, existe apenas um termo ligado a contribuição de $2$ polos,
\[\mathcal I^1_3\supseteq \frac{g^4}{2\qty(m^2+\qty(3+4)^2)\qty(m^2+\qty(1+2)^2)}\int\frac{\dd[4]{\ell}}{\qty(2\pi)^4}\frac{1}{\qty(\ell^2+m^2)\qty(\qty(\ell-3-4)^2+m^2)}.\]
Nós poderíamos ir ainda mais além e tentar calcular o corte simples,
\[\mathcal I^1_4\eval_{1\textnormal{ Cuts}}=\raisebox{+3pt}{\begin{tikzpicture}[baseline=(b.base)]
        \begin{feynman}
            \vertex (i1) at (-1,0.5);
            \vertex (i2) at (-1,-0.5);
            \vertex [solid blob=gray, minimum size=0.3cm] (b1) at (0,0) {};
            \vertex (c) at (0,0.7) {};
            \vertex (b) at (0,0) {};
            \vertex (f1) at (1,0.5);
            \vertex [particle =\(4\)] (f2) at (1,-0.5);
            \diagram*{
                (i1) [particle =\(2\)]-- [scalar] (b1),
                (i2) [particle =\(1\)]-- [scalar] (b1),
                % (b1) -- [scalar, cut2, momentum = \(\ell\), out=60, in=120, looseness=8] (b1),
                (b1) -- [scalar] (f1)[particle =\(3\)],
                (b1) -- [scalar] (f2),
            };
            \draw[scalar,cut2](b1) .. controls +(0.8,1) and +(-0.8,1) .. (b1);
        \end{feynman}
    \end{tikzpicture}}=\mathcal A_6\qty(1,2,3,4,\ell,-\ell),\]
porém, este tipo de corte não está bem definido em TQC no geral, principalmente devido a divergências no limite colinear de amplitudes, contudo, não há problema nisto, cortes como esses 
iriam apenas contribuir para \textit{tadpoles}, os quais após renormalização devem ser zero, portanto não iremos nos preocupar. Note que para o caso desta teoria teste é sim 
possível obter o integrando $\mathcal I^1_4$ completo partindo da unitariedade generalizada.
% Para obter tal formalismo precisamos supor várias coisas, primeiramente, LSZ. Isto é, um elemento de matriz S é obtido por,
% \[S_{\beta\alpha}=\im^n\int\prod\limits_{a=1}^n\dd[4]{x_a}\qty(\textnormal{K.O.})_a\epsilon_{\sigma_a}^{I_a}\qty(k_a)\exp\qty((\pm)_a\im k_a\cdot x_a)\qty(\Omega,\textnormal{T}\qty{\Phi^{I_a}_a\qty(x_a)\cdots}\Omega)\]
% como de costume, o sinal $\pm$ dentro da exponencial nos indica se determinada partícula faz parte da coleção inicial ou final, mas claro, isso só é verdade se $k_a^0>0$. Podemos 
% absorver este fator dentro dos momentos, assim, alguns possuirão $k_a<0$, nisto

\section{Plano de Atividades}

O plano de atividades para o próximo ano está descrito na \cref{tab:atividades},

\begin{table}[h!]
    \centering
    \caption{Plano de atividades.}
    \label{tab:atividades}
    \begin{tabular}{@{}llp{7cm}@{}}
        \toprule
        \textbf{Data} & \textbf{Categoria} & \textbf{Descrição} \\
        \midrule
        Mar–Jul 2025 & Disciplinas Cursadas & Teoria de Cordas.\newline Tópicos Avançados em Relatividade Geral. \\[2mm]

        Mar-Jul 2025 & Revisão Bibliográfica & Estudo do formalismo \textit{on-shell} \autocite{arkani-hamedScatteringAmplitudesAll2021,elvangScatteringAmplitudesGauge2015}, bem como 
        dos métodos e teorias utilizadas em \autocite{herrmannUVCancelationsGravity2019,johanssonConformalGravityGauge2017,johanssonUnravelingConformalGravity2018}.\\[2mm]

        Jul 2025 & Participação em Evento & XV Escola do CBPF. \\[2mm]

        Ago-Dez 2025 & Cálculos & Cálculos preliminares com a teoria teste proposta em \autocite{johanssonUnravelingConformalGravity2018}. \\[2mm]

        Nov-Dez 2025 & Escrita & Redação de capítulos preliminares sobre métodos \textit{on-shell} e unitariedade generalizada. \\[2mm]

        Out 2025 & Participação em Evento & XLVIII CPLF.\newline Agorá II. \\[2mm]

        Dez 2025 & Participação em Evento & QCD Meets Gravity School.\newline QCD Meets Gravity Conference.\\
        \bottomrule
    \end{tabular}
\end{table}

%%%%%%%%%%%%%%%%%%%%%%%%%%%%%%%%%%%%%%%%%%%%%%%%%%%%%%%%%%%%%

\newpage

\printbibliography

\end{document}