\documentclass[12pt]{amsart}

\usepackage[brazilian]{babel}
\usepackage{csquotes}
\usepackage[style=numeric-comp, backend=bibtex]{biblatex}
\usepackage{amsmath}
\usepackage{amssymb}
% \usepackage{bbm}
\usepackage{graphics, setspace}
\usepackage{mathtools}
\usepackage[hidelinks]{hyperref}
\usepackage{physics}
\usepackage{enumitem}
\usepackage{slashed}
\usepackage{tensor}
\usepackage[lmargin=1cm,rmargin=1cm, tmargin =2cm,bmargin =2cm]{geometry}
\usepackage{tensor}
\usepackage[brazilian]{cleveref}
\usepackage{tikz-feynman}
\usepackage{bbold}

\AtBeginDocument{\renewcommand*{\hbar}{{\mkern-1mu\mathchar'26\mkern-8mu\textnormal{h}}}}
\AtBeginDocument{\newcommand{\e}{\textnormal{e}}}
\AtBeginDocument{\newcommand{\im}{\textnormal{i}}}
\AtBeginDocument{\newcommand{\luz}{\textnormal{c}}}
\AtBeginDocument{\newcommand{\grav}{\textnormal{G}}}
\AtBeginDocument{\newcommand{\kb}{{\textnormal{k}_{\textnormal{B}}}}}
\newcommand{\Dd}[1]{\mathcal D #1}
\newcommand{\Det}[1]{\textup{Det} #1}
\newcommand{\sgn}[1]{\mbox{sgn}\qty(#1)}
\newcommand{\cqd}{\hfill$\blacksquare$}
\newcommand{\dbar}{\mbox{\dj}}
\newcommand{\HRule}[1]{\rule{\linewidth}{#1}}
\newcommand{\sumint}[1]{\ \ \mathclap{\displaystyle\int\limits_{#1}}\mathclap{\textstyle\sum}\ \ }

\numberwithin{equation}{section}

\newtheorem{teo}{Teorema}[section]
\newtheorem{defi}{Definição}[section]
\newtheorem{lem}{Lema}[section]
\newtheorem{hip}{Hipótese}[subsection]

\newcommand{\numberthis}{\addtocounter{equation}{1}\tag{\theequation}}

\pagestyle{plain}

%\AddToHook{cmd/section/before}{\clearpage}

\addbibresource{refs.bib}

\newtheorem{teorema}{Teorema}[section]
\newtheorem{definicao}{Definição}[section]
\newtheorem{lema}{Lema}[section]
\newtheorem{hipotese}{Hipótese}[section]
\newtheorem{postulado}{Postulado}[section]

\begin{document}

%\tableofcontents

%%%%%%%%%%%%%%%%%%%%%%%%%%%%%%%%%%%%%%%%%%%%%%%%%%%%%%%%%%%%%

\begin{titlepage}
    \begin{center}
        \vspace*{-1cm}{
            \setstretch{.5} 
            \textsc{Universidade de São Paulo} \\
            \HRule{.2pt}\\
            \textsc{Instituto de Física}
        }

        \vspace{5.5cm}
        
        \Large \textbf{\textsc{Amplitudes de espalhamento em teorias com derivadas de
            ordem superior}}
 	    \HRule{1.5pt} \\ [0.5cm]
        \linespread{1}
        \large Relatório de atividades anual de Mestrado. \\ 
   	    \HRule{1.5pt} \\ [0.5cm]       
        Projeto sob fomento da CAPES
        \\ [0.5cm]
        Pesquisador Responsável: Gabriel Santos Menezes \\
        Aluno: Vicente Viater Figueira
        \\ [0.5cm]
        Vigência: 01/03/2025 a 01/03/2027\\
        Período Coberto pelo Relatório: 01/03/2025 a 01/12/2025
        \vfill
        {\normalsize  São Paulo, \today}
    \end{center}
\end{titlepage}

\nocite{*}

\section{Resumo do projeto proposto}

Este plano de atividades se propõe a realizar um estudo de amplitudes de espalhamento na chamada
teoria $\qty(DF)^2$ e, utilizando-se do método da cópia dupla, estender esses resultados para o caso da super-
gravidade conforme do tipo Berkovits-Witten. Estes estudos também preveem uma maior compreensão
do método da unitariedade generalizada para o caso de partículas instáveis. Além disso, também nos
permitiria uma abordagem sistemática no estudo do comportamento a altas energias das amplitudes de
espalhamento em gravidade quadrática.

\section{Realizações no Período}

Disciplinas feitas no primeiro semestre: Teoria de Cordas, Tópicos Avançados em Relatividade Geral

Eventos participados: CBPF, CPLF e II Agorá Meeting

\subsection{Métodos On-Shell}

O principal ponto da abordagem relativamente moderna de métodos on-shell para o cálculo de amplitudes em teorias de campo 
é utilizar-se de uma informação subutilizada em Teoria Quântica de Campos (TQC) usual, transformações pelo \textit{Little-Group}. 
É de conhecimento geral que a álgebra de Poincaré admite dois invariantes de Casimir, a massa quadrada $-P^\mu P_\mu$ e o spin 
$W^\mu W_\mu$, para estados fisicamente aceitáveis é necessário $P^\mu P_\mu\leq 0$, desses dois casos
\subsubsection{teste}

\subsection{Unitariedade em TQC}

Unitariedade em TQC se refere a unitariedade da matrix $S$. Como revisão, a matrix $S$ é a amplitude de transição entre um estado \textit{in}, $\Psi^+_\alpha$, para um estado 
\textit{out}, $\Psi^-_\beta$, \[S_{\beta\alpha}=\qty(\Psi^-_\beta,\Psi^+_\alpha),\] aqui $\alpha$ e $\beta$ são índices que condensam toda a informação contida em seu respectivo 
estado do espaço de Hilbert. É assumido que tanto os estados \textit{in}, quanto os \textit{out}, sejam uma base completa do espaço de Hilbert, de forma que se a matrix $S$ 
é um mapa entre essas duas bases, é necessário ela ser um mapa unitário, e de fato, manipulando formalmente essa expressão, \[\int\dd{\beta}S^\ast_{\beta\gamma}S_{\beta\alpha}=\int\dd{\beta}\qty(\Psi^-_\beta,\Psi^+_\gamma)^\ast\qty(\Psi^-_\beta,\Psi^+_\alpha)=\int\dd{\beta}\qty(\Psi^+_\gamma,\Psi^-_\beta)\qty(\Psi^-_\beta,\Psi^+_\alpha)=\qty(\Psi^+_\gamma,\Psi^+_\alpha)=\delta_{\gamma\alpha}\]
Há fortes consequências dessa propriedades, a principal é chamada por motivos históricos de \textbf{Teorema Óptico}, primeiro, é necessário expandir, \[S_{\beta\alpha}=\delta_{\beta\alpha}+\im \qty(2\pi)^4\delta^{\qty(4)}\qty(p_\beta-p_\alpha)\mathcal A_{\beta\alpha}\] nessa forma, 
a condição de unitariedade implica,
\begin{align*}
    \delta_{\gamma\alpha}&=\int\dd{\beta}S^\ast_{\beta\gamma}S_{\beta\alpha}=\int\dd{\beta}\qty(\delta_{\beta\gamma}+\im \qty(2\pi)^4\delta^{\qty(4)}\qty(p_\beta-p_\gamma)\mathcal A_{\beta\gamma})^\ast\qty(\delta_{\beta\alpha}+\im \qty(2\pi)^4\delta^{\qty(4)}\qty(p_\beta-p_\alpha)\mathcal A_{\beta\alpha})\\
    \delta_{\gamma\alpha}&=\delta_{\gamma\alpha}-\im \qty(2\pi)^4\delta^{\qty(4)}\qty(p_\alpha-p_\gamma)\mathcal A^\ast_{\alpha\gamma}+\im \qty(2\pi)^4\delta^{\qty(4)}\qty(p_\gamma-p_\alpha)\mathcal A_{\gamma\alpha}+\qty(2\pi)^8\int\dd{\beta}\delta^{\qty(4)}\qty(p_\beta-p_\gamma)\delta^{\qty(4)}\qty(p_\beta-p_\alpha)\mathcal A^\ast_{\beta\gamma}\mathcal A_{\beta\alpha}\\
    0&=-\im \mathcal A^\ast_{\alpha\gamma}+\im \mathcal A_{\gamma\alpha}+\qty(2\pi)^4\int\dd{\beta}\delta^{\qty(4)}\qty(p_\beta-p_\alpha)\mathcal A^\ast_{\beta\gamma}\mathcal A_{\beta\alpha}
\end{align*}
A maior utilidade deste resultado é do ponto de vista de teoria de perturbação, certamente calculamos uma amplitude de espalhamento $\mathcal A_{\beta\alpha}$ em uma determinada 
ordem $\mathcal O(g^n)$ do parâmetro de acoplamento, porém, o resultado acima promove uma relação entre $\mathcal A$ e $\mathcal A^2$, ou seja, há relações entre amplitudes em 
diferentes ordens na expansão do parâmetro de acoplamento. A versão mais famosa deste resultado é para $\alpha=\gamma$, 
\begin{align*}
    \im \mathcal A^\ast_{\alpha\alpha}-\im \mathcal A_{\alpha\alpha}&=\qty(2\pi)^4\int\dd{\beta}\delta^{\qty(4)}\qty(p_\beta-p_\alpha)\mathcal A^\ast_{\beta\alpha}\mathcal A_{\beta\alpha}\\
    2\mathfrak{Im}\qty[\mathcal A_{\alpha\alpha}]&=\qty(2\pi)^4\int\dd{\beta}\delta^{\qty(4)}\qty(p_\beta-p_\alpha)\abs{\mathcal A_{\beta\alpha}}^2
\end{align*}
Trabalhando do ponto de vista de teoria de perturbação, podemos calcular a parte imaginária da contribuição de 1--\textit{loop} de $\mathcal A_{\alpha\alpha}$ apenas sabendo a contribuição 
de nível arvore para $\mathcal A_{\beta\alpha}$. Parte deste fato está relacionado ao teorema de Sokhotski–Plemelj, \[\frac{1}{p^2+m^2-\im\epsilon}=\im\pi\delta\qty(p^2+m^2)+\textnormal{P.V.}\frac{1}{p^2+m^2}.\] Que nos confirma que o propagador 
apenas possui parte imaginária para uma partícula \textit{on-shell}, porém, para diagramas a nível árvore não é cinematicamente permitido de uma partícula virtual 
interna ao diagrama entrar \textit{on-shell}, o que é compatível com o senso comum de contribuições à nível árvore serem polinômios de propagadores e numeradores cinemáticos, que certamente não 
possuem parte imaginária para partículas \textit{off-shell}. Agora, para contribuições de \textit{loop}, partículas virtuais internas podem ficarem \textit{on-shell}, e portanto, os diagramas 
podem possuírem parte imaginária.

Como exemplo tomemos a teoria $g\phi^3$,\[\mathcal L = -\frac12\phi\qty(-\Box+m^2)\phi+\frac1{3!}g\phi^3,\] A contribuição de 1--\textit{loop} para o processo $1\rightarrow 1$ é,
\begin{align*}
    \im\mathcal A^{\textnormal{1--loop}}_{1\rightarrow 1}&=\raisebox{+3pt}{\feynmandiagram [baseline = (b.base),horizontal=a to b,layered layout] {
        a -- [scalar,momentum =\(p\)] b -- [ half left,scalar, momentum=\(\ell\)] c --[half left,scalar, momentum=\(\ell-p\)] b,
        c -- [scalar,momentum =\(p\)] d,
    };}=\frac12 \qty(\im g)^2\frac{1}{\im^2}\int\frac{\dd[4]{\ell}}{\qty(2\pi)^4}\frac{1}{\ell^2+m^2-\im\epsilon}\frac{1}{\qty(\ell - p)^2+m^2-\im\epsilon}\\
    \mathcal A^{\textnormal{1--loop}}_{1\rightarrow 1}&=-\im\frac12g^2\int\frac{\dd[4]{\ell}}{\qty(2\pi)^4}\qty(\im\pi\delta\qty(\ell^2+m^2)+\textnormal{P.V.}\frac{1}{\ell^2+m^2})\qty(\im\pi\delta\qty(\qty(\ell-p)^2+m^2)+\textnormal{P.V.}\frac{1}{\qty(\ell-p)^2+m^2})\\
    \mathfrak{Im}\qty[\mathcal A^{\textnormal{1--loop}}_{1\rightarrow 1}]&=-\frac12g^2\int\frac{\dd[4]{\ell}}{\qty(2\pi)^4}\qty(-\pi^2\delta\qty(\ell^2+m^2)\delta\qty(\qty(\ell-p)^2+m^2)+\textnormal{P.V.}\frac{1}{\ell^2+m^2}\textnormal{P.V.}\frac{1}{\qty(\ell-p)^2+m^2})
\end{align*}
A parte dependente do valor principal resultará em zero, e portanto,
\begin{align*}
    \mathfrak{Im}\qty[\mathcal A^{\textnormal{1--loop}}_{1\rightarrow 1}]&=\frac12\pi^2g^2\int\frac{\dd[4]{\ell}}{\qty(2\pi)^4}\delta\qty(\ell^2+m^2)\delta\qty(\qty(\ell-p)^2+m^2)\\
    \mathfrak{Im}\qty[\mathcal A^{\textnormal{1--loop}}_{1\rightarrow 1}]&=\frac12\pi^2g^2\int\frac{\dd[4]{q}\dd[4]{\ell}}{\qty(2\pi)^4}\delta\qty(\ell^2+m^2)\delta\qty(q^2+m^2)\delta^{\qty(4)}\qty(q+\ell-p)\\
    \mathfrak{Im}\qty[\mathcal A^{\textnormal{1--loop}}_{1\rightarrow 1}]&=\frac12\pi^2g^2\int\frac{\dd[4]{q}\dd[4]{\ell}}{\qty(2\pi)^42\omega_{\boldsymbol \ell}2\omega_{\vb q}}\qty(\delta\qty(\ell^0-\omega_{\boldsymbol\ell})+\delta\qty(\ell^0+\omega_{\boldsymbol\ell}))\qty(\delta\qty(q^0-\omega_{\vb q})+\delta\qty(q^0+\omega_{\vb q}))\delta^{\qty(4)}\qty(q+\ell-p)\\
    \mathfrak{Im}\qty[\mathcal A^{\textnormal{1--loop}}_{1\rightarrow 1}]&=\frac12\pi^2g^2\int\frac{\dd[3]{\vb q}\dd[3]{\boldsymbol\ell}}{\qty(2\pi)^42\omega_{\boldsymbol \ell}2\omega_{\vb q}}\qty(\delta\qty(\omega_{\vb q}+\omega_{\boldsymbol\ell}-p^0)+\delta\qty(\omega_{\vb q}-\omega_{\boldsymbol\ell}-p^0)+\delta\qty(-\omega_{\vb q}+\omega_{\boldsymbol\ell}-p^0))\delta^{\qty(3)}\qty(\vb q+\boldsymbol\ell-\vb p)\\
    \mathfrak{Im}\qty[\mathcal A^{\textnormal{1--loop}}_{1\rightarrow 1}]&=\frac18\qty(2\pi)^4g^2\int\frac{\dd[3]{\vb q}\dd[3]{\boldsymbol\ell}}{\qty(2\pi)^62\omega_{\boldsymbol \ell}2\omega_{\vb q}}\delta^{\qty(4)}\qty(q+\ell- p)\\
    \mathfrak{Im}\qty[\mathcal A^{\textnormal{1--loop}}_{1\rightarrow 1}]&=\frac18\qty(2\pi)^4\int\frac{\dd[3]{\vb q}\dd[3]{\boldsymbol\ell}}{\qty(2\pi)^62\omega_{\boldsymbol \ell}2\omega_{\vb q}}\delta^{\qty(4)}\qty(q+\ell- p)\abs{\mathcal A^{\textnormal{tree}}_{1\rightarrow 2}}^2, \ \ \ \mathcal A^{\textnormal{tree}}_{1\rightarrow 2}=g
\end{align*}
Neste \textit{toy-model} podemos apreciar claramente a parte imaginária da amplitude $1\rightarrow 1$ a 1--\textit{loop} ser expressável em termos da amplitude nível árvore $1\rightarrow 2$. A integral que aparece, 
\[\int\dd{\beta}=\frac14\int\frac{\dd[3]{\boldsymbol\ell}\dd[3]{\vb q}}{\qty(2\pi)^62\omega_{\boldsymbol\ell}2\omega_{\vb q}},\] 
nada é além da medida Lorentz invariante do espaço de fase. Há uma maneira diagramática de obter essa igualdade entre a parte imaginária e produtos 
de amplitudes em menor ordem, elas vão pelo nome de \textbf{regras de corte de Cutkosky}, o procedimento é simples, escrevemos um diagrama de Feynman de n--\textit{loops} que 
contribua para o processo em análise, disto, \textit{cortamos} propagadores deste diagrama de forma a separar o diagrama inicial em dois diagramas de ordem menor. 
O procedimento de \textit{cortar} um propagador corresponde a substituir $\qty(p^2+m^2-\im\epsilon)^{-1}$ por $\im\pi\theta\qty(p^0)\delta\qty(p^2+m^2)$, diagramaticamente, 
representamos um propagador cortado por uma linha perpendicular passando por seu propagador, ao fim, multiplicamos as duas amplitudes restantes, com a da direita sendo 
conjugada, ao fim, integramos sobre o espaço de fase Lorentz invariante. Note que neste processo obtemos duas amplitudes \textit{on-shell}. Como exemplo,
\begin{align*}
    \mathfrak{Im}\qty[\mathcal A^{\textnormal{1--loop}}_{1\rightarrow 1}]&=\frac12
    \raisebox{+3pt}{\begin{tikzpicture}[baseline=(a.base)]
        \begin{feynman}\diagram* {
        a -- [scalar,momentum =\(p\)] b -- [ half left,scalar,momentum=\(\ell\), cut] c-- [scalar,momentum =\(p\)] d,c --[half left,scalar,reversed momentum=\(q\), cut] b
        };
        \end{feynman}
    \end{tikzpicture}}=\frac12\qty(2\pi)^4\int\dd{\beta}\delta^{\qty(4)}\qty(q+\ell-p)\mathcal A^{\textnormal{tree}}_{1\rightarrow 2}\qty(p;\ell,q)\mathcal A^{\textnormal{tree}\ast}_{1\rightarrow 2}\qty(p;\ell,q)
\end{align*}
Claramente, para esse exemplo simples, há apenas uma única maneira de se \textit{cortar} o diagrama de 1--\textit{loop} em 
duas partes de menor ordem. Porém, para um número de pernas externas maior, ou maior número de \textit{loops}, é necessário somar 
sobre todas as maneiras de se separar as amplitudes. Este conceito de unitariedade possui algumas limitações, 
primeiro, é possível apenas determinar a parte imaginária das amplitudes via amplitudes de ordem inferior, 
segundo, somente conseguimos aplicar este resultado para um espalhamento da forma $\alpha\rightarrow \alpha$, que 
é longe de ser a forma de espalhamento mais geral. Contudo, é possível obter um resultado mais geral, para isto, temos 
que relembrar a definição de estados \textit{in/out}. Dado um hamiltoniano $H=H_0+V$, e sendo $\Phi_\alpha$ autoestado de $H_0$ com autovalor $E_\alpha$, definimos 
$\Psi^\pm_\alpha$ autoestado de $H$ com autovalor $E_\alpha$ por \[\Psi^\pm_\alpha=\Phi_\alpha+\qty(E_\alpha-H_0\pm\im\epsilon)^{-1}V\Psi^\pm_\alpha,\ \ \ \epsilon>0.\]

Note a imposição $\epsilon>0$, e portanto, a troca $\epsilon\leftrightarrow-\epsilon$ corresponde a: $\Psi^-_\alpha\leftrightarrow\Psi^+_\alpha$. 
Assim, podemos retornar a expressão,
\[S_{\alpha\gamma}^\ast=\qty(\Psi^-_\alpha,\Psi^+_\gamma)^\ast = \qty(\Psi^+_\gamma,\Psi^-_\alpha)= \qty(\Psi^-_\gamma,\Psi^+_\alpha)\eval_{\epsilon<0}=S_{\gamma\alpha}\eval_{\epsilon<0},\]
portanto,\[\mathcal A_{\alpha\gamma}^\ast=\mathcal A_{\gamma\alpha}\eval_{\epsilon<0},\]
utilizando esse resultado podemos concluir,
\begin{align*}
    -\im\qty(\mathcal A_{\gamma\alpha}-\mathcal A^\ast_{\alpha\gamma})&=\int\dd{\beta}\qty(2\pi)^4\delta^{\qty(4)}\qty(p_\beta-p_\alpha)\mathcal A^\ast_{\beta\gamma}\mathcal A_{\beta\alpha}\\
    -\im\qty(\mathcal A_{\gamma\alpha}\eval_{\epsilon>0}-\mathcal A_{\gamma\alpha}\eval_{\epsilon<0})&=\int\dd{\beta}\qty(2\pi)^4\delta^{\qty(4)}\qty(p_\beta-p_\alpha)\mathcal A^\ast_{\beta\gamma}\mathcal A_{\beta\alpha}
\end{align*}
O lado esquerdo desta igualdade deve ser entendido como sendo o limite $\epsilon\rightarrow 0$, claramente, se $\mathcal A_{\gamma\alpha}$ fosse 
uma função contínua em $\epsilon$, o resultado seria zero, e como o lado direito da igualdade é não necessariamente zero, podemos apenas concluir que: 
Em geral $\mathcal A_{\gamma\alpha}$ é descontínuo em $\epsilon$, porém, como $\epsilon$ contribui para a amplitude somente dentro do propagador 
$-\im\qty(p^2+m^2-\im\epsilon)^{-1}$, e é acompanhado por um fator de $\im$, concluímos que genericamente as amplitudes, vistas como 
funções dos invariantes cinemáticos, possuem um \textit{branch cut} num subconjunto do eixo real, quando interpretamos os momentos podendo tomar 
valores em números complexos. 

Seja uma função $f:\mathbb C\rightarrow \mathbb C$ analítica em todo plano, exceto por possíveis polos e um \textit{branch cut} no eixo real, naturalmente 
isso significa que, 
\begin{align*}
    \exists s\in\mathbb R,\ \ 0&\neq\lim\limits_{\epsilon\rightarrow 0^+}\qty[f\qty(s-\im\epsilon)-f\qty(s+\im\epsilon)]= \mathfrak{Disc}[f],
\end{align*}
no qual já definimos o que chamamos de descontinuidade de uma função. Assim,
\begin{align*}
    -\im \mathfrak{Disc}\qty[\mathcal A_{\gamma\alpha}]&=\int\dd{\beta}\qty(2\pi)^4\delta^{\qty(4)}\qty(p_\beta-p_\alpha)\mathcal A^\ast_{\beta\gamma}\mathcal A_{\beta\alpha}.
\end{align*}

Agora, se supormos que $\mathcal A_{\gamma\alpha}$ é uma amplitude de nível árvore, claramente as sub-amplitudes 
$\mathcal A_{\beta\gamma},\mathcal A_{\beta\alpha}$ devem ser também de nível árvore, do contrário, seriam contribuições de ordem superior 
para $\mathcal A_{\gamma\alpha}$. Assim, necessariamente $\Phi_\beta$ é um estado de uma única partícula, portanto, $\dd{\beta}$ é a 
medida invariante de Lorentz do espaço de fase de uma única partícula,
\begin{align*}
    -\im \mathfrak{Disc}\qty[\mathcal A^{\textnormal{Tree}}_{\gamma\alpha}]&=\int\dd{\beta}\qty(2\pi)^4\delta^{\qty(4)}\qty(p_\beta- p_\alpha)\mathcal A^\ast_{\beta\gamma}\mathcal A_{\beta\alpha}\\
    -\im \mathfrak{Disc}\qty[\mathcal A^{\textnormal{Tree}}_{\gamma\alpha}]&=\sum\limits_\beta\int\frac{\dd[3]{\vb q}}{\qty(2\pi)^32\omega_{\vb q}}\qty(2\pi)^4\delta^{\qty(4)}\qty(q- p_\alpha)\mathcal A^\ast_{\beta\gamma}\mathcal A_{\beta\alpha}\\
    -\im \mathfrak{Disc}\qty[\mathcal A^{\textnormal{Tree}}_{\gamma\alpha}]&=\sum\limits_\beta2\pi\frac{1}{2\omega_{\vb q}}\delta\qty(q^0- p^0_\alpha)\mathcal A^\ast_{\beta\gamma}\mathcal A_{\beta\alpha}\eval_{\vb q = \vb p_\alpha=\vb p_\gamma}\\
    -\im \mathfrak{Disc}\qty[\mathcal A^{\textnormal{Tree}}_{\gamma\alpha}]&=\sum\limits_\beta2\pi\theta\qty(p_\alpha^0)\delta\qty(\qty(q^0)^2-\qty(p_\alpha^0)^2)\mathcal A^\ast_{\beta\gamma}\mathcal A_{\beta\alpha}\eval_{\vb q = \vb p_\alpha=\vb p_\gamma}\\
    \mathfrak{Disc}\qty[\mathcal A^{\textnormal{Tree}}_{\gamma\alpha}]&=\sum\limits_\beta2\pi\im\theta\qty(p_\alpha^0)\delta\qty(p_\alpha^2+m_\beta^2)\mathcal A^\ast_{\beta\gamma}\mathcal A_{\beta\alpha}.
\end{align*}

Note que $p_\alpha^2+m_\beta^2 = 0$ é impossível de ser satisfeito para diagramas nível árvore, pois, isso força $m_\beta^2> \sum\limits_\alpha m_\alpha^2$, 
e caso exista tal estado em nossa teoria, este não é estável, logo, não é possível de estar em um espalhamento de estados assintóticos. 
Claramente, essa expressão somente toma total sentido se estamos dispostos a interpretar $\mathcal A_{\gamma\alpha}$ como função dos momentos externos, 
e admitirmos estes a poderem tomar valores complexos. Nesse ponto de vista, olhemos para as amplitudes como funções de cada momento das partículas iniciais e 
finais: $p_{\alpha_i},p_{\gamma_j}$, e estudemos sua extensão no plano complexo definida por,
\begin{align*}
    \mathcal A_{\gamma \alpha} = \mathcal A_{\gamma \alpha}\qty(\qty{p_\alpha},\qty{p_\gamma})\rightarrow\tilde{\mathcal A}_{\gamma\alpha}\qty(\qty{p_\alpha+k_\alpha z},\qty{p_\gamma+k_\gamma z})=\tilde{\mathcal A}_{\gamma \alpha}\qty(z),
\end{align*}
no qual $z\in\mathbb C$ e $k_{\alpha_i},k_{\gamma_j}$ são momentos arbitrários satisfazendo:\[k_\alpha=k_\gamma,\ \ \ k_{\alpha_i}\cdot k_{\alpha_j}=k_{\alpha_i}\cdot k_{\gamma_j}=k_{\gamma_i}\cdot k_{\gamma_j}=0,\ \ \ p_{\alpha_i}\cdot k_{\alpha_i}=p_{\gamma_i}\cdot k_{\gamma_i}=0.\]

Desta forma temos que $\tilde{\mathcal A}_{\gamma \alpha}\qty(0)=\mathcal A_{\gamma \alpha}$, olhemos para expressão anterior quando sujeita a esta extensão complexa,
\begin{align*}
    \mathfrak{Disc}\qty[\tilde{\mathcal A}_{\gamma\alpha}]&=\sum\limits_\beta2\pi\im\theta\qty({\tilde{p}}_\alpha^0)\delta\qty({\tilde{p}}_\alpha^2+m_\beta^2)\tilde{\mathcal A}^\ast_{\beta\gamma}\tilde{\mathcal A}_{\beta\alpha}\\
    \mathfrak{Disc}\qty[\tilde{\mathcal A}_{\gamma\alpha}]&=\sum\limits_\beta2\pi\im\theta\qty(p_\alpha^0+k_\alpha^0 z)\delta\qty(p_\alpha^2+m_\beta^2+2p_\alpha\cdot k_\alpha z)\tilde{\mathcal A}^\ast_{\beta\gamma}\tilde{\mathcal A}_{\beta\alpha}\\
    \mathfrak{Disc}\qty[\tilde{\mathcal A}_{\gamma\alpha}]&=\sum\limits_\beta\frac{2\pi\im}{2\abs{p_\alpha\cdot k_\alpha}}\theta\qty(p_\alpha^0+k_\alpha^0 z)\delta\qty(z+\frac{p_\alpha^2+m_\beta^2}{2p_\alpha\cdot k_\alpha})\tilde{\mathcal A}^\ast_{\beta\gamma}\tilde{\mathcal A}_{\beta\alpha},\ \ \ z_\beta = -\frac{p_\alpha^2+m_\beta^2}{2p_\alpha\cdot k_\alpha}\\
    \mathfrak{Disc}\qty[\tilde{\mathcal A}_{\gamma\alpha}]&=\sum\limits_\beta \abs{z_\beta}\frac{2\pi\im}{\abs{p_\alpha^2+m_\beta^2}}\theta\qty(p_\alpha^0+k_\alpha^0 z_\beta)\delta\qty(z-z_\beta)\tilde{\mathcal A}^\ast_{\beta\gamma}\tilde{\mathcal A}_{\beta\alpha}\\
    \frac1z\mathfrak{Disc}\qty[\tilde{\mathcal A}_{\gamma\alpha}]&=\sum\limits_\beta \frac{\abs{z_\beta}}{z_\beta}\frac{2\pi\im}{\abs{p_\alpha^2+m_\beta^2}}\theta\qty(p_\alpha^0+k_\alpha^0 z_\beta)\delta\qty(z-z_\beta)\tilde{\mathcal A}^\ast_{\beta\gamma}\tilde{\mathcal A}_{\beta\alpha}.
\end{align*}

Sabemos que as amplitudes $\mathcal A_{\beta\gamma},\mathcal A_{\beta\alpha}$ são de nível árvore, portanto, pelo argumento anterior, não possuem singularidades em $z=z_\beta$, 
assim, podemos integrar a expressão anterior em um intervalo simplesmente conexo fechado qualquer que contenha $z_\beta$ e não passe por $z=0$. Chamemos este intervalo de $I_\beta$,
\begin{align*}
    \int\limits_{I_\beta}\dd{z}\frac1z\mathfrak{Disc}\qty[\tilde{\mathcal A}_{\gamma\alpha}]&=\int\limits_{I_\beta}\dd{z}\sum\limits_\beta \frac{\abs{z_\beta}}{z_\beta}\frac{2\pi\im}{\abs{p_\alpha^2+m_\beta^2}}\theta\qty(p_\alpha^0+k_\alpha^0 z_\beta)\delta\qty(z-z_\beta)\tilde{\mathcal A}^\ast_{\beta\gamma}\tilde{\mathcal A}_{\beta\alpha}\\
    \int\limits_{I_\beta}\dd{z}\frac1z\mathfrak{Disc}\qty[\tilde{\mathcal A}_{\gamma\alpha}]&=\sum\limits_\beta \frac{\abs{z_\beta}}{z_\beta}\frac{2\pi\im}{\abs{p_\alpha^2+m_\beta^2}}\theta\qty(p_\alpha^0+k_\alpha^0 z_\beta)\tilde{\mathcal A}^\ast_{\beta\gamma}\tilde{\mathcal A}_{\beta\alpha}\eval_{z=z_\beta},
\end{align*}
agora, utilizando a definição de descontinuidade, e sabendo que a descontinuidade de $z$ é zero neste domínio, podemos deformar o contorno $I_\beta$ para duas versões, uma passando por baixo da reta real, e outra 
por cima da reta real com orientação oposta, e devido à descontinuidade é possível unir estes contornos formando uma curva fechada no sentido anti-horário circundando o ponto $z_\beta$, 
chamemos este contorno de $C_\beta$,
\begin{align*}
    \oint\limits_{C_\beta}\dd{z}\frac1z\tilde{\mathcal A}_{\gamma\alpha}&=\sum\limits_\beta \frac{\abs{z_\beta}}{z_\beta}\frac{2\pi\im}{\abs{p_\alpha^2+m_\beta^2}}\theta\qty(p_\alpha^0+k_\alpha^0 z_\beta)\tilde{\mathcal A}^\ast_{\beta\gamma}\tilde{\mathcal A}_{\beta\alpha}\eval_{z=z_\beta}.
\end{align*}
Podemos então agora deformar o contorno $C_\beta$ em outros dois, um circundando o ponto $z=0$ no sentido horário, e outro no sentido anti-horário circundando $z=\infty$, assim,
\begin{align*}
    -\frac{1}{2\pi\im}\oint\limits_{C_0}\dd{z}\frac1z\tilde{\mathcal A}_{\gamma\alpha}-\frac{1}{2\pi\im}\oint\limits_{C_\infty}\dd{z}\frac1z\tilde{\mathcal A}_{\gamma\alpha}&=\sum\limits_\beta \frac{\abs{z_\beta}}{z_\beta}\frac{1}{\abs{p_\alpha^2+m_\beta^2}}\theta\qty(p_\alpha^0+k_\alpha^0 z_\beta)\tilde{\mathcal A}^\ast_{\beta\gamma}\tilde{\mathcal A}_{\beta\alpha}\eval_{z=z_\beta}\\
    -\mathfrak{Res}_{z=0}\qty[\frac1z\tilde{\mathcal A}_{\gamma\alpha}]-\mathfrak{Res}_{z=\infty}\qty[\frac1z\tilde{\mathcal A}_{\gamma\alpha}]&=\sum\limits_\beta \frac{\abs{z_\beta}}{z_\beta}\frac{1}{\abs{p_\alpha^2+m_\beta^2}}\theta\qty(p_\alpha^0+k_\alpha^0 z_\beta)\tilde{\mathcal A}^\ast_{\beta\gamma}\tilde{\mathcal A}_{\beta\alpha}\eval_{z=z_\beta}\\
    \mathcal A_{\gamma\alpha}&=-\sum\limits_\beta \frac{\abs{z_\beta}}{z_\beta}\frac{1}{\abs{p_\alpha^2+m_\beta^2}}\theta\qty(p_\alpha^0+k_\alpha^0 z_\beta)\tilde{\mathcal A}^\ast_{\beta\gamma}\tilde{\mathcal A}_{\beta\alpha}\eval_{z=z_\beta}-\mathfrak{Res}_{z=\infty}\qty[\frac1z\tilde{\mathcal A}_{\gamma\alpha}]\\
    \mathcal A_{\gamma\alpha}&=-\sum\limits_\beta \frac{\abs{z_\beta}}{z_\beta}\frac{1}{\abs{p_\alpha^2+m_\beta^2}}\theta\qty(p_\alpha^0+k_\alpha^0 z_\beta)\tilde{\mathcal A}^\ast_{\beta\gamma}\tilde{\mathcal A}_{\beta\alpha}\eval_{z=z_\beta}+\lim\limits_{\abs{z}\rightarrow \infty}\tilde{\mathcal A}_{\gamma\alpha},
\end{align*}
finalmente, sempre podemos tomar $z_\beta<0$, pois, $m_\beta^2>-p_\alpha^2$ e o sinal de $k_\alpha$ é fixado por se fazer $\theta\qty(p_\alpha^0+k_\alpha^0z_\beta)$, assim,
\begin{align*}
    \mathcal A_{\gamma\alpha}&=\sum\limits_\beta\frac{1}{p_\alpha^2+m_\beta^2}\tilde{\mathcal A}^\ast_{\beta\gamma}\tilde{\mathcal A}_{\beta\alpha}\eval_{z=z_\beta}+\lim\limits_{\abs{z}\rightarrow \infty}\tilde{\mathcal A}_{\gamma\alpha}\numberthis\label{residuotreelevelsmatrix}.
\end{align*}
Esta forma da condição de unitariedade é também conhecida como \textbf{relações de recursão à nível árvore}. Escolhas específicas de $k_\alpha,k_\gamma$ 
caracterizam diferentes métodos de recursão, os principais métodos são BCFW --- Britto, Cachazo, Feng e Witten --- e CSW --- Cachazo, Svrcek e Witten ---. 
Esses métodos de recursão, unidos com a simplicidade do formalismo de amplitudes \textit{on-shell}, providenciam uma abordagem robusta e --- comparativamente ao método usual --- simples para 
se obter amplitudes de $n\geq4$ pontos à nível árvore. Claramente, essas relações somente facilitam a obtenção de amplitudes se o requisito \[\lim\limits_{\abs{z}\rightarrow \infty}\tilde{\mathcal A}_{\gamma\alpha}=0,\] for satisfeito. Esta 
condição é altamente não trivial e não é válida para várias teorias, porém, para teorias do tipo Yang-Mills foi-se provado que é de fato verdadeira.

\section{Plano de Atividades}

O plano de atividades para o próximo ano está descrito na \cref{tab:atividades},

\begin{table}[h!]
    \centering
    \caption{Plano de atividades.}
    \label{tab:atividades}
    \begin{tabular}{@{}llp{7cm}@{}}
        \toprule
        \textbf{Data} & \textbf{Categoria} & \textbf{Descrição} \\
        \midrule
        Mar–Jul 2025 & Disciplinas Cursadas & Teoria de Cordas.\newline Tópicos Avançados em Relatividade Geral. \\[2mm]

        Mar-Jul 2025 & Revisão Bibliográfica & Estudo do formalismo \textit{on-shell} \autocite{arkani-hamedScatteringAmplitudesAll2021,elvangScatteringAmplitudesGauge2015}, bem como 
        dos métodos e teorias utilizadas em \autocite{herrmannUVCancelationsGravity2019,johanssonConformalGravityGauge2017,johanssonUnravelingConformalGravity2018}.\\[2mm]

        Jul 2025 & Participação em Evento & XV Escola do CBPF. \\[2mm]

        Ago-Dez 2025 & Cálculos & Cálculos preliminares com a teoria teste proposta em \autocite{johanssonUnravelingConformalGravity2018}. \\[2mm]

        Nov-Dez 2025 & Escrita & Redação de capítulos preliminares sobre métodos \textit{on-shell} e unitariedade generalizada. \\[2mm]

        Out 2025 & Participação em Evento & XLVIII CPLF.\newline Agorá II. \\[2mm]

        Dez 2025 & Participação em Evento & QCD Meets Gravity School.\newline QCD Meets Gravity Conference.\\
        \bottomrule
    \end{tabular}
\end{table}

%%%%%%%%%%%%%%%%%%%%%%%%%%%%%%%%%%%%%%%%%%%%%%%%%%%%%%%%%%%%%

\newpage

\printbibliography

\end{document}